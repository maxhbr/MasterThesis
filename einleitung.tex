\chapter{Introduction/motivation}

\PROBLEM[Danksagung? Erwähnen der software?]

\TODO[see~\cite{van2003galois} for better text]
The basic problem, from which this \rewrite{masters-thesis} arises, is that
there are differential equations $(\frac{d}{dt}-A)\hat v=w$ with coefficients
in convergent powers series, which have no solution with converging entries.
One can use the theory of meromorphic connections, to look at this problem and
the interesting object is then the classifying set\comm{, which is defined
as\dots}.

\TODO[Or: want to classify meromorphic connections with singularities.]

\TODO[motivation for meromorphic connections]

We will only be interested in local information and thus only in local
classification of meromorphic connections.
This classification can be splittet in the coarse formal classification and the
fine meromorphic classification.

The formal classification problem was solved by the Levelt-Turittin Theorem
(cf.\ Theorem~\ref{thm:leveltTurittin}). It states that in every formal
equivalence class are some meromorphic connections of special form, which will
be called models.
These models are meromorphic connections, which are defined to be isomorphic to
some direct sum of elementary meromorphic connections and elementary
meromorphic connections are well understood.

Starting with a formal equivalence class corresponding to some model $A^0$, we
can apply the meromorphic classification.
The obtained set of meromorphic classes will be called the classifying set
$\cH(A^0)$ (cf.\ Section~\ref{sec:classifyingSet}).
The tool to describe the classifying set will be the Stokes structures which
turn out to deliver exactly the needed information.
This idea is formulated in the Malgrange-Sibuya Theorem
(cf.\ Theorem~\ref{thm:mainThm1}), where the Stokes structures appear as the
first cohomology $H^1(S^1;\Lambda(A^0))$ of the Stokes sheaf
$\Lambda(A^0)$ on $S^1$ (cf.\ Definition~\ref{defn:StokesSheaf}).

The Malgrange-Sibuya theorem can be improved by showing that in each element
in the $H^1(S^1;\Lambda(A^0))$ contains a unique cocycle called the Stokes
cocyle (cf.\ Definition~\ref{defn:stokesCocycle}) of spezial form.
These Stokes cocycles are given by the elements in the product over some
spezial directions $\theta\in\A\subset S^1$ determined by $A^0$
(cf.\ Definition~\ref{defn:antiStokesDir}) of groups
$\Sto_\theta(A^0)\subset\Lambda_\theta(A^0)$
(cf.\ Definition~\ref{defn:stokesGroup}).
We will see that $\Sto_\theta(A^0)$ has faithful representation
$\SSto_\theta(A^0)$ (cf.\ Proposition~\ref{prop:representation}).
The elements of $\SSto_\theta(A^0)$ are the so-called Stokes matrices and it is
easy to see that they are nilpotent.
This characteristic can be used to define the structure of a nilpotent Lie
group on the classifying set $\cH(A^0)$.

Stokes matrices provide exactly the required information in order to describe a
meromorphic class of a meromorphic connection, and since we know which
restrictions hold for these matrices, we can explicitly give the isomorphism
$\C^n\to H^1(S^1;\Lambda(A^0))$ for a convenient $n$
(cf.\ Chapter~\ref{chp:WhichInformationIsNeeded}).

\begin{comment}
  \begin{enumerate}
    \item first introduce asymptotic analysis
    \item then define languages for meromorphic connections or systems
    \item talk about Stokes structures
    \item more?
  \end{enumerate}
\end{comment}

