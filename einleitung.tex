\chapter{Einleitung}
There is the problem of classifying meromorphic connections, which can be
splittet in the coarse formal classification and the fine meromorphic
classification. The formal classification problem was solved by the
Levelt-Turittin theorem, which states, that in every formal class are some
spezial meromorphic connections of special form, i.e.\ meromorphic connections,
which are isomorphic to some direct sum of elementary meromorphic connections.
These elementary meromorphic connections are well understood.

If we start with a formal equivalence class of meromorphic connections, which
is characterized by a model, we can apply the meromorphic classification.
The obtained set of meromorphic classes will be called the classifying set.
The tool to describe the classifying set will be the Stokes structures which
turn out, to supply exactly the needed information. Formulated is this in the
Malgrange-Sibuya theorem, in which the Stokes structures appear as the first
cohomology $H^1(S^1;\Lambda^{<0}(A^0))$ of the Stokes sheaf $\Lambda^{<0}(A^0)$
on $S^1$.

The Malgrange-Sibuya theorem can be improved by showing, that in each element
in the $H^1(S^1;\Lambda^{<0}(A^0))$ cotains a unique cocycle, called the Stokes
cocyle, of spezial form, i.e.\ is given by an element in the product of
subgroups $\Sto_\theta(A^0)$ of $\Lambda_\theta^{<0}(A^0)$ at
\rewrite{spezial directions} $\theta\in\A$.
We will see, that $\Sto_\theta(A^0)$ has faithful representation
$\SSto_\theta(A^0)$. The elements of $\SSto_\theta(A^0)$ are the so-called
Stokes matrices and it is easy to see, that they are nilpotent.
This can be used to define the structure of a nilpotent Lie group to the
classifying set.

Since we have seen, that the Stokes matrices are exactly the needed
information, to describe a meromorphic class of a meromorphic connection, and
since we know which restrictions hold for these matrices we can explicitly
write an isomorphism of $\C^n\to H^1(S^1;\Lambda^{<0}(A^0))$ for a convenient
$n$.

\begin{comment}
  \begin{enumerate}
    \item first introduce asymptotic analysis
    \item then define languages for meromorphic connections or systems
    \item talk about Stokes structures
    \item more?
  \end{enumerate}
\end{comment}

