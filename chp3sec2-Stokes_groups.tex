%%%%%%%%%%%%%%%%%%%%%%%%%%%%%%%%%%%%%%%%%%%%%%%%%%%%%%%%%%%%%%%%%%%%%%%%%%%%%%%
\section{The Stokes groups}\label{sec:StokesGroup}
Here we want to introduce the notion of Stokes groups. They are for example
also introduced by Loday-Richaud in~\cite{Loday1994,Loday2014} or section 4
of~\cite{Martinet1991} by Martinet and Ramis.

Let us recall, that the normal form $A^0$ can be written as
$A^0=Q'(t^{-1})+L\frac{1}{t}$ and a normal solution is given by
$\cY_0(t)=t^Le^{Q(t^{-1})}$ (cf.\ Proposition~\ref{prop:fundSolBuilder}), where
\begin{itemize}
\item $Q(t^{-1})=\bigoplus_{j\in\{1,\dots,s\}}q_j(t^{-1})\cdot\id_{n_j}$ and
\item the block structure of $L$ is finer then the structure of $Q$
  (cf.\ Definition~\ref{defn:structureComparison}).
\end{itemize}
Let $\left\{q_1(t^{-1}),\dots,q_s(t^{-1})\right\}$ be the \emph{set of all
determining polynomials of $[A^0]$} and denote by
\[
  \cQ(A^0):=\left\{q_j-q_l
    \mid
    \text{$q_j$ and $q_l$ determining polynomials of $[A^0]$, } q_j \neq q_l
  \right\}
\]
the \emph{set of all determining polynomials of $[\End A^0]$}.
Instead of $q_j-q_l\in\cQ(A^0)$ we will sometimes talk of (ordered) pairs
$(q_j,q_l)\in\cQ(A^0)$.

\begin{defn}\label{defn:determiningPolysOfEndA}
  We call
  \begin{itemize}
    \item $a_{jl}\in\C\backslash\{0\}$ the \emph{leading factor},
    \item $\frac{a_{jl}}{t^{k_{jl}}}$ the \emph{leading
      coefficient} and
    \item $k_{jl}\in\Q$ the \emph{degree}
  \end{itemize}
  of $q_j-q_l\in\cQ(A^0)$ if
  \[
    q_j-q_l\in\left\{\frac{a_{jl}}{t^{k_{jl}}}+h \mid h \in o(t^{-k_{jl}}),
      a_{jl}\neq0
    \right\}\,.
  \]
  \begin{s-rem}
    \begin{enumerate}
      \item It is obvious that $k_{jl}=k_{lj}$ and
        $\frac{a_{jl}}{t^{k_{jl}}}=\frac{-a_{lj}}{t^{k_{lj}}}$.
      \item In Boalch's paper \cite{boalch} (and also in \cite{thboalch}) are
        the \rewrite{degrees of the pairs always incremented by one}.
        We will prefer the \rewrite{other notion}, which is also used in
        Loday-Richaud's paper \cite{Loday1994}.
      \item In Loday-Richaud's book \cite[Def.4.3.6]{Loday2014} $a_{jl}$ is
        negated to be consistent with calculations at $\infty$.
        Here this is not necessary, since we use the clockwise orientation on
        $S^1$ (cf.\ Definition~\ref{defn:antiStokesDir}).
    \end{enumerate}
  \end{s-rem}
  The degrees of the elements in $\cQ(A^0)$ are defined to be  the
  \emph{levels} of $A^0$.
  The set of all levels of $A^0$ will be denoted by
  \[
    \cK=\{k_1<\dots<k_r\} \subset \Q \,.
  \]
  \begin{s-rem}
    The system $[A^0]$ is unramified if and only if $\cK\subset\Z$.
    Since we only want to consider the unramified case, this will be always the
    case.
  \end{s-rem}
\end{defn}

%%%%%%%%%%%%%%%%%%%%%%%%%%%%%%%%%%%%%%%%%%%%%%%%%%%%%%%%%%%%%%%%%%%%%%%%%%%%%%%
\pagebreak%%%%%%%%%%%%%%%%%%%%%%%%%%%%%%%%%%%%%%%%%%%%%%%%%%%%%%%%%%%%%%%%%%%%%
\subsection{Anti-Stokes directions and the Stokes group}
\marginnote{\cite[I.4]{Loday1994}}
\begin{defn}
  \marginnote{\cite[130]{hotta2008}, \cite[79]{Loday2014}}
  Let $k\in\N$ and $a\in\C$.
  We say that an exponential $e^{q(t^{-1})}$, where
  $q(t^{-1})\in\frac{a}{t^{k}}+o(t^{-k})$, has \emph{maximal decay in a
  direction $\theta\in S^1$} if and only if $ae^{-ik\tilde\theta}$ is real
  negative. \rewrite{We say that an matrix has maximal decay, if every entry
  has maximal decay.}
  \begin{comment}
    $be^0\in\C$ corresponding to has maximal decay if and only if \PROBLEM[$1$
    has maximal decay?]
  \end{comment}
\end{defn}

On the determining polynomials of $[A^0]$ we define the following (partial)
order relations:
\begin{defn}\label{defn:definingRelations}
  Let $\tilde\theta$ be a determination of $\theta\in S^1$.
  \begin{enumerate}
  \item We define the relation
    $\boldmath q_j \underset{\tilde\theta}{\prec} q_l$ to be equivalent to
    the condition
    \begin{einr}
      \rewrite{$e^{(q_j-g_l)(t^{-1})}$ is flat at $0$} in a neighbourhood of
      the direction $\tilde\theta$, i.e.\ if and only if
      $\Re(a_{jl}e^{-ik_{jl}\tilde\theta})<0$.
    \end{einr}
  \item Let us define another relation $\boldmath q_j\myrel{\tilde\theta}q_l$
    equivalent to
    \begin{einr}
      $e^{(q_j-g_l)(t^{-1})}$ is of maximal decay in the direction
      $\tilde\theta$,
    \end{einr}
    which by itself is equivalent to
    \begin{einr}
      $a_{jl}e^{-ik_{jl}\tilde\theta}$ is a real negative
      number, i.e.\ $q_j \underset{\tilde\theta}{\prec} q_l$ and
      $\Im(a_{jl}e^{-ik_{jl}\tilde\theta})=0$.
    \end{einr}
  \end{enumerate}
  \begin{s-rem}
    In the unramified case do these relations not depend on the determination
    $\tilde\theta$ of $\theta$. As a consequence we will only write
    $\underset{\theta}{\prec}$ and $\myrel{\theta}$.
  \end{s-rem}
\end{defn}
To understand the previous definition better, it is convenient to look closer at
functions of the form $f:\theta\mapsto ae^{-ik\theta}$, $k\in\Z$, corresponding
to some pair $(q_j,q_l)$.
Write $a$ as $a=|a|e^{i\arg(a)}$, thus the function $f$ writes as
\begin{align*}
  f(\theta)&=|a|e^{i(\arg(a)-k\theta)}
  \\&=|a|(\cos(\arg(a)-k\theta) + i\sin(\arg(a)-k\theta)) \,.
\end{align*}
In the Figure~\ref{fig:functionF}, we illustrate the real and the imaginary
part of $f$.
\begin{figure}[h!] %{{{
  \begin{flushright}
    \tikzmarkc{n2}{purple} here is $\Im(ae^{-ik\theta})=0$
  \end{flushright}
  \begin{center}
    \begin{tikzpicture}[scale=4]
      \pgfmathsetmacro{\k}{3}
      \pgfmathsetmacro{\argA}{0.3}
      \pgfmathsetmacro{\absA}{0.4}

      \begin{scope}[thick]
        \clip (-0.6,{\absA+0.1}) rectangle (2.6,{-\absA-0.1});
        \foreach \x in {-2,-1,...,3}{
          \pgfmathsetmacro{\s}{{\argA + \x / \k * 2 - 1/2/\k}};
          \fill[blue!10!white]
            ({\s + 2/2/\k},0) sin ({\s + 3/2/\k},{-\absA})
                              cos ({\s + 4/2/\k},0);
          \draw[blue!40!black] (\s,0) sin ({\s + 1/2/\k},\absA)
                       cos ({\s + 2/2/\k},0)
                       sin ({\s + 3/2/\k},{-\absA})
                       cos ({\s + 4/2/\k},0);
          \fill[white] ({\s + 3/2/\k},{-\absA}) circle (1pt);
          \fill[blue!40!black] ({\s + 3/2/\k},{-\absA}) circle (.4pt);
          \pgfmathsetmacro{\s}{{\argA + \x / \k * 2}};
          \draw[purple] (\s,0) sin ({\s + 1/2/\k},\absA)
                       cos ({\s + 2/2/\k},0)
                       sin ({\s + 3/2/\k},{-\absA})
                       cos ({\s + 4/2/\k},0);
          \fill[white] (\s,0) circle (1pt);
          \fill[purple] (\s,0) circle (.4pt);
          \fill[white] ({\s + 2/2/\k},0) circle (1pt);
          \fill[purple] ({\s + 2/2/\k},0) circle (.4pt);
        }
      \end{scope}
      \draw [blue!40!black,dashed]
        ({\argA+4/\k + 1/2/\k},{-\absA}) -- ({\argA+4/\k + 1/2/\k},0);
      \draw [blue!40!black,dashed]
        ({\argA+4/\k + 3/2/\k},{-\absA}) -- ({\argA+4/\k + 3/2/\k},0);
      \draw [blue!40!black, thick
            ,decorate
            ,decoration={brace,mirror,amplitude=10pt}
            ,xshift=0pt
            ,yshift=0pt]
        ({\argA+4/\k + 1/2/\k},{-\absA}) -- ({\argA+4/\k + 3/2/\k},{-\absA})
        node [midway,yshift=-7pt] {$\tikzmark{g0}$};

      \node at ({\argA+2 - 3/\k},0) {$\tikzmark{f-1}$};
      \node at ({\argA+2 - 2/\k},0) {$\tikzmark{f0}$};
      \node at ({\argA+2 - 1/\k},0) {$\tikzmark{f1}$};

      \node at ({\argA-2/\k + 2/2/\k},{-\absA}) {$\tikzmark{e-1}$};
      \node at ({\argA + 2/2/\k},{-\absA}) {$\tikzmark{e0}$};
      \node at ({\argA+2/\k + 2/2/\k},{-\absA}) {$\tikzmark{e1}$};
      \node at ({\argA+4/\k + 2/2/\k},{-\absA}) {$\tikzmark{e2}$};

      \draw[-latex'] (-0.7,0) -- (2.7,0);
      \draw[dotted] (-0.6,\absA)node[left,font=\tiny] {$|a|$} -- (2.6,\absA);
      \draw[dotted] (-0.6,{-\absA})node[left,font=\tiny] {$-|a|$} -- (2.6,{-\absA});

      \draw[dotted] ({-0.5},{-\absA}) -- ({-0.5},{\absA + 0.1})
        node [above,font=\tiny,] {-0.5 \pi};
      \draw[-latex'] ({0},{-\absA-0.1}) -- ({0},{\absA + 0.2});
      \foreach \x in {0.5,1,...,2.5}{%
        \draw[dotted] ({\x},{-\absA}) -- ({\x},{\absA + 0.1})
          node [above,font=\tiny,] {\x \pi};
      }

      \draw[thick, dotted, green!50!black] (\argA,{-\absA}) -- (\argA,{\absA + 0.1})
        node [above,font=\tiny,] {$\frac{\arg(a)}{k}$};
    \end{tikzpicture}
  \end{center}
  \begin{flushright}
    \tikzmarkc{n3}{blue} here is $q_j\underset{\theta}{\prec}q_l$
  \end{flushright}
  \begin{flushright}
    \tikzmarkc{n1}{blue} here is $q_j\myrel{\theta}q_l$
  \end{flushright}
  \begin{tikzpicture}[remember picture,overlay]
    \draw[->,blue!50!white,thick] (n3) to[out=150,in=270] (g0);
    \draw[->,purple!50!white,thick] (n2) to[out=240,in=70] (f-1);
    \draw[->,purple!50!white,thick] (n2) to[out=253,in=120] (f0);
    \draw[->,purple!50!white,thick] (n2) to[out=266,in=70] (f1);
    \draw[->,blue!50!white,thick] (n1) to[out=180,in=270] (e-1);
    \draw[->,blue!50!white,thick] (n1) to[out=170,in=270] (e0);
    \draw[->,blue!50!white,thick] (n1) to[out=160,in=270] (e1);
    % \draw[->,blue!40!black,thick] (n1) to[out=150,in=270] (e2);
  \end{tikzpicture}
  \caption{In this plot is the real part of $f(\theta)=ae^{-ik\theta}$,
    corresponding to some pair $(q_j,q_l)$, in
    \textcolor{blue!60!white}{blue} and the imaginary part in
    \textcolor{purple}{purple} sketched.
  }\label{fig:functionF}
\end{figure} %}}}

The graphs, corresponding to the flipped pair $(q_l,q_j)$ are then obtained by
the transformation $\arg(a)\to\arg(-a)=\arg(a)+\pi$, i.e.\ the shift by
$\frac{\pi}{k}$ to the right. This $\frac{\pi}{k}$ is exactly a half period,
thus the new graphs are also obtained by \rewrite{mirroring at the line $t=0$.}
\begin{rem}\label{rem:relationDistanceCondition}
  Let $k_{jl}$ be the degree of $q_j-q_l$.
  It is easy to see (cf.\ Figure~\ref{fig:functionF}), that the condition
  $q_j \underset{\theta}{\prec} q_l$ is equivalent to
  \begin{einr}
    there is a $\theta'\in U(\theta,\frac{\pi}{k_{jl}})$ such that
    $q_j\myrel{\theta}q_l$.
  \end{einr}
\end{rem}
Let us now use the defined relations to say, which are the interesting
directions of $S^1$.
\begin{defn}\label{defn:antiStokesDir}
  % \marginnote{See \cite[Def.I.4.5]{Loday1994}(for the ramified case)
  %   \cite[Def.3.2]{boalch}}
  Let $\theta\in S^1$ be an direction.
  \begin{enumerate}
    \item $\theta$ is an \emph{anti-Stokes direction} if there is at least one
      pair $(q_j,q_l)$ in $\cQ(A^0)$, which satisfies $q_j\myrel{\theta}q_l$.

      Let $\A=\{\alpha_1,\dots,\alpha_{\nu}\}$ denote the set of all
      anti-Stokes directions in a clockwise ordering. For a uniform
      notation later, \rewrite{define $\A$ to contain a single, arbitrary
        direction if $\cK=\{0\}$.}
    \item The direction $\theta$ is a \emph{Stokes direction} if there is at
      least one pair $(q_j,q_l)$ in $\cQ(A^0)$, which satisfies neither
      $q_j\underset{\theta}{\prec} q_l$ nor $q_l\underset{\theta}{\prec} q_j$.
      \begin{comment}
        Let $\S=\{\sigma_1<\cdots<\sigma_\mu\}$ be the set of Stokes directions.
      \end{comment}
  \end{enumerate}
\end{defn}
The clockwise ordering is chosen, similar to Loday-Richaud's
paper~\cite{Loday1994}, since the calculations are then compatible with the
calculations, which look at $\infty$ and take a counterclockwise ordering.
Boalch uses in~\cite{boalch} and~\cite{thboalch} the inverse ordering, but looks
also at $0$, thus there \rewrite{might be some} incompatibilities.
In Loday-Richaud's book~\cite{Loday2014} this problem is solved by an additional
minus sign for some coefficient.

We will use the Greek letter $\alpha$ whenever we want to emphasize that a
direction is an anti-Stokes direction.
For generic directions, we will use $\theta$.
In fact will most of the following definitions and
constructions work for every $\theta\in S^1$, but the Stokes group
(cf.\ Definition~\ref{defn:stokesGroup}) for example will be trivial
for every $\theta\notin\A$.
Thus the interesting directions are  only the anti-Stokes directions
$\alpha\in\A$.

\begin{lem}\label{lem:rotationalSym}%\label{rem:rotationalSymPrime}
  Let $\alpha\in\A$ together with a pair $(q_j-q_l)(t^{-1})\in\cQ(A^0)$ of
  degree $k_{jl}$, such that $q_j\myrel{\alpha}q_l$ be given.
  We then know for every $m\in\N$ that
  \[
  \underset{\alpha'}{
    \underset{\text{\rotatebox[origin=c]{-90}{$=:$}}}{%
      \underbrace{\alpha+m\frac{\pi}{k_{jl}}}}}
  \in \A \,.
  \]
  Especially is either $q_j\myrel{\alpha'}q_l$ (in the case, when $m$ is even)
  or $q_l\myrel{\alpha'}q_j$ (when $m$ is uneven) satisfied
  (see Figure~\ref{fig:functionF}).
  \begin{s-cor}
    It follows that in the case $\cK=\{k\}$, the set $\A$ has
    $\frac{\pi}{k}$-rotational symmetry.
  \end{s-cor}
\end{lem}
\begin{proof}
  \marginnote{\cite[8]{thboalch}}
  Let $(j,l)$ be a pair such that $q_j\myrel{\alpha}q_l$, i.e.\ such that
  $a_{jl}e^{-ik_{jl}\alpha}\in\R_{<0}$.
  Hence for $m\in\N$
  \begin{align*}
    a_{jl}e^{-ik_{jl}\left(\alpha+m\frac{\pi}{k_{jl}}\right)}
    &=a_{jl}e^{-ik_{jl}\alpha}e^{-im\pi}
    = \begin{cases}
      a_{jl}e^{-ik_{jl}\alpha}\in\R_{<0}
        & \text{, if $m$ is even}
    \\-a_{jl}e^{-ik_{jl}\alpha}\in\R_{>0}
        & \text{, if $m$ is uneven}
    \end{cases}
  \end{align*}
  is in the case when $m$ is even, also real and negative. In the other case,
  when $n$ is uneven, we use that $a_{jl}=-a_{lj}$ and $k_{jl}=k_{lj}$ to
  obtain
  $a_{lj}e^{-ik_{lj}\left(\alpha+m\frac{\pi}{k_{lj}}\right)} \in\R_{<0}$.

  Thus, for $\alpha':=\alpha+m\frac{\pi}{k_{jl}}$, we have $\alpha'\in\A$ since
  \begin{itemize}
    \item $q_j\myrel{\alpha'}q_l$ when $m$ is even or
    \item $q_l\myrel{\alpha'}q_j$ when $m$ is uneven.
  \end{itemize}
\end{proof}

As a \rewrite{subgroup of the stalk at $\theta$} of the in
Definition~\ref{defn:StokesSheaf} defined Stokes sheaf $\Lambda(A^0)$ we
define the Stokes group as follows.
\begin{defn}\label{defn:stokesGroup}
  Define the \emph{Stokes group}
  \[
    \Sto_\theta(A^0):=
    \left\{\phi_\theta\in\Lambda_\theta(A^0)
      \mid \phi_\theta \text{~has maximal decay at~} \theta
    \right\}
  \]
  whose elements are called \emph{Stokes germs}.
  \TODO[This is in fact a group, since\dots]
  \begin{s-rem}
    For $\theta\notin\A$ the group $\Sto_\theta(A^0)$ is trivial, since at
    $\theta$ no flat isotropy has maximal decay, but the identity.
  \end{s-rem}
\end{defn}

%%%%%%%%%%%%%%%%%%%%%%%%%%%%%%%%%%%%%%%%%%%%%%%%%%%%%%%%%%%%%%%%%%%%%%%%%%%%%%%
\subsection{Stokes matrices}\label{sec:matrixReps}
\marginnote{\cite[9f]{thboalch}, \cite[??]{Loday1994}}
Stokes matrices, which Wasow calls in his book~\cite{wasow2002asymptotic}
Stokes multipliers and Boalch calls Stokes factors in~\cite{boalch,thboalch},
arise either
\begin{einr}
  as faithful representations of Stokes germs
\end{einr}
or, if one starts by comparing the actual fundamental solutions on arcs, as
\begin{einr}
  the matrices describing the blending between two adjacent fundamental
  solutions, with some additional assumptions
  (cf.\ Definition~\cite[80]{Loday2014}).
\end{einr}
\begin{defn}\label{defn:groupOfFaithfullReps}
  Let us use
  \[
    \bdelta_{jl}:=
    \begin{cases}
      0 \in \C^{n_j\times n_l} & \text{,~if~} j\neq l
    \\\id \in \C^{n_j\times n_l} & \text{,~if~} j=l
    \end{cases}
  \]
  as a block version of Kronecker's delta corresponding to the structure of the
  normal solution $\cY_0$, which was fixed.
  Define the group
  \begin{align*}
    \SSto_\theta(A^0)= \Big\{K=(K_{jl})_{j,l\in\{1,\dots,s\}}\in\GL_n(\C) \mid
      K_{jl}=\bdelta_{jl} \text{~unless~} q_j\myrel{\theta}q_l \Big\}
  \end{align*}
  of all \emph{Stokes matrices} of $A^0$ in the direction $\theta$.
  They will arise as a faithful representation
  (cf.~\cite[Def.4.1]{hall2003lie}) of $\Sto_\theta(A^0)$.
  \begin{s-rem}\label{rem:groupOfFaithfullReps}
    There is obviously a bijection
    $\vartheta_\theta: \prod_{q_j\myrel{\theta}q_l}\C^{n_j \cdot n_l}
    \overset{\cong}{\longrightarrow} \SSto_\theta(A^0)$.
  \end{s-rem}
\end{defn}

\begin{prop}\label{prop:representation}
  \marginnote{\cite[Def.I.4.7]{Loday1994}\\\cite[78f]{Loday2014}}
  In this situation is
  \begin{align*}
    \rho_{\theta}:\Sto_\theta(A^0)&\longrightarrow\SSto_\theta(A^0)
    \\\phi_\theta
    &\longmapsto
    C_{\phi_\theta}:=\cY_{0}\phi_\theta\cY_{0}^{-1}
  \end{align*}
  \marginnote{Boalch uses $C_{\phi_\theta}:=\cY_{0}^{-1}\phi_\theta\cY_{0}$}
  an isomorphism which maps a germ of $\Sto_\theta(A^0)$ to the corresponding
  Stokes matrix $C_{\phi_\theta}$ such that
  \begin{equation}\label{eq:representation}
    \phi_\theta(t)\cY_{0}(t)=\cY_{0}(t)C_{\phi_\theta}
  \end{equation}
  near $\theta$.
  The matrix $C_{\phi_\theta}$ is then called a \emph{representation of
  $\phi_\theta$}.
  \begin{s-rem}\label{rem:representation}
    \begin{enumerate}
      \item In the ramified case does this morphism depend on the choice of the
        determination $\tilde\theta$ of $\theta$ and the corresponding choice of
        a realization of the fundamental solution with that determination of the
        argument near the direction $\theta$
        (cf.~\cite{Loday1994} or~\cite[78f]{Loday2014}).
      \item \marginnote{\cite[Defn.I.4.7]{Loday1994}}
        This construction defines also a morphism, which takes a germ
        $\phi_\theta\in\Lambda_\theta(A^0)$ into its
        unique representation matrix
        \[
          C_{\phi_\theta} \in
          \underset{\SSto_\theta(A^0)}{%
            \underset{\text{\rotatebox[origin=c]{90}{$\subset$}}}{%
              \hat\SSto_{\theta}(A^0)}}
          :=
          \left\{(K_{jl})_{j,l\in\{1,\dots,s\}}
            \in \GL_n(\C)\mid K_{jl}=\bdelta_{jl} \text{ unless }
            q_j \underset{\theta}{\prec} q_l \right\}
        \]
        and there is a bijection $\hat\vartheta_\theta:
        \prod_{q_j\underset{\theta}{\prec}q_l}\C^{n_j \cdot n_l}
        \overset{\cong}{\longrightarrow} \hat\SSto_\theta(A^0)$.
        \begin{comment}
          Does this define a local-constant sheaf
          \[
            I\mapsto \hat\SSto_{I}(A^0)
            :=
            \left\{(K_{jl})_{j,l\in\{1,\dots,s\}}
              \in \GL_n(\C)\mid K_{jl}=\bdelta_{jl} \text{ unless }
              q_j \underset{\theta}{\prec} q_l
              \text{ for some } \theta\in I\right\}
          \]
          and a skyscraper sheaf
          \[
            I\mapsto \SSto_{I}(A^0) \,.
          \]
          \PROBLEM
        \end{comment}
    \end{enumerate}
  \end{s-rem}
\end{prop}
\begin{proof}
  It is well known (cf.\ \cite[10]{thboalch}), that the morphism
  $\rho_{\theta}$, i.e.\ conjugation by the fundamental solution, relates
  solutions $\phi_\theta$ of $[\End(A^0)]=[A^0,A^0]$ to solutions of $[0,0]$
  which are the constant matrices $\GL_n(\C)$.
  Thus we have to show, that the image of $\Sto_\theta(A^0)$ under
  $\rho_{\theta}$ is $\SSto_\theta(A^0)$.

  To see that the obtained matrix has the necessary zeros, to lie in
  $\SSto_{\theta}(A^0)$ we look at Equation (\ref{eq:representation}) and
  deduce
  \begin{equation}\label{eq:repProof1}
    \phi_\theta(t)
    =t^L e^{Q(t^{-1})}C_{\phi_\theta}e^{-Q(t^{-1})}t^{-L}
  \end{equation}
  with the given choice of the argument near $\theta$.
  After decomposing $C_{\phi_\theta}$ into
  \begin{align*}
    C_{\phi_\theta}&=1_n+\begin{pmatrix}
      c_{(1,1)} & c_{(1,2)} & \cdots &\\
      c_{(2,1} & \ddots\\
      \vdots \\
      & & & c_{(s,s)}
    \end{pmatrix}
  \\&=1_n+
    \underset{C_{\phi_\theta}^{(1,1)}}{\underbrace{%
      \begin{pmatrix}
        c_{(1,1)} & 0 & \cdots &\\
        0\\
        \vdots&\\
        &
      \end{pmatrix}
    }}
    +
    \underset{C_{\phi_\theta}^{(1,2)}}{\underbrace{%
      \begin{pmatrix}
        0 & c_{(1,2)} & 0 & \cdots\\
        & 0 &\\
        &\vdots\\
        &
      \end{pmatrix}
    }}
    +\cdots+
    \underset{C_{\phi_\theta}^{(s,s)}}{\underbrace{%
      \begin{pmatrix}
        &\\
        & & & \vdots\\
        & & & 0\\
        & \cdots & 0 & c_{(s,s)}
      \end{pmatrix}
    }}
  \\&=1_n+\sum_{(l,j)}C_{\phi_\theta}^{(l,j)}
  \end{align*}
  where the $c_{(j,l)}$ are blocks\marginnote{One can ignore the block
  structure by using $1\times1$ sized blocks. But one looses the uniqueness of
  the $q_j$'s.} of size $n_j\times n_l$ which correspond to the structure of
  $Q$. After rewriting the Equation (\ref{eq:repProof1}) we get
  \[
    \phi_\theta=
      t^L\left(
        1_n+\sum_{(l,j)}C_{\phi_\theta}^{(l,j)}e^{(q_l-q_j)(t^{-1})}
      \right)t^{-L} \,.
  \]
  \begin{comment}
    \begin{align*}
      \phi_\theta(t)
      &=t^Le^{Q(t^{-1})}\left(
        1_n+C_{\phi_\theta}
      \right)e^{-Q(t^{-1})}t^{-L}
    \\&=t^Le^{Q(t^{-1})}\left(
        1_n+\sum_{(l,j)}C_{\phi_\theta}^{(l,j)}
      \right)e^{-Q(t^{-1})}t^{-L}
    \\&=t^L\left(
        1_n+\sum_{(l,j)}e^{Q(t^{-1})}C_{\phi_\theta}^{(l,j)}e^{-Q(t^{-1})}
      \right)t^{-L}
    \\&=t^L\left(
          1_n+\sum_{(l,j)}C_{\phi_\theta}^{(l,j)}e^{(q_l-q_j)(t^{-1})}
        \right)t^{-L} \,.
    \end{align*}
  \end{comment}
  Thus, for $\phi_{\theta}$ to be flat in direction $\theta$, it is
  necessary and sufficient that if $e^{(q_l-q_j)(t^{-1})}$ does not have
  maximal decay in direction $\theta$ the corresponding
  block $C_{\phi_\theta}^{(l,j)}$ vanishes.
  Thus we have seen, that $C_{\phi_\theta}$ is an
  element of $\SSto_\theta(A^0)$.

  The \textbf{surjectivity} can \rewrite{now be seen easily} since every
  constant matrix with zeros at the necessary positions characterizes a unique
  element of $\Sto_\theta(A^0)$:
  \begin{einr}
    Let $C=1_n + \sum_{(l,j)\mid q_j\myrel{\theta}q_l} C^{(l,j)}$ be an element
    of $\SSto_\theta(A^0)$.
    Then is a pre-image of $C$ given by
    $t^Le^{Q(t^{-1})}Ce^{-Q(t^{-1})}t^{-L}$ which lies in $\Sto_\theta(A^0)$,
    since it satisfies the condition discussed above.
  \end{einr}

  The map $\rho_{\tilde\theta}$ is also \textbf{injective}, since it is the
  conjugation by an invertible matrix.
\end{proof}

From the calculations in the proof it is clear that
\begin{enumerate}
  \item for $j=l$ the (diagonal) blocks $C_{\phi_\theta}^{(l,j)}$ vanish since
    $q_l-q_j=0$ does not have maximal decay and
    \label{page:firstStatementOnTheStructure}
  \item if $e^{q_j-q_l}$ has has maximal decay, then $e^{q_l-q_j}$ has not.
    Thus if $C_{\phi_\theta}^{(l,j)}$ is not equal to zero, the block
    $C_{\phi_\theta}^{(j,l)}$ is necessarily zero.
\end{enumerate}
This implies that the matrix $C_{\phi_\theta}$ is unipotent, and \rewrite{hence
is $\Sto_\theta(A^0)$ is a unipotent Lie group.}
\begin{prop}
  Some of the above results are also true for the groups
  $\hat\SSto_\theta(A^0)$. One could use these, to see that the sheaf
  $\Lambda(A^0)$ is a piecewise-constant sheaf of non-Abelian unipotent Lie
  groups (cf.~\cite[Prop.2.1]{Loday2004}).
\end{prop}

One can use the Stokes matrices to give an alternative characterization of
Stokes germs:
\begin{einr}
  a germ $\phi_\theta\in\Lambda_\theta(A^0)$ is in $\Sto_\theta(A^0)$ if and
  only if there exists a $C\in\SSto_\theta(A^0)$ such that
  $\phi_\theta=\cY_{0}C\cY_{0}^{-1}$.
\end{einr}
Formulated is this in the following corollary.
\begin{cor}
  \marginnote{\cite[Def.I.4.12]{Loday1994}}
  A germ $\phi_\theta\in\Lambda_\theta(A^0)$ is a Stokes germ, i.e.\ an element
  in $\Sto_\theta(A^0)$, if and only if it has a representation
  $C_{\phi_\theta}$ where
  \[
    C_{\phi_\theta}=
        1_n + \sum_{(l,j)\mid q_j\myrel{\theta}q_l}C_{\phi_\theta}^{(l,j)}
  \]
  and the $C_{\phi_\theta}^{(l,j)}$ have the necessary block structure, i.e.\ it
  is in $\SSto_\theta(A^0)$.
  \begin{s-rem}
    In Loday-Richaud's book~\cite[78]{Loday2014} are the elements of
    $\Sto_\theta(A^0)$ actually characterized as the flat transformations, such
    that Equation (\ref{eq:representation}) is satisfied for some unique
    constant invertible matrix $C\in\SSto_\theta(A^0)$.
  \end{s-rem}
\end{cor}

\begin{defn}
  We denote the set of \emph{levels of the germ}
  $\phi_{\theta}\in\Lambda_\theta(A^0)$ by
  \[
    \cK(\phi_\theta):= \left\{\deg(q_j-q_l)\mid C_{\phi_\theta}^{(l,j)}\neq0
      \text{ in some representation of }\phi_\theta\right\} \subset \cK \,.
  \]
  A germ $\phi_\theta$ is called a \emph{$k$-germ} when
  $\cK(\phi_{\theta})\subset\{k\}$, i.e.\ it has at most the level $k$.
\end{defn}

\begin{comment}
  \PROBLEM[This would be good]
  \begin{lem}
    Every $k$-germ in direction $\alpha$ can be extended to the Stokes arc
    $U(\frac{\pi}{k},\alpha)$.
    \begin{cor}
        Every germ $\phi_\alpha\in\Sto_\alpha(A^0)$ can be extended to the arc
        $U(\frac{\pi}{\max\cK(\phi_\alpha)},\alpha)$, i.e.\ there is a section
        $\phi\in\Gamma\left(U(\frac{\pi}{\max\cK(\phi_\alpha)},\alpha),
          \Lambda(A^0)\right)$ which has $\phi_\alpha$ as its germ at $\alpha$.
    \end{cor}
  \end{lem}
  Let $\phi_\alpha$ be a \textbf{simple} $k$-germ in the sense that it is build
  from a single block, i.e.\
  \[
    \phi_\theta=
      t^L\left(
        1_n+C_{\phi_\theta}^{(l,j)}e^{(q_l-q_j)(t^{-1})}
      \right)t^{-L} \,.
  \]
  for some pair $(j,l)$. Assume also that the block has size $1\times1$.

  Let $\phi$ be the extension of $\phi_\alpha$ around alpha, i.e.\ the matrix
  which has germ $\phi_\alpha$ at $\alpha$ and which solves $[\End A^0]$ and is
  multiplicatively flat.

  \begin{einr}
    The system $[\End A^0]$ is
    \[
    \frac{dF}{dt}=A^0F-FA^0 \,.
    \]
  \end{einr}

  \textbf{Question:} Which form has the extension $\phi$ around $\alpha$ of
  the germ $\phi_\alpha$, does it retain the structure?
  \begin{einr}
    \begin{enumerate}
    \item Look at a diagonal element $\phi^{j,j}$ an $(j,j)$:
      \begin{itemize}
      \item it satisfies $\phi_\alpha^{j,j}=1$ and
      \item is satisfies some complicated equation
        \textbf{Question:} Is it constantly $1$
        \begin{einr}
          hopefully yes
        \end{einr}
      \end{itemize}
    \item Look at an off-diagonal position at $(j,l)$:
      \begin{itemize}
      \item[case a:] $\phi_\alpha^{j,l}\neq 0$
      \item[case b:] $\phi_\alpha^{j,l}=0$
      \end{itemize}
    \end{enumerate}
  \end{einr}
\end{comment}

%%%%%%%%%%%%%%%%%%%%%%%%%%%%%%%%%%%%%%%%%%%%%%%%%%%%%%%%%%%%%%%%%%%%%%%%%%%%%%%
\subsection{Decomposition of the Stokes group by levels}
\marginnote{\cite{Loday1994}, \cite[362ff]{Martinet1991}}
\rewrite{The goal of this section is, to introduce a filtration of
$\Lambda(A^0)$, which will be restricted to $\Sto_\theta(A^0)$ and defines there
a filtration.} This leads to a decomposition of $\Sto_\theta(A^0)$ into a
semidirect product (cf.\ Proposition~\ref{prop:filtrationOfStokesGroup}).

Let us introduce a couple of notations and definitions, which coincide with the
notations used in Loday-Richaud's paper \cite{Loday1994}.
Another good resource, which uses similar notations, is for example the
paper~\cite[362f]{Martinet1991} from Martinet and Ramis.
\begin{notations}
  \marginnote{\cite[Not.I.4.15]{Loday1994},\\\cite[362]{Martinet1991}}
  For every level $k\in\cK$ and direction $\theta\in S^1$ we set
  \begin{itemize}
    \item $\Lambda^{k}(A^0)$ as the subsheaf of $\Lambda(A^0)$ of all germs,
      which are generated by $k$-germs;
    \item $\Lambda^{\leq k}(A^0)$ (resp. $\Lambda^{<k}(A^0)$ or
      $\Lambda^{\geq k}(A^0)$) as the subsheaf of $\Lambda(A^0)$ generated by
      $k'$-germs for all $k'\leq k$ (resp. $k'<k$ or $k'\geq k$).
  \end{itemize}
  Let $\star\in\{k,<k,\leq k,\dots\}$.
  The restrictions to $\Sto_\theta$ yield the groups
  \[
    \Sto_\theta^\star(A^0):=\Sto_\theta(A^0)\cap\Lambda_\theta^{\star}(A^0)
  \]
  and let us also define $\SSto^\star_\theta(A^0)$ as the groups of
  representations, which correspond to elements of  $\Sto^\star_\theta(A^0)$.
\end{notations}
Corresponding to the definitions above, one can define
$\A^\star:=\left\{\alpha\in\A\mid\Sto_\alpha^\star(A^0)\neq\{\id\}\right\}$
for $\star\in\{k,<k,\leq k,\dots\}$ and we say that \emph{$\alpha$ is bearing
the level $k$} if $\alpha\in\A^k$.
\begin{rem}
  It is clear that for every $k\in\cK$ we have the canonical inclusions
  $\A^k\hookrightarrow\A^{\leq k}$ and $\A^{<k}\hookrightarrow\A^{\leq k}$.
\end{rem}
Sometimes it is also useful to talk about the \emph{set of levels beared by an
direction $\alpha\in\A$}:
\[
  \cK_\alpha:=\left\{k\in\cK\mid\Sto_\alpha^k(A^0)\neq\{\id\}\right\} \,.
\]
\begin{cor}
  The Lemma~\ref{lem:rotationalSym} implies that from $k\in\cK_{\alpha}$ follows
  that $k\in\cK_{\alpha+m\frac{\pi}{k}}$ for $m\in\N$.
\end{cor}
Let us now study the sheaves $\Lambda^\star(A^0)$ and discuss how they
correlate \rewrite{and how they can be composed from the others}.

The following proposition can be found as~\cite[Prop.I.5.1]{Loday1994} and the
key-statement is also given in~\cite[Prop.4.10]{Martinet1991}.
\begin{prop}\label{prop:PropertiesOfStokesSheafSplitting}
  \marginnote{\cite[Prop.I.5.1]{Loday1994}}
  For any level $k\in\cK$ one has that $\Lambda^{k}(A^0)$,
  $\Lambda^{\leq k}(A^0)$ and $\Lambda^{<k}(A^0)$ are sheaves of subgroups of
  $\Lambda(A^0)$ and the sheaf $\Lambda^k(A^0)$ is normal in
  $\Lambda^{\leq k}(A^0)$.
  \begin{comment}
    A subgroup $N$ is normal in $G$ ($N\vartriangleleft G$) if it is stable
    under conjugation, i.e.
    \[
      N\vartriangleleft G \Leftrightarrow \forall n\in N \forall g\in G,
      gng^{-1}\in N ,.
    \]
  \end{comment}

  \marginnote{\cite[Proposition 10]{Martinet1991}}
  \rewrite{We even know more,} let
  \begin{itemize}
    \item $i:\Lambda^k(A^0)\hookrightarrow\Lambda^{\leq k}(A^0)$ be the
      canonical inclusion and
    \item $p:\Lambda^{\leq k}(A^0)\twoheadrightarrow\Lambda^{<k}(A^0)$ be the
      truncation to terms of levels $<k$.
  \end{itemize}
  Then does the exact sequence of sheaves
  \[
    1\longrightarrow\Lambda^k(A^0)
    \overset{i}\longrightarrow\Lambda^{\leq k}(A^0)
    \overset{p}\longrightarrow\Lambda^{<k}(A^0)
    \longrightarrow 1 \,,
  \]
  split.
\end{prop}
From the splitting of the sequence, we obtain immediately the following
decomposition into a semidirect product.
\begin{cor}\label{cor:factorStokesGerms}
  \marginnote{\cite[Cor.I.5.2]{Loday1994}}
  For any $k\in\cK$, there are the two following ways of factoring
  $\Lambda^{\leq k}(A^0)$ in a semidirect product:
  \begin{align*}
    \Lambda^{\leq k}(A^0)&\cong \Lambda^{<k}(A^0)\ltimes\Lambda^{k}(A^0)
  \\                    &\cong \Lambda^{k}(A^0)\ltimes\Lambda^{<k}(A^0)\,.
  \end{align*}
  This means that any germ $f^{\leq k}\in\Lambda^{\leq k}(A^0)$ can be uniquely
  written as
  \begin{itemize}
    \item $f^{\leq k}=f^{<k}g^k$, where $f^{<k}\in\Lambda^{<k}$ and
      $g^k\in\Lambda^k$, or
    \item $f^{\leq k}=f^kf^{<k}$, where $f^k\in\Lambda^k$ and
      $f^{<k}\in\Lambda^{<k}$.
  \end{itemize}
  \begin{s-rem}\label{rem:algFactorization}
    \marginnote{\cite[Cor.I.5.2(ii)]{Loday1994}}
    We can get the factor $f^{<k}$ common to both factorizations by truncation
    of $f^{\leq k}$ to terms of level $<k$, i.e.\ by applying the map $p$ from
    Proposition~\ref{prop:PropertiesOfStokesSheafSplitting}.
    This truncation can explicitly be achieved, in terms of Stokes matrices, by
    keeping in representations $1+\sum C^{(j,l)}$ of $f^{\leq k}$ only the
    blocks $C^{(j,l)}$ such that $\deg(q_j-q_l)<k$.

    A factorization algorithm could then be:
    \begin{einr}
      get the factor $f^{<k}$ common to both factorizations by truncation of
      $f^{\leq k}$ to terms of level $<k$ and set $g^k:=(f^{<k})^{-1}f^{\leq k}$
      and $f^k:=f^{\leq k}(f^{<k})^{-1}$.
    \end{einr}
  \end{s-rem}
  This decomposition in a semidirect product can be extended to all levels,
  since $\Lambda^{<k}(A^0)=\Lambda^{\leq\max\{k'\in\cK\mid k' < k\}}$.
  Thus
  \[
    \Lambda(A^0)\cong\underset{k\in\cK}\bigltimes\Lambda^k(A^0) \,,
  \]
  where the semidirect product is taken in an ascending or descending order of
  levels.
\end{cor}
\begin{rem}
  Loday-Richaud states in her paper \cite[Prop.I.5.3]{Loday1994} the following
  proposition, which is a more general version of
  Proposition~\ref{prop:PropertiesOfStokesSheafSplitting}.
  \begin{s-prop}
    \marginnote{\cite[Prop.I.5.3]{Loday1994}}
    For any levels $k$,$k'\in\cK$ with $k'<k$ one has:
    \begin{enumerate}
      \item the sheaf $\Lambda^{\geq k'}(A^0)\cap\Lambda^{\leq k'}(A^0)$ is
        normal in $\Lambda^{\leq k}(A^0)$;
      \item the exact sequence of sheaves
        \[
          1\longrightarrow\Lambda^{\geq k'}(A^0)\cap\Lambda^{\leq k}(A^0)
          \overset{i}\longrightarrow\Lambda^{\leq k}(A^0)
          \overset{p}\longrightarrow\Lambda^{<k'}(A^0)
          \longrightarrow 1 \,,
        \]
        where
        \begin{itemize}
          \item $i$ is the canonical inclusion and
          \item $p$ is the truncation to terms of levels $<k'$,
        \end{itemize}
        splits.
    \end{enumerate}
    \TODO[is $\Lambda^{\geq k'}(A^0)\cap\Lambda^{\leq k}(A^0)=\Lambda^k(A^0)$
    and thus the first proposition a corollary of this?]
  \end{s-prop}
  We can use this proposition to follow (cf.\ \cite[Cor.I.5.4]{Loday1994}) that
  \begin{enumerate}
    \item
      the filtration
      \[
        \Lambda^{k_r}(A^0)
        =
        \Lambda^{\geq k_r}(A^0)
        \subset
        \Lambda^{\geq k_{r-1}}(A^0)
        \subset
        \cdots
        \subset
        \Lambda^{\geq k_{1}}(A^0)
        =
        \Lambda(A^0)
      \]
      is normal and
    \item we can use this to achieve the decomposition
      \[
        \Lambda(A^0)\cong\underset{k\in\cK}\bigltimes\Lambda^k(A^0)
      \]
      taken in an arbitrary order. In fact, one can also extend the algorithm
      from Remark~\ref{rem:algFactorization} to an arbitrary order of levels.
  \end{enumerate}
\end{rem}
The important statement, which we will use later, is then the following
Proposition. It is stated by Loday-Richaud in her Paper~\cite{Loday1994} as
Proposition I.5.5 or in the Paper~\cite[Thm.4.8]{Martinet1991} by Martinet and
Ramis.
\begin{prop}\label{prop:filtrationOfStokesGroup}
  The results from above can be restricted to the Stokes groups.
  Thus, for $\alpha\in\A$, one has
  \[
    \Sto_\alpha(A^0)\cong\underset{k\in\cK_\alpha}\bigltimes\Sto_\alpha^k(A^0)
  \]
  the semidirect product being taken in an arbitrary order\comm{~(we will only
  be interested in the ascending order)}.
  \begin{s-defn}
    We will denote the map which gives the factors of this factorization by
    \[
      i_\alpha:
      \Sto_\alpha(A^0)
      \overset{\cong}\longrightarrow
      \prod_{k\in\cK_\alpha}\Sto_\alpha^k(A^0)\,,
    \]
    where the factorization is taken in ascending order.
  \end{s-defn}
  \begin{s-rem}\label{rem:filtrationOfStokesMats}
    Write $\rho_{\alpha}^k:\Sto^k_\alpha(A^0)\to\SSto^k_\alpha(A^0)$ for
    the \rewrite{restriction} of the map $\rho_{\alpha}$
    (cf.\ Proposition~\ref{prop:representation}) to the level $k$.
    \rewrite{Then, one can} denote, by ab the induced decomposition also by
    \[
      i_\alpha:
      \SSto_\alpha(A^0)
      \overset{\cong}\longrightarrow
      \prod_{k\in\cK_\alpha}\SSto_\alpha^k(A^0)
    \]
    and the corresponding diagram
    \[ \begin{tikzcd}
        \Sto_\alpha(A^0)
        \rar{i_\alpha}
        \dar{\rho_{\alpha}}
        & \prod_{k\in\cK_\alpha}\Sto_\alpha^k(A^0)\,,
        \dar{\prod_{k\in\cK}\rho_{\alpha}^k}
      \\\SSto_\alpha(A^0)
        \rar{i_\alpha}
        & \prod_{k\in\cK_\alpha}\SSto_\alpha^k(A^0)\,,
    \end{tikzcd} \]
    commutes.
  \end{s-rem}
\end{prop}

%%% Local Variables:
%%% TeX-master: "Maximilian_Huber-Masters_Thesis-with_notes.tex"
%%% End:
