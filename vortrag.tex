\chapter{Vortrag}
Here are the german notes of my Presentation of Stokes structures.
\section{Motivation} %{{{
Wir wollen \textbf{irregulär singulärere meromorphe Zusammenhänge}, welche zu
einem vorgegebenen `gutem Modell' \textbf{formal isomorph} sind, betrachten und
fragen uns, wie wir
\begin{center}
  \textbf{Isomorphieklassen dieser klassifizieren.}
\end{center}

Es stellt sich heraus, das die dafür benötigte Information genau durch die
\textbf{Stokes Struktur} gegeben ist. 

\begin{comment}
  \TODO[Siehe \textbf{\cite{van2003galois}: Chapter 7} für mehr Motivation]
  DGLs mit konvergenten Einträgen haben manchmal eine Lösung die dann nur
  formal ist.
\end{comment}

\begin{comment}
  Um diese zu erklären wir zunächst das Konzept einer \textbf{Asymptotischen
  Erweiterung} erläutert, danach sollen die Stokes Strukturen in einer
  \textbf{Matrix-} sowie einer \textbf{Garben-Variante} definiert werden.
  Zum Schluss werden noch die essentiellen Theoreme erwähnt.
\end{comment}
%}}}
\section{Meromorphe Zusammenhänge} %{{{
Sei
\begin{itemize}
  \item $M$ eine Riemann-Fläche\marginnote{$\C$},
  \item $D$ ein effektiver Divisor\marginnote{$1(0)$} auf $M$ und
  \item ein holomorphes Bündel $\sE$ ist ein lokal freier $\cO_M$-Modul
    \begin{itemize}
      \item $E$ das entsprechende Vektor Bündel über $M$
    \end{itemize}
  \item ein meromorphes Bündel $\sM$ ist ein lokal freier $\cO_M(*Z)$-Modul
\end{itemize}
\TODO[Divisor $D$ oder hyperfläche $Z$?]
\begin{defn}
  Ein \emph{meromorpher Zusammenhang auf $E$ mit Polen, welche durch $D$
  beschränkt sind,} ist eine $\C$-lineare Derivation
  \[
    \nabla:\sE\to\Omega_X^1(D)\otimes\sE
  \]
  welcher
  \begin{comment}
    \begin{itemize}
      \item für
        \begin{itemize}
          \item alle offenen Mengen $U\subset M$,
          \item alle Schnitte $s\in\Gamma(U,\sE)$ und
          \item alle holomorphen Funktionen $f\in\cO(U)$
        \end{itemize}
    \end{itemize}
  \end{comment}
  die \textbf{Leibnitz Regel}
  \[
    \nabla(fs)=f\nabla s+(df)\otimes s
    \qquad
    \in\Gamma(U,\Omega_M^1\otimes_{\cO_M}\sE)
  \]
  erfüllt.

  \TODO[(Iso)morphism of connections]

  \begin{comment}
    \begin{defn}
      Ein Zusammenhang heiß \emph{flach} oder \emph{integrabel} falls
      \begin{itemize}
        \item seine Krümmung $R_\nabla\equiv0$
          wobei
          \begin{itemize}
            \item $R_\nabla
              :=\nabla\circ\nabla:\sE\to\Omega_M^2\otimes_{\cO_M}\sE$
          \end{itemize}
      \end{itemize}
      \begin{rem}[ \cite{sabbah2007isomonodromic} Rem 0.12.5]
        Falls $\dim(M)=1$ ist jeder Zusammenhang flach.
      \end{rem}
    \end{defn}
  \end{comment}
\end{defn}
\begin{rem}
  Lokal sieht
  \begin{itemize}
    \item $\sE$ wie $U\times\C^n$ aus und
    \item ein $s\in\sE$ wie 
      $\begin{pmatrix}f_{1}\\ \vdots\\ f_{n} \end{pmatrix}=:f$
      \begin{itemize}
        \item mit $f_j$ holomorph.
      \end{itemize}
  \end{itemize}
  Dann
  \[
    \nabla s=df - A(z)f
  \]
  wobei $A(z)$ eine holomorphe\TODO[meromorph?] matrix-wertige Funktion ist.
  \begin{comment}
    Wollen diese Klassifizieren. Klassifiziere diese durch die Lösung von
    $\nabla s=0$. Dies ist eine DGL (ODE).
  \end{comment}
\end{rem}
\begin{comment}
  \TODO[$A^0=\dots$]
  \begin{defn}
    \begin{itemize}
      \item $G\{z\}:=\GL_n(\C\{z\})$ \emph{locale analytischen Gauge
        Tranformationen}
        \begin{itemize}
          \item mit Wirkung $F[A^0]=(dF)F^{-1}+FA^0F^{-1}$
        \end{itemize}
      \item $\hat G:=\GL_n(\C\llbracket z\rrbracket)$ \emph{formale
        Tranformationen}
    \end{itemize}
    definiere
    \[
      \Syst(A^0)
      :=\left\{A \mid A=\hat F[A^0]\text{ für ein }\hat F\in\hat G\right\}
    \]
  \end{defn}
  \begin{center}
    \textbf{Wir sind interessiert in $\Syst(A^0)/\C\{z\}$}
  \end{center}
\end{comment}
\subsection{Irregularität von meromorphen Zusammenhängen} %{{{
Wir wollen irregulär singuläre meromorphe Zusammenhänge betrachten. Die regulär
singulären sind durch \TODO

\TODO[\cite{sabbah2007isomonodromic}: p. 86]
\TODO[\cite{sabbah2007isomonodromic}: Def II.2.24]
%}}}
\subsection{Modelle und formale Zerlegung}%{{{
\begin{comment}
  Durch das Tensorieren mit formalen Potenzreihen erhält man einen Funktor.
\end{comment}
%}}}
\begin{defn}
  Sei $(\sM,\nabla)$ ein meromorphen Bündel mit Zusammenhang.
  \TODO[oder merom zsh?]
  \begin{itemize}
    \item $(\sM,\nabla)$ ist ein \emph{Model} falls
      \[
        (\sM,\nabla)
        \cong
        \bigoplus_\phi(\sE^\phi\otimes\sR_\phi)
      \]
      wobei
      \begin{itemize}
        \item \TODO[$\sE^\phi$ wird gelöst von\dots]
        \item die $\sR_\phi$ regulär singulär sind und
        \item die
          $\phi\in\C\{t\}[t^{-1}]$
          \begin{itemize}
            \item keinen holomorphen Anteil haben und
            \item paarweise unterschiedlich sind.
          \end{itemize}
      \end{itemize}
      Die $\sE^\phi\otimes\sR_\phi$ heißen elementare Zusammenhänge.
  \end{itemize}
\end{defn}
\begin{comment}
  \begin{thm}[Levelt-Turittin]
    Zu einem formalem meromorphen Zusammenhang gibt es, bis auf Zurückziehen,
    immer einen Isomorphismus zu einer direkten Summe formaler elementarer
    Zusammenhänge.
    \begin{rem}
      \textbf{Übergang:}
      Aber keinen konvergenten lift.
      Sektorweise aber schon mit asymptotischer Analysis
    \end{rem}
  \end{thm}
\end{comment}
%}}}
%}}}
\section{Asymptotische Erweiterungen} %{{{
\begin{comment}
  See \cite{van2003galois}
\end{comment}
\begin{paracol}{2}
\begin{defn} Sei \textcolor{red!60!black}{$U$} ein offener Intervall von $S^1$
  \begin{align*}
    \Delta_r^*(U):= \{z\in\Delta_r&\mid z=\rho e^{i\theta}
                                \\&,0<\rho<r
                                \\&,\theta\in U\}
  \end{align*}
\end{defn}
\switchcolumn
  \begin{tikzpicture}[scale=3]
    \node (zero) at (0,0) {};
    \node[below left] at (zero) {$0$};
    \draw[blue,dashed] (zero) circle (1cm);

    \filldraw[fill=green!20!white
      ,draw=green!60!black
      ,thick
    ,path fading=west] (0,0)
    -- ({cos( -30 )*.7},{sin( -30 )*.7}) arc (-30:70:.7) -- cycle;
    \node[green!40!black] at (.4,.3) {$\Delta_r^*(U)$};
    \node[green!40!black] at (.3,-.25) {$r$};

    \draw[thick,red!60!black] ({cos( -30 )},{sin( -30 )}) arc (-30:70:1);

    \node[red!60!black,right] at (1,0) {$U$};

    \fill[white] (zero) circle (1.5pt);
    \fill (zero) circle (.7pt);
  \end{tikzpicture}
\end{paracol}
\begin{defn}
  \def\myN{\textbf{\textcolor{blue!40!black}{N}}}
  \def\mySect{\textcolor{red!40!black}{W}}
  \def\myConst{\textcolor{green!40!black}{C(\myN,\mySect)}}
  $f$ hat die \textbf{formale Laurent Reihe} $\sum_{n\geq n_0}c_nz^n$ als
  \emph{asymptotische Erweiterung}, falls
  \begin{itemize}
    \item für alle $\myN\geq0$ und
    \item für alle abgeschlossenen Untersektoren $\mySect$ in $\Delta_r^*(U)$
  \end{itemize}
  eine Konstante $\myConst$ existiert, so dass
  \[
    \left|
      f(x)-\sum_{n_0\leq n\leq \myN-1}c_nz^n
    \right|
    \leq \myConst|z|^{\myN} \qquad \text{ für alle } z\in \mySect
  \]
  \begin{comment}
    \Leftrightarrow{}
    \[
      \lim_{z\to0,z\in{\mySect}}
      |z|^{-(\myN-1)}
      \left|
      f(x)-\sum_{n_0\leq n\leq \myN-1}c_nz^n
      \right|=0
      \qquad \text{ für alle } z\in \mySect
    \]
  \end{comment}
  Damit bekommt man die Garbe $\sA$ auf $S^1$, die zu einem offenem Intervall
  $U$ von $S^1$ die Menge $\sA(U)\subset\cO(\Delta_r^*(U))$ der Funktionen mit
  asymptotischer Entwicklung auf dem entsprechendem Sektor zuordnet.
\end{defn}

\begin{lem}[Borel-Ritt]
  Für jedes \textbf{echtes} offenes Intervall $X$ von $S^1$ ist die Abbildung
  \[
    \sA(U)\to \C((z))
  \]
  eine surjektion.
  \begin{comment}
    \begin{itemize}
      \item $x\mapsto e^{-\frac{1}{x}}$ hat verschwindende asymptotische
        Erweiterung in Sektoren um $\theta=0$
        \begin{itemize}
          \item geht schneller als jedes Polynom gegen $0$
        \end{itemize}
      \item $ 0\to \sA^{<0} \to \sA \overset{T}{\to}
          \underset{\hat\cO_{S^1}}{\underbrace{\pi^{-1}\hat\cO_{D}}} \to 0$
    \end{itemize}
  \end{comment}
\end{lem}
\begin{paracol}{2} %%%%%%%%%%%%%%%%%%%%%%%%%%%%%%%%%%%%%%%%%%%%%%%%%%%%%%%%%%%%
  \begin{thm}
    \begin{itemize}
      \item Für jedes $e^{i\theta^o}\in S^1$,
    \end{itemize}
    existiert ein Isomorphismus
    \[
      \tilde\lambda_{\theta^o}: \tilde\sM_{\theta^o}
      \overset{\sim}{\longrightarrow}\tilde\sM^{good}_{\theta^o},
    \]
    so dass das Diagramm
    \[ \begin{tikzcd}
        \tilde\sM_{\theta^o} \dar \rar{\tilde\lambda_{\theta^o}} &
        \tilde\sM^{good}_{\theta^o} \dar
        \\\hat\sM \rar{\hat\lambda} &
        \hat\sM^{good}
    \end{tikzcd} \]
    kommutiert
  \end{thm}
\switchcolumn %%%%%%%%%%%%%%%%%%%%%%%%%%%%%%%%%%%%%%%%%%%%%%%%%%%%%%%%%%%%%%%%%
\begin{thm}
  Zu jedem $\theta\in\S^1$ und jedem genügend kleinem Intervall um $V$ gilt
  \[
    \sA(V)\otimes\sM\cong\sA(V)\otimes\bigoplus_\phi(\sE^\phi\otimes\sR_\phi)
  \]
\end{thm}
\end{paracol} %%%%%%%%%%%%%%%%%%%%%%%%%%%%%%%%%%%%%%%%%%%%%%%%%%%%%%%%%%%%%%%%%
%}}}
\section{Stokes Strukturen} %{{{
Sei
\begin{itemize}
  \item $(\sM^{good},\nabla^{good})$ ein \textbf{Modell}
    \begin{itemize}
      \item $\sM^{good}$ ein meromorphes Bündel
        \begin{itemize}
          \item auf $D$
          \item mit Pol bei $\{0\}$
        \end{itemize}
        mit einem \textbf{flachem} Zusammenhang\marginnote{holomorph}
        \[
          \nabla^{good}:\sM\to\Omega_{D\times X}^1\otimes \sM
        \]
    \end{itemize}
\end{itemize}
\begin{defn}
  \begin{itemize}
    \item Definiere auf $S^1$ die Garbe
      \begin{itemize}
        \item $\Aut^{<0}(
          \underset{\sA_{\tilde D}\otimes\sM^{good}}
          {\underbrace{\tilde\sM^{good}}})$ der Automorphismen
          welche
          \begin{itemize}
            \item mit dem Zusammenhang kompatibel\TODO[in Formeln] sind und
            \item formal äquivalent zur Identität sind.
          \end{itemize}
      \end{itemize}
      Die Schnitte heißen \emph{Stokes Matrizen}.
    \item definiere
      \[
        \St(\sM^{good})
          :=H^1\left(S^1,\Aut^{<0}\left(\tilde \sM^{good}\right)\right)
      \]
  \end{itemize}
\end{defn}

\begin{comment}
  Definiere $\sH_X$ als die (Prä-)Garbe auf $X$ mit
  \[
    U\mapsto\sH_X(U)=\{(\sM,\nabla,\hat f) \mid \text{definiert auf $U$}\}
    /\sim
  \]
\end{comment}
Definiere 
\[
  \sH:=\left\{(\sM,\nabla,\hat f)\right\}/\sim
\]
\TODO[definieren isomorphie von `Keimen']

%}}}
\subsection{Theorem 1} %{{{
Sei $(\sM,\nabla,\hat f)\in\sH$. Dann gibt es eine Überdeckung
$\mathfrak{W}$ von $S^1$ und für jedes $W_i\in\mathfrak{W}$ einen
Lift\footnote{$\hat f_i=\hat f$} von $\hat f$
\[
  f_i:(\tilde\sM,\tilde\nabla)_{W_i}
  \overset{\sim}{\to}
  (\tilde\sM^{good},\tilde\nabla^{good})_{W_i} \,.
\]
Dann ist$(f_jf_i^{-1})_{i,j}$ ein Kozykel in der Garbe
$\Aut^{<0}(\tilde\sM^{good})$ relativ zur Überdeckung $\mathfrak{W}$.
Damit haben wir eine Abbildung
\[
  \sH\to H^1(S^1,\Aut^{<0}(\hat \sM^{good}))
\]
\begin{thm}
  Die Abbildung
  \[
    \sH\to\St(\sM^{good})
  \]
  ist ein Isomorphismus von punktierter Mengen.
\end{thm}
%}}}
\subsection{Theorem 2} %{{{
\TODO[From $\sH$ to $\Syst(A^0)/G\{z\}$]
\begin{comment}
  Sei $x^0\in X$ dann $\sH_{X,x_0}=\cH(A^0)$ für passendes $A^0$.
\end{comment}

\begin{thm}
  Es gibt einen Isomorphismus
  \[
    \Syst(A^0)/G\{z\}\cong(U_+\times U_-)^{k-1}/T
  \]
\end{thm}

\begin{center}
  \begin{tikzpicture}[scale=3]
    \node[] (zero) at (0,0) {};
    \fill[fill=green!20!white] (0,0) -- (1,0) arc (0:60:1.0cm) -- cycle;
    \draw[blue] (zero) circle (1cm);

    \foreach \w/\str in {10/$d_1\in S^1$,
                         20/$d_2$,
                         45/$d_3$,
                         55/$d_l$}
    {\draw[thick,purple!\w!blue,path fading=west]
        (0,0) -- +({cos( \w )},{sin( \w )}) node[right] {\str};
     \fill[blue!20!white] ({cos( \w )},{sin( \w )}) circle (1pt);
     \foreach \sep in {60,120,180,240,300}
     {\draw[green!20!white,thick] (zero) -- +({cos( \sep )},{sin( \sep )});
      \draw[purple!\w!blue] (0,0) -- +({cos( \w + \sep )},{sin( \w + \sep )});
      \fill[blue!20!white] ({cos( \w + \sep )},{sin( \w + \sep )}) circle (1pt);
     }
    };

    \foreach \sep/\str in {0/$1$
                          ,60/$2$
                          ,120/$k-1$
                          ,180/$4$
                          ,240/$5$
                          ,300/$2(k-1)$}
    {\node[green!40!black]
      at ({.6 * cos( \sep + 30 )},{.5 * sin( \sep + 30)}) {\str};
    };

    \fill[yellow!60!black] (0.8,0.07) circle (1pt);
    \node[yellow!60!black,right] at (0.8,0.07) {$p$};

    \fill (zero) circle (1pt);
  \end{tikzpicture}
\end{center}
%}}}
