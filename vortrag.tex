\chapter{Vortrag}
\section{Motivation} %{{{
Wir wollen \textbf{irregulär singulärere meromorphe Zusammenhänge}, welche zu
einem vorgegebenen gutem Modell formal isomorph sind, betrachten und fragen
uns, wie wir Isomorphieklassen dieser klassifizieren.

Es stellt sich heraus, das die dafür benötigte Information genau durch die
\textbf{Stokes Struktur} gegeben ist.

\begin{comment}
  Um diese zu erklären wir zunächst das Konzept einer \textbf{Asymptotischen
  Erweiterung} erläutert, danach sollen die Stokes Strukturen in einer
  \textbf{Matrix-} sowie einer \textbf{Garben-Variante} definiert werden.
  Zum Schluss werden noch die essentiellen Theoreme erwähnt.
\end{comment}
%}}}
\section{Meromorphe Zusammenhänge} %{{{
Sei
\begin{itemize}
  \item $M$ eine Riemann-Fläche,
  \item $D$ ein effektiver Divisor auf $M$ und
  \item $E$ ein holomorphes Vektor Bündel über $M$
    \begin{itemize}
      \item mit der Garbe $\cE$ als Garbe der holomorphen Schnitte.
    \end{itemize}
\end{itemize}
\begin{defn}
  Ein \emph{meromorpher Zusammenhang auf $E$ mit Polen, welche durch $D$
  beschränkt sind,} ist ein Differentialoperator
  \[
    \nabla:\cE\to\Omega_X^1(D)\otimes\cE
  \]
  welcher
  \begin{itemize}
    \item für
      \begin{itemize}
        \item alle offenen Mengen $U\subset M$,
        \item alle Schnitte $s\in\Gamma(U,\cE)$ und
        \item alle holomorphen Funktionen $f\in\cO(U)$
      \end{itemize}
  \end{itemize}
  die \textbf{Leibnitz Regel}
  \[
    \nabla(fs)=f\nabla s+(df)\otimes s
    \qquad
    \in\Gamma(U,\Omega_M^1\otimes_{\cO_M}\cE)
  \]
  erfüllt.
  \begin{comment}
    \begin{defn}
      Ein Zusammenhang heiß \emph{flach} oder \emph{integrabel} falls
      \begin{itemize}
        \item seine Krümmung $R_\nabla\equiv0$
          wobei
          \begin{itemize}
            \item $R_\nabla
              :=\nabla\circ\nabla:\cE\to\Omega_M^2\otimes_{\cO_M}\cE$
          \end{itemize}
      \end{itemize}
      \begin{rem}[ \cite{sabbah2007isomonodromic} Rem 0.12.5]
        Falls $\dim(M)=1$ ist jeder Zusammenhang flach.
      \end{rem}
    \end{defn}
  \end{comment}
\end{defn}
\subsection{Irregularität von meromorphen Zusammenhängen} %{{{
\TODO
%}}}
\subsection{Modelle und formale Zerlegung}%{{{
\begin{defn}
  Let $(\cM,\nabla)$ be a \textbf{germ} of a meromorphic bundle with
  connection.
  \begin{itemize}
    \item A germ $(\cM,\nabla)$ is \emph{elementary} if it is isomorphic to
      some germ
      \[
        (\cE^\phi,\nabla)\otimes(\cR,\nabla)
      \]
      where
      \begin{itemize}
        \item $(\cR,\nabla)$ ha regular singularity along $\{0\}\times X$
      \end{itemize}
    \item $(\cM,\nabla)$ is a \emph{model} if it is isomorphic to a direct sum
      of elementary models, written as
      \[
        \bigoplus_\phi(\cE^\phi\otimes\cR_\phi)
      \]
      where
      \begin{itemize}
        \item the meromorphic bundles with connection $\cR_\phi$ have regular
          singularity and
        \item the
          $\phi\in\C\{t\textcolor{gray}{\underset{\text{parameter}}
            {\underbrace{,x_1,\ldots,x_n}}}\}[t^{-1}]$
          \begin{itemize}
            \item have no holomorphic part and
            \item are pairwise distinct.
          \end{itemize}
      \end{itemize}
  \end{itemize}
  \begin{comment}
    \begin{itemize}
      \item We will say that a model is \emph{good} if,
        \begin{itemize}
          \item for all $\phi\neq\psi$
            \begin{itemize}
              \item such that $\cR_\phi$, $\cR_\psi$ are nonzero,
            \end{itemize}
            the order of the pole along $t=0$ of $(\phi-\psi)(t,x)$ does not
            depend on $x$ being in some neighbourhood of $x^o$.
        \end{itemize}
    \end{itemize}
  \end{comment}
\end{defn}
%}}}
%}}}
\section{Asymptotische Erweiterungen} %{{{
%}}}
\section{Stokes Strukturen} %{{{
%}}}
\section{Theoreme} %{{{
\begin{thm}
  Die Abbildung
  \[
    \sH_X\to\St_X(\sM^{good})
  \]
  ist ein Isomorphismus von Garben punktierter Mengen.
\end{thm}

\begin{thm}
  Es gibt einen Isomorphismus
  \[
    \Syst(A^0)/G\{z\}\cong(U_+\times U_-)^{k-1}/T
  \]
\end{thm}
%}}}
