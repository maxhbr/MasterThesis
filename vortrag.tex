\chapter{Vortrag}
\section{Motivation} %{{{
Wir wollen \textbf{irregulär singulärere meromorphe Zusammenhänge}, welche zu
einem vorgegebenen gutem Modell formal isomorph sind, betrachten und fragen
uns, wie wir Isomorphieklassen dieser klassifizieren.

Es stellt sich heraus, das die dafür benötigte Information genau durch die
\textbf{Stokes Struktur} gegeben ist.

\begin{comment}
  Um diese zu erklären wir zunächst das Konzept einer \textbf{Asymptotischen
  Erweiterung} erläutert, danach sollen die Stokes Strukturen in einer
  \textbf{Matrix-} sowie einer \textbf{Garben-Variante} definiert werden.
  Zum Schluss werden noch die essentiellen Theoreme erwähnt.
\end{comment}
%}}}
\section{Meromorphe Zusammenhänge} %{{{
Sei
\begin{itemize}
  \item $M$ eine Riemann-Fläche\marginnote{$\C$},
  \item $D$ ein effektiver Divisor\marginnote{$1(0)$} auf $M$ und
  \item $E$ ein holomorphes Vektor Bündel über $M$
    \begin{itemize}
      \item mit der Garbe $\cE$ als Garbe der holomorphen Schnitte.
    \end{itemize}
  \item \TODO[meromorphes Bündel!?!]
\end{itemize}
\begin{defn}
  Ein \emph{meromorpher Zusammenhang auf $E$ mit Polen, welche durch $D$
  beschränkt sind,} ist ein Differentialoperator
  \[
    \nabla:\cE\to\Omega_X^1(D)\otimes\cE
  \]
  welcher
  \begin{comment}
    \begin{itemize}
      \item für
        \begin{itemize}
          \item alle offenen Mengen $U\subset M$,
          \item alle Schnitte $s\in\Gamma(U,\cE)$ und
          \item alle holomorphen Funktionen $f\in\cO(U)$
        \end{itemize}
    \end{itemize}
  \end{comment}
  die \textbf{Leibnitz Regel}
  \[
    \nabla(fs)=f\nabla s+(df)\otimes s
    \qquad
    \in\Gamma(U,\Omega_M^1\otimes_{\cO_M}\cE)
  \]
  erfüllt.
  \begin{comment}
    \begin{defn}
      Ein Zusammenhang heiß \emph{flach} oder \emph{integrabel} falls
      \begin{itemize}
        \item seine Krümmung $R_\nabla\equiv0$
          wobei
          \begin{itemize}
            \item $R_\nabla
              :=\nabla\circ\nabla:\cE\to\Omega_M^2\otimes_{\cO_M}\cE$
          \end{itemize}
      \end{itemize}
      \begin{rem}[ \cite{sabbah2007isomonodromic} Rem 0.12.5]
        Falls $\dim(M)=1$ ist jeder Zusammenhang flach.
      \end{rem}
    \end{defn}
  \end{comment}
\end{defn}
\begin{rem}
  Lokal sieht $\cE$ wie $U\times\C^n$ aus und ein $s\in\cE$ wie 
  $\begin{pmatrix}f_{1}\\ \vdots\\ f_{n} \end{pmatrix}=:f$ mit holomorphen
  Funktionen $f_j$.
  Dann
  \[
    \nabla s=df - A(z)f
  \]
  wobei $A(z)$ eine holomorphe matrix-wertige Funktion ist.
  \begin{comment}
    Wollen diese Klassifizieren. Klassifiziere diese durch die Lösung von
    $\nabla s=0$. Dies ist eine DGL (ODE).
  \end{comment}
\end{rem}
\TODO[Zusammenhang zu (Partiellen-)DGLs]
\subsection{Formalisieren} %{{{
\TODO
%}}}
\subsection{Irregularität von meromorphen Zusammenhängen} %{{{
Wir wollen irregulär singuläre meromorphe Zusammenhänge betrachten. Die regulär
singulären sind durch \TODO
\TODO

\TODO[\cite{sabbah2007isomonodromic}: p. 86]
\TODO[\cite{sabbah2007isomonodromic}: Def II.2.24]
%}}}
\subsection{Modelle und formale Zerlegung}%{{{
\begin{defn}
  Let $(\cM,\nabla)$ be a \textbf{germ} of a meromorphic bundle with
  connection.
  \begin{itemize}
    \item A germ $(\cM,\nabla)$ is \emph{elementary} if it is isomorphic to
      some germ
      \[
        (\cE^\phi,\nabla)\otimes(\cR,\nabla)
      \]
      where
      \begin{itemize}
        \item $(\cR,\nabla)$ ha regular singularity along $\{0\}\times X$
      \end{itemize}
    \item $(\cM,\nabla)$ is a \emph{model} if it is isomorphic to a direct sum
      of elementary models, written as
      \[
        \bigoplus_\phi(\cE^\phi\otimes\cR_\phi)
      \]
      where
      \begin{itemize}
        \item the meromorphic bundles with connection $\cR_\phi$ have regular
          singularity and
        \item the
          $\phi\in\C\{t\textcolor{gray}{\underset{\text{parameter}}
            {\underbrace{,x_1,\ldots,x_n}}}\}[t^{-1}]$
          \begin{itemize}
            \item have no holomorphic part and
            \item are pairwise distinct.
          \end{itemize}
      \end{itemize}
  \end{itemize}
  \begin{comment}
    \begin{itemize}
      \item We will say that a model is \emph{good} if,
        \begin{itemize}
          \item for all $\phi\neq\psi$
            \begin{itemize}
              \item such that $\cR_\phi$, $\cR_\psi$ are nonzero,
            \end{itemize}
            the order of the pole along $t=0$ of $(\phi-\psi)(t,x)$ does not
            depend on $x$ being in some neighbourhood of $x^o$.
        \end{itemize}
    \end{itemize}
  \end{comment}
\end{defn}
%}}}
%}}}
\section{Asymptotische Erweiterungen} %{{{
Let \textcolor{red!60!black}{$U$} be an open interval in $S^1$
\begin{defn}
  \begin{itemize}
    \item $\Delta_r^*(U):=
      \{z\in\Delta_r\mid z=\rho e^{i\theta},0<\rho<r,\theta\in U\}$%
      \begin{center}
        \begin{tikzpicture}[scale=3]
          \node (zero) at (0,0) {};
          \node[below left] at (zero) {$0$};
          \draw[blue,dashed] (zero) circle (1cm);

          \filldraw[fill=green!20!white
            ,draw=green!60!black
            ,thick
          ,path fading=west] (0,0)
          -- ({cos( -30 )*.7},{sin( -30 )*.7}) arc (-30:70:.7) -- cycle;
          \node[green!40!black] at (.4,.3) {$\Delta_r^*(U)$};
          \node[green!40!black] at (.3,-.25) {$r$};

          \draw[thick,red!60!black] ({cos( -30 )},{sin( -30 )}) arc (-30:70:1);

          \node[red!60!black,right] at (1,0) {$U$};

          \fill[white] (zero) circle (1.5pt);
          \fill (zero) circle (.7pt);
        \end{tikzpicture}
      \end{center}
  \end{itemize}
\end{defn}
\begin{defn}
  \def\myN{\textbf{\textcolor{blue!40!black}{N}}}
  \def\mySect{\textcolor{red!40!black}{W}}
  \def\myConst{\textcolor{green!40!black}{C(\myN,\mySect)}}
  $f$ hat die \textbf{formale Laurent Reihe} $\sum_{n\geq n_0}c_nz^n$ als
  \emph{asymptotische Erweiterung}, falls
  \begin{itemize}
    \item für alle $\myN\geq0$ und
    \item für alle abgeschlossenen Untersektoren $\mySect$ in $\Delta_r^*(U)$
  \end{itemize}
  eine Konstante $\myConst$ existiert, so dass
  \[
    \left|
      f(x)-\sum_{n_0\leq n\leq \myN-1}c_nz^n
    \right|
    \leq \myConst|z|^{\myN} \qquad \text{ für alle } z\in \mySect
  \]
  \begin{comment}
    \Leftrightarrow{}
    \[
      \lim_{z\to0,z\in{\mySect}}
      |z|^{-(\myN-1)}
      \left|
      f(x)-\sum_{n_0\leq n\leq \myN-1}c_nz^n
      \right|=0
      \qquad \text{ für alle } z\in \mySect
    \]
  \end{comment}
\end{defn}

\TODO[The sheaf $\sA$]

\begin{lem}[Borel-Ritt]
  \TODO
\end{lem}
\begin{thm}[Sectorial lifting]
  \TODO
\end{thm}
%}}}
\section{Stokes Strukturen} %{{{
\TODO[Trivial parameter space?]
Sei
\begin{itemize}
  \item $X$ eine analytische Mannigfaltigkeit mit Koordinaten $x_1,\dots,x_n$
    \marginnote{Parameter Raum}
  \item $\sM^{good}$ ein meromorphes Bündel
    \begin{itemize}
      \item auf $D\times X$
      \item mit Polen entlang $\{0\}\times X$
    \end{itemize}
    mit einem \textbf{flachem} Zusammenhang
    \[
      \nabla^{good}:\sM\to\Omega_{D\times X}^1\otimes \sM
    \]
\end{itemize}
$(\sM^{good},\nabla^{good})$ soll ein \textbf{gutes Modell} in der Umgebung von
jedem $x^0\in X$ sein. Das bedeutet \TODO

\TODO[The sheaf $\Aut^{<X}(\sM^{good})$]

\begin{defn}
  Die Garbe $\Aut^{<X}(\sM^{good})$ ist die Garbe
  \begin{itemize}
    \item auf $S^1\otimes X$
    \item von Automorphismen von
      \[
        \tilde\sM^{good}:=\sA_{\tilde D\times X}\otimes\sM^{good}
      \]
      welche
      \begin{itemize}
        \item mit dem Zusammenhang kompatibel\TODO[in Formeln] sind und
        \item formal äquivalent zur Identität sind.
      \end{itemize}
  \end{itemize}
  Die Schnitte heißen \emph{Stokes Matrizen}.
\end{defn}
\begin{defn}
  Definiere die \emph{Stokes Garbe\footnote{von punktierten Mengen}
  $\St_X(\sM^{good})$ auf $X$} als die Garbe zur Prägarbe
  \[
    U\mapsto H^1(S^1\times U,\Aut^{<X}(\tilde \sM^{good}))
  \]
  \begin{thm}
    Die Stokes Garbe ist \textbf{lokal konstant}.
  \end{thm}
\end{defn}

\TODO[The sheaf $\sH_X$]
%}}}
\section{Theoreme} %{{{
\begin{thm}
  Die Stokes-Garbe $\St_X(\sM^{good})$ ist eine lokal konstante Garbe von
  punktierten Mengen.
\end{thm}
\begin{thm}
  Die Abbildung
  \[
    \sH_X\to\St_X(\sM^{good})
  \]
  ist ein Isomorphismus von Garben punktierter Mengen.
\end{thm}

\TODO[From $\sH_X$ to $\Syst(A^0)/G\{z\}$]

\begin{thm}
  Es gibt einen Isomorphismus
  \[
    \Syst(A^0)/G\{z\}\cong(U_+\times U_-)^{k-1}/T
  \]
\end{thm}
%}}}

