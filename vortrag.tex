\chapter{Vortrag: Stokes Strukturen meromorpher Zusammenhänge}
Here are the german notes of my Presentation of Stokes structures.
\setcounter{section}{-1} \section{Motivation} %{{{
% \marginnote{\tiny DGLs mit konvergenten Einträgen haben manchmal eine formale
% Lösung die dann dann aber nicht konvergiert.}
% Wir wollen \textbf{irregulär singulärere meromorphe Zusammenhänge}, welche zu
% einem vorgegebenen `gutem Modell' \textbf{formal isomorph} sind, betrachten und
% fragen uns, wie wir
% \begin{center}
%   \textbf{Isomorphieklassen dieser beschreiben können.}
% \end{center}

% Es stellt sich heraus, das hier die formalen Invarianten nicht ausreichen.
% Die benötigte Information ist aber genau durch die \textbf{Stokes Struktur}
% gegeben, welche aus der asymptotischen Analysis entsteht.
\marginnote{In diesem Vortrag werden Stokes-Strukturen sowie meromorphe
Zusammenhänge und alles benötigte aus der asymptotischen Analysis definiert.
Kenntnisse aus der Theorie der D-Moduln werden nicht benötigt.}
Es ist eine bekannte Tatsache, dass eine irregulär singuläre
Diffenentialgleichung mit konvergenten Einträgen eine formale Lösung haben
kann, welche divergent ist.  Deshalb will man Isomorphieklassen von (irregulär
singulären) meromorphen Zusammenhängen, welche zu einem vorgegebenen `gutem
Model' formal isomorph sind, beschreiben.

Es stellt sich heraus, das dafür die formalen Invarianten, wie beispielsweise
die formale Monodromie, nicht ausreichen. Die benötigte Information ist aber
durch die Stokes-Struktur gegeben, welche mit Hilfe der asymptotischen Analysis
entsteht.

%}}}
\section{Meromorphe Zusammenhänge} %{{{
Sei $M$ eine \textbf{Riemann-Fläche}\footnote{Komplexe Mannigfaltigkeit von
Dimension 1.}, $Z$ ein \textbf{effektiver Divisor}\footnote{Formale Summe von
Hyperflächen mit positiven Koeffizienten.} auf $M$.
\begin{comment}
  Starte mit einer Diffenentialgleichung.

  Über jedem Punkt von $M\backslash Z$ hat man einen endlichen Vektorraum von
  `initial data' und erhalte ein holomorphes Bündel $\sM$ auf $M$
  \textcolor{gray}{(durch fortsetzen)}.
  \\Weiter gibt es zu jedem $x\in X$ und jedem Keim $u\in\sM_x$ in der Faser
  bei $x$ gibt es einen eindeutig bestimmten Schnitt in der Umgebung von $x$.
\end{comment}
Sei $\sM$ ein \textbf{\textcolor{gray}{(holomorphes)} Bündel}
$:\Leftrightarrow{}$ lokal freier $\cO_M$-Modul.
\begin{defn}
  \def\myU{\textcolor{green!30!black}{U}}
  \def\mys{\textcolor{blue!60!black}{s}}
  \def\myf{\textcolor{red!60!black}{f}}
  Ein \emph{meromorpher Zusammenhang auf $\sM$ mit Polen auf $Z$}
  ist eine $\C$-lineare Abbildung
  \[
    \nabla:\sM\to\Omega_M^1(*Z)\otimes\sM
  \]
  welche für alle offenen Mengen $\myU\subset M$, Schnitte
  $\mys\textcolor{blue!60!black}{\in\Gamma(\myU,\sM)}$ und holomorphen
  Funktionen $\myf\textcolor{red!60!black}{\in\cO(\myU)}$ die \textbf{Leibniz
  Regel}
  \[
    \nabla(\myf\mys)=\myf\nabla\mys+(d\myf)\otimes\mys
  \]
  erfüllt.

  \begin{comment}
    \begin{defn}
      Ein Zusammenhang heißt \emph{flach} oder \emph{integrabel} falls
          seine Krümmung
      $R_\nabla:=\nabla\circ\nabla:\sM\to\Omega_M^2(*Z)\otimes_{\cO_M}\sM$
      identisch verschwindet.\marginnote{\textcolor{gray}{$R_\nabla\equiv0$}}
      \begin{rem}
        \marginnote{\tiny \cite{sabbah2007isomonodromic} Rem 0.12.5}
        Für $\dim(M)=1$ ist jeder Zusammenhang flach.
        \textcolor{gray}{Also hier nicht von Bedeutung.}
      \end{rem}
    \end{defn}
  \end{comment}
\end{defn}
\begin{rem}
  \marginnote{\tiny Siehe
  \\\bullet~\cite{pym}
  \\\bullet~\cite{sabbah2007isomonodromic} 0.11.a}
  Lokal sieht $\sM$ wie $U\times\C^n$ und ein $s\in\sM$ wie
  $\begin{pmatrix}f_{1}\\\vdots\\f_{n}\end{pmatrix}=:f\in\cO(U)^n$, aus.
  \\Dann
  \[
    \nabla s=df-Af
  \]
  wobei $A\in M(n\times n,\Omega_M^1(*Z)(U))$ ist die \emph{Zusammenhangs
  Matrix}.
  \begin{comment}
    Klassifiziere diese durch die Lösung der DGL $\nabla s=0$.
  \end{comment}
  Ein Wechsel $F\in\Gl_n(\cO_X(U))$ der Trivialisierung entspricht
  $F[A^0]=(dF)F^{-1}+FA^0F^{-1}$.
  \marginnote{\tiny see \cite{sabbah2007isomonodromic} II.2.a}
\end{rem}
\begin{comment}
  Ein Zusammenhang is irregulär singulär, falls er einen Pol hat, der auch nach
  Tranformation immer noch Ordnung $\geq1$ hat.
\end{comment}
\pagebreak \subsection{Modelle und formale Zerlegung}%{{{
\begin{defn}
  Ein Keim $(\cM,\nabla)$ ist ein \emph{Model} falls
  \[
    (\cM,\nabla)
    \cong
    \bigoplus_\phi
    \underset{\text{merom. Zus.}}{%
      \underset{\text{elementare}}{%
        \underbrace{\cE^\phi\otimes\cR_\phi}
      }
    }
  \]
  wobei
  \begin{itemize}
    \item die $\phi\in t^{-1}\C[t^{-1}]$ paarweise unterschiedlich sind,
    \item die $\cR_\phi$ regulär singulär sind und
    \item $\cE^\phi$ wird gelöst von $e^\phi$.
  \end{itemize}
\end{defn}
\begin{lem}
  Ist $(\sM,\nabla)$ ein Modell, so lässt sich die Zusammenhangs Matrix
  schreiben als
  \[
    A^0=dQ+\Lambda\frac{dt}{t}
  \]
  wobei
  \begin{itemize}
    \item $Q=\diag(\phi_1,\dots,\phi_n)$ und
    \item $\Lambda$ diagonal und konstant ist.
  \end{itemize}
\end{lem}
\begin{tthm}[Levelt-Turittin]
  Zu
  % einem Keim\footnote{denn möglicherweise muss man sich auf eine kleine Scheibe einschränken.}
  $(\cM,\nabla)$
  gibt es, bis auf Verzweigung, immer einen \textbf{formalem} Isomorphismus
  \[
    \hat\lambda:\hat\cM
    \overset{\cong}\longrightarrow
    \hat\cM^{good}
    =\hat\cO\otimes\cM^{good}
  \]
  zu einem \textbf{formalem Modell}.
  \begin{rem}
    Aber keinen konvergenten lift.
    \[ \Large\begin{tikzcd}
        \textcolor{green!60!black}{\tilde{\textcolor{black}{\sM}}_{\theta}}
        \dar
        \rar[green!60!black]{\tilde\lambda}&
        \textcolor{green!60!black}}}
%}}}
\pagebreak \section{Asymptotische Entwicklungen} %{{{
Betrachte nun $M=D=\{t\in\C\mid|t|<r\}$ für $r$ beliebig klein und $Z=\{0\}$.
\marginnote{\tiny See \cite{van2003galois}}
\begin{paracol}{2}
  \begin{defn}
    Sei $\textcolor{red!60!black}{\theta_0,\theta_1}\in S^1$.
    \begin{itemize}
      \item $\begin{aligned}
          \Sect_{\textcolor{blue!60!black}{r}}
            (\textcolor{red!60!black}{\theta_0,\theta_1})
            := \{t=\rho e^{i\theta}\in\C
              &\mid 0<\rho<\textcolor{blue!60!black}{r}
            \\&~,\theta\in(\textcolor{red!60!black}{\theta_0,\theta_1})\}
        \end{aligned}$
      \item $\Sect(\textcolor{red!60!black}{\theta_0,\theta_1}):=
        \Sect_{\textcolor{blue!60!black}{r}}
        (\textcolor{red!60!black}{\theta_0,\theta_1})$
        für $\textcolor{blue!60!black}{r}$ klein genug.
      \item $\Sect(\textcolor{red!60!black}{U})
        \!\!\overset{\textcolor{red!60!black}{U=(\theta_0,\theta_1)}}{:=}\!\!
        \Sect(\textcolor{red!60!black}{\theta_0,\theta_1})$
    \end{itemize}
  \end{defn}
\switchcolumn
  \begin{center}
    \begin{tikzpicture}[scale=2.5]
      \node (zero) at (0,0) {};
      \node[below left] at (zero) {$0$};
      \draw[blue,dashed] (zero) circle (1cm);

      \filldraw[fill=green!20!white
        ,draw=green!60!black
        ,thick
        ,path fading=west] (0,0)
      -- ({cos( -30 )*.65},{sin( -30 )*.65}) arc (-30:70:.65) -- cycle;
      \draw[blue!60!black,thick] (0,0) -- ({cos( -30 )*.65},{sin( -30)*.65});
      \node[green!40!black] at (.4,.3)
        {$\Sect_{\textcolor{blue!60!black}{r}}
        (\textcolor{red!60!black}{\theta_0,\theta_1})$};
      \node[green!40!black] at (.3,-.25) {$\textcolor{blue!60!black}{r}$};

      \draw[thick,red!60!black] ({cos( -30 )},{sin( -30 )}) arc (-30:70:1);

      \fill[red!60!black] ({cos( -30 )},{sin( -30 )}) circle(.7pt);
      \fill[red!60!black] ({cos( 70 )},{sin( 70 )}) circle(.7pt);

      \node[red!60!black] at ({1.1 * cos(70)},{1.1 * sin(70)}) {$\theta_1$};
      \node[red!60!black] at ({1.1 * cos(-30)},{1.1 * sin(-30)}) {$\theta_0$};

      \node[red!60!black,right] at (1,0) {$U$};

      \fill[white] (zero) circle (1.5pt);
      \fill (zero) circle (.7pt);
    \end{tikzpicture}
  \end{center}
\end{paracol}
\begin{defn}
  \def\myN{\textbf{\textcolor{blue!40!black}{N}}}
  \def\mySect{\textcolor{red!40!black}{W}}
  \def\myConst{\textcolor{green!40!black}{C(\myN,\mySect)}}
  % $f\in\cO_D(*Z)$
  $f\in\cO(\Sect(U))$
  hat
  $\sum_{n\geq n_0}c_nt^n\in\C(\!(t)\!)$\footnote{\textbf{formale Laurent Reihe}}
  als \emph{asymptotische Entwicklung auf $\Sect(U)$}, falls
  \begin{itemize}
    \item $\exists$ $\textcolor{yellow!30!black}{r}$ so dass
      $\forall$ $\myN\geq0$ und
      $\forall$ abgeschlossenen Untersektoren $\mySect$ in
      $\Sect_{\textcolor{yellow!30!black}{r}}(U)$
      \\eine Konstante $\myConst$ existiert, so dass
      \[
        \left|
          f(t)-\sum_{n_0\leq n\leq\myN-1}c_nt^n
        \right|
        \leq \myConst|t|^{\myN} \qquad \text{ für alle } t\in \mySect
      \]

      \begin{comment}
        oder äquivalent:
        $\lim_{z\to0,z\in{\mySect}}|t|^{-(\myN-1)}
          \left|
            f(t)-\sum_{n_0\leq n\leq \myN-1}c_nt^n
          \right|=0$
      \end{comment}
  \end{itemize}
  Erhalte die Garbe $\sA$ auf $S^1$:
  \[
    \underset{\text{von } S^1}{\underset{\text{off. Intervall}}{U}}
    \mapsto\sA(U)
    \subset\cO(\Sect(U)) \text{ die Funktionen mit asymptotischer
    Entwicklung auf } \Sect(U)
  \]
\end{defn}
\begin{llem}[Borel-Ritt]
  \marginnote{$t\mapsto e^{-\frac{1}{t}}$ hat verschwindende asy.\  Entw.\  in
  Sektoren um $\theta=0$ \cite[6]{Varadarajan96linearmeromorphic}}
  Für jedes \textbf{echte} offene Intervall $U$ von $S^1$ ist die Abbildung
  \[
    \textcolor{gray}{0\to \sA^{<0}(U) \to}
    \sA(U) \overset{T}\twoheadrightarrow \C(\!(t)\!)
    \textcolor{gray}{\to0}
    \qquad
    \qquad
    \qquad
    \textcolor{gray}{0\to \sA^{<0} \to
    \sA \to \pi^{-1}\hat\cO_{D}
    \to0}
  \]
  eine surjektion.
\end{llem}
\begin{thm}
  Zu jedem $\theta\in S^1$ und jedem genügend kleinem Intervall
  $V\ni\theta$ gibt es einen Lift
  $\textcolor{green!40!black}{\tilde\lambda(V)}$ so dass das Diagramm
  \[ \Large\begin{tikzcd}
      \sA(V)\otimes\sM \cong:&
      \textcolor{green!40!black}{\tilde{\textcolor{black}{\sM}}(V)}
      \dar \rar[green!40!black]{\tilde\lambda(V)}&
      \textcolor{green!40!black}}}
\section{Stokes-Strukturen} %{{{
Fixiere ein \textbf{Modell} $(\sM^{good},\nabla^{good})$ auf $D$ mit Pol bei
$\{0\}=Z$ und somit auch eine Matrix $A^0=dQ+\Lambda\frac{dt}{t}$.

\begin{center}
  \textbf{Wir sind interessiert an
    $\left\{(\sM,\nabla)
        \mid \hat f:(\hat\sM,\hat\nabla)
          \overset{\cong}\to
          (\hat\sM^{good},\hat\nabla^{good})
      \right\}\Big/\sim$.}
\end{center}
Wir betrachten dazu aber die größere die \textcolor{gray}{punktierte} Menge
\[
  \sH(\sM^{good}):=\left\{(\sM,\nabla,\hat f)
      \mid \hat f:(\hat\sM,\hat\nabla)
        \overset{\cong}\to
        (\hat\sM^{good},\hat\nabla^{good})
    \right\}\Big/\sim
    \qquad \qquad
    \textcolor{gray}{\ni (\sM^{good},\nabla^{good},\hat \Id)}\,.
\]
Indem wir zu Zusammenhangs Matrizen übergehen erhalten wir die isomorphen
Mengen
\[
  \underset{=:\Syst(A^0)}{\underbrace{%
      \left\{A \mid A=\hat F[A^0]\text{~für ein~}\hat F\in\hat G
      :=\GL_n(\C\llbracket t\rrbracket)\right\}
  }}\Big/G\{t\}
\]
und
\[
  \cH(A^0):=\left\{
      (A,\hat F)\in\Syst(A^0)\times\hat G\mid A=\hat F[A^0]
    \right\}\Big/G\{t\}
\]

% \begin{comment}
% Definiere
% \[
%   \Syst(A^0):=\left\{A \mid A=\hat F[A^0]\text{ für ein }\hat F\in\hat G
%     :=\GL_n(\C\llbracket t\rrbracket)\right\}
% \]
% \begin{center}
%   \textbf{Wir sind interessiert in $\Syst(A^0)/G\{t\}$.}
% \end{center}
% wobei $G\{t\}:=\GL_n(\C\{t\})$ \emph{lokale analytischen Gauge
% Tranformationen}
% \begin{itemize}
%   \item mit Wirkung $F[A^0]=(dF)F^{-1}+FA^0F^{-1}$.
%     \marginnote{\tiny see \cite{sabbah2007isomonodromic} II.2.a}
% \end{itemize}
% Wir benötigen aber den etwas größeren Raum
% \[
%   \widehat\Syst(A^0):=
%     \left\{(A,\hat F)\in\Syst(A^0)\times\hat G\mid A=\hat F[A^0]\right\}
% \]
% und damit
% \begin{paracol}{5}
%   $\cH(A^0):=\widehat\Syst(A^0)/G\{t\}$
% \switchcolumn
% \switchcolumn
%   $\sH(\sM^{good}):=\left\{(\sM,\nabla,\hat f)
%       \mid \hat f:(\sM,\nabla)\overset{\cong}\to(\sM^{good},\nabla^{good})
%     \right\}/\sim$
% \end{paracol}
% \TODO[ist $\cH(A^0)$ der Halm bei $0$ von $\sH(\sM^{good})$?]
% \end{comment}

%}}}
\pagebreak \subsection{Theorem 1} %{{{
\begin{comment}
  see:
  \begin{itemize}
    \item \cite[29]{Varadarajan96linearmeromorphic}
  \end{itemize}
\end{comment}
\begin{defn}
  Definiere den \emph{Stokes Raum}
  % \marginnote{Kompliziert, da dies über Garbenkohomologie von Garben von
  %   nichtabelschen Gruppen definiert.}
  \[
    \St(\sM^{good})
      :=H^1\left(S^1,\Aut^{<0}\left(\tilde \sM^{good}\right)\right)
  \]
  wobei
  \begin{itemize}
    \item $\Aut^{<0}(\!\!\!\underset{\sA_{\tilde D}\otimes\sM^{good}}
      {\underbrace{\tilde\sM^{good}}}\!\!\!)$
      die Garbe auf $S^1$ der Automorphismen welche
      \begin{itemize}
        \item mit dem Zusammenhang kompatibel sind\TODO und
        \item formal äquivalent zur Identität sind.
      \end{itemize}
      \textcolor{gray}{Die Schnitte heißen \emph{Stokes Matrizen}.}
  \end{itemize}
  \begin{thm}
    $\St(\sM^{good})$ ist ein $\C$-Vektorraum.
  \end{thm}
\end{defn}

Sei $(\sM,\nabla,\hat f)\in\sH(\sM^{good})$. Dann gibt es eine Überdeckung
$\mathfrak{W}$ von $S^1$ und für jedes $W_i\in\mathfrak{W}$ einen
Lift\footnote{$\hat f_i=\hat f$} von $\hat f$
\[
  f_i:(\tilde\sM,\tilde\nabla)_{|W_i}
  \overset{\sim}{\longrightarrow}
  (\tilde\sM^{good},\tilde\nabla^{good})_{|W_i} \,.
\]
Dann ist $(f_jf_i^{-1})_{i,j}$ ein Kozykel in der Garbe
$\Aut^{<0}(\tilde\sM^{good})$ relativ zur Überdeckung $\mathfrak{W}$.
Damit haben wir eine Abbildung
\[
  \sH(\sM^{good})\to H^1(S^1,\Aut^{<0}(\tilde\sM^{good}))=\St(\sM^{good})
\]
\begin{tthm}
  Das ist ein Isomorphismus \textcolor{gray}{von punktierter Mengen}.
\end{tthm}
%}}}
\pagebreak \subsection{Theorem 2} %{{{
\begin{tthm}[Balser, Jurkat, Lutz]
  Es gibt einen Isomorphismus
  \begin{align*}
    \cH(A^0)&\cong(U_+\times U_-)^{k-1}
    &\textcolor{gray}{\cong\C^{(k-1)n(n-1)}}
  \\ [(A,\hat F)]&\mapsto\textbf{S}=(S_1,\dots,S_{2k-2})
  \end{align*}
  \begin{cor}
    Es gibt einen Isomorphismus
    \[
      \Syst(A^0)/G\{t\}\cong(U_+\times U_-)^{k-1}/T
    \]
    \textcolor{gray}{Durch Rausteilen der Torus \marginnote{$T=(\C^*)^n$}
      Wirkung: $t(\textbf{S})=(tS_1t^{-1},\dots,tS_{2k-2}t^{-1})$.}
  \end{cor}
\end{tthm}

\begin{defn}
  Sei $\phi_{ij}(z)$ der führende Term von $\phi_i-\phi_j$
  \begin{itemize}
    \item \textcolor{gray}{für $\phi_i-\phi_j=a/z^{k-1}+b/z^{k-2}+\dots$ dann
      $\phi_{ij}=a/z^{k-1}$.}
  \end{itemize}
  \marginnote{\tiny Richtungen, entlang denen $e^{\phi_i-\phi_j}$ am
  schnellsten fällt}
  $d\in\A\subset S^1$
  $:\Leftrightarrow{}$
  es gibt $i\neq j$ so dass $\phi_{ij}(z)\in\R_{<0}$ für $z\to0$ auf dem
  `Strahl durch $d$'.
  \\Die Elemente in $\A$ heißen \emph{anti-Stokes-Richtungen}.
\end{defn}
\begin{paracol}{2}
\switchcolumn[0]
  Sei $r:=\#\A$, $l:=r/(2k-2)$ und $\textbf{\underline{d}}:=(d_1,\dots,d_l)$
  \emph{Halb-Periode}.
  \begin{defn}
    Definiere die \emph{totale Ordnung}
    \[
      \phi_i\underset{\textbf{\underline{d}}}{<}\phi_j
      \Leftrightarrow{}
      \phi_{ij}\in\R_{<0}\text{
      entlang einem } d\in\textbf{\underline{d}}
    \]
    und durch $\phi_i\underset{\textbf{\underline{d}}}{<}\phi_j
    \Leftrightarrow{}\pi_i<\pi_j$ die
    \[
      \text{\emph{Permutations Matrix} } (P)_{ij}=\delta_{\pi(i)j} \,.
    \]
  \end{defn}
\switchcolumn[1]
  \begin{center}
    \begin{tikzpicture}[scale=3]
      \node[] (zero) at (0,0) {};

      % %%%%%%%%%%%%%%%%%%%%%%%%%%%%%%%%%%%%%%%%%%%%%%%%%%%%%%%%%%%%%%%%%%%%%%%%%
      % \fill[fill=red!60!black] (0,0) -- ({cos( 75 )*1.1},{sin( 75 )*1.1}) arc
      %   (75:105:1.1) -- cycle;
      % \fill[fill=red!60!black] (0,0) -- ({cos( 115 )*1.1},{sin( 115 )*1.1}) arc
      %   (115:145:1.1) -- cycle;

      % \fill[fill=red!60!black] (0,0) -- ({cos( 75 )*1.05},{sin( 75 )*1.05}) arc
      %   (75:145:1.05) -- cycle;

      % \fill[fill=white] (0,0) -- ({cos( 75 )*1},{sin( 75 )*1}) arc
      %   (75:145:1) -- cycle;

      % \node[red!40!black] at (-0.5,1.1) {$\widehat\Sect_i$};

      % \draw[thick,red!40!black] (0,0) -- +({cos( 145 )},{sin( 145 )});
      % \draw[thick,red!40!black] (0,0) -- +({cos( 75 )},{sin( 75 )});
      % \fill[red!40!black] ({cos( 145 )},{sin( 145 )}) circle (1pt);
      % \fill[red!40!black] ({cos( 75 )},{sin( 75 )}) circle (1pt);
      % %%%%%%%%%%%%%%%%%%%%%%%%%%%%%%%%%%%%%%%%%%%%%%%%%%%%%%%%%%%%%%%%%%%%%%%%%

      \fill[fill=green!20!white] (0,0) -- (1,0) arc (0:60:1.0cm) -- cycle;
      \draw[blue] (zero) circle (1cm);

      \foreach \w/\str in {10/$d_1\in S^1$,
                           20/$d_2$,
                           45/$d_3$,
                           55/$d_l$}
      {\draw[thick,purple!\w!blue,path fading=west]
          (0,0) -- +({cos( \w )},{sin( \w )}) node[right] {\str};
       \fill[blue!20!white] ({cos( \w )},{sin( \w )}) circle (1pt);
       \foreach \sep in {60,120,180,240,300}
       {\draw[green!20!white,thick] (zero) -- +({cos( \sep )},{sin( \sep )});
        \draw[purple!\w!blue] (0,0) -- +({cos( \w + \sep )},{sin( \w + \sep )});
        \fill[blue!20!white] ({cos( \w + \sep )},{sin( \w + \sep )}) circle (1pt);
       }
      };
      \node[right,red!30!black] at ({cos( 355 )},{sin( 355 )}) {$d_r$};

      \foreach \sep/\str in {0/$1$
                            ,60/$2$
                            ,120/$k-1$
                            ,180/$4$
                            ,240/$5$
                            ,300/$2k-2$}
      {\node[green!40!black]
        at ({.6 * cos( \sep + 30 )},{.5 * sin( \sep + 30)}) {\str};
      };

      \fill[yellow!60!black] (0.8,0.07) circle (1pt);
      \node[yellow!60!black,right] at (0.8,0.07) {$p$};

      \fill[white] (zero) circle (1pt);
      \fill (zero) circle (.7pt);
    \end{tikzpicture}
  \end{center}
\end{paracol}
\begin{thm}
  Zu $(A,\hat F)$ gibt es auf jedem Sektor $\Sect\left(d_i,d_{i+1}\right)$
  einen \textcolor{gray}{(kanonischen)} Lift
  \marginnote{\tiny eine \textbf{invertierbare} $n\times n$ Matrix von
  holomorphen Einträgen}
  \[
    \Sigma_i(\hat F)\in\Gl_n(\cO_{\Sect_i})
  \]
  so dass $\Sigma_i(\hat F)[A^0]=A$.  \textcolor{gray}{(Durch Borel-Summation)}
  \begin{comment}
    $\Sigma_i(\hat F)$ kann analytisch auf den Supersektor $\widehat\Sect_i
    :=\Sect\left(d_i-\frac{\pi}{2k-2},d_{i+1}+\frac{\pi}{2k-2}\right)$
    fortgesetzt werden und hat dort asymptotische Entwicklung $\hat F$ bei $0$.
  \end{comment}
  % \marginnote{If $g\in G\{z\}$ and $t\in T$ then
  %   $\Sigma_i(g\circ\hat F \circ t^{-1})=g\circ\Sigma_i(\hat F)\circ t^{-1}$.}
\end{thm}
\begin{defn}
  Die \emph{Stokes Faktoren} zu $(A,\hat F)$ sind
  \marginnote{$\kappa_i$ ist formal äquivalent zu $\id$:
  $\kappa_i[A^0]=A^0$.}
  \[
    K_i:= e^{-Q}\cdot e^{-\Lambda} \cdot \underset{\kappa_i}{\underbrace{%
      \Sigma_i(\hat F)^{-1}\cdot \Sigma_{i-1}(\hat F)}}
    \cdot  e^{\Lambda}\cdot e^{Q}
  \]
  \begin{lem}
    $K_i$ ist in der \emph{Gruppe der Stokes Faktoren}
    \[
      \SSto_{d_i}(A^0) := \{K \in G \mid (K)_{ij}
      =\delta_{ij} \text{ außer } \phi_{ij}\in\R_{<0}\text{ entlang } d_i \}.
    \]
  \end{lem}
\end{defn}
\begin{lem}
  Sei $\textbf{d}=(d_1,\dots,d_l)$.
  \begin{enumerate}
    \item $\prod_{d\in\textbf{d}}\SSto_d(A^0)\cong PU_+P^{-1};
      \qquad
      (K_1,\dots,K_l)\mapsto K_l\dots K_2K_1\in G$
    \item $\prod_{d\in\A}\SSto_d(A^0)\cong (U_+\times U_-)^{k-1};
      \qquad
      (K_1,\dots,K_r)\mapsto (S_1,\dots,S_{2k-2})$
      \begin{itemize}
        \item wobei $S_i:=P^{-1}K_{il}\dots K_{(i-1)l+1}P\in U_{+/-}$ falls $i$
          ungerade/gerade die Stokes Matrizen.
      \end{itemize}
  \end{enumerate}
\end{lem}

\begin{thm}
  Die Abbildung
  \[ \Large\begin{tikzcd}[row sep=0em]
      \cH(A^0) \rar & (U_+\times U_-)^{k-1}
    \\(A,\hat F) \rar[mapsto] & (S_1,\dots,S_{2k-2})
  \end{tikzcd} \]
  ist ein Isomorphismus.
  \\Also ist insbesondere $\cH(A^0)$ isomorph zum VR $\C^{(k-1)n(n-1)}$.
\end{thm}
%}}}
