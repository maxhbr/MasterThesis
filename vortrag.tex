\chapter{Vortrag}
\setcounter{section}{-1}

Here are the german notes of my Presentation of Stokes structures.
\section{Motivation} %{{{
Wir wollen \textbf{irregulär singulärere meromorphe Zusammenhänge}, welche zu
einem vorgegebenen `gutem Modell' \textbf{formal isomorph} sind, betrachten und
fragen uns, wie wir
\begin{center}
  \textbf{Isomorphieklassen dieser klassifizieren.}
\end{center}

Es stellt sich heraus, das die dafür benötigte Information genau durch die
\textbf{Stokes Struktur} gegeben ist.
\TODO[und formale Monodromie?]

\begin{comment}
  \TODO[Siehe \textbf{\cite{van2003galois}: Chapter 7} für mehr Motivation]
  DGLs mit konvergenten Einträgen haben manchmal eine formale Lösung die dann
  dann aber nicht konvergiert.
\end{comment}
%}}}
\section{Meromorphe Zusammenhänge} %{{{
Sei
\begin{itemize}
  \item $M$ eine Riemann-Fläche\marginnote{$\C$},
  \item $Z$ ein effektiver Divisor auf $M$ und
  \item ein holomorphes Bündel $\sM$ ist ein lokal freier $\cO_M$-Modul
\end{itemize}
\begin{defn}
  Ein \emph{meromorpher Zusammenhang auf $\sM$ mit Polen, welche durch $Z$
  beschränkt sind}, ist eine $\C$-lineare Derivation
  \[
    \nabla:\sM\to\Omega_M^1(Z)\otimes\sM
  \]
  welcher
  \begin{comment}
    \begin{itemize}
      \item für
        \begin{itemize}
          \item alle offenen Mengen $U\subset M$,
          \item alle Schnitte $s\in\Gamma(U,\sM)$ und
          \item alle holomorphen Funktionen $f\in\cO(U)$
        \end{itemize}
    \end{itemize}
  \end{comment}
  die \textbf{Leibniz Regel}
  \[
    \nabla(fs)=f\nabla s+(df)\otimes s
  \]
  erfüllt.

  \begin{defn}
    Ein Zusammenhang heißt \emph{flach} oder \emph{integrabel} falls
    \begin{itemize}
      \item seine Krümmung $R_\nabla\equiv0$
    \end{itemize}
    wobei
    \begin{itemize}
      \item $R_\nabla
        :=\nabla\circ\nabla:\sM\to\Omega_M^2(Z)\otimes_{\cO_M}\sM$
    \end{itemize}
    \begin{rem}
      \marginnote{\cite{sabbah2007isomonodromic} Rem 0.12.5}
      Falls $\dim(M)=1$ ist jeder Zusammenhang flach.
    \end{rem}
  \end{defn}
\end{defn}
\begin{rem}
  \marginnote{Siehe
  \\\bullet\cite{pym}
  \\\bullet\cite{sabbah2007isomonodromic} 0.11.a}
  Lokal sieht
  \begin{itemize}
    \item $\sM$ wie $U\times\C^n$ und
    \item ein $s\in\sM$ wie
      $\begin{pmatrix}f_{1}\\ \vdots\\ f_{n} \end{pmatrix}=:f$,
      mit $f_j\in\cO(U)$,
  \end{itemize}
  aus.
  Dann
  \[
    \nabla s=df - A(t)f
  \]
  wobei $A(t)$ mit Einträgen in
  $\C\{t\}[1/t]$\TODO[Hat $A$ werte in den 1-Formen??] ist die
  \emph{Zusammenhangs Matrix}.

  \begin{comment}
    Wollen diese Klassifizieren. Klassifiziere diese durch die Lösung von
    $\nabla s=0$. Dies ist eine DGL (ODE).
  \end{comment}
\end{rem}
\begin{comment}
  \subsection{Irregularität von meromorphen Zusammenhängen} %{{{
  \TODO[\cite{sabbah2007isomonodromic}: p. 86]
  \TODO[\cite{sabbah2007isomonodromic}: Def II.2.24]
\end{comment}
%}}}
\subsection{Modelle und formale Zerlegung}%{{{
\begin{defn}
  Ein Keim $(\cM,\nabla)$ ist ein \emph{Model} falls
  \[
    (\cM,\nabla)
    \cong
    \bigoplus_\phi
    \underset{\text{merom. Zus.}}{%
      \underset{\text{elementare}}{%
        \underbrace{\cE^\phi\otimes\cR_\phi}
      }
    }
  \]
  wobei
  \begin{itemize}
    \item \TODO[$\cE^\phi$ wird gelöst von\dots]
    \item die $\cR_\phi$ regulär singulär sind und
    \item die $\phi\in t^{-1}\C[t^{-1}]
      \textcolor{gray}{=\C\{t\}[t^{-1}]\backslash\C\{t\}}$
      paarweise unterschiedlich sind.
  \end{itemize}
\end{defn}
\begin{comment}
  \begin{lem}
    Ist $(\sM,\nabla)$ ein Modell, so lässt sich die Zusammenhangs Matrix
    schreiben als
    \[
      A^0=dQ+\Lambda\frac{dt}{t}
    \]
    wobei
    \begin{itemize}
      \item $Q=\diag(q_1,\dots,q_n)$ mit $q_i$ meromorph und
      \item $\Lambda$ diagonal und konstant ist.
    \end{itemize}
  \end{lem}
\end{comment}
\begin{paracol}{2}
  \begin{thm}[Levelt-Turittin]
    Zu einem \textbf{formalem} meromorphen Zusammenhang gibt es, bis auf
    Zurückziehen, immer einen Isomorphismus
    \[
      \hat\lambda:\hat\sM_0
      \overset{\cong}\longrightarrow
      \hat\sM_0^{good}
      =\hat\cO_0\otimes\sM_0^{good}
    \]
    zu einem \textbf{formalem Modell}.
  \end{thm}
\switchcolumn
  \begin{rem}
    Aber keinen konvergenten lift.
    \[ \Large\begin{tikzcd}
        \textcolor{green!60!black}{\tilde{\textcolor{black}{\sM}}_{\theta}}\dar
        \rar[green!60!black]{\tilde\lambda}&
        \textcolor{green!60!black}{%
          \tilde{\textcolor{black}{\sM}}_{\theta}}^{good} \dar
        \\\hat\sM_0 \rar{\hat\lambda} &
        \hat\sM_0^{good}
    \end{tikzcd} \]
    \textcolor{green!60!black}{Sektorweise aber schon, mit asymptotischer
    Analysis}
  \end{rem}
\end{paracol}

\begin{itemize}
  \item $G\{t\}:=\GL_n(\C\{t\})$ \emph{locale analytischen Gauge
    Tranformationen}
    \begin{itemize}
      \item mit Wirkung $F[A^0]=(dF)F^{-1}+FA^0F^{-1}$
        \marginnote{see \cite{sabbah2007isomonodromic} II.2.a}
    \end{itemize}
  \item $\hat G:=\GL_n(\C\llbracket t\rrbracket)$ \emph{formale
    Tranformationen}
\end{itemize}
definiere
\[
  \Syst(A^0)
  :=\left\{A \mid A=\hat F[A^0]\text{ für ein }\hat F\in\hat G\right\}
\]
\begin{center}
  \textbf{Wir sind interessiert in $\Syst(A^0)/G\{z\}$.}
\end{center}
Wir benötigen aber den etwas größeren Raum
\[
  \widehat\Syst(A^0):=
    \left\{(A,\hat F)\in\Syst(A^0)\times\hat G\mid A=\hat F[A^0]\right\}
\]
und damit
\begin{paracol}{5}
  \[
    \cH(A^0):=\widehat\Syst(A^0)/G\{t\}
  \]
\switchcolumn
\switchcolumn
  \[
    \sH:=\left\{(\sM,\nabla,\hat f)
      \mid \hat f:(\sM,\nabla)\overset{\cong}\to(\sM^{good},\nabla^{good})
    \right\}/\sim
  \]
\end{paracol}
%}}}
%}}}
\section{Asymptotische Entwicklungen} %{{{
Betrachte nun $M=D=\{t\in\C\mid|t|<r\}$ für $r$ beliebig klein und $Z=\{0\}$.
\marginnote{See \cite{van2003galois}}
\begin{paracol}{2}
  \begin{defn}
    Sei $\textcolor{red!60!black}{\theta_0,\theta_1}\in S^1$.
    \begin{itemize}
      \item $\begin{aligned}
          \Sect_{\textcolor{blue!60!black}{r}}
            (\textcolor{red!60!black}{\theta_0,\theta_1})
            := \{t=\rho e^{i\theta}\in\C
              &\mid 0<\rho<\textcolor{blue!60!black}{r}
            \\&~,\theta\in(\textcolor{red!60!black}{\theta_0,\theta_1})\}
        \end{aligned}$
      \item $\Sect_r(\textcolor{red!60!black}{U})
        \!\!\overset{\textcolor{red!60!black}{U=(\theta_0,\theta_1)}}{:=}\!\!
        \Sect_{\textcolor{blue!60!black}{r}}
        (\textcolor{red!60!black}{\theta_0,\theta_1})$
      \item $\Sect(\theta_0,\theta_1):=
        \Sect_{\textcolor{blue!60!black}{r}}
        (\textcolor{red!60!black}{\theta_0,\theta_1})$
        für $\textcolor{blue!60!black}{r}$ klein genug.
    \end{itemize}
  \end{defn}
\switchcolumn
  \begin{center}
    \begin{tikzpicture}[scale=2.5]
      \node (zero) at (0,0) {};
      \node[below left] at (zero) {$0$};
      \draw[blue,dashed] (zero) circle (1cm);

      \filldraw[fill=green!20!white
        ,draw=green!60!black
        ,thick
        ,path fading=west] (0,0)
      -- ({cos( -30 )*.65},{sin( -30 )*.65}) arc (-30:70:.65) -- cycle;
      \draw[blue!60!black,thick] (0,0) -- ({cos( -30 )*.65},{sin( -30)*.65});
      \node[green!40!black] at (.4,.3)
        {$\Sect_{\textcolor{blue!60!black}{r}}
        (\textcolor{red!60!black}{\theta_0,\theta_1})$};
      \node[green!40!black] at (.3,-.25) {$\textcolor{blue!60!black}{r}$};

      \draw[thick,red!60!black] ({cos( -30 )},{sin( -30 )}) arc (-30:70:1);

      \fill[red!60!black] ({cos( -30 )},{sin( -30 )}) circle(.7pt);
      \fill[red!60!black] ({cos( 70 )},{sin( 70 )}) circle(.7pt);

      \node[red!60!black] at ({1.1 * cos(70)},{1.1 * sin(70)}) {$\theta_1$};
      \node[red!60!black] at ({1.1 * cos(-30)},{1.1 * sin(-30)}) {$\theta_0$};

      \node[red!60!black,right] at (1,0) {$U$};

      \fill[white] (zero) circle (1.5pt);
      \fill (zero) circle (.7pt);
    \end{tikzpicture}
  \end{center}
\end{paracol}
\begin{defn}
  \def\myN{\textbf{\textcolor{blue!40!black}{N}}}
  \def\mySect{\textcolor{red!40!black}{W}}
  \def\myConst{\textcolor{green!40!black}{C(\myN,\mySect)}}
  $f\in\cO(D\backslash\{0\})$ hat die \textbf{formale Laurent Reihe}
  $\sum_{n\geq n_0}c_nz^n$ als \emph{asymptotische Entwicklung auf
  $\Sect_r(U)$}, falls
  \begin{itemize}
    \item für alle $\myN\geq0$ und
    \item für alle abgeschlossenen Untersektoren $\mySect$ in $\Sect_r(U)$
  \end{itemize}
  eine Konstante $\myConst$ existiert, so dass
  \[
    \left|
      f(t)-\sum_{n_0\leq n\leq\myN-1}c_nt^n
    \right|
    \leq \myConst|t|^{\myN} \qquad \text{ für alle } t\in \mySect
  \]

  \begin{comment}
    oder äquivalent:
    $\lim_{z\to0,z\in{\mySect}}|t|^{-(\myN-1)}
      \left|
        f(t)-\sum_{n_0\leq n\leq \myN-1}c_nt^n
      \right|=0$
  \end{comment}
  Damit bekommt man
  \begin{itemize}
    \item $\sA(U,r)\subset\cO(\Sect_r(U))$ die Funktionen mit asymptotischer
      Entwicklung auf $\Sect_r(U)$
    \item die Garbe $\sA$ auf $S^1$,
      \[
        \underset{\text{von } S^1}{\underset{\text{off. Intervall}}{U}}
        \mapsto\sA(U):=\colim_r\sA(U,r)
      \]
  \end{itemize}
  \TODO[Garbe auf $\tilde D=S^1\times[0,r)$ anstatt auf $S^1$]
\end{defn}
\begin{lem}[Borel-Ritt]
  \marginnote{$t\mapsto e^{-\frac{1}{t}}$ hat verschwindende asy.\  Entw.\  in
  Sektoren um $\theta=0$}
  Für jedes \textbf{echtes} offenes Intervall $U$ von $S^1$ ist die Abbildung
  \[
    \textcolor{gray}{0\to \sA^{<0}(U) \to}
    \sA(U) \overset{T}\twoheadrightarrow \C((t))
    \textcolor{gray}{\to0}
    \qquad
    \qquad
    \qquad
    \textcolor{gray}{0\to \sA^{<0} \to
    \sA \to \pi^{-1}\hat\cO_{D}
    \to0}
  \]
  eine surjektion.
\end{lem}
\begin{paracol}{2} %%%%%%%%%%%%%%%%%%%%%%%%%%%%%%%%%%%%%%%%%%%%%%%%%%%%%%%%%%%%
  \begin{thm}
    \begin{itemize}
      \item Für jedes $e^{i\theta^o}\in S^1$,
    \end{itemize}
    existiert ein Isomorphismus
    \[
      \tilde\lambda_{\theta^o}: \tilde\sM_{\theta^o}
      \overset{\sim}{\longrightarrow}\tilde\sM^{good}_{\theta^o},
    \]
    so dass das Diagramm
    \[\Large\begin{tikzcd}
        \tilde\sM_{\theta^o} \dar \rar{\tilde\lambda_{\theta^o}} &
        \tilde\sM^{good}_{\theta^o} \dar
      \\\hat\sM_0 \rar{\hat\lambda} &
        \hat\sM_0^{good}
      \end{tikzcd} \]
    kommutiert
  \end{thm}
\switchcolumn %%%%%%%%%%%%%%%%%%%%%%%%%%%%%%%%%%%%%%%%%%%%%%%%%%%%%%%%%%%%%%%%%
\begin{thm}
  \textcolor{gray}{%
    Zu jedem $\theta\in S^1$ und jedem genügend kleinem Intervall um $V$ gilt
    \[
      \sA(V)\otimes\sM\cong\sA(V)\otimes\bigoplus_\phi(\sE^\phi\otimes\sR_\phi)
    \]
  }
\end{thm}
\end{paracol} %%%%%%%%%%%%%%%%%%%%%%%%%%%%%%%%%%%%%%%%%%%%%%%%%%%%%%%%%%%%%%%%%
%}}}
\section{Stokes Strukturen} %{{{
\begin{paracol}{2}
  Fixiere $(\sM^{good},\nabla^{good})$ ein \textbf{Modell}
  \begin{itemize}
    \item auf $D$ mit Pol bei $\{0\}=Z$
  \end{itemize}
\switchcolumn
  Fixiere $A^0=dQ+\Lambda\frac{dt}{t}$.
\end{paracol}
%}}}
\subsection{Theorem 1} %{{{
\begin{defn}
  Definiere
  \[
    \St(\sM^{good})
      :=H^1\left(S^1,\Aut^{<0}\left(\tilde \sM^{good}\right)\right)
  \]
  wobei
  \begin{itemize}
    \item $\Aut^{<0}(\underset{\sA_{\tilde D}\otimes\sM^{good}}
      {\underbrace{\tilde\sM^{good}}})$
      die Garbe auf $S^1$ der Automorphismen welche
      \begin{itemize}
        \item mit dem Zusammenhang kompatibel\TODO[in Formeln] sind und
        \item formal äquivalent zur Identität sind.
      \end{itemize}
      Die Schnitte heißen \emph{Stokes Matrizen}.
  \end{itemize}
\end{defn}

Sei $(\sM,\nabla,\hat f)\in\sH$. Dann gibt es eine Überdeckung
$\mathfrak{W}$ von $S^1$ und für jedes $W_i\in\mathfrak{W}$ einen
Lift\footnote{$\hat f_i=\hat f$} von $\hat f$
\[
  f_i:(\tilde\sM,\tilde\nabla)_{|W_i}
  \overset{\sim}{\longrightarrow}
  (\tilde\sM^{good},\tilde\nabla^{good})_{|W_i} \,.
\]
Dann ist $(f_jf_i^{-1})_{i,j}$ ein Kozykel in der Garbe
$\Aut^{<0}(\tilde\sM^{good})$ relativ zur Überdeckung $\mathfrak{W}$.
Damit haben wir eine Abbildung
\[
  \sH\to H^1(S^1,\Aut^{<0}(\hat \sM^{good}))
\]
\begin{tthm}
  Die Abbildung
  \[
    \sH\to\St(\sM^{good})
  \]
  ist ein Isomorphismus von punktierter Mengen.
\end{tthm}
%}}}
\subsection{Theorem 2} %{{{
\TODO[From $\sH$ to $\Syst(A^0)/G\{t\}$]

\begin{tthm}[Balser, Jurkat, Lutz]
  \marginnote{Isomorph zu $\C^{(k-1)n(n-1)}$}
  Es gibt einen Isomorphismus
  \begin{align*}
    \cH(A^0)&\cong(U_+\times U_-)^{k-1}
  \\ [(A,\hat F)]&\mapsto(S_1,\dots,S_{2k-2})=\textbf{S}
  \end{align*}
  es gibt einen Isomorphismus
  \[
    \Syst(A^0)/G\{t\}\cong(U_+\times U_-)^{k-1}/T
  \]
\end{tthm}

\begin{defn}
  \marginnote{Richtungen, entlang denen $e^{q_i-q_j}$ am schnellsten fällt}
  Sei $q_{ij}(z)$ der führende Koeffizient von $q_i-q_j$
  \begin{itemize}
    \item für $q_i-q_j=a/z^{k-1}+b/z^{k-2}+\dots$ dann $q_{ij}=a/z^{k-1}$.
  \end{itemize}
  $d\in\A\subset S^1$
  $:\Leftrightarrow{}$
  es gibt $i\neq j$ so dass $q_{ij}(z)\in\R_{<0}$ für $z\to0$ auf dem `Strahl
  durch $d$'.
\end{defn}
\begin{paracol}{2}
  \begin{defn}
    \begin{itemize}
      \item $r:=\#\A$
      \item $l:=r/(2k-2)$
      \item $\textbf{d}:=(d_1,\dots,d_l)$ Halb-Periode
    \end{itemize}
    Definiere die Totale Ordnung
    \[
      q_i\underset{\textbf{d}}{<}q_j\Leftrightarrow{}q_{ij}\in\R_{<0}\text{
      entlang einem } d\in\textbf{d}
    \]
    \begin{itemize}
      \item Permutations Matrix $(P)_{ij}=\delta_{\pi(i)j}$
    \end{itemize}
  \end{defn}
\switchcolumn
  \begin{center}
    \begin{tikzpicture}[scale=3]
      \node[] (zero) at (0,0) {};

      % %%%%%%%%%%%%%%%%%%%%%%%%%%%%%%%%%%%%%%%%%%%%%%%%%%%%%%%%%%%%%%%%%%%%%%%%%
      % \fill[fill=red!60!black] (0,0) -- ({cos( 75 )*1.1},{sin( 75 )*1.1}) arc
      %   (75:105:1.1) -- cycle;
      % \fill[fill=red!60!black] (0,0) -- ({cos( 115 )*1.1},{sin( 115 )*1.1}) arc
      %   (115:145:1.1) -- cycle;

      % \fill[fill=red!60!black] (0,0) -- ({cos( 75 )*1.05},{sin( 75 )*1.05}) arc
      %   (75:145:1.05) -- cycle;

      % \fill[fill=white] (0,0) -- ({cos( 75 )*1},{sin( 75 )*1}) arc
      %   (75:145:1) -- cycle;

      % \node[red!40!black] at (-0.5,1.1) {$\widehat\Sect_i$};

      % \draw[thick,red!40!black] (0,0) -- +({cos( 145 )},{sin( 145 )});
      % \draw[thick,red!40!black] (0,0) -- +({cos( 75 )},{sin( 75 )});
      % \fill[red!40!black] ({cos( 145 )},{sin( 145 )}) circle (1pt);
      % \fill[red!40!black] ({cos( 75 )},{sin( 75 )}) circle (1pt);
      % %%%%%%%%%%%%%%%%%%%%%%%%%%%%%%%%%%%%%%%%%%%%%%%%%%%%%%%%%%%%%%%%%%%%%%%%%

      \fill[fill=green!20!white] (0,0) -- (1,0) arc (0:60:1.0cm) -- cycle;
      \draw[blue] (zero) circle (1cm);

      \foreach \w/\str in {10/$d_1\in S^1$,
                           20/$d_2$,
                           45/$d_3$,
                           55/$d_l$}
      {\draw[thick,purple!\w!blue,path fading=west]
          (0,0) -- +({cos( \w )},{sin( \w )}) node[right] {\str};
       \fill[blue!20!white] ({cos( \w )},{sin( \w )}) circle (1pt);
       \foreach \sep in {60,120,180,240,300}
       {\draw[green!20!white,thick] (zero) -- +({cos( \sep )},{sin( \sep )});
        \draw[purple!\w!blue] (0,0) -- +({cos( \w + \sep )},{sin( \w + \sep )});
        \fill[blue!20!white] ({cos( \w + \sep )},{sin( \w + \sep )}) circle (1pt);
       }
      };

      \foreach \sep/\str in {0/$1$
                            ,60/$2$
                            ,120/$k-1$
                            ,180/$4$
                            ,240/$5$
                            ,300/$2k-2$}
      {\node[green!40!black]
        at ({.6 * cos( \sep + 30 )},{.5 * sin( \sep + 30)}) {\str};
      };

      \fill[yellow!60!black] (0.8,0.07) circle (1pt);
      \node[yellow!60!black,right] at (0.8,0.07) {$p$};

      \fill[white] (zero) circle (1pt);
      \fill (zero) circle (.7pt);
    \end{tikzpicture}
  \end{center}
\end{paracol}
\begin{thm}[3.1]
  Sei $(A,\hat F)\in\hat\Syst(A^0)$.
  \\Dann:
  \begin{itemize}
    \item
      \marginnote{eine \textbf{invertierbare} $n\times n$ Matrix von
      holomorphen Einträgen}
      gibt es auf jedem Sektor
      $\Sect_i:=\Sect\left(d_i,d_{i+1}\right)$ einen Lift
      \begin{itemize}
        \item $\Sigma_i(\hat F)\in\Gl_n(\cO_{\Sect_i})$
      \end{itemize}
      so dass $\Sigma_i(\hat F)[A^0]=A$.
    \item $\Sigma_i(\hat F)$ kann analytisch auf den Supersektor
      $\widehat\Sect_i
      :=\Sect\left(d_i-\frac{\pi}{2k-2},d_{i+1}+\frac{\pi}{2k-2}\right)$
      fortgesetzt werden und hat dort asymptotische Entwicklung $\hat F$ bei
      $0$.
  \end{itemize}
  \begin{comment}
    \begin{itemize}
      \item If $g\in G\{z\}$ and $t\in T$ then
        $\Sigma_i(g\circ\hat F \circ t^{-1})=g\circ\Sigma_i(\hat F)\circ t^{-1}$.
    \end{itemize}
  \end{comment}
\end{thm}
\begin{defn}
  \begin{itemize}
    \item Die \emph{Stokes Faktoren} zu $(A,\hat F)\in\widehat\Syst(A^0)$ sind
      \marginnote{$\kappa_i$ ist formal äquivalent zu $\id$.}
      \[
        K_i:= e^{-Q}\cdot e^{-\Lambda} \cdot \underset{\kappa_i}{\underbrace{%
          \Sigma_i(\hat F)^{-1}\cdot \Sigma_{i-1}(\hat F)}}
        \cdot  e^{\Lambda}\cdot e^{Q}
      \]
    \item Die \emph{Gruppe der Stokes Faktoren}
      \[
        \SSto_d(A^0) := \{K \in G \mid (K)_{ij}
          =\delta_{ij} \text{ unless } (ij) \text{ is a root of } d\}.
      \]
      \begin{lem}
        $K_i\in\SSto_{d_i}(A^0)$
      \end{lem}
  \end{itemize}
\end{defn}
\begin{lem}[3.2]
  Sei $\textbf{d}=(d_1,\dots,d_l)$.
  \begin{enumerate}
    \item $\prod_{d\in\textbf{d}}\SSto_d(A^0)\cong PU_+P^{-1};
      \qquad
      (K_1,\dots,K_l)\mapsto K_l\dots K_2K_1\in G$
    \item $\prod_{d\in\A}\SSto_d(A^0)\cong (U_+\times U_-)^{k-1};
      \qquad
      (K_1,\dots,K_r)\mapsto (S_1,\dots,S_{2k-2})$
      \begin{itemize}
        \item wobei $S_i:=P^{-1}K_{il}\dots K_{(i-1)l+1}P\in U_{+/-}$ falls $i$
          ungerade/gerade die Stokes Matrizen.
      \end{itemize}
  \end{enumerate}
\end{lem}
Torus-Wirkung auf Stokes Matrizen
\[
  t(\textbf{S})=(tS_1t^{-1},\dots,tS_{2k-2}t^{-1})
\]
%}}}
