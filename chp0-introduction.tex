\chapter{Introduction}
\TODO[see~\cite{van2003galois} for better text]
Let $t$ be a local coordinate around $0$ in $\C$.
The basic problem, from which this master's thesis arises, is that there are
differential equations $(\frac{d}{dt}-A)\hat v=w$ with coefficients in
convergent powers series, which have no solution with converging entries.
By looking at asymptotic solutions on sectors and their relations on overlaps
one obtains the Stokes phenomenon.
One can use the language of meromorphic connections or the language of systems,
to look at this problem and the interesting object is then the classifying
set\comm{, which is defined as\dots} (cf.\ Section~\ref{sec:classifyingSet}).
The information contained in the Stokes phenomenon will be exactly the
information needed to classify meromorphic connections up to convergent
transformation.

We will only be interested in local classification and thus only in local
information of meromorphic connections. This classification can be splitted in
the coarse formal classification and the fine meromorphic classification.

The formal classification problem was solved by the Levelt-Turittin Theorem
(cf.\ Theorem~\ref{thm:leveltTurittin}). It states that in every formal
equivalence class are some meromorphic connections of special form, which will
be called models.
These models are meromorphic connections, which are defined to be isomorphic to
some direct sum of elementary meromorphic connections and elementary
meromorphic connections are well understood.

Starting with a formal equivalence class corresponding to some model $A^0$, we
can apply the meromorphic classification.
The obtained set of meromorphic classes will be called the classifying set, but
we will look at the slightly larger space $\cH(A^0)$ of meromorphic pairs
(cf.\ Section~\ref{sec:classifyingSet}).
The tool to describe the set of meromorphic pairs will be the Stokes structures.
% which turn out to provide exactly the needed information.
This idea is formulated in the Malgrange-Sibuya Theorem
(cf.\ Theorem~\ref{thm:mainThm1}), where the Stokes structures appear as the
first cohomology $H^1(S^1;\Lambda(A^0))$ of the Stokes sheaf
$\Lambda(A^0)$ on $S^1$ (cf.\ Definition~\ref{defn:StokesSheaf}).

The Malgrange-Sibuya theorem can be improved by showing that in each element
in the $H^1(S^1;\Lambda(A^0))$ contains a unique cocycle called the Stokes
cocycle (cf.\ Definition~\ref{defn:stokesCocycle}) of special form.
These Stokes cocycles are given by the elements in the product over some
special directions $\theta\in\A\subset S^1$ determined by $A^0$
(cf.\ Definition~\ref{defn:antiStokesDir}) of Stokes groups
$\Sto_\theta(A^0)\subset\Lambda_\theta(A^0)$
(cf.\ Definition~\ref{defn:stokesGroup}).
We will see that $\Sto_\theta(A^0)$ has the faithful representation
$\SSto_\theta(A^0)$, which has a rather simple definition
(cf.\ Proposition~\ref{prop:representation}).
The elements of $\SSto_\theta(A^0)$ are the so-called Stokes matrices and it is
easy to see that they are nilpotent.
Since the Stokes matrices also provide the required information to describe a
meromorphic class of a meromorphic connection one obtains an isomorphism
$\C^N\to H^1(S^1;\Lambda(A^0))$, where $N$ is the irregularity of $A^0$
(cf.\ Section~\ref{sec:WhichInformationIsNeeded}).
This characteristic can be used to define the structure of a nilpotent Lie
group on the classifying set $\cH(A^0)$.

%%% Local Variables:
%%% TeX-master: "Maximilian_Huber-Masters_Thesis-with_notes.tex"
%%% End:
