\chapter{Introduction}
\TODO[see~\cite{van2003galois} for better text]
Let $t$ be a local coordinate around $0$ in $\C$.
The basic fact, from which this master's thesis arises, is that there are
differential equations $(\frac{d}{dt}-A)\hat v=w$ with coefficients in
convergent powers series, which have no solution with converging entries.
By looking at asymptotic solutions on sectors at $0$ and how they can be extended to
larger arcs, one discovers the Stokes phenomenon. It describes that there are
some directions, called Stokes directions, beyond which some solutions can not
be extended.
To overlapping sectors one obtains matrices, which describe how solutions on the
corresponding sectors correlate.
Some germs of these matrices, which satisfy some condition, will be enough to
classify corresponding to the differential equation up to convergent
transformation.
Instead of differential equations, one can use the language of meromorphic
connections and the Stokes structures, which will be encoded in the first
homology of the Stokes sheaf, will be used to describe the classifying set.
% We will only be interested in local classification and thus only in local
% information of meromorphic connections.

In the first chapter we will define the notion of asymptotic expansions. We will
also state the important Borel-Ritt Lemma.
The second chapter is dedicated to the theory of meromorphic connections and the
equivalent language of systems. We will talk about local expression of
meromorphic connections and the formal classification of such objects.

The classification of meromorphic connections can be splitted in the coarse
formal classification and the fine meromorphic classification.
The formal classification problem was solved by the Levelt-Turittin Theorem
(cf.\ Theorem~\ref{thm:leveltTurittin}). It states that in every formal
equivalence class are some meromorphic connections of special form, which will
be called models and that such models are unique up to meromorphic equivalence.
These models are meromorphic connections, which are defined to be isomorphic to
some direct sum of elementary meromorphic connections and elementary
meromorphic connections are well understood.
On the level of systems there are the normal forms, which correspond to models.

Starting with a formal equivalence class corresponding to some normal form
$A^0$, we can apply the meromorphic classification to the corresponding set of
ambassadors.
The obtained set of meromorphic classes will be called the classifying set. But
instead of the classifying set we will look at the slightly larger space
$\cH(A^0)$ of meromorphic pairs which also handles the information, how some
meromorphic connection is related to it's model
(cf.\ Section~\ref{sec:classifyingSet}).
The classifying set can be obtained from the set of meromorphic pairs in a
rather simple way (cf.\ Corollary~\ref{cor:isomorphyOfClassfset}).

We have already said that the tool to describe the set of meromorphic pairs will
be the Stokes structures.
% which turn out to provide exactly the needed information.
This idea is formulated in the Malgrange-Sibuya Theorem
(cf.\ Theorem~\ref{thm:mainThm1}), where the Stokes structures appear as the
first cohomology $H^1(S^1;\Lambda(A^0))$ of the Stokes sheaf
$\Lambda(A^0)$ on $S^1$ (cf.\ Definition~\ref{defn:StokesSheaf}).

The Malgrange-Sibuya theorem can be improved by showing that in each element
in the $H^1(S^1;\Lambda(A^0))$ contains a unique cocycle called the Stokes
cocycle (cf.\ Definition~\ref{defn:stokesCocycle}) of special form.
This goes back to Loday-Richaud's work and especially her
Paper~\cite{Loday1994}.
These Stokes cocycles are given by the elements in the product over some
special directions $\theta\in\A\subset S^1$ determined by $A^0$ of Stokes groups
$\Sto_\theta(A^0)\subset\Lambda_\theta(A^0)$ (cf.\ Section~\ref{sec:StokesGroup}).
We will see that $\Sto_\theta(A^0)$ has the faithful representation
$\SSto_\theta(A^0)$, which has a rather simple definition
(cf.\ Section~\ref{sec:matrixReps}).
The elements of $\SSto_\theta(A^0)$ are the so-called Stokes matrices and it is
easy to see that they are nilpotent.
Since the Stokes matrices also provide the required information to describe a
meromorphic class of a meromorphic connection one obtains a structure of a
nilpotent Lie group on the set of isomorphism classes of meromorphic pairs
$\cH(A^0)$ and an isomorphism $\C^N\to H^1(S^1;\Lambda(A^0))$, where $N$ is the
irregularity of $A^0$ (cf.\ Section~\ref{sec:WhichInformationIsNeeded}).

In the Section~\ref{sec:furtherImprovements} we will improve this even further
by collecting the data of multiple Stokes groups into one group which is more
stable under deformations on the chosen model. In the single-leveled case is
this extensively used by Boalch in his Publications~\cite{boalch,thboalch}.
After that we will mention some ideas from the summability theory, just to give
a rough view of the concept.
In the last Section~\ref{sec:theCompleteDiagram} we will draw a diagram which
will contain many isomorphisms and objects which were defined in the
Chapter~\label{chap:stokes}. It will also clearify, how some morphisms were
composed.

%%% Local Variables:
%%% TeX-master: "Maximilian_Huber-Masters_Thesis-with_notes.tex"
%%% End:
