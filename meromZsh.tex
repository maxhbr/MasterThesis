\chapter{Systems and meromorphic connections}
There are multiple languages, which can be used for talking about meromorphic
connections.
\rewrite{Firstly} the languages of meromorphic connections which can be used to
talk about global information.
\comm{Also there is an approach with $\cD$-modules.}
For local description one can use \textbf{germs of meromorphic connections} or
the coordinate dependent (local) \textbf{systems} and \textbf{connection
matrices}. 
\comm{Also, there is the coordinate independent approach with localized
holonomic $\cD$-modules.}

Meromorphic connections are introduced and discussed in many resources.
A good starting point are Sabbah's lecture notes \cite{sabbah_cimpa90}.
More advanced resources are for example
Sabbah's book \cite{sabbah2007isomonodromic},
Varadarajan's \cite{Varadarajan96linearmeromorphic} or
the book \cite{hotta2008} from Hotta et al.
The necessary facts about meromorphic connections are also stated in
Boalch's paper \cite{boalch} (resp.\ his thesis \cite{thboalch}),
and Loday-Richaud's paper \cite{Loday1994}.

Systems are, for example, discussed in 
the book \cite{hotta2008} from Hotta et al,
Loday-Richaud's paper \cite{Loday1994} and his book \cite{Loday2014} and
Boalch's paper \cite{boalch} (resp.\ his thesis \cite{thboalch}).

Let $M$ be a riemanian surface and let $Z=k_1(a_1)+\cdots+k_m(a_m)>0$ be an
effective divisor\footnote{The $a_i$ are distinct points and the $k_i$ are
positive integers.} on $M$.
It is sufficient to think $M=\P^1$ and $0\in|Z|$\footnote{If
$Z=k_1(a_1)+\cdots+k_m(a_m)$ then $|Z|:=\{a_1,\dots,a_m\}$.}, since we will
only be interested in local information (at $0$).

Let $\sM$ be a holomorphic Bundle over $M$ i.e.\ a locally free $\cO_M$-module
of rank $n$.

A meromorphic connection is then defined as follows.
\begin{defn}\label{defn:mercon}
  \def\myU{\textcolor{green!30!black}{U}}
  \def\mys{\textcolor{blue!60!black}{s}}
  \def\myf{\textcolor{red!60!black}{f}}
  A \emph{meromorphic connection $(\sM,\nabla)$ on $\sM$ with poles on $Z$}
  is defined by a $\C$-linear morphism of sheaves
  \[
    \nabla:\sM\to\Omega_M^1(*Z)\otimes\sM
  \]
  satisfying, for each $\myU\underset{\text{op.}}{\subset} M$, the
  \emph{Leibniz rule}
    \[
      \nabla(\myf\mys)=\myf\nabla\mys+(d\myf)\otimes\mys
    \]
  for $\mys\textcolor{blue!60!black}{\in\Gamma(\myU,\sM)}$ and
  $\myf\textcolor{red!60!black}{\in\cO_M(\myU)}$.
  The \emph{rank} of the meromorphic connection $(\sM,\nabla)$ is defined as
  the rank of the Bundle $\sM$.
  \begin{s-rem}
    Some authors use the factors $k_i$ of the divisor $Z$ to limit the pole
    orders at the points $a_i$. Since we do not need this restriction, we allow
    arbitrary pole orders. Denoted is this by the $*$ in $\Omega_M^1(*Z)$.
    The sheaf $\cO_M(*Z)$ of functions, which are meromorphic along $Z$, is
    defined in Sabbah's book~\cite[Sec.0.8]{sabbah2007isomonodromic} and
    $\Omega_M^1(*Z)$ is then defined as
    \[
      \Omega_M^1(*Z):=\cO_M(*Z)\underset{\cO_M}\otimes\Omega_M^1
    \]
    \rewrite{the \emph{sheaf of meromorphic differential $1$-forms}} (cf.\
    \cite[Sec.0.9.b]{sabbah2007isomonodromic}).
  \end{s-rem}
\end{defn}
\begin{rem}
  \begin{enumerate}
    \item We will occasionally omit the $\nabla$ and simply call $\sM$ the
      meromorphic connection.
    \item Here, the variant `holomorphic bundle with meromorphic connection' is
      chosen, like in \cite{boalch}.
      There is also the twisted\TODO[better word?] description `meromorphic
      bundle with holomorphic connection' which is for example used in
      Sabbah's book~\cite{sabbah2007isomonodromic}.
      \\By choosing a lattice of a meromorphic bundle, one gets a holomorphic
      bundle but if the meromorphic bundle had a holomorphic connection, the
      induced connection \rewrite{on the lattice} is no longer guaranteed to be
      holomorphic.
      Thus we obtain a meromorphic connection on a holomorphic bundle in our
      sense.
  \end{enumerate}
\end{rem}

\begin{defn}
  \marginnote{\cite[0.12.2]{sabbah2007isomonodromic}}
  The connection $\nabla:\cM\to \Omega_M^1\otimes_{\cO_M}\cM$ is said to be
  \emph{integrable} or \emph{flat}, if
  \begin{einr}
    its curvature vanishes, i.e.\ $R_\nabla\equiv0$ where
    \begin{itemize}
      \item $R_\nabla:=\nabla\circ\nabla:\cE\to\Omega_M^2\otimes_{\cO_M}\cE$
        is a $\cO_M$-linear morphism.
    \end{itemize}
  \end{einr}
  \begin{s-prop}
    \marginnote{\cite[0.12.4]{sabbah2007isomonodromic}}
    The connection $\nabla$ is flat if and only if, in any local basis $e$ of
    $\cM$, the connection matrix $\Omega$ satisfies
    \[
      d\omega + \omega \wedge \omega = 0.
    \]
    This means, that the flatness condition is sufficient to assure the
    existence of local fundamental solutions.
  \end{s-prop}
  \begin{s-rem}
    Here are all connections flat, since only look at Dimension one.
  \end{s-rem}
  \begin{comment}
    We will say that a connection on a meromorphic bundle is \emph{integrable}
    or \emph{flat} if its restriction to $M\backslash Z$ is an integrable
    connection on the holomorphic bundle $\sM_{|M\backslash Z}$.
  \end{comment}
\end{defn}

%%%%%%%%%%%%%%%%%%%%%%%%%%%%%%%%%%%%%%%%%%%%%%%%%%%%%%%%%%%%%%%%%%%%%%%%%%%%%%%
\section{Local expression of meromorphic connections}
\marginnote{\cite[28]{sabbah2007isomonodromic}, \cite[2]{thboalch} and
  \cite[11]{babbitt1989local}}
We will usually\TODO[only?] be interested in local information of meromorphic
connections.  This means, that we look at a connection in a neighbourhood of $0$
\rewrite{and allow only one singularity at $0$}.
There are many ways of expressing the local information, we will either talk
about germs of meromorphic connections or systems, which are coordinate-
and trivialization-dependent.
\begin{prop}
  \marginnote{\textbf{\cite[Rem.5.2.4]{hotta2008}}\\\cite[Def.4.2.1]{Loday2014}}
  A germ of a meromorphic connection $(\sM,\nabla)$ is just the sheaf-theoretic
  germ (at $t=0$), thus is given by a tuple $(\cM,\nabla)$ where
  \begin{itemize}
    \item $\cM$ is the germ at $0$ of the holomorphic bundle $\sM$ and thus a
      $\C(\!\{t\}\!)$-vectorspace of dimension $n$, since the ring of germs of
      meromorphic functions with poles at $0$ is the ring $\C(\!\{t\}\!)$, and
    \item $\nabla:\cM\to \cM$ \TODO[one forms?] is a additive map, which
      satisfies the \emph{Leibniz rule}
      \[
        \nabla(fm)=\partial f\cdot m + f\nabla(m)
      \]
      for all $f\in\C(\!\{t\}\!)$ and $m\in \cM$.
  \end{itemize}
  \begin{comment}
      \begin{s-rem}
      \marginnote{\cite{sabbah2007isomonodromic}}
      It is a $(\C(\!\{t\}\!),\nabla)$-vectorspace.
    \end{s-rem}
    \begin{s-rem}
      Loday-Richaud calls this in \cite[Def.4.2.1]{Loday2014} a
      \emph{differential module}.
    \end{s-rem}
  \end{comment}
\end{prop}
\begin{rem}
  From now on we will mostly talk about \textbf{germs of} meromorphic
  connections $(\cM,\nabla)$ and we will call them meromorphic connection. If
  we want to talk about meromorphic connections in the sense of
  definition~\ref{defn:mercon} we will emphasize this by the word `global' or
  by talking about a meromorphic connection \textbf{on $M$}.
\end{rem}
\begin{defn}
  \marginnote{\cite[Def.5.2.1]{hotta2008}}
  A \emph{(iso-)\,morphism of meromorphic connections}
  $\Phi:(\cM,\nabla)\overset{\sim}\to(\cM',\nabla')$ is a (iso-)\,morphism of
  $\C(\!\{t\}\!)$-vectorspaces $\Phi:\cM\overset{\sim}\to\cM'$ which commutes
  with the connections, i.e.\ which satisfies
  $\nabla\circ\phi = (\id\otimes\phi)\circ\nabla$.
  \TODO[Category of meromorphic connections!]
\end{defn}
\marginnote{\cite[65]{Loday2014}, \cite[129]{hotta2008}}
Choose a $\C(\!\{t\}\!)$-basis $\underline{e}=(e_1,e_2,\dots,e_n)$ of $\cM$.
Let $A=(a_{jk})$ be a $n\times n$ matrix with entries in $\C(\!\{t\}\!)$
\rewrite{such that the it describes the action of $\nabla$,} i.e.\ it satisfies
\[
  \nabla e_k
  =
  -\sum_{1\leq j\leq n} a_{jk}(t)e_j \,.
\]
Let $x=\sum_{0\leq j\leq n}x_je_j$ be an arbitrary element of $\cM$ which is in
matrix notation described as $x=\underline{e}\cdot X$ with the column matrix
$X={}^t\!(x_1,x_2 ,\dots,x_n)$.
Then, applying the Leibniz rule, yields
\begin{align*}
  \nabla x&=\nabla\left(\underline{e}\cdot X\right)
  \\&=\underline{e} \cdot dX - \underline{e}\cdot \nabla X
  \\&=\underline{e}\left(dX-AX\right).
\end{align*}
Such that horizontal sections of $(\cM,\nabla)$, i.e.\ sections which satisfy
$\nabla x=0$, correspond to solutions of
\begin{equation}\label{eq:ode}
  \frac{d}{dt}x=Ax \,.
\end{equation}
Thus, with the connection $\nabla$ and the $K$-basis $\underline{e}$ is
naturally associate the differential operator $\triangle=d-A$, which has order
one and dimension $n$.
\TODO[possibly multivalued solutions ($\tilde K$)?~\cite{hotta2008} on page
128]

\begin{comment}
  This means that the connection $\nabla$ is fully determined by the matrix $A$
  and thus is fully determined by $A'$.
\end{comment}
\begin{defn}
  \begin{enumerate}
    \item This matrix $A$ is called a \emph{connection matrix} of
      $(\cM,\nabla)$. It depends on the choice of the $\C(\!\{t\}\!)$-basis.
      \begin{comment}
        \begin{s-cor}
          As we have seen above, \rewrite{is a connection} fully determined by
          its connection matrix.
        \end{s-cor}
      \end{comment}
    \item A \emph{germ of a meromorphic linear differential system} of rank
      $n$, or just a \emph{system}, is a germ of a meromorphic connection on
      the trivial vector bundle \textbf{with a chosen trivialization}
      \rewrite{of the fiber $E\cong\C^n$}.
      \begin{s-prop}
        Thus, the set of systems is isomorphic to the set
        \[
          \End(E)\otimes\C(\!\{t\}\!)=\gl_n(\C(\!\{t\}\!))
        \]
        of all connection matrices.
      \end{s-prop}
      Such a system will be denoted by $[A]=d-A$ where $A$ is the connection
      matrix in the chosen trivialization.
      \TODO[Category of systems!]
  \end{enumerate}
\end{defn}

%%%%%%%%%%%%%%%%%%%%%%%%%%%%%%%%%%%%%%%%%%%%%%%%%%%%%%%%%%%%%%%%%%%%%%%%%%%%%%%
\subsubsection{System \rightarrow{} germ of a meromorphic connection}
If we want to reverse the base choice to get the corresponding meromorphic
connection back, we can do this in the following way.
\begin{prop}\label{prop:systToMeromConn}
  If we start with either a system $[A]$ of rank $n$, or a connection matrix
  $A\in\gl_n(\C(\!\{t\}\!)$, we get a germ of a meromorphic connection via
  \[
    (\cM_A,\nabla_A)=(\C(\!\{t\}\!)^n,d-A)
  \]
  which has $A$ as its connection matrix.
  \begin{s-rem}
    Since systems are meromorphic connections together with a base choice, we
    could alternatively forget the base choice. This yields clearly an
    isomorphic meromorphic connection.
  \end{s-rem}
\end{prop}
\begin{proof}
  \TODO{}
\end{proof}

\begin{prop}
  Let $(\cM_1,\nabla_1)$ and $(\cM_2,\nabla_2)$ be two meromorphic
  connections
  with the connection matrices $A_1$ and $A_2$.
  A connection matrix of $(\cM_1,\nabla_1)\oplus(\cM_1,\nabla_1)$ is then
  given
  by the block-diagonal matrix $\diag(A_1,A_2)$.
\end{prop}
\begin{proof}
  Use Proposition~\ref{prop:systToMeromConn} to write the connection as
  \[
    (\C(\!\{t\}\!)^{n_1},d-A_1)\oplus(\C(\!\{t\}\!)^{n_2},d-A_2)
  \]
  and we want to show, that it is isomorphic to
  \[
    \left(\C(\!\{t\}\!)^{n_1+n_2},d-
    \begin{pmatrix} A_1 & 0 \\0 & A_2 \end{pmatrix}\right) \,.
  \]
  Denote by $A:=\diag(A_1,A_2)$ the block-diagonal matrix build from $A_1$ and
  $A_2$.
  For every $j\in\{1,2\}$ we have the corresponding inclusion
  $i_j:\C(\!\{t\}\!)^{n_j}\hookrightarrow\C(\!\{t\}\!)^{n_1+n_2}$ and the
  diagram
  \[ \begin{tikzcd}
      \C(\!\{t\}\!)^{n_j} \rar{d-A_j}\dar{i_j} & \C(\!\{t\}\!)^{n_j} \dar{i_j}
    \\\C(\!\{t\}\!)^{n_1+n_2} \rar{d-A} & \C(\!\{t\}\!)^{n_1+n_2}
  \end{tikzcd} \]
  which commutes, since the derivation commutes with the inclusion and the
  matrix $A$ is build in the \rewrite{correct way}, to satisfy
  $i_j(A_jx)=A_j(i_j(x))$:
  \begin{align*}
    i_j(dx-A_jx) &= i_j(dx)-i_j(A_jx)
    \\&=d(i_j(x))-A_j(i_j(x))
    \\&=((d-A)\circ i_j)(x) \,.
  \end{align*}
\end{proof}

\begin{rem}
  \marginnote{\cite[129f]{hotta2008}}
  Let $(\cM_1,\nabla_1)$ and $(\cM_2,\nabla_2)$ be meromorphic connections.
  Then
  \begin{enumerate}
    \item $\cM_1\otimes\cM_2$ is endowed with the structure of a meromorphic
      connection by
      \[
        \nabla(u_1\otimes u_2)=\nabla_1u_x\otimes u_2+u_1\otimes\nabla_2u_2
      \]
      where $u_i\in\cM_i$ and
    \item $\Hom_{\C(\!\{t\}\!)}(\cM_1,\cM_2)$ is endowed with the structure of
      a meromorphic
      connection by
      \[
        (\nabla\phi)(u_1)=\nabla_2(\phi(u_1))-\phi(\nabla_1 u_1)
      \]
      where $\phi\in\Hom_{\C(\!\{t\}\!)}(\cM_1,\cM_2)$ and $u_i\in\cM_i$.
  \end{enumerate}
\end{rem}

%%%%%%%%%%%%%%%%%%%%%%%%%%%%%%%%%%%%%%%%%%%%%%%%%%%%%%%%%%%%%%%%%%%%%%%%%%%%%%%
\subsubsection{Local \rightarrow{} global}
\begin{comment}
  Maybe see: \cite[Thm.3.3.1]{sibuya1990Linear}: G. D. Birkhoff
\end{comment}
\begin{thm}
  \begin{comment}
    Quelle?
  \end{comment}
  If we start with a germ $(\cM,\nabla)$ of a meromorphic connection there is
  a unique meromorphic connection $(\sM,\nabla)$ such that
  \begin{itemize}
    \item $(\sM,\nabla)$ has only singularities at $0$ and $\infty$
    \item the singularity at $\infty$ is only \TODO{} and
    \item $(\cM,\nabla)$ is the germ at $0$ of $(\sM,\nabla)$.
  \end{itemize}
\end{thm}
\begin{proof}
  \TODO{}
\end{proof}

%%%%%%%%%%%%%%%%%%%%%%%%%%%%%%%%%%%%%%%%%%%%%%%%%%%%%%%%%%%%%%%%%%%%%%%%%%%%%%%
\subsubsection{Formalization}
\begin{multicols}{2}
  Let $[A]$ be a system. We \rewrite{view it as a formal system, by} allowing
  formal solutions.
  \TODO{}

\columnbreak

  Let $(\cM,\nabla)$ be a meromorphic connection. The connection $\nabla$
  naturally extends to $\hat\cM:=\cM\otimes\C(\!(t)\!)$ and
  $\tilde\cM_\theta:=\cM\otimes\cA_\theta$.
  \TODO{}
\end{multicols}


%%%%%%%%%%%%%%%%%%%%%%%%%%%%%%%%%%%%%%%%%%%%%%%%%%%%%%%%%%%%%%%%%%%%%%%%%%%%%%%
\subsubsection{As differential operator}
\begin{comment}
  \begin{itemize}
    \item \cite[Sec.4.2]{Loday2014}
  \end{itemize}
\end{comment}

From the theory of ordinary differential equations we know that to
(\ref{eq:ode}) there is a equivalent ordinary differential equation of order
$n$ which can be written as
\[
  \underset{=:P}{\underbrace{%
      (a_n\partial_t^n+a_{n-1}\partial_t^{n-1}+\cdots a_{1}\partial_t+a_{0})
  }} \cdot v=0
\]
where $a_i\in\C(\!\{t\}\!)$.
This leads to the theory of $\cD$-modules.

\begin{comment}
  See \cite[Sec.1.4]{babbitt1983} for \textbf{ode of rank $n$} to
  \textbf{system}.
\end{comment}

\begin{comment}
%%%%%%%%%%%%%%%%%%%%%%%%%%%%%%%%%%%%%%%%%%%%%%%%%%%%%%%%%%%%%%%%%%%%%%%%%%%%%%%
\subsubsection{As $\cD$-module}
\marginnote{\cite[Sec.4.2.2]{Loday2014}}

In the other direction, from $\cD$-modules to meromorphic connections, there is
the lemma of the cyclic vector. \TODO{}
\marginnote{\textbf{\cite[Prop.4.2.5]{Loday2014}},
  \cite[Rem.4.2.6]{Loday2014}}
\end{comment}

%%%%%%%%%%%%%%%%%%%%%%%%%%%%%%%%%%%%%%%%%%%%%%%%%%%%%%%%%%%%%%%%%%%%%%%%%%%%%%%
\subsection{Transformation of systems}
\begin{notations}
  We will use the following notations
  \begin{itemize}
    \item $G=\Gl_n(\C)$;
    \item $G[t]=\Gl_n(\C[t])$;
    \item $G\{t\}=\Gl_n(\C\{t\})$ analytic transformations;
    \item $G(\!\{t\}\!)=\Gl_n\left(\C\{t\}[t^{-1}]\right)$ meromorphic
      transformations;
    \item $G\llbracket t\rrbracket=\Gl_n\left(\C\llbracket t\rrbracket\right)$
      (maybe not applicable) formal transformations;
    \item $G(\!(t)\!)=\Gl_n\left(\C\llbracket t\rrbracket[t^{-1}]\right)$
      (maybe not applicable) formal meromorphic transformations.
  \end{itemize}
  \begin{comment}
    We will always use the meromorphic ones, in contrast
    to~\cite{boalch,thboalch} where analytic classification is used.
  \end{comment}
\end{notations}
\marginnote{\cite[Sec.4.3.1]{Loday2014}}
By \emph{meromorphic\footnote{We use the term meromorphic in the sens of
convergent meromorphic. Otherwise we say formal meromorphic.} transformation},
or just \emph{transformation}, of a system we mean a $\C(\!\{t\}\!)$-linear
change of the trivialization.
Such a change is given by a matrix $F\in G(\!\{t\}\!)$ and the transformed
connection matrix ${}^F\!A$ is obtained through
\[
  {}^F\!A=(dF)F^{-1} + FAF^{-1} \,.
\]
If $F$ is formal i.e.\ $F\in G(\!(t)\!)$, it will usually be denoted by
$\hat F$.
The transformation of $A$ by $\hat F$ is not guaranteed to have convergent
entries.
We denote by $\hat G(A)$ the set of all \emph{(applicable) formal
transformations}
\[
  \hat G(A):=\left\{\hat F\in G(\!(t)\!)
    \mid {}^{\hat F}\!A \text{ has convergent entries i.e.\ }
    {}^{\hat F}\!A\in G(\!\{t\}\!)
  \right\}\,.
\]
Let $\hat{F'}\in\hat G(A^0)$ and $A':={}^{\hat{F'}}\!A$, then are the sets
$\hat G(A)$ and $\hat G(A')$ related by
\[
  \hat G(A')=\hat G(A)\hat{F'}^{-1}=\left\{
    \hat F\in G(\!\{t\}\!) \mid \hat F\hat{F'}\in\hat G(A)
  \right\} \,.
\]

\begin{rem}
  The condition
  \begin{einr}
    $B$ is obtained from $A$ by transformation $F$
  \end{einr}
  is clearly equivalent to
  \begin{einr}
    $F$ solves the linear differential system
    \[
      \frac{dF}{dt}=BF-FA
    \]
    which is denoted by $[A,B]$.
  \end{einr}
  \begin{s-rem}
    If we start with a isomorphism
    $\Phi:(\cM,\nabla)\overset{\sim}\longrightarrow(\cM',\nabla')$ and two
    base choices $\cM\overset{\sim}\longrightarrow\C(\!\{t\}\!)^n$ and
    $\cM'\overset{\sim}\longrightarrow\C(\!\{t\}\!)^n$ we have the following
    commutative diagram:
    \[ \begin{tikzcd}[column sep=.7cm,row sep=.7cm]
        \C(\!\{t\}\!)^n \arrow{rrrrr}{F}\arrow{dddd}{d-B}&&&&&
          \C(\!\{t\}\!)^n\arrow{dddd}{d-A}
          \\ & \cM \arrow{dd}{\nabla}\arrow{rrr}{\Phi}\ular&&& \cM'\arrow{dd}{\nabla'}\urar
          \\
        \\ & \cM \arrow{rrr}{\Phi}\dlar&&& \cM'\drar
        \\ \C(\!\{t\}\!)^n \arrow{rrrrr}{F}&&&&& \C(\!\{t\}\!)^n
    \end{tikzcd} \]
    Thus the commutation property for the outer rectangle is given by
    \[
      (d-B)\circ F=F\circ(d-A)
    \]
    which is equivalent \TODO[really] to
    \[
      \frac{dF}{dt}=BF-FA.
    \]
    \begin{comment}
      \begin{align*}
        ((d-B)\circ F) x &= (F\circ(d-A)) x
      \\(d\circ F) x - (B\circ F) x &= F(dx -Ax)
      \\d(Fx) - B(Fx) &= F(x' - Ax)
      \\(F'x)x' - BFx &= Fx' - FAx
      \\(F'x)x' - Fx' &= BFx - FAx
      \\(F'x)x' - Fx' &= (BF - FA)x
      \\&~\!~\vdots
      \\F'x &= (BF - FA)x
      \end{align*}
    \end{comment}
  \end{s-rem}
\end{rem}
\begin{comment}
  \begin{rem}
    \[
      {}^{(F_2F_1)}\!A =
      {}^{F_2}\!\left({}^{F_1}\!A\right)
    \]
    \rewrite{since}
    \begin{align*}
      {}^{F_2F_1}\!A^0
      &= d(F_2F_1)(F_2F_1)^{-1}+F_2F_1 A(F_2F_1)^{-1}
    \\&=\left(
        \left(dF_2\right)̂F_1
        +F_2\left(dF_1\right)̂
      \right) F_1^{̀-1} F_2^{̀-1}
      +F_2F_1 A F_1^{̀-1}F_2^{̀-1}
    \\&= \left(dF_2\right)̂F_2^{-1}
       +F_2\left(dF_1\right)̂ F_1^{̀-1} F_2^{̀-1}
       +F_2
       \left(
         {}^{F_1}\!A-\left(dF_1\right)F^{-1}
       \right)
       F_2^{̀-1}
    \\&= \left(dF_2\right)̂F_2^{-1} +F_2 {}^{F_1}\!A F_2^{̀-1}
    \\&= {}^{F_2}\!\left({}^{F_1}\!A\right)
    \end{align*}
  \end{rem}
\end{comment}
\begin{defn}
  We define the \emph{(formal) equivalence relation on the connection matrices}
  as
  \begin{einr}
    \textbf{\boldmath$A$ is (formally) equivalent to $B$}
  \end{einr}
  if and only if
  \begin{einr}
    \textbf{\boldmath$B$ is obtained from $A$ by (formal) transformation}.
  \end{einr}
  The \emph{class of a connection matrix} is the orbit under the gauge
  transformation \rewrite{in} $G(\!\{t\}\!)$. The \emph{formal class} ist the
  orbit \rewrite{in} $\hat G(A)$.
  \begin{s-rem}
    \begin{enumerate}
      \item Thus $A$ is (formally) equivalent to $B$ if and only if there is a
        (formal) solution of $[A,B]$.
      \item This implies also an equivalence relation and a classification on
        the systems.
    \end{enumerate}
  \end{s-rem}
\end{defn}

\begin{prop}
  \marginnote{\cite[Lem.5.1.3]{hotta2008}}
  Two germs of meromorphic connections are (formally) isomorphic if and only if
  their corresponding connection matrices are (formally) equivalent.
\end{prop}
\begin{proof}
  \TODO{}
\end{proof}

\begin{defn}
  \begin{itemize}
    \item An \emph{isotropy} of $A^0$ or of $[A^0]$ is a transformation $\hat
      F$ which satisfy ${}^{\hat F}\!A^0=A^0$.
      Thus, the isotropies are the solutions of the system
      $[\End A^0]:=[A^0,A^0]$.
      \marginnote{\cite[853]{Loday1994}}
    \item Let $G_0(A^0)$ denote the set of all isotropies of $A^0$.
      \begin{s-rem}
        They are, a priori, formal transformations. Actually $G(A^0)$ is a
        subgroup of $\Gl_n(\C[1/x,x])$
        (cf.~\cite[853]{Loday1994}\TODO[~(cf.~\cite{BJL1979Birkhoff})]).
      \end{s-rem}
  \end{itemize}
\end{defn}
\begin{lem}
  \marginnote{\cite[854]{Loday1994}}
  Two formal transformations $\hat F_1$ and $\hat F_2$ take $A^0$ into
  equivalent matrices ${}^{\hat F_1}\!A^0$ and  ${}^{\hat F_2}\!A^0$ if and
  only if there exists $f_0\in G_0(A^0)$ such that $\hat F_1=\hat F_2f_0$.
\end{lem}
\begin{proof}
  \TODO{}
\end{proof}

%%%%%%%%%%%%%%%%%%%%%%%%%%%%%%%%%%%%%%%%%%%%%%%%%%%%%%%%%%%%%%%%%%%%%%%%%%%%%%%
\subsubsection{Regular / irregular singularities}
\marginnote{\cite[Defn.5.1.6]{hotta2008}}
\begin{defn}
  \marginnote{\cite[86]{sabbah2007isomonodromic}}
  A connection with connection matrix $A$ has \emph{regular singularity} at $0$
  if there exists a konvergent transformation, by which $A$ is obtained from a
  matrix with at most a simple pole at $t=0$.
  Otherwise, the singularity is called \emph{irregular}.
  \begin{s-rem}
    This implies that, if $A$ has
    $\left\{\substack{\text{irregular}\\\text{regular}}\right\}$
    singularity, then also all
    meromorphic equivalent matrices ${}^{F}\!A$ have
    $\left\{\substack{\text{irregular}\\\text{regular}}\right\}$
    singularity.
  \end{s-rem}
  \begin{comment}
    \begin{s-rem}
      \marginnote{\cite[150]{van2003galois}}
      One can express this notion of regular singular also in terms of
      $\delta:=t\frac{d}{dt}$. A system has regular singularity if it is
      equivalent to an equation $\delta-A$ where $A$ has entries in
      holomorphic functions in a neighbourhood of $z=0$.
    \end{s-rem}
  \end{comment}
\end{defn}
\begin{thm}
  Let $(\cM,\nabla)$ be a regular singular meromorphic connection and $A$ its
  connection matrix.
  Then there exists a matrix $F\in G(\!\{t\}\!)$ such that after transformation
  by $F$ the matrix $B={}^F\!A$ is constant i.e.\ ${}^F\!A\in G$
  (cf.~\cite[Thm.II.2.8]{sabbah2007isomonodromic}
  or~\cite[Sec.5.1.2]{hotta2008}).
\end{thm}

%%%%%%%%%%%%%%%%%%%%%%%%%%%%%%%%%%%%%%%%%%%%%%%%%%%%%%%%%%%%%%%%%%%%%%%%%%%%%%%
\subsection{Fundamental solutions and monodromy of a system}
\marginnote{\cite[Sec.4.3.2]{Loday2014}}
\marginnote{\cite[130]{hotta2008}}
It is well known in the theory of linear ODEs that the set of solutions to
(\ref{eq:ode}) forms a vector space of dimension $n$ over $\C$, i.e.\ if
$x_0(t)$ and $x_{00}(t)$ are two solution of (\ref{eq:ode}) and $C_1$,
$C_2\in\C$ are two constants, then is also $C_1x_0(t)+C_2x_{00}(t)$ a solution.

\begin{defn}
  \marginnote{\cite[4f]{thboalch}}
  A \emph{fundamental solution} $\cY$ of the system $[A]$ is an invertible
  $n\times n$ matrix, which solves $[0,A]$.
  \begin{s-rem}
    This means, that the columns of $\cY$ are $n$ $\C$-linearly independent
    solutions of the system $[A]$.
  \end{s-rem}
\end{defn}

\begin{rem}
  If the trivialization is changed by $F$ (resp.\ $\hat F$) the fundamental
  solution $\cY\in G(\!\{t\}\!)$ changes to $F\cY$ (resp.\ $\hat F\cY$).
\end{rem}

\begin{comment}
  Unique \textbf{up to permutation}?\ or up to basis change?\ in the ramified
  case up to \TODO{}
\end{comment}

\marginnote{\cite[130]{hotta2008}, \cite[6]{heu2010}}
Choose a fundamental solution $\cY$. Then analytic continuation along a closed
path $\gamma$ in $M\backslash Z$ provides another fundamental solution
$\cY'=\rho^{-1}(\gamma)\cY$ where $\rho(\gamma)$ is called the \emph{monodromy
along the path $\gamma$}.
\begin{defn}
  Let $[A]$ be a system with fundamental solution $\cY$.
  The analyic continuation of $\cY$ along a circle around $t=0$ yields the
  fundamental solution
  \[
    \lim_{s\to2\pi}\cY(e^{\sqrt{-1}s}t)=\cY(t)\Lambda
  \]
  where $\Lambda\in G$ is called the \emph{monodromy matrix} of $[A]$.
\end{defn}

\TODO[monodromy representation] \TODO[Riemann hilbert korrespondenz]

\begin{comment}
%%%%%%%%%%%%%%%%%%%%%%%%%%%%%%%%%%%%%%%%%%%%%%%%%%%%%%%%%%%%%%%%%%%%%%%%%%%%%%%
  \subsection{Ramification}
  \marginnote{\cite[I.5.4.1]{sabbah_cimpa90}}
\end{comment}

%%%%%%%%%%%%%%%%%%%%%%%%%%%%%%%%%%%%%%%%%%%%%%%%%%%%%%%%%%%%%%%%%%%%%%%%%%%%%%%
\section{Formal classification}\label{sec:formalClassification}
\marginnote{\cite[Thm.4.3.1]{Loday2014}}
In every formal equivalence class of meromorphic connections, there are some
meromorphic connections of special form, which we will call models. They are
not unique but all of them, which are formally isomorphic to a given
meromorphic connection, lie in the same convergent equivalence class.
In fact, every element of this convergent equivalence class will be a model in
our definition.

\rewrite{The first part is given by} the Levelt-Turittin theorem, which says,
that each meromorphic connection is, after potentially needed ramification,
formally isomorphic to such a model.
Thus the Levelt-Turittin theorem solves the \emph{formal classification
problem}.
\begin{defn}\label{defn:elemnMerConnBausteine}
  \begin{itemize}
    \item For a $\phi\in\C(\!\{t\}\!)$ we use $\cE^{\phi}$ to denote the germ
      \[
        (\cE^{\phi},\nabla)=(\C(\!\{t\}\!),d-\phi')\,.
      \]
      This corresponds to the system satisfied by the function $e^\phi$.
      \begin{s-cor}
        $\cE^\phi$ is determined by the class of $\phi$ in
        $\C(\!\{t\}\!)/\C\{t\}=t^{-1}\C[t^{-1}]$. In the following, we will
        only consider the unique ambassador $\phi$ in each class which has no
        holomorphic part.
      \end{s-cor}
    \item For $\alpha\in\C$, define the \emph{elementary regular meromorphic
      connection of rank one} $\cN_{\alpha,0}$ as the germ
      \[
        (\cN_{\alpha,0},\nabla)=\left(\C(\!\{t\}\!),d+\frac{\alpha}{t}\right)
        \,.
      \]
      This corresponds to the system satisfied by $t^{-\alpha}$.

      \marginnote{\cite[Defn.II.2.5]{sabbah2007isomonodromic}}
      An \emph{elementary regular model of arbitrary rank} is a meromorphic
      connection which has a basis, in which the connection Matrix can be
      written as
      \[
        \frac{1}{t} (\alpha\id+N)
      \]
      where $N$ is a nilpotent matrix.
      If \rewrite{$\alpha\id + N$} is a single Jordan Block, we denote the
      corresponding connection by $\cN_{\alpha,d}$ where $d$ is the dimension
      \rewrite{minus one}.
  \end{itemize}
\end{defn}
\begin{prop}
  \begin{enumerate}
    \item
      \marginnote{\cite[Cor.II.2.9]{sabbah2007isomonodromic}}
      Every regular germ of a meromorphic connection is isomorphic to some
      direct sum
      \[
        (\cR,\nabla)=\bigoplus_{\alpha,d}(\cN_{\alpha,d},\nabla)\,.
        \TODO[not only of rank one]
      \]
      \begin{s-rem}
        For a detailed analysis of regular meromorphic connections
        see~\cite[Sec.II.2]{sabbah2007isomonodromic} or
        \cite[Sec.5.2]{hotta2008}.
      \end{s-rem}
    \item Every germ of a meromorphic connection of rank one is isomorphic to
      some germ
      \[
        (\cE^\phi,\nabla)\otimes(\cN_{\alpha,0},\nabla) \,.
      \]
    \item Two such germs corresponding to $(\phi_1,\alpha_1)$ and
      $(\phi_2,\alpha_2)$ are isomorphic if and only if
      \begin{itemize}
        \item $\phi_1-\phi_2$ has no pole and
        \item $\alpha_1-\alpha_2\in\Z$.
      \end{itemize}
  \end{enumerate}
\end{prop}
\begin{proof}
  See~\cite[Prop.II.5.1]{sabbah2007isomonodromic}
\end{proof}
\begin{defn}
  \marginnote{\cite[Def.II.5.2]{sabbah2007isomonodromic}}
  A germ $(\cM,\nabla)$ is called \emph{elementary} if it is isomorphic to
  some germ $(\cE^\phi,\nabla)\otimes(\cR,\nabla)$ where
  \begin{itemize}
    \item $(\cR,\nabla)$ has regular singularity at $\{0\}$ but has not to be
      of rank $1$, i.e.\ is isomorphic to a direct sum of regular elementary
      meromorphic connections.
  \end{itemize}
  \marginnote{\cite[II.2.f]{sabbah2007isomonodromic}}
\end{defn}
\begin{defn}\label{defn:model}
  \def\myPhi{\textcolor{red!60!black}{\phi}}
  \def\myE{\textcolor{green!40!black}{\cE^{\myPhi}}}
  A germ $(\cM',\nabla')$ is a \emph{model} if there exists, after ramification
  $\cM=\pi^*\cM'$ by $\pi$, a isomorphism to a direct sum of elementary
  meromorphic connections:
  \begin{multicols}{2}
    \[
      \lambda:(\cM,\nabla)
      \overset{\cong}{\longrightarrow}
      % \cong
      \bigoplus_{~\tikzmark{e3}\!\!\myPhi}
      \overset{\tikzmark{e2}}{\myE}
      \otimes
      \overset{\tikzmark{e1}}{\textcolor{blue!40!black}{\cR_{\myPhi}}}
      \,.
    \]
    \columnbreak{}
    \begin{itemize}
      \item[\tikzmarkb{n2}{green}] is irregular singular
      \item[\tikzmarkc{n1}{blue}] has regular singularity at $\{0\}$
      \item[\tikzmarkc{n3}{red}] $\myPhi\in t^{-1}\C[t^{-1}]$ pairwise distinct
    \end{itemize}
    \begin{tikzpicture}[remember picture,overlay]
      \draw[->,blue!50!white,thick] (n1) to[out=180,in=70] (e1);
      \draw[->,green!40!black,thick] (n2) to[out=180,in=70] (e2);
      \draw[->,red!50!white,thick] (n3) to[out=205,in=-30] (e3);
    \end{tikzpicture}
  \end{multicols}

\end{defn}
The important theorem here is the Levelt-Turittin theorem, which solves the
formal classification problem.
\begin{thm}[Levelt-Turittin]\label{thm:leveltTurittin}
  To each germ $(\cM',\nabla')$ of a meromorphic connection there exists, after
  potentially needed pullback $\pi^{*}\textcolor{black}{\cM'}=:\cM$ by some
  suitable ramification $t=z^q$ of order $q\geq1$, a
  \textcolor{green!30!black}{\textbf{formal}} isomorphism
  \[
    \textcolor{green!30!black}{\hat{\textcolor{black}{\lambda}}}:
    \textcolor{green!30!black}{\hat{\textcolor{black}{\cM}}}
    \overset{\cong}\longrightarrow
    \textcolor{green!30!black}{\hat{\textcolor{black}{\cM}}^{nf}}
    :=\textcolor{green!30!black}{\hat\cO_M\otimes}\cM^{nf}
  \]
  to a model $\cM^{nf}$.
  We then call $\cM^{nf}$ a \emph{formal decomposition} or \emph{formal model}
  of $\cM$ or $\cM'$.
  \begin{rem}
    There, in general is \textbf{no} lift of the isomorphism $\hat\lambda$,
    i.e.\ there is no isomorphism making the diagram
    \[ \begin{tikzcd}
        \cM \dar \arrow[dotted]{r}[description]{?} & \cM^{nf} \dar
        \\\hat\cM \rar{\hat\lambda} & \hat\cM^{nf}
    \end{tikzcd} \]
    kommutative.
    But sectorwise, there are lifts given by the main asymptotic existence
    theorem (cf.\ Theorem~\ref{thm:meat}).
  \end{rem}
\end{thm}
\begin{proof}
  See \TODO{}
\end{proof}
\TODO[Full set of formal invariants]
\begin{comment}
  \begin{prop}
    \marginnote{This condition \textbf{might be} equivalent to the
      condition of being \textbf{nice} in~\cite{thboalch}.}
    Let $(\cM,\nabla)$ be a germ, equipped with a basis in which the matrix $A$
    takes the form
    \[
      A=t^{-r}A(t)
    \]
    with
    \begin{itemize}
      \item $r\geq1$,
      \item $A$ has holomorphic entries, and
      \item $A_0:=A(0)$ being regular semisimple, i.e.\ with pairwise distinct
        eigenvalues.
    \end{itemize}
    Then there is no ramification needed, to apply the Levelt-Turittin-theorem.
    \comm{Further, all the summands $\cR_\phi$ have rank one, which is not the
    case in general.}
  \end{prop}
  \begin{proof}
    See~\cite[Thm.II.5.7]{sabbah2007isomonodromic}.
  \end{proof}
\end{comment}

%%%%%%%%%%%%%%%%%%%%%%%%%%%%%%%%%%%%%%%%%%%%%%%%%%%%%%%%%%%%%%%%%%%%%%%%%%%%%%%
\subsection{In the language of systems: normal forms}
In the language of system, the equivalent of models are normal forms, which are
characterizes by the structure of their fundamental solutions.

The normal forms are then the systems $[A^0]$, with a fundamental solution
$\cY$ in a special form and from the Levelt-Turittin we then can deduce that
every system $[{}^{\hat F}\!A^0]$, with $\hat F\in G(\!(t)\!)$, has a
fundamental solution in the form $\hat F\cY$.
\begin{defn}\label{defn:normSol}
  Let $[A]$ be a system.
  We call $[A]$ (or $A$) a \emph{normal form} if its fundamental solution can
  be written as
  \[
    \mathcal{Y}_0(t)=F t^L e^{Q(t^{-1})}
  \]
  with
  \begin{itemize}
    \item \emph{irregular part} $e^{Q(t^{-1})}$ of $\mathcal{Y}_0$ defind by
      \[
        Q(t^{-1})=\underset{j=1}{\overset{s}{\bigoplus}}q_j(t^{-1})1_{n_j}
          =\diag(\underset{n_1\text{-times}}{\underbrace{%
          q_1,\dots,q_1}},q_2,\dots,q_s)
      \]
      where the $q_i$ are polynomials in $\frac{1}{t}$ or in a fractional power
      $\frac{1}{s}=\frac{1}{t^{1/p}}$ of $\frac{1}{t}$ such that $q_j(0)=0$,
      i.e.\ without constant term,
    \item $L\in\gl_n(\C)$ constant matrix called the \emph{matrix of formal
      monodromy}, where $t^L$ means $e^{L\ln t}$ and
      \marginnote{in \cite[1]{Remy2014} $L$ is just a Jordan normal form.  Is
      this generic enough?}
    \item $F\in G(\!\{t\}\!)$ a transformation.
  \end{itemize}
  The normal forms will often be denoted $A^0$.
  If $A$ is formally equivalent to a normal form $A^0$ we say, that $A^0$
  \emph{is a normal form for} $A$.

  The fundamental solution $F t^L e^{Q(t^{-1})}$ of a normal form will be
  called \emph{normal solution} and will often be denoted by $\cY_0$.
  \begin{s-rem}
    \begin{enumerate}
      \item The system $[A^0]$ has the same monodromy as $t^L$. \comm{In fact,
        every formally equivalent system has the same monodromy.}
      \item \TODO[Cor from base change for fundamental solutions]
        By changing the basis via $F^{-1}$ wo obtain from $[A^0]$ the
        equivalent system $[{}^{F^{-1}}\!A^0]$ with the normal solution
        \[
          F^{-1}\mathcal{Y}_0(t)=t^L e^{Q(t^{-1})} \,.
        \]
        Thus it is always possible to assume, that the transformation matrix
        $F$ is trivial.
    \end{enumerate}
  \end{s-rem}
\end{defn}
\begin{prop}\label{prop:fundSolBuilder}
  Let $\mathcal{Y}=\hat F t^L e^{Q(t^{-1})}$ be a fundamental solution.
  The matrix
  \[
    A={}^{\hat F}\left(t^LQ'(t^{-1})t^{-L}+L\frac{1}{t}\right)
  \]
  defines a system $[A]$ which has $\cY$ as fundamental solution.
\end{prop}
\begin{proof}
  It is sufficient, to proof this in the case $\hat F=\id$, i.e.\ the case
  where $\cY=\cY_0=t^L e^{Q(t^{-1})}$ is a normal solution.

  \begin{comment}
    Since we are dealing with diagonal matrices it is easy to see, that
    \begin{align*}
      \frac{d}{dt}e^{Q(t^{-1})}
      &=\diag\left(\frac{d}{dt}e^{q_1(t^{-1})},\frac{d}{dt}e^{q_2(t^{-1})}
        ,\dots,
        \frac{d}{dt}e^{q_n(t^{-1})}\right)
        \\&=\diag\left(\frac{d}{dt}q_1(t^{-1})e^{q_1(t^{-1})}
                      ,\frac{d}{dt}q_1(t^{-1})e^{q_2(t^{-1})}
                      ,\dots
                      ,\frac{d}{dt}q_1(t^{-1})e^{q_n(t^{-1})}\right)
    \\&=Q'(t^{-1})e^{Q(t^{-1})} \,.
    \end{align*}
    and, since the function $t^L$ is defined as $e^{L\ln t}$,
    \begin{align*}
      \frac{d}{dt}t^L&=\frac{d}{dt}e^{L\ln t}
      =Le^{(L-\id)\ln t}
      =L\frac{1}{t}t^L \,.
    \end{align*}
  \end{comment}
  It is easy to see, that $\frac{d}{dt}e^{Q(t^{-1})}=Q'(t^{-1})e^{Q(t^{-1})}$
  and $\frac{d}{dt}t^L=L\frac{1}{t}t^L$, thus we can prove, that $\cY_0$ is a
  matrix consisting of solutions:
  \begin{align*}
    \frac{d}{dt}\cY_0
    &=\frac{d}{dt}\left(t^Le^{Q(t^{-1})}\right)
  \\&=\frac{d}{dt}t^Le^{Q(t^{-1})}+t^L\frac{d}{dt}e^{Q(t^{-1})}
  \\&=L\frac{1}{t}t^Le^{Q(t^{-1})}+t^LQ'(t^{-1})e^{Q(t^{-1})}
  % \\&=L\frac{1}{t}t^Le^{Q(t^{-1})}+Q'(t^{-1})t^Le^{Q(t^{-1})}
  \\&=\left(t^LQ'(t^{-1})t^{-L}+L\frac{1}{t}\right)t^Le^{Q(t^{-1})}
  \\&=A^0\cY_0 \,.
  \end{align*}
  The invertability condition is clear.
\end{proof}
\begin{thm}\label{thm:modelEqNormalForm}
  A meromorphic connection $(\cM,\nabla)$ is a model if and only if its
  connection matrix $A$ defines a normal form $[A]$.
  \begin{s-rem}
    This allows us, to say that a model $(\cM^{nf},\nabla^{nf})$ is a normal
    form and \rewrite{vice versa}.
    \rewrite{This} explains, why we mark models with ${}^{nf}$.
  \end{s-rem}
\end{thm}
\begin{proof}
  \textbf{``\Rightarrow{}'':}
  Let $\cM=\bigoplus_{\phi}\cE^\phi\otimes\cR_\phi$ be a model and let us first
  fix a $\phi$ and look at the connection $\cE^\phi\otimes\cR_\phi$.
  We know that $\cR_\phi$ is a direct sum of regular elementary meromorphic
  connections whose fundamental matrices are of the form
  $\frac{1}{t}(\alpha\id+N)$ where $\alpha\in\C$ and $N$ is nilpotent
  (cf.\ Definition~\ref{defn:elemnMerConnBausteine}).
  Thus the connection matrix for $\cR_\phi$ writes as
  \[
    \frac{1}{t} \bigoplus_{(\alpha,N)}\left( \alpha\id+N\right)
    \,.
  \]
  A connection matrix of $\cE^\phi\otimes\cR_\phi$ is then given by
  \[
    -\phi'\id + \frac{1}{t} \bigoplus_{(\alpha,N)}\left( \alpha\id+N\right) \,.
  \]
  The connection matrix of $\cM$ thus writes as
  \[
    \bigoplus_\phi\left(
      -\phi'\id + \frac{1}{t} \bigoplus_{(\alpha,N)}\left( \alpha\id+N\right)
    \right)
    =
    \underset{Q'(t^{-1})}{
      \underset{\text{\rotatebox[origin=c]{-90}{$=$}}}{%
        \underbrace{-\bigoplus_\phi \phi'\id}
      }
    }
    +
    \frac{1}{t}
    \underset{L}{
      \underset{\text{\rotatebox[origin=c]{-90}{$=$}}}{%
        \underbrace{%
          \bigoplus_\phi\left(
            \bigoplus_{(\alpha,N)}\left( \alpha\id+N\right)
          \right)
        }
      }
    }
    \,.
  \]

  \TODO{}

  \textbf{``\Leftarrow{}'':}
  Let $A^0$ be a normal form with $\mathcal{Y}_0(t)=F t^L e^{Q(t^{-1})}$ as a
  normal solution. We use proposition~\ref{prop:fundSolBuilder} to assume that
  the connection matrix $A^0$ has the form
  \[
    A^0=t^LQ'(t^{-1})t^{-L}+L\frac{1}{t}
  \]
  and the proposition~\ref{prop:systToMeromConn} yields our meromorphic
  connection $(\cM_{A^0},\nabla_{A^0})=(\C(\!\{t\}\!)^n,d-A^0)$ for
  \rewrite{which we want to prove, to be} a model.
  \TODO[Jordan-NF]
\end{proof}
\begin{cor}
  From the Levelt-Turittin theorem\comm{, the fundamental solution
  transformation rules} and theorem~\ref{thm:modelEqNormalForm} we
  deduce that every system $[A]$ with normal form $[A^0]$ and normal solution
  $t^L e^{Q(t^{-1})}$, has the fundamental solution
  \[
    \mathcal{Y}=\hat F t^L e^{Q(t^{-1})}
  \]
  where $\hat F$ is a solution of $[A^0,A]$.
\end{cor}

\begin{rem}
  Let $[A^0]$ be a normal form with normal solution
  $\mathcal{Y}_0(t)=t^L e^{Q(t^{-1})}$.
  \begin{enumerate}
    \item In the unramified case are \rewrite{the $q_i(t^{-1})$ the
      $\phi_i(t)$} from definition~\ref{defn:model}.
      \TODO[Why/realy?]
    \item In the unramified case $Q$ and $L$ commute thus $L$ can be supposed
      in Jordan form (cf.\ \cite[Sec.4]{Martinet1991}).
    \item It is always possible to permutate the columns of a fundamental
      solution by
      \[
        P^{-1}\mathcal{Y}P=\hat F t^{P^{-1}LP} e^{P^{-1}Q(t^{-1})P}
      \]
      with a permutation matrix $P$ and \rewrite{obtain another fundamental
      solution for the same system} (cf.\ \cite[73]{Loday2014}).
  \end{enumerate}
\end{rem}

%%%%%%%%%%%%%%%%%%%%%%%%%%%%%%%%%%%%%%%%%%%%%%%%%%%%%%%%%%%%%%%%%%%%%%%%%%%%%%%
\section{The main asymptotic existence theorem}
\TODO[move] \TODO[maybe \textbf{not necessary}, since we have Borel-Ritt!]
\begin{comment}
  \begin{multicols}{2}
    \textbf{Classical:}
    \begin{itemize}
      \item \cite[Thm.4.4.1]{Loday2014}
      \item \cite[Thm.7.10]{van2003galois}{\tiny\cite[Thm.7.12]{van2003galois}}
      \item \cite[Thm.12.1]{wasow2002asymptotic}
      \item \cite[5.3.Thm.1]{Varadarajan96linearmeromorphic}
      \item \cite[207]{Balser2000Formal}: Some historical remarks
      \item \cite[Thm.A]{BJL1979Birkhoff}
    \end{itemize}
  \columnbreak
    \textbf{Sheafical:}
    \begin{itemize}
      \item \cite[Thm.2.3.1]{sabbah_cimpa90}
      \item \cite[Sec.4.4]{Loday2014}
    \end{itemize}
  \end{multicols}
\end{comment}
Here we want to state the main asymptotic expansion theorem (or often M.A.E.T.)
which is essentially a deduction from the Borel-Ritt lemma.
\marginnote{\cite[207]{Balser2000Formal}}
It states that to every formal solution of a system of meromorphic differential
equations, and every sector with sufficiently small opening, one can find a
solution of the system having the formal one as its asymptotic expansion.

Look at the situation, where $A$ is via $\hat F$ formally equivalent to $A^0$.

\begin{defn}
  \marginnote{\cite[855]{Loday1994}}
  We call $F$ a \emph{lift of $\hat F$ on $I\subset S^1$} if
  \begin{itemize}
    \item $F\sim_I\hat F$
      (cf.\ Page~\pageref{page:notationForAsymptoticExpansion}) and
    \item $F$ satisfies the same system $[A^0,A]$ as $\hat F$.
  \end{itemize}
\end{defn}
The main asymptotic existence theorem states
\begin{thm}[M.E.A.T]\label{thm:meat}
  \marginnote{\cite[Thm.3.1]{boalch}, \cite[Sec.4.4]{Loday2014}}
  To every $\hat F\in G(\!(t)\!)$ and to every small enough arc
  $I\subsetneq S^1$ there exists a lift $F$ on $I$.
  \begin{s-rem}
    \marginnote{\cite[Thm.4.4.1]{Loday2014}}
    If we write the system $[A^0,A]$ as a differential operator $D$.
    The theorem then states, that $D$ acts linearly and surjectively on the
    sheaf $\cA^{<0}$, i.e.\ the sequence
    \[
      \cA^{<0}\overset{D}\longrightarrow\cA^{<0} \longrightarrow 0
    \]
    are exact sequences of sheaves of $\C$-vector spaces.
    \begin{comment}
      Proof in \textbf{[Mal91a, App 1; Thm 1]}
    \end{comment}
  \end{s-rem}
\end{thm}
\begin{proof}
  \TODO{}
\end{proof}

\begin{comment}
  \begin{rem}
    We are then able to find a \rewrite{(cyclic)} covering of the $S^1$ of
    arcs such that on every arc there exists a lift $\tilde F$ of $\hat F$.
  \end{rem}
\end{comment}
\TODO{}

Translated to the language of meromorphic connection the theorem~\ref{thm:meat}
becomes the following.
\begin{thm}[Sectorial decomposition]
  \marginnote{\cite[Thm.II.5.12]{sabbah2007isomonodromic}}
  Let $(\cM,\nabla)$ be a meromorphic connection and let
  $\hat\lambda:\hat\cM\to\hat\cM^{nf}$ be the isomorphism given by
  theorem~\ref{thm:leveltTurittin} together with the model $\cM^{nf}$.
  There exists then, for any $e^{i\theta^0}\in S^1$, an isomorphism
  $\tilde\lambda_{\theta^0}:
  \tilde\cM_{\theta^0}=\cA_{\theta^0}\otimes\cM\to\tilde\cM_{\theta^0}^{nf}$
  lifting $\hat\lambda$, that is, such that the following diagram
  \[ \begin{tikzcd}
      \tilde\cM_{\theta^o} \dar \rar{\tilde\lambda_{\theta^o}} &
      \tilde\cM^{nf}_{\theta^o} \dar
      \\\hat\cM \rar{\hat\lambda} &
      \hat\cM^{nf}
  \end{tikzcd} \]
  commutes
\end{thm}
\begin{proof}
  \rewrite{This is clear,} since the tensor is right exact and we have the
  Borel-Ritt Lemma.
  A proof can be found in~\cite[Sec.II.2.4]{sabbah_cimpa90}.
\end{proof}

\begin{comment}
  \begin{multicols}{2}
    \begin{thm}
      \marginnote{\cite[Thm.II.2.3.1]{sabbah_cimpa90}}
      \rewrite{Let $\cM_{K}$ be a meromorphic connection. There exists an
        integer $q\geq 1$ such that, after the ramification $x=t^q$, one has,
        for all $\theta\in S^1$ and each sufficiently small interval $V$
        centered at $\theta$}
      \[
        \cA_L(V)\otimes_L\cM_L\cong\cA_L(V)\otimes_L
        \left(\cF_L^R\otimes\cG_L\right)
      \]
    \end{thm}
    \begin{proof}
      See \cite[Sec.II.2.4]{sabbah_cimpa90}
    \end{proof}
  \columnbreak
  \end{multicols}
\end{comment}

%%%%%%%%%%%%%%%%%%%%%%%%%%%%%%%%%%%%%%%%%%%%%%%%%%%%%%%%%%%%%%%%%%%%%%%%%%%%%%%
\section{The classifying set}\label{sec:classifyingSet}
We want to understand the Set
$\Big\{\big[(\cM,\nabla)\big]\Big\}$ of the (convergent)
isomorphism classes of all meromorphic connections. Since we have also the
formal classification (cf.\ Section~\ref{sec:formalClassification}) and we
know, that all elements in a convergent isomorphism class lie in the same
formal isomorphism class, i.e.\ the convergent classification is
\rewrite{finer} than the formal classification, we can reduce the problem by
fixing a model $(\cM^{nf},\nabla^{nf})$ with the corresponding normal form
$A^0$.
\rewrite{Thus we can} restrict ourself to the subset
\begin{multline*}
  {}^0C(\cM^{nf},\nabla^{nf})=\Big\{
    \big[(\cM,\nabla)\big]
    \mid \text{there exists a formal isomorphism }
  \\\qquad\hat f:(\hat\cM,\hat\nabla)
      \overset{\sim}\longrightarrow
      (\hat\cM^{nf},\hat\nabla^{nf})
  \Big\}
\end{multline*}
of all isomorphism classes of meromorphic connections, which are formally
isomorphic to $(\cM^{nf},\nabla^{nf})$. This is the set, that we will be
calling the \emph{classifying set} and we will also denote it ${}^0C(A^0)$.
\begin{rem}
  Note that we classify
  \begin{einr}
    meromorphic connections within fixed \textbf{formal meromorphic classes,
    modulo meromorphic equivalence}.
  \end{einr}
  Whereas in \cite{boalch} and \cite{thboalch}
  \begin{einr}
    meromorphic connections within fixed \textbf{formal analytic classes,
    modulo analytic equivalence}
  \end{einr}
  are classified as is done in the older literature. \rewrite{The resulting
  classifying sets are isomorphic.}
  \\This distinction relates to the difference between \textbf{‘regular
  singular’} connections and \textbf{‘logarithmic’} connections.
\end{rem}
\marginnote{\cite[6]{thboalch}
  \\\tiny{(\cite[19]{boalch})}
  \\\cite[852]{Loday1994}
  \\\cite[111]{sabbah2007isomonodromic}
  \\\cite{babbitt1983}}
It is convenient to look at the slightly larger space of \emph{marked pairs}
\[
  \cH=\cH(\cM^{nf},\nabla^{nf})=\left\{
    \left[(\cM,\nabla,\hat f)\right]
      \mid
      \hat f:(\hat\cM,\hat\nabla)
        \overset{\sim}\longrightarrow
        (\hat\cM^{nf},\hat\nabla^{nf})
  \right\}
\]
in which we also handle the additional information, of the formal isomorphism,
by which the meromorphic connection is formally isomorphic to the model.
Where the isomorphisms of marked pairs are defined as follows:
\begin{defn}
  Two germs $(\cM,\nabla,\hat f)$ and $(\cM',\nabla',\hat f')$ are
  isomorphic if there exists an isomorphism
  $g:(\cM,\nabla)\overset{\sim}\longrightarrow(\cM',\nabla')$ such that
  $\hat f=\hat f'\circ \hat g$.
  \begin{s-rem}
    \rewrite{Sabbah states in} \cite[111]{sabbah2007isomonodromic}, that such
    an isomorphism is then unique.
  \end{s-rem}
\end{defn}

Equivalently, one can talk in terms of systems. We then denote by
\[
  \Syst_m(A^0):=\{[A]
    \mid A={}^{\hat F}\!A^0 \text{ for some } \hat F\in G(\!(t)\!)\}
\]
the set of systems formally meromorphic equivalent to $A^0$.
Since we use meromorphic equivalences, in contrast to \cite{boalch,thboalch},
we denote that in $\Syst_m$ by the subscript ${}_m$.
Thus ${}^0C(\cM^{nf},\nabla^{nf})$ corresponds to
the set ${}^0C(A^0):=\Syst_m(A^0)/G(\!\{t\}\!)$ of meromorphic classes which
are formally equivalent to $A^0$.
\TODO[\cite{thboalch} p. 3: In the logarithmic case\dots]
Analogous, $\cH(\cM^{nf},\nabla^{nf})$ corresponds to the set $\cH(A^0)$ of
equivalence classes\rewrite{, i.e. orbits of $G(\!\{t\}\!)$,} in
\[
  \hat\Syst_m(A^0):=\{(A,\hat F)
    \mid A={}^{\hat F}\!A^0 \text{ for some } \hat F\in G(\!(t)\!)\} \,.
\]

\begin{lem}
  Since $G_0(A^0)$ is by definition the stabilizer\TODO[correct?] of $A^0$ we
  deduce
  \[
    \Syst_m(A^0)\cong \hat G(A^0)/G_0(A^0) \,.
  \]
  \begin{s-cor}
    \marginnote{\cite[Eq.1.9b]{babbitt1983}}
    Thus the \emph{set of meromorphic classas of systems formally equivalent
      to $A^0$} are just the orbits of $G\{t\}$, that is
    \[
      {}^0C(A^0)\cong G(\!\{t\}\!)\backslash\hat G(A^0)/G_0(A^0)
    \]
    whereas the \emph{set of meromorphic classes of transformations $\cH(A^0)$
    of $[A^0]$} is canonically isomorphic to the left quotient
    $G(\!\{t\}\!)\backslash\hat G(A^0)$ (cf.\ \cite[Lem.1.17]{thboalch}).
  \end{s-cor}
\end{lem}
\begin{proof}
  See
  \begin{itemize}
    \item~\cite[6]{thboalch}: in the case $G_0(A^0)=T$
  \end{itemize}
\end{proof}

The group $G_0(A^0)$ is easy to compute and is often trivial. In fact, the
elements are block-diagonal, see~\cite[77]{Loday2014}.
Thus the structure of ${}^0C(A^0)$ is easily deduced from the structure of
$G(\!\{t\}\!)\backslash\hat G(A^0)$.
