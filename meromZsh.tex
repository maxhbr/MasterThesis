\chapter{Meromorphic connections}
\begin{comment}
  \begin{multicols}{2}
    \textbf{Global}
    \begin{itemize}
      \item Meromorphic connection
      \item $\cD$-modules (global)
    \end{itemize}
    \columnbreak
    \textbf{Local}
    \begin{itemize}
      \item Germ of a meromorphic connection
      \item $\cD$-modules (local)
      \item System
        \begin{itemize}
          \item coordinate dependent
        \end{itemize}
      \item Conection matrix
        \begin{itemize}
          \item coordinate dependent
        \end{itemize}
    \end{itemize}
  \end{multicols}
\end{comment}
There are multiple languages, which can be used for talking about meromorphic
connections. Firstly the languages of meromorphic connections, and also
$\cD$-modules\TODO[global]. These two can be used to talk about global
information. For local description one can use germs of meromorphic connections
or $\cD$-modules\TODO[local] which are coordinate independent. Coordinate
dependent alternatives are (local) systems and connection matrices. Also, one
can use the fundamental solutions to describe local informations of meromorphic
connections.
\begin{comment}
  Siehe:
  \begin{multicols}{3}
    \begin{itemize}
      \item \cite{boalch} and \cite{thboalch}
      \item \cite{sabbah2007isomonodromic}
    \end{itemize}
    \columnbreak
    \begin{itemize}
      \item \cite{Varadarajan96linearmeromorphic}
    \end{itemize}
    \columnbreak
    \begin{itemize}
      \item \cite{Loday1994}
      \item \cite{Loday2014}
    \end{itemize}
  \end{multicols}
\end{comment}

Let $M$ be a riemanian surface and let $Z=k_1(a_1)+\cdots+k_m(a_m)>0$ be an
effective divisor\footnote{The $a_i$ are distinct points and the $k_i$ are
positive integers.} on $M$.
It is sufficient to think $M=\P^1$ and $Z=1(0)+1(\infty)$\TODO[\infty?], since
we will only be interested in local information (at $0$).

Let $\sM$ be a holomorphic Bundle over $M$ i.e.\ a locally free $\cO_M$-module
of rank $n$.

A meromorphic connection is then defined as follows.
\begin{defn}
  \def\myU{\textcolor{green!30!black}{U}}
  \def\mys{\textcolor{blue!60!black}{s}}
  \def\myf{\textcolor{red!60!black}{f}}
  A \emph{meromorphic connection $(\sM,\nabla)$ on $\sM$ with poles on $Z$}
  is defined by a $\C$-linear morphism of sheaves
  \[
    \nabla:\sM\to\Omega_M^1(*Z)\otimes\sM
  \]
  satisfying, for each $\myU\underset{\text{op.}}{\subset} M$, the
  \emph{Leibniz rule}
    \[
      \nabla(\myf\mys)=\myf\nabla\mys+(d\myf)\otimes\mys
    \]
  for $\mys\textcolor{blue!60!black}{\in\Gamma(\myU,\sM)}$ and
  $\myf\textcolor{red!60!black}{\in\cO_M(\myU)}$.
  The rank of the Bundle $\sM$ is the \emph{rank} of the meromorphic connection
  $(\sM,\nabla)$.
\end{defn}
\begin{rem}
  \begin{enumerate}
    \item We will occasionally omit the $\nabla$ and simply call $\sM$ the
      meromorphic connection.
    \item Here, the variant `holomorphic bundle with meromorphic connection' is
      chosen, like in \cite{boalch}.
      There is also the twisted\TODO[better word?] description `meromorphic
      bundle with holomorphic connection' which is used in
      \cite{sabbah2007isomonodromic}.
      \\ By choosing a lattice of a meromorphic bundle, one gets a holomorphic
      bundle but the connection is no longer holomorphic. Such that we obtain a
      meromorphic connection in our\dots\TODO
  \end{enumerate}
\end{rem}

\begin{defn}
  \marginnote{\cite[0.12.2]{sabbah2007isomonodromic}}
  The connection $\nabla:\cM\to \Omega_M^1\otimes_{\cO_M}\cM$ is said to be
  \emph{integrable} or \emph{flat}, if
  \begin{itemize}
    \item its curvature $R_\nabla\equiv0$
    where
    \begin{itemize}
      \item $R_\nabla:=\nabla\circ\nabla:\cE\to\Omega_M^2\otimes_{\cO_M}\cE$
        is a $\cO_M$-linear morphism.
    \end{itemize}
  \end{itemize}
  \begin{prop}[0.12.4]
    The connection $\nabla$ is flat if and only if, in any local basis $e$ of
    $\cM$, the connection matrix $\Omega$ satisfies
    \[
      d\omega + \omega \wedge \omega = 0.
    \]
  \end{prop}
  \begin{rem}
    Here are all connections flat, since \TODO{}
  \end{rem}
  \begin{comment}
    We will say that a connection on a meromorphic bundle is \emph{integrable}
    or \emph{flat} if its restriction to $M\backslash Z$ is an integrable
    connection on the holomorphic bundle $\sM_{|M\backslash Z}$.
  \end{comment}
\end{defn}

%%%%%%%%%%%%%%%%%%%%%%%%%%%%%%%%%%%%%%%%%%%%%%%%%%%%%%%%%%%%%%%%%%%%%%%%%%%%%%%
\section{Local expression of meromorphic connections}
\TODO[Germ of a meromorphic Connection / Connection matrices
  / systems]
\begin{comment}
  \begin{itemize}
    \item \cite{sabbah2007isomonodromic} p.28
    \item \cite{thboalch} p.2
    \item \cite{babbitt1989local} p. 11
  \end{itemize}
\end{comment}
We will usually\TODO[only?] be interested in local information of a meromorphic
connection.  This means, that we look at a connection in a neighbourhood of $0$
and allow only one singularity at $0$.
There are many ways of expressing the local information, we will either talk
about germs of meromorphic connections or systems, which is coordinate-
and trivialization-dependent.

\begin{comment}
  \begin{paracol}{2}\sloppy
  \switchcolumn[0]\noindent
    Let $U$ be a trivializing open set for $M$ and let
    $\textbf{e}=(e_1,\dots,e_n)$ be a basis of $\Gamma(U,\sM)$.
    There exists then a $n\times n$ matrix $A$ of meromorphic 1-forms such
    that\dots
    \begin{defn}
      This matrix is called \emph{connection matrix} for $(\sM,\nabla)$.
    \end{defn}
  \switchcolumn[1]\noindent
    Let $U$ be a neighbourhood of $0$ and $t$ a coordinate on $U$ vanishing at
    $0$.
  \end{paracol}
\end{comment}
\begin{comment}
  Some Literature like \cite{sabbah_cimpa90} always talks about germs\dots
\end{comment}
\begin{prop}
  \marginnote{\cite[Def.4.2.1]{Loday2014}}
  A germ $(\cM,\nabla)$ of a meromorphic connection $(\sM,\nabla)$ is just the
  sheaf-theoretic germ, thus has the form $(\cM,\nabla)$ where
  \begin{itemize}
    \item $\cM$ is the germ at $0$ of the holomorphic bundle $\sM$ and thus a
      $\C(\!\{t\}\!)$-vectorspace of dimension $n$, since the ring of germs of
      meromorphic functions with poles at $0$ is the ring $\C(\!\{t\}\!)$, and
    \item $\nabla:\cM\to \cM$ is a additive map, which satisfies the
      \emph{Leibniz rule}
      \[
        \nabla(fm)=\partial f\cdot m + f\nabla(m)
      \]
      for all $f\in\C(\!\{t\}\!)$ and $m\in \cM$.
  \end{itemize}
  \begin{rem}
    \marginnote{\cite{sabbah2007isomonodromic}}
    It is a $(\C(\!\{t\}\!),\nabla)$-vectorspace.
  \end{rem}
  \begin{comment}
    \begin{rem}
      Loday-Richaud calls this in \cite[Def.4.2.1]{Loday2014} a
      \emph{differential module}.
    \end{rem}
  \end{comment}
\end{prop}
\begin{defn}
  A \emph{(iso-)\,morphism of systems} $\Phi:(\cM,\nabla)\to(\cM',\nabla')$
  is a (iso-)\,morphism of vectorspaces $\Phi:\cM\to\cM'$ which commutes with the
  connections.
  \TODO[Category of systems!]
\end{defn}
\marginnote{\cite[65]{Loday2014}}
\begin{paracol}{2}\sloppy
\switchcolumn[0]\noindent
  % Choose a $\C(\!\{t\}\!)$-basis $\underline{e}=(e_1,e_2,\dots,e_n)$ of $\cM$
  % and let $A\in\End(E)\otimes\C(\!\{t\}\!)$\TODO[In one forms? $A=A'dt$?] be a
  % matrix, such that
  % \[
  %   (\epsilon_1,\epsilon_2,\dots,\epsilon_n)=-(e_1,e_2,\dots,e_n)A
  % \]
  % is the image of $\nabla$.
  \textcolor{gray}{Choose a $\C(\!\{t\}\!)$-basis
    $\underline{e}=(e_1,e_2,\dots,e_n)$ of $\cM$.  Let $A'$ be a $n\times n$
    matrix with entries in $\C(\!\{t\}\!)$ such that
    \[
      \nabla\left(e_1,\dots,e_n\right)
      =
      -(e_1,e_2,\dots,e_n)A'
    \]
    and let $A=A'dt$.}
\switchcolumn[1]\noindent
  Choose a $\C(\!\{t\}\!)$-basis $\underline{e}=(e_1,e_2,\dots,e_n)$ of $\cM$.
  Let $A'$ be a $n\times n$ matrix with entries in $\C(\!\{t\}\!)$ such that
  the matrix $A=A'dt$ describes the action of $\nabla$:
  \[
    \nabla\left(e_1,\dots,e_n\right)
    =
    -(e_1,e_2,\dots,e_n)A \,.
  \]
\end{paracol}
Let $x=\sum_{0\leq j\leq n}x_je_j$ be an arbitrary element of $\cM$ which is in
matrix notation described as $x=\underline{e}X$ with the column matrix
$X={}^t\!(x_1,x_2 ,\dots,x_n)$.
Then, applying the Leibniz rule, yields
\begin{align*}
  \nabla x&=\nabla\left(\underline{e}\cdot X\right)
  \\&=\underline{e} \cdot dX - \underline{e}\cdot \nabla X
  \\&=\underline{e}\left(dX-AX\right).
\end{align*}
Thus, with the connection $\nabla$ and the $K$-basis $\underline{e}$ is
naturally associate the differential operator $\nabla=[A]=d-A$, which has order
one and dimension $n$. This means that the connection $\nabla$ is fully
determined by the matrix $A$ and thus is fully determined by $A'$.
\begin{defn}
  \begin{enumerate}
    \item This matrix $A$ is called a \emph{connection matrix} of
      $(\cM,\nabla)$. It depends on the choice of the $\C(\!\{t\}\!)$-basis.
    \item A \emph{germ of a meromorphic linear differential system} of rank
      $n$, or just a \emph{system}, is a germ of a meromorphic connection on
      the trivial vector bundle \textbf{with a chosen trivialization} of the
      fiber $E\cong\C^n$.
      \begin{prop}
        Thus, the set of systems is isomorphic to the set
        $G(\!\{t\}\!)$ of all connection matrices.
      \end{prop}
      It well be denoted by $[A]$ where $A$ is the connection matrix.
  \end{enumerate}
\end{defn}

\begin{comment}
  \cite{boalch} wants \textbf{generic} meromorphic connections
  \begin{itemize}
    \item\dots simplest jet sufficient\dots
  \end{itemize}
\end{comment}

%%%%%%%%%%%%%%%%%%%%%%%%%%%%%%%%%%%%%%%%%%%%%%%%%%%%%%%%%%%%%%%%%%%%%%%%%%%%%%%
\subsubsection{System \rightarrow{} germ of a meromorphic connection}
\begin{comment}
  There is a thm in \cite{sabbah2007isomonodromic}
\end{comment}
\begin{prop}
  If we start with a system $[A]$ we get a germ of a meromorphic connection via
  \[
    (\cM,\nabla)=(\C(\!\{t\}\!))^n,d-A)
  \]
  and $A$ is a connection matrix for $(\cM,\nabla)$.
\end{prop}
\begin{proof}
  \TODO{}
\end{proof}

%%%%%%%%%%%%%%%%%%%%%%%%%%%%%%%%%%%%%%%%%%%%%%%%%%%%%%%%%%%%%%%%%%%%%%%%%%%%%%%
\subsubsection{Local \rightarrow{} global}
\begin{comment}
  Quelle?
  \begin{itemize}
    \item \cite{sabbah2007isomonodromic}???
  \end{itemize}
\end{comment}
\begin{thm}
  If we start with a germ at $0$ of a meromorphic connection $(\cM,\nabla)$
  there is a unique meromorphic connection $(\sM,\nabla)$ such that
  \begin{itemize}
    \item $(\sM,\nabla)$ has only singularities at $0$ and $\infty$
    \item the singularity at $\infty$ is only \TODO{} and
    \item $(\cM,\nabla)$ is the germ at $0$ of $(\sM,\nabla)$.
  \end{itemize}
\end{thm}
\begin{proof}
  \TODO{}
\end{proof}

%%%%%%%%%%%%%%%%%%%%%%%%%%%%%%%%%%%%%%%%%%%%%%%%%%%%%%%%%%%%%%%%%%%%%%%%%%%%%%%
\subsection{Meromorphic / formal transformation}
\begin{comment}\footnotesize
  see \cite{thboalch} \textbf{Rem 1.41 on p. 16}:
  \begin{rem}
    Note that in most of the recent references we have used, Stokes matrices
    are used to classify
    \begin{itemize}
      \item meromorphic connections within fixed \textbf{formal meromorphic
        classes, modulo meromorphic equivalence}.
    \end{itemize}
    Whereas here we classify
    \begin{itemize}
      \item meromorphic connections within fixed \textbf{formal analytic
        classes, modulo analytic equivalence},
    \end{itemize}
    as is done in the older literature.  The fact is that the sets equivalence
    classes are the same in both cases. It is important for us to work with
    analytic, rather than meromorphic gauge transformations, because then the
    $\C^\infty$ viewpoint in Chapter 3 is cleaner. This distinction relates to
    the difference between \textbf{‘regular singular’} connections and
    \textbf{‘logarithmic’} connections.
  \end{rem}
\end{comment}
\begin{notations}
  We will use the following notations
  \begin{itemize}
    \item $G=\Gl_n(\C)$;
    \item $G[t]=\Gl_n(\C[t])$;
    \item $G\{t\}=\Gl_n(\C\{t\})$ analytic tranformations;
    \item $G(\!\{t\}\!)=\Gl_n(\C\{t\}[t^{-1}])$ meromorphic transformations;
    \item $G\llbracket t\rrbracket=\Gl_n(\C\llbracket t\rrbracket)$
      (maby not applicable) formal tranformations;
    \item $G(\!(t)\!)=\Gl_n(\C\llbracket t\rrbracket[t^{-1}])$
      (maby not applicable) formal meromorphic tranformations;
  \end{itemize}
\end{notations}
By \emph{transformation} (or \emph{meromorphic transformation}) of a system we
mean a linear change of the trivialization.
Such a change is given by a matrix $F\in G(\!\{t\}\!)$ and the transformed
connection matrix ${}^F\!A$ is obtained through
\[
  {}^F\!A=(dF)F^{-1} + FAF^{-1} \,.
\]
If $F$ is formal i.e.\ $F\in G(\!(t)\!)$, it will usually be denoted by
$\hat F$.
The transformation of $A$ by $\hat F$ is not guaranteed to have convergent
entries.
We denote by $\hat G(A)$ the set of all \emph{(applicable) formal
transformations}
\[
  \hat G(A):=\left\{\hat F\in G(\!(t)\!)
    \mid {}^{\hat F}\!A \text{ has convergent entries i.e.\ } 
    {}^{\hat F}\!A\in G(\!\{t\}\!)
  \right\}\,.
\]
\begin{rem}
  \def\myB{\textcolor{blue!60!black}{B}}
  \def\myA{\textcolor{green!30!black}{A}}
  \def\myF{\textcolor{red!60!black}{F}}
  The condition
  \begin{itemize}
    \item[] $\myB$ is obtained from $\myA$ by transformation $\myF$
  \end{itemize}
  is clearly equivalent to
  \begin{itemize}
    \item[]  $\myF$ solves the linear differential system
      \[
        \frac{d\myF}{dt}=\myB\myF-\myF\myA
      \]
      which is denoted by $[\myA,\myB]$.
  \end{itemize}
\end{rem}
We define the (formal) equivalence relation on the connection matrices as
\textbf{\boldmath$A$ is (formally) equivalent to $B$} if and only if
\textbf{\boldmath$B$ is obtained from $A$ by (formal) transformation}.
Thus $A$ is (formally) equivalent to $B$ if and only if there is a (formal)
solution of $[A,B]$.

\begin{defn}
  \TODO[is this a proposition]
  Two germs of meromorphic connections are (formally) isomorphic if and only if
  their corresponding connection matrices are (formally) equivalent.
\end{defn}

%%%%%%%%%%%%%%%%%%%%%%%%%%%%%%%%%%%%%%%%%%%%%%%%%%%%%%%%%%%%%%%%%%%%%%%%%%%%%%%
\subsubsection{Regular / irregular singularities}
\begin{defn}
  \marginnote{\cite[86]{sabbah2007isomonodromic}}
  A connection defined by a matrix $A$ has \emph{regular singularity} at $0$ if
  there exists a konvergent transformation, by which $A$ is obtained from a
  matrix with at most a simple pole at $t=0$.
  Otherwise, the singularity is called \emph{irregular}.
\end{defn}
\begin{thm}
  \marginnote{\cite[Thm.II.2.8]{sabbah2007isomonodromic}}
  Let $(\cM,\nabla)$ be a regular singular meromorphic connection and $A$ its
  connection matrix.
  Then there exists a matrix $F\in G(\!\{t\}\!)$ such that after transformation
  by $F$ the matrix $B={}^F\!A$ takes the form
  \[
    B=B'dt
  \]
  with a constant matrix $B'\in G$.
\end{thm}

%%%%%%%%%%%%%%%%%%%%%%%%%%%%%%%%%%%%%%%%%%%%%%%%%%%%%%%%%%%%%%%%%%%%%%%%%%%%%%%
\subsection{Fundamental solution}

%%%%%%%%%%%%%%%%%%%%%%%%%%%%%%%%%%%%%%%%%%%%%%%%%%%%%%%%%%%%%%%%%%%%%%%%%%%%%%%
\subsection{As differential operator}
\begin{comment}
  \begin{itemize}
    \item \cite[Sec.4.2]{Loday2014}
  \end{itemize}
\end{comment}

%%%%%%%%%%%%%%%%%%%%%%%%%%%%%%%%%%%%%%%%%%%%%%%%%%%%%%%%%%%%%%%%%%%%%%%%%%%%%%%
\section{Models and formal decomposition of a germ / formal classification}
\begin{paracol}{2}
\switchcolumn[0]\noindent
  \textcolor{gray}{As something like building blocks of some meromorphic
    connections we introduce the elementary irregular models. Into which
    meromorphic connections, after potentially needed pullback, decompose.
    \\The building blocks of these building blocks are defined as follows.}
\switchcolumn[1]\noindent
  In every formal equivalence class of meromorphic connections, there are some
  meromorphic connection of special form, which we will call models. They are
  not unique but all of them lie in the same convergent equivalence class.

  On the other hand\TODO[same direction?], the Levelt-Turittin theorem says,
  that  each meromorphic connection is, after potentially needed ramification,
  formally isomorphic to such a model, that means, that in each formal class
  there are some models.
  \TODO[Problems with ramification??]
  This solves the formal classificaton problem.
\end{paracol}
\begin{defn}
  \begin{itemize}
    \item For a $\phi\in\C\{t\}[t^{-1}]$ we use $\cE^{\phi}$ to denote the germ
      \[
        (\cE^{\phi},\nabla)=(\C\{t\},d-d\phi)\,.
      \]
      This is the system satisfied by the function $e^\phi$.
      \begin{cor}
        $\cE^\phi$ is determined by the class of $\phi$ in
        $\C\{t\}[t^{-1}]/\C\{t\}=t^{-1}\C[t^{-1}]$. In the following, we will
        only consider the unique element $\phi$ in each class which has no
        holomorphic part.
      \end{cor}
    \item For $\alpha\in\C$, define $\cN_{\alpha,0}$ as the germ
      \[
        (\cN_{\alpha,0},\nabla)=(\C\{t\}[t^{-1}],d+\alpha dt/t)\,.
      \]
      This is the system satisfied by \TODO{}.
  \end{itemize}
\end{defn}
\begin{prop}
  \begin{enumerate}
    \item Every regular germ of a meromorphic connection is isomorphic to some
      \TODO{}
      \begin{rem}
        For a detailed analysis of regular meromorphic connections
        see~\cite{sabbah2007isomonodromic} chapter II.2.
      \end{rem}
    \item Every germ of a meromorphic connection of rank one is isomorphic to
      some germ
      \[
        \cE^\phi\otimes\cN_{\alpha,0} \,.
      \]
    \item Two such germs corresponding to $(\phi_1,\alpha_1)$ and
      $(\phi_2,\alpha_2)$ are isomorphic if and only if
      \begin{itemize}
        \item $\phi_1-\phi_2$ has no pole and
        \item $\alpha_1-\alpha_2\in\Z$.
      \end{itemize}
  \end{enumerate}
\end{prop}
\begin{proof}
  See~\cite{sabbah2007isomonodromic} Proposition II.5.1.
\end{proof}
\begin{defn}
  \marginnote{\cite{sabbah2007isomonodromic} Definition II.5.2}
  A Germ $(\cM,\nabla)$ is called \emph{elementary} if it is isomorphic to
  some germ $(\cE^\phi,\nabla)\otimes(\cR,\nabla)$ where
  \begin{itemize}
    \item $(\cR,\nabla)$ has regular singularity at $\{0\}$ but has not to be
      of rank $1$.
  \end{itemize}
  \marginnote{\cite{sabbah2007isomonodromic} II.2.f}
\end{defn}
\begin{defn}
  \def\myPhi{\textcolor{red!60!black}{\phi}}
  \def\myE{\textcolor{green!40!black}{\cE^{\myPhi}}}
  A germ $(\cM,\nabla)$ is a \emph{model} if there exists a isomorphism
  \begin{multicols}{2}
    \[
      \lambda:(\cM,\nabla)
      \overset{\cong}{\longrightarrow}
      % \cong
      \bigoplus_{~\tikzmark{e3}\!\!\myPhi}
      \underset{\text{merom. Zus.}}{%
        \underset{\text{elementare}}{%
          \underbrace{%
            \overset{\tikzmark{e2}}{\myE}
            \otimes
            \overset{\tikzmark{e1}}{\textcolor{blue!40!black}{\cR_{\myPhi}}}
          }
        }
      }\,.
    \]
    \columnbreak{}
    \begin{itemize}
      \item[\tikzmarkb{n2}{green}] is irregular singular
      \item[\tikzmarkc{n1}{blue}] has regular singularity at $\{0\}$
      \item[\tikzmarkc{n3}{red}] $\myPhi\in t^{-1}\C[t^{-1}]$ pairwise distinct
    \end{itemize}
    \begin{tikzpicture}[remember picture,overlay]
      \draw[->,blue!50!white,thick] (n1) to[out=180,in=70] (e1);
      \draw[->,green!40!black,thick] (n2) to[out=180,in=70] (e2);
      \draw[->,red!50!white,thick] (n3) to[out=205,in=-70] (e3);
    \end{tikzpicture}
  \end{multicols}
\end{defn}
\begin{lem}
  Let $(\cM,\nabla)$ be a model of a meromorphic connection, which is
  represented by the system $[A^0]$.
  \begin{enumerate}
    \item 
      If $(\cM,\nabla)$ is a model, if and only if there is a basis in which
      the fundamental solution\footnote{A $n\times n$ matrix whose
      columns are $n$ $\C$-linearly independent solutions of the system
      $[A^0]$.} $\mathcal{Y}_0$ has the form
      \[
        \mathcal{Y}_0(t)=t^L e^{Q(t^{-1})}
      \]
      with
      \begin{itemize}
        \item
          $Q(t^{-1})=\underset{j=1}{\overset{r}{\bigoplus}}q_j(t^{-1})1_{r_j}$
          \begin{itemize}
            \item $q_j$ is the function, such that $q_j(t^{-1})=\phi_j(t)$
              \TODO[replace all $q$ by $\phi$?]
            \item the $r_j$ are \TODO{} and satisfy therefore $\sum r_j=n$.
          \end{itemize}
          This means, that $Q=\diag(\underset{r_1\text{-times}}{\underbrace{%
            q_1,\dots,q_1}},q_2,\dots,q_r)$.
          \TODO[ramified case]
        \item $L\in\gl_n(\C)$ constant matrix called the \emph{matrix of
          formal monodromy}. ($t^L$ means $e^{L\ln t}$.)
          \marginnote{in \cite[1]{Remy2014} $L$ is just a Jordan normal form.
          Is this generic enough?}
      \end{itemize}
      The matrix $\mathcal{Y}_0$ is called a \emph{normal solution}.
    \item The matrix ${A^0}':=dQ+L\frac{dt}{t}$ is a connection matrix for
      $(\cM,\nabla)$.
      \TODO[Therefore it is obtained from $A^0$ via an isotropy?]
    \item If $(\cM',\nabla')$ is obtained from $(\cM,\nabla)$ via the formal
      meromorphic transformation $\hat F$, then is
      \[
        \mathcal{Y}=\hat F t^L e^{Q(1/t)}
      \]
      a fundamental solution for $Y'={}^{\hat F}\!A^0(t)Y$.
  \end{enumerate}
\end{lem}
\begin{proof}
  \TODO{}
\end{proof}
We will denote
\[
  \cQ_{[A^0]}:=\left\{q_1(t^{-1}),\dots,q_n(t^{-1})\right\} \,.
\]

The important theorem here is the Levelt-Turittin theorem, which solves the
formal classification problem.
\begin{thm}[Levelt-Turittin]
  To each germ $(\cM,\nabla)$ of a meromorphic connection there exists, after
  potentially needed \textcolor{blue!60!black}{pullback by some suitable
  ramification $t=z^q$ of order $q\geq1$}, a
  \textcolor{green!30!black}{\textbf{formal}} isomorphism
  \[
    \textcolor{green!30!black}{\hat{\textcolor{black}{\lambda}}}:
    \textcolor{blue!60!black}{\pi^{+}}
    \textcolor{green!30!black}{\hat{\textcolor{black}{\cM}}}
    \overset{\cong}\longrightarrow
    \textcolor{green!30!black}{\hat{\textcolor{black}{\cM}}^{model}}
    :=\textcolor{green!30!black}{\hat\cO_M\otimes}\cM^{model}
  \]
  to a model $\cM^{model}$.
  We then call $\cM^{model}$ a \emph{formal decomposition} or \emph{formal
  model} of $\cM$.
\end{thm}
\begin{proof}
  See \TODO{}
\end{proof}
\begin{cor}
  \marginnote{\cite[Thm.4.3.1]{Loday2014}}
  To any system $[A]$ there exists a fundamental solution $\mathcal{Y}$ of the
  form $\mathcal{Y}=\hat F t^\Lambda e^{Q(1/t)}$.
\end{cor}

\begin{prop}
  \marginnote{This condition \textbf{might be} equivalent to the
    condition of being \textbf{nice} in~\cite{thboalch}.}
  Let $(\cM,\nabla)$ be a germ,
  \begin{itemize}
    \item equipped with a basis in which the matrix $A$ takes the form
      \[
        A=t^{-r}A(t)\frac{dt}{t}
      \]
      with
      \begin{itemize}
        \item $r\geq1$,
        \item $A$ has holomorphic entries, and
        \item $A_0:=A(0)$ being regular semisimple\footnote{i.e.\ with
          pairwise distinct eigenvalues.}.
      \end{itemize}
  \end{itemize}
  Then there is no ramification needed, to apply the Levelt-Turittin-theorem.
  \begin{comment}
    Further, all the summands $\cR_\phi$ have rank one, which is not the case
    in general.
  \end{comment}
\end{prop}
\begin{proof}
  See~\cite{sabbah2007isomonodromic} theorem II.5.7.
\end{proof}

%%%%%%%%%%%%%%%%%%%%%%%%%%%%%%%%%%%%%%%%%%%%%%%%%%%%%%%%%%%%%%%%%%%%%%%%%%%%%%%
\subsection{The main asymptotic existence theorem}
\begin{comment}
  \begin{multicols}{2}
    \textbf{Classical:}
    \begin{itemize}
      \item \cite[Thm.4.4.1]{Loday2014}
      \item \cite[Thm.7.10]{van2003galois}{\tiny\cite[Thm.7.12]{van2003galois}}
      \item \cite[Thm.12.1]{wasow2002asymptotic}
      \item \cite[5.3.Thm.1]{Varadarajan96linearmeromorphic}
    \end{itemize}
  \columnbreak
    \textbf{Sheafical:}
    \begin{itemize}
      \item \cite[Thm.2.3.1]{sabbah_cimpa90}
    \end{itemize}
  \end{multicols}
\end{comment}
Here we want to state the main asymptotic existence theorem (or often M.A.E.T.)
which is essentially a deduction from the Borel-Ritt lemma.

%%%%%%%%%%%%%%%%%%%%%%%%%%%%%%%%%%%%%%%%%%%%%%%%%%%%%%%%%%%%%%%%%%%%%%%%%%%%%%%
\section{Meromorphic / formal transformation II}
\begin{defn}
  \begin{itemize}
    \item An \emph{isotropy} of $A^0$ is a transformation $\hat F$ which
      satisfy ${}^{\hat F}\!A^0=A^0$.
      Thus, the isotropies are the solutions of the system
      $[\End A^0]:=[A^0,A^0]$.
      \marginnote{\cite[853]{Loday1994}}
      \begin{rem}
        They are, a priori, formal transformations. Actually $G(A^0)$ is a
        subgroup of $\Gl_n(\C[1/x,x])$.
      \end{rem}
    \item Let $G_0(A^0)$ denote the set of all isotropies of $A^0$.
      \begin{comment}
        In the nice case this is only $T$?
      \end{comment}
  \end{itemize}
\end{defn}
\begin{lem}
  \marginnote{\cite[854]{Loday1994}}
  Two formal transformations $\hat F_1$ and $\hat F_2$ take $A^0$ into
  equivalent matrices ${}^{\hat F_1}\!A^0$ and  ${}^{\hat F_2}\!A^0$ if and
  only if there exists $f_0\in G_0(A^0)$ such that $\hat F_1=\hat F_2f_0$.
  \TODO[\Leftrightarrow{} $f_0$ is a isotropy?]
\end{lem}
\begin{proof}
  \TODO{}
\end{proof}

%%%%%%%%%%%%%%%%%%%%%%%%%%%%%%%%%%%%%%%%%%%%%%%%%%%%%%%%%%%%%%%%%%%%%%%%%%%%%%%
\section{The classifying set}
We want to understand the Set $\{[(\cM,\nabla)]\}$ of the (convergent)
isomorphism classes of all meromorphic connections. Since we have also the
formal classification and we know, that all elements in a convergent
isomorphism class lie in the same formal isomorphism class, we can reduce the
problem by fixing a model $(\cM^{model},\nabla^{model})$ with the corresponding
connection matrix $A^0$. Thus we can restrict ourself to the subset
\begin{align*}
  {}^0C(\cM^{model},\nabla^{model})=\big\{
    \left[(\cM,\nabla)\right]
    &\mid \text{there exists a formal isomorphism }
  \\&\qquad\hat f:(\hat\cM,\hat\nabla)
      \overset{\sim}\longrightarrow
      (\hat\cM^{model},\hat\nabla^{model})
  \big\}
\end{align*}
of all isomorphism classes of
meromorphic connections, which are formally isomorphic to
$(\cM^{model},\nabla^{model})$. This is the set, that we will be calling the
\emph{classifying set}.
\begin{comment}
  \begin{itemize}
    \item \cite{thboalch} p.6
      \begin{itemize}
        \item \cite{boalch} p.19
      \end{itemize}
    \item \cite{Loday1994} p.852
    \item \cite{sabbah2007isomonodromic} p.111
  \end{itemize}
\end{comment}

It is convenient to look at the slightly larger space of \emph{marked pairs}
\[
  \cH=\cH(\cM^{model},\nabla^{model})=\left\{
    \left[(\cM,\nabla,\hat f)\right]
      \mid
      \hat f:(\hat\cM,\hat\nabla)
        \overset{\sim}\longrightarrow
        (\hat\cM^{model},\hat\nabla^{model})
  \right\}
\]
in which we also keep the additional information, of the formal isomorphism, by
which the meromorphic connection is formally isomorphic to the model.
Where the isomorphisms of marked pairs are defined as follows:
\begin{defn}
  Two germs $(\cM,\nabla,\hat f)$ and $(\cM',\nabla',\hat f')$ are
  isomorphic if there exists an isomorphism
  $g:(\cM,\nabla)\overset{\sim}\longrightarrow(\cM',\nabla')$ such that
  $\hat f=\hat f'\circ \hat g$.
  \begin{comment}
    \cite[111]{sabbah2007isomonodromic}:\dots it is important to remark that
    such an isomorphism is then unique.
  \end{comment}
\end{defn}

Equivalently, one can talk in terms of systems. We then denote by
\[
  \Syst_m(A^0):=\{d-A
    \mid A={}^{\hat F}\!A^0 \text{ for some } \hat F\in G(\!(t)\!)\}
\]
the set of systems formally meromorphic equivalent to $A^0$.
Since we use meromorphic equivalences, in contrast to \cite{boalch,thboalch},
we denote that in $\Syst_m$ by the subscript ${}_m$.
Thus ${}^0C(\cM^{model},\nabla^{model})$ corresponds to
the set ${}^0C(A^0):=\Syst_m(A^0)/G(\!\{t\}\!)$ of meromorphic classes which
are formally equivalent to $A^0$.
\TODO[\cite{thboalch} p. 3: In the logarithmic case\dots]
Analogous, $\cH(\cM^{model},\nabla^{model})$ corresponds to the set $\cH(A^0)$ of
equivalence classes in
\[
  \hat\Syst_m(A^0):=\{(A,\hat F)
    \mid A={}^{\hat F}\!A^0 \text{ for some } \hat F\in G(\!(t)\!)\} \,.
\]

\begin{lem}
  Since $G_0(A^0)$ is defined as the stabilizer\TODO[correct?] of $A^0$ we deduce
  \[
    \Syst_m(A^0)\cong \hat G(A^0)/G_0(A^0) \,.
  \]
  \begin{cor}
    Thus the \emph{set of meromorphic classas of systems formally equivalent
      to $A^0$} are just the orbits of $G\{t\}$, that is
    \[
      {}^0C(A^0)\cong G\{t\}\backslash\hat G(A^0)/G_0(A^0)
    \]
    whereas the \emph{set of meromorphic classes of transformations of $[A^0]$}
    is represented by the left quotient $G\backslash\hat G(A^0)$.
  \end{cor}
\end{lem}
\begin{proof}
  See
  \begin{itemize}
    \item~\cite[6]{thboalch}: in the case $G_0(A^0)=T$
  \end{itemize}
\end{proof}

The group $G_0(A^0)$ is easy to compute and is often trivial. In fact, the
elements are block-diagonal, see~\cite[77]{Loday2014}.
Thus the structure of ${}^0C(A^0)$ is easily deduced from the structure of
$G\backslash\hat G(A^0)$.

\begin{lem}
  The set $G\backslash\hat G(A^0)$ is canonically isomorphic to $\cH(A^0)$.
\end{lem}
\begin{proof}
  See~\cite{thboalch}: Lemma 1.17
\end{proof}

