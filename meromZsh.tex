\chapter{Meromorphe Zusammenhänge}
\begin{comment}
  Siehe:
  \begin{itemize}
    \item \cite{boalch} and \cite{thboalch}
    \item \cite{sabbah2007isomonodromic}
  \end{itemize}
\end{comment}

\section{From \cite{thpym} and \cite{citeulike:12387686}}
Consider a Riemann surface $X$ and an effective divisor $D$.
\begin{defn}
  If $E$ is a holomorphic vector bundle over $X$, a \emph{meromorphic connection
  on $E$ with poles bounded by $D$} is a differential operator
  \[
    \nabla:\cE\to\Omega_X^1(D)\otimes\cE
  \]
  on $\cE$
  \begin{itemize}
    \item the sheaf of holomorphic sections on $E$
  \end{itemize}
  that satisfies the Leibnitz rule
  \[
    \nabla(fs)=f\nabla s + (df)\otimes s
  \]
  \begin{rem}
    Thus, in a local trivialization of $E$ near a point $p\in D$ of
    multiplicity $k$ and a coordinate $z$ centered at $p$, we may write
    \[
      \nabla=d-\frac{1}{z^k}A(z)dz
    \]
    for a holomorphic matrix $A(z)$.
  \end{rem}
\end{defn}

\section{Matrix version from \cite{boalch} and \cite{thboalch}}
%{{{
Let
\begin{itemize}
  \item $D=k_1(a_1)+\dots+k_m(a_m)$ be an \textbf{effective divisor} on $\P^1$
    so that
    \begin{itemize}
      \item $a_1,\dots,a_m\in\P^1$ are distinct points,
      \item $k_1,\dots,k_m>0$ positive integers
    \end{itemize}
    and
  \item $V\to\P^1$ be a rank $n$ vector bundle.
\end{itemize}
\begin{defn}[2.1]
  A \emph{meromorphic connection} $\nabla$ on $V$ with poles on $D$ is
  \begin{itemize}
    \item a map $\nabla:V\to V\times K(D)$ where
      \begin{itemize}
        \item from the sheaf of holomorphic sections of $V$
        \item to the sheaf of sections of $V\otimes K(D)$\footnote{for more
          information about $K(D)$ see \cite{beauville1996complex}}
      \end{itemize}
    \item satisfying the Leibnitz rule
      \begin{equation}
        \nabla(fv)=(df)\otimes v + f\nabla v
      \end{equation}
      where
      \begin{itemize}
        \item $v$ is a local section of $V$,
        \item $f$ is a local holomorphic function and
        \item $K$ is the sheaf or holomorphic one-forms on $\P^1$
      \end{itemize}
  \end{itemize}
\end{defn}

\begin{paracol}{2}\sloppy
\switchcolumn[0]\noindent
\begin{rem}
  If we choose a local coordinate $z$ on $\P^1$ vanishing at $a_i$ then
  \begin{itemize}
    \item in terms of local trivialization of $V$,
  \end{itemize}
  $\nabla$ has the form
  \[
    \nabla=d-A=d-{}^iA\footnote{the presuperscript is used to signify local
    information}
  \]
  where
  \begin{itemize}
    \item $A=\left(\sum^{0}_{j=k_i}A_j\frac{dz}{z^{j}}\right)+A_0dz+\dots$
      \begin{itemize}
        \item is a matrix of meromorphic one-forms and
        \item $A_j\in\End(\C^n)$.
      \end{itemize}
  \end{itemize}
\end{rem}
\switchcolumn[1]\noindent
  Let $E=\C^n$ be
  \begin{itemize}
    \item a fixed complex vector space
      \begin{itemize}
        \item with preferred basis.
      \end{itemize}
  \end{itemize}
  \begin{defn}
    \cite[Def 1.5]{thboalch}.
    A \emph{germ of a meromorphic linear differential system}\footnote{or just
    \emph{system} from now on} (of rank $n$) is
    \begin{itemize}
      \item a germ of a meromorphic connection
        \begin{itemize}
          \item on the trivial vector bundle with fibre $E$.
        \end{itemize}
    \end{itemize}
  \end{defn}
  \begin{rem}
    \begin{itemize}
      \item The set of systems is isomorphic to $\End(E)\otimes\C\{z\}[z^{-1}]$
      \item A matrix of germs of meromorphic functions
        $A'\in\End(E)\otimes\C\{z\}[z^{-1}]$ determines
        \[
          \frac{dv}{dz}=A'v
        \]
        \begin{itemize}
          \item a system of equations for $v(z)\in E$
          \item which corresponds to the connection germ $\nabla=d_A=d-A$
            \begin{itemize}
              \item on $E$
              \item where $A=A'dz$.
            \end{itemize}
        \end{itemize}
    \end{itemize}
  \end{rem}
\end{paracol}
\begin{paracol}{2}\sloppy
\switchcolumn[0]\noindent
\begin{defn}[\cite{boalch} Def 2.2]
  A meromorphic connection is a \emph{nice (or generic) meromorphic connection}
  if
  \begin{itemize}
    \item at each $a_i$
  \end{itemize}
  the leading coefficient ${}^iA_{k_i}$ is
  \begin{itemize}
    \item diagonalisable with distinct eigenvalues and $k_i\geq2$, or
    \item diagonalisable with distinct eigenvalues $\mod\Z$ and $k_i=1$
  \end{itemize}
  \begin{rem}
    \begin{itemize}
      \item This condition is independent of the trivialization and coordinate
        choice.
      \item We will restrict to nice connections since they are simplest yet
        sufficient\footnote{to describe the symplectic nature of isomonodromic
        deformations} for our purposes.
    \end{itemize}
  \end{rem}
\end{defn}
\switchcolumn[1]\noindent
  \begin{defn}
    A \emph{nice formal normal form} $A^0$ is
    \begin{itemize}
      \item a nice diagonal system with no holomorphic part.
    \end{itemize}
    \TODO[Entspricht dies der Levelt-Turittin-Zerlegung??]
    \begin{rem}
      Such $A^0$ can be uniquely written as
      \[
        A^0:= dQ + \Lambda \frac{dz}{z},
      \]
      where
      \begin{itemize}
        \item $Q := \diag(q_1,\dots ,q_n)$
          \begin{itemize}
            \item $q_1,\dots ,q_n\in z^{-1}C[z^{-1}]$ are polynomials
              \begin{itemize}
                \item of degree $k-1$ in $z^{-1}$
                \item with no constant term
              \end{itemize}
          \end{itemize}
          and
        \item $\Lambda = Res_0(A_0)$ is a constant diagonal matrix.
      \end{itemize}
    \end{rem}
  \end{defn}
\end{paracol}
%}}}

\subsubsection{Local moduli spaces} %{{{
\begin{defn}
  \begin{itemize}
    \item $\Syst(A^0):=\{d-A \mid A=\hat{F}[A^0] \text{ for some } \hat{F}\in
      G\llbracket z\rrbracket\}$\footnote{the set of germs at $0\in\C$ of
      meromorphic connections on the trivial rank $n$ vector bundle, that are
      formally equivalent to $d-A^0$.}
      where
      \begin{itemize}
        \item $A$ is a matrix of germs of meromorphic one-forms
        \item $\hat{F}[A^0]=(d\hat{F})\hat{F}^{-1}+\hat{F}A^0\hat{F}^{-1}$
        \item $G\llbracket z\rrbracket:=\GL_n(\C\llbracket z\rrbracket)$
          \begin{itemize}
            \item group of \emph{formal transformations}
            \item does not act on $\Syst(A^0)$
          \end{itemize}
      \end{itemize}
  \end{itemize}
  The group $G\{z\}:=\GL_n(\C\{z\})$
  \begin{itemize}
    \item the group of \emph{local analytic gauge transformations}
  \end{itemize}
  acts on $\Syst(A^0)$.
  \begin{center}
    \textbf{We are interested in $\bf\Syst(A^0)/G\{z\}$}\footnote{The set of
    isomorphism classes of germs of meromorphic connections formally equivalent
    to $A^0$. Note that any generic connection is formally equivalent to some
    such $A^0$.}
  \end{center}
\end{defn}
\begin{rem}
  In the abelian and the simple pole case $\Syst(A^0)/G\{z\}$ is only a point.
\end{rem}
\subsubsection{Add a little bit more information}
\begin{defn}
  A \emph{compatible framing} at $a_i$ of a vector bundle $V$ with
  generic\footnote{The precise notion of ‘generic’ is given in
    \cite{thboalch} Definition 1.2 and we will refer to such connections as
    \emph{nice}.} connection $\nabla$ is
  \begin{itemize}
    \item an isomorphism $g_0:V_{a_i}\to\C^n$
      \begin{itemize}
        \item between the firbre $V_{a_i}$ and $\C^n$
      \end{itemize}
      such that
      \begin{itemize}
        \item the leading coefficient of $\nabla$ is diagonal in any local
          trivialization of $V$ extending $g_0$.
      \end{itemize}
  \end{itemize}
  \begin{rem}
    Given a trivialization of $V$ in a neighbourhood of $a_i$ so that
    $\nabla=d-A$ as above, then
    \begin{itemize}
      \item a compatible framing is represented by a constant matrix
        $g_0\in G$ such that $g_0A_{k_i}g_0^{-1}$ is diagonal
    \end{itemize}
  \end{rem}
\end{defn}
\begin{paracol}{2}\sloppy
\switchcolumn[0]\noindent
  \begin{defn}[2.4]
    A connection $(V,\nabla)$ with compatible framing $g_0$ at $a_i$ has
    \emph{irregular type ${}^iA^0$} if
    \begin{itemize}
      \item $g_0$ extends to a formal trivialization of $V$ at $a_i$, in which
        $\nabla$ differs from $d-{}^iA^0$ by a matrix of one-forms with just
        simple poles.
    \end{itemize}
  \end{defn}
\switchcolumn[1]\noindent
  \begin{defn}
    The set of \emph{applicable formal transformations} is
    \[
      \hat G(A^0):=\left\{\hat F\in\hat G
        \mid\hat F[A^0] \text{ is convergent}\right\}\,.
    \]
  \end{defn}

  \begin{defn}[marked pair]
    A \emph{marked pair} is a pair $(A, \hat F)$ consisting of
    \begin{itemize}
      \item a nice system $A$ and
      \item a choice of formal isomorphism $\hat F \in \hat G$ such that
        $A = \hat F[A^0]$
    \end{itemize}
  \end{defn}
\end{paracol}
\begin{defn}
  Define
  \begin{itemize}
    \item $\widehat\Syst_{cf}(A^0)$~:\Leftrightarrow{} the set of compatibly
      framed connection germs with both irregular and formal type $A^0$.
    \item $\widehat\Syst_{mp}(A^0):=\{(A,\hat{F})\mid A\in\Syst(A^0)
      ,\hat{F}\in G\llbracket z\rrbracket
      ,A=\hat{F}[A^0]\}$
      the set of \emph{marked pairs}.
      \begin{itemize}
        \item $G\{z\}$ action on marked pairs:
          $g(A,\hat{F})=(g[A],g\circ\hat{F})$
      \end{itemize}
  \end{itemize}

  \begin{lem}
    There is a canonical isomorphism
    \[
      \widehat\Syst_{cf}(A^0)\cong\widehat\Syst_{mp}(A^0) \,,
    \]
  \end{lem}
  Let $\widehat\Syst(A^0)$ denote either of these two sets.
\end{defn}
\begin{paracol}{2}\sloppy %%%%%%%%%%%%%%%%%%%%%%%%%%%%%%%%%%%%%%%%%%%%%%%%%%%%%
  \begin{defn}
    \[
      \mathcal{H}(A^0):=\widehat\Syst(A^0)/G\{z\}
    \]
  \end{defn}
\switchcolumn %%%%%%%%%%%%%%%%%%%%%%%%%%%%%%%%%%%%%%%%%%%%%%%%%%%%%%%%%%%%%%%%%
  \begin{defn} Definition from \cite{thboalch}:
    \[
      \mathcal{H}(A^0):=G\{z\}\backslash\hat G(A^0)
    \]
  \end{defn}
  \begin{lem}
    The set is canonically isomorphic to the set of isomorphism classes of
    compatibly framed systems having associated formal normal form $A^0$.
  \end{lem}
\end{paracol} %%%%%%%%%%%%%%%%%%%%%%%%%%%%%%%%%%%%%%%%%%%%%%%%%%%%%%%%%%%%%%%%%
\begin{lem}
The actions of $T$ and $G\{z\}$ on $\Syst(A^0)$ commute:
\[
  \Syst(A^0)/G\{z\}\cong\mathcal{H}(A^0)/T \,.
\]
\end{lem}
%}}}

\subsubsection{Moduli spaces} %{{{
Let $\textbf{a}$\footnote{In \cite{thboalch} denoted by $\textbf{A}$} denote
the choice of
\begin{itemize}
  \item an effective divisor $D=k_1(a_1)+\dots+k_m(a_m)$ and
  \item at each point $a_i$
    \begin{itemize}
      \item a germ $d-{}^iA^0$
        \begin{itemize}
          \item of a \textbf{diagonal generic meromorphic connection}
            \begin{itemize}
              \item on the trivial rank $n$ vector bundle
            \end{itemize}
          \item write ${}^iA^0=d({}^iQ)+{}^i\Lambda^0\frac{dz}{z}$
        \end{itemize}
    \end{itemize}
\end{itemize}
\begin{defn}[2.5]
  The \emph{moduli space $\cM^*(\textbf{a})$
  \textcolor{green!40!black}{($\cM(\textbf{a})$)}} is
  \begin{itemize}
    \item the set of isomorphism classes of pairs $(V,\nabla)$ where
      \begin{itemize}
        \item a trivial \textcolor{green!40!black}{(degree zero)} rank $n$
          holomorphic vector bundle $V$ over $\P^1$
        \item a meromorphic connection $\nabla$ (with poles on $D$) on $V$
          which is formally equivalent to $d-{}^iA^0$ at $a_i$ for each
          $i$\footnote{and has no other poles}
      \end{itemize}
  \end{itemize}
\end{defn}
\begin{defn}[2.6]
  The \emph{extended moduli space $\widetilde\cM^*(\textbf{a})$
  \textcolor{green!40!black}{($\widetilde\cM(\textbf{a})$)}} is
  \begin{itemize}
    \item the set of isomorphism classes of triples $(V,\nabla,\textbf{g})$
      where
      \begin{itemize}
        \item a trivial \textcolor{green!40!black}{(degree zero)}
          \textcolor{gray}{rank $n$} holomorphic vector bundle $V$ over $\P^1$
        \item a generic \textcolor{gray}{meromorphic} connection $\nabla$
          (with poles on $D$) on $V$
        \item compatible framins $\textbf{g}=({}^1g_0,\dots,{}^mg_0)$
      \end{itemize}
      such that $(V,\nabla,\textbf{g})$ has irregular type ${}^iA^0$ at each
      $a_i$
  \end{itemize}
\end{defn}
Since $\cM^*(\textbf{a})$ and $\widetilde\cM^*(\textbf{a})$ are moduli spaces
of connections on trivial bundles we can obtain explicit descriptions of them.
See \cite{thboalch} page 10ff.
%}}}

\begin{comment}
  \section{Definition from \cite{sabbah_cimpa90}}
  \begin{defn}
    A \emph{meromorphic connection} $\cM_K$ is a 
    \begin{itemize}
      \item $K$-vector space of finite dimension
      \item equipped with
        \begin{itemize}
          \item a $\C$-linear derivation $\partial_x: \cM_K\to\cM_K$
        \end{itemize}
        such that, for all $f\in K$ and all $m\in\cM_K$ one has
        \[
          \partial_x(fm)=\frac{\partial f}{\partial m}+f\partial_xm.
        \]
    \end{itemize}
  \end{defn}

  \begin{thm}
    Let $\cM_{\hat K}$ be a formal meromorphic connection. There exists an
    integer $q$ such that the connection $\pi^*\cM_{\hat K}=\cM_{\hat L}$ is
    isomorphic to a direct sum of elementary formal meromorphic connections.
  \end{thm}
\end{comment}

\section{Sheaf version from \cite{sabbah2007isomonodromic}} %{{{
\begin{defn}[Holomorphic budles (0.3.?)]
  Let
  \begin{itemize}
    \item $\pi:E\to M$ be a holomorphic mapping between two complex analytic
      manifolds.
  \end{itemize}
  We will say that
  \begin{itemize}
    \item $\pi$ is a \emph{vector fibration of rank $d$}, or
    \item $\pi$ makes $E$ a \emph{vector bundle of rank $d$ on $M$}
  \end{itemize}
  if there exists a open covering\dots
\end{defn}
Set $\sE(U)$ as the set of holomorphic sections, where
\begin{itemize}
  \item a \emph{holomorphic section} of $E$ on $U$ is a holomorphic mapping
    $\sigma:U\to E$ which
    \begin{itemize}
      \item is a section of the projection, i.e., which satisfies
        $\pi\circ\sigma=\Id_U$.
    \end{itemize}
\end{itemize}
This defines a sheaf $\sE$ of modules over $\cO_M(U)$.
\begin{defn}[Meromorphic bundles (0.8.?)]
  A \emph{meromorphic bundle on $M$ with poles along $Z$} is a locally free
  sheaf of $\cO_M(*Z)$-modules of finite rank
  where
  \begin{itemize}
    \item $Z$ is a smooth hypersurface in a complex analytic manifold $M$
    \item define the sheaf $\cO_M(*Z)$ by
      \begin{itemize}
        \item $U\mapsto$ functions which are \emph{meromorphic along $Z\cap U$}
          i.e.\ which
          \begin{itemize}
            \item are holomorphic on $U\backslash Z\cap U$ and
            \item for any chart $V$ of $M$
              \begin{itemize}
                \item contained in $U$ and
                \item in which $Z\cap V$ is defined by the vanishing of some
                  coordinate $z_1$
              \end{itemize}
              there exists
              \begin{itemize}
                \item an integer $m$
              \end{itemize}
              such that
              \begin{itemize}
                \item $z_1^mf(z_1,\dots,z_n)$ is locally bounded in the
                  neighbourhood of any point of $Z\cap V$.
              \end{itemize}
          \end{itemize}
      \end{itemize}
  \end{itemize}
\end{defn}
\begin{defn}[Holomorphic / meromorphic connections]
  \begin{itemize}
    \item A \emph{holomorphic connection} $\nabla$ on a holomorphic vector
      bundle $\pi:E\to M$ is a $\C$-linear homomorphism of sheaves
      \[
        \nabla:\sE\to\Omega_M^1\otimes_{\cO_M}\sE
      \]
      satisfying the Leibnitz rule.
    \item A \emph{connection} on a meromorphic bundle $\sM$ is defined as a
      $\C$-linear homomorphism
      \[
        \nabla:\sM\to\Omega_M^1\otimes_{\cO_M}\sM
      \]
      satisfying
      \begin{itemize}
        \item for
          \begin{itemize}
            \item any open set $U$ of $M$,
            \item any section $s\in\Gamma(U,\sE)$ and
            \item any holomorphic function $f\in\cO(U)$
          \end{itemize}
      \end{itemize}
      the \emph{Leibnitz rule}:
      \[
        \nabla(f\cdot s)=\nabla(s)+df\otimes
        s\in\Gamma(U,\Omega_M^1\otimes_{\cO_M}\sE)
      \]
  \end{itemize}
  \begin{comment}
    What is the difference between
    \begin{itemize}
      \item \textbf{meromorphic} bundle with \textbf{holomorphic} connection,
      \item \textbf{holomorphic} bundle with \textbf{meromorphic} connection
        and
      \item \textbf{meromorphic} bundle with \textbf{meromorphic} connection.
    \end{itemize}
  \end{comment}
\end{defn}
\begin{defn}[Flatness (0.12.2)]
  The connection $\nabla:\cE\to \Omega_M^1\otimes_{\cO_M}\cE$ is said to be
  \emph{integrable} or \emph{flat}, if
  \begin{itemize}
    \item its curvature $R_\nabla\equiv0$

    where
    \begin{itemize}
      \item $R_\nabla:=\nabla\circ\nabla:\cE\to\Omega_M^2\otimes_{\cO_M}\cE$
        is a $\cO_M$-linear morphism.
    \end{itemize}
  \end{itemize}
  \begin{prop}[0.12.4]
    The connection $\nabla$ is flat if and only if, in any local basis $e$ of
    $\cE$, the connection matrix $\Omega$ satisfies
    \[
      d\omega + \omega \wedge \omega = 0.
    \]
  \end{prop}
  We will say that a connection on a meromorphic bundle is \emph{integrable} or
  \emph{flat} if its restriction to $M\backslash Z$ is an integrable connection
  on the holomorphic bundle $\sM_{|M\backslash Z}$.
\end{defn}

\subsection{Models and formal decomposition of a germ}
Let $\textbf{k}$ denote $\C\{t\}[1/t]$.
\subsubsection{Regular singularities}
\begin{defn}
  An \emph{elementary regular model} is
  \begin{itemize}
    \item a $(\textbf{k},\nabla)$-vector space
      equipped with a basis in which
      \begin{itemize}
        \item the connection matrix is written as
          $\Omega(t)=(\alpha\Id+N)\frac{dt}{t}$
          where
          \begin{itemize}
            \item $\alpha\in\C$ and
            \item $N$ is a nilpotent matrix
          \end{itemize}
      \end{itemize}
  \end{itemize}
\end{defn}
\begin{defn}
  The horizontal sections of an elementary regular model have \emph{moderate
  growth near the origin} if and only if for
  \begin{itemize}
    \item any horizontal section $s$ in the neighbourhood of a closed angular
      sector with angel $<2\pi$,
  \end{itemize}
  there exist
  \begin{itemize}
    \item an integer $n\geq0$ and
    \item a constant $C>0$
  \end{itemize}
  such that, on this sector,
  \[
    \left\Vert s(t)\right\Vert\leq C|t|^{-n} \,.
  \]
\end{defn}
\begin{cor}[II.2.9]
  Any $(\textbf{k},\nabla)$-vector space with regular singularity is isomorphic
  to a direct sum of elementary regular models.
\end{cor}
Let
\begin{itemize}
  \item $x^0\in X$ and
  \item $(x_1,\dots,x_n)$ be a system of coordinates centered at $x^0$.
  \item $\Omega=A(t,x)\frac{dt}{t}+\sum_{i=1}^nC^{i}(t,x)dx_i$ be the
    connection matrix
    \begin{itemize}
      \item in some basis $\textbf{e}$ of $\sM$
      \item in the neighbourhood of $(0,x^0)$.
    \end{itemize}
\end{itemize}
If the matrices $A$ and $C^{(i)}$ have holomorphic entries, $(\sM,\nabla)$ has
regular singularity in the neighbourhood of $(0,x^0)$. For more criteria see
\cite[II.4.a]{sabbah2007isomonodromic}: \textbf{How to recognize a regular
  singularity}.
\begin{prop}[Any formal solution is convergent(II.2.18)]
  Let $A(t)$ be a matrix in $M_d(\C\{t\})$. Any vector $u(t)$ with entries in
  $\C\llbracket t\rrbracket$ which is solution of the system
  $tu'(t)+A(t)u(t)=0$ has converging entries.
\end{prop}
\begin{defn}[II.2.24]
  A meromorphic bundle $\sM$ on $D\times X$, with poles along $\{0\}\times X$,
  equipped with a flat connection $\nabla$, has \emph{regular singularity} if
  \begin{itemize}
    \item in the neighbourhood of any point $(0,x^0)$ of $\{0\}\times X$
      \begin{itemize}
        \item there exists a logarithmic lattice of $(\sM,\nabla)$.
      \end{itemize}
  \end{itemize}
\end{defn}
\begin{thm}[Normal form of regular singularities with parameter (II.2.25)]
  There exists a basis of the germ $\sM_{(0,x^0)}$ in which the matrix of
  $\nabla$ can be written as $B\frac{dt}{t}$, where $B$ is constant.
\end{thm}
\subsubsection{Irregular singularities in rank one}
Let
\begin{itemize}
  \item $\phi\in\C\{t,x_1,\dots,x_n\}[t^{-1}]$.
\end{itemize}
Denote by $\cE^{\phi}$
\begin{itemize}
  \item the germ $(\cM,\nabla):=(\C\{t,x_1,\dots,x_n\}[t^{-1}],d-d\phi)$
    \begin{itemize}
      \item this is the system satisfied by the function $e^\phi$
    \end{itemize}
\end{itemize}
For $\alpha\in\C$, let $\cN_{\alpha,0}$ be
\begin{itemize}
  \item the germ
    $(\cM,\nabla):=(\C\{t,x_1,\dots,x_n\}[t^{-1}],d-\alpha dt/t)$.
\end{itemize}
\begin{prop}[Classification of irregular singularities in rank one (II.5.1)]
  Any germ $(\cM,\nabla)$ of rank one is isomorphic to some germ
  $\cE^{\phi}\otimes\cN_{\alpha,0}$.

  Tow such germs are corresponding to $(\phi_1,\alpha_1)$ and
  $(\phi_2,\alpha_2)$ are isomorphic iff
  \begin{itemize}
    \item $\phi_1-\phi_2$ has no pole and
      \begin{itemize}
        \item Therefore, the class of $\phi$ in
          $\C\{t,x_1,\dots,x_n\}[t^{-1}]/\C\{t,x_1,\dots,x_n\}$ determines the
          germ $\cE^\phi$.
      \end{itemize}
    \item $\alpha_1-\alpha_2\in\Z$.
  \end{itemize}
\end{prop}
\subsubsection{Irregular singularities in arbitrary rank}
\begin{defn}
  Let $(\cM,\nabla)$ be a \textbf{germ} of a meromorphic bundle with
  connection.
  \begin{itemize}
    \item A germ $(\cM,\nabla)$ is \emph{elementary} if it is isomorphic to
      some germ
      \[
        (\cE^\phi,\nabla)\otimes(\cR,\nabla)
      \]
      where
      \begin{itemize}
        \item $(\cR,\nabla)$ ha regular singularity along $\{0\}\times X$
      \end{itemize}
    \item $(\cM,\nabla)$ is a \emph{model} if it is isomorphic to a direct sum
      of elementary models, written as
      \[
        \bigoplus_\phi(\cE^\phi\otimes\cR_\phi)
      \]
      where
      \begin{itemize}
        \item the meromorphic bundles with connection $\cR_\phi$ have regular
          singularity and
        \item the
          $\phi\in\C\{t\textcolor{gray}{\underset{\text{parameter}}
            {\underbrace{,x_1,\ldots,x_n}}}\}[t^{-1}]$
          \begin{itemize}
            \item have no holomorphic part and
            \item are pairwise distinct.
          \end{itemize}
      \end{itemize}
    \item We will say that a model is \emph{good} if,
      \begin{itemize}
        \item for all $\phi\neq\psi$
          \begin{itemize}
            \item such that $\cR_\phi$, $\cR_\psi$ are nonzero,
          \end{itemize}
          the order of the pole along $t=0$ of $(\phi-\psi)(t,x)$ does not
          depend on $x$ being in some neighbourhood of $x^o$.
      \end{itemize}
  \end{itemize}
\end{defn}
\begin{thm}[Formal decomposition (II.5.7)]
  Let $(\cM,\nabla)$ be a germ of meromorphic bundle with connection,
  \begin{itemize}
    \item equipped with a basis
      \begin{itemize}
        \item
          \marginnote{This condition \textbf{might be} equivalent to the
            condition of being \textbf{nice} in \cite{thboalch}.}
          in which the matrix $\Omega$ takes the form
          \[
            \Omega=t^{-r}\left[A(t,x)\frac{dt}{t}
              +\sum_{i=1}^nC^{(i)}(t,x)dx_i\right]
          \]
        with
        \begin{itemize}
          \item $r\geq1$
          \item $A$ and the $C^{(i)}$ having holomorphic entries, and
          \item $A_0:=A(0,x^0)$ being regular semisimple\footnote{i.e.\ with
            pairwise distinct eigenvalues.}
        \end{itemize}
      \end{itemize}
      \TODO[is this the condition, such that no ramification is necessary?]
  \end{itemize}
  Then there exist
  \begin{itemize}
    \item a good model $(\cM^{good},\nabla)
      =\left(\bigoplus_\phi(\cE^\phi\otimes\cR_\phi) \right)$ and
    \item a \textcolor{red!40!black}{`formal'} isomorphism
      \[
        \textcolor{red!40!black}{\hat\cO_{D\times X,x_0}\otimes}(\cM,\nabla)
        \overset{\sim}{\longrightarrow}
      \textcolor{red!40!black}{\hat\cO_{D\times X,x_0}\otimes}
        \left(\bigoplus_\phi(\cE^\phi\otimes\cR_\phi) \right)
      \]
  \end{itemize}
\end{thm}
\begin{comment}
  \begin{rem}
    \begin{itemize}
      \item In the present situation, no ramification is needed.
        \TODO[Why?]
      \item Moreover, we will see that all the components $\cR^\phi$ occurring
        in the model have rank one, which is not the case in general, even when
        no ramification is needed.
    \end{itemize}
  \end{rem}
\end{comment}
%}}}

% vim:set ft=tex foldmethod=marker foldmarker=%{{{,%}}}:
