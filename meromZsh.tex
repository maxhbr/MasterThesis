\chapter{Meromorphe Zusammenhänge}
\begin{comment}
  Siehe:
  \begin{itemize}
    \item \cite{boalch} and \cite{thboalch}
    \item \cite{sabbah2007isomonodromic}
  \end{itemize}
  and
  \begin{itemize}
    \item \cite{Varadarajan96linearmeromorphic}
  \end{itemize}
\end{comment}

Let $M$ be a riemanian surface and let $Z=k_1(a_1)+\cdots+k_m(a_m)>0$ be an
effective divisor\footnote{The $a_i$ are distinct points and the $k_i$ are
positive integers.} on $M$.
It is sufficient to think $M=\P^1$.

Let $\sM$ be a holomorphic Bundle i.e.\ a locally free $\cO_m$-module of rank
$n$.

A meromorphic connection is then defined as follows.
\begin{defn}
  \def\myU{\textcolor{green!30!black}{U}}
  \def\mys{\textcolor{blue!60!black}{s}}
  \def\myf{\textcolor{red!60!black}{f}}
  A \emph{meromorphic connection $(\sM,\nabla)$ on $\sM$ with poles on $Z$}
  is a $\C$-linear morphism of sheaves
  \[
    \nabla:\sM\to\Omega_M^1(*Z)\otimes\sM
  \]
  satisfying, for each $\myU\overset{\text{op.}}{\subset} M$, the
  \emph{Leibniz rule}
  \begin{multicols}{2}
    \[
      \nabla(\tikzmark{e2}\myf\overset{\tikzmark{e1}}\mys)
        =\myf\nabla\mys+(d\myf)\otimes\mys \,.
    \]
    \columnbreak
    \begin{itemize}
      \item[\tikzmarkc{n1}{blue}]
        $\mys\textcolor{blue!60!black}{\in\Gamma(\myU,\sM)}$
      \item[\tikzmarkc{n2}{red}]
        $\myf\textcolor{red!60!black}{\in\cO_M(\myU)}$
    \end{itemize}
  \end{multicols}
  \begin{tikzpicture}[remember picture,overlay]
    \draw[->,blue!50!white,thick] (n1) to[out=180,in=70] (e1);
    \draw[->,red!50!white,thick] (n2) to[out=180,in=-70] (e2);
  \end{tikzpicture}
\end{defn}
\begin{rem}
  \begin{enumerate}
    \item We will occasionally omit the $\nabla$ and simply call $\sM$ the
      meromorphic connection.
    \item Here, the variant `holomorphic bundle with meromorphic connection' is
      chosen, like in \cite{boalch}.  There is also the twisted description
      `meromorphic bundle with holomorphic connection' which is used in
      \cite{sabbah2007isomonodromic}.
      \\ By choosing a lattice of a meromophic bundle, one gets a holomorphic
      bundle but the connection is no longer holomorphic. Such that we obtain a
      meromorphic connection in our\dots\TODO
  \end{enumerate}
\end{rem}

\begin{defn}[Flatness (0.12.2)]
  The connection $\nabla:\cM\to \Omega_M^1\otimes_{\cO_M}\cM$ is said to be
  \emph{integrable} or \emph{flat}, if
  \begin{itemize}
    \item its curvature $R_\nabla\equiv0$
    where
    \begin{itemize}
      \item $R_\nabla:=\nabla\circ\nabla:\cE\to\Omega_M^2\otimes_{\cO_M}\cE$
        is a $\cO_M$-linear morphism.
    \end{itemize}
  \end{itemize}
  \begin{prop}[0.12.4]
    The connection $\nabla$ is flat if and only if, in any local basis $e$ of
    $\cM$, the connection matrix $\Omega$ satisfies
    \[
      d\omega + \omega \wedge \omega = 0.
    \]
  \end{prop}
  We will say that a connection on a meromorphic bundle is \emph{integrable} or
  \emph{flat} if its restriction to $M\backslash Z$ is an integrable connection
  on the holomorphic bundle $\sM_{|M\backslash Z}$.
\end{defn}

\section{Germ of a meromophic Connection / local expression / Connection
  matrices / systems}
\begin{comment}
  \cite{sabbah2007isomonodromic} p.28
\end{comment}
\begin{comment}
  \cite{boalch} wants \textbf{generic} meromorphic connections
  \begin{itemize}
    \item\dots simplest jet sufficient\dots
  \end{itemize}
\end{comment}

Let $U$ be a trivializing open set for $M$ and let $\textbf{e}=(e_1,\dots,e_n)$
be a basis of $\Gamma(U,\sM)$.
There exists then a $n\times n$ matrix $A$ of meromophic 1-forms such
that\dots
\begin{defn}
  This matrix is called \emph{connection matrix} for $(\sM,\nabla)$.
\end{defn}

\begin{comment}
  \TODO[Gauge transformation]
  \begin{itemize}
    \item meromorphic transformation \cite{Loday1994} p. 852
    \item holomorphic trasformation
  \end{itemize}
\end{comment}

\begin{cor}
  The connection matrices describe the germs of meromophic connections.
\end{cor}

\subsection{Connection matrix \leftrightarrow{} system}
\subsection{Connection matrix \rightarrow{} connection}
\begin{comment}
  There is a thm in \cite{sabbah2007isomonodromic}
\end{comment}

\subsection{Meromorphic / formal transformation}
\begin{comment}\footnotesize
  see \cite{thboalch} \textbf{Rem 1.41 on p. 16}:
  \begin{rem}
    Note that in most of the recent references we have used, Stokes matrices
    are used to classify
    \begin{itemize}
      \item meromorphic connections within fixed formal \textbf{meromorphic
        classes, modulo meromorphic equivalence}.
    \end{itemize}
    Whereas here we classify
    \begin{itemize}
      \item meromorphic connections within fixed \textbf{formal analytic
        classes, modulo analytic equivalence},
    \end{itemize}
    as is done in the older literature.  The fact is that the sets equivalence
    classes are the same in both cases. It is important for us to work with
    analytic, rather than meromorphic gauge transformations, because then the
    $\C^\infty$ viewpoint in Chapter 3 is cleaner. This distinction relates to
    the difference between \textbf{‘regular singular’} connections and
    \textbf{‘logarithmic’} connections.
  \end{rem}
\end{comment}
By \emph{transformation} (or \emph{meromophic transformation}) of a connection
matrix $A$ we mean a linear change of the trivialzation. Such a change is given
by a matrix $F\in\Gl_n(\C\{t\}[1/t])$ and the transformed matrix $[{}^F\!A]$ is
obtained through
\[
  {}^F\!A=(dF)F^{-1} + FAF^{-1} \,.
\]
If $F$ is formal i.e. $F\in\Gl_n(\C\llbracket t\rrbracket[1/t])$, it will
usually be denoted by $\hat F$. The transformation of $A$ by $\hat F$ is not
guaranteed to have konvergent entries. We denote by $\hat G(A)$ the set of all
formal transformations
\[
  \hat G(A):=\left\{\hat F\in\Gl_n(\C\llbracket t\rrbracket[1/t])
    \mid {}^{\hat F}\!A \text{ has konvergent entries}
  \right\}\,.
\]
Denote the set of all systems, which are formally equivalent to $d-A^0$ by
\[
  \Syst(A^0):=\{d-A \mid A={}^{\hat F}A^0 \text{ for some }\hat G(A^0)\}
\]
\begin{rem}
  \def\myB{\textcolor{blue!60!black}{B}}
  \def\myA{\textcolor{green!30!black}{A}}
  \def\myF{\textcolor{red!60!black}{F}}
  The condition 
  \begin{itemize}
    \item[] $\myB$ is obtained from $\myA$ by transformation $\myF$
  \end{itemize}
  is clearly equivalent to
  \begin{itemize}
    \item[]  $\myF$ solves the linear differential system
      \[
        \frac{d\myF}{dt}=\myB\myF-\myF\myA
      \]
      which is denoted by $[\myA,\myB]$.
  \end{itemize}
\end{rem}

This defines equivalence relations\dots \TODO

\begin{defn}
  Two germs of meromophic connections are isomorphic if and only if \TODO
\end{defn}

\begin{defn}
  \begin{itemize}
    \item An \emph{isotropy} of $A^0$ is a transformation $\hat F$ which
      satisfy ${}^{\hat F}\!A=A$. Thus, the isotropies are the solutions of the
      system $[\End A^0]:=[A^0,A^0]$.
      \marginnote{\cite{Loday1994} p. 853}
    \item Let $G_0(A^0)$ denote the set of all isotropies of $A^0$.
  \end{itemize}
\end{defn}

\TODO[Define marked pairs, twice]

\section{Models and formal decomposition of a germ}
\begin{paracol}{2}\sloppy
\switchcolumn[0]\noindent
  For a $\phi\in\C\{t\}[t^{-1}]$ we use $\cE^{\phi}$ to denote the germ
  \[
    (\cE^{\phi},\nabla)=(\C\{t\},d-d\phi)\,.
  \]
  This is the system satisfied by the function $e^\phi$.
  \\For $\alpha\in\C$, define $\cN_{\alpha,0}$ as the germ
  \[
    (\cN_{\alpha,0},\nabla)=(\C\{t\}[t^{-1}],d+\alpha dt/t)\,.
  \]
  \begin{prop}
    \begin{enumerate}
      \item Every germ of a meromophic connection of rank one is isomorphic to
        some germ
        \[
          \cE^\phi\otimes\cN_{\alpha,0}
        \]
      \item Two such germs corresponding to $(\phi_1,\alpha_1)$ and
        $(\phi_2,\alpha_2)$ are isomorphic if and only if
        \begin{itemize}
          \item $\phi_1-\phi_2$ has no pole and
          \item $\alpha_1-\alpha_2\in\Z$.
        \end{itemize}
    \end{enumerate}
  \end{prop}
  \begin{proof}
    See \cite{sabbah2007isomonodromic} Proposition II.5.1.
  \end{proof}
\switchcolumn[1]\noindent
  As something like building blocks of some meromophic
  connections\footnote{Which we will call models} we introduce the elementary
  irregular models.
  \begin{defn}[Elementary irregular models]
    \marginnote{\cite{sabbah2007isomonodromic} Definition II.5.2}
    A Germ $(\cM,\nabla)$ is called \emph{elementary} if it is isomorphic to
    some germ $(\cE^\phi,\nabla)\otimes(\cR,\nabla)$ where
    \begin{itemize}
      \item $(\cR,\nabla)$ has regular singularity at $\{0\}$.
    \end{itemize}
    \marginnote{\cite{sabbah2007isomonodromic} II.2.f}
  \end{defn}
\end{paracol}
\begin{cor}
  $\cE^\phi$ is determined by the class of $\phi$ in
  $\C\{t\}[t^{-1}]/\C\{t\}$. In the following, we will only consider the
  unique element $\phi$ in each class which has no holomorphic part.
\end{cor}
\begin{defn}
  \def\myPhi{\textcolor{red!60!black}{\phi}}
  \def\myE{\textcolor{green!40!black}{\cE^{\myPhi}}}
  A germ $(\cM,\nabla)$ is a \emph{model} or a \emph{formal decomposition} if
  there exists a isomorphism
  \begin{multicols}{2}
    \[
      \lambda:(\cM,\nabla)
      \overset{\cong}{\longrightarrow}
      % \cong
      \bigoplus_{~\tikzmark{e3}\!\!\myPhi}
      \underset{\text{merom. Zus.}}{%
        \underset{\text{elementare}}{%
          \underbrace{%
            \overset{\tikzmark{e2}}{\myE}
            \otimes
            \overset{\tikzmark{e1}}{\textcolor{blue!40!black}{\cR_{\myPhi}}}
          }
        }
      }\,.
    \]
    \columnbreak
    \begin{itemize}
      \item[\tikzmarkb{n2}{green}] irregular singular
      \item[\tikzmarkc{n1}{blue}] has regular singularity at $\{0\}$
      \item[\tikzmarkc{n3}{red}] $\myPhi\in t^{-1}\C[t^{-1}]$ pairwise distinct
    \end{itemize}
    \begin{tikzpicture}[remember picture,overlay]
      \draw[->,blue!50!white,thick] (n1) to[out=180,in=70] (e1);
      \draw[->,green!40!black,thick] (n2) to[out=180,in=70] (e2);
      \draw[->,red!50!white,thick] (n3) to[out=205,in=-70] (e3);
    \end{tikzpicture}
  \end{multicols}
\end{defn}
\begin{lem}
  If $(\cM,\nabla)$ is a model, if and only if there is a basis in which the
  connection matrix has the form
  \begin{multicols}{2}
    \[
      A^0=d\overset{~\tikzmark{e1}}{\textcolor{green!40!black}{Q}}
         +~\tikzmark{e2}\!\!\Lambda\frac{dt}{t} \,.
    \]
    \columnbreak
    \begin{itemize}
      \item[\tikzmarkb{n1}{green}]
        $\textcolor{green!40!black}{Q
          =\diag(\textcolor{red!60!black}{\phi_1,\dots,\phi_n})}$
      \item[\tikzmarkc{n2}{blue}]
        $\Lambda$ constant
    \end{itemize}
    \begin{tikzpicture}[remember picture,overlay]
      \draw[->,green!40!black,thick] (n1) to[out=180,in=70] (e1);
      \draw[->,blue!50!white,thick] (n2) to[out=180,in=-70] (e2);
    \end{tikzpicture}
  \end{multicols}
\end{lem}
\begin{proof}
  \TODO{}
\end{proof}

The important theoreme here is the Levelt-Turittin-theoreme.
\begin{thm}[Levelt-Turittin]
  To each germ $(\cM,\nabla)$ of a meromorphic connection there exists, after
  pontentially needed \textcolor{blue!60!black}{pullback by some suitable
  ramification $t=z^q$ of order $q\geq1$}, a
  \textcolor{green!40!black}{\textbf{formal}} isomorphism
  \[
    \textcolor{green!40!black}{\hat{\textcolor{black}{\lambda}}}:
    \textcolor{blue!60!black}{\pi^{+}}
    \textcolor{green!40!black}{\hat{\textcolor{black}{\cM}}}
    \overset{\cong}\longrightarrow
    \textcolor{green!40!black}{\hat{\textcolor{black}{\cM}}^{good}}
    :=\textcolor{green!40!black}{\hat\cO_M\otimes}\cM^{good}
  \]
  to a model $\cM^{good}$\footnote{The word `good' only means, that it is a
  model. In \cite{sabbah2007isomonodromic} it means good-model,
  which\dots\TODO}.
\end{thm}
\begin{proof}
  See \TODO
\end{proof}

\begin{prop}
  \marginnote{This condition \textbf{might be} equivalent to the
    condition of being \textbf{nice} in \cite{thboalch}.}
  Let $(\cM,\nabla)$ be a germ,
  \begin{itemize}
    \item equipped with a basis in which the matrix $A$ takes the form
      \[
        A=t^{-r}A(t)\frac{dt}{t}
      \]
      with
      \begin{itemize}
        \item $r\geq1$,
        \item $A$ has holomorphic entries, and
        \item $A_0:=A(0)$ being regular semisimple\footnote{i.e.\ with
          pairwise distinct eigenvalues.}
      \end{itemize}
  \end{itemize}
  Then there is no ramification needed, to apply the Levelt-Turittin-theoreme.
  \begin{comment}
    Further, all the summands $\cR_\phi$ have rank one, which is not the case
    in general.
  \end{comment}
\end{prop}
\begin{proof}
  See \cite{sabbah2007isomonodromic} theoreme II.5.7.
\end{proof}

