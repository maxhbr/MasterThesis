\chapter{Meromorphe Zusammenhänge}
\begin{comment}
Siehe:
\begin{itemize}
  \item \cite{sabbah2007isomonodromic}
\end{itemize}
\end{comment}
%
\ccite[Def 2.1]{boalch}
Sei $D=k_1(a_1)+\dots+k_m(a_m)$ ein effektiver Divisor auf $\P^1$ und sei
$V\to\P^1$ ein Rang $n$ Vektorbündel.
\begin{defn}[Meromorpher Zusammenhang]
Ein meromorpher Zusammenhang $\nabla$ auf $V$ mit Polen auf $D$ ist eine
Abbildung
\[
  \nabla: V\to V\otimes K(D)
\]
so dass die Leibnitzregel
\begin{equation}
  \nabla(fv)=(df)\otimes v + f\nabla v
\end{equation}
erfüllt ist. Hierbei ist $v$ ein lokaler Schnitt von $V$, $f$ eine lokale
holomorphe Funktion und $K$ die Garbe der holomorphen Eins-Formen.
\comm{ Sabbah schreibt $\Omega_{\P^1}^1$ für $K$ }
\end{defn}
%
\begin{comment}
\begin{prop}
Die Differenz zweier meromorphen Zusammenhänge ist $???$-linear.
\end{prop}
\begin{proof}
Denn für zwei meromorphe Zusammenhänge $\nabla_1$ und $\nabla_2$ gilt:
\begin{align*}
(\nabla_1 - \nabla_2)(fv) &= \nabla_1(fv) - \nabla_2(fv)
\\&=(df)\otimes v + f\nabla_1 v -(df)\otimes v - f\nabla_2 v
\\&=f\nabla_1 v - f\nabla_2 v
\\&=f(\nabla_1 - \nabla_2) v
\end{align*}
und da ??? gilt reicht dies um die Aussage zu zeigen
\end{proof}
\end{comment}
%
\begin{prop}[Proposition 2.1]
\end{prop}

% vim:set ft=tex foldmethod=marker foldmarker={{{,}}}:
