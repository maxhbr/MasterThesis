\chapter{Meromorphe Zusammenhänge}
\begin{comment}
Siehe:
\begin{itemize}
  \item \cite{sabbah2007isomonodromic}
\end{itemize}
\end{comment}
%
\ccite[Def 2.1]{boalch}
Sei $D=k_1(a_1)+\dots+k_m(a_m)$ ein effektiver Divisor auf $\P^1$ und sei
$V\to\P^1$ ein Rang $n$ Vektorbündel.
\begin{defn}[Meromorpher Zusammenhang]
Ein meromorpher Zusammenhang $\nabla$ auf $V$ mit Polen auf $D$ ist eine
Abbildung
\[
  \nabla: V\to V\otimes K(D)
\]
so dass die Leibnitzregel
\begin{equation}
  \nabla(fv)=(df)\otimes v + f\nabla v
\end{equation}
erfüllt ist. Hierbei ist $v$ ein lokaler Schnitt von $V$, $f$ eine lokale
holomorphe Funktion und $K$ die Garbe der holomorphen Eins-Formen.
\comm{Sabbah schreibt $\Omega_{\P^1}^1$ für $K$ }
\end{defn}
%
\begin{comment}
\begin{prop}
Die Differenz zweier meromorphen Zusammenhänge ist $???$-linear.
\end{prop}
\begin{proof}
Denn für zwei meromorphe Zusammenhänge $\nabla_1$ und $\nabla_2$ gilt:
\begin{align*}
(\nabla_1 - \nabla_2)(fv) &= \nabla_1(fv) - \nabla_2(fv)
\\&=(df)\otimes v + f\nabla_1 v -(df)\otimes v - f\nabla_2 v
\\&=f\nabla_1 v - f\nabla_2 v
\\&=f(\nabla_1 - \nabla_2) v
\end{align*}
und da ??? gilt, reicht dies um die Aussage zu zeigen
\end{proof}
\end{comment}

\begin{paracol}{2}
Wählt man eine lokale Koordinate $z$ auf $\P^1$ welche bei $a_i$
verschwindet. Dann hat, in terms of local trivialization of $V$, $\nabla$ die
form:
\[
\nabla = d-{}^iA
\]
wobei
\[
{}^iA=\left(\sum^{0}_{j=k_i}{}^iA_j\frac{dz}{z^{j}}\right)+A_0dz+\dots
\]
\switchcolumn %%%%%%%%%%%%%%%%%%%%%%%%%%%%%%%%%%%%%%%%%%%%%%%%%%%%%%%%%%%%%%%%%
\begin{defn}
\cite[Def 1.5]{thboalch}.
A \emph{germ of a meromorphic linear differential system} (of rank $n$), or
just \emph{system} from now on, is a germ of a meromorphic connection on the
trivial vector bundle with fibre $E$.
\end{defn}
Dieses $\nabla = d-{}^iA$ definiert ein \emph{System}. Der Buchstabe $k$ wird
die Ordnung des Pols bezeichnen und $\Syst_k$ den Vektorraum aller Systeme mit
Polordnung von höchstens $k$.

Sind Zwei Systeme $A$ und $B$ gegeben, so definiert $dX=BX-XA$ ein System auf
$\End(E)$ welches wir mit $\Hom(A,B)$ bezeichnen wollen.
\end{paracol}

\begin{paracol}{2}
\switchcolumn %%%%%%%%%%%%%%%%%%%%%%%%%%%%%%%%%%%%%%%%%%%%%%%%%%%%%%%%%%%%%%%%%
\begin{defn}
A \emph{nice formal normal form} $A^0$ is a nice diagonal system with no
holomorphic part. Such $A^0$ can be uniquely written as
\[
A^0:= dQ + \Lambda \frac{dz}{z},
\qquad
Q := \diag(q1,\dots ,qn)
\]
where $q_1,\dots ,q_n\in z^{−1}C[z^{−1}]$ are diagonal polynomials of degree
$k-1$ in $z^{−1}$ with no
constant term and $\Lambda = Res_0(A_0)$ is a constant diagonal matrix.
\end{defn}
\end{paracol}


\begin{paracol}{2}
\begin{defn}[compatible framing]
A \textbf{compatible framing} at $a_i$ of a vector bundle $V$ with nice
connection $\nabla$ is a choice of isomorphism $g$ between the fibre
$V_{a_i}$ and $\C^n$ which is compatible with $\nabla$ in the sense that the
leading term ${}^iA_{k_i}$ of $\nabla$ is diagonal in any local
trivialisation of $V$ extending this isomorphism.
\end{defn}
\switchcolumn %%%%%%%%%%%%%%%%%%%%%%%%%%%%%%%%%%%%%%%%%%%%%%%%%%%%%%%%%%%%%%%%%
\begin{defn}[marked pair]
A marked pair is a pair $(A, \hat F)$ consisting of a nice system $A$ and a choice of formal
isomorphism
$\hat F \in \hat G$
between $A$ and some formal normal form $A^0(A = \hat F[A0])$
\end{defn}
\end{paracol}

% vim:set ft=tex foldmethod=marker foldmarker={{{,}}}:
