\chapter{Meromorphe Zusammenhänge}
\begin{comment}
  Siehe:
  \begin{itemize}
    \item \cite{boalch} and \cite{thboalch}
    \item \cite{sabbah2007isomonodromic}
  \end{itemize}
  and
  \begin{itemize}
    \item \cite{Varadarajan96linearmeromorphic}
  \end{itemize}
\end{comment}

Let $M$ be a riemanian surface and let $Z=k_1(a_1)+\cdots+k_m(a_m)>0$ be an
effective divisor\footnote{The $a_i$ are distinct points and the $k_i$ are
positive integers.} on $M$.
It is sufficient to think $M=\P^1$.

Let $\sM$ be a holomorphic Bundle i.e.\ a locally free $\cO_m$-module of rank
$n$.

A meromorphic connection is then defined as follows.
\begin{defn}
  \def\myU{\textcolor{green!30!black}{U}}
  \def\mys{\textcolor{blue!60!black}{s}}
  \def\myf{\textcolor{red!60!black}{f}}
  A \emph{meromorphic connection $(\sM,\nabla)$ on $\sM$ with poles on $Z$}
  is a $\C$-linear morphism of sheaves
  \[
    \nabla:\sM\to\Omega_M^1(*Z)\otimes\sM
  \]
  satisfying, for each $\myU\overset{\text{op.}}{\subset} M$, the
  \emph{Leibniz rule}
  \begin{multicols}{2}
    \[
      \nabla(\tikzmark{e2}\myf\overset{\tikzmark{e1}}\mys)
        =\myf\nabla\mys+(d\myf)\otimes\mys \,.
    \]
    \columnbreak
    \begin{itemize}
      \item[\tikzmarkc{n1}{blue}]
        $\mys\textcolor{blue!60!black}{\in\Gamma(\myU,\sM)}$
      \item[\tikzmarkc{n2}{red}]
        $\myf\textcolor{red!60!black}{\in\cO_M(\myU)}$
    \end{itemize}
  \end{multicols}
  \begin{tikzpicture}[remember picture,overlay]
    \draw[->,blue!50!white,thick] (n1) to[out=180,in=70] (e1);
    \draw[->,red!50!white,thick] (n2) to[out=180,in=-70] (e2);
  \end{tikzpicture}
\end{defn}
\begin{rem}
  \begin{enumerate}
    \item We will occasionally omit the $\nabla$ and simply call $\sM$ the
      meromorphic connection.
    \item Here, the variant `holomorphic bundle with meromorphic connection' is
      chosen, like in \cite{boalch}.  There is also the twisted description
      `meromorphic bundle with holomorphic connection' which is used in
      \cite{sabbah2007isomonodromic}.
      \\ By choosing a lattice of a meromorphic bundle, one gets a holomorphic
      bundle but the connection is no longer holomorphic. Such that we obtain a
      meromorphic connection in our\dots\TODO
  \end{enumerate}
\end{rem}

\begin{defn}[Flatness (0.12.2)]
  The connection $\nabla:\cM\to \Omega_M^1\otimes_{\cO_M}\cM$ is said to be
  \emph{integrable} or \emph{flat}, if
  \begin{itemize}
    \item its curvature $R_\nabla\equiv0$
    where
    \begin{itemize}
      \item $R_\nabla:=\nabla\circ\nabla:\cE\to\Omega_M^2\otimes_{\cO_M}\cE$
        is a $\cO_M$-linear morphism.
    \end{itemize}
  \end{itemize}
  \begin{prop}[0.12.4]
    The connection $\nabla$ is flat if and only if, in any local basis $e$ of
    $\cM$, the connection matrix $\Omega$ satisfies
    \[
      d\omega + \omega \wedge \omega = 0.
    \]
  \end{prop}
  We will say that a connection on a meromorphic bundle is \emph{integrable} or
  \emph{flat} if its restriction to $M\backslash Z$ is an integrable connection
  on the holomorphic bundle $\sM_{|M\backslash Z}$.
\end{defn}

\section{Local expression of meromorphic connections}
\TODO[Germ of a meromorphic Connection / Connection matrices
  / systems]
\begin{comment}
  \begin{itemize}
    \item \cite{sabbah2007isomonodromic} p.28
    \item \cite{thboalch} p.2
    \item \cite{babbitt1989local} p. 11
  \end{itemize}
\end{comment}
We will usually be interested in local information of a meromorphic connection.
This means, that we look at a connection in a neighbourhood of $0$ and  allow
only one singularity at $0$.
There are many ways of expressing the local information, we will either talk
about germs of meromorphic connections or systems, which is coordinate-
and trivialization-dependent.

\begin{paracol}{2}\sloppy
\switchcolumn[0]\noindent
  Let $U$ be a trivializing open set for $M$ and let
  $\textbf{e}=(e_1,\dots,e_n)$ be a basis of $\Gamma(U,\sM)$.
  There exists then a $n\times n$ matrix $A$ of meromorphic 1-forms such
  that\dots
  \begin{defn}
    This matrix is called \emph{connection matrix} for $(\sM,\nabla)$.
  \end{defn}
\switchcolumn[1]\noindent
  Let $U$ be a neighbourhood of $0$ and $t$ a coordinate on $U$ vanishing at
  $0$.
\end{paracol}

\begin{comment}
  Some Literature like \cite{sabbah_cimpa90} always talks about germs\dots
\end{comment}
\begin{defn}
  A \emph{germ of a meromorphic linear differential system} of rank $n$, or
  just a \emph{system}, is a germ of a meromorphic connection on the trivial
  vector bundle\footnote{With a chosen trivialization.} with fibre $E=\C^n$.
\end{defn}
The set of systems is, via the Taylor expansion, isomorphic to
$\End(E)\otimes\C(\!\{t\}\!)$. A matrix
$A'\in\End(E)\otimes\C(\!\{t\}\!)$ determines a system
\[
  \frac{dv}{dt}=A'v
\]
which corresponds to the connection germ $\nabla=[A]=d-A$ on $E$ where
$A=A'dt$.
\begin{paracol}{2}\sloppy
\switchcolumn[1]\noindent
  Thus we obtain the definition of the connection matrices.
\end{paracol}
\begin{comment}
  \cite{boalch} wants \textbf{generic} meromorphic connections
  \begin{itemize}
    \item\dots simplest jet sufficient\dots
  \end{itemize}
\end{comment}

\begin{comment}
  \TODO[Gauge transformation]
  \begin{itemize}
    \item meromorphic transformation \cite{Loday1994} p. 852
    \item holomorphic trasformation
  \end{itemize}
\end{comment}

\begin{cor}
  The connection matrices describe the germs of meromorphic connections.
\end{cor}
\begin{proof}
  \TODO
\end{proof}

\subsection{Connection matrix \leftrightarrow{} system}
\subsection{Connection matrix \rightarrow{} connection}
\begin{comment}
  There is a thm in \cite{sabbah2007isomonodromic}
\end{comment}

\subsection{Meromorphic / formal transformation}
\begin{comment}\footnotesize
  see \cite{thboalch} \textbf{Rem 1.41 on p. 16}:
  \begin{rem}
    Note that in most of the recent references we have used, Stokes matrices
    are used to classify
    \begin{itemize}
      \item meromorphic connections within fixed \textbf{formal meromorphic
        classes, modulo meromorphic equivalence}.
    \end{itemize}
    Whereas here we classify
    \begin{itemize}
      \item meromorphic connections within fixed \textbf{formal analytic
        classes, modulo analytic equivalence},
    \end{itemize}
    as is done in the older literature.  The fact is that the sets equivalence
    classes are the same in both cases. It is important for us to work with
    analytic, rather than meromorphic gauge transformations, because then the
    $\C^\infty$ viewpoint in Chapter 3 is cleaner. This distinction relates to
    the difference between \textbf{‘regular singular’} connections and
    \textbf{‘logarithmic’} connections.
  \end{rem}
\end{comment}
\begin{notations}
  We will use the following notations
  \begin{itemize}
    \item $G=\Gl_n(\C)$
    \item $G[t]=\Gl_n(\C[t])$
    \item $G\{t\}=\Gl_n(\C\{t\})$ analytic tranformations
    \item $G(\!\{t\}\!)=\Gl_n(\C\{t\}[t^{-1}])$ meromorphic transformations
    \item $G\llbracket t\rrbracket=\Gl_n(\C\llbracket t\rrbracket)$
      (maby not applicable) formal tranformations
    \item $G(\!(t)\!)=\Gl_n(\C\llbracket t\rrbracket[t^{-1}])$
      (maby not applicable) formal meromorphic tranformations
  \end{itemize}
\end{notations}
By \emph{transformation} (or \emph{meromorphic transformation}) of a connection
matrix $A$ we mean a linear change of the trivialization. Such a change is given
by a matrix $F\in G(\!\{t\}\!)$ and the transformed matrix ${}^F\!A$ is
obtained through
\[
  {}^F\!A=(dF)F^{-1} + FAF^{-1} \,.
\]
If $F$ is formal i.e. $F\in G(\!(t)\!)$, it will usually be denoted by
$\hat F$. The transformation of $A$ by $\hat F$ is not guaranteed to have
convergent entries.
We denote by $\hat G(A)$ the set of all \emph{(applicable) formal
transformations}
\[
  \hat G(A):=\left\{\hat F\in G(\!(t)\!)
    \mid {}^{\hat F}\!A \text{ has convergent entries}
  \right\}\,.
\]
\begin{rem}
  \def\myB{\textcolor{blue!60!black}{B}}
  \def\myA{\textcolor{green!30!black}{A}}
  \def\myF{\textcolor{red!60!black}{F}}
  The condition
  \begin{itemize}
    \item[] $\myB$ is obtained from $\myA$ by transformation $\myF$
  \end{itemize}
  is clearly equivalent to
  \begin{itemize}
    \item[]  $\myF$ solves the linear differential system
      \[
        \frac{d\myF}{dt}=\myB\myF-\myF\myA
      \]
      which is denoted by $[\myA,\myB]$.
  \end{itemize}
\end{rem}

This defines equivalence relations\dots \TODO and one optains the notions of
(formal) equivalence classes of meromorphic connections.

\begin{defn}
  Two germs of meromorphic connections are (formally) isomorphic if and only if
  their corresponding connection matrices are (formally) equivalent.
  \TODO[is this a proposition]
\end{defn}

\subsection{Regular / irregular singularities}
\TODO[regular / irregular singularity]

\section{Models and formal decomposition of a germ}
\begin{paracol}{2}
\switchcolumn[0]\noindent
  As something like building blocks of some meromorphic connections we introduce
  the elementary irregular models. Into which meromorphic connections, after
  potentially needed pullback, decompose.

  The building blocks of these building blocks are defined as follows.
\switchcolumn[1]\noindent
  In every formal equivalence class of meromorphic connections, there are some
  meromorphic connection of special form, which we will call models. They are
  not unique but all of them lie in the same convergent equivalence class.

  On the other hand, the Levelt-Turittin theoreme says, that  each meromorphic
  connection is, after potentially needed ramification, formally isomorphic to
  such a model, that means, that in each formal class there are some models.
  \TODO[Problems with ramification??]
\end{paracol}
\begin{defn}
  \begin{itemize}
    \item For a $\phi\in\C\{t\}[t^{-1}]$ we use $\cE^{\phi}$ to denote the germ
      \[
        (\cE^{\phi},\nabla)=(\C\{t\},d-d\phi)\,.
      \]
      This is the system satisfied by the function $e^\phi$.
      \begin{cor}
        $\cE^\phi$ is determined by the class of $\phi$ in
        $\C\{t\}[t^{-1}]/\C\{t\}=t^{-1}\C[t^{-1}]$. In the following, we will
        only consider the unique element $\phi$ in each class which has no
        holomorphic part.
      \end{cor}
    \item For $\alpha\in\C$, define $\cN_{\alpha,0}$ as the germ
      \[
        (\cN_{\alpha,0},\nabla)=(\C\{t\}[t^{-1}],d+\alpha dt/t)\,.
      \]
      This is the system satisfied by \TODO{}.
  \end{itemize}
\end{defn}
\begin{prop}
  \begin{enumerate}
    \item Every regular germ of a meromorphic connection is isomorphic to some
      \TODO{}
      \begin{rem}
        For a detailed analysis of regular meromorphic connections see
        \cite{sabbah2007isomonodromic} chapter II.2.
      \end{rem}
    \item Every germ of a meromorphic connection of rank one is isomorphic to
      some germ
      \[
        \cE^\phi\otimes\cN_{\alpha,0} \,.
      \]
    \item Two such germs corresponding to $(\phi_1,\alpha_1)$ and
      $(\phi_2,\alpha_2)$ are isomorphic if and only if
      \begin{itemize}
        \item $\phi_1-\phi_2$ has no pole and
        \item $\alpha_1-\alpha_2\in\Z$.
      \end{itemize}
  \end{enumerate}
\end{prop}
\begin{proof}
  See \cite{sabbah2007isomonodromic} Proposition II.5.1.
\end{proof}
\begin{defn}
  \marginnote{\cite{sabbah2007isomonodromic} Definition II.5.2}
  A Germ $(\cM,\nabla)$ is called \emph{elementary} if it is isomorphic to
  some germ $(\cE^\phi,\nabla)\otimes(\cR,\nabla)$ where
  \begin{itemize}
    \item $(\cR,\nabla)$ has regular singularity at $\{0\}$ but has not to be
      of rank $1$.
  \end{itemize}
  \marginnote{\cite{sabbah2007isomonodromic} II.2.f}
\end{defn}
\begin{defn}
  \def\myPhi{\textcolor{red!60!black}{\phi}}
  \def\myE{\textcolor{green!40!black}{\cE^{\myPhi}}}
  A germ $(\cM,\nabla)$ is a \emph{model} if there exists a isomorphism
  \begin{multicols}{2}
    \[
      \lambda:(\cM,\nabla)
      \overset{\cong}{\longrightarrow}
      % \cong
      \bigoplus_{~\tikzmark{e3}\!\!\myPhi}
      \underset{\text{merom. Zus.}}{%
        \underset{\text{elementare}}{%
          \underbrace{%
            \overset{\tikzmark{e2}}{\myE}
            \otimes
            \overset{\tikzmark{e1}}{\textcolor{blue!40!black}{\cR_{\myPhi}}}
          }
        }
      }\,.
    \]
    \columnbreak
    \begin{itemize}
      \item[\tikzmarkb{n2}{green}] is irregular singular
      \item[\tikzmarkc{n1}{blue}] has regular singularity at $\{0\}$
      \item[\tikzmarkc{n3}{red}] $\myPhi\in t^{-1}\C[t^{-1}]$ pairwise distinct
    \end{itemize}
    \begin{tikzpicture}[remember picture,overlay]
      \draw[->,blue!50!white,thick] (n1) to[out=180,in=70] (e1);
      \draw[->,green!40!black,thick] (n2) to[out=180,in=70] (e2);
      \draw[->,red!50!white,thick] (n3) to[out=205,in=-70] (e3);
    \end{tikzpicture}
  \end{multicols}
\end{defn}
\begin{lem}
  If $(\cM,\nabla)$ is a model, if and only if there is a basis in which the
  connection matrix has the form
  \begin{multicols}{2}
    \[
      A^0=d\overset{~\tikzmark{e1}}{\textcolor{green!40!black}{Q}}
        +~\tikzmark{e2}\!\!\textcolor{blue!40!black}{\Lambda}\frac{dt}{t}
    \]
    \columnbreak
    \begin{itemize}
      \item[\tikzmarkb{n1}{green}]
        $\textcolor{green!40!black}{Q
          =\diag(\textcolor{red!60!black}{\phi_1,\dots,\phi_n})}$
      \item[\tikzmarkc{n2}{blue}]
        $\textcolor{blue!40!black}{\Lambda}$ constant \emph{matrix of exponets
        of formal Monodromy}\footnote{usually not of rank $1$}
    \end{itemize}
    \begin{tikzpicture}[remember picture,overlay]
      \draw[->,green!40!black,thick] (n1) to[out=180,in=70] (e1);
      \draw[->,blue!50!white,thick] (n2) to[out=180,in=-70] (e2);
    \end{tikzpicture}
  \end{multicols}
  where the $\textcolor{red!60!black}{\phi_i}$ are the same as in the above
  definition.
\end{lem}
\begin{proof}
  \TODO{}
\end{proof}

The important theoreme here is the Levelt-Turittin-theoreme.
\begin{thm}[Levelt-Turittin]
  To each germ $(\cM,\nabla)$ of a meromorphic connection there exists, after
  potentially needed \textcolor{blue!60!black}{pullback by some suitable
  ramification $t=z^q$ of order $q\geq1$}, a
  \textcolor{green!30!black}{\textbf{formal}} isomorphism
  \[
    \textcolor{green!30!black}{\hat{\textcolor{black}{\lambda}}}:
    \textcolor{blue!60!black}{\pi^{+}}
    \textcolor{green!30!black}{\hat{\textcolor{black}{\cM}}}
    \overset{\cong}\longrightarrow
    \textcolor{green!30!black}{\hat{\textcolor{black}{\cM}}^{good}}
    :=\textcolor{green!30!black}{\hat\cO_M\otimes}\cM^{good}
  \]
  to a model $\cM^{good}$\footnote{The word `good' only means, that it is a
  model. In \cite{sabbah2007isomonodromic} it means good-model,
  which\dots\TODO}.
  We then call $\cM^{good}$ a \emph{formal decomposition} or \emph{formal
  model} of $\cM$.
\end{thm}
\begin{proof}
  See \TODO
\end{proof}

\begin{prop}
  \marginnote{This condition \textbf{might be} equivalent to the
    condition of being \textbf{nice} in \cite{thboalch}.}
  Let $(\cM,\nabla)$ be a germ,
  \begin{itemize}
    \item equipped with a basis in which the matrix $A$ takes the form
      \[
        A=t^{-r}A(t)\frac{dt}{t}
      \]
      with
      \begin{itemize}
        \item $r\geq1$,
        \item $A$ has holomorphic entries, and
        \item $A_0:=A(0)$ being regular semisimple\footnote{i.e.\ with
          pairwise distinct eigenvalues.}.
      \end{itemize}
  \end{itemize}
  Then there is no ramification needed, to apply the Levelt-Turittin-theoreme.
  \begin{comment}
    Further, all the summands $\cR_\phi$ have rank one, which is not the case
    in general.
  \end{comment}
\end{prop}
\begin{proof}
  See \cite{sabbah2007isomonodromic} theoreme II.5.7.
\end{proof}

\begin{comment}
  \subsection{By fundamental solutions\dots}
  \marginnote{\cite[853]{Loday1994}, \cite[Th.4.3.1]{Loday2014}}
  A system
  \[
    \frac{dX}{dt}=A^0X
  \]
  is called a \emph{normal form} if it satisfies the following Properties:
  there exists a fundamental solution $X_0$ of $[A^0]$, i.e., a $n\times n$
  matrix

  \dots

  of the form $X_0=t^Le^{Q(t^{-1})}$ where
  \begin{itemize}
    \item $L$ is a constant matrix called the \emph{matrix of exponents of
      formal Monodromy} and
    \item $Q(t^{-1})=\diag\left(q_1(t^{-1}),\dots,q_n(t^{-1})\right)$ is a
      diagonal matrix where
      \begin{itemize}
        \item the $q_j(t^{-1})$ are polynomials in a root $z^{-1}=t^{-1/p}$,
          $p\in\N^*$, of the variable $t$ without constant term.
      \end{itemize}
  \end{itemize}
  Such a solution will be called \emph{normal solution}.
\end{comment}

\section{Meromorphic / formal transformation II}
\begin{defn}
  \begin{itemize}
    \item An \emph{isotropy} of $A^0$ is a transformation $\hat F$ which
      satisfy ${}^{\hat F}\!A=A$. Thus, the isotropies are the solutions of the
      system $[\End A^0]:=[A^0,A^0]$.
      \marginnote{\cite[853]{Loday1994}}
      \begin{rem}
        They are, a priori, formal transformations. Actually $G(A^0)$ is a
        subgroup of $\Gl_n(\C[1/x,x])$.
      \end{rem}
    \item Let $G_0(A^0)$ denote the set of all isotropies of $A^0$.
      \begin{comment}
        In the nice case this is only $T$?
      \end{comment}
  \end{itemize}
\end{defn}
\begin{lem}
  \marginnote{\cite[854]{Loday1994}}
  Two formal transformations $\hat F_1$ and $\hat F_2$ take $A^0$ into
  equivalent matrices ${}^{\hat F_1}\!A^0$ and  ${}^{\hat F_2}\!A^0$ if and
  only if there exists $f_0\in G_0(A^0)$ such that $\hat F_1=\hat F_2f_0$.
  \TODO[\Leftrightarrow{} $f_0$ is a isotropy?]
\end{lem}
\begin{proof}
  \TODO
\end{proof}

%%%%%%%%%%%%%%%%%%%%%%%%%%%%%%%%%%%%%%%%%%%%%%%%%%%%%%%%%%%%%%%%%%%%%%%%%%%%%%%
\section{The classifying set}
We want to understand the Set $\{[(\cM,\nabla)]\}$ of the (convergent)
isomorphism classes of all meromorphic connections. Since we have also the
formal classification and we know, that all elements in a convergent
isomorphism class lie in the same formal isomorphism class, we can reduce the
problem by fixing a model $(\cM^{good},\nabla^{good})$ with the corresponding
connection matrix $A^0$. Thus we can restrict ourself to the subset
\begin{align*}
  {}^0C(\cM^{good},\nabla^{good})=\left\{
    \left[(\cM,\nabla)\right]
      \mid \text{there exists a formal isomorphism }
      \hat f:(\hat\cM,\hat\nabla)
        \overset{\sim}\longrightarrow
        (\hat\cM^{good},\hat\nabla^{good})
  \right\}
\end{align*}
of all isomorphism classes of
meromorphic connections, which are formally isomorphic to
$(\cM^{good},\nabla^{good})$. This is the set, that we will be calling the
\emph{classifying set}.
\begin{comment}
  \begin{itemize}
    \item \cite{thboalch} p.6
      \begin{itemize}
        \item \cite{boalch} p.19
      \end{itemize}
    \item \cite{Loday1994} p.852
    \item \cite{sabbah2007isomonodromic} p.111
  \end{itemize}
\end{comment}

It is convenient to look at the slightly larger space of \emph{marked pairs}
\[
  \cH=\cH(\cM^{good},\nabla^{good})=\left\{
    \left[(\cM,\nabla,\hat f)\right]
      \mid
      \hat f:(\hat\cM,\hat\nabla)
        \overset{\sim}\longrightarrow
        (\hat\cM^{good},\hat\nabla^{good})
  \right\}
\]
in which we also keep the additional information, of the formal isomorphism, by
which the meromorphic connection is formally isomorphic to the model.
Where the isomorphisms of marked pairs are defined as follows:
\begin{defn}
  Two germs $(\cM,\nabla,\hat f)$ and $(\cM',\nabla',\hat f')$ are
  isomorphic if there exists an isomorphism
  $g:(\cM,\nabla)\overset{\sim}\longrightarrow(\cM',\nabla')$ such that
  $\hat f=\hat f'\circ \hat g$.
  \begin{comment}
    \cite[111]{sabbah2007isomonodromic}:\dots it is important to remark that
    such an isomorphism is then unique.
  \end{comment}
\end{defn}

Equivalently, one can talk in terms of systems. We then denote by
\[
  \Syst_m(A^0):=\{d-A
    \mid A={}^{\hat F}\!A^0 \text{ for some } \hat F\in G(\!(t)\!)\}
\]
the set of systems formally meromorphic equivalent to $A^0$\footnote{Since we
use meromorphic equivalences we denote that by the subscript ${}_m$.}.
Thus ${}^0C(\cM^{good},\nabla^{good})$ corresponds to
the set ${}^0C(A^0):=\Syst_m(A^0)/G(\!\{t\}\!)$ of meromorphic classes which
are formally equivalent to $A^0$.
\TODO[\cite{thboalch} p. 3: In the logarithmic case\dots]
Analogous, $\cH(\cM^{good},\nabla^{good})$ corresponds to the set $\cH(A^0)$ of
equivalence classes in
\[
  \hat\Syst_m(A^0):=\{(A,\hat F)
    \mid A={}^{\hat F}\!A^0 \text{ for some } \hat F\in G(\!(t)\!)\} \,.
\]

\begin{lem}
  Since $G_0(A^0)$ is defined as the stabilizer\TODO[correct?] of $A^0$ we deduce
  \[
    \Syst_m(A^0)\cong \hat G(A^0)/G_0(A^0) \,.
  \]
  \begin{cor}
    Thus the \emph{set of meromorphic classas of systems formally equivalent
      to $A^0$} are just the orbits of $G\{t\}$, that is
    \[
      {}^0C(A^0)\cong G\{t\}\backslash\hat G(A^0)/G_0(A^0)
    \]
    whereas the \emph{set of meromorphic classes of transformations of $[A^0]$}
    is represented by the left quotient $G\backslash\hat G(A^0)$.
  \end{cor}
\end{lem}
\begin{proof}
  See
  \begin{itemize}
    \item \cite[6]{thboalch}: in the case $G_0(A^0)=T$
  \end{itemize}
\end{proof}

The group $G_0(A^0)$ is easy to compute and is often trivial. In fact, the
elements are block-diagonal, see \cite[77]{Loday2014}.
Thus the structure of ${}^0C(A^0)$ is easily deduced from the structure of
$G\backslash\hat G(A^0)$.

\begin{lem}
  The set $G\backslash\hat G(A^0)$ is canonically isomorphic to $\cH(A^0)$.
\end{lem}
\begin{proof}
  See \cite{thboalch}: Lemma 1.17
\end{proof}

