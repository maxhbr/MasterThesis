\chapter{Meromorphe Zusammenhänge}
\begin{comment}
Siehe:
\begin{itemize}
  \item \cite{sabbah2007isomonodromic}
\end{itemize}
\end{comment}
Let
\begin{itemize}
  \item $D=k_1(a_1)+\dots+k_m(a_m)$ be an \textbf{effective divisor} on $\P^1$
    so that
    \begin{itemize}
      \item $a_1,\dots,a_m\in\P^1$ are distinct points,
      \item $k_1,\dots,k_m>0$ positive integers
    \end{itemize}
    and
  \item let $V\to\P^1$ be a rank $n$ vector bundle.
\end{itemize}
\begin{defn}[2.1]
  A \emph{meromorphic connection} $\nabla$ on $V$ with poles on $D$ is
  \begin{itemize}
    \item a map $\nabla:V\to V\times K(D)$ where
      \begin{itemize}
        \item from the sheaf of holomorphic sections of $V$
        \item to the sheaf of sections of $V\otimes K(D)$
      \end{itemize}
    \item satisfying the Leibnitz rule
      \begin{equation}
        \nabla(fv)=(df)\otimes v + f\nabla v
      \end{equation}
      where
      \begin{itemize}
        \item $v$ is a local section of $V$,
        \item $f$ is a local holomorphic function and
        \item $K$ is the sheaf or holomorphic one-forms on $\P^1$
      \end{itemize}
  \end{itemize}
\end{defn}

\begin{paracol}{2}\sloppy
\switchcolumn[0]\noindent
  If we choose a local coordinate $z$ on $\P^1$ vanishing at $a_i$ then in
  terms of local trivialization of $V$, $\nabla$ has the form
  $\nabla=d-A=d-{}^iA$\footnote{the presuperscript is used to signify local
  information} where
  \[
    A=\left(\sum^{0}_{j=k_i}A_j\frac{dz}{z^{j}}\right)+A_0dz+\dots
  \]
  is a matrix of meromorphic one-forms and $A_j\in\End(\C^n)$.
\switchcolumn[1]\noindent
  \begin{defn}
    \cite[Def 1.5]{thboalch}.
    A \emph{germ of a meromorphic linear differential system}\footnote{or just
    \emph{system} from now on} (of rank $n$) is
    \begin{itemize}
      \item a germ of a meromorphic connection
        \begin{itemize}
          \item on the trivial vector bundle with fibre $E$.
        \end{itemize}
    \end{itemize}
  \end{defn}
\end{paracol}
\begin{paracol}{2}\sloppy
\switchcolumn[0]\noindent
\begin{defn}
  A \emph{nice meromorphic connection} is\dots 
\end{defn}
\switchcolumn[1]\noindent
  \begin{defn}
    A \emph{nice formal normal form} $A^0$ is
    \begin{itemize}
      \item a nice diagonal system with no holomorphic part.
    \end{itemize}
    \begin{rem}
      Such $A^0$ can be uniquely written as
      \[
      A^0:= dQ + \Lambda \frac{dz}{z},
      \]
      where
      \begin{itemize}
        \item $Q := \diag(q1,\dots ,qn)$
          \begin{itemize}
            \item $q_1,\dots ,q_n\in z^{−1}C[z^{−1}]$ are diagonal polynomials
              \begin{itemize}
                \item of degree $k-1$ in $z^{−1}$ with no constant term
              \end{itemize}
          \end{itemize}
          and
        \item $\Lambda = Res_0(A_0)$ is a constant diagonal matrix.
      \end{itemize}
    \end{rem}
  \end{defn}
\end{paracol}
\begin{paracol}{2}\sloppy
\switchcolumn[0]\noindent
  % \begin{defn}[compatible framing]
  %   A \emph{compatible framing} at $a_i$ of a vector bundle $V$ with nice
  %   connection $\nabla$ is a choice of isomorphism $g$ between the fibre
  %   $V_{a_i}$ and $\C^n$ which is compatible with $\nabla$ in the sense that
  %   the leading term ${}^iA_{k_i}$ of $\nabla$ is diagonal in any local
  %   trivialisation of $V$ extending this isomorphism.
  % \end{defn}
  \begin{defn}
    A \emph{compatible framing} at $a_i$ of a vector bundle $V$ with generic
    connection $\nabla$ is
    \begin{itemize}
      \item an isomorphism $g_0:V_{a_i}\to\C^n$
        \begin{itemize}
          \item between the firbre $V_{a_i}$ and $\C^n$
        \end{itemize}
        such that
        \begin{itemize}
          \item the leading coefficient of $\nabla$ is diagonal in any local
            trivialization of $V$ extending $g_0$.
        \end{itemize}
    \end{itemize}
    \begin{rem}
      Given a trivialization of $V$ in a neighbourhood of $a_i$ so that
      $\nabla=d-A$ as above, then
      \begin{itemize}
        \item a compatible framing is represented by a constant matrix
          $g_0\in G$ such that $g_0A_{k_i}g_0^{-1}$ is diagonal
      \end{itemize}
    \end{rem}
  \end{defn}
\switchcolumn[1]\noindent
  \begin{defn}[marked pair]
    A \emph{marked pair} is a pair $(A, \hat F)$ consisting of
    \begin{itemize}
      \item a nice system $A$ and
      \item a choice of formal isomorphism $\hat F \in \hat G$ such that
        $A = \hat F[A^0]$
    \end{itemize}
  \end{defn}
\end{paracol}

\begin{defn}[2.4]
  A connection $(V,\nabla)$ with compatible framing $g_0$ at $a_i$ has
  \emph{irregular type ${}^iA^0$} if
  \begin{itemize}
    \item $g_0$ extends to a formal trivialization of $V$ at $a_i$, in which
    $\nabla$ differs from $d-{}^iA^0$ by a matrix of one-forms witch just
    simple poles.
  \end{itemize}
\end{defn}

\subsection{Local moduli spaces} %{{{
\begin{defn}
  \begin{itemize}
    \item $\Syst(A^0):=\{d-A \mid A=\hat{F}[A^0] \text{ for some } \hat{F}\in
      G\llbracket z\rrbracket\}$\footnote{the set of germs at $0\in\C$ of
      meromorphic connections on the trivial rank $n$ vector bundle, that are
      formally equivalent to $d-A^0$.}
      where
      \begin{itemize}
        \item $A$ is a matrix of germs of meromorphic one-forms
        \item $\hat{F}[A^0]=(d\hat{F})\hat{F}^{-1}+\hat{F}A^0\hat{F}^{-1}$
        \item $G\llbracket z\rrbracket:=\GL_n(\C\llbracket z\rrbracket)$
          \begin{itemize}
            \item does not act on $\Syst(A^0)$
          \end{itemize}
      \end{itemize}
  \end{itemize}
  The group $G\{z\}:=\GL_n(\C\{z\})$ acts on $\Syst(A^0)$, we are interested
  in
  \begin{center}
    $\Syst(A^0)/G\{z\}$\footnote{The set of isomorphism classes of germs of
    meromorphic connections formally equivalent to $A^0$. Note that any
    generic connection is formally equivalent to some such $A^0$.}
  \end{center}
\end{defn}
\begin{rem}
  In the abelian and the simple pole case $\Syst(A^0)/G\{z\}$ is only a point.
\end{rem}
\begin{defn}
  \begin{itemize}
    \item $\widehat\Syst_{cf}(A^0)$~:\Leftrightarrow{} the set of compatibly
      framed connection germs with both irregular and formal type $A^0$.
    \item $\widehat\Syst_{mp}(A^0):=\{(A,\hat{F})\mid A\in\Syst(A^0)
      ,\hat{F}\in G\llbracket z\rrbracket
      ,A=\hat{F}[A^0]\}$
      the set of \emph{marked pairs}.
  \end{itemize}
\end{defn}
\begin{lem}
  There is a canonical isomorphism
  $\widehat\Syst_{cf}(A^0)\cong\widehat\Syst_{mp}(A^0)$\footnote{Let
  $\Syst(A^0)$ denote either of these two sets.}.
\end{lem}

$G\{z\}$ action on marked pairs: $g(A,\hat{F})=(g[A],g\circ\hat{F})$
\[
  \mathcal{H}(A^0):=\widehat\Syst(A^0)/G\{z\}
\]
\begin{comment}
  The actions of $T$ and $G\{z\}$ on $\Syst(A^0)$ commute so
  \[
    \Syst(A^0)/G\{z\}\cong\mathcal{H}(A^0)/T \,.
  \]
\end{comment}
%}}}

\subsection{Moduli spaces}
At each point $a_i$ choose
\begin{itemize}
  \item a germ $d-{}^iA^0$
    \begin{itemize}
      \item of a diagonal generic meromorphic connection on the trivial rank
      $n$ vector bundle
    \end{itemize}
    such that
    \begin{itemize}
      \item ${}^iA^0$ is a matrix of germs of meromorphic one-forms, which we
        require\footnote{without loss of generality} to be diagonal.
    \end{itemize}
\end{itemize}
If $z_i$ is a local coordinate vanishing at $a_i$, write
\begin{itemize}
  \item ${}^iA^0=d({}^iQ)+{}^i\Lambda^0\frac{dz}{z}$ where
    \begin{itemize}
      \item ${}^i\Lambda^0$ is constant diagonal
      \item ${}^iQ=diag(q_1,\dots,q_n)$ diagonal matrix of meromorphic
        functions
    \end{itemize}
\end{itemize}
Let  $\textbf{a}$ denote the choice of
\begin{itemize}
  \item the effectrive divisor $D$ and
  \item all the germs ${}^iA^0$
\end{itemize}
\begin{defn}[2.5]
  The \emph{moduli space $\cM^*(\textbf{a})$
  \textcolor{green!40!black}{($\cM(\textbf{a})$)}} is
  \begin{itemize}
    \item the set of isomorphism classes of pairs $(V,\nabla)$ where
      \begin{itemize}
        \item a trivial \textcolor{green!40!black}{(degree zero)} rank $n$
          holomorphic vector bundle $V$ over $\P^1$
        \item a meromorphic connection $\nabla$ (with poles on $D$) on $V$
          which is formally equivalent to $d-{}^iA^0$ at $a_i$ for each
          $i$\footnote{and has no other poles}
      \end{itemize}
  \end{itemize}
\end{defn}
\begin{defn}[2.6]
  The \emph{extended moduli space $\widetilde\cM^*(\textbf{a})$
  \textcolor{green!40!black}{($\widetilde\cM(\textbf{a})$)}} is
  \begin{itemize}
    \item the set of isomorphism classes of triples $(V,\nabla,\textbf{g})$
      where
      \begin{itemize}
        \item a trivial \textcolor{green!40!black}{(degree zero)}
          \textcolor{gray}{rank $n$} holomorphic vector bundle $V$ over $\P^1$
        \item a generic \textcolor{gray}{meromorphic} connection $\nabla$
          (with poles on $D$) on $V$
        \item compatible framins $\textbf{g}=({}^1g_0,\dots,{}^mg_0)$
      \end{itemize}
      such that $(V,\nabla,\textbf{g})$ has irregular type ${}^iA^0$ at each
      $a_i$
  \end{itemize}
\end{defn}
Since $\cM^*(\textbf{a})$ and $\widetilde\cM^*(\textbf{a})$ are moduli spaces
of connections on trivial bundles we can obtain explicit descriptions of them.
See \cite{thboalch} page 10ff.

\section{\cite{sabbah2007isomonodromic}} %{{{
\begin{defn}[Holomorphic budles (0.3.?)]
  Let
  \begin{itemize}
    \item $\pi:E\to M$ be a holomorphic mapping between two complex analytic
      manifolds.
  \end{itemize}
  We will say that
  \begin{itemize}
    \item $\pi$ is a \emph{vector fibration of rank $d$}, or
    \item $\pi$ makes $E$ a \emph{vector bundle of rank $d$ on $M$}
  \end{itemize}
  if there exists a open covering\dots
\end{defn}
Set $\sE(U)$ as the set of holomorphic sections, where
\begin{itemize}
  \item a \emph{holomorphic section} of $E$ on $U$ is a holomorphic mapping
    $\sigma:U\to E$ which
    \begin{itemize}
      \item is a section of the projection, i.e., which satisfies
        $\pi\circ\sigma=\Id_U$.
    \end{itemize}
\end{itemize}
This defines a sheaf $\sE$ of modules over $\cO_M(U)$.
\begin{defn}[Meromorphic bundles (0.8.?)]
  A \emph{meromorphic bundle on $M$ with poles along $Z$} is a locally free
  sheaf of $\cO_M(*Z)$-modules of finite rank
  where 
  \begin{itemize}
    \item $Z$ is a smooth hypersurface in a complex analytic manifold $M$
    \item define the sheaf $\cO_M(*Z)$ by
      \begin{itemize}
        \item $U\mapsto$ functions which are \emph{meromorphic along $Z\cap U$}
          i.e.\ which
          \begin{itemize}
            \item are holomorphic on $U\backslash Z\cap U$ and
            \item for any chart $V$ of $M$
              \begin{itemize}
                \item contained in $U$ and
                \item in which $Z\cap V$ is defined by the vanishing of some
                  coordinate $z_1$
              \end{itemize}
              there exists
              \begin{itemize}
                \item an integer $m$
              \end{itemize}
              such that 
              \begin{itemize}
                \item $z_1^mf(z_1\dots z_n)$ is locally bounded in the
                  neighbourhood of any point of $Z\cap V$.
              \end{itemize}
          \end{itemize}
      \end{itemize}
  \end{itemize}
\end{defn}
\begin{defn}[Holomorphic connections (0.11.1)]
  A \emph{holomorphic connection} $\nabla$ on a holomorphic vector bundle
  $\pi:E\to M$ is a $\C$-linear homomorphism of sheaves
  \[
    \nabla:\sE\to\Omega_M^1\otimes_{\cO_M}\sE
  \]
  satisfying
  \begin{itemize}
    \item for
      \begin{itemize}
        \item any open set $U$ of $M$,
        \item any section $s\in\Gamma(U,\sE)$ and
        \item any holomorphic function $f\in\cO(U)$
      \end{itemize}
      the \emph{Leibnitz rule}:
      \[
        \nabla(f\cdot s)=\nabla(s)+df\otimes
        s\in\Gamma(U,\Omega_M^1\otimes_{\cO_M}\sE)
      \]
  \end{itemize}
\end{defn}
\begin{defn}[Meromorphic connections]
  A \emph{connection on $\sM$} is defined as a $\C$-linear homomorphism
  \[
    \nabla:\sM\to\Omega_M^1\otimes_{\cO_M}\sM
  \]
  satisfying the Leibnitz rule.
\end{defn}
\begin{defn}[Flatness (0.12.2)]
  The connection $\nabla:\cE\to \Omega_M^1\otimes_{\cO_M}\cE$ is said to be
  \emph{integrable} or \emph{flat}, if
  \begin{itemize}
    \item its curvature $R_\nabla\equiv0$

    where
    \begin{itemize}
      \item $R_\nabla:=\nabla\circ\nabla:\cE\to\Omega_M^2\otimes_{\cO_M}\cE$
        is a $\cO_M$-linear morphism.
    \end{itemize}
  \end{itemize}
  \begin{prop}[0.12.4]
    The connection $\nabla$ is flat if and only if, in any local basis $e$ of
    $\cE$, the connection matrix $\Omega$ satisfies
    \[
      d\omega + \omega \wedge \omega = 0.
    \]
  \end{prop}
  We will say that a connection on a meromorphic bundle is \emph{integrable} or
  \emph{flat} if its restriction to $M\backslash Z$ is an integrable connection
  on the holomorphic bundle $\sM_{|M\backslash Z}$.
\end{defn}

\subsection{Models and formal decomposition of a germ}
Let $\textbf{k}$ denote $\C\{t\}[1/t]$.
\subsubsection{Regular singularities}
\begin{defn}
  An \emph{elementary regular model} is
  \begin{itemize}
    \item a $(\textbf{k},\nabla)$-vector space
      equipped with a basis in which
      \begin{itemize}
        \item the connection matrix is written as
          $\Omega(t)=(\alpha\Id+N)\frac{dt}{t}$
          where
          \begin{itemize}
            \item $\alpha\in\C$ and
            \item $N$ is a nilpotent matrix
          \end{itemize}
      \end{itemize}
  \end{itemize}
\end{defn}
\begin{cor}[II.2.9]
  Any $(\textbf{k},\nabla)$-vector space with regular singularity is isomorphic
  to a direct sum of elementary regular models.
\end{cor}
\begin{prop}[Any formal solution is convergent(II.2.18)]
  Let $A(t)$ be a matrix in $M_d(\C\{t\})$. Any vector $u(t)$ with entries in
  $\C\llbracket t\rrbracket$ which is solution of the system
  $tu'(t)+A(t)u(t)=0$ has converging entries.
\end{prop}
\subsubsection{Irregular singularities}
\begin{defn}
  Let $(\cM,\nabla)$ be a \textbf{germ} of a meromorphic bundle with
  connection.
  \begin{itemize}
    \item A germ $(\cM,\nabla)$ is \emph{elementary} if it is isomorphic to
      some germ
      \[
        (\cE^\phi,\nabla)\otimes(\cR,\nabla)
      \]
      where
      \begin{itemize}
        \item $(\cR,\nabla)$ ha regular singularity along $\{0\}\times X$
      \end{itemize}
    \item $(\cM,\nabla)$ is a \emph{model} if it is isomorphic to a direct sum
      of elementary models, written as
      \[
        \bigoplus_\phi(\cE^\phi\otimes\cR_\phi)
      \]
      where we assume that
      \begin{itemize}
        \item the meromorphic bundles with connection $\cR_\phi$ have regular
          singularity and
        \item the
          $\phi\in\C\{t\textcolor{gray}{\underset{\text{parameter}}
            {\underbrace{,x_1,\ldots,x_n}}}\}[t−1]$
          \begin{itemize}
            \item have no holomorphic part and
            \item are pairwise distinct.
          \end{itemize}
      \end{itemize}
    \item We will say that a model is \emph{good} if,
      \begin{itemize}
        \item for all $\phi\neq\psi$
          \begin{itemize}
            \item such that $\cR_\phi$, $\cR_\psi$ are nonzero,
          \end{itemize}
          the order of the pole along $t=0$ of $(\phi-\psi)(t,x)$ does not
          depend on $x$ being in some neighbourhood of $x^o$.
      \end{itemize}
  \end{itemize}
\end{defn}
\begin{thm}[Formal decomposition (II.5.7)]
  Let $(\cM,\nabla)$ be a germ of meromorphic bundle with connection,
  \begin{itemize}
    \item equipped with a basis
      \begin{itemize}
        \item
          in which the matrix $\Omega$ takes the form
          \[
            \Omega=t^{-r}\left[A(t,x)\frac{dt}{t}
              +\sum_{i=1}^nC^{(i)}(t,x)dx_i\right]
          \]
        with
        \begin{itemize}
          \item $r\geq1$
          \item $A$ and the $C^{(i)}$ having holomorphic entries, and
          \item $A_0:=A(0,x^0)$ being regular semisimple\footnote{i.e.\ with
            pairwise distinct eigenvalues.}
        \end{itemize}
      \end{itemize}
  \end{itemize}
  Then there exist
  \begin{itemize}
    \item a good model $(\cM^{good},\nabla)
      =\left(\bigoplus_\phi(\cE^\phi\otimes\cR_\phi) \right)$ and
    \item a \textcolor{red!40!black}{`formal'} isomorphism
      \[
        \textcolor{red!40!black}{\hat\cO_{D\times X,x_0}\otimes}(\cM,\nabla)
        \overset{\sim}{\longrightarrow}
      \textcolor{red!40!black}{\hat\cO_{D\times X,x_0}\otimes}
        \left(\bigoplus_\phi(\cE^\phi\otimes\cR_\phi) \right)
      \]
  \end{itemize}
\end{thm}
\begin{comment}
  \begin{rem}
    \begin{itemize}
      \item In the present situation, no ramification is needed.
        \TODO[Why?]
      \item Moreover, we will see that all the components $\cR^\phi$ occurring
        in the model have rank one, which is not the case in general, even when
        no ramification is needed.
    \end{itemize}
  \end{rem}
\end{comment}
%}}}

% vim:set ft=tex foldmethod=marker foldmarker=%{{{,%}}}:
