\chapter{Meromorphic connections}
\begin{comment}
  \begin{multicols}{2}
    \textbf{Global}
    \begin{itemize}
      \item Meromorphic connection
      \item $\cD$-modules (global)
    \end{itemize}
    \columnbreak
    \textbf{Local}
    \begin{itemize}
      \item Germ of a meromorphic connection
      \item $\cD$-modules (local)
      \item System
        \begin{itemize}
          \item coordinate dependent
        \end{itemize}
      \item Conection matrix
        \begin{itemize}
          \item coordinate dependent
        \end{itemize}
    \end{itemize}
  \end{multicols}
\end{comment}
There are multiple languages, which can be used for talking about meromorphic
connections. \rewrite{Firstly} the languages of meromorphic connections, and also
$\cD$-modules\TODO[global].
These two can be used to talk about global information.
For local description one can use germs of meromorphic connections or
$\cD$-modules\TODO[local] which are coordinate independent. Coordinate
dependent alternatives are (local) systems and connection matrices. Also,
\rewrite{one can use the fundamental solutions to describe local informations
of meromorphic connections.}

\begin{comment}
  Siehe:
  \begin{multicols}{3}
    \begin{itemize}
      \item \cite{boalch} and \cite{thboalch}
      \item \cite{sabbah2007isomonodromic}
      \item \cite{Varadarajan96linearmeromorphic}
      \item \textbf{\cite[Chap.5]{hotta2008}}
    \end{itemize}
    \columnbreak
    Differential modules:
    \begin{itemize}
      \item \cite{Loday1994}
      \item \cite{Loday2014}
    \end{itemize}
  \end{multicols}
\end{comment}

Let $M$ be a riemanian surface and let $Z=k_1(a_1)+\cdots+k_m(a_m)>0$ be an
effective divisor\footnote{The $a_i$ are distinct points and the $k_i$ are
positive integers.} on $M$.
It is sufficient to think $M=\P^1$ and $Z=1(0)+1(\infty)$\TODO[\infty?], since
we will only be interested in local information (at $0$).

Let $\sM$ be a holomorphic Bundle over $M$ i.e.\ a locally free $\cO_M$-module
of rank $n$.

A meromorphic connection is then defined as follows.
\begin{defn}
  \def\myU{\textcolor{green!30!black}{U}}
  \def\mys{\textcolor{blue!60!black}{s}}
  \def\myf{\textcolor{red!60!black}{f}}
  A \emph{meromorphic connection $(\sM,\nabla)$ on $\sM$ with poles on $Z$}
  is defined by a $\C$-linear morphism of sheaves
  \[
    \nabla:\sM\to\Omega_M^1(*Z)\otimes\sM
  \]
  satisfying, for each $\myU\underset{\text{op.}}{\subset} M$, the
  \emph{Leibniz rule}
    \[
      \nabla(\myf\mys)=\myf\nabla\mys+(d\myf)\otimes\mys
    \]
  for $\mys\textcolor{blue!60!black}{\in\Gamma(\myU,\sM)}$ and
  $\myf\textcolor{red!60!black}{\in\cO_M(\myU)}$.
  The rank of the Bundle $\sM$ is the \emph{rank} of the meromorphic connection
  $(\sM,\nabla)$.
  \begin{s-rem}
    Some authors use the factors $k_i$ of the divisor $Z$ to limit the pole
    orders at the points $a_i$. Since we do not need this restriction, we allow
    arbitrary pole orders. Denoted is this by the $*$ in $\Omega_M^1(*Z)$.
    A definition of $\Omega_M^1(*Z)$ of $\cO_M(*Z)$ can be found in
    \cite[Sec.0.8]{sabbah2007isomonodromic} and 
    $\Omega_M^1(*Z)$ is then defined as
    \[
      \Omega_M^1(*Z):=\cO_M(*Z)\underset{\cO_M}\otimes\Omega_M^1
    \]
    the \emph{sheaf of differential $1$-forms} (cf.\
    \cite[Sec.0.9.b]{sabbah2007isomonodromic}).
  \end{s-rem}
\end{defn}
\begin{rem}
  \begin{enumerate}
    \item We will occasionally omit the $\nabla$ and simply call $\sM$ the
      meromorphic connection.
    \item Here, the variant `holomorphic bundle with meromorphic connection' is
      chosen, like in \cite{boalch}.
      There is also the twisted\TODO[better word?] description `meromorphic
      bundle with holomorphic connection' which is used in
      \cite{sabbah2007isomonodromic}.
      \\ By choosing a lattice of a meromorphic bundle, one gets a holomorphic
      bundle but the connection is no longer holomorphic. Such that we obtain a
      meromorphic connection in our\dots\TODO
  \end{enumerate}
\end{rem}

\begin{defn}
  \marginnote{\cite[0.12.2]{sabbah2007isomonodromic}}
  The connection $\nabla:\cM\to \Omega_M^1\otimes_{\cO_M}\cM$ is said to be
  \emph{integrable} or \emph{flat}, if
  \begin{itemize}
    \item its curvature $R_\nabla\equiv0$
    where
    \begin{itemize}
      \item $R_\nabla:=\nabla\circ\nabla:\cE\to\Omega_M^2\otimes_{\cO_M}\cE$
        is a $\cO_M$-linear morphism.
    \end{itemize}
  \end{itemize}
  \begin{s-prop}
    \marginnote{\cite[0.12.4]{sabbah2007isomonodromic}}
    The connection $\nabla$ is flat if and only if, in any local basis $e$ of
    $\cM$, the connection matrix $\Omega$ satisfies
    \[
      d\omega + \omega \wedge \omega = 0.
    \]
    \begin{comment}
      This condition is sufficient to assure the existence of local
      fundamental solutions.
    \end{comment}
  \end{s-prop}
  \begin{s-rem}
    Here are all connections flat, since \TODO[dim $1$?]
  \end{s-rem}
  \begin{comment}
    We will say that a connection on a meromorphic bundle is \emph{integrable}
    or \emph{flat} if its restriction to $M\backslash Z$ is an integrable
    connection on the holomorphic bundle $\sM_{|M\backslash Z}$.
  \end{comment}
\end{defn}

%%%%%%%%%%%%%%%%%%%%%%%%%%%%%%%%%%%%%%%%%%%%%%%%%%%%%%%%%%%%%%%%%%%%%%%%%%%%%%%
\section{Local expression of meromorphic connections}
\begin{comment}
  \cite[28]{sabbah2007isomonodromic}, \cite[2]{thboalch} and
  \cite[11]{babbitt1989local}
\end{comment}
We will usually\TODO[only?] be interested in local information of a meromorphic
connection.  This means, that we look at a connection in a neighbourhood of $0$
and allow only one singularity at $0$.
There are many ways of expressing the local information, we will either talk
about germs of meromorphic connections or systems, which are coordinate-
and trivialization-dependent.
\begin{prop}
  \marginnote{\textbf{\cite[Rem.5.2.4]{hotta2008}}\\\cite[Def.4.2.1]{Loday2014}}
  A germ of a meromorphic connection $(\sM,\nabla)$ is just the sheaf-theoretic
  germ (at $t=0$), thus has the form $(\cM,\nabla)$ where
  \begin{itemize}
    \item $\cM$ is the germ at $0$ of the holomorphic bundle $\sM$ and thus a
      $\C(\!\{t\}\!)$-vectorspace of dimension $n$, since the ring of germs of
      meromorphic functions with poles at $0$ is the ring $\C(\!\{t\}\!)$, and
    \item $\nabla:\cM\to \cM$ \TODO[one forms?] is a additive map, which
      satisfies the \emph{Leibniz rule}
      \[
        \nabla(fm)=\partial f\cdot m + f\nabla(m)
      \]
      for all $f\in\C(\!\{t\}\!)$ and $m\in \cM$.
  \end{itemize}
  \begin{s-rem}
    \marginnote{\cite{sabbah2007isomonodromic}}
    It is a $(\C(\!\{t\}\!),\nabla)$-vectorspace.
  \end{s-rem}
  \begin{comment}
    \begin{s-rem}
      Loday-Richaud calls this in \cite[Def.4.2.1]{Loday2014} a
      \emph{differential module}.
    \end{s-rem}
  \end{comment}
\end{prop}
\begin{defn}
  A \emph{(iso-)\,morphism of meromorphic connections}
  $\Phi:(\cM,\nabla)\to(\cM',\nabla')$ is a (iso-)\,morphism of
  $\C(\!\{t\}\!)$-vectorspaces $\Phi:\cM\to\cM'$ which commutes with the
  connections.
  \TODO[Category of systems!]
\end{defn}
\marginnote{\cite[65]{Loday2014}}
\begin{multicols}{2}
%   % Choose a $\C(\!\{t\}\!)$-basis $\underline{e}=(e_1,e_2,\dots,e_n)$ of $\cM$
%   % and let $A\in\End(E)\otimes\C(\!\{t\}\!)$\TODO[In one forms? $A=A'dt$?] be a
%   % matrix, such that
%   % \[
%   %   (\epsilon_1,\epsilon_2,\dots,\epsilon_n)=-(e_1,e_2,\dots,e_n)A
%   % \]
%   % is the image of $\nabla$.
%   \textcolor{gray}{Choose a $\C(\!\{t\}\!)$-basis
%     $\underline{e}=(e_1,e_2,\dots,e_n)$ of $\cM$.  Let $A'$ be a $n\times n$
%     matrix with entries in $\C(\!\{t\}\!)$ such that
%     \[
%       \nabla\left(e_1,\dots,e_n\right)
%       =
%       -(e_1,e_2,\dots,e_n)A'
%     \]
%     and let $A=A'dt$.}
  \textcolor{gray}{%
    Choose a $\C(\!\{t\}\!)$-basis $\underline{e}=(e_1,e_2,\dots,e_n)$ of $\cM$.
    Let $A'$ be a $n\times n$ matrix with entries in $\C(\!\{t\}\!)$
    \rewrite{such that the matrix $A=A'dt$ describes the action of $\nabla$:}
    \[
      \nabla\left(e_1,\dots,e_n\right)
      =
      -(e_1,e_2,\dots,e_n)A \,.
    \]
  }

\columnbreak

  Choose a $\C(\!\{t\}\!)$-basis $\underline{e}=(e_1,e_2,\dots,e_n)$ of $\cM$.
  Let $A=(a_{jk})$ be a $n\times n$ matrix with entries in $\C(\!\{t\}\!)$
  \rewrite{such that the it describes the action of $\nabla$:}
  \[
    \nabla e_k
    =
    -\sum_{1\leq j\leq n} a_{jk}(t)e_j \,.
  \]
\end{multicols}
Let $x=\sum_{0\leq j\leq n}x_je_j$ be an arbitrary element of $\cM$ which is in
matrix notation described as $x=\underline{e}\cdot X$ with the column matrix
$X={}^t\!(x_1,x_2 ,\dots,x_n)$.
Then, applying the Leibniz rule, yields
\begin{align*}
  \nabla x&=\nabla\left(\underline{e}\cdot X\right)
  \\&=\underline{e} \cdot dX - \underline{e}\cdot \nabla X
  \\&=\underline{e}\left(dX-AX\right).
\end{align*}
Hence the condition $\nabla x=0$ is equivalent to the system of ODEs
\begin{equation} \label{eq:ode}
  \frac{d}{dt}x=Ax
\end{equation}
such that horizontal sections of $(\cM,\nabla)$ correspond to solutions of
(\ref{eq:ode}).
Thus, with the connection $\nabla$ and the $K$-basis $\underline{e}$ is
naturally associate the differential operator $\triangle=d-A$, which has order
one and dimension $n$.
\TODO[possibly multivalued solutions ($\tilde K$)?~\cite{hotta2008} on page
128]

\begin{comment}
  This means that the connection $\nabla$ is fully determined by the matrix $A$
  and thus is fully determined by $A'$.
\end{comment}
\begin{defn}
  \begin{enumerate}
    \item This matrix $A$ is called a \emph{connection matrix} of
      $(\cM,\nabla)$. It depends on the choice of the $\C(\!\{t\}\!)$-basis.
      \begin{s-rem}
        We have seen above, that a connection is fully determined by its
        connection matrix.
      \end{s-rem}
    \item A \emph{germ of a meromorphic linear differential system} of rank
      $n$, or just a \emph{system}, is a germ of a meromorphic connection on
      the trivial vector bundle \textbf{with a chosen trivialization}
      \rewrite{of the fiber $E\cong\C^n$}.
      \begin{s-prop}
        Thus, the set of systems is isomorphic to the set
        $A\in\End(E)\otimes\C(\!\{t\}\!)=\gl_n(\C(\!\{t\}\!))$ of all
        connection matrices.
      \end{s-prop}
      It well be denoted by $[A]=d-A$ where $A$ is the connection matrix in the
      chosen trivialization.
  \end{enumerate}
\end{defn}

\begin{comment}
  \cite{boalch} wants \textbf{generic} meromorphic connections
  \begin{itemize}
    \item\dots simplest jet sufficient\dots
  \end{itemize}
\end{comment}

%%%%%%%%%%%%%%%%%%%%%%%%%%%%%%%%%%%%%%%%%%%%%%%%%%%%%%%%%%%%%%%%%%%%%%%%%%%%%%%
\subsubsection{System \rightarrow{} germ of a meromorphic connection}
\begin{prop}
  If we start with a system $[A]$ of rank $n$, or a connection matrix $A$, we
  get a germ of a meromorphic connection via
  \[
    (\cM_A,\nabla_A)=(\C(\!\{t\}\!)^n,d-A)
  \]
  and $A$ is a connection matrix for $(\cM,\nabla)$.
\end{prop}
\begin{proof}
  \TODO{}
\end{proof}

%%%%%%%%%%%%%%%%%%%%%%%%%%%%%%%%%%%%%%%%%%%%%%%%%%%%%%%%%%%%%%%%%%%%%%%%%%%%%%%
\subsubsection{Local \rightarrow{} global}
\begin{comment}
  Maybe see: \cite[Thm.3.3.1]{sibuya1990Linear}: G. D. Birkhoff
\end{comment}
\begin{thm}
  \begin{comment}
    Quelle?
    \begin{itemize}
      \item \cite{sabbah2007isomonodromic}???
    \end{itemize}
  \end{comment}
  If we start with a germ $(\cM,\nabla)$ of a meromorphic connection there is
  a unique meromorphic connection $(\sM,\nabla)$ such that
  \begin{itemize}
    \item $(\sM,\nabla)$ has only singularities at $0$ and $\infty$
    \item the singularity at $\infty$ is only \TODO{} and
    \item $(\cM,\nabla)$ is the germ at $0$ of $(\sM,\nabla)$.
  \end{itemize}
\end{thm}
\begin{proof}
  \TODO{}
\end{proof}

%%%%%%%%%%%%%%%%%%%%%%%%%%%%%%%%%%%%%%%%%%%%%%%%%%%%%%%%%%%%%%%%%%%%%%%%%%%%%%%
\subsubsection{Formalization}
\begin{multicols}{2}
  Let $[A]$ be a system. We \rewrite{view it as a formal system, by} allowing
  formal solutions.
  \TODO{}

\columnbreak

  Let $(\cM,\nabla)$ be a meromorphic connection. The connection $\nabla$
  naturally extends to $\hat\cM:=\cM\otimes\C(\!(t)\!)$ and
  $\tilde\cM_\theta:=\cM\otimes\cA_\theta$.
  \TODO{}
\end{multicols}

%%%%%%%%%%%%%%%%%%%%%%%%%%%%%%%%%%%%%%%%%%%%%%%%%%%%%%%%%%%%%%%%%%%%%%%%%%%%%%%
\subsubsection{Fundamental solution}
\begin{comment}
  See \cite[Sec.4.3.2]{Loday2014}
\end{comment}
Fundamental solutions \rewrite{beschreiben} also systems of ordinary
differential equations.

\begin{defn}
  A \emph{fundamental solution} $\cY$ of the system $[A]$ is a $n\times n$
  matrix \comm{with entries in \TODO{}} whose columns are $n$ $\C$-linearly
  independent solutions of the system $[A]$.
\end{defn}

\begin{rem}
  If the trivialization is changed by $F$ (resp.\ $\hat F$) the fundamental
  solution $\cY\in G(\!\{t\}\!)$ changes to $F\cY$ (resp.\ $\hat F\cY$).
\end{rem}

\begin{comment}
  Unique \textbf{up to permutation}?\ or up to basis change?
\end{comment}

%%%%%%%%%%%%%%%%%%%%%%%%%%%%%%%%%%%%%%%%%%%%%%%%%%%%%%%%%%%%%%%%%%%%%%%%%%%%%%%
\subsubsection{As differential operator}
\begin{comment}
  \begin{itemize}
    \item \cite[Sec.4.2]{Loday2014}
  \end{itemize}
\end{comment}

From the theory of ordinary differential equations we know that to
(\ref{eq:ode}) there is a equivalent ordinary differential equation of order
$n$ which can be written as
\[
  \underset{=:P}{\underbrace{%
      (a_n\partial_t^n+a_{n-1}\partial_t^{n-1}+\cdots a_{1}\partial_t+a_{0})
  }} \cdot v=0
\]
where $a_i\in\C(\!\{t\}\!)$. \TODO{}

\begin{comment}
  See \cite[Sec.1.4]{babbitt1983} for \textbf{ode of rank $n$} to
  \textbf{system}.
\end{comment}

%%%%%%%%%%%%%%%%%%%%%%%%%%%%%%%%%%%%%%%%%%%%%%%%%%%%%%%%%%%%%%%%%%%%%%%%%%%%%%%
\subsubsection{As $\cD$-module}
\begin{comment}
  \cite[Sec.4.2.2]{Loday2014}
\end{comment}

In the other direction, from $\cD$-modules to meromorphic connections, there is
the lemma of the cyclic vector. \TODO{}
\begin{comment}
  \begin{itemize}
    \item \cite[Rem.4.2.6]{Loday2014}
  \end{itemize}
\end{comment}

%%%%%%%%%%%%%%%%%%%%%%%%%%%%%%%%%%%%%%%%%%%%%%%%%%%%%%%%%%%%%%%%%%%%%%%%%%%%%%%
\subsection{Transformation of systems}
\begin{comment}\footnotesize
  see \cite{thboalch} \textbf{Rem 1.41 on p. 16}:
  \begin{rem}
    Note that in most of the recent references we have used, Stokes matrices
    are used to classify
    \begin{itemize}
      \item meromorphic connections within fixed \textbf{formal meromorphic
        classes, modulo meromorphic equivalence}.
    \end{itemize}
    Whereas here we classify
    \begin{itemize}
      \item meromorphic connections within fixed \textbf{formal analytic
        classes, modulo analytic equivalence},
    \end{itemize}
    as is done in the older literature.  The fact is that the sets equivalence
    classes are the same in both cases. \comm{It is important for us to work
      with analytic, rather than meromorphic gauge transformations, because
    then the $\C^\infty$ viewpoint in Chapter 3 is cleaner.} This distinction
    relates to the difference between \textbf{‘regular singular’} connections
    and \textbf{‘logarithmic’} connections.
  \end{rem}
\end{comment}
\begin{notations}
  We will use the following notations
  \begin{itemize}
    \item $G=\Gl_n(\C)$;
    \item $G[t]=\Gl_n(\C[t])$;
    \item $G\{t\}=\Gl_n(\C\{t\})$ analytic transformations;
    \item $G(\!\{t\}\!)=\Gl_n\left(\C\{t\}[t^{-1}]\right)$ meromorphic
      transformations;
    \item $G\llbracket t\rrbracket=\Gl_n\left(\C\llbracket t\rrbracket\right)$
      (maybe not applicable) formal transformations;
    \item $G(\!(t)\!)=\Gl_n\left(\C\llbracket t\rrbracket[t^{-1}]\right)$
      (maybe not applicable) formal meromorphic transformations.
  \end{itemize}
  \begin{comment}
    We will always use the meromorphic ones, in contrast
    to~\cite{boalch,thboalch} where analytic classification is used.
  \end{comment}
\end{notations}
\marginnote{\cite[Sec.4.3.1]{Loday2014}}
By \emph{meromorphic\footnote{We use the term meromorphic in the sens of
convergent meromorphic. Otherwise we specify formal meromorphic or simply
formal.} transformation}, or just \emph{transformation}, of a system we mean a
$\C(\!\{t\}\!)$-linear change of the trivialization.  Such a change is given by
a matrix $F\in G(\!\{t\}\!)$ and the transformed connection matrix ${}^F\!A$ is
obtained through
\[
  {}^F\!A=(dF)F^{-1} + FAF^{-1} \,.
\]
If $F$ is formal i.e.\ $F\in G(\!(t)\!)$, it will usually be denoted by
$\hat F$.
The transformation of $A$ by $\hat F$ is not guaranteed to have convergent
entries.
We denote by $\hat G(A)$ the set of all \emph{(applicable) formal
transformations}
\[
  \hat G(A):=\left\{\hat F\in G(\!(t)\!)
    \mid {}^{\hat F}\!A \text{ has convergent entries i.e.\ }
    {}^{\hat F}\!A\in G(\!\{t\}\!)
  \right\}\,.
\]
\begin{rem}
  \def\myB{\textcolor{blue!60!black}{B}}
  \def\myA{\textcolor{green!30!black}{A}}
  \def\myF{\textcolor{red!60!black}{F}}
  The condition
  \begin{itemize}
    \item[] $\myB$ is obtained from $\myA$ by transformation $\myF$
  \end{itemize}
  is clearly equivalent to
  \begin{itemize}
    \item[]  $\myF$ solves the linear differential system
      \[
        \frac{d\myF}{dt}=\myB\myF-\myF\myA
      \]
      which is denoted by $[\myA,\myB]$.
  \end{itemize}
  \begin{comment}
    \begin{s-rem}
      $[A]=[0,A]$ \TODO[korrekt?]
    \end{s-rem}
  \end{comment}
\end{rem}
\begin{defn}
  We define the \emph{(formal) equivalence relation on the connection matrices}
  as
  \begin{einr}
    \textbf{\boldmath$A$ is (formally) equivalent to $B$}
  \end{einr}
  if and only if
  \begin{einr}
    \textbf{\boldmath$B$ is obtained from $A$ by (formal) transformation}.
  \end{einr}
  The \emph{class of a connection matrix} is the orbit under the gauge
  transformation in $G(\!\{t\}\!)$. The \emph{formal class} ist the orbit in
  $\hat G(A)$.
  \begin{s-rem}
    \begin{enumerate}
      \item Thus $A$ is (formally) equivalent to $B$ if and only if there is a
        (formal) solution of $[A,B]$.
      \item This implies also a equivalence relation and a classification on
        the systems.
    \end{enumerate}
  \end{s-rem}
\end{defn}

\begin{prop}
  Two germs of meromorphic connections are (formally) isomorphic if and only if
  their corresponding connection matrices are (formally) equivalent.
\end{prop}
\begin{proof}
  \TODO{}
\end{proof}

\begin{defn}
  \begin{itemize}
    \item An \emph{isotropy} of $A^0$ is a transformation $\hat F$ which
      satisfy ${}^{\hat F}\!A^0=A^0$.
      Thus, the isotropies are the solutions of the system
      $[\End A^0]:=[A^0,A^0]$.
      \marginnote{\cite[853]{Loday1994}}
    \item Let $G_0(A^0)$ denote the set of all isotropies of $A^0$.
      \begin{s-rem}
        They are, a priori, formal transformations. Actually $G(A^0)$ is a
        subgroup of $\Gl_n(\C[1/x,x])$.
      \end{s-rem}
      \begin{comment}
        In the nice case this is only $T$?
      \end{comment}
  \end{itemize}
\end{defn}
\begin{lem}
  \marginnote{\cite[854]{Loday1994}}
  Two formal transformations $\hat F_1$ and $\hat F_2$ take $A^0$ into
  equivalent matrices ${}^{\hat F_1}\!A^0$ and  ${}^{\hat F_2}\!A^0$ if and
  only if there exists $f_0\in G_0(A^0)$ such that $\hat F_1=\hat F_2f_0$.
\end{lem}
\begin{proof}
  \TODO{}
\end{proof}

%%%%%%%%%%%%%%%%%%%%%%%%%%%%%%%%%%%%%%%%%%%%%%%%%%%%%%%%%%%%%%%%%%%%%%%%%%%%%%%
\subsubsection{Regular / irregular singularities}
\begin{defn}
  \marginnote{\cite[86]{sabbah2007isomonodromic}}
  A connection with connection matrix $A$ has \emph{regular singularity} at $0$
  if there exists a konvergent transformation, by which $A$ is obtained from a
  matrix with at most a simple pole at $t=0$.
  Otherwise, the singularity is called \emph{irregular}.
  \begin{s-rem}
    This implies that, if $A$ has
    $\left\{\substack{\text{irregular}\\\text{regular}}\right\}$
    singularity, then also all
    meromorphic equivalent matrices ${}^{F}\!A$ have
    $\left\{\substack{\text{irregular}\\\text{regular}}\right\}$
    singularity.
  \end{s-rem}
\end{defn}
\begin{thm}
  \begin{comment}
    see
    \begin{itemize}
      \item \cite[Thm.II.2.8]{sabbah2007isomonodromic}
      \item \cite[5.1.2]{hotta2008}
    \end{itemize}
  \end{comment}
  Let $(\cM,\nabla)$ be a regular singular meromorphic connection and $A$ its
  connection matrix.
  Then there exists a matrix $F\in G(\!\{t\}\!)$ such that after transformation
  by $F$ the matrix $B={}^F\!A$ is constant i.e.\ $B\in G$.
\end{thm}

%%%%%%%%%%%%%%%%%%%%%%%%%%%%%%%%%%%%%%%%%%%%%%%%%%%%%%%%%%%%%%%%%%%%%%%%%%%%%%%
\subsection{Ramification}
\begin{comment}
  \cite[I.5.4.1]{sabbah_cimpa90}
\end{comment}

%%%%%%%%%%%%%%%%%%%%%%%%%%%%%%%%%%%%%%%%%%%%%%%%%%%%%%%%%%%%%%%%%%%%%%%%%%%%%%%
\section{Models / formal decomposition of a germ / formal classification}
\marginnote{\cite[Thm.4.3.1]{Loday2014}}
In every formal equivalence class of meromorphic connections, there are some
meromorphic connection of special form, which we will call models. They are not
unique but all of them, which are formally isomorphic to a given meromorphic
connection, lie in the same convergent equivalence class.
In fact, every element of this convergent equivalence class will be a model in
our definition.

The first part is given by the Levelt-Turittin theorem, which says, that
each meromorphic connection is, after potentially needed ramification, formally
isomorphic to such a model.
\TODO[Problems with ramification??]
Thus the Levelt-Turittin theorem solves the \emph{formal classification
problem}.
\begin{defn}
  \begin{itemize}
    \item For a $\phi\in\C\{t\}[t^{-1}]$ we use $\cE^{\phi}$ to denote the germ
      \[
        (\cE^{\phi},\nabla)=(\C\{t\},d-d\phi)\,.
      \]
      This is the system satisfied by the function $e^\phi$.
      \begin{s-cor}
        $\cE^\phi$ is determined by the class of $\phi$ in
        $\C\{t\}[t^{-1}]/\C\{t\}=t^{-1}\C[t^{-1}]$. In the following, we will
        only consider the unique ambassador $\phi$ in each class which has no
        holomorphic part.
      \end{s-cor}
    \item For $\alpha\in\C$, define the \emph{elementary regular meromorphic
      connection} $\cN_{\alpha,0}$ as the germ
      \[
        (\cN_{\alpha,0},\nabla)=(\C\{t\}[t^{-1}],d+\alpha dt/t)\,.
      \]
      This is the system satisfied by $t^\alpha$\TODO[?].
  \end{itemize}
\end{defn}
\begin{prop}
  \begin{enumerate}
    \item Every regular germ of a meromorphic connection is isomorphic to some
      direct sum
      \[
        (\cR,\nabla)=\bigoplus_\alpha(\cN_{\alpha,0},\nabla)\,.
      \]
      \begin{s-rem}
        For a detailed analysis of regular meromorphic connections
        see~\cite[Sec.II.2]{sabbah2007isomonodromic} or
        \cite[Sec.5.2]{hotta2008}.
      \end{s-rem}
    \item Every germ of a meromorphic connection of rank one is isomorphic to
      some germ
      \[
        (\cE^\phi,\nabla)\otimes(\cN_{\alpha,0},\nabla) \,.
      \]
    \item Two such germs corresponding to $(\phi_1,\alpha_1)$ and
      $(\phi_2,\alpha_2)$ are isomorphic if and only if
      \begin{itemize}
        \item $\phi_1-\phi_2$ has no pole and
        \item $\alpha_1-\alpha_2\in\Z$.
      \end{itemize}
  \end{enumerate}
\end{prop}
\begin{proof}
  See~\cite[Prop.II.5.1]{sabbah2007isomonodromic}
\end{proof}
\begin{defn}
  \marginnote{\cite[Def.II.5.2]{sabbah2007isomonodromic}}
  A germ $(\cM,\nabla)$ is called \emph{elementary} if it is isomorphic to
  some germ $(\cE^\phi,\nabla)\otimes(\cR,\nabla)$ where
  \begin{itemize}
    \item $(\cR,\nabla)$ has regular singularity at $\{0\}$ but has not to be
      of rank $1$, i.e.\ is isomorphic to a direct sum of regular elementary
      meromorphic connections.
  \end{itemize}
  \marginnote{\cite[II.2.f]{sabbah2007isomonodromic}}
\end{defn}
\begin{defn} \label{defn:model}
  \def\myPhi{\textcolor{red!60!black}{\phi}}
  \def\myE{\textcolor{green!40!black}{\cE^{\myPhi}}}
  \begin{multicols}{2}
    \textcolor{gray}{%
      A germ $(\cM,\nabla)$ is a \emph{model} or \emph{normal form} if there
      exists a isomorphism}

    \columnbreak{}

    A germ $(\cM',\nabla')$ is a \emph{model} or \emph{normal form}\footnote{If
    we talk about systems, we will prefer to use the term ``normal form''} if
    there exists, after ramification $\cM=\pi^*\cM'$ by $\pi$, a isomorphism
  \end{multicols}
  \begin{multicols}{2}
    \[
      \lambda:(\cM,\nabla)
      \overset{\cong}{\longrightarrow}
      % \cong
      \bigoplus_{~\tikzmark{e3}\!\!\myPhi}
      \underset{\text{merom. Zus.}}{%
        \underset{\text{elementare}}{%
          \underbrace{%
            \overset{\tikzmark{e2}}{\myE}
            \otimes
            \overset{\tikzmark{e1}}{\textcolor{blue!40!black}{\cR_{\myPhi}}}
          }
        }
      }\,.
    \]
    \columnbreak{}
    \begin{itemize}
      \item[\tikzmarkb{n2}{green}] is irregular singular
      \item[\tikzmarkc{n1}{blue}] has regular singularity at $\{0\}$
      \item[\tikzmarkc{n3}{red}] $\myPhi\in t^{-1}\C[t^{-1}]$ pairwise distinct
    \end{itemize}
    \begin{tikzpicture}[remember picture,overlay]
      \draw[->,blue!50!white,thick] (n1) to[out=180,in=70] (e1);
      \draw[->,green!40!black,thick] (n2) to[out=180,in=70] (e2);
      \draw[->,red!50!white,thick] (n3) to[out=205,in=-70] (e3);
    \end{tikzpicture}
  \end{multicols}
\end{defn}
The important theorem here is the Levelt-Turittin theorem, which solves the
formal classification problem.
\begin{thm}[Levelt-Turittin]
  To each germ $(\cM,\nabla)$ of a meromorphic connection there exists, after
  potentially needed \textcolor{blue!60!black}{pullback by some suitable
  ramification $t=z^q$ of order $q\geq1$}, a
  \textcolor{green!30!black}{\textbf{formal}} isomorphism
  \[
    \textcolor{green!30!black}{\hat{\textcolor{black}{\lambda}}}:
    \textcolor{blue!60!black}{\pi^{*}}
    \textcolor{green!30!black}{\hat{\textcolor{black}{\cM}}}
    \overset{\cong}\longrightarrow
    \textcolor{green!30!black}{\hat{\textcolor{black}{\cM}}^{nf}}
    :=\textcolor{green!30!black}{\hat\cO_M\otimes}\cM^{nf}
  \]
  to a model $\cM^{nf}$.
  We then call $\cM^{nf}$ a \emph{formal decomposition}, \emph{formal model} or
  \emph{formal normal form} of $\cM$.
\end{thm}
\begin{proof}
  See \TODO{}
\end{proof}
\TODO[Full set of formal invariants]
\begin{prop}
  \marginnote{This condition \textbf{might be} equivalent to the
    condition of being \textbf{nice} in~\cite{thboalch}.}
  Let $(\cM,\nabla)$ be a germ, equipped with a basis in which the matrix $A$
  takes the form
  \[
    A=t^{-r}A(t)\frac{dt}{t}
  \]
  with
  \begin{itemize}
    \item $r\geq1$,
    \item $A$ has holomorphic entries, and
    \item $A_0:=A(0)$ being regular semisimple\footnote{i.e.\ with
      pairwise distinct eigenvalues.}.
  \end{itemize}
  Then there is no ramification needed, to apply the Levelt-Turittin-theorem.
  \begin{comment}
    Further, all the summands $\cR_\phi$ have rank one, which is not the case
    in general.
  \end{comment}
\end{prop}
\begin{proof}
  See~\cite[Thm.II.5.7]{sabbah2007isomonodromic}.
\end{proof}

%%%%%%%%%%%%%%%%%%%%%%%%%%%%%%%%%%%%%%%%%%%%%%%%%%%%%%%%%%%%%%%%%%%%%%%%%%%%%%%
\subsection{In the language of systems and fundamental solutions}
There is also another way of classifying normal forms, namely by fundamental
solutions. The normal forms are then the systems $[A^0]$, with a fundamental
solution $\cY$ in a special form and from the Levelt-Turittin we then can
deduce that every system $[{}^{\hat F}\!A^0]$, with $\hat F\in G(\!(t)\!)$, has
a fundamental solution in the form $\hat F\cY$.
\begin{lem} \label{lem:normSol}
  % Let $(\cM^{nf},\nabla^{nf})$ be a model of a meromorphic connection,
  % which is represented by the system $[A^0]$.

  \marginnote{\cite[Thm.4.3.1]{Loday2014}}
  The meromorphic connection $(\cM^{nf},\nabla^{nf})$ is a model, if and
  only if there is a basis such that the defined system $[A^0]$ has a
  fundamental solution $\mathcal{Y}_0$ of the form
  \[
    \mathcal{Y}_0(t)=t^L e^{Q(t^{-1})}
  \]
  with
  \begin{itemize}
    \item \emph{irregular part} $e^{Q(t^{-1})}$ of $\mathcal{Y}_0$ defind by
      \[
        Q(t^{-1})=\underset{j=1}{\overset{s}{\bigoplus}}q_j(t^{-1})1_{n_j}
          =\diag(\underset{n_1\text{-times}}{\underbrace{%
          q_1,\dots,q_1}},q_2,\dots,q_s)
      \]
      where the $q_i$ are polynomials in $\frac{1}{t}$ or in a fractional power
      $\frac{1}{s}=\frac{1}{t^{1/p}}$ of $\frac{1}{t}$ such that $q_j(0)=0$,
      i.e.\ without constant term,
    \item $L\in\gl_n(\C)$ constant matrix called the \emph{matrix of formal
      monodromy}, where $t^L$ means $e^{L\ln t}$ and
      \marginnote{in \cite[1]{Remy2014} $L$ is just a Jordan normal form.  Is
      this generic enough?}
  \end{itemize}
  \begin{comment}
    The models $(\cM^{nf},\nabla^{nf})$ only yield the unramified case!
  \end{comment}
  The fundamental solution, i.e.\ matrix $\mathcal{Y}_0$, is called a
  \emph{normal solution}.
  \begin{comment}
    \begin{s-cor}
      A system $[A]$ is a model, if it has, up to meromorphic base change, a
      fundamental solution of the form $t^L e^{Q(t^{-1})}$, i.e.\ it has a
      fundamental solution of the form $Ft^L e^{Q(t^{-1})}$ with
      $F\in G(\!\{t\}\!)$.
    \end{s-cor}
  \end{comment}
\end{lem}
\begin{proof}
  \TODO{}
\end{proof}
\begin{rem}
  Let $[A^0]$ be a normal form with normal solution
  $\mathcal{Y}_0(t)=t^L e^{Q(t^{-1})}$.
  \begin{enumerate}
    \item In the unramified case are the $q_i(t^{-1})$ the $\phi_i(t)$ from
      definition~\ref{defn:model}.
      \TODO[Why/realy?]
    \item In the unramified case
      $[Q,L]=0$, i.e.\ $Q$ and $L$ commute and $L$
      can be supposed in Jordan form (cf.  \cite[Sec.4]{Martinet1991}).
    \item If $A'={}^{\hat F}\!A^0$ is obtained from $A^0$ via the formal
      meromorphic transformation $\hat F$ then is
      \[
        \mathcal{Y}=\hat F t^L e^{Q(t^{-1})}
      \]
      a fundamental solution for $[A']$.
      \TODO[Why?]
    \item \marginnote{\cite[73]{Loday2014}}
      It is always possible to permutate the columns of a fundamental
      solution by
      \[
        P^{-1}\mathcal{Y}P=\hat F t^{P^{-1}LP} e^{P^{-1}Q(t^{-1})P}
      \]
      with a permutation matrix $P$ and \rewrite{obtain another fundamental
      solution for the same system.}
  \end{enumerate}
\end{rem}
\begin{comment}
  \begin{lem}
    Let $\triangle$ be a system with fundamental solution
    $\mathcal{Y}_0(t)=t^L e^{Q(t^{-1})}$ then is the matrix
    ${A^0}:=dQ+L\frac{dt}{t}$ is a connection matrix for $\triangle$.
  \TODO[\textbf{Only} in the unramified/simple case?]
  \TODO[Therefore it is obtained from $A^0$ via an isotropy?]
  \end{lem}
\end{comment}

%%%%%%%%%%%%%%%%%%%%%%%%%%%%%%%%%%%%%%%%%%%%%%%%%%%%%%%%%%%%%%%%%%%%%%%%%%%%%%%
\subsection{The main asymptotic existence theorem}
\TODO[move] \TODO[maybe \textbf{not necessary}, since we have Borel-Ritt!]
\begin{comment}
  \begin{multicols}{2}
    \textbf{Classical:}
    \begin{itemize}
      \item \cite[Thm.4.4.1]{Loday2014}
      \item \cite[Thm.7.10]{van2003galois}{\tiny\cite[Thm.7.12]{van2003galois}}
      \item \cite[Thm.12.1]{wasow2002asymptotic}
      \item \cite[5.3.Thm.1]{Varadarajan96linearmeromorphic}
      \item \cite[207]{Balser2000Formal}: Some historical remarks
    \end{itemize}
  \columnbreak
    \textbf{Sheafical:}
    \begin{itemize}
      \item \cite[Thm.2.3.1]{sabbah_cimpa90}
    \end{itemize}
  \end{multicols}
\end{comment}
Here we want to state the main asymptotic existence theorem (or often M.A.E.T.)
which is essentially a deduction from the Borel-Ritt lemma.
\marginnote{\cite[207]{Balser2000Formal}}
It states that to every formal solution of a system of meromorphic differential
equations, and every sector with sufficiently small opening, one can find a
solution of the system having the formal one as its asymptotic expansion.
\begin{rem}
  We will use this in the case, when $A$ is via $\hat F$ formally equivalent to
  $A^0$. We are then able to find a \rewrite{(cyclic)} covering of the $S^1$ of
  arcs such that on every arc there exists a lift $\tilde F$ of $\hat F$.
\end{rem}
\TODO{}
\begin{comment}
  \begin{multicols}{2}
    \begin{thm}
      \marginnote{\cite[Thm.II.2.3.1]{sabbah_cimpa90}}
      \rewrite{Let $\cM_{K}$ be a meromorphic connection. There exists an
        integer $q\geq 1$ such that, after the ramification $x=t^q$, one has,
        for all $\theta\in S^1$ and each sufficiently small interval $V$
        centered at $\theta$}
      \[
        \cA_L(V)\otimes_L\cM_L\cong\cA_L(V)\otimes_L
        \left(\cF_L^R\otimes\cG_L\right)
      \]
    \end{thm}
    \begin{proof}
      See \cite[Sec.II.2.4]{sabbah_cimpa90}
    \end{proof}
  \columnbreak
    We set
    \[
      \hat\cM=\hat\cO\otimes \cM
    \]
    and for each $e^{i\theta^0}\in S^1$ we set
    \[
      \tilde\cM_{\theta^0}=\cA_{\theta^0}\otimes\cM\,.
    \]
    \begin{thm}[Sectorial decomposition]
      \marginnote{\cite[Thm.II.5.12]{sabbah2007isomonodromic}}
      Let $(\cM,\nabla)$ be a meromorphic connection and let us assume that
      there exist a model $\cM^{nf}$ together with an isomorphism
      $\hat\lambda:\hat\cM\to\hat\cM^{nf}$. There exists then, for any
      $e^{i\theta^0}\in S^1$, an isomorphism $\tilde\lambda_{\theta^0}:
      \tilde\cM_{\theta^0}\to\tilde\cM_{\theta^0}^{nf}$
      lifting $\hat\lambda$, that is, such that the following diagram
      \[ \begin{tikzcd}
          \tilde\cM_{\theta^o} \dar \rar{\tilde\lambda_{\theta^o}} &
          \tilde\cM^{nf}_{\theta^o} \dar
          \\\hat\cM \rar{\hat\lambda} &
          \hat\cM^{nf}
      \end{tikzcd} \]
      commutes
    \end{thm}
    \begin{proof}
      This is clear, since the tensor is right exact and we have the Borel-Ritt
      Lemma.
    \end{proof}
  \end{multicols}
\end{comment}

%%%%%%%%%%%%%%%%%%%%%%%%%%%%%%%%%%%%%%%%%%%%%%%%%%%%%%%%%%%%%%%%%%%%%%%%%%%%%%%
\section{The classifying set}
We want to understand the Set
$\bigg\{\Big[(\cM,\nabla)\Big]\bigg\}$ of the (convergent)
isomorphism classes of all meromorphic connections. Since we have also the
formal classification and we know, that all elements in a convergent
isomorphism class lie in the same formal isomorphism class, we can reduce the
problem by fixing a model $(\cM^{nf},\nabla^{nf})$ with the corresponding
connection matrix $A^0$. Thus we can restrict ourself to the subset
\begin{multline*}
  {}^0C(\cM^{nf},\nabla^{nf})=\bigg\{
    \Big[(\cM,\nabla)\Big]
    \mid \text{there exists a formal isomorphism }
  \\\qquad\hat f:(\hat\cM,\hat\nabla)
      \overset{\sim}\longrightarrow
      (\hat\cM^{nf},\hat\nabla^{nf})
  \bigg\}
\end{multline*}
of all isomorphism classes of meromorphic connections, which are formally
isomorphic to $(\cM^{nf},\nabla^{nf})$. This is the set, that we will be
calling the \emph{classifying set} and we will also denote it ${}^0C(A^0)$.
\begin{comment}
  \begin{itemize}
    \item \cite{thboalch} p.6
      \begin{itemize}
        \item \cite{boalch} p.19
      \end{itemize}
    \item \cite{Loday1994} p.852
    \item \cite{sabbah2007isomonodromic} p.111
    \item \cite{babbitt1983}
  \end{itemize}
\end{comment}

It is convenient to look at the slightly larger space of \emph{marked pairs}
\[
  \cH=\cH(\cM^{nf},\nabla^{nf})=\left\{
    \left[(\cM,\nabla,\hat f)\right]
      \mid
      \hat f:(\hat\cM,\hat\nabla)
        \overset{\sim}\longrightarrow
        (\hat\cM^{nf},\hat\nabla^{nf})
  \right\}
\]
in which we also keep the additional information, of the formal isomorphism, by
which the meromorphic connection is formally isomorphic to the model.
Where the isomorphisms of marked pairs are defined as follows:
\begin{defn}
  Two germs $(\cM,\nabla,\hat f)$ and $(\cM',\nabla',\hat f')$ are
  isomorphic if there exists an isomorphism
  $g:(\cM,\nabla)\overset{\sim}\longrightarrow(\cM',\nabla')$ such that
  $\hat f=\hat f'\circ \hat g$.
  \begin{comment}
    \cite[111]{sabbah2007isomonodromic}:\dots it is important to remark that
    such an isomorphism is then unique.
  \end{comment}
\end{defn}

Equivalently, one can talk in terms of systems. We then denote by
\[
  \Syst_m(A^0):=\{d-A
    \mid A={}^{\hat F}\!A^0 \text{ for some } \hat F\in G(\!(t)\!)\}
\]
the set of systems formally meromorphic equivalent to $A^0$.
Since we use meromorphic equivalences, in contrast to \cite{boalch,thboalch},
we denote that in $\Syst_m$ by the subscript ${}_m$.
Thus ${}^0C(\cM^{nf},\nabla^{nf})$ corresponds to
the set ${}^0C(A^0):=\Syst_m(A^0)/G(\!\{t\}\!)$ of meromorphic classes which
are formally equivalent to $A^0$.
\TODO[\cite{thboalch} p. 3: In the logarithmic case\dots]
Analogous, $\cH(\cM^{nf},\nabla^{nf})$ corresponds to the set $\cH(A^0)$ of
equivalence classes in
\[
  \hat\Syst_m(A^0):=\{(A,\hat F)
    \mid A={}^{\hat F}\!A^0 \text{ for some } \hat F\in G(\!(t)\!)\} \,.
\]

\begin{lem}
  Since $G_0(A^0)$ is defined as the stabilizer\TODO[correct?] of $A^0$ we deduce
  \[
    \Syst_m(A^0)\cong \hat G(A^0)/G_0(A^0) \,.
  \]
  \begin{s-cor}
    \marginnote{\cite[Eq.1.9b]{babbitt1983}}
    Thus the \emph{set of meromorphic classas of systems formally equivalent
      to $A^0$} are just the orbits of $G\{t\}$, that is
    \[
      {}^0C(A^0)\cong G\{t\}\backslash\hat G(A^0)/G_0(A^0)
    \]
    whereas the \emph{set of meromorphic classes of transformations of $[A^0]$}
    is represented by the left quotient $G\backslash\hat G(A^0)$.
  \end{s-cor}
\end{lem}
\begin{proof}
  See
  \begin{itemize}
    \item~\cite[6]{thboalch}: in the case $G_0(A^0)=T$
  \end{itemize}
\end{proof}

The group $G_0(A^0)$ is easy to compute and is often trivial. In fact, the
elements are block-diagonal, see~\cite[77]{Loday2014}.
Thus the structure of ${}^0C(A^0)$ is easily deduced from the structure of
$G\backslash\hat G(A^0)$.

\begin{lem}
  The set $G\backslash\hat G(A^0)$ is canonically isomorphic to $\cH(A^0)$.
\end{lem}
\begin{proof}
  See~\cite{thboalch}: Lemma 1.17
\end{proof}

