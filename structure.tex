\chapter{Structure of Stokes germs and Stokes
  matrices}\label{chp:WhichInformationIsNeeded}
Here we want to discuss, which information is required to describe the Stokes
cocycle corresponding to a multileveled system.
\section{Simplest, multileveled case}
Let $A^0$ be a normal form with dimension $n=3$ and two levels
$\cK=\{k_1<k_2\}$, which satisfies that there is at least one anti-Stokes
direction $\theta_0$ which is beared by both levels.
Let $q_j(t^{-1})$ be the determining polynomials and let $k_{jl}$ be the
degrees of $(q_j-q_l)(t^{-1})$.
Since it is the only way to get two levels in dimension $3$ we have, up to
permutation: $k_1=k_{1,2}\geq k_{1,3}$ and $k_2=k_{2,3}$.
We know also that the leading terms of $(q_1-q_2)(t^{-1})$ and
$(q_1-q_3)(t^{-1})$ are equal and thus determine the same anti-Stokes
directions.
The set of all anti-Stokes directions is then given as
\[
  \A=\left\{\theta_0+\frac{\pi}{k}\cdot j\mid k\in\cK\text{, }j\in\N\right\}
    =:\{
      \underset{\theta_0}{%
        \underset{\text{\rotatebox[origin=c]{-90}{$=$}}}{%
          \theta_1
      }}
    ,\dots,\theta_\nu\}\,.
\]
Denote by $\cY_0(t)$ a normal solution of $[A^0]$.

\marginnote{\cite[79]{Loday2014}}
Fix a set of determinations $\tilde\theta_1,\dots,\tilde\theta_\nu$ of the
anti-Stokes directions $\theta_1,\dots,\theta_\nu$ such that all
$\tilde\theta_j$ lie in the same interval $[2m\pi,2(m+1)\pi[$.
For every $\theta\in\A$ denote by $\cY_{0,\tilde\theta}(t)$ the normal solution
with that choice of a determination of the argument. We have already used this
notation on page~\pageref{page:alreadyUsedDefn}.

\subsection{What do the Stokes germs $\phi_\theta$ look like?}
\rewrite{First} we want to use the Stokes matrices of Stokes germs, discussed
in Section~\ref{sec:matrixReps}, to \rewrite{obtain some statements} about the
\rewrite{structure} of the Stokes germs in our situation.

The Proposition~\ref{prop:representation} states that every Stokes germ
$\phi_\theta$ can be written as its matrix representation conjugated by the
normal solution. That means that there is a Stokes matrix in
$\SSto_\theta(A^0)$, which gets via
\begin{align*}
  \rho_{\tilde\theta}^{-1}:
  \SSto_\theta(A^0)
  &\overset{\cong}\longrightarrow
  \Sto_\theta(A^0)
  \\C_{\cY_{0,\tilde\theta}} &\longmapsto
  \cY_{0,\tilde\theta} C_{\cY_{0,\tilde\theta}} \cY_{0,\tilde\theta}^{-1} \,,
\end{align*}
mapped to the Stokes germ $\phi_\theta$.
\TODO[This depends on the determination $\tilde\theta$ of $\theta$]

\paragraph{Using the structure of $\SSto_\theta(A^0)$}
First look at an example in which we will explain, from which relations on the
determining polynomials which restriction on the form of the Stokes matrices
follow.
\begin{exmp}
  Let $\theta\in\A$ be an anti-Stokes direction.
  From the definition of $\SSto_\theta(A^0)$ (cf.\
  Definition~\ref{defn:groupOfFaithfullReps}) we know that, if one has $q_1
  \underset{\theta,\max}{\prec} q_2$, the Stokes matrix has the form
  \[
    \begin{pmatrix}
      1 & \text{\boldmath$c_1$} & \star
    \\\text{\boldmath$0$} & 1 & \star
    \\\star & \star & 1
    \end{pmatrix}
  \]
  where $c_j\in\C$ and $\star\in\C$.
  \begin{s-rem}
    In our situation, we also know that
    \begin{einr}
      $q_1 \underset{\theta,\max}{\prec} q_2$
      \Leftrightarrow{}
      $q_1 \underset{\theta,\max}{\prec} q_3$
      \qquad \rewrite{respectively} \qquad
      $q_2 \underset{\theta,\max}{\prec} q_1$
      \Leftrightarrow{}
      $q_3 \underset{\theta,\max}{\prec} q_1$
    \end{einr}
    since the leading terms of $q_1-q_2$ and $q_1-q_3$ are equal.
  \end{s-rem}
  Thus the representation has the \rewrite{form}
  \[
    \begin{pmatrix}
      1 & c_1 & \text{\boldmath$c_2$}
    \\0 & 1 & \star
    \\\text{\boldmath$0$} & \star & 1
    \end{pmatrix}\,.
  \]
  If we also know that neither $q_2 \underset{\theta,\max}{\prec} q_3$ nor
  $q_3 \underset{\theta,\max}{\prec} q_2$ it has the \rewrite{form}
  \[
    \begin{pmatrix}
      1 & c_1 & c_2
    \\0 & 1 & \text{\boldmath$0$}
    \\0 & \text{\boldmath$0$} & 1
    \end{pmatrix}\,.
  \]
  We also know that every matrix of this \rewrite{form} is a representation to
  some Stokes germ.
  Thus we have an isomorphism
  \begin{align*}
    \vartheta_\theta:\C^2 &\longrightarrow \SSto_\theta(A^0)
  \\(c_1,c_2)&\longmapsto
    \begin{pmatrix}
      1 & c_1 & c_2
    \\0 & 1 & 0
    \\0 & 0 & 1
    \end{pmatrix}
  \end{align*}
\end{exmp}
In fact, the following $9$ cases of Stokes matrices can arise.
\begin{center}
  \def\arraystretch{1.3}
  \setlength\tabcolsep{4mm}
  \begin{tabular}{r|c|c|c}
    & $q_2 \underset{\theta,\max}{\prec} q_3$
    & $q_3 \underset{\theta,\max}{\prec} q_2$
    & else
    \tabularnewline
    \hline
    $\substack{q_1 \underset{\theta,\max}{\prec} q_2\\\text{and}
    \\q_1 \underset{\theta,\max}{\prec} q_3}$
    & $\begin{pmatrix} 1 & c_2 & c_3 \\0 & 1 & c_1 \\0 & 0 & 1 \end{pmatrix}$
   \cellcolor{blue!15}
    & $\begin{pmatrix} 1 & c_2 & c_3 \\0 & 1 & 0 \\0 & c_1 & 1 \end{pmatrix}$
   \cellcolor{blue!15}
    & $\begin{pmatrix} 1 & c_2 & c_3 \\0 & 1 & 0 \\0 & 0 & 1 \end{pmatrix}$
   \cellcolor{green!15}
    \tabularnewline
    \hline
    $\substack{q_2 \underset{\theta,\max}{\prec} q_1\\\text{and}
    \\q_3 \underset{\theta,\max}{\prec} q_1}$
    & $\begin{pmatrix} 1 & 0 & 0 \\c_2+c_1c_3 & 1 & c_1 \\c_3 & 0 & 1 \end{pmatrix}$
   \cellcolor{blue!15}
    & $\begin{pmatrix} 1 & 0 & 0 \\c_2 & 1 & 0 \\c_1c_2+c_3 & c_1 & 1 \end{pmatrix}$
   \cellcolor{blue!15}
    & $\begin{pmatrix} 1 & 0 & 0 \\c_2 & 1 & 0 \\c_3 & 0 & 1 \end{pmatrix}$
   \cellcolor{green!15}
    \tabularnewline
    \hline
    else
    & $\begin{pmatrix} 1 & 0 & 0 \\0 & 1 & c_1 \\0 & 0 & 1 \end{pmatrix}$
   \cellcolor{purple!15}
    & $\begin{pmatrix} 1 & 0 & 0 \\0 & 1 & 0 \\0 & c_1 & 1 \end{pmatrix}$
   \cellcolor{purple!15}
    & $\begin{pmatrix} 1 & 0 & 0 \\0 & 1 & 0 \\0 & 0 & 1 \end{pmatrix}$
  \end{tabular}
\end{center}
In the \textcolor{blue!75!black}{blue} cases we have $\cK_\theta=\cK$ and
$\C^3\overset{\vartheta_\theta}{\underset{\cong}{\longrightarrow}}\SSto_\theta(A^0)$.
In the \textcolor{green!50!black}{green} cases $\cK_\theta=\{k_2\}$ and
$\C^2\overset{\vartheta_\theta}{\underset{\cong}{\longrightarrow}}\SSto_\theta(A^0)$ as
well as in the \textcolor{purple!75!black}{purple} cases $\cK_\theta=\{k_1\}$
and $\C^1\overset{\vartheta_\theta}{\underset{\cong}{\longrightarrow}}\SSto_\theta(A^0)$.
Thus, for every $\theta\in\A$, we have an isomorphism
$\rho_{\tilde\theta}^{-1}\circ\vartheta_\theta$.
Two of the cases are more complicated than necessary, to be consistent with
the decomposition in the next part (cf.\ Example~\ref{exmp:decompositionHere}).

\paragraph{Decomposition by levels}
In proposition~\ref{prop:filtrationOfStokesGroup} and especially
remark~\ref{rem:filtrationOfStokesMats} we have defined a decomposition of the
Stokes group $\Sto_\theta(A^0)$ in subgroups generated by $k$-germs for
$k\in\cK$.
In our case, we have at most two level, such that this decomposition is given
by
\[
  \phi_\theta=\phi_\theta^{k_1} \phi_\theta^{k_2}
  \overset{i_\theta}\longmapsto
    \left(\phi_\theta^{k_1},\phi_\theta^{k_2}\right)
      \in\Sto_\theta^{k_1}(A^0)\times\Sto_\theta^{k_2}(A^0) \,,
\]
where $\phi_\theta^{k_1}\in\Sto_\theta^{k_1}(A^0)=\Sto_\theta^{<k_2}(A^0)$,
$\phi_\theta^{k_2}\in\Sto_\theta^{k_2}(A^0)$  and $i_\theta$ is the map, wich
gives the factors of this factorization in ascending order.

This decomposition of a germ $\phi_\theta$ is trivial if
$\#\cK(\phi_\theta)\leq1$, thus the interesting cases are the
\textcolor{blue!75!black}{blue} cases.

\begin{exmp}\label{exmp:decompositionHere}
  Look at the example
  \[
    \vartheta_\theta(c_1,c_2,c_3)=
    \cY_{0,\tilde\theta}
    \begin{pmatrix} 1 & 0 & 0 \\c_2 & 1 & 0 \\c_1c_2+c_3 & c_1 & 1 \end{pmatrix}
    \cY_{0,\tilde\theta}^{-1}
    =\phi_\theta
    \,.
  \]
  According to remark~\ref{rem:algFactorization} the factor
  $\phi_\theta^{k_1}$, generated by the $k_1$-germs, is given by
  \[
    \phi_\theta^{k_1}=
    \cY_{0,\tilde\theta}
    \begin{pmatrix}
      1 & 0 & 0
    \\\text{\boldmath $0$} & 1 & 0
    \\\text{\boldmath $0$} & c_1 & 1
    \end{pmatrix}
    \cY_{0,\tilde\theta}^{-1}
    \,.
  \]
  The other factor $\phi_\theta^{k_2}$ is then obtained as
  \begin{align*}
    \phi_\theta^{k_2}&=
    \left(\phi_\theta^{k_1}\right)^{-1}
    \phi_\theta^{k_2}
  \\&=\cY_{0,\tilde\theta}
    \begin{pmatrix}
      1     & 0    & 0
    \\0     & 1    & 0
    \\0     & -c_1 & 1
    \end{pmatrix}
    \underset{=\id}{\underbrace{%
        \cY_{0,\tilde\theta}^{-1}
        \cY_{0,\tilde\theta}
    }}
    \begin{pmatrix} 1 & 0 & 0 \\c_2 & 1 & 0 \\c_1c_2+c_3 & c_1 & 1 \end{pmatrix}
    \cY_{0,\tilde\theta}^{-1}
  \\&=\cY_{0,\tilde\theta}
    \begin{pmatrix}
      1     & 0 & 0
    \\c_2     & 1          & 0
    \\c_3     & 0          & 1
    \end{pmatrix}
    \cY_{0,\tilde\theta}^{-1}
    \,.
  \end{align*}
\end{exmp}
The four nontrivial decomposition in our situation, are given by
\begin{enumerate}
  \item $\begin{pmatrix} 1 & 0 & 0 \\0 & 1 & c_1 \\0 & 0 & 1 \end{pmatrix}
  \cdot\begin{pmatrix} 1 & c_2 & c_3 \\0 & 1 & 0 \\0 & 0 & 1 \end{pmatrix}=
  \begin{pmatrix} 1 & c_2 & c_3 \\0 & 1 & c_1 \\0 & 0 & 1 \end{pmatrix}$
  \item $\begin{pmatrix} 1 & 0 & 0 \\0 & 1 & 0 \\0 & c_1 & 1 \end{pmatrix}
  \cdot\begin{pmatrix} 1 & c_2 & c_3 \\0 & 1 & 0 \\0 & 0 & 1 \end{pmatrix}=
  \begin{pmatrix} 1 & c_2 & c_3 \\0 & 1 & 0 \\0 & c_1 & 1 \end{pmatrix}$
  \item $\begin{pmatrix} 1 & 0 & 0 \\0 & 1 & c_1 \\0 & 0 & 1 \end{pmatrix}
  \cdot\begin{pmatrix} 1 & 0 & 0 \\c_2 & 1 & 0 \\c_3 & 0 & 1 \end{pmatrix}=
  \begin{pmatrix} 1 & 0 & 0 \\c_2+c_1c_3 & 1 & c_1 \\c_3 & 0 & 1 \end{pmatrix}$
  \item $\begin{pmatrix} 1 & 0 & 0 \\0 & 1 & 0 \\0 & c_1 & 1 \end{pmatrix}
  \cdot\begin{pmatrix} 1 & 0 & 0 \\c_2 & 1 & 0 \\c_3 & 0 & 1 \end{pmatrix}=
  \begin{pmatrix} 1 & 0 & 0 \\c_2 & 1 & 0 \\c_1c_2+c_3 & c_1 & 1 \end{pmatrix}$
\end{enumerate}
We will use this decompositions to write the isomorphisms
$\vartheta_\theta:\C^\star\to \SSto_\theta(A^0)$ as
\[ \begin{tikzcd}
  \C^\star
  \rar{j_\theta}
  \arrow[rr,out=-30,in=-150,"\vartheta_\theta"]
  &
  \SSto_\theta^{k_1}(A^0)\times\SSto_\theta^{k_2}(A^0)
  \rar{i_\theta^{-1}}
  &
  \SSto_\theta(A^0)
\end{tikzcd} \]
where $\star\in\{1,2,3\}$.

\subsection{What do Stokes cocycles look like?}
\begin{prop}\label{prop:decompositionDiagram}
  By taking the product over $\A$ of all
  $\rho_{\tilde\theta}^{-1}\circ\vartheta_\theta$ we obtain the isomorphism
  \[
    \prod_{\theta\in\A}\rho_{\tilde\theta}^{-1}\circ\vartheta_\theta:
    \C^{k_1+2\cdot k_2}
    \overset\cong\longrightarrow \prod_{\theta\in\A}\SSto_\theta(A^0)
    \overset\cong\longrightarrow \prod_{\theta\in\A}\Sto_\theta(A^0)
  \]
  which can be concatenated with $h$ to obtain an element in $\St(A^0)$.
  We can also define the isomorphism
  $\C^{2\cdot(k_1+2\cdot k_2)}\textcolor{purple}{\longrightarrow} \St(A^0)$
  which makes the following diagram commute.
  \[ \begin{tikzcd}[column sep=1.5cm,row sep=2cm]
      &\C^{2\cdot(k_1+2\cdot k_2)}
      \dlar[blue]{\prod_{\theta\in\A}\rho_{\tilde\theta}^{-1}\circ\vartheta_\theta}
      \arrow[dr,purple]
    \\\prod_{\theta\in\A}\Sto_\theta(A^0)
      \arrow[rr,green!50!black,"h"]
      &&\St(A^0)
  \end{tikzcd} \]
  Using the knowledge, how the morphism were build and the fact that some of
  the morphism are equal, we can rewrite this diagram as follows
  \[ \begin{tikzcd}
      &\C^{2\cdot(k_1+2\cdot k_2)}
      \dar[xshift=-1pt,blue]
      \dar[xshift=1pt,purple,"\prod_{\theta\in\A}j_\theta"]
    \\
      &\prod_{k\in\{k_1,k_2\}}\prod_{\theta\in\A}\SSto_{\theta}^{k}(A^0)
       \arrow[dl,blue,"\prod_{\theta\in\A}i_\theta^{-1}"]
       \arrow[ddr,purple,
         "\prod_{k\in\{k_1,k_2\}}\prod_{\theta\in\A}(\rho_{\tilde\theta}^k)^{-1}"]
    \\ \prod_{\theta\in\A} \SSto_{\theta}(A^0)
        \dar[blue]{\prod_{\theta\in\A}\rho_{\tilde\theta}^{-1}}
    \\\prod_{\theta\in\A} \Sto_{\theta}(A^0)
      \arrow[rr,green!50!black,"\prod_{\theta\in\A}i_\theta"]
      &&\prod_{k\in\{k_1,k_2\}}\prod_{\theta\in\A}\Sto_{\theta}^{k}(A^0)
      \dar[xshift=-1pt,green!50!black]
      \dar[xshift=1pt,purple]{\prod_{k\in\{k_1,k_2\}}i^k}
    \\
      && \prod_{k\in\{k_1,k_2\}}\Gamma(\dot\cU^k;\Lambda^k(A^0))
      \dar[xshift=-1pt,green!50!black]
      \dar[xshift=1pt,purple]{\cT}
    \\&& \St(A^0)
  \end{tikzcd} \]
\end{prop}

\begin{comment}
  Decompose $\C^{k_1+2\cdot k_2}\cong
  \prod_{\theta\in\A^{k_1}} \C \times \prod_{\theta\in\A^{k_2}} \C^2$ and
  define the isomorphism
  \[
    \C^{k_1+2\cdot k_2}\overset{\cong}\longrightarrow
    \prod_{\theta\in\A^{k_1}} \SSto_{\theta}^{k_1}(A^0) \times
    \prod_{\theta\in\A^{k_2}} \SSto_{\theta}^{k_2}(A^0)
  \]
  levelwise as
  \begin{enumerate}
    \item an element
      $\left( (a_{\theta_{\nu_1}},b_{\theta_{\nu_1}})
        ,(a_{\theta_{\nu_2}},b_{\theta_{\nu_2}})
        ,\dots
      \right)\in\prod_{\theta\in\A^{k_2}} \C^2$
      gets mapped to
      \[
        \left.
        \begin{cases}
          \left(
          \begin{pmatrix} 1 & a_{\theta_{\nu_1}} & b_{\theta_{\nu_1}} \\0 & 1 & 0 \\0 & 0 & 1 \end{pmatrix}
          ,\begin{pmatrix} 1 & 0 & 0 \\a_{\theta_{\nu_2}} & 1 & 0 \\b_{\theta_{\nu_2}} & 0 & 1 \end{pmatrix}
          ,\begin{pmatrix} 1 & a_{\theta_{\nu_3}} & b_{\theta_{\nu_3}} \\0 & 1 & 0 \\0 & 0 & 1 \end{pmatrix}
            ,\dots
          \right)
          & ,\substack{\text{~if~} q_1 \underset{\theta_0,\max}{\prec} q_2
            \\\text{~and~} q_1 \underset{\theta_0,\max}{\prec} q_3}
          \\\left(
          \begin{pmatrix} 1 & 0 & 0 \\a_{\theta_{\nu_1}} & 1 & 0 \\b_{\theta_{\nu_1}} & 0 & 1 \end{pmatrix}
          ,\begin{pmatrix} 1 & a_{\theta_{\nu_2}} & b_{\theta_{\nu_2}} \\0 & 1 & 0 \\0 & 0 & 1 \end{pmatrix}
          ,\begin{pmatrix} 1 & 0 & 0 \\a_{\theta_{\nu_3}} & 1 & 0 \\b_{\theta_{\nu_3}} & 0 & 1 \end{pmatrix}
            ,\dots
          \right)
          & ,\substack{\text{~if~} q_2 \underset{\theta_0,\max}{\prec} q_1
            \\\text{~and~} q_3 \underset{\theta_0,\max}{\prec} q_1}
        \end{cases}
        \right\}
        =:\left(
          C_{\theta_{\nu_1}}^{k_2}
          ,C_{\theta_{\nu_2}}^{k_2}
          ,\dots
        \right)
      \]
    \item
      $\left(
        c_{\theta_{\mu_1}}
        ,c_{\theta_{\mu_2}}
        ,\dots
      \right)\in\prod_{\theta\in\A^{k_1}} \C$
      gets mapped to
      \[
        \left.
        \begin{cases}
          \left(
          \begin{pmatrix} 1 & 0 & 0 \\0 & 1 & c_{\theta_{\mu_1}} \\0 & 0 & 1 \end{pmatrix}
          ,\begin{pmatrix} 1 & 0 & 0 \\0 & 1 & 0 \\0 & c_{\theta_{\mu_2}} & 1 \end{pmatrix}
          ,\begin{pmatrix} 1 & 0 & 0 \\0 & 1 & c_{\theta_{\mu_3}} \\0 & 0 & 1 \end{pmatrix}
            ,\dots
          \right)
          &\text{,~if~} q_2 \underset{\theta_0,\max}{\prec} q_3
          \\\left(
          \begin{pmatrix} 1 & 0 & 0 \\0 & 1 & 0 \\0 & c_{\theta_{\mu_1}} & 1 \end{pmatrix}
          ,\begin{pmatrix} 1 & 0 & 0 \\0 & 1 & c_{\theta_{\mu_2}} \\0 & 0 & 1 \end{pmatrix}
          ,\begin{pmatrix} 1 & 0 & 0 \\0 & 1 & 0 \\0 & c_{\theta_{\mu_2}} & 1 \end{pmatrix}
            ,\dots
          \right)
          &\text{,~if~} q_3 \underset{\theta_0,\max}{\prec} q_2
        \end{cases}
      \right\}
      =:\left(
        C_{\theta_{\mu_1}}^{k_1}
        ,C_{\theta_{\mu_2}}^{k_1}
        ,\dots
      \right)
      \]
  \end{enumerate}
\end{comment}

%%%%%%%%%%%%%%%%%%%%%%%%%%%%%%%%%%%%%%%%%%%%%%%%%%%%%%%%%%%%%%%%%%%%%%%%%%%%%%%
\subsubsection{Explicit example}
\def\kOne{1}
\def\kTwo{3}
\def\zkOnepzKtwo{14} % 2\cdot(\kOne+2\cdot\kTwo
\def\zkOne{2} % 2*\kOne
\def\zkTwo{6} % 2*\kTwo

Even more explicit, we can fix the levels $k_1=\kOne$ and $k_2=\kTwo$ together
with $\theta_0=0$.
Assume without any restriction that $q_1 \underset{\theta_0,\max}{\prec} q_2$
and $q_1 \underset{\theta_0,\max}{\prec} q_3$ as well as
$q_2 \underset{\theta_0,\max}{\prec} q_3$.

The classification space is in this case isomorphic to
$\C^{2\cdot(\kOne+2\cdot\kTwo)}=\C^{\zkOnepzKtwo}$.
The element
\[
  ({}^1c_1,{}^2c_1,
  {}^1c_2,{}^1c_3,{}^2c_2,{}^2c_3,\dots,{}^{\zkTwo}c_2,{}^{\zkTwo}c_3)
  \in\C^{\zkOnepzKtwo}
\]
gets, via the morphism $\prod_{\theta\in\A}j_\theta$, mapped to
\begin{align*}
  &\left(
  \left(
    \begin{pmatrix} 1 & 0 & 0 \\0 & 1 & {}^1c_1 \\0 & 0 & 1 \end{pmatrix},
    \begin{pmatrix} 1 & 0 & 0 \\0 & 1 & 0 \\0 & {}^2c_1 & 1 \end{pmatrix}
  \right),
  \right.
\\&\qquad\left(
  \left.
    \begin{pmatrix} 1 & {}^1c_2 & {}^1c_3 \\0 & 1 & 0 \\0 & 0 & 1 \end{pmatrix},
    \begin{pmatrix} 1 & 0 & 0 \\{}^2c_2 & 1 & 0 \\{}^2c_3 & 0 & 1 \end{pmatrix},
    \dots,
    \begin{pmatrix} 1 & 0 & 0 \\{}^{\zkTwo}c_2 & 1 & 0 \\{}^{\zkTwo}c_3 & 0 & 1 \end{pmatrix}
  \right)
  \right)
\end{align*}
in $\prod_{\theta\in\A^{\kOne}}\SSto_{\theta}^{\kOne}(A^0) \times
\prod_{\theta\in\A^{\kTwo}}\SSto_{\theta}^{\kTwo}(A^0)$ and thus the element
\begin{align*}
  &\left(
  \left(
    \begin{pmatrix} 1 & 0 & 0 \\0 & 1 & {}^1c_1 \\0 & 0 & 1 \end{pmatrix},
    \id,\id,
    \begin{pmatrix} 1 & 0 & 0 \\0 & 1 & 0 \\0 & {}^2c_1 & 1 \end{pmatrix},
    \id,\id
  \right),
  \right.
\\&\qquad\left(
  \left.
    \begin{pmatrix} 1 & {}^1c_2 & {}^1c_3 \\0 & 1 & 0 \\0 & 0 & 1 \end{pmatrix},
    \begin{pmatrix} 1 & 0 & 0 \\{}^2c_2 & 1 & 0 \\{}^2c_3 & 0 & 1 \end{pmatrix},
    \dots,
    \begin{pmatrix} 1 & 0 & 0 \\{}^{\zkTwo}c_2 & 1 & 0 \\{}^{\zkTwo}c_3 & 0 & 1 \end{pmatrix}
  \right)
  \right)
\end{align*}
in
$\prod_{\theta\in\A}\SSto_{\theta}^{\kOne}(A^0) \times
\prod_{\theta\in\A}\SSto_{\theta}^{\kTwo}(A^0)$.
Using the morphism $\prod_{\theta\in\A}i_\theta^{-1}$ we get a complete set of
Stokes matrices as
\begin{align*}
  &\left(
    \begin{pmatrix} 1 & {}^1c_2 & {}^1c_3 \\0 & 1 & {}^1c_1 \\0 & 0 & 1 \end{pmatrix},
    \begin{pmatrix} 1 & 0 & 0 \\{}^2c_2 & 1 & 0 \\{}^2c_3 & 0 & 1 \end{pmatrix},
    \begin{pmatrix} 1 & {}^3c_2 & {}^3c_3 \\0 & 1 & 0 \\0 & 0 & 1 \end{pmatrix},
  \right.
\\&\qquad
  \left.
    \begin{pmatrix} 1 & 0 & 0 \\{}^4c_2 & 1 & 0 \\{}^2c_1{}^4c_2+{}^4c_3 & {}^2c_1 & 1 \end{pmatrix},
    \begin{pmatrix} 1 & {}^5c_2 & {}^5c_3 \\0 & 1 & 0 \\0 & 0 & 1 \end{pmatrix},
    \begin{pmatrix} 1 & 0 & 0 \\{}^{\zkTwo}c_2 & 1 & 0 \\{}^{\zkTwo}c_3 & 0 & 1 \end{pmatrix}
  \right)
  \in
  \prod_{\theta\in\A}\SSto_{\theta}(A^0)
\end{align*}

\section{A more general multileveled case}
Let us assume, that our normal form $[A^0]$ of dimension $n$ has the levels
$\cK$ and start at the situation
$\prod_{k\in\cK}\prod_{\theta\in\A}\SSto_{\theta}^{k}(A^0)$
of Proposition~\ref{prop:decompositionDiagram}.
Since the factors $\prod_{\theta\in\A}\SSto_{\theta}^{k}(A^0)$ are products
over a single level, we are able to apply the theory discussed by Boalch
in~\cite[Section 3]{boalch}.
The goal is, to group the set $\A^k$ of all anti-Stokes directions of level $k$
into $2k$ distinct subsets of consecutive anti-Stokes directions of length
$\frac{\#\A^k}{2k}$. Such a subset of $\A^k$ will be called a
\emph{half-period}.
Let $\theta\in S^1$ be a direction which satisfies that
$\theta\pm\frac{\pi}{2k}\notin\A$, then is a half-period obtained by
$\A^k\cap U(\theta,\frac{\pi}{k})$ where $U(\theta,\frac{\pi}{k})$ is an arc of
width $\frac{\pi}{k}$ with center $\theta$
(cf.\ Remark~\ref{rem:arcsByWidthAndCenter}).
This can be reformulated to:
\begin{einr}
  every arc of width $\frac{\pi}{k}$, which has no anti-Stokes direction of
  level $k$ on the border, contains $\frac{\#\A^k}{2k}$ anti-Stokes directions
\end{einr}
or
\begin{einr}
  that for every sector $I$ of width $\frac{\pi}{k}$, which satisfies the
  condition on the border, there is to any unordered pair in $\cQ(A^0)$, which
  has level $k$,
  \[
    [(q_j,q_l)]\in
    \{(q_j,q_l)\mid(q_j,q_l)\in \cQ(A^0), k_{j,l}=k\}/(q_j,q_l)-(q_l,q_j)
  \]
  exactly one direction $\theta\in\A^{k}\cap I$ which satisfies either
  $q_j\underset{\theta,\max}{\prec}q_l$ or
  $q_l\underset{\theta,\max}{\prec}q_j$.
\end{einr}
We then are able to express the information of the Stokes matrices
corresponding to such an half-period in a single matrix of special form.

\begin{comment}
  We know that
  \[
    \prod_{\theta\in\A}\SSto_{\theta}^{k}(A^0)\cong
    \prod_{\theta\in\A^k}\SSto_{\theta}^{k}(A^0) \,,
  \]
  since $\SSto_{\theta}^{k}(A^0)=\{\id\}$ when $k\notin\cK_\theta$.
\end{comment}

Let $I=U(\theta,\frac{\pi}{k})$ be an arc width $\frac{\pi}{k}$ such that
$\theta\pm\frac{\pi}{2k}\notin\A^k$.
\begin{rem}\label{rem:remarkAboutCommonStructure}
  For every $\theta\in\A^k\cap I$ we can write the Stokes matrix as
  \[
    K^\theta=(K^\theta_{jl})_{j,l\in\{1,\dots,s\}}\in\SSto_\theta^k(A^0)
  \]
  where the $K^\theta_{jl}\in\C^{n_j\times n_l}$ are blocks corresponding to
  the structure of $Q$ (cf.\ Definition~\ref{defn:groupOfFaithfullReps}).
  We then know for $j\neq l$ that, if $K^\theta_{jl}\neq0$ then
  \begin{itemize}
    \item is $K^\theta_{lj}=0$ and
    \item for every $\theta'\in(\A^k\cap I)\backslash\{\theta\}$ is
      $K^{\theta'}_{jl}=0$ as well as $K^{\theta'}_{lj}=0$.
  \end{itemize}
\end{rem}
\begin{rem}
  In the situation of Remark~\ref{rem:remarkAboutCommonStructure}
  there exists a common (block) permutation matrix $P\in G$ given by
  $(P)_{jl}=\delta_{\pi(j)l}$ where
  \begin{itemize}
    \item $\delta_{jl}$ is the block version of Kronecker's delta which was
      introduced in Definition~\ref{defn:groupOfFaithfullReps} and
    \item $\pi$ is the permutation of $\{1,\dots,s\}$ corresponding to
      $q_j\underset{\theta,\max}{\prec}q_l$ \Leftrightarrow{} $\pi(j)<\pi(l)$
      \marginnote{\cite[18]{boalch}}
  \end{itemize}
  such that every matrix $P^{-1}K^{\theta}P$ is upper triangular.
  \begin{s-rem}
    After moving the sector $U(\theta,\frac{\pi}{k})$ to
    $U(\theta,\frac{\pi}{k})+\frac{\pi}{k}
    =U(\theta+\frac{\pi}{k},\frac{\pi}{k})$ we obtain also a corresponding
    permutation $\pi'$ which is inverse to $\pi$.
    The corresponding permutation matrix $P'$ which is then given by
    $P'=P^{-1}$.

    Such that the permutation $P$ transforms Stokes matrices on the sector
    $I+\frac{\pi}{k}$ to lower triangular matrices
  \end{s-rem}
  In the single leveled case of this phenomenon is discussed on page 18 of
  Boalch's paper~\cite{boalch}.
\end{rem}

In the paper \cite{BJL1979Birkhoff} from Balser, Jurkat and Lutz is the
following Lemma stated as Lemma 2 on page 75.
\begin{lem}
  Let $T\subset\{1,\dots,n\}\times\{1,\dots,n\}$ be a position set, which
  satisfies the \emph{completeness property}:
  \begin{einr}
    if $(j,k)$ and $(k,l)\in T$ then is also $(j,l)\in T$.
  \end{einr}
  Choose a indexing $i:\{1,\dots,
  \underset{\mu}{\underset{\text{\rotatebox[origin=c]{-90}{$=$}}}{\#T}}\}
  \overset{\cong}{\to}T$ of the position set, then there exists for every
  \[
    K\in S_T:=\left\{K=(K_{jl})_{j,l\in\{1,\dots,n\}}\in G \mid
      K_{jl}=\delta_{jl} \text{~unless~} (j,l)\in T \right\}
  \]
  unique scalars $t_m\in\C$ such that
  \[
    K=(\id + t_1E_{i(1)})\cdots(\id + t_{\mu}E_{i(\mu)}) \,.
  \]
  \begin{s-rem}
    \begin{enumerate}
      \item The completeness property is reasonable, since for example
        $S_{\{(2,3),(3,2)\}}$, corresponding to the not complete set
        $\{(2,3),(3,2)\}$, is not stable under the product:
        \begin{align*}
          \begin{pmatrix}
            1 & a & 0
          \\0 & 1 & 0
          \\0 & 0 & 1
          \end{pmatrix}
          \begin{pmatrix}
            1 & 0 & 0
          \\0 & 1 & b
          \\0 & 0 & 1
          \end{pmatrix}
          =
          \begin{pmatrix}
            1 & a & \textcolor{red!60!black}{ab}
          \\0 & 1 & b
          \\0 & 0 & 1
          \end{pmatrix}
          \notin S_{\{(2,3),(3,2)\}}
          \,.
        \end{align*}
      \item We can write the group $\SSto_\theta(A^0)$ as
        \[
          \SSto_\theta(A^0)=
          S_{\{(j,l)\mid q_j \underset{\theta,\max}{\prec} q_l\}} \,.
        \]
    \end{enumerate}
  \end{s-rem}
\end{lem}
\begin{cor}\label{cor:composeLevelwise}
  Let $T_1,\dots,T_r$ be a subsets of the complete set $T$, such that
  \begin{itemize}
    \item $T=T_1\dot\cup\dots\dot\cup T_r$ and
    \item for every $m$ does $T_m$ satisfy the completeness property.
  \end{itemize}
  Then is by
  \begin{align*}
    S_{T_1}\times\cdots\times S_{T_m} &\longrightarrow S_T
  \\(K_1,\dots,K_m) &\longmapsto K_1\cdots K_m
  \end{align*}
  an isomorphism defined.
\end{cor}
We can apply the Corollary~\ref{cor:composeLevelwise} in our situation. Let
$I=U(\theta,\frac{\pi}{k})$ be an arc of width $\frac{\pi}{k}$ which satisfies,
that $\theta\pm\frac{\pi}{2k}\notin\A^k$.
We then have
\begin{itemize}
  \item for every $\theta'\in I\cap \A^k$ satisfies the set
    $T_{\theta'}:=\{(j,l)\mid q_j\underset{\theta_{\theta'},\max}{\prec}q_l\}$
    the completeness property, since it is defined via a \rewrite{transitiv
    relation}, and
  \item the union of all $T_{\theta'}$ is then
    \begin{align*}
      T&:=\dot\bigcup_{\theta'\in I\cap\A}T_{\theta'}
      \\&=\{(j,l) \mid q_j \underset{\theta',\max}{\prec} q_l
          \text{~for some~}\theta'\in\A^k\cap I \}
      \\&=\{(j,l) \mid q_j \underset{\theta}{\prec} q_l \}
    \end{align*}
    and is also complete, since $\underset{\theta}{\prec}$ is also a
    \rewrite{transitive relation.}
\end{itemize}
Thus we have an isomorphism
\[
  \prod_{\theta\in\A^k\cap I}\SSto_\theta^k(A^0)
  \longrightarrow
  S_T=:\SSto_I^k(A^0)
\]
This yields directly an isomorphism
\[
  \prod_{\theta\in\A^k}\SSto_\theta^k(A^0)
  \longrightarrow
  \prod_{j\in\{1,\dots,2k\}}
  \SSto_{U(\theta+j\frac{\pi}{k},\frac{\pi}{k})}^k(A^0) \,.
\]
This means, that the information of all Stokes-matrices can be grouped into
$2k$ matrices, which are products of the corresponding Stokes matrices.

\begin{cor}
  We now can improve the diagram of isomorphisms from
  Propositios~\label{prop:decompositionDiagram} by writing it in a more general
  form and adding the found isomorphism.
  Let $\theta$ be a direction, which satisfies
  $\theta\pm\frac{\pi}{2k}\notin\A^k$ for every $k\in\cK$.
  \[ \begin{tikzcd}
      % &\C^{2\cdot(k_1+2\cdot k_2)}
      % \dar[xshift=-1pt,blue]
      % \dar[xshift=1pt,purple,"\prod_{\theta'\in\A}j_\theta'"]
    % \\
      &\displaystyle\prod_{k\in\cK}\prod_{\theta'\in\A}\SSto_{\theta'}^{k}(A^0)
       \arrow[dl,"\prod_{\theta'\in\A}i_\theta'^{-1}"]
       \arrow[ddr,
         "\prod_{k\in\cK}\prod_{\theta'\in\A}(\rho_{\tilde\theta'}^k)^{-1}"]
       \arrow[r]
      &\displaystyle\prod_{k\in\cK} \prod_{j\in\{1,\dots,2k\}}
       \SSto_{U(\theta+j\frac{\pi}{k},\frac{\pi}{k})}^k(A^0)
    \\ \displaystyle\prod_{\theta'\in\A} \SSto_{\theta'}(A^0)
        \dar{\prod_{\theta'\in\A}\rho_{\tilde\theta'}^{-1}}
    \\\displaystyle\prod_{\theta'\in\A} \Sto_{\theta'}(A^0)
      \arrow[rr,"\prod_{\theta'\in\A}i_\theta'"]
      &&\displaystyle\prod_{k\in\cK}\prod_{\theta'\in\A}\Sto_{\theta'}^{k}(A^0)
      \dar{\prod_{k\in\cK}i^k}
    \\
      && \displaystyle\prod_{k\in\cK}\Gamma(\dot\cU^k;\Lambda^k(A^0))
      \dar{\cT}
    \\&& \St(A^0)
      \arrow[purple,uuuu,out=30,in=-60,thick]
  \end{tikzcd} \]
  \[
    \St(A^0) \textcolor{purple}{\longrightarrow}
    \prod_{k\in\cK} \prod_{j\in\{1,\dots,2k\}}
    \SSto_{U(\theta+j\frac{\pi}{k},\frac{\pi}{k})}^k(A^0)
  \]

  \begin{s-rem}
    The contained isomorphism
    \[
      \prod_{\theta'\in\A}\SSto_{\theta'}(A^0) \longrightarrow
      \prod_{k\in\cK} \prod_{j\in\{1,\dots,2k\}}
      \SSto_{U(\theta+j\frac{\pi}{k},\frac{\pi}{k})}^k(A^0)
    \]
    is in the single-leveled case with $n=s$, i.e.\ all diagonal elements of
    $Q$ are different, given in Lemma 3.2 of Boalch's paper
    \cite[Lem.3.2]{boalch}.
    In this case is every
    $\SSto_{U(\theta+j\frac{\pi}{k},\frac{\pi}{k})}^k(A^0)$ isomorphic to some
    $PU_+P^{-1}$ where $U_+$ is the group of all upper triangular matrices,
    with ones on the diagonal, and $P$ is a permutation matrix.
  \end{s-rem}
\end{cor}

\PROBLEM{}
