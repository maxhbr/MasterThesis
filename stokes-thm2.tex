%%%%%%%%%%%%%%%%%%%%%%%%%%%%%%%%%%%%%%%%%%%%%%%%%%%%%%%%%%%%%%%%%%%%%%%%%%%%%%%
\section{Stokes structures: using Stokes groups}\label{sec:mainThm2}
\marginnote{\cite{Loday1994},~\cite[Thm.4.3.11]{Loday2014}
  \\and~\cite{boalch,thboalch}
  \\and~\cite{babbitt1989local}
  \\and~\cite{BJL1979Birkhoff}
  \\and~\cite[Chapter 4]{Martinet1991}}

The goal in this section is to prove that there is a bijective and natural map
\[
  h:\prod_{\alpha\in\A}\Sto_\alpha(A^0)\longrightarrow\St(A^0) \,.
\]
And since $\Sto_\alpha(A^0)$ has $\SSto_\alpha(A^0)$ as a faithful
representation, we also get the isomorphism
$\prod_{\alpha\in\A}\SSto_\alpha(A^0)\cong\St(A^0)$ as a corollary.
\TODO[This goes back to \cite{BJL1979Birkhoff}?]

Let us recall, that $\St(A^0)$ is defined to be $H^1(S^1;\Lambda(A^0))$
(cf.\ Section~\ref{sec:mainThm1}).
The elements of $\prod_{\alpha\in\A}\Sto_\alpha(A^0)$ define in a canonical way
cocycles of the sheaf $\Lambda(A^0)$ (cf.\ Equation (\ref{eq:mapStoToCocy})),
called Stokes cocycles (cf.\ Definition~\ref{defn:stokesCocycle}).
In fact, will $h$ map such cocycles to the cohomology class, to which they
correspond.
Thus the statement, that $h$ is a bijection, is equivalent to the statement
that
\begin{einr}
  in each cohomology class of $\St(A^0)$ is an unique $1$-cocycle, which is a
  Stokes cocycle.
\end{einr}

%%%%%%%%%%%%%%%%%%%%%%%%%%%%%%%%%%%%%%%%%%%%%%%%%%%%%%%%%%%%%%%%%%%%%%%%%%%%%%%
\subsubsection{Cyclic coverings}
To formulate the following theorem, we use the notion of cyclic coverings and
nerves of such coverings, which are defined as follows.

\begin{defn}
  \marginnote{\cite[Sec.II.1]{Loday1994} and \cite[Sec.II.3.1]{Loday1994}}
  Let $J$ be a finite set, identified to $\{1,\dots,p\}\subset\Z$.
  \begin{enumerate}
    \item A \emph{cyclic covering} of $S^1$ is a finite covering
      $\cU=\left(U_j=U(\theta_j,\epsilon_j)\right)_{j\in J}$ consisting of
      arcs, which satisfies that
      \begin{enumerate}
        \item $\tilde\theta_j \geq \tilde\theta_{j+1}$ for $j\in\{1,\dots,p-1\}$,
        \item $\tilde\theta_j+\frac{\epsilon_j}{2}\geq
          \tilde\theta_{j+1}+\frac{\epsilon_{j+1}}{2}$ for
          $j\in\{1,\dots,p-1\}$ and
          $\tilde\theta_p+\frac{\epsilon_p}{2}\geq
          \tilde\theta_{1}-2\pi+\frac{\epsilon_{1}}{2}$, i.e.\ the arcs are not
          encased by another arc,
      \end{enumerate}
      where the $\tilde\theta_j\in [0,2\pi[$ are determinations of the
      $\theta_j\in S^1$.
      \begin{comment}
        \begin{enumerate}
          \item the $\theta_j$ are in ascending order with respect to the
            clockwise orientation of $S^1$;
          \item the $U_j\cap U_{j+1}$ have only one connected component when
            $\#J>2$;
          \item the $U_j$ are not encased by another arc, this means that the
            open sets $U_j\backslash U_l$ are connected for all $j,l\in J$.
        \end{enumerate}
      \end{comment}
    \item The \emph{nerve} of a cyclic covering $\cU=\{U_j;j\in J\}$ is the
      family $\dot\cU=\{\dot U_j;j\in J\}$ defined by:
      \begin{itemize}
        \item $\dot U_j=U_j\cap U_{j+1}$ when $\#J>2$,
        \item $\dot U_1$ and $\dot U_2$ the connected components of
          $U_1\cap U_2$ when $\#J=2$.
      \end{itemize}
      \begin{s-rem}
        The nerve of the cyclic covering
        $\cU=\left(U(\theta_j,\epsilon_j)\right)_{j\in J}$ is explicitly given
        by
        \[
          \dot\cU=\left(
            \left(\theta_{j+1}-\frac{\epsilon_{j+1}}{2},
            \theta_{j}+\frac{\epsilon_{j}}{2}\right)
          \right)_{j\in J} \,.
        \]
      \end{s-rem}
  \end{enumerate}
\end{defn}
The cyclic coverings correspond one-to-one to nerves of cyclic coverings. If
one starts with a nerve $\{\dot U_j \mid j\in J\}$, one obtains a cyclic
covering as $\cU=\{U_j \mid j\in J\}$ where the arc $U_j$ is the connected
clockwise hull from $\dot U_{j-1}$ to $\dot U_j$.

\begin{defn}
  A covering $\cV$ is said to \emph{refine} a covering $\cU$ if, to each open
  set $V\in\cV$ there is at least one $U\in\cU$ with $V\subset U$.
\end{defn}
\begin{prop}
  \marginnote{\cite[Prop.II.1.3]{Loday1994}}
  The covering $\cV$ refines $\cU$ if and only if the corresponding nerves
  $\dot\cU=\{\dot U_j\}$ and $\dot\cV=\{\dot V_l\}$ satisfy
  \begin{einr}
    each $\dot U_j$ contains at least one $\dot V_l$.
  \end{einr}
\end{prop}

%%%%%%%%%%%%%%%%%%%%%%%%%%%%%%%%%%%%%%%%%%%%%%%%%%%%%%%%%%%%%%%%%%%%%%%%%%%%%%%
\subsection{The theorem}
\marginnote{\cite[868]{Loday1994}}
Let $\{\theta_j\mid j\in J\}\subset S^1$ be a finite set and
$\dot\phi=(\dot\phi_{\theta_j})_{j\in J}
\in\prod_{j\in J}\Lambda_{\theta_j}(A^0)$ be a finite family of germs.
Let $\dot\phi_j$ be the function representing the germ $\dot\phi_{\theta_j}$
on its (maximal) arc of definition $\Omega_j$ around $\theta_j$.
In the following way, one can associate a cohomology class in $\St(A^0)$ to
$\dot\phi$:
\begin{einr}
  for every cyclic covering $\cU=(U_j)_{j\in J}$ which satisfies
  $\dot U_j\subset\Omega_j$ for all $j\in J$, one can define the $1$-cocycle
  $(\dot\phi_{j|\dot U_j})_{j\in J}\in\Gamma(\dot\cU;\Lambda(A^0))$.
\end{einr}
To a different cyclic covering, satisfying the condition above, this
construction yields a cohomologous $1$-cocycle, thus the induced map
\begin{equation}\label{eq:mapStoToCocy}
  \prod_{j\in J}\Lambda_{\theta_j}(A^0)
  \longrightarrow
  H^1(S^1;\Lambda(A^0))=\St(A^0)
\end{equation}
is welldefined (cf.\ \cite[868]{Loday1994}).
\begin{defn}\label{defn:stokesCocycle}
  \marginnote{\cite[Def.II.1.8]{Loday1994}
    \\, \cite[4.3.10]{Loday2014}
    \\, \cite[Defn 6 on p 374]{Martinet1991}}
  Let $\nu=\#\A$ the number of all anti-Stokes directions and write
  $\A=\{\alpha_1,\dots,\alpha_\nu\}$.

  A \emph{Stokes cocycle} is a 1-cocycle $(\phi_j)_{j\in\{1,\dots,\nu\}}\in
  \prod_{j\in\{1,\dots,\nu\}}\Gamma(U_j;\Lambda(A^0))$ corresponding to some
  cyclic covering with nerve $\dot\cU=(\dot U_j)_{j\in\{1,\dots,\nu\}}$,
  which satisfies for every $j\in\{1,\dots,\nu\}$
  \begin{itemize}
    \item $\alpha_j\in\dot U_j$ and
    \item the germ $\phi_{\alpha_j}:=\phi_{j,\alpha_j}$ of $\phi_j$ at
      $\alpha_j$ is an element of $\Sto_{\alpha_j}(A^0)$.
  \end{itemize}
  \begin{s-rem}\label{rem:inclusionGermRemark}
    \PROBLEM[refactor!]
    The sections in $\Gamma(\dot U_j;\Lambda(A^0))$ are uniquely
    determined as the extension of the germ at $\alpha_j$, since the sheaf
    $\Lambda(A^0)$ defined via the system $[A^0,A^0]$
    (cf.\ Definition~\ref{defn:StokesSheaf}).
    We thus have an injective map
    \[
      \prod_{j\in\{1,\dots,\nu\}}\Gamma(\dot U_j;\Lambda(A^0))
      \hookrightarrow
      \prod_{j\in\{1,\dots,\nu\}}\Sto_{\alpha_j}(A^0) \,,
    \]
    which takes an Stokes cocycle and yields the corresponding Stokes germs.
    For a fine enough covering $\cU$, i.e.\ a covering $\cU$ with a nerve
    $\dot\cU$ which consists of small enough arcs satisfying the conditions
    above, is this map a bijection.

    We will use this fact implicitly and assume that the covering is always
    fine enough to call elements of $\prod_{\alpha\in\A}\Sto_\alpha(A^0)$
    Stokes cocycles.
  \end{s-rem}
\end{defn}
We can use the Equation (\ref{eq:mapStoToCocy}) to obtain for Stokes cocycles
a mapping
\[ \begin{tikzcd}
    h:\prod_{\alpha\in\A}\Sto_\alpha(A^0)
    \rar[hook]&
    \prod_{\alpha\in\A}\Lambda_\alpha(A^0)
    \rar{\text{(\ref{eq:mapStoToCocy})}}&
    \St(A^0),
\end{tikzcd} \]
which takes a Stokes cocycle to its corresponding cohomology class.
\begin{center}
  \begin{minipage}[t]{0.8\textwidth}
    \begin{tthm}\label{thm:mainThm2}
      The map
      \[ \begin{tikzcd}
          h:\prod_{\alpha\in\A}\Sto_\alpha(A^0) \rar& \St(A^0)
      \end{tikzcd} \]
      is a bijection and natural.
      \begin{s-rem}
        \marginnote{\cite[869]{Loday1994},\cite[Sec.III.3.3]{Loday1994}}
        Natural means that $h$ commutes to isomorphisms and constructions over
        systems or connections they represent.
      \end{s-rem}
    \end{tthm}
  \end{minipage}
\end{center}
From Theorem~\ref{thm:mainThm2} and Proposition~\ref{prop:representation} we
get the following corollary.
\begin{cor}\label{cor:isomToChochN}
  \PROBLEM[mentioned twice]
  Using the isomorphisms $\Sto_\theta(A^0)\cong\SSto_\theta(A^0)$ from
  Proposition~\ref{prop:representation} we obtain
  \[
    \St(A^0) \cong \prod_{\alpha\in\A}\SSto_\alpha(A^0)
  \]
  which endows $\St(A^0)$ with the structure of an unipotent Lie group with the
  finite complex dimension $N:=\dim_\C\St(A^0)$
  (cf.~\cite[Sec.III.1]{Loday1994}).
  This can be rewritten in the following way:
  \[
    N=\sum_{\alpha\in\A}\dim_\C\SSto_\alpha(A^0)
      =\sum_{\alpha\in\A}\sum_{q_j\myrel{\alpha}q_l}n_j\cdot n_l
      =\sum_{\substack{1\leq j,l\leq n\\j<l}}2\cdot\deg(q_j-q_l)\cdot
        n_j\cdot n_l \,.
  \]
  \begin{s-rem}
    This number $N$ is known to be the \emph{irregularity} of $[\End A^0]$.
  \end{s-rem}
\end{cor}
\begin{comment}
  \marginnote{\cite[880f]{Loday1994}}
  We have also two structures of a linear affine variety on the set $\St(A^0)$.
  \begin{rem}
    Let $\sto_\alpha(A^0)$ be the Lie algebra corresponding to
    $\Sto_\alpha(A^0)$. The exponential map\footnote{This is not the map $\exp$
    from Section~\ref{sec:mainThm1}.} induces an homomorphism
    $\exp:\sto_\alpha(A^0)\to\Sto_\alpha(A^0)$ and denote by $\ln=\exp^{-1}$
    the inverse map.
    \begin{enumerate}
      \item The \emph{tangent linear structure} is defined as \TODO
      \item Using the map
        \begin{align*}
          \sto_{\alpha}(A^0) &\overset{\id+\cdot}\longrightarrow
          \Sto_{\alpha}(A^0)
        \\\dot{f}_\alpha & \longmapsto \id+\dot{f}_\alpha \,.
        \end{align*}
    \end{enumerate}
  \end{rem}
\end{comment}

\begin{rem}
  To define the inverse map of $h$, one has to find in each cocycle in
  $\St(A^0)$ the Stokes cocycle. Loday-Richaud gives an algorithm in section
  II.3.4 of her paper~\cite{Loday1994}, which takes a cocycle over an arbitrary
  cyclic covering and outputs cohomologous Stokes cocycle and thus solves this
  problem.
\end{rem}

%%%%%%%%%%%%%%%%%%%%%%%%%%%%%%%%%%%%%%%%%%%%%%%%%%%%%%%%%%%%%%%%%%%%%%%%%%%%%%%
\subsection{Proof of Theorem~\ref{thm:mainThm2}}\label{sec:proofOfMatrixThm}
We will only look at the unramified case, for which we refer to
\cite[Sec.II.3]{Loday1994}.
The proof in the ramified case can be found in \cite[Sec.II.4]{Loday1994}.
We first have to introduce adequate coverings, which will be used in the proof.

%%%%%%%%%%%%%%%%%%%%%%%%%%%%%%%%%%%%%%%%%%%%%%%%%%%%%%%%%%%%%%%%%%%%%%%%%%%%%%%
\subsubsection{Adequate coverings}
\begin{defn}
  \marginnote{\cite[371]{Martinet1991} defines adapted coverings}
  Let $\star\in\{k,<k,\leq k,\dots\}$.
  A covering $\cU$ beyond which the inductive limit
  $\underset{\cU}{\underrightarrow{\lim}}H^1(\cU;\Lambda^\star(A^0))$ is
  stationary is said to be \emph{adequate} to describe
  $H^1(S^1;\Lambda^\star(A^0))$ or \emph{adequate} to $\Lambda^\star(A^0)$.
  \begin{comment}
    A covering $\cU$ is said to be \emph{adequate} to describe
    $H^1(S^1;\Lambda^\star(A^0))$ or \emph{adequate} to $\Lambda^\star(A^0)$ if
    for every element in
    $\underset{\cU}{\underrightarrow{\lim}}H^1(\cU;\Lambda^\star(A^0))$
    given by some covering $\cU'$ and an element of
    $\Gamma(\cU';\Lambda^\star(A^0))$
    there exists
    \begin{itemize}
      \item an element in $\Gamma(\cU;\Lambda^\star(A^0))$ and
      \item an common refinement of $\cU$ and $\cU'$
    \end{itemize}
    such that \PROBLEM[the elements are~?? on the refined covering.]
  \end{comment}

  In other words is a covering $\cU$ adequate, if and only if the quotient map
  \[
      \Gamma(\cU;\Lambda^\star(A^0))\longrightarrow H^1(S^1;\Lambda^\star(A^0))
  \]
  is surjective.\TODO[Proof? check??]
  \begin{comment}
    \cite[371]{Martinet1991} introduces the following definition
    \begin{s-defn}
      A covering $\cU$ is \emph{adapted} if every anti-Stokes direction is
      contained in exactly one element of the nerve $\dot\cU$.
    \end{s-defn}
  \end{comment}
\end{defn}
The following proposition is in Loday-Richaud's paper~\cite{Loday1994} given as
Proposition II.1.7. It contains a simple characterization, which will be used
to see, that our defined coverings are adequate.
\begin{prop}\label{prop:adeqCovCondition}
  \marginnote{\cite[Prop.II.1.7]{Loday1994}}
  Let $k\in\cK_\alpha$.
  \begin{s-defn}
    Let $\alpha\in\A^k$.
    An arc $U(\alpha,\frac{\pi}{k})$ is called a \emph{Stokes arc of level $k$
    at $\alpha$}.
  \end{s-defn}
  A cyclic covering $\cU=(U_j)_{j\in J}$, which satisfies
  \begin{einr}
    for every $\alpha\in\A^k$ contains the Stokes arc $U(\alpha,\frac{\pi}{k})$
    at least one arc $\dot U_j$ from the nerve $\dot\cU$ of $\cU$
  \end{einr}
  is adequate to $\Lambda^k(A^0)$.

  The covering $\cU$ is adequate to $\Lambda^{\leq k}(A^0)$ (resp.\
  $\Lambda^{<k}(A^0)$) if it is adequate to $\Lambda^{k'}(A^0)$ for every
  $k'\leq k$ (resp.\ $k'<k$).
\end{prop}

Let $k\in\cK$.
We want to define the three cyclic coverings $\cU^{k}$, $\cU^{\leq k}$ and
$\cU^{<k}$ which will be adequate to $\Lambda^k(A^0)$, $\Lambda^{\leq k}(A^0)$
and $\Lambda^{<k}(A^0)$. Furthermore will the coverings be comparable at the
different levels.

\begin{enumerate}
  \item The first covering $\cU^{k}=\{\dot U_\alpha^k\mid\alpha\in\A^k\}$
    is the cyclic covering with nerve
    \[
      \dot\cU^k:=
      \left\{\dot U_\alpha^k=U(\alpha,\frac{\pi}{k})\mid\alpha\in\A^k\right\}
    \]
    consisting of all Stokes arcs of level $k$ for anti-Stokes directions
    bearing the level $k$.
\end{enumerate}

\begin{rem}\label{rem:superSectors}
  Boalch introduces in his publications~\cite[19]{boalch}
  and~\cite[Def.1.23]{thboalch} the notion of \emph{supersectors}, they are in
  the case of a single level $k$, defined as follows:
  \begin{einr}
    write the anti-Stokes directions as $\A=\{\alpha_1,\dots,\alpha_\nu\}$
    arranged according to the clockwise ordering, then is the $i$-th
    supersector defined as the arc
    \[
      \hat\Sect^k_i:=
        \left(\alpha_i-\frac{\pi}{2k},\alpha_{i+1}+\frac{\pi}{2k}\right) \,.
    \]
  \end{einr}
  This yields a cyclic covering $(\hat\Sect^k_i)_{i\in\{1,\dots,\nu\}}$ whose
  nerve is exactly $\dot\cU^k$ defined above.
  \begin{comment}
    \begin{center}
      \begin{tikzpicture}[scale=3]
        \node[] (zero) at (0,0) {};
        \draw[blue] (zero) circle (1cm);

        \fill[fill=green!20!white] (0,0) -- ({cos( 15 )},{sin( 15 )}) arc
          (15:85:1) -- cycle;

        \fill[fill=red!60!black] (0,0) -- ({cos( 15 )*0.5},{sin( 15 )*0.5}) arc
          (15:45:0.5) -- cycle;
        \fill[fill=red!60!black] (0,0) -- ({cos( 55 )*0.5},{sin( 55 )*0.5}) arc
          (55:85:0.5) -- cycle;

        \fill[fill=green!20!white] (0,0) -- ({cos( 15 )*0.4},{sin( 15 )*0.4}) arc
          (15:85:0.4) -- cycle;

        \node[green!40!black] at (1,1) {$\widehat\Sect_i$};

        \node[] (lft) at ({cos( 85 )},{sin( 85 )}) {};
        \node[] (rgt) at ({cos( 15 )},{sin( 15 )}) {};

        \draw[->,red!40!black] ({cos( 45 )},{sin( 45 )})
          to [out=35, in=25] (rgt);

        \draw[->,red!40!black] ({cos( 55 )},{sin( 55 )})
          to [out=65, in=75] (lft);

        \draw[thick,red!40!black,path fading=west] (0,0) -- +({cos( 85 )},{sin( 85 )});
        \draw[thick,red!40!black,path fading=west] (0,0) -- +({cos( 15 )},{sin( 15 )});
        \fill[red!40!black] ({cos( 85 )},{sin( 85 )}) circle (1pt);
        \fill[red!40!black] ({cos( 15 )},{sin( 15 )}) circle (1pt);

        \foreach \w/\str in {10/,
                             20/,
                             45/$\alpha_{i+1}$,
                             55/$\alpha_{i}$}
        {\draw (0,0) -- +({cos( \w )},{sin( \w )}) node[right] {\str};
         \fill[blue!20!white] ({cos( \w )},{sin( \w )}) circle (.7pt);
         \foreach \sep in {60,120,180,240,300}
         {\draw (0,0) -- +({cos( \w + \sep )},{sin( \w + \sep )});
          \fill[blue!20!white]
            ({cos( \w + \sep )},{sin( \w + \sep )}) circle (.7pt);
         }
        };


        \fill (zero) circle (1pt);
      \end{tikzpicture}
    \end{center}
  \end{comment}
\end{rem}
If we extend to more then one level level, $\#\cK>1$, the set
$\bigcup_{k\in\cK}\left\{U(\alpha,\frac{\pi}{k})\mid\alpha\in\A^{k}\right\}$ is
no longer a nerve.
Hence we have to define the coverings $\cU^{\leq k}$ and $\cU^{<k}$ in a
different way.
Denote by
\[
  \left\{K_1<\cdots<K_s=k\right\}
  =\left\{\max\left(\cK_\alpha\cap[0,k]\right)\mid\alpha\in\A^{\leq k}\right\}
\]
the set of all \emph{$k$-maximum levels} for $\alpha\in\A^{\leq k}$.
\begin{enumerate}
  \setcounter{enumi}{1}
  \item The cyclic covering
    $\cU^{\leq k}=\left\{U_\alpha^{\leq k}\mid\alpha\in\A^{\leq k}\right\}$
    will be defined by induction.
    Let us assume that
    \begin{einr}
      the $\dot U_\alpha^{\leq k}$ are defined for all $\alpha\in\A^{\leq k}$
      with $k$-maximum level greater than $K_i$ such that their complete family
      is a nerve.
    \end{einr}
    Let
    \begin{itemize}
      \item $\alpha$ be a anti-Stokes direction with $k$-maximum level $K_i$
        and
      \item $\alpha^-$ (resp.\ $\alpha^+$) be the next anti-Stokes direction
        with $k$-maximum level greater then $K_i$ on the left (resp.\ on the
        right) and define $\dot U_{\alpha^-,\alpha^+}$ as the clockwise hull of
        the arcs $\dot U_{\alpha^-}^{\leq k}$ and $\dot U_{\alpha^+}^{\leq k}$
        already defined by induction.
        If there are no anti-Stokes directions with $k$-maximum level greater
        then $K_i$ we set $\dot U_{\alpha^-,\alpha^+}=S^1$.
    \end{itemize}
    We then set
    \[
      \dot U_\alpha^{\leq k}
        :=U\left(\alpha,\frac{\pi}{K_i}\right)\cap\dot U_{\alpha^-,\alpha^+}
    \]
    and the family of all $\dot U_\alpha^{\leq k}$ is a nerve.
    \begin{rem}
      If $\alpha$ has a $k$-maximum level equal to $k$ then is
      $\dot U_\alpha^{\leq k}$ equal to the Stokes arc
      $U\left(\alpha,\frac{\pi}{k}\right)=\dot U_\alpha^k$.
      \comm{\dots{}and then no $0$-cochain with level $k$ or $\geq k$ can
      exists on the covering $\cU^{\leq{k}}$.}
    \end{rem}
\end{enumerate}

\begin{enumerate}
  \setcounter{enumi}{2}
\item The last cyclic covering,
  $\cU^{<k}=\left\{U_{\alpha}^{<k}\mid\alpha\in\A^{<k}\right\}$, is defined as
  $\cU^{<k}:=\cU^{\leq k'}$ where $k':=\max\{k''\in\cK\mid k''<k\}$.
\end{enumerate}

\begin{figure} %{{{
  \centering

  \def\kOne{7}
  \def\kTwo{10}
  \begin{tikzpicture}[scale=5] %{{{

    \newcommand{\myDrawArcA}[4]{%{{{
      % Parameter: radius , center , width , color
      \pgfmathsetmacro\hwdth{#3 / 2}
      \draw[#4]
        ({cos( #2 )},{sin( #2 )})
        --
        ({cos( #2 ) * #1 },{sin( #2 ) * #1 });
      \draw[ultra thick,#4]
        ({cos(#2 - \hwdth) * #1},{sin(#2 - \hwdth) * #1})
        arc
        ({#2 - \hwdth}:{#2 + \hwdth}:#1);
      \draw[dotted,#4]
        ({cos(#2 - \hwdth) * #1},{sin(#2 - \hwdth) * #1})
        --
        ({cos(#2 - \hwdth)},{sin(#2 - \hwdth)});
      \draw[dotted,#4]
        ({cos(#2 + \hwdth) * #1},{sin(#2 + \hwdth) * #1})
        --
        ({cos(#2 + \hwdth)},{sin(#2 + \hwdth)});
      \filldraw[white] ({cos(#2)},{sin(#2)}) circle (0.5pt);
      \filldraw[red] ({cos(#2)},{sin(#2)}) circle (0.2pt);
    }%}}}
    \newcommand{\myDrawArcB}[5]{%{{{
      % Parameter: radius , center , start , stop , color
      \draw[#5]
        ({cos( #2 )},{sin( #2 )})
        --
        ({cos( #2 ) * #1 },{sin( #2 ) * #1 });
      \draw[ultra thick,#5]
        ({cos( #3 ) * #1},{sin( #3 ) * #1})
        arc
        ({ #3 }:{ #4 }:#1);
      \draw[dotted,#5]
        ({cos( #3 ) * #1},{sin( #3 ) * #1})
        --
        ({cos( #3 )},{sin( #3 )});
      \draw[dotted,#5]
        ({cos( #4 ) * #1},{sin( #4 ) * #1})
        --
        ({cos( #4 )},{sin( #4 )});
      \filldraw[white] ({cos(#2)},{sin(#2)}) circle (0.5pt);
      \filldraw[red] ({cos(#2)},{sin(#2)}) circle (0.2pt);
    }%}}}

    \node (zero) at (0,0) {};
    \draw (zero) circle (1cm);

    %%%%%%%%%%%%%%%%%%%%%%%%%%%%%%%%%%%%%%%%%%%%%%%%%%%%%%%%%%%%%%%%%%%%%%%%%
    %% Inner:
    \foreach \n/\mycol/\baseR in {\kOne/orange/0.8
                                 ,\kTwo/purple/0.9}
    {%{{{
      \pgfmathsetmacro\wdth{180/\n}
      \foreach \i in {1,2,...,\n}}}

    %%%%%%%%%%%%%%%%%%%%%%%%%%%%%%%%%%%%%%%%%%%%%%%%%%%%%%%%%%%%%%%%%%%%%%%%%
    %% Outer:
    \def\mycol{brown}
    \pgfmathsetmacro{\wdth}{180/\kTwo}
    \foreach \i in {1,2,...,\kTwo}{%
      \pgfmathsetmacro\r{{1.05 + mod(\i+1,2)*0.05}}
      \pgfmathsetmacro\angl{\i* \wdth}
      \myDrawArcA{\r}{\angl}{\wdth}{\mycol}

      \pgfmathsetmacro\r{{1.05 + mod(\i + mod(\kTwo+1,2),2)*0.05}}
      \pgfmathsetmacro\angl{\i* \wdth+180}
      \myDrawArcA{\r}{\angl}{\wdth}{\mycol}
    }
    \foreach \i in {1,2,...,\kOne}{%
      \foreach \j in {1,...,\kTwo}{%
        \pgfmathparse{\i/\kOne <= \j/\kTwo ? 0 : 1}
        \ifnum\pgfmathresult=0{%
          \pgfmathparse{\i/\kOne < \j/\kTwo ? 0 : 1}
          \ifnum\pgfmathresult=0{%
            \pgfmathsetmacro\r{{1.15 + mod(\i+1,2)*0.05}}
            \pgfmathsetmacro\center{\i/\kOne*180}
            \pgfmathsetmacro\start{(\i-0.5)*180/\kOne > (\j-1.5)*180/\kTwo
              ? (\i-0.5)*180/\kOne : (\j-1.5)*180/\kTwo}
            \pgfmathsetmacro\stop{(\i+0.5)*180/\kOne > (\j+0.5)*180/\kTwo
              ? (\j+0.5)*180/\kTwo : (\i+0.5)*180/\kOne}
            \myDrawArcB{\r}{\center}{\start}{\stop}{\mycol}

            \pgfmathsetmacro\r{{1.15 + mod(\i,2)*0.05}}
            \pgfmathsetmacro\center{\center+180}
            \pgfmathsetmacro\start{\start+180}
            \pgfmathsetmacro\stop{\stop+180}
            \myDrawArcB{\r}{\center}{\start}{\stop}{\mycol}
          }\fi
          \breakforeach
        }\fi
      }
    }

    \fill[white] (zero) circle (1.5pt);
    \fill (zero) circle (.7pt);
  \end{tikzpicture} %}}}
  \caption{The adequate coverings for an example with $\cK=\{\kOne,\kTwo\}$ and
    $\A=\left\{ \frac{j\cdot\pi}{k}\mid k\in\cK\text{, } j\in\N \right\}$.
    The anti-Stokes directions are marked by the \textcolor{red}{red} dots.
    The arcs of $\dot\cU^{\kOne}=\dot\cU^{\leq\kOne}$ are
    \textcolor{orange}{orange}, the arcs of $\dot\cU^{\kTwo}$ are
    \textcolor{purple}{purple} and the arcs of $\dot\cU^{\leq\kTwo}=\dot\cU$
    are \textcolor{brown}{brown}.
  }\label{fig:adequateCovering}
\end{figure}%}}}

\begin{rem}
  The coverings $\cU^{k}$, $\cU^{\leq k}$ and $\cU^{<k}$ depend only on
  $\cQ(A^0)$. Hence they depend only on the determining polynomials.
\end{rem}
It is obvious, that for every $k\in\cK$ the covering $\cU^{\leq k}$ refines
$\cU^{k}$ and $\cU^{<k}$.
Furthermore are the coverings defined, such that they satisfy the
condition in Proposition~\ref{prop:adeqCovCondition}.
Thus the first property in the following proposition is satisfied. The other
two can be found at \cite[Prop.II.3.1 (iv)]{Loday1994}.
\begin{prop}\label{prop:adequateProperties}
  \marginnote{\cite[Prop.II.3.1]{Loday1994}}
  \PROBLEM[Was bedeutet das?]
  Let $k\in\cK$, then
  \begin{enumerate}
    \item the coverings $\cU^{k}$, $\cU^{\leq k}$ and $\cU^{<k}$ are adequate
      to $\Lambda^k(A^0)$, $\Lambda^{\leq k}(A^0)$ and $\Lambda^{<k}(A^0)$, respectively;
    \item there exists no $0$-cochain in $\Lambda^k(A^0)$ on $\cU^k$;
    \item on $\cU^{\leq k}$ there is no $0$-cochain in $\Lambda^{\leq k}(A^0)$
      of level $k$, i.e.\ all $0$-cochains of $\Lambda^{\leq k}(A^0)$ belong to
      $\Lambda^{<k}(A^0)$.
  \end{enumerate}
\end{prop}
To have a shorter notation, we denote the product
$\prod_{\alpha\in\A^\star}\Gamma(\dot U_\alpha^\star;\Lambda^\star(A^0))$ by
$\Gamma(\dot U^\star;\Lambda^\star(A^0))$ for every
$\star\in\{k,<k,\leq k,\dots\}$.

%%%%%%%%%%%%%%%%%%%%%%%%%%%%%%%%%%%%%%%%%%%%%%%%%%%%%%%%%%%%%%%%%%%%%%%%%%%%%%%
\subsubsection{The case of a unique level}
\marginnote{\cite[II.3.2]{Loday1994}}
First we will proof Theorem~\ref{thm:mainThm2} in the case of a unique level.
This means that
\begin{itemize}
  \item either $\Lambda(A^0)$ has only one level $k$, thus
    \begin{itemize}
      \item $\Lambda(A^0)=\Lambda^k(A^0)$ and
      \item $\Sto_\theta(A^0)=\Sto_\theta^k(A^0)$ for every $\theta$,
    \end{itemize}
  \item or we restrict to a given level $k\in\cK$.
\end{itemize}
\begin{lem}
  Let $k\in\cK$.
  The morphism $h$ from Theorem~\ref{thm:mainThm2} is in the case of an unique
  level build as
  \[ \begin{tikzcd}[row sep=0cm]
    \underset{\alpha\in\A}\prod\Sto_\alpha^k(A^0)
    \rar{i^k}
    & \Gamma(\dot\cU^k;\Lambda^k(A^0))
    \rar{s^k}
    & H^1(S^1;\Lambda^k(A^0))
    \\
    \text{\rotatebox[origin=c]{-90}{$=$}}
    \tikzmark{e1}
    &&
    \tikzmark{e2}
    \text{\rotatebox[origin=c]{-90}{$=$}}
    \\
    \underset{\alpha\in\A}\prod\Sto_\alpha(A^0)
    \arrow{rr}{h}
    && H^1(S^1;\Lambda(A^0))
  \end{tikzcd} \]
  \begin{flushright}
    \tikzmarkc{n1}{blue} only in the single leveled case
    \begin{tikzpicture}[remember picture,overlay]
      \draw[->,blue!50!white,thick] (n1) to[out=180,in=0] (e1);
      \draw[->,blue!50!white,thick] (n1) to[out=180,in=180,distance=3cm] (e2);
    \end{tikzpicture}
  \end{flushright}
  from
  \begin{itemize}
    \item the canonical injective map
      \[ \begin{tikzcd}
        i^k:\underset{\alpha\in\A}\prod\Sto_\alpha^k(A^0) \rar&
        \Gamma(\dot\cU^k;\Lambda^k(A^0)) \,,
      \end{tikzcd} \]
      i.e.\ the map which is the canonical extension of germs to their natural
      arc of definition, and
    \item the quotient map
      \[ \begin{tikzcd}
        s^k:\Gamma(\dot\cU^k;\Lambda^k(A^0)) \rar& H^1(S^1;\Lambda^k(A^0))
      \end{tikzcd} \]
  \end{itemize}
  which are both isomorphisms.
\end{lem}
\begin{proof}
  \begin{enumerate}
    \item The map
      \[ \begin{tikzcd}
        i^k: \underset{\alpha\in\A}\prod\Sto_\alpha^k(A^0)
        \rar &
        \underset{\Gamma(\dot\cU^k;\Lambda^k(A^0))}{%
          \underset{\text{\rotatebox[origin=c]{-90}{$=$}}}{%
            \underbrace{%
              \prod_{\alpha\in\A^k}\Gamma(\dot\cU_\alpha^k;\Lambda^k(A^0))
            }
        }}
      \end{tikzcd} \]
      is welldefined, since the sections of $\Lambda^k(A^0)$ are solution of
      the system $[A^0,A^0]$ and it is very well known from the theory of
      differential equations\TODO[source?] that an element
      $f_\alpha\in\Gamma(\dot\cU_\alpha^k;\Lambda^k(A^0))$ is uniquely
      determined as the extension of its germ at some point $\alpha$.

      It is also a group isomorphism (cf.~\cite{Loday1994}).

      \PROBLEM[$\Sto_\alpha^k(A^0)\subsetneq\Lambda_\alpha^k(A^0)$ SEE bjl p.72]
      \begin{comment}
        It is a group isomorphism, since
        \\Problems:
        \begin{itemize}
          \item Show that a element of $\Lambda_\alpha^k(A^0)$ is extensionable
            to the arc $U_\alpha\in\dot\cU_\alpha^k$ if and only if it has
            maximal decay in direction $\alpha$.
          \item \PROBLEM[$\Sto_\alpha^k(A^0)\subsetneq\Lambda_\alpha^k(A^0)$]
            and thus
            \[
              i^k:
              \underset{\alpha\in\A}\prod\Sto_\alpha^k(A^0)
              \subsetneq
              \underset{\alpha\in\A}\prod\Lambda_\alpha^k(A^0)
              \to
              \prod_{\alpha\in\A^k}\Gamma(\dot\cU_\alpha^k;\Lambda^k(A^0))
            \]
        \end{itemize}
        IDEAS:
        \begin{itemize}
          \item Are sections fully determined by there germs at some points?
          \item $U_\alpha^k$ is the largest arc, to contain no corresponding
            Stokes ray
            \begin{itemize}
              \item Then might \cite[Lemma 1]{BJL1979Birkhoff} on p.\ 73 help
            \end{itemize}
          \item Every arc $U\in\dot\cU_\alpha^k$ has the width $\frac{\pi}{k}$
            and is delimited by Stokes directions.
          \item \textbf{See \cite{babbitt1989local}}
          \item See \cite[375]{Martinet1991} and
            \textbf{\cite[Def.5 on 372]{Martinet1991}}
          \item \textbf{\textcolor{green}{\cite[72]{babbitt1989local}}}
          \item or maybe \cite{wasow2002asymptotic}
        \end{itemize}
      \end{comment}
    \item The second map
      \[ \begin{tikzcd}
        s^k:
        \overset{\Gamma(\dot\cU^k;\Lambda^k(A^0))}{%
          \overset{\text{\rotatebox[origin=c]{-90}{$=$}}}{%
            \overbrace{%
              \prod_{\alpha\in\A^k}\Gamma(\dot\cU_\alpha^k;\Lambda^k(A^0))
            }
        }}
        \rar &
        H^1(S^1;\Lambda^k(A^0))
      \end{tikzcd} \]
      is a bijection, \rewrite{since} from
      Proposition~\ref{prop:adequateProperties} we know that it is
      \begin{itemize}
        \item \textbf{surjective}, \rewrite{since} $\cU^k$ is adequate to
          $\Lambda^k(A^0)$ and
        \item \textbf{injective}, \rewrite{since} on $\cU^k$ there is no
          $0$-cochain in $\Lambda^k(A^0)$.
      \end{itemize}
  \end{enumerate}
  \PROBLEM[Show that this is the correct $h$]
  \PROBLEM[Naturality?]
\end{proof}

%%%%%%%%%%%%%%%%%%%%%%%%%%%%%%%%%%%%%%%%%%%%%%%%%%%%%%%%%%%%%%%%%%%%%%%%%%%%%%%
\subsubsection{The case of several levels}
\rewrite{In the proof of the case of several levels, we will still use}
Loday-Richaud's paper~\cite{Loday1994} as reference.
\begin{defn}\label{defn:firstSetOfInlusions}
  Here we want to define a \emph{product map of cocycles}
  $\mathfrak{S}^{\leq k}$.
  This map will be composed from the following injective maps:
  \begin{enumerate}
    \item The first map is defined as
      \begin{align*}
        \sigma^k:\Gamma(\dot\cU^k;\Lambda^k(A^0))
        &\longrightarrow \Gamma(\dot\cU^{\leq k};\Lambda^{\leq k}(A^0))
      \\\dot f=(\dot f_\alpha)_{\alpha\in\A^k}
        &\longmapsto (\dot G_\alpha)_{\alpha\in\A^{\leq k}}
      \end{align*}
      where
      \[
        \dot G_\alpha=\begin{cases}
          \dot f_\alpha \text{~restricted to } \dot U_\alpha^{\leq k}
          \text{~and seen as being in } \Lambda^{\leq k}(A^0)
          & \text{~when } \alpha\in\A^k
        \\\id \text{~(the identity) }
          & \text{~when } \alpha\notin\A^k
        \end{cases}
      \]
    \item and the second map
      \begin{align*}
        \sigma^{<k}:\Gamma(\dot\cU^{<k};\Lambda^{<k}(A^0))
        &\longrightarrow \Gamma(\dot\cU^{\leq k};\Lambda^{\leq k}(A^0))
      \\\dot f=(\dot f_\alpha)_{\alpha\in\A^{<k}}
        &\longmapsto (\dot F_\alpha)_{\alpha\in\A^{\leq k}}
      \end{align*}
      is defined, in a similar way, as
      \[
        \dot F_\alpha=\begin{cases}
          \dot f_\alpha \text{~restricted to } \dot U_\alpha^{\leq k}
          \text{~and seen as being in } \Lambda^{\leq k}(A^0)
          & \text{~when } \alpha\in\A^{<k}
        \\\id \text{~(the identity) }
          & \text{~when } \alpha\notin\A^{<k}
        \end{cases}
      \]
  \end{enumerate}
  Thus we can define
  \begin{align*}
    \mathfrak{S}^{\leq k}:
    \Gamma(\dot\cU^{<k};\Lambda^{<k}(A^0))
    \times
    \Gamma(\dot\cU^{k};\Lambda^{k}(A^0))
    &\longrightarrow
    \Gamma(\dot\cU^{\leq k};\Lambda^{\leq k}(A^0))
  \\(\dot f, \dot g)
    &\longmapsto
    (\dot F_\alpha\dot G_\alpha)_{\alpha\in\A^{\leq k}}
  \end{align*}
  where $(\dot F_\alpha)_{\alpha\in\A^{\leq k}}=\sigma^{<k}(\dot f)$ and
  $(\dot G_\alpha)_{\alpha\in\A^{\leq k}}=\sigma^k(\dot g)$ are defined as
  above.
  \begin{s-rem}
    This map $\mathfrak{S}^{\leq k}$ is injective, since injectivity for germs
    implies injectivity for sections.
  \end{s-rem}
\end{defn}

\begin{lem}
  \marginnote{\cite[Lem.II.3.3]{Loday1994}}
  Let $k\in\cK$.
  \begin{enumerate}
    \item If the cocycles $\mathfrak{S}^{\leq k}(\dot f,\dot g)$ and
      $\mathfrak{S}^{\leq k}(\dot f',\dot g')$ are cohomologous in
      $\Gamma(\dot\cU^{\leq k};\Lambda^{\leq k}(A^0))$
      then $\dot f$ and $\dot f'$ are cohomologous in
      $\Gamma(\dot\cU^{<k};\Lambda^{<k}(A^0))$.
    \item Any cocycle in $\Gamma(\dot\cU^{\leq k},\Lambda^{\leq k}(A^0))$ is
      cohomologous to a cocycle in the range of $\mathfrak{S}^{\leq k}$.
  \end{enumerate}
\end{lem}
\begin{proof}
  \begin{enumerate}
    \item Denote by $\alpha^+$ the nearest anti-Stokes direction in
      $\A^{\leq k}$ on the right\footnote{In clockwise direction.} of $\alpha$.
      The cocycles $\mathfrak{S}^{\leq k}(\dot f,\dot g)$ and
      $\mathfrak{S}^{\leq k}(\dot f',\dot g')$ are cohomologous if and only
      if there is a $0$-cochain $c=(c_\alpha)_{\alpha\in\A^{\leq k}}
      \in\Gamma(\cU^{\leq k},\Lambda^{\leq k}(A^0)$ such that
      \begin{equation}\label{eq:UniqueString:urdtindfgupndtcn}
        \dot F_\alpha\dot G_\alpha =
        c_\alpha^{-1}\dot F_\alpha'\dot G_\alpha'c_{\alpha^+}
      \end{equation}
      for every $\alpha\in\A$. From Proposition~\ref{prop:adequateProperties}
      follows, that $c$ is with values in $\Lambda^{<k}(A^0)$.
      The fact that $\Lambda^k(A^0)$ is normal in $\Lambda^{\leq k}(A^0)$ in
      Proposition~\ref{prop:PropertiesOfStokesSheafSplitting}, can be used to
      see that $ c_{\alpha^+}^{-1}G_\alpha'c_{\alpha^+}
      \in\Gamma(\cU^{\leq k};\Lambda^{k}(A^0))$.
      Thus, we rewrite the relation (\ref{eq:UniqueString:urdtindfgupndtcn}) to
      \[
        \dot F_\alpha\dot G_\alpha =
        (c_\alpha^{-1}\dot F_\alpha'c_{\alpha^+})
        (c_{\alpha^+}^{-1}G_\alpha'c_{\alpha^+})
        \,,\qquad\text{~for~} \alpha\in\A^{\leq k} \,.
      \]
      Since Corollary~\ref{cor:factorStokesGerms} tells us, that the
      factorization into the factors of the semidirect product are unique, we
      get for all $\alpha\in\A^k$
      \[
        \dot F_\alpha=c_\alpha^{-1}\dot F_\alpha'c_{\alpha^+}
        \qquad \text{~and~} \qquad
        \dot G_\alpha=c_{\alpha^+}^{-1}\dot G_\alpha'c_{\alpha^+}.
      \]
      The former relation implies that $(\dot F_\alpha)$ and $(\dot F_\alpha')$
      are cohomologous with values in $\Lambda^{<k}(A^0)$ on $\cU^{\leq k}$.
      Since $\cU^{<k}$ is already adequate to $\Lambda^{<k}(A^0)$, \rewrite{are
      $(\dot F_\alpha)$ and $(\dot F_\alpha')$} already on $\cU^{<k}$,
      i.e.\ in $\Gamma(\dot\cU^{<k};\Lambda^{<k}(A^0))$, cohomologous.
    \item The proof of part 2.\ (together with a proof of part 1.) can be
      found in Loday-Richaud's paper \cite[Proof of Lem.II.3.3]{Loday1994}.
  \end{enumerate}
\end{proof}
Let $k\in\cK$ and $k'=\max\{k'\in\cK\mid k'<k\}$. We then know by definition
that $\cU^{<k}=\cU^{\leq k'}$ as well as
$\Lambda^{< k}(A^0)=\Lambda^{\leq k'}(A^0)$ and thus
$\Gamma(\dot\cU^{< k};\Lambda^{< k}(A^0))=
\Gamma(\dot\cU^{\leq k'};\Lambda^{\leq k'}(A^0))$ and obtain the following
proposition.
\begin{prop}\label{prop:theMapTau}
  By applying $\mathfrak{S}^{\leq k}$ successively for different $k$'s
  in decending order, one obtains the \emph{product map of single leveled
  cocycles $\tau$} in the following way
  \[ \begin{tikzcd}[column sep=1.4cm,row sep=.9cm]
      \underset{\tikzmark{e1}}{\underbrace{%
        \Gamma(\dot\cU^{<k_r};\Lambda^{<k_r}(A^0))}}
      \times
      \Gamma(\dot\cU^{k_r};\Lambda^{k_r}(A^0))
      \rar{\mathfrak{S}^{\leq k_r}}&
      \overset{\Gamma(\dot\cU;\Lambda(A^0))}{%
        \overset{\text{\rotatebox[origin=c]{-90}{$=$}}}{%
          \Gamma(\dot\cU^{\leq k_r};\Lambda^{\leq k_r}(A^0))
      }}
      \\\underset{\tikzmark{e2}}{\underbrace{%i
        \Gamma(\dot\cU^{<k_{r-1}};\Lambda^{<k_{r-1}}(A^0))}}
      \times
      \Gamma(\dot\cU^{k_{r-1}};\Lambda^{k_{r-1}}(A^0))
      \rar{\mathfrak{S}^{\leq k_{r-1}}}&
      \overset{\tikzmark{n1}}{%
        \Gamma(\dot\cU^{\leq k_{r-1}};\Lambda^{\leq k_{r-1}}(A^0))}
      \\\hspace{6cm}\cdots \rar{\mathfrak{S}^{\leq k_{r-2}}}&
      \overset{\tikzmark{n2}}{%
        \Gamma(\dot\cU^{\leq k_{r-2}};\Lambda^{\leq k_{r-2}}(A^0))}
      \\\underset{\tikzmark{eEND}}{\underbrace{%
        \Gamma(\dot\cU^{<k_3};\Lambda^{<k_3}(A^0))}}
      \times
      \Gamma(\dot\cU^{k_3};\Lambda^{k_3}(A^0))
      \rar{\mathfrak{S}^{\leq k_3}}&
      \cdots\hspace{3cm}
      \\
      \Gamma(\dot\cU^{k_1};\Lambda^{k_1}(A^0))
      \times
      \Gamma(\dot\cU^{k_2};\Lambda^{k_2}(A^0))
      \rar{\mathfrak{S}^{\leq k_2}}&
      \overset{\tikzmark{nEND}}{%
        \Gamma(\dot\cU^{\leq k_2};\Lambda^{\leq k_2}(A^0))}
  \end{tikzcd} \]
  \begin{tikzpicture}[remember picture,overlay]
    \draw[<-] (n1) to[out=90,in=270] node[midway,fill=white]{$=$}
      ([yshift=.3em]e1);
    \draw[<-] (n2) to[out=90,in=270] node[midway,fill=white]{$=$}
      ([yshift=.3em]e2);
    \draw[<-] (nEND) to[out=90,in=270] node[midway,fill=white]{$=$}
      ([yshift=.3em]eEND);
  \end{tikzpicture}
  which can be written in the following compact form
  \begin{align*}
    \tau:\prod_{k\in\cK}\Gamma(\dot \cU^k;\Lambda^k(A^0))
    &\longrightarrow
    \Gamma(\dot\cU;\Lambda(A^0))
  \\(\dot f^k)_{k\in\cK}
    &\longmapsto
    \prod_{k\in\cK}\tau^k(\dot f^k)
  \end{align*}
  where the product is following an ascending order of levels and the maps
  $\tau_k$ are defined as
  \[ \begin{tikzcd}[row sep=0cm]
    \tau^k:\Gamma(\dot\cU^k;\Lambda^k(A^0))
    \arrow{r}{\sigma^k}&
    \Gamma(\dot\cU^{\leq k};\Lambda^{\leq k}(A^0))
    \rar &
    \Gamma(\dot\cU;\Lambda(A^0))
  \\~~~(\dot f_\alpha)_{\alpha\in\A^k}
    \arrow[|->]{rr}
    &&
    (\dot G_\alpha)_{\alpha\in\A}
  \end{tikzcd} \]
  with
  \[
    \dot G_\alpha=\begin{cases}
      \dot f_\alpha \text{~restricted to } \dot U_\alpha
      \text{~and seen as being in } \Lambda(A^0)
      & \text{~when } \alpha\in\A^k
    \\\id \text{~(the identity on $\dot U_\alpha$) }
      & \text{~when } \alpha\notin\A^k
    \end{cases}
  \]
  The defined map $\tau$ is clearly injective and it can be extended to an
  arbitrary order of levels (cf.\ Remark~\cite[Rem.II.3.5]{Loday1994}).
\end{prop}
\begin{comment}
  \begin{defn}\label{defn:theMapTau}
    Define the injective map
    \begin{align*}
      \tau^k:\Gamma(\dot\cU^k;\Lambda^k(A^0))
      &\longrightarrow \Gamma(\dot\cU;\Lambda(A^0))
    \\\dot f=(\dot f_\alpha)_{\alpha\in\A^k}
      &\longmapsto
      (\dot F_\alpha)_{\alpha\in\A}
    \end{align*}
    where
    \[
      \dot F_\alpha=\begin{cases}
        \dot f_\alpha \text{~restricted to } \dot U_\alpha
        \text{~and seen as being in } \Lambda(A^0)
        & \text{~when } \alpha\in\A^k
      \\\id \text{~(the identity on $\dot U_\alpha$) }
        & \text{~when } \alpha\notin\A^k
      \end{cases}
    \]
    \marginnote{\cite[Prop.II.3.4]{Loday1994}}
    The \emph{product map of single-leveled cocycles} is then defined as
    \begin{align*}
      \tau:\prod_{k\in\cK}\Gamma(\dot \cU^k;\Lambda^k(A^0))
      &\longrightarrow
      \Gamma(\dot\cU;\Lambda(A^0))
    \\(\dot f^k)_{k\in\cK}
      &\longmapsto
      \prod_{k\in\cK}\tau^k(\dot f^k)
    \end{align*}
    following an ascending order of levels.
    \begin{s-rem}
      The map $\tau$
      \begin{enumerate}
        \item is injective since it is composed from $\sigma^k$ and the clearly
          injetive mapping
          \begin{align*}
            \Gamma(\dot\cU^{\leq k};\Lambda^{\leq k}(A^0))
            &\longrightarrow
            \Gamma(\dot\cU;\Lambda(A^0))
          \\(\dot F_\alpha)_{\alpha\in\A^{\leq k}}
            &\longmapsto
            (\dot F'_\alpha)_{\alpha\in\A}
          \end{align*}
          where
          \[
            \dot F'_\alpha=\begin{cases}
              \dot F_\alpha \text{~restricted to } \dot U_\alpha
              \text{~and seen as being in } \Lambda(A^0)
              & \text{~when } \alpha\in\A^{\leq k}
              \\\id \text{~(the identity on $\dot U_\alpha$) }
              & \text{~when } \alpha\notin\A^{\leq k}
            \end{cases}
          \]
          and
        \item it can be extended to an arbitrary order of levels
          (cf.\ Remark~\cite[Rem.II.3.5]{Loday1994}).
      \end{enumerate}
    \end{s-rem}
  \end{defn}
\end{comment}
\begin{cor}
  The product map of single-leveled cocycles $\tau$ induces on the cohomology
  a bijective and natural map
  \[ \begin{tikzcd}
    \cT:
    \underset{\underset{k\in\cK}\prod H^1(S^1;\Lambda^k(A^0))}{%
      \underset{\text{\rotatebox[origin=c]{-90}{$\cong$}}}{%
        \prod_{k\in\cK}\Gamma(\dot\cU^k;\Lambda^k(A^0))}}
    \rar&
    \underset{H^1(S^1;\Lambda(A^0))}{%
      \underset{\text{\rotatebox[origin=c]{-90}{$\cong$}}}{%
        H^1(\cU;\Lambda(A^0))}}\,.
  \end{tikzcd} \]
\end{cor}

%%%%%%%%%%%%%%%%%%%%%%%%%%%%%%%%%%%%%%%%%%%%%%%%%%%%%%%%%%%%%%%%%%%%%%%%%%%%%%%
\paragraph{Composing functions to obtain $h$}
We have the ingredients to define the function $h$ from
Theorem~\ref{thm:mainThm2} by composition of already bijective maps.
\begin{proof}[Proof of Theorem~\ref{thm:mainThm2}]
  Let $i_\alpha:\Sto_\alpha(A^0)\to\prod_{k\in\cK}\Sto_\alpha^k(A^0)$ be the
  map which corresponds to the filtration from
  Proposition~\ref{prop:filtrationOfStokesGroup} and
  denote the composition
  \[ \begin{tikzcd}[column sep=1.8cm,row sep=0]
      \displaystyle \prod_{\alpha\in\A}\Sto_\alpha(A^0)
      \rar{\prod_{\alpha\in\A}i_\alpha}
      \arrow[ddrr, out=270,in=200,"\mathfrak{T}"]
      &
      \displaystyle \prod_{\alpha\in\A}\prod_{k\in\cK}\Sto_\alpha^k(A^0)
    \\&\text{\rotatebox[origin=c]{-90}{$\equiv$}}
    \\&\displaystyle \prod_{k\in\cK}\prod_{\alpha\in\A}\Sto_\alpha^k(A^0)
      \rar{\prod_{k\in\cK}i^k}&
      \displaystyle \prod_{k\in\cK}\Gamma(\dot\cU^k;\Lambda^k(A^0))
  \end{tikzcd} \]
  by $\mathfrak{T}$. The bijection $h$ is then obtained as
  \[
    \cT\circ\mathfrak{T}: \prod_{\alpha\in\A}\Sto_\alpha(A^0)
    \longrightarrow H^1(\cU;\Lambda(A^0)) \,.
  \]
  \PROBLEM[naturality (is obvious?)]
  \PROBLEM[Show that this is the correct $h$]
\end{proof}

%%%%%%%%%%%%%%%%%%%%%%%%%%%%%%%%%%%%%%%%%%%%%%%%%%%%%%%%%%%%%%%%%%%%%%%%%%%%%%%
\subsection{Some exemplary calculations}\label{sec:WhichInformationIsNeeded}
Here we want to discuss, which information is required to describe the Stokes
cocycle corresponding to a multileveled system in more depth.
We will look at a sigle-leveled system corresponding to a normal form
$A^0\in \GL_3(\C(\!\{t\}\!))$ with exactly $2$ levels and will apply the
techniques developed in the previous sections in an rather explicit way.

Let $A^0$ be a normal form with dimension $n=3$ and two levels
$\cK=\{k_1<k_2\}$, which satisfies that there is at least one anti-Stokes
direction $\theta$ which is beared by both levels.
Let $q_j(t^{-1})$ be the determining polynomials and let $k_{jl}$ be the
degrees of $(q_j-q_l)(t^{-1})$.
Up to permutation, we know that in our case are the leading terms of
$(q_1-q_2)(t^{-1})$ and $(q_1-q_3)(t^{-1})$ equal and thus
\begin{itemize}
  \item up to permutation is $k_2=k_{1,2}=k_{1,3}$ and $k_1=k_{2,3}$, i.e.\ the
    larger degree appears twice, and
    \begin{comment}
      let $q_1,q_2,q_3$ be polynomials, such that
      \[
        \deg(q_1-q_2) =: k_2 > k_1 := \deg(q_2-q_3)
      \]
      then is the degree of $q_1-q_3$ given by
      \begin{itemize}
        \item[\textbf{case 1}] $\deg(q_1)<k_2$: then is $\deg(q_2)=k_2$ and
          thus $\deg(q_3)=k_2$.
        \item[\textbf{case 2}] $\deg(q_2)<k_2$: then is $\deg(q_1)=k_2$ and
          $\deg(q_3)\leq\deg(q_2)$.
        \item[\textbf{case 3}] $\deg(q_1)=\deg(q_2)=k_2$: thus follows that the
          leading term of $q_1$ and $q_2$ are different.
          \begin{itemize}
            \item[\textbf{subcase 3.a}] $\deg(q_3)<k_2$: everything is clear
            \item[\textbf{subcase 3.b}] $\deg(q_3)=k_2$: here has the leading
              term of $q_3$ be equal to them from $q_1$ to satisfy that
              $k_2>\deg(q_2-q_3)$.
          \end{itemize}
      \end{itemize}
      From this follows that $\deg(q_1-q_3)=k_2$.
    \end{comment}
  % \item $q_1 \underset{\alpha,\max}{\prec} q_2$
  %   \Leftrightarrow{}
  %   $q_1 \underset{\alpha,\max}{\prec} q_3$
  %   \rewrite{respectively}
  %   $q_2 \underset{\alpha,\max}{\prec} q_1$
  %   \Leftrightarrow{}
  %   $q_3 \underset{\alpha,\max}{\prec} q_1$
  \item $q_1\myrel{\alpha}q_2$ (resp.~$q_2\myrel{\alpha}q_1$) if and only if
    $q_1\myrel{\alpha}q_3$ (resp.~$q_3\myrel{\alpha}q_1$) and thus do they
    determine the same anti-Stokes directions.
\end{itemize}
The set of all anti-Stokes directions is then given as
\[
  \A=\left\{\theta+\frac{\pi}{k}\cdot j\mid k\in\cK\text{, }j\in\N\right\}
    % =:\{
    %   \underset{\theta}{%
    %     \underset{\text{\rotatebox[origin=c]{-90}{$=$}}}{%
    %       \alpha_1
    %   }}
    % ,\dots,\alpha_\nu\}\,.
\]
Denote by $\cY_0(t)$ a normal solution of $[A^0]$.

Let us start by looking at a single germ in depth.
The Proposition~\ref{prop:representation} states that every Stokes germ
$\phi_\alpha$ can be written as its matrix representation conjugated by the
normal solution, i.e.\ as $\phi_\alpha=\cY_{0}C_{\phi_\alpha}\cY_{0}^{-1}
=\rho_{\alpha}^{-1}(C_{\phi_\alpha})$.

Look at an example in which we will demonstrate, from which relations on the
determining polynomials which restriction on the form of the Stokes matrices
arise.
\begin{exmp}
  Let $\alpha\in\A$ be an anti-Stokes direction.
  From the definition of $\SSto_\alpha(A^0)$ (cf.\
  Definition~\ref{defn:groupOfFaithfullReps}) we know that, if one has
  $q_1\myrel{\alpha}q_2$, the Stokes matrix has the form
  \[
    \begin{pmatrix}
      1 & \text{\boldmath$c_1$} & \star
    \\\text{\boldmath$0$} & 1 & \star
    \\\star & \star & 1
    \end{pmatrix}
  \]
  where $c_j\in\C$ and $\star\in\C$.

  We have seen that $q_1\myrel{\alpha}q_2$ \Rightarrow{}
  $q_1\myrel{\alpha}q_3$ thus the representation has the
  \rewrite{form}
  \[
    \begin{pmatrix}
      1 & c_1 & \text{\boldmath$c_2$}
    \\0 & 1 & \star
    \\\text{\boldmath$0$} & \star & 1
    \end{pmatrix}
  \]
  and if we also know that neither $q_2\myrel{\alpha}q_3$ nor
  $q_3\myrel{\alpha}q_2$ it has the \rewrite{form}
  \[
    \begin{pmatrix}
      1 & c_1 & c_2
    \\0 & 1 & \text{\boldmath$0$}
    \\0 & \text{\boldmath$0$} & 1
    \end{pmatrix}\,.
  \]
  \begin{comment}
    We also know that every matrix of this \rewrite{form} is a representation
    to some Stokes germ.
    Thus we have an isomorphism
    \begin{align*}
      \vartheta_\alpha:\C^2 &\longrightarrow \SSto_\alpha(A^0)
    \\(c_1,c_2)&\longmapsto
      \begin{pmatrix}
        1 & c_1 & c_2
      \\0 & 1 & 0
      \\0 & 0 & 1
      \end{pmatrix}
    \end{align*}
  \end{comment}
\end{exmp}
In fact, the following $9$ cases of Stokes matrices can arise:
\begin{center}
  \def\arraystretch{1.3}
  \setlength\tabcolsep{4mm}
  \begin{tabular}{r|c|c|c}
    & $q_2\myrel{\alpha}q_3$ & $q_3\myrel{\alpha}q_2$ & else
    \tabularnewline
    \hline
    $\substack{q_1\myrel{\alpha}q_2\\\text{and} \\q_1\myrel{\alpha}q_3}$
    & $\begin{pmatrix} 1 & c_2 & c_3 \\0 & 1 & c_1 \\0 & 0 & 1 \end{pmatrix}$
   \cellcolor{blue!15}
    & $\begin{pmatrix} 1 & c_2 & c_3 \\0 & 1 & 0 \\0 & c_1 & 1 \end{pmatrix}$
   \cellcolor{blue!15}
    & $\begin{pmatrix} 1 & c_2 & c_3 \\0 & 1 & 0 \\0 & 0 & 1 \end{pmatrix}$
   \cellcolor{green!15}
    \tabularnewline
    \hline
    $\substack{q_2\myrel{\alpha}q_1\\\text{and} \\q_3\myrel{\alpha}q_1}$
    & $\begin{pmatrix} 1 & 0 & 0 \\c_2' & 1 & c_1 \\c_3 & 0 & 1 \end{pmatrix}$
   \cellcolor{blue!15}
    & $\begin{pmatrix} 1 & 0 & 0 \\c_2 & 1 & 0 \\c_3' & c_1 & 1 \end{pmatrix}$
   \cellcolor{blue!15}
    & $\begin{pmatrix} 1 & 0 & 0 \\c_2 & 1 & 0 \\c_3 & 0 & 1 \end{pmatrix}$
   \cellcolor{green!15}
    \tabularnewline
    \hline
    else
    & $\begin{pmatrix} 1 & 0 & 0 \\0 & 1 & c_1 \\0 & 0 & 1 \end{pmatrix}$
   \cellcolor{purple!15}
    & $\begin{pmatrix} 1 & 0 & 0 \\0 & 1 & 0 \\0 & c_1 & 1 \end{pmatrix}$
   \cellcolor{purple!15}
    & $\begin{pmatrix} 1 & 0 & 0 \\0 & 1 & 0 \\0 & 0 & 1 \end{pmatrix}$
  \end{tabular}
\end{center}
In the \textcolor{blue!75!black}{blue} cases we have $\cK_\alpha=\cK$ and
$\C^3\overset{\vartheta_\alpha}{\underset{\cong}{\longrightarrow}}\SSto_\alpha(A^0)$.
In the \textcolor{green!50!black}{green} cases $\cK_\alpha=\{k_2\}$ and
$\C^2\overset{\vartheta_\alpha}{\underset{\cong}{\longrightarrow}}\SSto_\alpha(A^0)$
as well as in the \textcolor{purple!75!black}{purple} cases
$\cK_\alpha=\{k_1\}$ and
$\C^1\overset{\vartheta_\alpha}{\underset{\cong}{\longrightarrow}}\SSto_\alpha(A^0)$.
\comm{Thus, for every $\alpha\in\A$, we have an isomorphism
$\rho_{\alpha}^{-1}\circ\vartheta_\alpha$.}
We will replace $c_2'$ by $c_2+c_1c_3$ and $c_3'$ by $c_1c_2+c_3$ to be
consistent with the decomposition in the next part
(cf.\ Example~\ref{exmp:decompositionHere}).
\begin{cor}
  The morphism $\prod_{\alpha\in\A}\vartheta_\alpha$ is an isomorphism of
  pointed sets, which maps the element only containing zeros to
  \[
    (\id,\id,\dots,\id)\in\prod_{\alpha\in\A}\SSto_\alpha(A^0),
  \]
  which gets by $\left(\prod_{\alpha\in\A}\right)^{-1}\circ h$ mapped to the
  trivial cohomology class in $\St(A^0)$.
\end{cor}

%%%%%%%%%%%%%%%%%%%%%%%%%%%%%%%%%%%%%%%%%%%%%%%%%%%%%%%%%%%%%%%%%%%%%%%%%%%%%%%
In proposition~\ref{prop:filtrationOfStokesGroup} and especially
Remark~\ref{rem:filtrationOfStokesMats} we have defined a decomposition of the
Stokes group $\Sto_\alpha(A^0)$ in subgroups generated by $k$-germs for
$k\in\cK$.
In our case, we have at most two nontrivial factors. Especially is this
decomposition given by
\[
  \phi_\alpha=\phi_\alpha^{k_1} \phi_\alpha^{k_2}
  \overset{i_\alpha}\longmapsto
    \left(\phi_\alpha^{k_1},\phi_\alpha^{k_2}\right)
      \in\Sto_\alpha^{k_1}(A^0)\times\Sto_\alpha^{k_2}(A^0) \,,
\]
and $i_\alpha$ is the map, wich gives the factors of this factorization in
ascending order.
This decomposition, of a germ $\phi_\alpha$, is trivial if
$\#\cK(\phi_\alpha)\leq1$, thus the interesting cases are the
\textcolor{blue!75!black}{blue} cases.

\begin{exmp}\label{exmp:decompositionHere}
  Look at the example
  \[
    \vartheta_\alpha(c_1,c_2,c_3)=
    \cY_{0}
    \begin{pmatrix} 1 & 0 & 0 \\c_2 & 1 & 0 \\c_1c_2+c_3 & c_1 & 1 \end{pmatrix}
    \cY_{0}^{-1}
    =\phi_\alpha
    \,.
  \]
  According to Remark~\ref{rem:algFactorization} the factor
  $\phi_\alpha^{k_1}\in\Sto_\alpha^{k_1}(A^0)$, is given by
  \[
    \phi_\alpha^{k_1}=
    \cY_{0}
    \begin{pmatrix}
      1 & 0 & 0
    \\\text{\boldmath $0$} & 1 & 0
    \\\text{\boldmath $0$} & c_1 & 1
    \end{pmatrix}
    \cY_{0}^{-1}
    \,.
  \]
  The other factor $\phi_\alpha^{k_2}$ is then obtained as
  \begin{align*}
    \phi_\alpha^{k_2}&=
    \left(\phi_\alpha^{k_1}\right)^{-1}
    \phi_\alpha^{k_2}
  \\&=\cY_{0}
    \begin{pmatrix}
      1     & 0    & 0
    \\0     & 1    & 0
    \\0     & -c_1 & 1
    \end{pmatrix}
    \underset{=\id}{\underbrace{%
        \cY_{0}^{-1}
        \cY_{0}
    }}
    \begin{pmatrix} 1 & 0 & 0 \\c_2 & 1 & 0 \\c_1c_2+c_3 & c_1 & 1 \end{pmatrix}
    \cY_{0}^{-1}
  \\&=\cY_{0}
    \begin{pmatrix}
      1     & 0 & 0
    \\c_2     & 1          & 0
    \\c_3     & 0          & 1
    \end{pmatrix}
    \cY_{0}^{-1}
    \,.
  \end{align*}
\end{exmp}
The four nontrivial decomposition in our situation, are given by:
\begin{enumerate}
  \item $\begin{pmatrix} 1 & 0 & 0 \\0 & 1 & c_1 \\0 & 0 & 1 \end{pmatrix}
  \cdot\begin{pmatrix} 1 & c_2 & c_3 \\0 & 1 & 0 \\0 & 0 & 1 \end{pmatrix}=
  \begin{pmatrix} 1 & c_2 & c_3 \\0 & 1 & c_1 \\0 & 0 & 1 \end{pmatrix}$
  \item $\begin{pmatrix} 1 & 0 & 0 \\0 & 1 & 0 \\0 & c_1 & 1 \end{pmatrix}
  \cdot\begin{pmatrix} 1 & c_2 & c_3 \\0 & 1 & 0 \\0 & 0 & 1 \end{pmatrix}=
  \begin{pmatrix} 1 & c_2 & c_3 \\0 & 1 & 0 \\0 & c_1 & 1 \end{pmatrix}$
  \item $\begin{pmatrix} 1 & 0 & 0 \\0 & 1 & c_1 \\0 & 0 & 1 \end{pmatrix}
  \cdot\begin{pmatrix} 1 & 0 & 0 \\c_2 & 1 & 0 \\c_3 & 0 & 1 \end{pmatrix}=
  \begin{pmatrix} 1 & 0 & 0 \\c_2+c_1c_3 & 1 & c_1 \\c_3 & 0 & 1 \end{pmatrix}$
  \item $\begin{pmatrix} 1 & 0 & 0 \\0 & 1 & 0 \\0 & c_1 & 1 \end{pmatrix}
  \cdot\begin{pmatrix} 1 & 0 & 0 \\c_2 & 1 & 0 \\c_3 & 0 & 1 \end{pmatrix}=
  \begin{pmatrix} 1 & 0 & 0 \\c_2 & 1 & 0 \\c_1c_2+c_3 & c_1 & 1 \end{pmatrix}$
\end{enumerate}

%%%%%%%%%%%%%%%%%%%%%%%%%%%%%%%%%%%%%%%%%%%%%%%%%%%%%%%%%%%%%%%%%%%%%%%%%%%%%%%
\subsubsection{Explicit example}
\def\kOne{1}
\def\kTwo{3}
\def\zkOnepzKtwo{14} % 2\cdot(\kOne+2\cdot\kTwo
\def\zkOne{2} % 2*\kOne
\def\zkTwo{6} % 2*\kTwo

Even more explicit, we can fix the levels $k_1=\kOne$ and $k_2=\kTwo$ together
with $\theta=0$.
Assume without any restriction that $q_1\myrel{\theta}q_2$ and
$q_1\myrel{\theta}q_3$ as well as $q_2\myrel{\theta}q_3$.
Other choices would result \rewrite{in reordering of the tuples below.}
\comm{Let the matrix $L$ be given as $L=\diag(l_1,l_2,l_3)\in\Gl_n(\C)$.}

The classification space is in this case isomorphic to
$\C^{2\cdot(\kOne+2\cdot\kTwo)}=\C^{\zkOnepzKtwo}$.
The element
\[
  ({}^1c_1,{}^2c_1,
  {}^1c_2,{}^1c_3,{}^2c_2,{}^2c_3,\dots,{}^{\zkTwo}c_2,{}^{\zkTwo}c_3)
  \in\C^{\zkOnepzKtwo}
\]
gets, via the isomorphism $\prod_{\alpha\in\A}j_\alpha$, mapped to
\begin{align*}
  &\left(
  \left(
    \begin{pmatrix} 1 & 0 & 0 \\0 & 1 & {}^1c_1 \\0 & 0 & 1 \end{pmatrix},
    \begin{pmatrix} 1 & 0 & 0 \\0 & 1 & 0 \\0 & {}^2c_1 & 1 \end{pmatrix}
  \right),
  \right.
\\&\qquad\left(
  \left.
    \begin{pmatrix} 1 & {}^1c_2 & {}^1c_3 \\0 & 1 & 0 \\0 & 0 & 1 \end{pmatrix},
    \begin{pmatrix} 1 & 0 & 0 \\{}^2c_2 & 1 & 0 \\{}^2c_3 & 0 & 1 \end{pmatrix},
    \dots,
    \begin{pmatrix} 1 & 0 & 0 \\{}^{\zkTwo}c_2 & 1 & 0 \\{}^{\zkTwo}c_3 & 0 & 1 \end{pmatrix}
  \right)
  \right)
\end{align*}
in $\prod_{\alpha\in\A^{\kOne}}\SSto_{\alpha}^{\kOne}(A^0) \times
\prod_{\alpha\in\A^{\kTwo}}\SSto_{\alpha}^{\kTwo}(A^0)$ and thus the element
\begin{align*}
  &\left(
  \left(
    \begin{pmatrix} 1 & 0 & 0 \\0 & 1 & {}^1c_1 \\0 & 0 & 1 \end{pmatrix},
    \id,\id,
    \begin{pmatrix} 1 & 0 & 0 \\0 & 1 & 0 \\0 & {}^2c_1 & 1 \end{pmatrix},
    \id,\id
  \right),
  \right.
\\&\qquad\left(
  \left.
    \begin{pmatrix} 1 & {}^1c_2 & {}^1c_3 \\0 & 1 & 0 \\0 & 0 & 1 \end{pmatrix},
    \begin{pmatrix} 1 & 0 & 0 \\{}^2c_2 & 1 & 0 \\{}^2c_3 & 0 & 1 \end{pmatrix},
    \dots,
    \begin{pmatrix} 1 & 0 & 0 \\{}^{\zkTwo}c_2 & 1 & 0 \\{}^{\zkTwo}c_3 & 0 & 1 \end{pmatrix}
  \right)
  \right)
\end{align*}
in
$\prod_{\alpha\in\A}\SSto_{\alpha}^{\kOne}(A^0) \times
\prod_{\alpha\in\A}\SSto_{\alpha}^{\kTwo}(A^0)$.
Using the morphism $\prod_{\alpha\in\A}i_\alpha^{-1}$ we get a complete set of
Stokes matrices as
\begin{align*}
  &\left(
    \begin{pmatrix} 1 & {}^1c_2 & {}^1c_3 \\0 & 1 & {}^1c_1 \\0 & 0 & 1 \end{pmatrix},
    \begin{pmatrix} 1 & 0 & 0 \\{}^2c_2 & 1 & 0 \\{}^2c_3 & 0 & 1 \end{pmatrix},
    \begin{pmatrix} 1 & {}^3c_2 & {}^3c_3 \\0 & 1 & 0 \\0 & 0 & 1 \end{pmatrix},
  \right.
\\&\qquad
  \left.
    \begin{pmatrix} 1 & 0 & 0 \\{}^4c_2 & 1 & 0 \\{}^2c_1{}^4c_2+{}^4c_3 & {}^2c_1 & 1 \end{pmatrix},
    \begin{pmatrix} 1 & {}^5c_2 & {}^5c_3 \\0 & 1 & 0 \\0 & 0 & 1 \end{pmatrix},
    \begin{pmatrix} 1 & 0 & 0 \\{}^{\zkTwo}c_2 & 1 & 0 \\{}^{\zkTwo}c_3 & 0 & 1 \end{pmatrix}
  \right)
  \in
  \prod_{\alpha\in\A}\SSto_{\alpha}(A^0) \,.
\end{align*}
Applying the isomorphism $\prod_{\alpha\in\A}\rho_\alpha^{-1}$, i.e.\
conjugation by the fundamental solution $\cY_0(t)=t^Le^{Q(t^{-1})}$
(cf.\ Proposition~\ref{prop:representation}), yields then the corresponding
Stokes cocycle in $\prod_{\alpha\in\A}\Sto_{\alpha}(A^0)$ and thus an element
in $\St(A^0)$.
\begin{comment}
  This element is explicitly given as
  \begin{align*}
    &\left(
      \begin{pmatrix}
          1 & {}^1c_2 t^{l_2-l_1}e^{(q_2-q_1)(t^{-1})}
            & {}^1c_3 t^{l_3-l_1}e^{(q_3-q_1)(t^{-1})}
        \\0 & 1 & {}^1c_1 t^{l_3-l_2}e^{(q_3-q_2)(t^{-1})}
        \\0 & 0 & 1
      \end{pmatrix},
      \cY_0^{-1}
      \begin{pmatrix}
          1 & 0 & 0
        \\{}^2c_2 & 1 & 0
        \\{}^2c_3 & 0 & 1
      \end{pmatrix}
      \cY_0
      ,
    \right.
  \\&\qquad
      \cY_0^{-1}
      \begin{pmatrix}
          1 & {}^3c_2 & {}^3c_3
        \\0 & 1 & 0
        \\0 & 0 & 1
      \end{pmatrix}
      \cY_0
      ,
      \cY_0^{-1}
      \begin{pmatrix}
          1 & 0 & 0
        \\{}^4c_2 & 1 & 0
        \\{}^2c_1{}^4c_2+{}^4c_3 & {}^2c_1 & 1
      \end{pmatrix}
      \cY_0
      ,
  \\&\qquad\qquad
    \left.
      \cY_0^{-1}
      \begin{pmatrix}
          1 & {}^5c_2 & {}^5c_3
        \\0 & 1 & 0
        \\0 & 0 & 1
      \end{pmatrix}
      \cY_0
      ,
      \cY_0^{-1}
      \begin{pmatrix}
          1 & 0 & 0
        \\{}^{\zkTwo}c_2 & 1 & 0
        \\{}^{\zkTwo}c_3 & 0 & 1
      \end{pmatrix}
      \cY_0
    \right)
    \in
    \prod_{\alpha\in\A}\Sto_{\alpha}(A^0) \,.
  \end{align*}
\end{comment}
