\chapter{Grundlagen}
\section{Mannigfaltigkeiten}
\begin{comment}
Hier sollen einige Grundlagen über differenzierbare Mannigfaltigkeiten
wiederholt werden. Siehe dazu \cite[1.22ff]{warnerLie}.
\end{comment}
\begin{comment}
\begin{defn} \ccite[Defn 1.22]{warnerLie}
Sei $\psi :M\overset{\cC^\infty}\to N$ eine Abbildung und sei $m\in M$. Das
\emph{Differenzial} von $\psi$ an $m$ ist die lineare Abbildung
$d\psi:M_m\to N_{\psi(m)}$, welche wie folgt definiert ist. ...
\end{defn}
\end{comment}
\begin{comment}
\begin{defn}
... der \emph{Tangentialraum $G_e$ am Element $e$} ist definiert als...
\end{defn}
\end{comment}

\begin{defn}
\begin{itemize}
\item 
\ccite[2.5 Definition]{warnerLie}
$\Lambda_k(V):=V_{k,0}/I_k(V)$ für $k\geq2$ und $\Lambda_0(V)=\C$ sowie
$\Lambda_1(V)=V$. Wobei
\begin{itemize}
\item $I(V)$ das zweiseitige Ideal\comm{ in $C(V)$}, welches von Elementen der
Form $v\otimes v$ für alle $v\in V$ erzeugt wird
\item $I_k(V):=I(V)\cap V_{k,0}$
\item $V_{k,0}$ der Tensorraum von Typ $(k,0)$, assoziiert zu $V$ 
\ccite[2.3 Definition]{warnerLie}
\end{itemize}
\item 
\ccite[2.14 Definition]{warnerLie}
$\Lambda_k^*(M):=\underset{m\in M}\bigcup\Lambda_k^*(M_m)$ das \emph{äußere $k$
Bündel von $M$}
\item 
\ccite[2.15 Definition]{warnerLie}
Eine \emph{$k$-Form} ist eine $\cC^\infty$-Abbildung von $M$ nach
$\Lambda_k^*(M)$ welche, verknüpft mit der kanonischen Projektion, die
Identität ergibt
\end{itemize}
\end{defn}

\section{\boldmath$\Proj$-Konstruktion für Projektive Schemata}
\begin{comment}
siehe dazu
\begin{itemize}
\item 
\cite[Section 2.3.3]{liu2002algebraic}
\item 
\cite[Section 4.5]{ravil} 
\end{itemize}
\end{comment}
\begin{defn} \ccite[Section 2.3.3]{liu2002algebraic}
Sei $A$ ein Ring und $S=\bigoplus_{d\geq0}S_d$ eine graduierte $A$-Algebra. 
\begin{itemize}
\item
Ein Ideal $I$ in $S$ wird \emph{homogen} genannt, falls es von homogenen
Elementen erzeugt wird. \comm{$\Leftrightarrow$ $I=\bigoplus_{d\geq0}(I\cap
S_d)$}
\item
$S_+:=\bigoplus_{d>0}S_d \vartriangleleft S$
\item
$\Proj S:=\{\mathfrak p \in\Spec S \text{ homogen} \mid
  S_+\notin\mathfrak p\}$
\item
$V_+(I):=...$
\item
Sei $f$ ein Homogenes Element aus $S$.
\begin{align*}
D_+(f)&:=\Proj S \ V_+(fS)
\\&=\left\{\mathfrak p \in\Proj S\mid f\notin \mathfrak p \right\}
  \overset{\text{offen}}\subset S
\end{align*}
\end{itemize}
\end{defn}

Es ist $(\Proj \C[T_0,T_1],\cO_{\Proj \C[T_0,T_1]})$ ein lokal geringter Raum.

\section{(Lokal) freie Modulgarben / Vektorbündel}
\begin{defn}
Ein $\cO_X$ Modul $\cM$ heißt
\begin{itemize}
\item 
\emph{frei}, wenn es eine Menge $I$ und einen $\cO_X$ Modul-Isomorphismus 
\[ \cO_X^{(I)} := \bigoplus_{i\in I} \cO_X \overset{\cong} \to \cM\]
gibt,
\item 
\emph{lokal frei} oder \emph{Vektorbündel von Rang $r$}, wenn es zu jedem $x\in
X$ eine Teilmenge $x \in U\overset{\text{offen}}\subset X$ und einen
$\cO_U$-Modul-Isomorphismus
\[ \cO_U^r \overset{\cong}\to \cM|_U\]
gibt.
\end{itemize}
\end{defn}
\begin{comment}
Sei ein Vektorbündel $V\overset{\pi}\to \P^1$ vom Rang $n$ gegeben. Dann setze 
\[ 
\cV : U \mapsto \cV(U) := \{\sigma: U \to \pi^{-1}(U) \subseteq V \mid
  \sigma\text{ stetig, } \pi \circ \sigma = \id_U\}.
\]
Dies ist eine lokal freie $\cO_{\Proj \C[T_0,T_1]}$-Modulgarbe vom Rang $n$.

\begin{defn}
Eine \emph{lokale Trivialisierung} ist ein isomorphismus
\[
\cV|_U\overset{\cong}\to \cO_{\P^1}|_U^n=\cO_U^n
\]
\end{defn}

\begin{defn}[Holomorphes Vektorbündel vom Rang $n$ auf $V$]
\end{defn}
\end{comment}

\section{(Weil-)Divisoren}
\begin{comment}
Aus dem Schematheorie Skript:
\begin{defn}
Sei $X$ noethersch.
\begin{enumerate}
\item
Ein \emph{Primzykel in $X$} ist eine irreduzible, abgeschlossene
Teilmenge.
\item
Ein \emph{Zykel in $X$} ist ein Element der abelschen Gruppe
\[\Z^{(X)} := \{Z = \sum_{x\in X} n_x \bar{\{x\}} \mid
    n_x \in \Z,\ n_x = 0\text{ für fast alle }x\}.\]
\item
Für $Z = \sum n_x \bar{\{x\}}$ heißt
\[ \supp Z := \bigcup_{x\in X\atop n_x\neq 0}\]
der \emph{Träger von $Z$}.
\item
Ein Zykel $Z$ heißt von \emph{Kodimension 1}, wenn alle $x$ mit
$n_x \neq 0$ von Kodimension 1 sind, d.h. 
$\codim_X \bar{\{x\}} = 1$. Äquivalent dazu ist zu fordern, dass
$\dim \O_{X,x} = 1$.\\
$Z^1(X) \subseteq \Z^{(X)}$ bezeichne die Untergruppe dieser.
\end{enumerate}
\end{defn}

\begin{defn}[Weil-Divisoren]
    Sei $X$ noethersch und integer, so heißt $Z^1(X)$ die Gruppe der
    \emph{Weil-Divisoren}.
\end{defn}
%
\begin{defn}[(Effectiver) Divisor]
Ein \emph{(effektiver) Divisor} ist...
\end{defn}
%

\end{comment}

%
% vim:set ft=tex foldmethod=marker foldmarker={{{,}}}:
