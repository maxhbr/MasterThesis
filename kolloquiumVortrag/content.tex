\section{Meromorphe Zusammenhängen und Systeme}
\begin{frame}{Meromorphe Zusammenhänge}
  \begin{defn}
    Ein \emph{(Keim eines) Meromorphen Zusammenhangs} ist ein Tupel
    $(\cM,\nabla)$\uncover<2->{, wobei}
    \begin{itemize}
    \item<2-> $\cM$ ist ein $\C(\!\{t\}\!)$-Vektorraum von Dimension $n$
      \uncover<3->{und}
    \item<3-> $\nabla:\cM\to \cM$, welcher die \emph{Leibniz Regel}
      \[
        \nabla(fm)=\frac{d}{dt} f \cdot m + f \nabla(m)
      \]
      für alle $f\in\C(\!\{t\}\!)$ und $m\in\cM$ erfüllt.
    \end{itemize}
  \end{defn}
\end{frame}
\begin{frame}{Systeme}
  Wähle eine $\C(\!\{t\}\!)$-Basis $\underline{e}=(e_1,e_2,\dots,e_n)$ von
  $\cM$.
  \uncover<2->{%
    \begin{defn}
      \begin{enumerate}
      \item Die \emph{Zusammenhangs-Matrix (von \boldmath$(\cM,\nabla)$ zu
          $\underline{e}$)} ist die Matrix $A\in\Gl_n(\C(\!\{t\}\!))$ so dass
        \[
          \nabla\left(\underline{e}\cdot X\right)
          =\underline{e}\left(dX-AX\right)
        \]
        für jedes $x=\underline{e}\cdot X$.
      \item<3-> Bezeichne das System linearer gewöhnlicher komplexer
        Differentialgleichungen
        \[
          dX=AX
        \]
        als \emph{System \boldmath$[A]$}.
      \end{enumerate}
    \end{defn}
  }
\end{frame}
\begin{frame}{Systeme und die Gauge Transformation}
  \begin{prop}
    Durch einen Wechsel $F\in\Gl_n\left(\C(\!\{t\}\!)\right)$ der Basis erhält
    man
    das System $\bigl[{}^F\!A\bigr]$, gegeben durch die
    \emph{Gauge-Transformation}
    \[
      {}^F\!A=(dF)F^{-1} + FAF^{-1} \,.
    \]
  \end{prop}
  \uncover<2->{%
    \begin{defn}
      Die Zusammenhangs-Matrizen $A$ und $B$ sind \emph{(formal) äquivalent},
      falls es einen (formalen) Basiswechsel $F$ gibt, so dass ${}^F\!A=B$.
    \end{defn}
  }
  \uncover<3->{%
    \begin{lem}
      Zwei meromorphe Zusammenhänge sind (formal) isomorph wenn sie (formal)
      äquivalente Zusammenhangs-Matrizen haben.
    \end{lem}
  }
\end{frame}

\subsection{Levelt-Turittin und die Normalform}
\begin{frame}{Die Normalform}
  Aus dem Levelt-Turittin Theorem\TODO[skizze] erhalten wir:
  \begin{prop}
    \uncover<2->{%
      Jedes (unverzweigte) System $[A]$ ist formal äquivalent zu einem
      System $[A^0]$ wobei
      \[
        A^0=Q'(t^{-1})+\frac{1}{t}L
      \]
      und
      \begin{itemize}
      \item $Q(t^{-1}):=\bigoplus_{j=1}^sq_j(t^{-1})\cdot\id_{n_j}$ und
      \item $L:=\bigoplus_{j=1}^sL_j$ mit $L_j$ in Jordan-Normalform.
      \end{itemize}
      \uncover<3->{%
        \begin{defn}
          Jedes System $\bigl[B^0\bigr]$, das äquivalent zu einem System
          $\bigl[A^0\bigr]$ wie oben ist, nennen wir \emph{normal Form}.
        \end{defn}
      }
    }
  \end{prop}
\end{frame}

\subsection{Die klassifizierende Menge der Systeme}
\begin{frame}{Die klassifizierende Menge}
  Die \textit{interessante Menge} ist die Menge
  \[
    {}^0\!C(A^0):=\left\{\left[A\right]
      \mid A={}^{\hat F}\!A^0 \text{ für ein } \hat F\in G(\!(t)\!)\right\}
  \]
  der Äquivalenzklassen vom Zusammenhangs-Matrizen formal äquivalent zu
  einem Model.

  \uncover<2->{%
    \vspace{1em}%%%%%%%%%%%%%%%%%%%%%%%%%%%%%%%%%%%%%%%%%%%%%%%%%%%%%%%%%%%%%%%%
    Es ist aber sinnvoll die folgende größere Menge zu betrachten:
    \begin{defn}
      Die \emph{klassifizierende Menge}
      \[
        \cH(A^0):=\left\{\left[\!\bigl(A,\hat F\bigl)\!\right]
          \mid A={}^{\hat F}\!A^0 \text{ mit } \hat F\in G(\!(t)\!)\right\}
      \]
      ist die Menge der Äquivalenzklassen der \emph{Markierten Paare} zu einem
      Model $A^0$.
      \TODO[Äquivalenzrelation zwischen markierten Paaren]
    \end{defn}
  }
\end{frame}

\section{Die Stokes Matrizen}
\begin{frame}{Die Stokes Keime}
  \begin{defn}
    \vspace{-3em}
    \begin{flushright}
      \begin{minipage}{.55\textwidth}
        \begin{itemize}
        \item[\tikzmarkc{n1}{blue}] Funktionen mit asymptotischer Erweiterung
          in Richtung $\theta$.
        \end{itemize}
      \end{minipage}
    \end{flushright}
    \begin{flushright}
    \end{flushright}
    \vspace{-2em}
    Ein Keim $F\in\Gl_n(\overset{\tikzmark{e1}}{\cA_\theta})$ in Richtung
    $\theta\in S^1$ ist
    \begin{itemize}
    \item ein \emph{Keim der Stokes Garbe \boldmath$\Lambda(A^0)$}, falls er
      \begin{enumerate}
      \item\emph{multiplikativ flach} ($F$ ist asymptotisch zu $\id$ bei
        $\theta$) \uncover<2->{und}
      \item<2-> eine \emph{Isotropie von \boldmath$A^0$} (${}^FA^0=A^0$)
      \end{enumerate}
      ist.
    \item<3-> Falls für $F$ weiter auch noch \textcolor{purple}{jeder Eintrag}
      \begin{enumerate}
        \setcounter{enumi}{2}
      \item von
        \emph{$\underset{\tikzmarkt{e2}}{%
              \only<-3>{%
                \textcolor{white}{\underbrace{\text{%
                      \textcolor{black}{maximal decay}}}}
              }
              \only<4->{%
                \textcolor{blue!50!white}{\underbrace{\text{%
                      \textcolor{black}{maximal decay}}}}
              }
            }$
          in Richtung \boldmath$\theta$}
        \uncover<4->{
          \begin{minipage}{\textwidth}
            \begin{flushright}
              \begin{minipage}{.6\textwidth}
                ~\tikzmark{n2}
                \vspace{-2em}
                \begin{defn}
                  % Eine Funktion
                  $e^{q(t^{-1})}$ mit $q(t^{-1})\in\frac{a}{t^{k}}+o(t^{-k})$
                  hat \emph{\textcolor{purple}{maximal decay} in Richtung
                    \boldmath$\theta$} genau dann wenn $ae^{-ik\theta}$ reell
                  negativ ist.
                \end{defn}
              \end{minipage}
            \end{flushright}
          \end{minipage}
        }
      \end{enumerate}
      ist, ist $F$ ein \emph{Stokes Keim}.
      \uncover<5->{
        Die Gruppe dieser Keime wird mit $\Sto_\theta(A^0)$ bezeichnet.
      }
    \end{itemize}
  \end{defn}
  \begin{tikzpicture}[remember picture,overlay]
    \draw[->,blue!50!white,thick] (n1) to[out=180,in=90] (e1);
    \uncover<4->{
      \draw[->,blue!50!white,thick] (n2) to[out=180,in=270] (e2);
    }
  \end{tikzpicture}
\end{frame}
\setbeamercolor{background canvas}{bg=gray!20}
\begin{frame}{Wann ist $F\in\Lambda_\theta(A^0)$ von \textcolor{purple}{maximal decay}?}
  Ein Keim $F\in\Gl_n(\cA_\theta)$, der Isotropie von $A^0$ ist, sieht wie folgt
  aus
  \begin{align*}
    F  &=
         \only<1>{
         \underset{\tikzmarkt{e2}}{
         \textcolor{blue!50!white}{\underbrace{\textcolor{black}{
         t^L e^{Q(t^{-1})}
         }}}}}
         \only<2->{
         t^L e^{Q(t^{-1})}
         }
         \only<1>{
         \underset{\tikzmarkt{e1}}{
         \textcolor{blue!50!white}{\underbrace{\textcolor{black}{
         \rho_\theta(F)
         }}}}}
         \only<2->{
         \rho_\theta(F)
         }
         \only<1>{
         \underset{\tikzmarkt{e3}}{
         \textcolor{blue!50!white}{\underbrace{\textcolor{black}{
         e^{-Q(t^{-1})}t^{-L}
         }}}}}
         \only<2->{
         e^{-Q(t^{-1})}t^{-L}
         }
    \\ &\onslide<2->{=
         t^L e^{Q(t^{-1})}
         \left(1_n+\sum_{(l,j)}C^{(l,j)}\right)
         e^{-Q(t^{-1})}t^{-L}
         }
    \\ &\onslide<3->{=
         t^L\left(
         1_n+\sum_{(l,j)}C^{(l,j)}e^{(q_l-q_j)(t^{-1})}
         \right)t^{-L} \,.}
  \end{align*}
  \uncover<1>{%
    \vspace{-11em}
    \begin{flushright}
      \tikzmarkc{n3}{blue} $\bigl(t^L e^{Q(t^{-1})}\bigr)^{-1}$

      \vspace{1em}
      \tikzmarkc{n1}{blue} konstant, also in $\Gl_n(\C)$

      \vspace{1em}
      \tikzmarkc{n2}{blue} Fundamentallösung von $A^0=(dF)F^{-1} + FA^0F^{-1}$
      \begin{tikzpicture}[remember picture,overlay]
        \draw[->,blue!50!white,thick] (n1) to[out=180,in=270] (e1);
        \draw[->,blue!50!white,thick] (n2) to[out=180,in=270] (e2);
        \draw[->,blue!50!white,thick] (n3) to[out=180,in=270] (e3);
      \end{tikzpicture}
    \end{flushright}
  }
  \vspace{3.5em}
  \uncover<4->{%
    Also ist $F$
    \begin{enumerate}
    \item multiplikativ flach falls falls für jedes $C^{(l,j)}\neq0$ der Faktor
      $e^{(q_l-q_j)(t^{-1})}$ asymptotisch zu $0$ ist \uncover<4->{und}
    \item<5-> von \textcolor{purple}{maximal decay} falls für jedes
      $C^{(l,j)}\neq0$ der Faktor $e^{(q_l-q_j)(t^{-1})}$ von \textcolor{purple}{maximal decay} in
      Richtung $\theta$ ist.
    \end{enumerate}
  }
\end{frame}
\setbeamercolor{background canvas}{bg=}

\subsection{Die anti-Stokes Richtungen}
\begin{frame}{Die Stokes Matrizen und die anti-Stokes Richtungen}
  \begin{defn}
    Definiere die Relation $q_j\myrel{\theta}q_l$ als äquivalent zu
    \begin{einr}
      $e^{(q_l-q_j)(t^{-1})}$ ist von \textcolor{purple}{maximal decay} in
      Richtung $\theta$.
    \end{einr}
  \end{defn}
  \uncover<2->{%
    \begin{defn}
      \begin{enumerate}
      \item<2-> Definiere die \emph{Gruppe der Stokes Matrizen} als
        \begin{align*}
          \SSto_\theta(A^0)&:= \Big\{
                             C=(c_{jl})_{j,l\in\{1,\dots,n\}}
                             \in\Gl_n(\C)
                             \mathlarger{\mathlarger{\mid}}
          \\              & \qquad\qquad
                            c_{(l,j)}=\delta_{jl}
                            \text{~außer wenn~}
                            q_j\myrel{\theta}q_l
                            \Big\} \,.
        \end{align*}
      \item<3-> Die \emph{anti-Stokes Richtungen} sind die $\theta\in S^1$ für
        die
        \[
          \Sto_\theta(A^0)\neq\{\id\} \,.
        \]
        \uncover<4->{%
          Das sind genau die $\theta$ für die es $j$ und $l$ gibt so dass
          $q_j\myrel{\theta}q_l$.
        }
      \end{enumerate}
    \end{defn}
  }
\end{frame}
\begin{frame}[fragile]{Level von Stokes Matrizen}
  \begin{defn}
    \begin{enumerate}
    \item Schreibe
      $\rho_\theta(F)=1_n+\sum_{(l,j)}C^{(l,j)}\in\SSto_\theta(A^0)$, dann sind
      die \emph{Level} von $F$ definiert als
      \[
        \cK(F):= \left\{\deg(q_j-q_l)\mid Ĉ{(j,l)}\neq0\right\} \subset \cK \,.
      \]
    \item<2-> Ein Keim $F$ ist ein \emph{\boldmath$k$-Keim}, wenn
      $\cK(F)\subset\{k\}$.

      \uncover<3->{
        Die Gruppe all dieser ist $\SSto_\theta^k(A^0)$.
      }
    \end{enumerate}
  \end{defn}
\end{frame}
\begin{frame}[fragile]{\textcolor{purple}{\dots}}
  \uncover<1>{
    Fixiere $(j,l)$ mit $j\neq l$ und damit
    $(q_l-q_j)(t^{-1})\in\frac{a}{t^{k}}+o(t^{-k})$.
    \begin{flushright}
      \tikzmarkc{n1}{blue} Reeller Teil von $e^{\frac{a}{t^{k}}}$

      \vspace{1em}
      \tikzmarkb{n2}{green} Imag.\ Teil von $e^{\frac{a}{t^{k}}}$
    \end{flushright}
  }
  \vspace{-4em}
  \begin{center}
    \begin{tikzpicture}[scale=3.1]
      \pgfmathsetmacro{\k}{1}
      \pgfmathsetmacro{\abs}{0.4}

       \node at (0.25,0.2) {$\tikzmark{e1}$};
       \node at (1.45,0.2) {$\tikzmark{e2}$};

      \uncover<7->{
        \draw[thick, dotted, green!50!black] (0.9,0) -- (0.9,\abs)
        node [above,font=\tiny,] {$\theta$};
      }

      \uncover<4>{
        \draw ({0.1},{-0.27}) -- ({0.1},{-0.47});
        \draw ({1.1},{-0.27}) -- ({1.1},{-0.37});
        \draw ({2.1},{-0.27}) -- ({2.1},{-0.47});
        \draw[<->] ({0.1},{-0.34}) -- ({1.1},{-0.34})
        node [midway,above,font=\tiny,] {$\frac{\pi}{k}$};
        \draw[<->] ({0.1},{-0.44}) -- ({2.1},{-0.44})
        node [midway,below,font=\tiny,] {$\frac{2\pi}{k}$};
      }

      \foreach \x in {-2,-1,...,3}{
        \foreach \argA/\absA in {0.1/0.27}{
          \begin{scope}
            \clip (-0.65,{\abs+0.1}) rectangle (2.63,{-\abs-0.1});
            \pgfmathsetmacro{\s}{{\argA + \x / \k * 2 - 1/2/\k}};

            \uncover<3->{
              \draw[blue!40!white] (\s,0)
              sin ({\s + 1/2/\k},{-\absA})
              cos ({\s + 2/2/\k},0)
              sin ({\s + 3/2/\k},\absA)
              cos ({\s + 4/2/\k},0);
              \uncover<6->{
                \draw[dotted] ({\s + 1/2/\k},{-\absA}) -- ({\s + 1/2/\k},-0.5);
              }
              \fill[white] ({\s + 1/2/\k},{-\absA}) circle (1pt);
              \fill[red] ({\s + 1/2/\k},{-\absA}) circle (.4pt);
            }

            \draw[blue!40!black] (\s,0)
            sin ({\s + 1/2/\k},\absA)
            cos ({\s + 2/2/\k},0)
            sin ({\s + 3/2/\k},{-\absA})
            cos ({\s + 4/2/\k},0);
            \uncover<6->{
              \draw[dotted] ({\s + 3/2/\k},{-\absA}) -- ({\s + 3/2/\k},-0.5);
            }
            \fill[white] ({\s + 3/2/\k},{-\absA}) circle (1pt);
            \fill[red] ({\s + 3/2/\k},{-\absA}) circle (.4pt);

            \only<1>{
              \draw[green!40!black] ({\s          - 1/2/\k},0)
                                sin ({\s + 1/2/\k - 1/2/\k},\absA)
                                cos ({\s + 2/2/\k - 1/2/\k},0)
                                sin ({\s + 3/2/\k - 1/2/\k},{-\absA})
                                cos ({\s + 4/2/\k - 1/2/\k},0);
            }
            \uncover<6->{
              \draw[dotted] ({\s + 3/2/\k},{-\absA}) -- ({\s + 3/2/\k},-0.5);
            }
          \end{scope}
          \uncover<8->{
            \begin{scope}[thick]
              \clip (0.4,{\abs+0.1}) rectangle (1.4,{-\abs-0.1});
              \pgfmathsetmacro{\s}{{\argA + \x / \k * 2 - 1/2/\k}};

              \draw[blue!40!white] (\s,0)
              sin ({\s + 1/2/\k},{-\absA})
              cos ({\s + 2/2/\k},0)
              sin ({\s + 3/2/\k},\absA)
              cos ({\s + 4/2/\k},0);
              \fill[white] ({\s + 1/2/\k},{-\absA}) circle (1pt);
              % \draw[dotted] ({\s + 1/2/\k},{-\absA}) -- ({\s + 1/2/\k},0.5);
              \fill[red] ({\s + 1/2/\k},{-\absA}) circle (.4pt);

              \draw[blue!40!black] (\s,0)
              sin ({\s + 1/2/\k},\absA)
              cos ({\s + 2/2/\k},0)
              sin ({\s + 3/2/\k},{-\absA})
              cos ({\s + 4/2/\k},0);
              \fill[white] ({\s + 3/2/\k},{-\absA}) circle (1pt);
              % \draw[dotted] ({\s + 3/2/\k},{-\absA}) -- ({\s + 3/2/\k},0.5);
              \fill[red] ({\s + 3/2/\k},{-\absA}) circle (.4pt);
            \end{scope}
          }
        }
        \uncover<5->{
          \foreach \argA/\absA in {-0.5/0.4
            ,0.3/0.35
            ,0.65/0.23}{
            \begin{scope}
              \clip (-0.65,{\abs+0.1}) rectangle (2.63,{-\abs-0.1});
              \pgfmathsetmacro{\s}{{\argA + \x / \k * 2 - 1/2/\k}};

              \draw[blue!40!white] (\s,0)
              sin ({\s + 1/2/\k},{-\absA})
              cos ({\s + 2/2/\k},0)
              sin ({\s + 3/2/\k},\absA)
              cos ({\s + 4/2/\k},0);
              \uncover<6->{
                \draw[dotted] ({\s + 1/2/\k},{-\absA}) -- ({\s + 1/2/\k},-0.5);
              }
              \fill[white] ({\s + 1/2/\k},{-\absA}) circle (1pt);
              \fill[red] ({\s + 1/2/\k},{-\absA}) circle (.4pt);

              \draw[blue!40!black] (\s,0)
              sin ({\s + 1/2/\k},\absA)
              cos ({\s + 2/2/\k},0)
              sin ({\s + 3/2/\k},{-\absA})
              cos ({\s + 4/2/\k},0);
              \uncover<6->{
                \draw[dotted] ({\s + 3/2/\k},{-\absA}) -- ({\s + 3/2/\k},-0.5);
              }
              \fill[white] ({\s + 3/2/\k},{-\absA}) circle (1pt);
              \fill[red] ({\s + 3/2/\k},{-\absA}) circle (.4pt);
            \end{scope}
            \uncover<8->{
              \begin{scope}[thick]
                \clip (0.4,{\abs+0.1}) rectangle (1.4,{-\abs-0.1});
                \pgfmathsetmacro{\s}{{\argA + \x / \k * 2 - 1/2/\k}};

                \draw[blue!40!white] (\s,0)
                sin ({\s + 1/2/\k},{-\absA})
                cos ({\s + 2/2/\k},0)
                sin ({\s + 3/2/\k},\absA)
                cos ({\s + 4/2/\k},0);
                \fill[white] ({\s + 1/2/\k},{-\absA}) circle (1pt);
                % \draw[dotted] ({\s + 1/2/\k},{-\absA}) -- ({\s + 1/2/\k},0.5);
                \fill[red] ({\s + 1/2/\k},{-\absA}) circle (.4pt);

                \draw[blue!40!black] (\s,0)
                sin ({\s + 1/2/\k},\absA)
                cos ({\s + 2/2/\k},0)
                sin ({\s + 3/2/\k},{-\absA})
                cos ({\s + 4/2/\k},0);
                \fill[white] ({\s + 3/2/\k},{-\absA}) circle (1pt);
                % \draw[dotted] ({\s + 3/2/\k},{-\absA}) -- ({\s + 3/2/\k},0.5);
                \fill[red] ({\s + 3/2/\k},{-\absA}) circle (.4pt);
              \end{scope}
            }
          }
        }
      }
      \uncover<6->{
        \node[font=\tiny] at (-0.5,{-\abs - 0.15}) {$\alpha_{\nu-1}$};
        \node[font=\tiny] at (-0.35,{-\abs - 0.15}) {$\alpha_{\nu}$};
        \node[font=\tiny] at (0.1,{-\abs - 0.15}) {$\alpha_1$};
        \node[font=\tiny] at (0.3,{-\abs - 0.15}) {$\alpha_2$};
        \node[font=\tiny] at (0.5,{-\abs - 0.15}) {$\alpha_3$};
        \node[font=\tiny] at (0.65,{-\abs - 0.15}) {$\alpha_4$};
        \node[font=\tiny] at (1.1,{-\abs - 0.15}) {$\alpha_5$};
        \node[font=\tiny] at (1.3,{-\abs - 0.15}) {$\alpha_6$};
        \node[font=\tiny] at (1.5,{-\abs - 0.15}) {$\alpha_7$};
        \node[font=\tiny] at (1.65,{-\abs - 0.15}) {$\alpha_8$};
        \node[font=\tiny] at (2.1,{-\abs - 0.15}) {$\alpha_9$};
        \node[font=\tiny] at (2.3,{-\abs - 0.15}) {$\alpha_{10}$};
        \node[font=\tiny] at (2.5,{-\abs - 0.15}) {$\alpha_{11}$};
      }

      \draw[-latex'] (-0.7,0) -- (2.7,0) node [right] {$S^1$};
      \draw[-latex'] ({0},{-\abs-0.1}) -- ({0},{\abs + 0.2});

      \uncover<8->{
        \begin{scope}[dashed]
          \clip (-0.65,{\abs+0.1}) rectangle (2.6,{-\abs-0.4});
          \foreach \x in {-2,-1,1,2} {
            % \draw [purple]
            % ({0.4 + \x},{-\abs - 0.2}) -- ({0.4 + \x},0);
            \draw [purple]
            ({1.4 + \x},{-\abs - 0.2}) -- ({1.4 + \x},0);
            \draw [purple
            ,decorate
            ,decoration={brace,mirror,amplitude=10pt}
            ,xshift=0pt
            ,yshift=0pt]
            ({0.4 + \x},{-\abs - 0.2}) -- ({1.4 + \x},{-\abs - 0.2});
          }
        \end{scope}
        \draw [purple,dashed]
        (1.4,{-\abs - 0.2}) -- (1.4,0);
        \draw [purple, thick
        ,decorate
        ,decoration={brace,mirror,amplitude=10pt}
        ,xshift=0pt
        ,yshift=0pt]
        (0.4,{-\abs - 0.2}) -- (1.4,{-\abs - 0.2})
        node [midway,yshift=-7pt] {$\tikzmark{g0}$};
        \draw [purple,dashed,
        ,decorate
        ,decoration={brace,mirror,amplitude=10pt}
        ,xshift=0pt
        ,yshift=0pt]
        (1.4,{-\abs - 0.2}) -- (2.4,{-\abs - 0.2})
        node [midway,yshift=-7pt] {$\tikzmark{g1}$};
      }

      % \draw[dotted] (-0.6,\abs)node[left,font=\tiny] {$|a|$} -- (2.6,\abs);
      % \draw[dotted] (-0.6,{-\abs})node[left,font=\tiny] {$-|a|$} -- (2.6,{-\abs});

      % \draw[dotted] ({-0.5},{-\abs}) -- ({-0.5},{\abs + 0.1})
      % node [above,font=\tiny,] {-0.5 \pi};
      % \foreach \x in {0.5,1,...,2.5}{%
      % \draw[dotted] ({\x},{-\abs}) -- ({\x},{\abs + 0.1})
      % node [above,font=\tiny,] {\x \pi};
      % }
    \end{tikzpicture}
  \end{center}
  \uncover<1>{
    \begin{flushright}
      \begin{tikzpicture}[remember picture,overlay]
        \draw[->,blue!50!white,thick] (n1) to[out=180,in=45] (e1);
        \draw[->,green!50!black,thick] (n2) to[out=180,in=135] (e2);
      \end{tikzpicture}
    \end{flushright}
  }
\end{frame}
\begin{frame}[fragile]{Zerlegung von $\Sto_\theta(A^0)$}
  \begin{thm}
    Für alle $\theta\in S^1$ gibt es eine eindeutige Zerlegung
    \[ \begin{tikzcd}[row sep=0cm]
        \Sto_\theta(A^0)
        \arrow{dddd}
        & \prod_{k\in\cK}\Sto_\theta^k(A^0) \lar
        \arrow{dddd}
        \\
        \hspace{1em}
        \\
        \hspace{1em}
        \\
        \hspace{1em}
        \\
        \SSto_\theta(A^0)
        & \prod_{k\in\cK}\SSto_\theta^k(A^0) \lar
        \\ C_{k_1}\cdot C_{k_2}\cdots C_{k_s}
        & \left(C_{k_1}, C_{k_2},\dots, C_{k_s}\right) \lar[maps to]
      \end{tikzcd} \]
  \end{thm}
\end{frame}

\subsection{Verbesserung der Stokes Matrizen}
\begin{frame}{Weitere \textcolor{purple}{Verbesserungen}}
  Sei $T\subset\{1,\dots,n\}\times\{1,\dots,n\}$ eine Positions Menge, welche
  die Eigenschaft
  \begin{einr}
    wenn $(j,k),(k,l)\in T$ dann ist auch $(j,l)\in T$
  \end{einr}
  erfüllt, zusammen mit einer Bijektion
  $i:\{1,2,\dots,\mu\}\overset{\cong}{\to}T$.
  \uncover<2->{
    \begin{lem}
      Zu jeder Matrix
      \[
        K \in \{K=(K_{jl})_{j,l\in\{1,\dots,n\}}\in\GL_n(\C) \mid
        K_{jl}=\delta_{jl} \text{~wenn~}
        (j,l)\notin T \}
      \]
      gibt es eindeutige $t_1,\dots,t_\mu\in\C$, so dass
      \[
        K=(\id + t_1E_{i(1)})\cdots(\id + t_{\mu}E_{i(\mu)}) \,.
      \]
    \end{lem}
    \TODO[Visualisieren / an der Tafel?]
  }
\end{frame}

\subsection{Beispiel}
\setbeamercolor{background canvas}{bg=gray}
\begin{frame}{Beispiel: Situation}
  Sei $[A]$ ein System so dass \dots
  \begin{itemize}
  \item
    Dimension $3$\footnote{Kleinste Dimension mit $2$ Level}
  \item
    fixiere $q_1$, $q_2$ und $q_3$ so dass
    \begin{itemize}
    \item $\cK=\{k_1<k_2\}$ und
    \item $\theta\in S^1$, so dass
      \begin{einr}
        $q_1\myrel{\theta}q_2$ und $q_1\myrel{\theta}q_3$

        \Rightarrow{} anti-Stokes Richtungen sind eindeutig bestimmt
      \end{einr}
    \end{itemize}
  \end{itemize}
\end{frame}
\setbeamercolor{background canvas}{bg=}

\section*{Fragen?}

\section{Die affine Struktur auf \dots}
\begin{frame}{Die erste affine Struktur}

\end{frame}
\begin{frame}{Die zweite affine Struktur}

\end{frame}

%%% Local Variables:
%%% mode: latex
%%% TeX-master: "kolloquiumVortrag"
%%% End:
