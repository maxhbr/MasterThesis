\section{Von meromorphen Zusammenhängen zu Systemen}
\begin{frame}[t]{Meromorphe Zusammenhänge}
  \begin{defn}[Meromorphe Zusammenhänge]
    Ein \emph{(Keim eines) Meromorphen Zusammenhangs} ist ein Tupel
    $(\cM,\nabla)$\only<2->{, wobei}
    \begin{itemize}
    \item<2-> $\cM$ ist ein $\C(\!\{t\}\!)$-Vektorraum von Dimension $n$
      \only<3->{und}
    \item<3-> $\nabla:\cM\to \cM$, welcher die \emph{Leibniz Regel}
      \[
        \nabla(fm)=\frac{d}{dt} f \cdot m + f \nabla(m)
      \]
      für alle $f\in\C(\!\{t\}\!)$ und $m\in\cM$ erfüllt.
    \end{itemize}
  \end{defn}
\end{frame}
\begin{frame}[t]{Systeme und die Gauge Transformation}
  Wähle eine $\C(\!\{t\}\!)$-Basis $\underline{e}=(e_1,e_2,\dots,e_n)$ von
  $\cM$.
  Sei $A=(a_{jk})_{j,k\in\{1,\dots,n\}}\in\Gl_n(\C(\!\{t\}\!))$ so dass
  \[
    \nabla e_k=-\!\!\sum_{1\leq j\leq n}\!\! a_{jk}(t)e_j
    \text{, für alle $k\in\{1,\dots,n\}$.}
  \]
  \only<2->{
    Für ein $x=\underline{e}\cdot X$ erfüllt diese Matrix
    $\nabla\left(\underline{e}\cdot X\right)=\underline{e}\left(dX-AX\right)$.
  }

  \only<3->{
    \begin{defn}
      \begin{enumerate}
      \item \textcolor{purple}{The matrix $A$ is called a \emph{connection
            matrix} of $(\cM,\nabla)$.}
      \item \dots \emph{das System $[A]$} \dots
      \end{enumerate}
    \end{defn}
    \TODO[Gauge trafo]
  }
\end{frame}
\begin{frame}[t]{Systeme und die Gauge Transformation}
  \textcolor{purple}{
    By \emph{meromorphic transformation}, or just \emph{transformation}, of a
    system we mean a
    $\C(\!\{t\}\!)$-linear\footnote{$\C(\!\{t\}\!):=\C\{t\}[t^{-1}]$.} change of
    the trivialization.
    Such a change is given by a matrix $F$ in $\Gl_n\left(\C(\!\{t\}\!)\right)$
    and the transformed connection matrix ${}^F\!A$ is obtained through the
    Gauge transformation
  }
  \[
    {}^F\!A=(dF)F^{-1} + FAF^{-1} \,.
  \]
  \textcolor{purple}{
    Let $B:={}^F\!A$ be the transformed matrix and rewrite the equation
    to obtain the linear differential equation
  }
  \[
    \frac{dF}{dt}=BF-FA
  \]
  \textcolor{purple}{
    which will be denoted by $[A,B]$.
  }
  \begin{rem}
    \textcolor{purple}{
      The system matrix $B$ is obtained from $A$ by transformation $F$ if and
      only if $F$ solves the linear differential system $[A,B]$.
    }
  \end{rem}
\end{frame}

\subsection{Die klassifizierende Menge der Systeme}
\begin{frame}[t]{Die klassifizierende Menge der Systeme}
  \begin{defn}
    Die \emph{klassifizierende Menge}
    \[
      \cH(A^0):=\left\{\left[\bigl(A,\hat F\bigl)\right]
        \mid A={}^{\hat F}\!A^0 \text{ for some } \hat F\in G(\!(t)\!)\right\} \,.
    \]
    ist die Menge der Äquivalenzklassen der Markierten Paare \TODO[\dots]
  \end{defn}
\end{frame}

\subsection{Levelt-Turittin und die Normalform}


\setbeamercolor{background canvas}{bg=gray}
\subsection{Fundamentallösungen}
\begin{frame}[t]{Fundamentallösungen}
  \TODO[nötig?]
  \begin{defn}
    Eine \emph{Fundamentallösung (auf einem Sektor \boldmath$\mathfrak{s}$)}
    eines Systems $[A]$ ist eine Matrix $\cY$ bestehend aus $n$ linear
    unabhängigen Lösungen als Spalten.
  \end{defn}
\end{frame}
\setbeamercolor{background canvas}{bg=}

\section{Die Stokes Matrizen}
\begin{frame}[t]{Die Stokes Keime}
  \setbeamercovered{transparent}
  \TODO[alternativ: S.Garbe -> S.Garbe.Repr -> S.Mat -> S.Grp]
  \begin{defn}[Stokes Gruppe]
    Ein \emph{Stokes Keim} ist ein Keim $F\in\Gl_n(\cA_\theta)$ in Richtung
    $\theta\in S^1$ welcher
    \begin{enumerate}
    \item<1-> \emph{multiplikativ flach} ($F$ ist asymptotisch zu $\id$ bei $\theta$),
    \item<2-> eine \emph{Isotropie von \boldmath$A^0$} (${}^FA^0=A^0$) und
    \item<4-> von \emph{maximal decay} \TODO[siehe unten]
    \end{enumerate}
    ist. Die Gruppe dieser Keime wird mit $\Sto_\theta(A^0)$ bezeichnet.
  \end{defn}
  \only<3->{
    \begin{defn}
      Die \emph{Keime bei \boldmath$\theta\in S^1$ der Stokes Garbe
        \boldmath$\Lambda_\theta(A^0)$} sind die Keime $F\in\Gl_n(\cA_\theta)$,
      welche die ersten zwei Bedingungen erfüllen.
    \end{defn}
  }
\end{frame}
\begin{frame}[t]{Maximal decay?}
  \begin{defn}
    \begin{enumerate}
    \item<1-> Eine Funktion $e^{q(t^{-1})}$, wobei
      $q(t^{-1})\in\frac{a}{t^{k}}+o(t^{-k})$, hat \emph{maximal decay in
        Richtung \boldmath$\theta$} genau dann wenn $ae^{-ik\theta}$ reell
      negativ ist.
    \item<2-> Definiere die Relation $q_j\myrel{\theta}q_l$ als äquivalent zu
      \begin{einr}
        $e^{(q_l-q_j)(t^{-1})}$ ist von maximal decay in Richtung $\theta$.
      \end{einr}
    \end{enumerate}
  \end{defn}
\end{frame}
\begin{frame}[t]{Wann ist $F\in\Lambda_\theta(A^0)$ von maximal decay?}
  Ein Keim $F\in\Lambda_\theta(A^0)$, der die ersten zwei
  Eigenschaften\only<-2>{\footnote{Multiplikativ flach und Isotropie von
      $A^0$.}} erfüllt, sieht wie folgt aus
  \begin{align*}
    F  &=t^L e^{Q(t^{-1})}
         \only<1>{
         \underset{\tikzmarkt{e1}}{
         \textcolor{blue!50!white}{\underbrace{\textcolor{black}{
         \left(1_n+\sum_{(l,j)}C^{(l,j)}\right)
         }}}}}
         \only<2->{
         \left(1_n+\sum_{(l,j)}C^{(l,j)}\right)
         }
         e^{-Q(t^{-1})}t^{-L}
    \\ &\onslide<2->{=
         t^L\left(
         1_n+\sum_{(l,j)}C^{(l,j)}e^{(q_l-q_j)(t^{-1})}
         \right)t^{-L} \,.}
  \end{align*}
  \only<1>{
    \begin{flushright}
      \tikzmarkc{n1}{blue} konstant, also in $\Gl_n(\C)$
      \begin{tikzpicture}[remember picture,overlay]
        \draw[->,blue!50!white,thick] (n1) to[out=180,in=270] (e1);
      \end{tikzpicture}
    \end{flushright}
  }
  \only<3->{
    Also ist $F$
    \begin{enumerate}
    \item multiplikativ flach falls falls für jedes $C^{(l,j)}\neq0$ der Faktor
      $e^{(q_l-q_j)(t^{-1})}$ multiplikativ flach ist\only<4->{, und}
    \item<4-> von maximal decay falls für jedes $C^{(l,j)}\neq0$ die Relation
      $q_j\myrel{\theta}q_l$ erfüllt ist.
    \end{enumerate}
  }
\end{frame}
\begin{frame}[t]{Die Stokes Matrizen}
  \begin{defn}
    Definiere die \emph{Gruppe der Stokes Matrizen} als
    \begin{align*}
      \SSto_\theta(A^0)&:= \Big\{
                         C=(c_{jl})_{j,l\in\{1,\dots,n\}}
                         \in\Gl_n(\C)
                         \mathlarger{\mathlarger{\mid}}
      \\              & \qquad\qquad
                        c_{(l,j)}=\delta_{jl}
                        \text{~außer wenn~}
                        q_j\myrel{\theta}q_l
                        \Big\} \,.
    \end{align*}
  \end{defn}
  \TODO[Beispiel]
  \only<2->{
    \begin{defn}
      Schreibe $C=1_n+\sum_{(l,j)}C^{(l,j)}\in\SSto_\theta(A^0)$, dann sind die
      \emph{Level} von $C$ definiert als die Grade der Polynome $q_j-q_l$ für
      die $C^{(j,l)}\neq0$.
    \end{defn}
  }
\end{frame}

\subsection{Die anti-Stokes Richtungen}
\begin{frame}[t,fragile]{Die anti-Stokes Richtungen}
  \begin{defn}
    Die \emph{anti-Stokes Richtungen} sind die $\theta\in S^1$ für die
    \[
      \Sto_\theta(A^0)\neq\{\id\} \,.
    \]
  \end{defn}
  Das sind genau die $\theta$ für die es $j$ und $l$ gibt so dass
  $q_j\myrel{\theta}q_l$.
  \vspace{-1em}
  \begin{tikzpicture}[scale=3.5]
    \def\kOne{10}
    \def\kTwo{14}
    \clip (-1.1,-0.2) rectangle (1.3,1.3);
    \node[] (zero) at (0,0) {};
    \node[blue] at (-0.9,0) {$S^1$};

    {
      \only<1>{
        \fill[purple!60!white] ({cos(20)},{sin(20)}) circle (.7pt);
        \fill[white] ({cos(20)},{sin(20)}) circle (.4pt);
      }
      \only<2->{
      \foreach \i in {1,2,...,\kTwo}{%
        \pgfmathsetmacro\ang{{20 + \i * 360 / \kTwo}}
        \fill[purple!60!white] ({cos(\ang)},{sin(\ang)}) circle (.7pt);
        \fill[white] ({cos(\ang)},{sin(\ang)}) circle (.4pt);
      }
      \only<4->{
        \foreach \i in {1,2,...,\kTwo}{%
          \pgfmathsetmacro\ang{{15 + \i * 360 / \kTwo}}
          \fill[purple!60!white] ({cos(\ang)},{sin(\ang)}) circle (.7pt);
          \fill[white] ({cos(\ang)},{sin(\ang)}) circle (.4pt);
        }
        \foreach \i in {1,2,...,\kTwo}{%
          \pgfmathsetmacro\ang{{1 + \i * 360 / \kTwo}}
          \fill[purple!60!white] ({cos(\ang)},{sin(\ang)}) circle (.7pt);
          \fill[white] ({cos(\ang)},{sin(\ang)}) circle (.4pt);
        }
      }
      }
      \fill[white] (zero) circle (1cm);
    }

    \only<1-2>{
      \clip (0,0) circle (1cm);
      \fill[brown!60!white] ({cos(20)},{sin(20)}) circle (.7pt);
      \fill[white] ({cos(20)},{sin(20)}) circle (.4pt);
    }
    \only<3->{
      \clip (0,0) circle (1cm);
      \foreach \i in {1,2,...,\kOne}{%
        \pgfmathsetmacro\ang{{20 + \i * 360 / \kOne}}
        \fill[brown!60!white] ({cos(\ang)},{sin(\ang)}) circle (.7pt);
        \fill[white] ({cos(\ang)},{sin(\ang)}) circle (.4pt);
      }
    }

    \only<4->{
      \clip (0,0) circle (1cm);
      \foreach \i in {1,2,...,\kOne}{%
        \pgfmathsetmacro\ang{{8 + \i * 360 / \kOne}}
        \fill[brown!60!white] ({cos(\ang)},{sin(\ang)}) circle (.7pt);
        \fill[white] ({cos(\ang)},{sin(\ang)}) circle (.4pt);
      }
    }

    \draw[blue] (zero) circle (1cm);
    \fill (zero) circle (1pt);
  \end{tikzpicture}
\end{frame}
\begin{frame}[t,fragile]{Zerlegung von $\Sto_\theta(A^0)$}
  \begin{defn}
    \begin{enumerate}[<+->]
    \item Die \emph{Level von
        \boldmath$C=(c_{jl})_{j,l\in\{1,\dots,n\}}\in\SSto_\theta(A^0)$} sind
      \[
        \cK(C):= \left\{\deg(q_j-q_l)\mid c_{jl}\neq0\right\} \subset \cK \,.
      \]
    \item Ein Keim $F$ ist ein \emph{\boldmath$k$-Keim}, wenn
      $\cK(F)\subset\{k\}$, die Gruppe all dieser ist $\SSto_\theta^k(A^0)$.
    \end{enumerate}
  \end{defn}
  % \only<3->{
  \begin{thm}
    Für alle $\theta\in S^1$ gibt es eine eindeutige Zerlegung
    \[ \begin{tikzcd}[row sep=0cm]
        \SSto_\theta(A^0)
        & \prod_{k\in\cK}\SSto_\theta^k(A^0) \lar
        \\ C_{k_1}\cdot C_{k_2}\cdots C_{k_s}
        & \left(C_{k_1}, C_{k_2},\dots, C_{k_s}\right) \lar[maps to]
      \end{tikzcd} \]
  \end{thm}
  % }
\end{frame}

\subsection{Verbesserung der Stokes Matrizen}

\begin{frame}[t]{\dots}
  Sei $T\subset\{1,\dots,n\}\times\{1,\dots,n\}$ eine Positions Menge, welche
  die \emph{vollständigkeits Eigenschaft}
    \begin{einr}
      wenn $(j,k),(k,l)\in T$ dann ist auch $(j,l)\in T$
    \end{einr}
    erfüllt zusammen mit einer Bijektion
    $i:\{1,2,\dots,\mu\} \overset{\cong}{\to}T$.
  \begin{lem}
    Zu jeder Matrix
    \[
      K\in \left\{K=(K_{jl})_{j,l\in\{1,\dots,n\}}\in\GL_n(\C)\mid
        K_{jl}=\delta_{jl} \text{~unless~} (j,l)\in T \right\}
    \]
    gibt es eindeutige $t_1,\dots,t_\mu\in\C$, so dass
    \[
      K=(\id + t_1E_{i(1)})\cdots(\id + t_{\mu}E_{i(\mu)}) \,.
    \]
  \end{lem}
\end{frame}

\section*{Fragen?}

\section{Die affine Struktur auf \dots}
\begin{frame}[t]{Die erste affine Struktur}

\end{frame}
\begin{frame}[t]{Die zweite affine Struktur}

\end{frame}

%%% Local Variables:
%%% mode: latex
%%% TeX-master: "kolloquiumVortrag"
%%% End:
