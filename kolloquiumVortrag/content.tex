\section{Von meromorphen Zusammenhängen zu Systemen}
\begin{frame}[t]{Meromorphe Zusammenhänge}
  \begin{defn}[Meromorphe Zusammenhänge]
  \end{defn}
  \begin{defn}[Zusammenhangsmatrix]
  \end{defn}
\end{frame}
\begin{frame}[t]{Systeme und die Gauge Transformation}
\end{frame}

\subsection{Die klassifizierende Menge der Systeme}
\begin{frame}[t]{\dots}
  \begin{defn}[Markierte Paare]
  \end{defn}
  \begin{defn}[Isomorphismen von Markierten Paaren]
  \end{defn}
  \[
    \hat\Syst_m(A^0):=\bigl\{\bigl(A,\hat F\bigl)
    \mid A={}^{\hat F}\!A^0 \text{ for some } \hat F\in G(\!(t)\!)\bigl\} \,.
  \]
\end{frame}

\subsection{Levelt-Turittin und die Normalform}

\subsection{Fundamentallösungen}
\begin{frame}[t]{Fundamentallösungen}
\end{frame}

\section{Die Stokes Matrizen}
\begin{frame}[t]{Die Stokes Keime}
  \setbeamercovered{transparent}
  \TODO[alternativ: S.Garbe -> S.Garbe.Repr -> S.Mat -> S.Grp]
  \begin{defn}[Stokes Gruppe]
    Ein \emph{Stokes Keim} ist ein Keim $F\in\Gl_n(\cA_\theta)$ in Richtung
    $\theta\in S^1$ welcher
    \begin{enumerate}[<+->]
    \item \emph{multiplikativ flach} ($F$ ist asymptotisch zu $\id$ bei $\theta$),
    \item eine \emph{Isotropie von \boldmath$A^0$} (${}^FA^0=A^0$) und
    \item von \emph{maximal decay} \TODO[siehe unten]
    \end{enumerate}
    ist. Die Gruppe dieser Keime wird mit $\Sto_\theta(A^0)$ bezeichnet.
  \end{defn}
\end{frame}
\begin{frame}[t]{Wann ist $F$ von maximal decay?}
  Ein Keim $F$ der die ersten zwei
  Eigenschaften\only<-2>{\footnote{Multiplikativ flach und Isotropie von
      $A^0$.}} erfüllt, sieht wie folgt aus
    \begin{align*}
      F  &=t^L e^{Q(t^{-1})}
           \only<1>{
             \underset{\tikzmarkt{e1}}{
               \textcolor{blue!50!white}{\underbrace{\textcolor{black}{
                     \left(1_n+\sum_{(l,j)}C^{(l,j)}\right)
           }}}}}
           \only<2->{
             \left(1_n+\sum_{(l,j)}C^{(l,j)}\right)
           }
           e^{-Q(t^{-1})}t^{-L}
      \\ &\onslide<2->{=
           t^L\left(
           1_n+\sum_{(l,j)}C^{(l,j)}e^{(q_l-q_j)(t^{-1})}
           \right)t^{-L} \,.}
    \end{align*}
  \only<1>{
    \begin{flushright}
      \tikzmarkc{n1}{blue} konstant, also in $\Gl_n(\C)$
      \begin{tikzpicture}[remember picture,overlay]
        \draw[->,blue!50!white,thick] (n1) to[bend left] (e1);
      \end{tikzpicture}
    \end{flushright}
  }
  \only<3->{
    \begin{defn}
      \begin{enumerate}
      \item Eine Funktion $e^{q(t^{-1})}$, wobei
        $q(t^{-1})\in\frac{a}{t^{k}}+o(t^{-k})$, hat \emph{maximal decay in
          Richtung \boldmath$\theta$} genau dann wenn $ae^{-ik\theta}$ reell
        negativ ist.
      \item<4-> Definiere die Relation $q_j\myrel{\theta}q_l$ als äquivalent zu
        \begin{einr}
          $e^{(q_l-q_j)(t^{-1})}$ ist von maximal decay in Richtung $\theta$.
        \end{einr}
      \end{enumerate}
    \end{defn}
  }
\end{frame}
\begin{frame}[t]{Die Stokes Matrizen}
  \begin{defn}
    Definiere die \emph{Gruppe der Stokes Matrizen} als
    \begin{align*}
      \SSto_\theta(A^0)&:= \Big\{
                         C=(c_{jl})_{j,l\in\{1,\dots,n\}}
                         \in\Gl_n(\C)
                         \mathlarger{\mathlarger{\mid}}
      \\              & \qquad\qquad
                        c_{(l,j)}=\delta_{jl}
                        \text{~außer wenn~}
                        q_j\myrel{\theta}q_l
                        \Big\} \,.
    \end{align*}
  \end{defn}
  \TODO[Beispiel]
  \only<2->{
    \begin{defn}
      Schreibe $C=1_n+\sum_{(l,j)}C^{(l,j)}\in\SSto_\theta(A^0)$, dann sind die
      \emph{Level} von $C$ definiert als die Grade der Polynome $q_j-q_l$ für
      die $C^{(j,l)}\neq0$.
    \end{defn}
  }
\end{frame}

\subsection{Die anti-Stokes Richtungen}
\begin{frame}[t]{Die anti-Stokes Richtungen}

\end{frame}

\subsection{Verbesserung der Stokes Matrizen}
\begin{frame}[t]{\dots}
  In the paper \cite{BJL1979Birkhoff} from Balser, Jurkat and Lutz is the
  following Lemma stated as Lemma 2 on page 75.
  \begin{lem}\label{lem:UniqueDecompositionWotBlocks}
    Let $T\subset\{1,\dots,n\}\times\{1,\dots,n\}$ be a position set, which
    satisfies the \emph{completeness property}:
    \begin{einr}
      if $(j,k)$ and $(k,l)\in T$ then is also $(j,l)\in T$.
    \end{einr}
    Choose a indexing $i:\{1,2,\dots,\mu \} \overset{\cong}{\to}T$ of the
    position set and denote by $\delta_{jl}\in\C$ the ordinary Kronecker's
    delta, then there exists for every matrix
    \[
      K\in \left\{K=(K_{jl})_{j,l\in\{1,\dots,n\}}\in\GL_n(\C)\mid
        K_{jl}=\delta_{jl} \text{~unless~} (j,l)\in T \right\}
    \]
    unique scalars $t_1,\dots,t_\mu\in\C$ such that
    \[
      K=(\id + t_1E_{i(1)})\cdots(\id + t_{\mu}E_{i(\mu)}) \,.
    \]
  \end{lem}
\end{frame}

\section*{Fragen?}

\section{Die affine Struktur auf \dots}

%%% Local Variables:
%%% mode: latex
%%% TeX-master: "kolloquiumVortrag"
%%% End:
