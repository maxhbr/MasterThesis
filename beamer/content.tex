\def\tilde{\widetilde}
\def\hat{\widehat}

\section{Meromorphe Zusammenhänge}
\subsection{Definition}
\begin{frame}[t]{Meromorphe Zusammenhänge}
  Sei $~\tikzmark{e2}\!\!M$ eine \textbf{Riemann-Fläche} und
  $~\tikzmark{e1}\!\!Z$ ein \textbf{effektiver Divisor} auf $M$.
  \only<1| handout:0>{\vspace{1cm}\begin{flushright}
      \tikzmarkc{n1}{blue} denke an $Z=1\cdot\{0\}$
    \end{flushright}
    \vspace{1cm}
    \begin{flushright}
      \tikzmarkc{n2}{blue} denke an $M=D:=\{t\in\C\mid|t|<r\}$
    \end{flushright}
    \begin{tikzpicture}[remember picture,overlay]
      \draw[->,blue!50!white,thick] (n1) to[bend left] (e1);
      \draw[->,blue!50!white,thick] (n2) to[bend left] (e2);
    \end{tikzpicture}
  }

  \visible<2-3>{%
    Sei $\sM$ ein \textbf{holomorphes Bündel}
    $:\Leftrightarrow{}$ lokal freier $\cO_M$-Modul.
  }

  \visible<3>{%
    \begin{defn}
      \def\myU{\textcolor{green!30!black}{U}}
      \def\mys{\textcolor{blue!60!black}{s}}
      \def\myf{\textcolor{red!60!black}{f}}
      Ein \emph{meromorpher Zusammenhang auf $\sM$ mit Polen auf $Z$}
      ist ein $\C$-linearer Garbenmorphismus
      \[
        \nabla:\sM\to\Omega_M^1(*Z)\otimes\sM
      \]
      welcher für alle $\myU\overset{\text{off.}}{\subset} M$ die
      \textbf{Leibniz Regel}
      \begin{multicols}{2}
        \[
          \nabla(\tikzmark{e2}\myf\overset{\tikzmark{e1}}\mys)
            =\myf\nabla\mys+(d\myf)\otimes\mys
        \]
        \columnbreak
        \begin{itemize}
          \item[\tikzmarkc{n1}{blue}]
            $\mys\textcolor{blue!60!black}{\in\Gamma(\myU,\sM)}$
          \item[\tikzmarkc{n2}{red}]
            $\myf\textcolor{red!60!black}{\in\cO_M(\myU)}$
        \end{itemize}
      \end{multicols}
      erfüllt.
      \begin{tikzpicture}[remember picture,overlay]
        \draw[->,blue!50!white,thick] (n1) to[out=180,in=70] (e1);
        \draw[->,red!50!white,thick] (n2) to[out=180,in=-70] (e2);
      \end{tikzpicture}
    \end{defn}
  }
\end{frame}
\begin{frame}[t,fragile]{Keim eines Meromorphen Zusammenhangs}
  Wenn man an lokaler Information an einer Singulariät interessiert ist,
  redet man oft von Keimen meromorpher Zusammenhänge.
  \vfill
  \only<2>{%
    Ein solcher Keim ist dann gegeben durch
    \begin{itemize}
      \item einen $\C\{t\}$-Modul $~~\tikzmark{e1}\!\!\!\!\cM$ und
        \begin{flushright}
          \tikzmarkc{n1}{gray}
          \textcolor{gray}{Halm von $\sM$}
        \end{flushright}
        \begin{tikzpicture}[remember picture,overlay]
          \draw[->,gray!50!white,thick] (n1) to[out=180,in=-70] (e1);
        \end{tikzpicture}
      \item einer $\C$-linearen Derivation
        $\nabla:\cM\to\C\{t\}[t^{-1}]\otimes\cM$ welche
        \begin{itemize}
          \item die \emph{Leibniz Regel}
            \begin{multicols}{2}
              \[
                \nabla(\tikzmark{e2}f\overset{\tikzmark{e1}}m)
                  =\frac{\partial f}{\partial t}m+f\nabla m
              \]
              \columnbreak
              \begin{itemize}
                \item[\tikzmarkc{n1}{blue}] $m\in\cM$
                \item[\tikzmarkc{n2}{red}] $f\in\C\{t\}$
              \end{itemize}
              \begin{tikzpicture}[remember picture,overlay]
                \draw[->,blue!50!white,thick] (n1) to[out=180,in=70] (e1);
                \draw[->,red!50!white,thick] (n2) to[out=180,in=-70] (e2);
              \end{tikzpicture}
            \end{multicols}
            erfüllt.
        \end{itemize}
    \end{itemize}
  }
\end{frame}
\begin{frame}[t]{Zusammenhangs Matrix}
  Zu einer Basis $\textbf{e}$ von $\cM$ gibt es eine \emph{Zusammenhangs
  Matrix} $A=A'dt$ für ein $A'\in M(n\times n,\C\{t\}[t^{-1}])$ so dass
  $\nabla\textbf{e}=A\textbf{e}$. Damit ist lokal
  \[
    \nabla s=ds-As
  \]
  \vfill
  \visible<2->{%
    Ein Wechsel $F\in G\{t\}:=\Gl_n(\C\{t\})$ der Trivialisierung
    entspricht der \emph{Gauge Transformation} auf Zusammenhangs Matrizen
    \[
      F[A]=(dF)F^{-1}+FAF^{-1} \,.
    \]
  }
  \vfill
  \visible<3->{%
    \begin{defn}
      \begin{itemize}
        \item $A$ \emph{äquivalent zu} $B$ 
          :\Leftrightarrow{}
          $\exists F\in G\{t\}$
          mit $F[A]=B$.
        \item $A$ \emph{formal äquivalent zu} $B$ 
          :\Leftrightarrow{}
          $\exists \hat F\in G\llbracket t\rrbracket$
          mit $\hat F[A]=B$.
      \end{itemize}
    \end{defn}
  }
  \vfill
\end{frame}
\iffalse
% {\setbeamercolor{background canvas}{bg=gray!40!white}
%   \begin{frame}[t]{Zusammenhangs Matrix}
%     Lokal lässt sich ein Schnitt $s\in\sM$ schreiben als
%     $\begin{pmatrix}s_{1}\\\vdots\\s_{n}\end{pmatrix}=s\in\cO_M(U)^n$.
%     \begin{lemdef}
%       Dann
%       \[
%         \nabla s=ds-As
%       \]
%       für eine \emph{Zusammenhangs Matrix}
%       $A\in M\left(n\!\times\! n,\Omega_M^1(*Z)(U)\right)$.
%     \end{lemdef}
%     \visible<2>{%
%       Ein Wechsel $F\in\Gamma(U,\Gl_n(\cO_M))$ der Trivialisierung entspricht der
%       \emph{Gauge Transformation} auf Zusammenhangs Matrizen
%       \[
%         F[A]=(dF)F^{-1}+FAF^{-1} \,.
%       \]
%       \begin{defn}
%         $A$ \emph{äquivalent zu} $B$ 
%         :\Leftrightarrow{}
%         $\exists F\in$
%         mit $F[A]=B$.
%       \end{defn}
%     }
%     % \begin{rem}
%     %   Zwei Zusammenhänge sind isomorph, genau dann, wenn ihre Zusammenhangs
%     %   Matrizen isomorph sind.
%     % \end{rem}
%   \end{frame}
% }
% {\setbeamercolor{background canvas}{bg=gray!40!white}
%   \begin{frame}[t]{Zusammenhangs Matrix}
%     Sei $z\in|Z|$ und $U\ni z$ eine Trivialisierende Teilmenge von $M$.
%     Sei $t$ eine lokale Koordinate, welche an $z$ verschwindet.
%     \begin{lemdef}
%       Dann hat $\nabla$, bezüglich der Trivialisierung, die Form 
%       \[
%         \nabla s=ds-\underline{A}s
%       \]
%       für eine \emph{Zusammenhangs
%       Matrix}
%       \[
%         \underline{A}\in M\left(n\!\times\! n,\Omega_{M}^1(*Z)(U)\right)
%       \]
%     \end{lemdef}
%   \end{frame}
% }
% {\setbeamercolor{background canvas}{bg=gray!40!white}
%   \begin{frame}[t]{Zusammenhangs Matrix: Äquivalenz}
%     \def\mySet{\Gamma(U,\Gl_n(\cO_M))}
%     Ein Wechsel $F\in\mySet$ der Trivialisierung entspricht der
%     \emph{Gauge Transformation} auf Zusammenhangs Matrizen
%     \[
%       F[A]=(dF)F^{-1}+FAF^{-1} \,.
%     \]
%     \begin{defn}
%       $A$ \emph{äquivalent zu} $B$ 
%       :\Leftrightarrow{}
%       $\exists F\in\mySet$
%       mit $F[A]=B$.
%     \end{defn}
%     \begin{defn}
%       $A$ \emph{formal äquivalent zu} $B$ 
%       :\Leftrightarrow{}
%       $\exists F\in\text{?}$
%       mit $F[A]=B$.
%     \end{defn}
%   \end{frame}
% }
\fi

\subsection{Modelle und formale Zerlegung}
\begin{frame}[t]{Formale Zerlegung}
  \def\myPhi{\textcolor{red!60!black}{\phi}}
  \def\myE{\textcolor{green!40!black}{\cE^{\myPhi}}}
  \begin{defn}
    Ein Keim $(\cM,\nabla)$ ist ein \emph{Model} oder eine \emph{formale
    Zerlegung} falls
    \begin{multicols}{2}
      \[
        \lambda:(\cM,\nabla)
        \overset{\cong}{\longrightarrow}
        % \cong
        \bigoplus_{~\tikzmark{e3}\!\!\myPhi}
        \underset{\text{merom. Zus.}}{%
          \underset{\text{elementare}}{%
            \underbrace{%
              \overset{\tikzmark{e2}}{\myE}
              \otimes
              \overset{\tikzmark{e1}}{\textcolor{blue!40!black}{\cR_{\myPhi}}}
            }
          }
        }
      \]
      \columnbreak
      \begin{itemize}
        \item[\tikzmarkb{n2}{green}] irregulär singulär
        \item[\tikzmarkc{n1}{blue}] regulär singulär
        \item[\tikzmarkc{n3}{red}] $\myPhi\in t^{-1}\C[t^{-1}]$ paarweise
          verschieden
      \end{itemize}
      \begin{tikzpicture}[remember picture,overlay]
        \draw[->,blue!50!white,thick] (n1) to[out=180,in=70] (e1);
        \draw[->,green!40!black,thick] (n2) to[out=180,in=70] (e2);
        \draw[->,red!50!white,thick] (n3) to[out=205,in=-70] (e3);
      \end{tikzpicture}
    \end{multicols}
    wobei $(\myE,\nabla)=(\C\{t\},d-d\phi)$.
  \end{defn}
\end{frame}
\begin{frame}[t]{Formale Zerlegung}
  \begin{lem}
    Ist $(\cM,\nabla)$ ein Modell, so lässt sich die Zusammenhangs Matrix
    schreiben als
    \begin{multicols}{2}
      \[
        A^0=d\overset{~\tikzmark{e1}}{\textcolor{green!40!black}{Q}}
           +~\tikzmark{e2}\!\!\Lambda\frac{dt}{t}
      \]
      \columnbreak
      \begin{itemize}
        \item[\tikzmarkb{n1}{green}]
          $\textcolor{green!40!black}{Q
            =\diag(\textcolor{red!60!black}{\phi_1,\dots,\phi_n})}$
        \item[\tikzmarkc{n2}{blue}]
          $\Lambda$ diagonal und konstant
      \end{itemize}
    \end{multicols}
  \end{lem}
  \begin{tikzpicture}[remember picture,overlay]
    \draw[->,green!40!black,thick] (n1) to[out=180,in=70] (e1);
    \draw[->,blue!50!white,thick] (n2) to[out=180,in=-70] (e2);
  \end{tikzpicture}
  \vfill
  \visible<2>{%
    \begin{defn}
      Der Zusammenhang heißt \emph{generisch} falls der Grad der Singularität
      von $\textcolor{red!60!black}{\phi_i}-\textcolor{red!60!black}{\phi_j}=k$
      für alle $i,j$.
    \end{defn}
    Wir beschränken uns auf generische Zusammenhänge.
  }
  \vfill
\end{frame}
\subsection{Levelt-Turittin-Theorem}
\begin{frame}{Levelt-Turittin-Theorem}
  \begin{tthm}[Levelt-Turittin]
    Zu einem Keim $(\cM,\nabla)$ eines meromorphen Zusammenhang gibt es, bis
    auf Verzweigung, immer einen \textbf{formalen} Isomorphismus
    \[
      \hat\lambda:\hat\cM
      \overset{\cong}\longrightarrow
      \hat\cM^{good}
      :=~\tikzmark{e1}\!\!\hat\cO_M\otimes\cM^{good}
    \]
    zu einem \textbf{formalem Modell}.
  \end{tthm}
  \begin{flushright}
    \tikzmarkc{n1}{gray}
    \textcolor{gray}{formale Vervollständigung von $\cO_M$.}
  \end{flushright}
  \begin{tikzpicture}[remember picture,overlay]
    \draw[->,gray!50!white,thick] (n1) to[out=180,in=-90] (e1);
  \end{tikzpicture}
\end{frame}

\section{Asymptotische Entwicklungen}
\subsection{Definition}
\begin{frame}[t]
  % Ab jetzt: $M=D=\{t\in\C\mid|t|<r\}$ für $r$ klein genug und $Z=\{0\}$.
  \begin{defn}
    Sei $\textcolor{red!60!black}{\theta_0,\theta_1}\in S^1$.
    \begin{itemize}
      \item $\textcolor{green!40!black}{\Sect_{\textcolor{blue!60!black}{r}}
        (\textcolor{red!60!black}{\theta_0,\theta_1})}
        := \{t=\rho e^{i\theta}\in\C
          \mid 0<\rho<\textcolor{blue!60!black}{r},
              \theta\in(\textcolor{red!60!black}{\theta_0,\theta_1})\}$
      \item $\Sect(\textcolor{red!60!black}{\theta_0,\theta_1}):=
        \textcolor{green!40!black}{\Sect_{\textcolor{blue!60!black}{r}}
        (\textcolor{red!60!black}{\theta_0,\theta_1})}$
        für $\textcolor{blue!60!black}{r}$ klein genug
      \item $\Sect(\textcolor{red!60!black}{U})
        \!\!\overset{\textcolor{red!60!black}{U=(\theta_0,\theta_1)}}{:=}\!\!
        \Sect(\textcolor{red!60!black}{\theta_0,\theta_1})$
    \end{itemize}
  \end{defn}
  \begin{center}
    \begin{tikzpicture}[scale=2.5]
      \node (zero) at (0,0) {};
      \node[below left] at (zero) {$0$};
      \draw[blue,dashed] (zero) circle (1cm);
      \node[blue,right] at (-1,-.7) {$S^1$};

      \filldraw[fill=green!20!white
        ,draw=green!60!black
        ,thick
        ,path fading=west] (0,0)
      -- ({cos( -30 )*.65},{sin( -30 )*.65}) arc (-30:70:.65) -- cycle;
      \draw[blue!60!black,thick] (0,0) -- ({cos( -30 )*.65},{sin( -30)*.65});
      \node[green!40!black] at (.4,.3)
        {$\Sect_{\textcolor{blue!60!black}{r}}
        (\textcolor{red!60!black}{\theta_0,\theta_1})$};
      \node[green!40!black] at (.3,-.25) {$\textcolor{blue!60!black}{r}$};

      \draw[thick,red!60!black] ({cos( -30 )},{sin( -30 )}) arc (-30:70:1);

      \fill[red!60!black] ({cos( -30 )},{sin( -30 )}) circle(.7pt);
      \fill[red!60!black] ({cos( 70 )},{sin( 70 )}) circle(.7pt);

      \node[red!60!black] at ({1.1 * cos(70)},{1.1 * sin(70)}) {$\theta_1$};
      \node[red!60!black] at ({1.1 * cos(-30)},{1.1 * sin(-30)}) {$\theta_0$};

      \node[red!60!black,right] at (.8,.6) {$U$};

      \fill[white] (zero) circle (1.5pt);
      \fill (zero) circle (.7pt);
    \end{tikzpicture}
  \end{center}
\end{frame}
\begin{frame}[t]{Asymptotische Entwicklung: Definition}
  \begin{defn}
    \def\myN{\textbf{\textcolor{blue!40!black}{N}}}
    \def\mySet{\textcolor{red!40!black}{V}}
    \def\myConst{\textcolor{green!40!black}{C(\myN,\mySet)}}
    $f\in\cO_M(\Sect(U))$ hat $\sum_{n\geq n_0}c_nt^n\in\C(\!(t)\!)$ als
    \emph{asymptotische Entwicklung auf $\Sect(U)$}, falls
    \begin{itemize}
      \item[] $\exists$ $\textcolor{orange!50!black}{r}$ so dass
        \only<1| handout:0>{\begin{itemize}
            \item[] $\forall$ $\myN\geq0$ und
            \item[] $\forall$ abgeschlossenen Intervall $\mySet$ in $U$
          \end{itemize}
        }
        \only<2>{%
          $\forall$ $\myN\geq0$ und
          $\forall$ $\mySet$ abg. Intervall in $U$
        }
        eine Konstante $\myConst$ existiert, so dass
        \[
          \left|
            f(t)-\sum_{n_0\leq n\leq\myN-1}c_nt^n
          \right|
          \leq \myConst|t|^{\myN} \qquad \text{ für alle } 
          t\in\Sect_{\textcolor{orange!50!black}{r}}(\mySet) \,.
        \]
    \end{itemize}
    \only<1| handout:0>{\vspace{-9mm}}
    \visible<2>{%
      Erhalte die Garbe $\sA$ auf $S^1$:
      \[
        \underset{\text{von } S^1}{\underset{\text{off. Intervall}}{U}}
        \mapsto\sA(U)
        \textcolor{gray}{\subset\cO(\Sect(U))}
      \]
      \begin{itemize}
        \item[] wobei $\sA(U)$ die Funktionen mit asymptotischer Entwicklung
          auf $\Sect(U)$.
      \end{itemize}
    }
  \end{defn}
\end{frame}
\subsection{Borel-Ritt Lemma}
\begin{frame}[t,fragile]{Borel-Ritt}
  \begin{llem}[Borel-Ritt]
    Für jedes \textbf{ec~\tikzmark{e1}\!\!hte} offene Intervall $U$ von $S^1$ ist die Abbildung
    \[
      \visible<2->{%
        \textcolor{blue!60!black}{0\to
        ~~~\tikzmark{e2}\!\!\!\!\!\!\sA^{<0}(U) \to}
      }
      \sA(U) \twoheadrightarrow \C(\!(t)\!)
      \visible<2->{%
        \textcolor{blue!60!black}{\to0}
      }
    \]
    welche $f$ die die asymptotische Entwicklung $\hat f$ zuordnet, eine
    surjektion.
  \end{llem}
  \only<1| handout:0>{%
    \begin{flushright}
      \tikzmarkc{n1}{gray}
      \textcolor{gray}{denn $\sA(S^1)=\C\{t\}$}
    \end{flushright}
    \begin{tikzpicture}[remember picture,overlay]
      \draw[->,gray!50!white,thick] (n1) to[out=180,in=-88] (e1);
    \end{tikzpicture}
  }
  \only<2| handout:0>{%
    \begin{flushright}
      \tikzmarkc{n1}{gray}
      \textcolor{gray}{$t\mapsto e^{-\frac{1}{t}}
        \in\sA^{<0}\left(\Sect\left(-\frac{\pi}{2},\frac{\pi}{2}\right)\right)$}
    \end{flushright}
    \begin{tikzpicture}[remember picture,overlay]
      \draw[->,gray!50!white,thick] (n1) to[out=180,in=-90] (e2);
    \end{tikzpicture}
  }
  \vfill
  \only<3>{%
    \begin{thm}
      Für jedes genügend kleinem Intervall $V\subset S^1$ gibt es einen Lift
      \[
        \tilde\lambda:
        \textcolor{gray}{%
          \underset{=:\tilde\cM(V)}{\underbrace{%
              \textcolor{black}{\sA(V)\otimes\cM}
            }
          }
        }
        \overset{\cong}{\longrightarrow}
        \sA(V)\otimes\cM^{good}
      \]
      von $\hat \lambda$.
    \end{thm}
  }
  \vfill
\end{frame}

\section{Stokes-Strukturen}
\begin{frame}[t]{Modulräume}
  Fixiere ein \textbf{Modell} $(\sM^{good},\nabla^{good})$ auf $M=D$ mit Pol
  bei $\{0\}=Z$ und somit auch eine Zusammenhangs Matrix
  $A^0=dQ+\Lambda\frac{dt}{t}$.

  \textbf{Wir sind interessiert an:}
  \begin{center}
      $\left\{(\sM,\nabla)
          \mid \hat f:(\hat\sM,\hat\nabla)
            ~~~~\!\tikzmark{e2}\!\!\!\!\!\!\!\overset{\cong}\longrightarrow
            (\hat\sM^{good},\hat\nabla^{good})
        \right\}\Big
        /~~\tikzmark{e1}\!\!\!\!\sim$.
  \end{center}
  \only<1| handout:0>{\vspace{5mm}\begin{flushright}
      \tikzmarkc{n1}{blue} Isomorphie
    \end{flushright}
    \vspace{5mm}
    \begin{flushright}
      \tikzmarkc{n2}{blue} formale Isomorphie
    \end{flushright}
    \begin{tikzpicture}[remember picture,overlay]
      \draw[->,blue!50!white,thick] (n1) to[out=180,in=-90] (e1);
      \draw[->,blue!50!white,thick] (n2) to[out=180,in=-90] (e2);
    \end{tikzpicture}
  }
  \visible<2-3>{%
    Wir betrachten dazu aber die größere die \textcolor{gray}{punktierte} Menge
    \[
      \sH(\sM^{good}):=\left\{(\sM,\nabla,~\tikzmark{e1}\!\!\hat f)
          \mid \hat f:(\hat\sM,\hat\nabla)
            \overset{\cong}\longrightarrow
            (\hat\sM^{good},\hat\nabla^{good})
        \right\}\Big/\sim
    \]
    \only<2| handout:0>{\vspace{5mm}\begin{flushright}
        \tikzmarkc{n1}{blue} merke den formalen Isomorphismus.
      \end{flushright}
      \begin{tikzpicture}[remember picture,overlay]
        \draw[->,blue!50!white,thick] (n1) to[out=180,in=-90] (e1);
      \end{tikzpicture}
    }
  }

  \only<3>{%
    Als Zusammenhangs Matrizen:
    \[
      \underset{=:\Syst(A^0)}{\underbrace{\!
          \left\{A \mid A=\hat F[A^0]\text{~für ein~}
            \hat F\in G\llbracket t\rrbracket\right\}
      \!}\,}\Big/G\{t\}
    \]
    und
    \[
      \sH(A^0):=\left\{
          (A,\hat F)\in\Syst(A^0)\times G\llbracket t\rrbracket
          \mid A=\hat F[A^0] \right\}\Big/G\{t\}
    \]
  }
\end{frame}

\subsection{Garben Version}
\begin{frame}[t]{Garben Version: Stokes Raum}
  \begin{defn}
    Definiere
    \begin{itemize}
      \item $\Aut^{<0}(\tilde\sM^{good})$
        die Garbe auf $S^1$ der Automorphismen welche die Identität als
        asymptotische Entwicklung haben
        \textcolor{gray}{(Die Schnitte heißen \emph{Stokes Matrizen})}
      \item<2-3> und den \emph{Stokes Raum}
        \[
          \St(\sM^{good})
          :=H^1\left(S^1,\Aut^{<0}\left(\tilde \sM^{good}\right)\right)
        \]
    \end{itemize}
    \visible<3>{%
      \begin{thm}
        $\St(\sM^{good})$ ist ein $\C$-Vektorraum.
      \end{thm}
    }
  \end{defn}
\end{frame}
\begin{frame}[t]{Garben Version}
  Sei $\left[(\sM,\nabla,\hat f)\right]\in\sH(\sM^{good})$.  Es gibt dann eine
  (zyklische\footnote{Schnitt zweier Mengen in $\mathfrak{W}$ ist immer ein
  Intervall.}) Überdeckung $\mathfrak{W}$ von $S^1$ so dass für jedes
  $W_i\in\mathfrak{W}$ ein Lift
  \[
    f_i:(\tilde\sM,\tilde\nabla)_{|W_i}
    \overset{\sim}{\longrightarrow}
    (\tilde\sM^{good},\tilde\nabla^{good})_{|W_i}
  \]
  von $\hat f$ existiert.
  \visible<2->{%
    \\Dann ist $(f_jf_i^{-1})_{i,j}$ ein Kozykel in der Garbe
    $\Aut^{<0}(\tilde\sM^{good})$ relativ zur Überdeckung $\mathfrak{W}$ und
    damit haben wir eine Abbildung
    \[
      \sH(\sM^{good})\longrightarrow
      H^1\left(S^1,\Aut^{<0}\left(\tilde\sM^{good}\right)\right)=\St(\sM^{good})
    \]
  }
  \visible<3>{%
    \begin{tthm}[Malgrange-Sibuya]
      Das ist ein Isomorphismus \textcolor{gray}{von punktierter Mengen}.
    \end{tthm}
  }
\end{frame}

\subsection{Matrix Version}
\begin{frame}[t]{Matrix Version: Modulräume}
  \textbf{Wir sind interessiert an:}
  \[
    \underset{=:\Syst(A^0)}{\underbrace{\!
        \left\{A \mid A=\hat F[A^0]\text{~für ein~}
          \hat F\in G\llbracket t\rrbracket\right\}
    \!}\,}\Big/G\{t\}
  \]
  und betrachten dazu aber die größere die Menge
  \[
    \sH(A^0):=\left\{
      (A,~\tikzmark{e1}\!\!\hat F)\in\Syst(A^0)\times G\llbracket t\rrbracket
        \mid A=\hat F[A^0] \right\}\Big/G\{t\}
  \]
  \vspace{5mm}\begin{flushright}
    \tikzmarkc{n1}{blue} merke die formale Transformation.
  \end{flushright}
  \begin{tikzpicture}[remember picture,overlay]
    \draw[->,blue!50!white,thick] (n1) to[out=180,in=-90] (e1);
  \end{tikzpicture}

\end{frame}
\begin{frame}{Matrix Version}
  \begin{tthm}[Balser, Jurkat, Lutz]
    Es gibt einen Isomorphismus
    \begin{align*}
      \sH(A^0)&\cong(U_+\times U_-)^{k-1}
      &\textcolor{gray}{\cong\C^{(k-1)n(n-1)}}
    \\ [(A,\hat F)]&\mapsto\textbf{S}=(S_1,\dots,S_{2k-2})
    \end{align*}
    \visible<2>{%
      \begin{cor}
        Damit gibt es einen Isomorphismus
        \[
          \Syst(A^0)/G\{t\}\cong(U_+\times U_-)^{k-1}/~\tikzmark{e1}\!\!T \,.
        \]
      \end{cor}
    }
  \end{tthm}
  \visible<2>{%
    \begin{flushright}
      \tikzmarkc{n1}{gray}
      \textcolor{gray}{Torus Wirkung\\durch Konjugation}
    \end{flushright}
    \begin{tikzpicture}[remember picture,overlay]
      \draw[->,gray!50!white,thick] (n1) to[out=180,in=-90] (e1);
    \end{tikzpicture}
  }
\end{frame}
\begin{frame}[t]{Matrix Version: Definitionen}
  \begin{defn}
    Sei $\phi_{ij}(z)$ der führende Term von $\phi_i-\phi_j$.
    \begin{align*}
    d\in\A\subset S^1
    :\Leftrightarrow{} &
    \begin{cases}
      \text{es gibt $i\neq j$ so dass $\phi_{ij}(z)\in\R_{<0}$}
    \\\text{für $z\to0$ auf dem `Strahl durch $d$'.}
    \end{cases}
    \end{align*}
    Die Elemente in $\A$ heißen \emph{anti-Stokes-Richtungen}.
  \end{defn}
  Sei
  \begin{itemize}
    \item $r:=\#\A$,
    \item $l:=r/(2k-2)$ und
    \item $\textbf{d}:=(d_1,\dots,d_l)$ \emph{Halb-Periode}.
  \end{itemize}
  % \begin{defn}
  %   Definiere die \emph{totale Ordnung}
  %   \[
  %     \phi_i\underset{\textbf{\underline{d}}}{<}\phi_j
  %     \Leftrightarrow{}
  %     \phi_{ij}\in\R_{<0}\text{
  %     entlang einem } d\in\textbf{\underline{d}}
  %   \]
  %   und durch $\phi_i\underset{\textbf{\underline{d}}}{<}\phi_j
  %   \Leftrightarrow{}\pi_i<\pi_j$ die
  %   \[
  %     \text{\emph{Permutations Matrix} } (P)_{ij}=\delta_{\pi(i)j} \,.
  %   \]
  % \end{defn}
\end{frame}
\begin{frame}{Matrix Version: Definitionen}
  \begin{multicols}{3}
    \only<1| handout:0>{%
      \begin{tikzpicture}[scale=3]
        \node[] (zero) at (0,0) {};

        \fill[fill=green!20!white] (0,0) -- (1,0) arc (0:60:1.0cm) -- cycle;
        \draw[blue] (zero) circle (1cm);

        \foreach \w/\str in {10/$d_1$}
        {\draw[thick,purple!\w!blue,path fading=west]
            (0,0) -- +({cos( \w )},{sin( \w )}) node[right] {\str};
         \fill[blue!20!white] ({cos( \w )},{sin( \w )}) circle (1pt);
         \foreach \sep in {60,120,180,240,300}
         {\draw[green!20!white,thick] (zero) -- +({cos( \sep )},{sin( \sep )});
          \draw[purple!\w!blue] (0,0) -- +({cos( \w + \sep )},{sin( \w + \sep )});
          \fill[blue!20!white] ({cos( \w + \sep )},{sin( \w + \sep )}) circle (1pt);
         }
        };

        \foreach \sep/\str in {0/$1$
                              ,60/$2$
                              ,120/$k-1$
                              ,180/$4$
                              ,240/$5$
                              ,300/$2k-2$}
        {\node[green!40!black]
          at ({.6 * cos( \sep + 30 )},{.5 * sin( \sep + 30)}) {\str};
        };

        \fill[yellow!60!black] (0.8,0.07) circle (1pt);
        \node[yellow!60!black,right] at (0.8,0.07) {$p$};

        \fill[white] (zero) circle (1pt);
        \fill (zero) circle (.7pt);
      \end{tikzpicture}
    }
    \only<2->{%
      \begin{tikzpicture}[scale=3]
        \node[] (zero) at (0,0) {};

        \fill[fill=green!20!white] (0,0) -- (1,0) arc (0:60:1.0cm) -- cycle;
        \draw[blue] (zero) circle (1cm);

        \foreach \w/\str in {10/$d_1$,
                             20/$d_2$,
                             45/$d_3$,
                             55/$~\tikzmark{e2}\!\!d_l$}
        {\draw[thick,purple!\w!blue,path fading=west]
            (0,0) -- +({cos( \w )},{sin( \w )}) node[right] {\str};
         \fill[blue!20!white] ({cos( \w )},{sin( \w )}) circle (1pt);
         \foreach \sep in {60,120,180,240,300}
         {\draw[green!20!white,thick] (zero) -- +({cos( \sep )},{sin( \sep )});
          \draw[purple!\w!blue] (0,0) -- +({cos( \w + \sep )},{sin( \w + \sep )});
          \fill[blue!20!white] ({cos( \w + \sep )},{sin( \w + \sep )}) circle (1pt);
         }
        };
        \node[right,red!30!black] at ({cos( 355 )},{sin( 355 )})
          {$~\tikzmark{e1}\!\!d_{r}$};

        \foreach \sep/\str in {0/$1$
                              ,60/$2$
                              ,120/$k-1$
                              ,180/$4$
                              ,240/$5$
                              ,300/$2k-2$}
        {\node[green!40!black]
          at ({.6 * cos( \sep + 30 )},{.5 * sin( \sep + 30)}) {\str};
        };

        \fill[yellow!60!black] (0.8,0.07) circle (1pt);
        \node[yellow!60!black,right] at (0.8,0.07) {$p$};

        \fill[white] (zero) circle (1pt);
        \fill (zero) circle (.7pt);
      \end{tikzpicture}
    }
    \columnbreak
    \columnbreak
    \only<2->{%
      \begin{itemize}
        \item[\tikzmarkc{n2}{red}] $l:=r/(2k-2)$
        \item[\tikzmarkc{n1}{blue}] $r:=\#\A$
        % \item $\textbf{\underline{d}}:=(d_1,\dots,d_l)$
      \end{itemize}
      \begin{tikzpicture}[remember picture,overlay]
        \draw[->,blue!50!white,thick] (n1) to[bend left,out=0] (e1);
        \draw[->,red!50!white,thick] (n2) to[bend right,out=0] (e2);
      \end{tikzpicture}
    }
  \end{multicols}
\end{frame}
\begin{frame}[t]{Matrix Version: Asymptotische Analysis}
  \begin{thm}
    \begin{itemize}
      \item Zu $(A,\hat F)$ gibt es auf jedem Sektor $\Sect$ einen kanonischen
        Lift
        \[
          \Sigma_i(\hat F)
          \in\Gl_n\left(\cO_M(\Sect\left(d_i,d_{i+1}\right))\right)
        \]
        so dass $\Sigma_i(\hat F)[A^0]=A$.
      \visible<2->{%
        \item Die analyt. Fortsetzung auf
          $\Sect\left(d_i-\frac{\pi}{2k-2},d_{i+1}+\frac{\pi}{2k-2}\right)$
          von $\Sigma_i(\hat F)$ ist dort immer noch asymptotisch zu $\hat F$.
        }
    \end{itemize}
  \end{thm}
  \visible<3>{%
    \begin{defn}
      Die begrenzenden Richtungen von
      $\Sect\left(d_i-\frac{\pi}{2k-2},d_{i+1}+\frac{\pi}{2k-2}\right)$ heißen
      \emph{Stokes-Richtungen}.
    \end{defn}
  }
\end{frame}
\begin{frame}[t]{Matrix Version: Stokes Faktoren}
  \begin{defn}
    Die \emph{Stokes Faktoren} zu $(A,\hat F)$ sind
    \[
      K_i:= e^{-Q}\cdot e^{-\Lambda} \cdot
        \underset{\kappa_i}{\underbrace{%
        \Sigma_i(\hat F)^{-1}\cdot \Sigma_{i-1}(\hat F)}}
      \cdot  e^{\Lambda}\cdot e^{Q}
    \]
    \visible<2>{%
      \begin{lem}
        $K_i$ ist in der \emph{Gruppe der Stokes Faktoren}
        \[
          \SSto_{d_i}(A^0) := \{K \in \Gl_n(\C) \mid (K)_{ij}
          =\delta_{ij} \text{ außer }
            \phi_{ij}\in\R_{<0}\text{ entlang } d_i \}.
        \]
      \end{lem}
    }
  \end{defn}
\end{frame}
\begin{frame}[t,fragile]
  \begin{lem}
    Sei $\textbf{d}=(d_1,\dots,d_l)$ eine Halb-Periode.
    \begin{align*}
      \prod_{d\in\textbf{d}}\SSto_d(A^0) &\cong U_+\\
      (K_1,\dots,K_l) &\mapsto P^{-1}K_l\dots K_2K_1P
    \end{align*}
    für eine Permutations Matrix $P$.
  \end{lem}
  \visible<2>{%
    \begin{cor}
      \begin{align*}
        \prod_{d\in\A}\SSto_d(A^0)&\cong (U_+\times U_-)^{k-1}\\
        (K_1,\dots,K_r)&\mapsto (S_1,\dots,S_{2k-2})
      \end{align*}
      mit $S_i:=P^{-1}K_{il}\dots K_{(i-1)l+1}P\in U_{+/-}$ falls $i$
      ungerade/gerade die \emph{Stokes Matrizen}.
    \end{cor}
  }
\end{frame}
\begin{frame}{Matrix Version}
  \begin{tthm}[Balser, Jurkat, Lutz]
    Die hier definierte Abbildung
    \begin{align*}
      \sH(A^0)&\cong(U_+\times U_-)^{k-1}
      &\textcolor{gray}{\cong\C^{(k-1)n(n-1)}}
    \\ [(A,\hat F)]&\mapsto\textbf{S}=(S_1,\dots,S_{2k-2})
    \end{align*}
    ist ein Isomorphismus
    \begin{cor}
      Es ist auch 
      \[
        \Syst(A^0)/G\{t\}\cong(U_+\times U_-)^{k-1}/T
      \]
      ein Isomorphismus
    \end{cor}
  \end{tthm}
\end{frame}
