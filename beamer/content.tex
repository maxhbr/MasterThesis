\def\tilde{\widetilde}
\def\hat{\widehat}

\section{Meromorphe Zusammenhänge}
\subsection{Definition}
\begin{frame}[t]{Meromorphe Zusammenhänge}
  Sei $M$ eine \textbf{Riemann-Fläche} und $Z$ ein \textbf{effektiver Divisor}
  auf $M$.
  \\Sei $\sM$ ein \textbf{holomorphes Bündel}
  $:\Leftrightarrow{}$ lokal freier $\cO_M$-Modul.
  \visible<2>{%
    \begin{defn}
      \def\myU{\textcolor{green!30!black}{U}}
      \def\mys{\textcolor{blue!60!black}{s}}
      \def\myf{\textcolor{red!60!black}{f}}
      Ein \emph{meromorpher Zusammenhang auf $\sM$ mit Polen auf $Z$}
      ist eine $\C$-lineare Abbildung
      \[
        \nabla:\sM\to\Omega_M^1(*Z)\otimes\sM
      \]
      welche für alle $\myU\overset{\text{off.}}{\subset} M$ die
      \textbf{Leibniz Regel}
      \begin{multicols}{2}
        \[
          \nabla(\tikzmark{e2}\myf\overset{\tikzmark{e1}}\mys)
            =\myf\nabla\mys+(d\myf)\otimes\mys
        \]
        \columnbreak
        \begin{itemize}
          \item[\tikzmarkc{n1}{blue}]
            $\mys\textcolor{blue!60!black}{\in\Gamma(\myU,\sM)}$ und
          \item[\tikzmarkc{n2}{red}]
            $\myf\textcolor{red!60!black}{\in\cO(\myU)}$
        \end{itemize}
      \end{multicols}
      erfüllt.
      \begin{tikzpicture}[remember picture,overlay]
        \draw[->,blue,opacity=.5] (n1) to[bend right] (e1);
        \draw[->,red,opacity=.5] (n2) to[bend left] (e2);
      \end{tikzpicture}
    \end{defn}
  }
\end{frame}
\begin{frame}[t]{Zusammenhangs Matrix}
  Lokal lässt sich ein Schnitt $s\in\sM$ schreiben als
  $\begin{pmatrix}s_{1}\\\vdots\\s_{n}\end{pmatrix}=s\in\cO(U)^n$.
  Dann
  \[
    \nabla s=ds-As
  \]
  für eine \emph{Zusammenhangs Matrix}
  $A\in M\left(n\times n,\Omega_M^1(*Z)(U)\right)$.
  \vfill
  \visible<2>{%
    Ein Wechsel $F\in\Gl_n(\cO_X(U))$ der Trivialisierung entspricht
    \[
      F[A^0]=(dF)F^{-1}+FA^0F^{-1} \,.
    \]
  }
  % \begin{rem}
  %   Zwei Zusammenhänge sind isomorph, genau dann, wenn ihre Zusammenhangs
  %   Matrizen isomorph sind.
  % \end{rem}
\end{frame}

\subsection{Modelle und formale Zerlegung}
\begin{frame}[t]
  \begin{defn}
    Ein Keim $(\cM,\nabla)$ ist ein \emph{Model} falls
    \begin{multicols}{2}
      \def\myPhi{\textcolor{red!60!black}{\phi}}
      \[
        \lambda:(\cM,\nabla)
        \overset{\cong}{\longrightarrow}
        % \cong
        \bigoplus_{\tikzmark{e3}\myPhi}
        \underset{\text{merom. Zus.}}{%
          \underset{\text{elementare}}{%
            \underbrace{%
              \overset{\tikzmark{e2}}{\cE^{\myPhi}}
              \otimes
              \overset{\tikzmark{e1}}{\cR_{\myPhi}}
            }
          }
        }
      \]
      \columnbreak
      \begin{itemize}
        \item[\tikzmarkc{n2}{blue}] irregulär singulär
        \item[\tikzmarkc{n1}{blue}] regulär singulär
        \item[\tikzmarkc{n3}{red}] $\myPhi\in t^{-1}\C[t^{-1}]$ paarweise
          verschwinden
      \end{itemize}
    \end{multicols}
  \end{defn}
  \begin{tikzpicture}[remember picture,overlay]
    \draw[->,blue,opacity=.5] (n1) to[bend right] (e1);
    \draw[->,blue,opacity=.5] (n2) to[bend right] (e2);
    \draw[->,red,opacity=.5] (n3) to[bend left] (e3);
  \end{tikzpicture}
  \visible<2>{%
    \begin{lem}
      Ist $(\sM,\nabla)$ ein Modell, so lässt sich die Zusammenhangs Matrix
      schreiben als
      \begin{multicols}{2}
        \[
          A^0=d\overset{\tikzmark{e1}}{\textcolor{red!60!black}{Q}}
            +\underset{\tikzmark{e2}}{\Lambda}\frac{dt}{t}
        \]
        \columnbreak
        \begin{itemize}
          \item[\tikzmarkc{n1}{red}]
            $\textcolor{red!60!black}{Q=\diag(\phi_1,\dots,\phi_n)}$
          \item[\tikzmarkc{n2}{blue}]
            $\Lambda$ diagonal und konstant ist.
        \end{itemize}
      \end{multicols}
    \end{lem}
    \begin{tikzpicture}[remember picture,overlay]
      \draw[->,red,opacity=.5] (n1) to[bend right] (e1);
      \draw[->,blue,opacity=.5] (n2) to[bend left] (e2);
    \end{tikzpicture}
  }
\end{frame}
\begin{frame}{Levelt-Turittin-Theorem}
  \begin{tthm}[Levelt-Turittin]
    Zu einem Keim $(\cM,\nabla)$ eines \textbf{formalem} meromorphen
    Zusammenhang gibt es, bis auf Verzweigung, immer einen Isomorphismus
    \[
      \hat\lambda:\hat\cM
      \overset{\cong}\longrightarrow
      \hat\cM^{good}
      =\hat\cO\otimes\cM^{good}
    \]
    zu einem \textbf{formalem Modell}.
  \end{tthm}
\end{frame}

\section{Asymptotische Entwicklungen}
\subsection{Definition}
\begin{frame}[t]
  % Ab jetzt: $M=D=\{t\in\C\mid|t|<r\}$ für $r$ klein genug und $Z=\{0\}$.
  \begin{defn}
    Sei $\textcolor{red!60!black}{\theta_0,\theta_1}\in S^1$.
    \begin{itemize}
      \item $\textcolor{green!40!black}{\Sect_{\textcolor{blue!60!black}{r}}
        (\textcolor{red!60!black}{\theta_0,\theta_1})}
        := \{t=\rho e^{i\theta}\in\C
          \mid 0<\rho<\textcolor{blue!60!black}{r},
              \theta\in(\textcolor{red!60!black}{\theta_0,\theta_1})\}$
      \item $\Sect(\textcolor{red!60!black}{\theta_0,\theta_1}):=
        \Sect_{\textcolor{blue!60!black}{r}}
        (\textcolor{red!60!black}{\theta_0,\theta_1})$
        für $\textcolor{blue!60!black}{r}$ klein genug.
      \item $\Sect(\textcolor{red!60!black}{U})
        \!\!\overset{\textcolor{red!60!black}{U=(\theta_0,\theta_1)}}{:=}\!\!
        \Sect(\textcolor{red!60!black}{\theta_0,\theta_1})$
    \end{itemize}
  \end{defn}
  \begin{center}
    \begin{tikzpicture}[scale=2.5]
      \node (zero) at (0,0) {};
      \node[below left] at (zero) {$0$};
      \draw[blue,dashed] (zero) circle (1cm);

      \filldraw[fill=green!20!white
        ,draw=green!60!black
        ,thick
        ,path fading=west] (0,0)
      -- ({cos( -30 )*.65},{sin( -30 )*.65}) arc (-30:70:.65) -- cycle;
      \draw[blue!60!black,thick] (0,0) -- ({cos( -30 )*.65},{sin( -30)*.65});
      \node[green!40!black] at (.4,.3)
        {$\Sect_{\textcolor{blue!60!black}{r}}
        (\textcolor{red!60!black}{\theta_0,\theta_1})$};
      \node[green!40!black] at (.3,-.25) {$\textcolor{blue!60!black}{r}$};

      \draw[thick,red!60!black] ({cos( -30 )},{sin( -30 )}) arc (-30:70:1);

      \fill[red!60!black] ({cos( -30 )},{sin( -30 )}) circle(.7pt);
      \fill[red!60!black] ({cos( 70 )},{sin( 70 )}) circle(.7pt);

      \node[red!60!black] at ({1.1 * cos(70)},{1.1 * sin(70)}) {$\theta_1$};
      \node[red!60!black] at ({1.1 * cos(-30)},{1.1 * sin(-30)}) {$\theta_0$};

      \node[red!60!black,right] at (1,0) {$U$};

      \fill[white] (zero) circle (1.5pt);
      \fill (zero) circle (.7pt);
    \end{tikzpicture}
  \end{center}
\end{frame}
\begin{frame}[t]{Asymptotische Entwicklung}
  \begin{defn}
    \def\myN{\textbf{\textcolor{blue!40!black}{N}}}
    \def\mySect{\textcolor{red!40!black}{W}}
    \def\myConst{\textcolor{green!40!black}{C(\myN,\mySect)}}
    $f\in\cO(\Sect(U))$ hat $\sum_{n\geq n_0}c_nt^n\in\C(\!(t)\!)$ als
    \emph{asymptotische Entwicklung auf $\Sect(U)$}, falls
    \begin{itemize}
      \item $\exists$ $\textcolor{yellow!30!black}{r}$ so dass
        $\forall$ $\myN\geq0$ und
        $\forall$ $\mySect$ abg. Untersektor von
        $\Sect_{\textcolor{yellow!30!black}{r}}(U)$
        eine Konstante $\myConst$ existiert, so dass
        \[
          \left|
            f(t)-\sum_{n_0\leq n\leq\myN-1}c_nt^n
          \right|
          \leq \myConst|t|^{\myN} \qquad \text{ für alle } t\in \mySect \,.
        \]
    \end{itemize}
    \visible<2>{%
      Erhalte die Garbe $\sA$ auf $S^1$:
      \[
        \underset{\text{von } S^1}{\underset{\text{off. Intervall}}{U}}
        \mapsto\sA(U)
        \textcolor{gray}{\subset\cO(\Sect(U))}
      \]
      wobei $\sA(U)$ die Funktionen mit asymptotischer Entwicklung auf
      $\Sect(U)$.
    }
  \end{defn}
\end{frame}
\subsection{Borel-Ritt Lemma}
\begin{frame}[t,fragile]{Borel-Ritt}
  \begin{llem}[Borel-Ritt]
    Für jedes \textbf{echte} offene Intervall $U$ von $S^1$ ist die Abbildung
    \[
      \visible<2-3>{%
        \textcolor{gray}{0\to \sA^{<0}(U) \to}
      }
      \sA(U) \overset{T}\twoheadrightarrow \C((t))
      \visible<2-3>{%
        \textcolor{gray}{\to0}
      }
    \]
    eine surjektion.
  \end{llem}
  \visible<3>{%
    \begin{thm}
      Für jedes genügend kleinem Intervall $V\subset S^1$ gibt es einen lift
      \[
        \tilde\lambda:
        \underset{=:\tilde\sM(V)}{\underbrace{\sA(V)\otimes\sM}}
        \overset{\cong}{\longrightarrow}
        \sA(V)\otimes\bigoplus_\phi(\sE^\phi\otimes\sR_\phi)
      \]
      von $\hat \lambda$.
    \end{thm}
  }
\end{frame}

\section{Stokes-Strukturen}
\begin{frame}[t]{Modulräume}
  Fixiere ein \textbf{Modell} $(\sM^{good},\nabla^{good})$ auf $D$ mit Pol bei
  $\{0\}=Z$ und somit auch eine Matrix $A^0=dQ+\Lambda\frac{dt}{t}$.

  \textbf{Wir sind interessiert an:}
  \begin{center}
      $\left\{(\sM,\nabla)
          \mid \hat f:(\hat\sM,\hat\nabla)
            \overset{\cong}\longrightarrow
            (\hat\sM^{good},\hat\nabla^{good})
        \right\}\Big/\sim$.
  \end{center}
  \visible<2-3>{%
    Wir betrachten dazu aber die größere die \textcolor{gray}{punktierte} Menge
    \[
      \sH(\sM^{good}):=\left\{(\sM,\nabla,\hat f)
          \mid \hat f:(\hat\sM,\hat\nabla)
            \overset{\cong}\longrightarrow
            (\hat\sM^{good},\hat\nabla^{good})
        \right\}\Big/\sim
    \]
  }

  \visible<3>{%
    Als Zusammenhangs Matrizen:
    \[
      \underset{=:\Syst(A^0)}{\underbrace{%
          \left\{A \mid A=\hat F[A^0]\text{~für ein~}\hat F\in\hat G
          :=\GL_n(\C\llbracket t\rrbracket)\right\}
      }}\Big/G\{t\}
    \]
    und
    \[
      \sH(A^0):=\left\{
          (A,\hat F)\in\Syst(A^0)\times\hat G\mid A=\hat F[A^0]
        \right\}\Big/G\{t\}
    \]
  }
\end{frame}

\subsection{Garben Version}
\begin{frame}[t]{Garben Version: Stokes Raum}
  \begin{defn}
    Definiere den \emph{Stokes Raum}
    \[
      \St(\sM^{good})
        :=H^1\left(S^1,\Aut^{<0}\left(\tilde \sM^{good}\right)\right)
    \]
    wobei
    \begin{itemize}
      \item $\Aut^{<0}(\!\!\!\underset{\sA_{\tilde D}\otimes\sM^{good}}
        {\underbrace{\tilde\sM^{good}}}\!\!\!)$
        die Garbe auf $S^1$ der Automorphismen welche eine asymptotische
        Entwicklung der Identität sind.
        \textcolor{gray}{Die Schnitte heißen \emph{Stokes Matrizen}.}
    \end{itemize}
    \visible<2>{%
      \begin{thm}
        $\St(\sM^{good})$ ist ein $\C$-Vektorraum.
      \end{thm}
    }
  \end{defn}
\end{frame}
\begin{frame}[t]{Garben Version}
  Sei $\left[(\sM,\nabla,\hat f)\right]\in\sH(\sM^{good})$. Dann gibt es eine
  Überdeckung $\mathfrak{W}$ von $S^1$ und für jedes $W_i\in\mathfrak{W}$ einen
  Lift
  \[
    f_i:(\tilde\sM,\tilde\nabla)_{|W_i}
    \overset{\sim}{\longrightarrow}
    (\tilde\sM^{good},\tilde\nabla^{good})_{|W_i}
  \]
  von $\hat f$.
  Dann ist $(f_jf_i^{-1})_{i,j}$ ein Kozykel in der Garbe
  $\Aut^{<0}(\tilde\sM^{good})$ relativ zur Überdeckung $\mathfrak{W}$.
  Damit haben wir eine Abbildung
  \[
    \sH(\sM^{good})\to H^1(S^1,\Aut^{<0}(\tilde\sM^{good}))=\St(\sM^{good})
  \]
  \visible<2>{%
    \begin{tthm}[Malgrange-Sibuya]
      Das ist ein Isomorphismus \textcolor{gray}{von punktierter Mengen}.
    \end{tthm}
  }
\end{frame}

\subsection{Matrix Version}
\begin{frame}{Matrix Version}
  \begin{tthm}[Balser, Jurkat, Lutz]
    Es gibt einen Isomorphismus
    \begin{align*}
      \sH(A^0)&\cong(U_+\times U_-)^{k-1}
      &\textcolor{gray}{\cong\C^{(k-1)n(n-1)}}
    \\ [(A,\hat F)]&\mapsto\textbf{S}=(S_1,\dots,S_{2k-2})
    \end{align*}
    \visible<2>{%
      \begin{cor}
        Es gibt einen Isomorphismus
        \[
          \Syst(A^0)/G\{t\}\cong(U_+\times U_-)^{k-1}/T
        \]
      \end{cor}
    }
  \end{tthm}
\end{frame}
\begin{frame}[t]{Matrix Version: Definitionen}
  \begin{defn}
    Sei $\phi_{ij}(z)$ der führende Term von $\phi_i-\phi_j$.
    \begin{align*}
    d\in\A\subset S^1
    :\Leftrightarrow{} &
    \begin{cases}
      \text{es gibt $i\neq j$ so dass $\phi_{ij}(z)\in\R_{<0}$}
    \\\text{für $z\to0$ auf dem `Strahl durch $d$'.}
    \end{cases}
    \end{align*}
    Die Elemente in $\A$ heißen \emph{anti-Stokes-Richtungen}.
  \end{defn}
  Sei
  \begin{itemize}
    \item $r:=\#\A$,
    \item $l:=r/(2k-2)$ und
    \item $\textbf{d}:=(d_1,\dots,d_l)$ \emph{Halb-Periode}.
  \end{itemize}
  % \begin{defn}
  %   Definiere die \emph{totale Ordnung}
  %   \[
  %     \phi_i\underset{\textbf{\underline{d}}}{<}\phi_j
  %     \Leftrightarrow{}
  %     \phi_{ij}\in\R_{<0}\text{
  %     entlang einem } d\in\textbf{\underline{d}}
  %   \]
  %   und durch $\phi_i\underset{\textbf{\underline{d}}}{<}\phi_j
  %   \Leftrightarrow{}\pi_i<\pi_j$ die
  %   \[
  %     \text{\emph{Permutations Matrix} } (P)_{ij}=\delta_{\pi(i)j} \,.
  %   \]
  % \end{defn}
\end{frame}
\begin{frame}{Matrix Version: Definitionen}
  \begin{multicols}{3}
    \begin{tikzpicture}[scale=3]
      \node[] (zero) at (0,0) {};

      % %%%%%%%%%%%%%%%%%%%%%%%%%%%%%%%%%%%%%%%%%%%%%%%%%%%%%%%%%%%%%%%%%%%%%%%%%
      % \fill[fill=red!60!black] (0,0) -- ({cos( 75 )*1.1},{sin( 75 )*1.1}) arc
      %   (75:105:1.1) -- cycle;
      % \fill[fill=red!60!black] (0,0) -- ({cos( 115 )*1.1},{sin( 115 )*1.1}) arc
      %   (115:145:1.1) -- cycle;

      % \fill[fill=red!60!black] (0,0) -- ({cos( 75 )*1.05},{sin( 75 )*1.05}) arc
      %   (75:145:1.05) -- cycle;

      % \fill[fill=white] (0,0) -- ({cos( 75 )*1},{sin( 75 )*1}) arc
      %   (75:145:1) -- cycle;

      % \node[red!40!black] at (-0.5,1.1) {$\widehat\Sect_i$};

      % \draw[thick,red!40!black] (0,0) -- +({cos( 145 )},{sin( 145 )});
      % \draw[thick,red!40!black] (0,0) -- +({cos( 75 )},{sin( 75 )});
      % \fill[red!40!black] ({cos( 145 )},{sin( 145 )}) circle (1pt);
      % \fill[red!40!black] ({cos( 75 )},{sin( 75 )}) circle (1pt);
      % %%%%%%%%%%%%%%%%%%%%%%%%%%%%%%%%%%%%%%%%%%%%%%%%%%%%%%%%%%%%%%%%%%%%%%%%%

      \fill[fill=green!20!white] (0,0) -- (1,0) arc (0:60:1.0cm) -- cycle;
      \draw[blue] (zero) circle (1cm);

      \foreach \w/\str in {10/$d_1\in S^1$,
                           20/$d_2$,
                           45/$d_3$,
                           55/$d_{\overset{\tikzmark{e2}}{l}}$}
      {\draw[thick,purple!\w!blue,path fading=west]
          (0,0) -- +({cos( \w )},{sin( \w )}) node[right] {\str};
       \fill[blue!20!white] ({cos( \w )},{sin( \w )}) circle (1pt);
       \foreach \sep in {60,120,180,240,300}
       {\draw[green!20!white,thick] (zero) -- +({cos( \sep )},{sin( \sep )});
        \draw[purple!\w!blue] (0,0) -- +({cos( \w + \sep )},{sin( \w + \sep )});
        \fill[blue!20!white] ({cos( \w + \sep )},{sin( \w + \sep )}) circle (1pt);
       }
      };
      \node[right,red!30!black] at ({cos( 355 )},{sin( 355 )})
        {$d_{\tikzmark{e1}r}$};

      \foreach \sep/\str in {0/$1$
                            ,60/$2$
                            ,120/$k-1$
                            ,180/$4$
                            ,240/$5$
                            ,300/$2k-2$}
      {\node[green!40!black]
        at ({.6 * cos( \sep + 30 )},{.5 * sin( \sep + 30)}) {\str};
      };

      \fill[yellow!60!black] (0.8,0.07) circle (1pt);
      \node[yellow!60!black,right] at (0.8,0.07) {$p$};

      \fill[white] (zero) circle (1pt);
      \fill (zero) circle (.7pt);
    \end{tikzpicture}
    \columnbreak
    \columnbreak
    \begin{itemize}
      \item[\tikzmarkc{n2}{red}] $l:=r/(2k-2)$
      \item[\tikzmarkc{n1}{blue}] $r:=\#\A$
      % \item $\textbf{\underline{d}}:=(d_1,\dots,d_l)$
    \end{itemize}
    \begin{tikzpicture}[remember picture,overlay]
      \draw[->,blue,opacity=.5] (n1) to[bend left] (e1);
      \draw[->,red,opacity=.5] (n2) to[bend right] (e2);
    \end{tikzpicture}
  \end{multicols}
\end{frame}
\begin{frame}[t]{Garben Version: Asymptotische Analysis}
  \begin{thm}
    \begin{itemize}
      \item Zu $(A,\hat F)$ gibt es auf jedem Sektor
        $\Sect\left(d_i,d_{i+1}\right)$ einen \textcolor{gray}{(kanonischen)}
        Lift
        \[
          \Sigma_i(\hat F)\in\Gl_n(\cO_{\Sect_i})
        \]
        so dass $\Sigma_i(\hat F)[A^0]=A$.
      \visible<2>{%
        \item 
          Die analyt. Fortsetzung auf
          $\Sect\left(d_i-\frac{\pi}{2k-2},d_{i+1}+\frac{\pi}{2k-2}\right)$
          von $\Sigma_i(\hat F)$ ist dort immer noch asymptotisch zu $\hat F$.
        }
    \end{itemize}
  \end{thm}
\end{frame}
\begin{frame}[t]{Garben Version: Stokes Faktoren}
  \begin{defn}
    Die \emph{Stokes Faktoren} zu $(A,\hat F)$ sind
    \[
      K_i:= e^{-Q}\cdot e^{-\Lambda} \cdot
        \underset{\kappa_i}{\underbrace{%
        \Sigma_i(\hat F)^{-1}\cdot \Sigma_{i-1}(\hat F)}}
      \cdot  e^{\Lambda}\cdot e^{Q}
    \]
    \visible<2>{%
      \begin{lem}
        $K_i$ ist in der \emph{Gruppe der Stokes Faktoren}
        \[
          \SSto_{d_i}(A^0) := \{K \in G \mid (K)_{ij}
          =\delta_{ij} \text{ außer }
            \phi_{ij}\in\R_{<0}\text{ entlang } d_i \}.
        \]
      \end{lem}
    }
  \end{defn}
\end{frame}
\begin{frame}[t]
  \begin{lem}
    Sei $\textbf{d}=(d_1,\dots,d_l)$ eine Halb-Periode.
    \begin{align*}
      \prod_{d\in\textbf{d}}\SSto_d(A^0) &\cong PU_+P^{-1}\\
      (K_1,\dots,K_l) &\mapsto K_l\dots K_2K_1\in G
    \end{align*}
  \end{lem}
  \visible<2>{%
    \begin{cor}
      \begin{align*}
        \prod_{d\in\A}\SSto_d(A^0)&\cong (U_+\times U_-)^{k-1}\\
        (K_1,\dots,K_r)&\mapsto (S_1,\dots,S_{2k-2})
      \end{align*}
      mit $S_i:=P^{-1}K_{il}\dots K_{(i-1)l+1}P\in U_{+/-}$ falls $i$
      ungerade/gerade die Stokes Matrizen.
    \end{cor}
  }
\end{frame}
% \begin{frame}[t,fragile]{Matrix Version}
%   \begin{thm}
%     Die Abbildung
%     \[ \begin{tikzcd}[row sep=0em]
%         \sH(A^0) \rar & (U_+\times U_-)^{k-1}
%       \\(A,\hat F) \rar[mapsto] & (S_1,\dots,S_{2k-2})
%     \end{tikzcd} \]
%     ist ein Isomorphismus.
%   \end{thm}
% \end{frame}
\begin{frame}{Matrix Version}
  \begin{tthm}[Balser, Jurkat, Lutz]
    Die hier definierte Abbildung
    \begin{align*}
      \sH(A^0)&\cong(U_+\times U_-)^{k-1}
      &\textcolor{gray}{\cong\C^{(k-1)n(n-1)}}
    \\ [(A,\hat F)]&\mapsto\textbf{S}=(S_1,\dots,S_{2k-2})
    \end{align*}
    ist ein Isomorphismus
    \begin{cor}
      Es ist auch 
      \[
        \Syst(A^0)/G\{t\}\cong(U_+\times U_-)^{k-1}/T
      \]
      ein Isomorphismus
    \end{cor}
  \end{tthm}
\end{frame}
