\section{Meromorphe Zusammenhänge}
\begin{frame}[t]{Meromorpher Zusammenhang}
  Sei $M$ eine \textbf{Riemann-Fläche}, $Z$ ein \textbf{effektiver Divisor} auf
  $M$.
  \\Sei $\sM$ ein \textbf{holomorphes Bündel}
  $:\Leftrightarrow{}$ lokal freier $\cO_M$-Modul.
  \begin{defn}
    \def\myU{\textcolor{green!30!black}{U}}
    \def\mys{\textcolor{blue!60!black}{s}}
    \def\myf{\textcolor{red!60!black}{f}}
    Ein \emph{meromorpher Zusammenhang auf $\sM$ mit Polen auf $Z$}
    ist eine $\C$-lineare Abbildung
    \[
      \nabla:\sM\to\Omega_M^1(*Z)\otimes\sM
    \]
    welche für alle offenen Mengen $\myU\subset M$, Schnitte
    $\mys\textcolor{blue!60!black}{\in\Gamma(\myU,\sM)}$ und holomorphen
    Funktionen $\myf\textcolor{red!60!black}{\in\cO(\myU)}$ die \textbf{Leibniz
    Regel}
    \[
      \nabla(\myf\mys)=\myf\nabla\mys+(d\myf)\otimes\mys
    \]
    erfüllt.
  \end{defn}
\end{frame}
\begin{frame}[t]{Zusammenhangs Matrix}
  Lokal sieht $\sM$ wie $U\times\C^n$ und ein $s\in\sM$ wie
  $\begin{pmatrix}f_{1}\\\vdots\\f_{n}\end{pmatrix}=:f\in\cO(U)^n$, aus.
  \\Dann
  \[
    \nabla s=df-Af
  \]
  wobei $A\in M(n\times n,\Omega_M^1(*Z)(U))$ ist die \emph{Zusammenhangs
  Matrix}.

  Ein Wechsel $F\in\Gl_n(\cO_X(U))$ der Trivialisierung entspricht
  \[
    F[A^0]=(dF)F^{-1}+FA^0F^{-1} \,.
  \]
\end{frame}

\subsection{Modelle und formale Zerlegung}
\begin{frame}[t]
  \begin{defn}
    Ein Keim $(\cM,\nabla)$ ist ein \emph{Model} falls
    \[
      (\cM,\nabla)
      \cong
      \bigoplus_\phi
      \underset{\text{merom. Zus.}}{%
        \underset{\text{elementare}}{%
          \underbrace{\cE^\phi\otimes\cR_\phi}
        }
      }
    \]
    wobei
    \begin{itemize}
      \item die $\phi\in t^{-1}\C[t^{-1}]$ paarweise unterschiedlich sind,
      \item die $\cR_\phi$ regulär singulär sind und
      \item $\cE^\phi$ wird gelöst von $e^\phi$.
    \end{itemize}
  \end{defn}
\end{frame}
\begin{frame}[t]{Zusammenhangs Matrix\dots}
  
\end{frame}
\begin{frame}[t]{Levelt-Turittin-Theorem}
  \begin{tthm}[Levelt-Turittin]
    Zu einem Keim $(\cM,\nabla)$ eines \textbf{formalem} meromorphen
    Zusammenhang gibt es, bis auf Verzweigung, immer einen Isomorphismus
    \[
      \hat\lambda:\hat\cM
      \overset{\cong}\longrightarrow
      \hat\cM^{good}
      =\hat\cO_{x^0}\otimes\cM^{good}
    \]
    zu einem \textbf{formalem Modell}.
  \end{tthm}
\end{frame}

\section{Asymptotische Entwicklungen}
\begin{frame}[t]{defns}
  Sei $r:=\#\A$, $l:=r/(2k-2)$ und $\textbf{\underline{d}}:=(d_1,\dots,d_l)$
  \emph{Halb-Periode}.
  \begin{defn}
    Definiere die \emph{totale Ordnung}
    \[
      \phi_i\underset{\textbf{\underline{d}}}{<}\phi_j
      \Leftrightarrow{}
      \phi_{ij}\in\R_{<0}\text{
      entlang einem } d\in\textbf{\underline{d}}
    \]
    und durch $\phi_i\underset{\textbf{\underline{d}}}{<}\phi_j
    \Leftrightarrow{}\pi_i<\pi_j$ die
    \[
      \text{\emph{Permutations Matrix} } (P)_{ij}=\delta_{\pi(i)j} \,.
    \]
  \end{defn}
\end{frame}
\begin{frame}[t]{title}
  \begin{center}
    \begin{tikzpicture}[scale=3]
      \node[] (zero) at (0,0) {};

      % %%%%%%%%%%%%%%%%%%%%%%%%%%%%%%%%%%%%%%%%%%%%%%%%%%%%%%%%%%%%%%%%%%%%%%%%%
      % \fill[fill=red!60!black] (0,0) -- ({cos( 75 )*1.1},{sin( 75 )*1.1}) arc
      %   (75:105:1.1) -- cycle;
      % \fill[fill=red!60!black] (0,0) -- ({cos( 115 )*1.1},{sin( 115 )*1.1}) arc
      %   (115:145:1.1) -- cycle;

      % \fill[fill=red!60!black] (0,0) -- ({cos( 75 )*1.05},{sin( 75 )*1.05}) arc
      %   (75:145:1.05) -- cycle;

      % \fill[fill=white] (0,0) -- ({cos( 75 )*1},{sin( 75 )*1}) arc
      %   (75:145:1) -- cycle;

      % \node[red!40!black] at (-0.5,1.1) {$\widehat\Sect_i$};

      % \draw[thick,red!40!black] (0,0) -- +({cos( 145 )},{sin( 145 )});
      % \draw[thick,red!40!black] (0,0) -- +({cos( 75 )},{sin( 75 )});
      % \fill[red!40!black] ({cos( 145 )},{sin( 145 )}) circle (1pt);
      % \fill[red!40!black] ({cos( 75 )},{sin( 75 )}) circle (1pt);
      % %%%%%%%%%%%%%%%%%%%%%%%%%%%%%%%%%%%%%%%%%%%%%%%%%%%%%%%%%%%%%%%%%%%%%%%%%

      \fill[fill=green!20!white] (0,0) -- (1,0) arc (0:60:1.0cm) -- cycle;
      \draw[blue] (zero) circle (1cm);

      \foreach \w/\str in {10/$d_1\in S^1$,
                           20/$d_2$,
                           45/$d_3$,
                           55/$d_l$}
      {\draw[thick,purple!\w!blue,path fading=west]
          (0,0) -- +({cos( \w )},{sin( \w )}) node[right] {\str};
       \fill[blue!20!white] ({cos( \w )},{sin( \w )}) circle (1pt);
       \foreach \sep in {60,120,180,240,300}
       {\draw[green!20!white,thick] (zero) -- +({cos( \sep )},{sin( \sep )});
        \draw[purple!\w!blue] (0,0) -- +({cos( \w + \sep )},{sin( \w + \sep )});
        \fill[blue!20!white] ({cos( \w + \sep )},{sin( \w + \sep )}) circle (1pt);
       }
      };

      \foreach \sep/\str in {0/$1$
                            ,60/$2$
                            ,120/$k-1$
                            ,180/$4$
                            ,240/$5$
                            ,300/$2k-2$}
      {\node[green!40!black]
        at ({.6 * cos( \sep + 30 )},{.5 * sin( \sep + 30)}) {\str};
      };

      \fill[yellow!60!black] (0.8,0.07) circle (1pt);
      \node[yellow!60!black,right] at (0.8,0.07) {$p$};

      \fill[white] (zero) circle (1pt);
      \fill (zero) circle (.7pt);
    \end{tikzpicture}
  \end{center}
\end{frame}
\begin{frame}[t]{Asymptotische Entwicklung}
  
\end{frame}
\begin{frame}[t]{Borel-Ritt}
  \begin{llem}[Borel-Ritt]
    Für jedes \textbf{echte} offene Intervall $U$ von $S^1$ ist die Abbildung
    \[
      \textcolor{gray}{0\to \sA^{<0}(U) \to}
      \sA(U) \overset{T}\twoheadrightarrow \C((t))
      \textcolor{gray}{\to0}
    \]
    % \[
    %   \textcolor{gray}{0\to \sA^{<0} \to
    %   \sA \to \pi^{-1}\hat\cO_{D}
    %   \to0}
    % \]
    eine surjektion.
  \end{llem}
  \begin{thm}
    Für jedes genügend kleinem Intervall $V\subset S^1$ gilt
    \[
      \sA(V)\otimes\sM\cong\sA(V)\otimes\bigoplus_\phi(\sE^\phi\otimes\sR_\phi)
    \]
  \end{thm}
\end{frame}

\section{Stokes-Strukturen}
\begin{frame}[t]{Modulräume?}
  
\end{frame}

\subsection{Garben Version}
\begin{frame}[t]{title}
  \begin{defn}
    Definiere den \emph{Stokes Raum}
    \[
      \St(\sM^{good})
        :=H^1\left(S^1,\Aut^{<0}\left(\tilde \sM^{good}\right)\right)
    \]
    wobei
    \begin{itemize}
      \item $\Aut^{<0}(\!\!\!\underset{\sA_{\tilde D}\otimes\sM^{good}}
        {\underbrace{\tilde\sM^{good}}}\!\!\!)$
        die Garbe auf $S^1$ der Automorphismen welche
        \begin{itemize}
          \item mit dem Zusammenhang kompatibel sind und
          \item formal äquivalent zur Identität sind.
        \end{itemize}
        \textcolor{gray}{Die Schnitte heißen \emph{Stokes Matrizen}.}
    \end{itemize}
    \begin{thm}
      $\St(\sM^{good})$ ist ein $\C$-Vektorraum.
    \end{thm}
  \end{defn}
\end{frame}
\begin{frame}[t]{title}
  Sei $(\sM,\nabla,\hat f)\in\sH(\sM^{good})$. Dann gibt es eine Überdeckung
  $\mathfrak{W}$ von $S^1$ und für jedes $W_i\in\mathfrak{W}$ einen
  Lift\footnote{$\hat f_i=\hat f$} von $\hat f$
  \[
    f_i:(\tilde\sM,\tilde\nabla)_{|W_i}
    \overset{\sim}{\longrightarrow}
    (\tilde\sM^{good},\tilde\nabla^{good})_{|W_i} \,.
  \]
  Dann ist $(f_jf_i^{-1})_{i,j}$ ein Kozykel in der Garbe
  $\Aut^{<0}(\tilde\sM^{good})$ relativ zur Überdeckung $\mathfrak{W}$.
  Damit haben wir eine Abbildung
  \[
    \sH(\sM^{good})\to H^1(S^1,\Aut^{<0}(\tilde\sM^{good}))=\St(\sM^{good})
  \]
\end{frame}

\subsection{Matrix Version}
\begin{frame}[t]{title}
  \begin{tthm}[Balser, Jurkat, Lutz]
    Es gibt einen Isomorphismus
    \begin{align*}
      \cH(A^0)&\cong(U_+\times U_-)^{k-1}
      &\textcolor{gray}{\cong\C^{(k-1)n(n-1)}}
    \\ [(A,\hat F)]&\mapsto\textbf{S}=(S_1,\dots,S_{2k-2})
    \end{align*}
    \begin{cor}
      Es gibt einen Isomorphismus
      \[
        \Syst(A^0)/G\{t\}\cong(U_+\times U_-)^{k-1}/T
      \]
    \end{cor}
  \end{tthm}
\end{frame}

