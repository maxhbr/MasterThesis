\chapter{Lie Gruppen und Lie Algebren}
Siehe \cite[Kapittel 3]{warnerLie}.
\begin{comment}
Weitere
\begin{itemize}
\item MakeMS.dvi
\item lie.pdf
\end{itemize}
\end{comment}

\section{Lie Gruppen}
\begin{defn}[Lie Gruppe]
Eine \emph{Lie Gruppe} ist eine differenzierbare Mannigfaltigkeit welche auch
eine Gruppenstruktur trägt, bei der die Abbildung
\begin{align*}
G\times G \to G, & (\sigma,\tau)\mapsto \sigma\tau
\end{align*}
$\cC^\infty$ ist.
\end{defn}
\begin{exmp}
Die Mannigfaltigkeit $\Gl(n,\R)$ aller $n\times n$ Matritzen, welche nicht
singulär sind, ist durch die Matrixmultiplikation eine Lie Gruppe.
\end{exmp}

\section{Lie Algebren}
\begin{defn}
Eine \emph{Lie Algebra} $\mathfrak g$ über $\R$ ist ein realer Vektor Raum
$\mathfrak g$ zusammen mit einem bilinearen Operator 
$[~,~]: \mathfrak g\times\mathfrak g\to\mathfrak g$, welche die \emph{Klammer}
(engl.  \emph{bracket}) genannt wird, die die folgenden Eigenschaften für alle
$x,y,z\in \mathfrak g$ erfüllt:
\begin{enumerate}
\item $[x,y]=-[y,x]$                    \hfill (anti-kommutativität)
\item $[[x,y],z]+[[y,z],x]+[[z,x],y]=0$ \hfill (Jacobi identität)
\end{enumerate}
\end{defn}
\begin{defn} \comm{(aus \protect{\cite[3.6 Definition]{warnerLie}})}
\begin{itemize}
\item 
\emph{Links Translation $l_\sigma:\tau \mapsto \sigma\tau$}
und 
\emph{Rechts Translation $l_\sigma:\tau \mapsto \tau\sigma$}.
\item 
Ein Vektorfeld $X$ auf $G$ heißt \emph{links-invariant}, falls es für jedes
$\sigma\in G$ die Bedingung
\[
dl_\sigma \circ X = X\circ l_\sigma
\]
\comm{oder alternativ $l_\sigma'(X(\tau))=X(\sigma\tau)$ für alle $\tau$} gilt.
\end{itemize}
\end{defn}
\begin{defn} \comm{(aus \protect{\cite[3.8 Definition]{warnerLie}})}
Die \emph{Lie Algebra zu einer Lie Gruppe $G$} ist die Lie Algebra $\mathfrak
g$ der links-invarianten Vektor Felder auf $G$. Alternativ\comm{, und besser},
können wir die Lie Algebra von $G$ auch als den tangentlialen Vektor Raum $G_e$
an der Identität mit der Lie Struktur...
\end{defn}
\begin{exmp}\comm{(aus \protect\cite[Example 3.10 (d)]{warnerLie})}
Sei $\mathfrak{gl}(n,\C)$ die Menge aller complexen $n\times n$ Matritzen und
sei $\Gl(n,\C)\subset\mathfrak{gl}(n,\C)$ die Teilmenge der nicht-singulären.
$\mathfrak{gl}(n,\C)$ ist ein $2n^2$-dimensionaler reeler Vektor Raum, dessen
Basis aus den Matritzen $\delta_{ij}$ sowie den $\sqrt{-1}\delta_{ij}$
($i,j=1,\dots,n$), wobei $\delta_{ij}$ die Matrix ist, die als einzigen Eintrag
eine $1$ im an der $ij$-ten Stelle hat.
$\mathfrak{gl}(n,\C)$ formt eine Lie Algebra, falls wir $[A,B]=AB-BA$ setzen.
$\Gl(n,\C)$ beinhaltet eine mannigfaltigkeits Struktur ...
\comm{(weiter wie in \protect\cite[Example 3.10 (b)]{warnerLie})}
\end{exmp}

\subsection{Koadjungierte Orbiten}
\begin{comment}
\begin{itemize}
\item 
siehe \cite{bryant} auf Seite 86ff
\begin{itemize}
\item 
vor allem \cite[Proposition 3]{bryant} auf Seite 86.
\end{itemize}
\item 
siehe \cite{warnerLie} auf Seite 112ff
\item
siehe \cite{ki99}
für die Symplektische Struktur
\end{itemize}
\end{comment}
\begin{defn}
Für eine Lie Gruppe $G$ ist die \emph{adjungierte Abbildung} 
$\Ad: G \to \End(\mathfrak g)$ 
\comm{, wobei $\End(\mathfrak g)\cong \mathfrak g \otimes \mathfrak g^*$ ist,}
ist gegeben durch
\[
\Ad(\sigma)=\left(L_\sigma\circ (R_\sigma)^{-1}\right)'(e):G_e\to G_e \,.
\]
\end{defn}
\begin{comment}
\begin{defn} \comm{(siehe Seite 16 \protect\cite{bryant})}
 ...die \emph{adjungierte Repräsentation von $G$}...
\end{defn}
\end{comment}
\begin{defn}
Sei $\Ad^*:G\to\Gl(\mathfrak g^*)$ die \emph{koadjungierte Repräsentation}
von $G$.
\comm{This is the so-called "contragredient" representation to the adjoint
representation.}
So dass, für jedes $a\in G$ und $\xi \in \mathfrak g^*$, das Element
$\Ad^*(a)(\xi)\in \mathfrak g^*$ durch die Regel
\[
\Ad^*(a)(\xi)(x)=\xi(\Ad(a^{-1})(x)) \qquad \text{ für alle } x\in \mathfrak g
\]
gegeben ist.
\end{defn}
\begin{bem} \comm{(von \protect\cite[P. 86]{bryant})}
Es ist der entsprechende Lie Algebren Homomorphismus $\ad^*:\mathfrak g \to
\mathfrak{gl(g^*)}$ gegeben durch
\[
\ad^*(\sigma)(\chi)(\tau)=-\chi([x,y]) \,.
\]
\end{bem}
\begin{defn}[Koadjungierte Orbiten] \comm{(von \protect\cite[P. 86]{bryant})}
Die Orbiten $G\cdot \xi$ in $\mathfrak g^*$ sind die
\emph{koadjungierten Orbiten}.
\end{defn}
Jeder koadjungierter Orbit trägt in natürlicher weise eine Symplektische
Struktur, welche gegeben ist, durch...
\begin{comment}
\begin{ex} \comm{(von \protect\cite[p. 96]{bryant})}
\textbf{10.} For any Lie group $G$ and any $\xi\in\mathfrak g^∗$, show that the
symplectic structures $\Omega_\xi$ and $\Omega_{\sigma\cdot\xi}$ on
$G\cdot\xi$ are the same for any $\sigma\in G$.
\end{ex}
\end{comment}

% vim:set ft=tex foldmethod=marker foldmarker={{{,}}}:
