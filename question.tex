%%%%%%%%%%%%%%%%%%%%%%%%%%%%%%%%%%%%%%%%%%%%%%%%%%%%%%%%%%%%%%%%%%%%%%%%%%%%%%%
\chapter{The main question}
Here we want to answer the following question:
\begin{einr}
  Which information is needed to describe the Stokes structure of a
  (unramified) \textbf{multileveled} meromorphic connection?
\end{einr}
To be more precise: we are interested in the representation of an element from
$\prod_{\theta\in\A}\Sto_\theta(A^0)$ and how the information about the
different levels is stored.

\bgroup
\columnratio{0.3}
\begin{paracol}{2}\sloppy
\switchcolumn[0]\noindent
  \vspace*{17mm}
  \begin{itemize}
    \item[] Step 1 \tikzmarkc{n0}{black}
    \item[] Step 2 \tikzmarkc{n1}{black}
  \end{itemize}
\switchcolumn[1]\noindent
  \[ \begin{tikzcd}[column sep=0,row sep=1.8cm]
      G\backslash\hat G(A^0)\dar{\tikzmark{e0}}
      \arrow[dd,dashed, out=0,in=0,"\exp"]
    \\\displaystyle \prod_{\theta\in\cA}\Sto_\theta(A^0)
      \dar{\tikzmark{e1}h}
    \\\St(A^0)
  \end{tikzcd} \]
  \begin{tikzpicture}[remember picture,overlay]
    \draw[->,black!50!white,thick] (n0) to[out=0,in=180] (e0);
    \draw[->,black!50!white,thick] (n1) to[out=0,in=180] (e1);
  \end{tikzpicture}
\end{paracol}
\egroup
In the step one, we will associate to an ambassador $\hat F$ in
$G\backslash\hat G(A^0)$ the corresponding element
$(\phi_{\theta_1},\dots,\phi_{\theta_\nu})$ in
$\prod_{\theta\in\cA}\Sto_\theta(A^0)$. \TODO[We will also discuss \dots]

\rewrite{In the second step, we will} map
$(\phi_{\theta_1},\dots,\phi_{\theta_\nu})$ via the isomorphism $h$ from
theorem~\ref{thm:mainThm2} to an element in $\St(A^0)$.

\begin{comment}
  As an third step, we will apply the morphism $\exp$ from the
  theorem~\ref{thm:mainThm1} and hopefully get the same element in $\St(A^0)$
\end{comment}

%%%%%%%%%%%%%%%%%%%%%%%%%%%%%%%%%%%%%%%%%%%%%%%%%%%%%%%%%%%%%%%%%%%%%%%%%%%%%%%
\section{Step 0: Building our example}
The aim of this section is to build, step by step, an unramified connection
with more than one level.
In fact we wont to have two levels, which might be enough.

\paragraph{Fix the determining polynomials}
Let us start by defining the determining polynomials $q_j(t^{-1})$ of $[A^0]$
such that
\begin{enumerate}
  \item $\cK$ has only two levels, i.e.\  $\cK=\{k_1<k_2\}$.
    This implies, that we need at least three different determining
    polynomials and
  \item all Stokes levels are beared by one direction $\theta_0\in S^1$,
    i.e.\ $\cK_{\theta_0}=\cK$.
\end{enumerate}
\begin{rem}
  In this situation we know, that:
  \begin{enumerate}
    \item
      Let $l_j:=\deg(q_j)$.
      If we assume that all determining polynomials have different leading
      coefficients, we \rewrite{require} (up to permutation) that the first
      $n'$, $0<n'<n$, polynomials should have degree $l_j=k_2$.
      For $n'<j<n$ the polynomials $q_j$ should have degree $l_j\leq k_1$ and
      the last polynomial $q_n$ is allowed to be of degree $l_n\leq k_1$.
    \item In dimension $n=3$ are all anti-Stokes directions determined by
      $k_1$, $k_2$ and $\theta_0$.  The set of all anti-Stokes directions is
      then given by
      \[
        \A=\left\{
            \theta_0+\frac{\pi}{k}\cdot j\mid k\in\cK\text{, } j\in\N
          \right\}\,.
      \]
  \end{enumerate}
\end{rem}
To keep the example as small as possible, we will choose $n=3$.
We will \rewrite{choose without restriction} $\theta_0=0$.

\paragraph{Fix the exponent of formal monodromy}
The information of formal monodromy is given by an invertible constant matrix
$L\in G$.
We then construct the connection matrix by
\[
  A^0:=t^LQ'(t^{-1})t^{-L}+L\frac{1}{t} \in G(\!\{t\}\!)
\]
where $Q(t^{-1})=\diag(q_1(t^{-1}),q_2(t^{-1}),\dots,q_n(t^{-1}))$ and
a corresponding normal solution is given by $\cY_0:=t^Le^{Q(t^{-1})}$
(cf.\ proposition~\ref{prop:fundSolBuilder}).
\begin{comment}
  If we assume that $L$ is diagonal, we can use the commutation of diagonal
  matrices to write $A^0:=Q'(t^{-1})+L\frac{1}{t}$.
\end{comment}

\paragraph{Fix the formal transformation}
Choose a formal transformation $\hat F\in\hat G(A^0)$, i.e.\ an ambassador of
an class in $G\backslash\hat G(A^0)$, and thus a connection matrix
${}^{\hat F}\!A^0$ and the corresponding fundamental solution
$\cY=\hat F\cY_0$.

\begin{comment}
  From the theorem~\ref{thm:summaFaktori} we know in our case, that $\hat F$
  can be factored in
  \[
    \hat F=\hat F_1 \hat F_2
  \]
  where $\hat F_j\in\hat G(A^0)$ is
  \begin{itemize}
    \item $k_j$-summable and
    \item with singular directions belonging to $\A^{k_j}$.
  \end{itemize}
  \TODO[build this way?]
\end{comment}

%%%%%%%%%%%%%%%%%%%%%%%%%%%%%%%%%%%%%%%%%%%%%%%%%%%%%%%%%%%%%%%%%%%%%%%%%%%%%%%
\section{Step 1: The corresponding Stokes cycle in
  $\prod_{\theta\in\A}\Sto_\theta(A^0)$}
\begin{comment}
  \textbf{summability:}
  \cite[III.2]{Loday1994} \cite[78f,80]{Loday2014} (\cite[9]{thboalch} only
  mentions multisummability)
\end{comment}
Fix a set of determinations $\tilde\theta_1,\dots,\tilde\theta_\nu$ of the
anti Stokes directions $\theta_1 ,\dots,\theta_\nu$.

%%%%%%%%%%%%%%%%%%%%%%%%%%%%%%%%%%%%%%%%%%%%%%%%%%%%%%%%%%%%%%%%%%%%%%%%%%%%%%%
\subsection{How do the Stokes germs $\phi_\theta$ look like?}
Here we want to use the faithful matrix representations of Stokes germs,
discussed in section~\ref{sec:matrixReps}, to \rewrite{obtain some statements}
about the \rewrite{structure} of the Stokes germs in our situation.

The proposition~\ref{prop:representation} states, that a matrix representation
of a Stokes germ, i.e.\ an element of $\SSto_\theta(A^0)$, gets mapped to a
Stokes germ via
\begin{align*}
  \rho_{\tilde\theta}^{-1}:
  \SSto_\theta(A^0)
  &\overset{\cong}\to
  \Sto_\theta(A^0)
  \\C_{\cY_{0,\tilde\theta}} &\mapsto
  t^{L}e^{Q(t^{-1})}C_{\cY_{0,\tilde\theta}} e^{-Q(t^{-1})}t^{-L} \,,
\end{align*}
which depends on the determination $\tilde\theta$ of $\theta$ \TODO[is this
dependency only important, in the ramified case?]
(cf.\ proposition~\ref{prop:representation}).

\paragraph{Using the structure of $\SSto_\theta(A^0)$}
\begin{exmp}
  Let $\theta\in\A$ be an anti Stokes direction.
  From the definition of $\SSto_\theta(A^0)$ (cf.\
  definition~\ref{defn:groupOfFaithfullReps}) we know that, if one has $q_1
  \underset{\theta,\max}{\prec} q_2$, the representation has the form
  \[
    \begin{pmatrix}
      1 & \text{\boldmath$c_1$} & \star
    \\\text{\boldmath$0$} & 1 & \star
    \\\star & \star & 1
    \end{pmatrix}
  \]
  where $c_j\in\C$ and $\star\in\C$.
  \begin{s-rem}
    In our situation, we also know, that
    \begin{einr}
      $q_1 \underset{\theta,\max}{\prec} q_2$
      \Leftrightarrow{}
      $q_1 \underset{\theta,\max}{\prec} q_3$
      \qquad \rewrite{respectively} \qquad
      $q_2 \underset{\theta,\max}{\prec} q_1$
      \Leftrightarrow{}
      $q_3 \underset{\theta,\max}{\prec} q_1$
    \end{einr}
    since the leading terms of $q_1-q_2$ and $q_1-q_3$ are equal.
  \end{s-rem}
  Thus the representation has the \rewrite{form}
  \[
    \begin{pmatrix}
      1 & c_1 & \text{\boldmath$c_2$}
    \\0 & 1 & \star
    \\\text{\boldmath$0$} & \star & 1
    \end{pmatrix}\,.
  \]
  If we also know, that neither $q_2 \underset{\theta,\max}{\prec} q_3$ nor
  $q_3 \underset{\theta,\max}{\prec} q_2$ it has the \rewrite{form}
  \[
    \begin{pmatrix}
      1 & c_1 & c_2
    \\0 & 1 & \text{\boldmath$0$}
    \\0 & \text{\boldmath$0$} & 1
    \end{pmatrix}\,.
  \]
  We also know, that every matrix of this \rewrite{form} is a representation to
  some Stokes germ.
  Thus we have an isomorphism
  \begin{align*}
    \vartheta_\theta:\C^2 &\to \SSto_\theta(A^0)
  \\(c_1,c_2)&\mapsto
    \begin{pmatrix}
      1 & c_1 & c_2
    \\0 & 1 & 0
    \\0 & 0 & 1
    \end{pmatrix}
  \end{align*}
\end{exmp}
In fact, we have the following $9$ cases 
\begin{center}
  \def\arraystretch{1.3}
  \setlength\tabcolsep{4mm}
  \begin{tabular}{r|c|c|c}
    & $q_2 \underset{\theta,\max}{\prec} q_3$
    & $q_3 \underset{\theta,\max}{\prec} q_2$
    & else
    \tabularnewline
    \hline
    $q_1 \underset{\theta,\max}{\prec} q_2$ and
    $q_1 \underset{\theta,\max}{\prec} q_3$
    & $\begin{pmatrix} 1 & c_1 & c_2 \\0 & 1 & c_3 \\0 & 0 & 1 \end{pmatrix}$
   \cellcolor{blue!15}
    & $\begin{pmatrix} 1 & c_1 & c_2 \\0 & 1 & 0 \\0 & c_3 & 1 \end{pmatrix}$
   \cellcolor{blue!15}
    & $\begin{pmatrix} 1 & c_1 & c_2 \\0 & 1 & 0 \\0 & 0 & 1 \end{pmatrix}$
   \cellcolor{green!15}
    \tabularnewline
    \hline
    $q_2 \underset{\theta,\max}{\prec} q_1$ and
    $q_3 \underset{\theta,\max}{\prec} q_1$
    & $\begin{pmatrix} 1 & 0 & 0 \\c_1 & 1 & c_3 \\c_2 & 0 & 1 \end{pmatrix}$
   \cellcolor{blue!15}
    & $\begin{pmatrix} 1 & 0 & 0 \\c_1 & 1 & 0 \\c_2 & c_3 & 1 \end{pmatrix}$
   \cellcolor{blue!15}
    & $\begin{pmatrix} 1 & 0 & 0 \\c_1 & 1 & 0 \\c_2 & 0 & 1 \end{pmatrix}$
   \cellcolor{green!15}
    \tabularnewline
    \hline
    else
    & $\begin{pmatrix} 1 & 0 & 0 \\0 & 1 & c_3 \\0 & 0 & 1 \end{pmatrix}$
   \cellcolor{purple!15}
    & $\begin{pmatrix} 1 & 0 & 0 \\0 & 1 & 0 \\0 & c_3 & 1 \end{pmatrix}$
   \cellcolor{purple!15}
    & $\begin{pmatrix} 1 & 0 & 0 \\0 & 1 & 0 \\0 & 0 & 1 \end{pmatrix}$
  \end{tabular}
\end{center}
In the \textcolor{blue!75!black}{blue} cases we have $\cK_\theta=\cK$ and
$\C^3\overset{\vartheta_\theta}{\underset{\cong}{\longrightarrow}}\SSto_\theta(A^0)$.
In the \textcolor{green!75!black}{green} cases $\cK_\theta=\{k_2\}$ and
$\C^2\overset{\vartheta_\theta}{\underset{\cong}{\longrightarrow}}\SSto_\theta(A^0)$ as
well as in the \textcolor{purple!75!black}{purple} cases $\cK_\theta=\{k_1\}$
and $\C^1\overset{\vartheta_\theta}{\underset{\cong}{\longrightarrow}}\SSto_\theta(A^0)$.
Thus, for every $\theta\in\A$, we have an isomorphism
$\rho_{\tilde\theta}^{-1}\circ\vartheta_\theta$.

\paragraph{Decomposition by levels}
In proposition~\ref{prop:filtrationOfStokesGroup} we have defined a
decomposition of the Stokes group $\Sto_\theta(A^0)$ in subgroups generated by
$k$-germs for $k\in\cK$.
In our case, we have at most two level, such that this decomposition looks like
\[
  \phi_\theta=\phi_\theta^{k_1} \phi_\theta^{k_2}
  \overset{i_\theta}\longmapsto
    \left(\phi_\theta^{k_1},\phi_\theta^{k_2}\right)
      \in\Sto_\theta^{k_1}(A^0)\times\Sto_\theta^{k_2}(A^0) \,,
\]
where $\phi_\theta^{k_1}\in\Sto_\theta^{k_1}(A^0)=\Sto_\theta^{<k_2}(A^0)$,
$\phi_\theta^{k_2}\in\Sto_\theta^{k_2}(A^0)$  and $i_\theta$ is the map, wich
gives the factors of this factorization in ascending order.

This decomposition of a germ $\phi_\theta$ is trivial if
$\#\cK(\phi_\theta)\leq1$, thus the interesting cases are the
\textcolor{blue!75!black}{blue} cases.

\begin{exmp}
  Look at the example
  \[
    \vartheta_\theta(c_1,c_2,c_3)=
    t^{L}e^{Q(t^{-1})}
    \begin{pmatrix}
      1 & 0 & 0
    \\c_1 & 1 & 0
    \\c_2 & c_3 & 1
    \end{pmatrix}
    e^{-Q(t^{-1})}t^{-L}
    =\phi_\theta
    \,.
  \]
  According to remark~\ref{rem:algFactorization} the factor
  $\phi_\theta^{k_1}$, generated by the $k_1$-germs, is given by
  \[
    \phi_\theta^{k_1}=
    t^{L}e^{Q(t^{-1})}
    \begin{pmatrix}
      1 & 0 & 0
    \\\text{\boldmath $0$} & 1 & 0
    \\\text{\boldmath $0$} & c_3 & 1
    \end{pmatrix}
    e^{-Q(t^{-1})}t^{-L} \,.
  \]
  The other factor $\phi_\theta^{k_2}$ is then obtained as
  \begin{align*}
    \phi_\theta^{k_2}&=
    \left(\phi_\theta^{k_1}\right)^{-1}
    \phi_\theta^{k_2}
  \\&=t^{L}e^{Q(t^{-1})}
    \begin{pmatrix}
      1     & 0    & 0
    \\0     & 1    & 0
    \\0     & c_3 & 1
    \end{pmatrix}^{-1}
    \underset{=\id}{\underbrace{%
        e^{-Q(t^{-1})}t^{-L}
        t^{L}e^{Q(t^{-1})}
    }}
    \begin{pmatrix}
      1     & 0 & 0
    \\c_1     & 1     & 0
    \\c_2     & c_3 & 1
    \end{pmatrix}
    e^{-Q(t^{-1})}t^{-L}
  \\&=t^{L}e^{Q(t^{-1})}
    \begin{pmatrix}
      1     & 0    & 0
    \\0     & 1    & 0
    \\0     & -c_3 & 1
    \end{pmatrix}
    \begin{pmatrix}
      1     & 0 & 0
    \\c_1     & 1     & 0
    \\c_2     & c_3 & 1
    \end{pmatrix}
    e^{-Q(t^{-1})}t^{-L}
  \\&=t^{L}e^{Q(t^{-1})}
    \begin{pmatrix}
      1     & 0 & 0
    \\c_1     & 1          & 0
    \\c_2-c_1c_3     & 0          & 1
    \end{pmatrix}
    e^{-Q(t^{-1})}t^{-L}
    \,.
  \end{align*}
\end{exmp}
The four interesting decomposition, in our situation are given by
\begin{enumerate}
  \item $\begin{pmatrix} 1 & c_1 & c_2 \\0 & 1 & c_3 \\0 & 0 & 1 \end{pmatrix}
    = \begin{pmatrix} 1 & 0 & 0 \\0 & 1 & c_3 \\0 & 0 & 1 \end{pmatrix}
    \begin{pmatrix} 1 & c_1 & c_2 \\0 & 1 & 0 \\0 & 0 & 1 \end{pmatrix}$
  \item $\begin{pmatrix} 1 & c_1 & c_2 \\0 & 1 & 0 \\0 & c_3 & 1 \end{pmatrix}
    = \begin{pmatrix} 1 & 0 & 0 \\0 & 1 & 0 \\0 & c_3 & 1 \end{pmatrix}
    \begin{pmatrix} 1 & c_1 & c_2 \\0 & 1 & 0 \\0 & 0 & 1 \end{pmatrix}$
  \item $\begin{pmatrix} 1 & 0 & 0 \\c_1 & 1 & c_3 \\c_2 & 0 & 1 \end{pmatrix}
    = \begin{pmatrix} 1 & 0 & 0 \\0 & 1 & c_3 \\0 & 0 & 1 \end{pmatrix}
    \begin{pmatrix} 1 & 0 & 0 \\c_1 -c_2c_3& 1 & 0 \\c_2 & 0 & 1 \end{pmatrix}$
  \item $\begin{pmatrix} 1 & 0 & 0 \\c_1 & 1 & 0 \\c_2 & c_3 & 1 \end{pmatrix}
    = \begin{pmatrix} 1 & 0 & 0 \\0 & 1 & 0 \\0 & c_3 & 1 \end{pmatrix}
    \begin{pmatrix} 1 & 0 & 0 \\c_1 & 1 & 0 \\c_2-c_1c_3 & 0 & 1 \end{pmatrix}$
\end{enumerate}
We will use this decompositions in the other direction, to write the
isomorphisms $\vartheta_\theta:\C^\star\to \SSto_\theta(A^0)$ as
\[
  \bar\vartheta_\theta:
  \C^\star
  \longrightarrow\SSto_\theta^{k_1}(A^0)\times\SSto_\theta^{k_2}(A^0)
  \overset{i_\theta^{-1}}\longrightarrow\SSto_\theta(A^0)
\]
where $\star\in\{1,2,3\}$.

\begin{prop}
  By taking the product of all $\rho_{\tilde\theta}^{-1}\circ\vartheta_\theta$
  we obtain the isomorphism
  \[
    \prod_{\theta\in\A}\rho_{\tilde\theta}^{-1}\circ\bar\vartheta_\theta:
    \C^{2\cdot k_2 + k_1}
    \overset\cong\longrightarrow
    \prod_{\theta\in\A}\Sto_\theta(A^0) \,.
  \]
\end{prop}

\begin{comment}
\begin{thm}
  The set $\Sto(A^0)$, and thus the classifying set, is in our case isomorphic
  to $\C^{k_1+2\cdot k_2}$.
\end{thm}
\end{comment}
We compose this isomorphism as follows:
\begin{enumerate}
  \item Decompose $\C^{k_1+2\cdot k_2}\cong
    \prod_{\theta\in\A^{k_1}} \C \times \prod_{\theta\in\A^{k_2}} \C^2$ and
    define the isomorphism
    \[
      \C^{k_1+2\cdot k_2}\overset{\cong}\longrightarrow
      \prod_{\theta\in\A^{k_1}} \SSto_{\theta}^{k_1}(A^0) \times
      \prod_{\theta\in\A^{k_1}} \SSto_{\theta}^{k_1}(A^0)
    \]
    piecewise as
    \begin{enumerate}
      \item an element
        $\left( (a_{\theta_{\nu_1}},b_{\theta_{\nu_1}})
          ,(a_{\theta_{\nu_2}},b_{\theta_{\nu_2}})
          ,\dots
        \right)\in\prod_{\theta\in\A^{k_2}} \C^2$
        gets mapped to
        \[
          \left.
          \begin{cases}
            \left(
            \begin{pmatrix} 1 & a_{\theta_{\nu_1}} & b_{\theta_{\nu_1}} \\0 & 1 & 0 \\0 & 0 & 1 \end{pmatrix}
            ,\begin{pmatrix} 1 & 0 & 0 \\a_{\theta_{\nu_2}} & 1 & 0 \\b_{\theta_{\nu_2}} & 0 & 1 \end{pmatrix}
            ,\begin{pmatrix} 1 & a_{\theta_{\nu_3}} & b_{\theta_{\nu_3}} \\0 & 1 & 0 \\0 & 0 & 1 \end{pmatrix}
              ,\dots
            \right)
            & ,\substack{\text{~if~} q_1 \underset{\theta_0,\max}{\prec} q_2
              \\\text{~and~} q_1 \underset{\theta_0,\max}{\prec} q_3}
            \\\left(
            \begin{pmatrix} 1 & 0 & 0 \\a_{\theta_{\nu_1}} & 1 & 0 \\b_{\theta_{\nu_1}} & 0 & 1 \end{pmatrix}
            ,\begin{pmatrix} 1 & a_{\theta_{\nu_2}} & b_{\theta_{\nu_2}} \\0 & 1 & 0 \\0 & 0 & 1 \end{pmatrix}
            ,\begin{pmatrix} 1 & 0 & 0 \\a_{\theta_{\nu_3}} & 1 & 0 \\b_{\theta_{\nu_3}} & 0 & 1 \end{pmatrix}
              ,\dots
            \right)
            & ,\substack{\text{~if~} q_2 \underset{\theta_0,\max}{\prec} q_1
              \\\text{~and~} q_3 \underset{\theta_0,\max}{\prec} q_1}
          \end{cases}
          \right\}
          =:\left(
            C_{\theta_{\nu_1}}^{k_2}
            ,C_{\theta_{\nu_2}}^{k_2}
            ,\dots
          \right)
        \]
      \item
        $\left(
          c_{\theta_{\mu_1}}
          ,c_{\theta_{\mu_2}}
          ,\dots
        \right)\in\prod_{\theta\in\A^{k_1}} \C$
        gets mapped to
        \[
          \left.
          \begin{cases}
            \left(
            \begin{pmatrix} 1 & 0 & 0 \\0 & 1 & c_{\theta_{\mu_1}} \\0 & 0 & 1 \end{pmatrix}
            ,\begin{pmatrix} 1 & 0 & 0 \\0 & 1 & 0 \\0 & c_{\theta_{\mu_2}} & 1 \end{pmatrix}
            ,\begin{pmatrix} 1 & 0 & 0 \\0 & 1 & c_{\theta_{\mu_3}} \\0 & 0 & 1 \end{pmatrix}
              ,\dots
            \right)
            &\text{,~if~} q_2 \underset{\theta_0,\max}{\prec} q_3
            \\\left(
            \begin{pmatrix} 1 & 0 & 0 \\0 & 1 & 0 \\0 & c_{\theta_{\mu_1}} & 1 \end{pmatrix}
            ,\begin{pmatrix} 1 & 0 & 0 \\0 & 1 & c_{\theta_{\mu_2}} \\0 & 0 & 1 \end{pmatrix}
            ,\begin{pmatrix} 1 & 0 & 0 \\0 & 1 & 0 \\0 & c_{\theta_{\mu_2}} & 1 \end{pmatrix}
              ,\dots
            \right)
            &\text{,~if~} q_3 \underset{\theta_0,\max}{\prec} q_2
          \end{cases}
        \right\}
        =:\left(
          C_{\theta_{\mu_1}}^{k_1}
          ,C_{\theta_{\mu_2}}^{k_1}
          ,\dots
        \right)
        \]
    \end{enumerate}
  \item We use the inclusions $\A^k\hookrightarrow\A$ and the fact, that
    $\SSto_\theta^k(A^0)=\{\id\}$ if $k\notin\cK_\theta$ to get an isomorphism
    \begin{align*}
      \prod_{\theta\in\A^{k}}\SSto_\theta^{k}(A^0)&\to
      \prod_{\theta\in\A}\SSto_\theta^{k}(A^0)
    \\\left(
        C_{\theta_{\mu_1}}^{k_1}
        ,C_{\theta_{\mu_2}}^{k_1}
        ,\dots
      \right)
      &\mapsto
      \left(
        C_{\theta_{1}}^{k_1}
        ,C_{\theta_{2}}^{k_1}
        ,\dots
      \right)
    \\\left(
        C_{\theta_{\mu_1}}^{k_2}
        ,C_{\theta_{\mu_2}}^{k_2}
        ,\dots
      \right)
      &\mapsto
      \left(
        C_{\theta_{1}}^{k_2}
        ,C_{\theta_{2}}^{k_2}
        ,\dots
      \right)
    \end{align*}
    for every $k\in\cK$.
  \item the next map is induced by the matrix product and given by
    \begin{align*}
      \prod_{\theta\in\A}\SSto_\theta^{k_1}(A^0)
      \times
      \prod_{\theta\in\A}\SSto_\theta^{k_2}(A^0)
      &\to
      \prod_{\theta\in\A}\SSto_\theta(A^0)
    \\\left(\left(
          C_{\theta_{1}}^{k_1}
          ,C_{\theta_{2}}^{k_1}
          ,\dots
        \right),
        \left(
          C_{\theta_{1}}^{k_2}
          ,C_{\theta_{2}}^{k_2}
          ,\dots
      \right)\right)
      &\longmapsto
        \left(
          C_{\theta_{1}}^{k_1}\cdot C_{\theta_{1}}^{k_2}
          ,C_{\theta_{2}}^{k_1}\cdot C_{\theta_{2}}^{k_2}
          ,\dots
      \right)
    \end{align*}
  \item The product of the maps
    \begin{align*}
      \rho_{\tilde\theta}^{-1}:
      \SSto_\theta(A^0)
      &\overset{\cong}\to
      \Sto_\theta(A^0)
      \\C_{\cY_{0,\tilde\theta}} &\mapsto
      t^{L}e^{Q(t^{-1})}C_{\cY_{0,\tilde\theta}} e^{-Q(t^{-1})}t^{-L} \,,
    \end{align*}
    gives an isomorphism $\prod_{\theta\in\A}\SSto_\theta(A^0)
    \overset{\cong}\to \prod_{\theta\in\A}\Sto_\theta(A^0)$.
\end{enumerate}

%%%%%%%%%%%%%%%%%%%%%%%%%%%%%%%%%%%%%%%%%%%%%%%%%%%%%%%%%%%%%%%%%%%%%%%%%%%%%%%
%%%%%%%%%%%%%%%%%%%%%%%%%%%%%%%%%%%%%%%%%%%%%%%%%%%%%%%%%%%%%%%%%%%%%%%%%%%%%%%
%%%%%%%%%%%%%%%%%%%%%%%%%%%%%%%%%%%%%%%%%%%%%%%%%%%%%%%%%%%%%%%%%%%%%%%%%%%%%%%
\newpage
%%%%%%%%%%%%%%%%%%%%%%%%%%%%%%%%%%%%%%%%%%%%%%%%%%%%%%%%%%%%%%%%%%%%%%%%%%%%%%%
%%%%%%%%%%%%%%%%%%%%%%%%%%%%%%%%%%%%%%%%%%%%%%%%%%%%%%%%%%%%%%%%%%%%%%%%%%%%%%%
%%%%%%%%%%%%%%%%%%%%%%%%%%%%%%%%%%%%%%%%%%%%%%%%%%%%%%%%%%%%%%%%%%%%%%%%%%%%%%%
\section{Step 2: The image in $\St(A^0)$}
The map $h$ from theorem~\ref{thm:mainThm2} maps the element
$(\phi_{\theta_1},\dots,\phi_{\theta_\nu})$ to an element of $\St(A^0)$.
It was build by concatenation in the following way
\bgroup
\columnratio{0.35}
\begin{paracol}{2}\sloppy
\switchcolumn[0]\noindent
  \vspace*{40mm}
  \begin{itemize}
    \item[] Step 2.a \tikzmarkc{n1}{black}
    \item[] \qquad Step 2.b \tikzmarkc{n3}{black}
    \item[] \qquad \qquad Step 2.c \tikzmarkc{n4}{black}
  \end{itemize}
\switchcolumn[1]\noindent
  \[ \begin{tikzcd}[column sep=0,row sep=1.8cm]
      \displaystyle \prod_{\theta\in\cA}\Sto_\theta(A^0)
      \dar{\tikzmark{e1}\prod_{\theta\in\A}i_\theta}
      \arrow[dddrrrr, out=0,in=90,"h"]
    \\\displaystyle \prod_{\theta\in\cA}\prod_{k\in\cK}\Sto_\theta^k(A^0)
    &\tikzmark{e2}\!\!\!\!\equiv
    &\displaystyle \prod_{k\in\cK}\prod_{\theta\in\cA}\Sto_\theta^k(A^0)
      \dar{\tikzmark{e3}\prod_{k\in\cK}i^k}
    \\&&\displaystyle \prod_{k\in\cK}\Gamma(\dot\cU^k;\Lambda^k(A^0))
      \dar{\tikzmark{e4}\cT}
    \\&&H^1(\cU;\Lambda(A^0))&=&\St(A^0)
  \end{tikzcd} \]
  \begin{tikzpicture}[remember picture,overlay]
    \draw[->,black!50!white,thick] (n1) to[out=0,in=180] (e1);
    % \draw[->,black!50!white,thick] (n2) to[out=0,in=225] (e2);
    \draw[->,black!50!white,thick] (n3) to[out=0,in=180] (e3);
    \draw[->,black!50!white,thick] (n4) to[out=0,in=180] (e4);
  \end{tikzpicture}
\end{paracol}
\egroup

%%%%%%%%%%%%%%%%%%%%%%%%%%%%%%%%%%%%%%%%%%%%%%%%%%%%%%%%%%%%%%%%%%%%%%%%%%%%%%%
\subsection{Step 2.a: Decompose by levels}
In proposition~\ref{prop:filtrationOfStokesGroup} we have defined a
decomposition of the Stokes group $\Sto_\theta(A^0)$ in subgroups generated by
$k$-germs for $k\in\cK$.
This decomposition of a germ $\phi_\theta$ is trivial if
$\#\cK(\phi_\theta)\leq1$.

In our case, we have only two level, such that this decomposition looks like
\[
  \phi_\theta=\phi_\theta^{k_1} \phi_\theta^{k_2}
  \overset{i_\theta}\longmapsto
    \left(\phi_\theta^{k_1},\phi_\theta^{k_2}\right)
      \in\Sto_\theta^{k_1}(A^0)\times\Sto_\theta^{k_2}(A^0) \,,
\]
where $\phi_\theta^{k_1}\in\Sto_\theta^{k_1}(A^0)=\Sto_\theta^{<k_2}(A^0)$,
$\phi_\theta^{k_2}\in\Sto_\theta^{k_2}(A^0)$  and $i_\theta$ is the map, wich
gives the factors of this factorization in ascending order.

\begin{exmp}
  Look at the example
  \[
    \vartheta_\theta(c_1,c_2,c_3)=
    t^{L}e^{Q(t^{-1})}
    \begin{pmatrix}
      1 & 0 & 0
    \\c_1 & 1 & 0
    \\c_2 & c_3 & 1
    \end{pmatrix}
    e^{-Q(t^{-1})}t^{-L}
    =\phi_\theta
    \,.
  \]
  According to remark~\ref{rem:algFactorization} the factor
  $\phi_\theta^{k_1}$, generated by the $k_1$-germs, is given by
  \[
    \phi_\theta^{k_1}=
    t^{L}e^{Q(t^{-1})}
    \begin{pmatrix}
      1 & 0 & 0
    \\\text{\boldmath $0$} & 1 & 0
    \\\text{\boldmath $0$} & c_3 & 1
    \end{pmatrix}
    e^{-Q(t^{-1})}t^{-L} \,.
  \]
  The other factor $\phi_\theta^{k_2}$ is then obtained as
  \begin{align*}
    \phi_\theta^{k_2}&=
    \left(\phi_\theta^{k_1}\right)^{-1}
    \phi_\theta^{k_2}
  \\&=t^{L}e^{Q(t^{-1})}
    \begin{pmatrix}
      1     & 0    & 0
    \\0     & 1    & 0
    \\0     & c_3 & 1
    \end{pmatrix}^{-1}
    \underset{=\id}{\underbrace{%
        e^{-Q(t^{-1})}t^{-L}
        t^{L}e^{Q(t^{-1})}
    }}
    \begin{pmatrix}
      1     & 0 & 0
    \\c_1     & 1     & 0
    \\c_2     & c_3 & 1
    \end{pmatrix}
    e^{-Q(t^{-1})}t^{-L}
  \\&=t^{L}e^{Q(t^{-1})}
    \begin{pmatrix}
      1     & 0    & 0
    \\0     & 1    & 0
    \\0     & -c_3 & 1
    \end{pmatrix}
    \begin{pmatrix}
      1     & 0 & 0
    \\c_1     & 1     & 0
    \\c_2     & c_3 & 1
    \end{pmatrix}
    e^{-Q(t^{-1})}t^{-L}
  \\&=t^{L}e^{Q(t^{-1})}
    \begin{pmatrix}
      1     & 0 & 0
    \\c_1     & 1          & 0
    \\c_2-c_1c_3     & 0          & 1
    \end{pmatrix}
    e^{-Q(t^{-1})}t^{-L}
    \,.
  \end{align*}
\end{exmp}
The four interesting decomposition, in our situation are given by
\begin{enumerate}
  \item $\begin{pmatrix} 1 & c_1 & c_2 \\0 & 1 & c_3 \\0 & 0 & 1 \end{pmatrix}
    = \begin{pmatrix} 1 & 0 & 0 \\0 & 1 & c_3 \\0 & 0 & 1 \end{pmatrix}
    \begin{pmatrix} 1 & c_1 & c_2 \\0 & 1 & 0 \\0 & 0 & 1 \end{pmatrix}$
  \item $\begin{pmatrix} 1 & c_1 & c_2 \\0 & 1 & 0 \\0 & c_3 & 1 \end{pmatrix}
    = \begin{pmatrix} 1 & 0 & 0 \\0 & 1 & 0 \\0 & c_3 & 1 \end{pmatrix}
    \begin{pmatrix} 1 & c_1 & c_2 \\0 & 1 & 0 \\0 & 0 & 1 \end{pmatrix}$
  \item $\begin{pmatrix} 1 & 0 & 0 \\c_1 & 1 & c_3 \\c_2 & 0 & 1 \end{pmatrix}
    = \begin{pmatrix} 1 & 0 & 0 \\0 & 1 & c_3 \\0 & 0 & 1 \end{pmatrix}
    \begin{pmatrix} 1 & 0 & 0 \\c_1 -c_2c_3& 1 & 0 \\c_2 & 0 & 1 \end{pmatrix}$
  \item $\begin{pmatrix} 1 & 0 & 0 \\c_1 & 1 & 0 \\c_2 & c_3 & 1 \end{pmatrix}
    = \begin{pmatrix} 1 & 0 & 0 \\0 & 1 & 0 \\0 & c_3 & 1 \end{pmatrix}
    \begin{pmatrix} 1 & 0 & 0 \\c_1 & 1 & 0 \\c_2-c_1c_3 & 0 & 1 \end{pmatrix}$
\end{enumerate}

%%%%%%%%%%%%%%%%%%%%%%%%%%%%%%%%%%%%%%%%%%%%%%%%%%%%%%%%%%%%%%%%%%%%%%%%%%%%%%%
\subsection{Step 2.b}
\begin{comment}
  \begin{rem}
    Since we know, that $\Sto_{\theta}^k$ is trivial, if $k\notin\cK_\theta$ we
    can write
    \begin{align*}
      \prod_{\theta\in\cA}\prod_{k\in\cK}\Sto_\theta^k(A^0)
      &\equiv \prod_{k\in\cK}\prod_{\theta\in\cA}\Sto_\theta^k(A^0)
      \\&= \prod_{k\in\cK} \prod_{j=1}^{2\cdot k}
     \Sto_{\theta_0+\frac{\pi}{k}\cdot j}^k(A^0) \,.
    \end{align*}
  \end{rem}
\end{comment}

Let $k\in\cK$.

The sections of $\Lambda(A^0)$ are the flat isotropies (cf.\
definition~\ref{defn:StokesSheaf}). 
The \textbf{isotropy property} for a stokes germ $\phi^k_\theta$ is in our case
satisfied, since a constant matrix conjugated by a fundamental solution of
$A^0$ is a solution of $[A^0,A^0]$:
\begin{align*}
\frac{d}{dt}\phi^k_\theta
  &=\frac{d}{dt}\left(
    t^Le^{Q(t^{-1})}C_{\cY_0,\tilde\theta}e^{-Q(t^{-1})}t^{-L}
  \right) 
\\&=\frac{1}{t}Lt^Le^{Q(t^{-1})}C_{\cY_0,\tilde\theta}e^{-Q(t^{-1})}t^{-L}i
  + t^LQ'(t^{-1})e^{Q(t^{-1})}C_{\cY_0,\tilde\theta}e^{-Q(t^{-1})}t^{-L}
\\&\qquad+ t^Le^{Q(t^{-1})}C_{\cY_0,\tilde\theta}(-Q'(t^{-1}))e^{-Q(t^{-1})}t^{-L}
  + t^Le^{Q(t^{-1})}C_{\cY_0,\tilde\theta}e^{-Q(t^{-1})}\frac{-1}{t}Lt^{-L}
\\&= \left(t^LQ'(t^{-1})t^{-L}+L\frac{1}{t}\right)
  t^Le^{Q(t^{-1})}C_{\cY_0,\tilde\theta}e^{-Q(t^{-1})}t^{-L}
\\&\qquad-t^Le^{Q(t^-1)}C_{\cY_0,\tilde\theta}e^{-Q(t^-1)}t^{-L}
  \left(t^LQ'(t^{-1})t^{-L}+L\frac{1}{t}\right)
\\&=A^0 \phi^k_\theta-\phi^k_\theta A^0 \,.
\end{align*}
This property \rewrite{is preserved, when} we extend the germ $\phi_\theta^k$
to the section $f_\theta^k\in\Gamma(\dot\cU_\theta^k;\Lambda^k(A^0))$. From the
theory of differential equations, we know, that \rewrite{this extension is
unique}.

The \textbf{flatness property} of an section $f_\theta^k$,
i.e.\ $f_\theta^k\sim_{\mathfrak{s}_{\cU_\theta^k}}\id$, is equivalent to
\[
  \left(f_\theta^k\right)_{j,l}\sim_{\mathfrak{s}_{\cU_\theta^k}}
  \begin{cases}
    1 & \text{,~if~} j=l
  \\0 & \text{,~if~} j\neq l
  \,.
  \end{cases}
\]
\TODO{}

%%%%%%%%%%%%%%%%%%%%%%%%%%%%%%%%%%%%%%%%%%%%%%%%%%%%%%%%%%%%%%%%%%%%%%%%%%%%%%%
\subsection{Step 2.c}
The map $\cT$ is induced by the map
$\tau:\prod_{k\in\cK}\Gamma(\dot \cU^k;\Lambda^k(A^0))
\to\Gamma(\dot\cU;\Lambda(A^0))$ (cf.\ definition~\ref{defn:theMapTau}).
\begin{comment}
  For every $\theta\in\A$,
  \begin{align*}
    \tau_\theta:
    \prod_{k\in\cK_\theta}\Gamma(\dot \cU^k;\Lambda^k(A^0))
    &\to\Gamma(\dot\cU_\theta;\Lambda(A^0))
  \\(f_\theta^{k})&\mapsto f_{\theta|\dot\cU_\theta}^{k}
    &&\text{,~if~} \cK_\theta=\{k\}
  \\(f_\theta^{k_1},f_\theta^{k_2})&\mapsto
    f_{\theta|\dot\cU_\theta}^{k_1}\cdot f_{\theta|\dot\cU_\theta}^{k_2}
    &&\text{,~if~} \cK_\theta=\{k_1,h_2\}
  \end{align*}
\end{comment}

%%%%%%%%%%%%%%%%%%%%%%%%%%%%%%%%%%%%%%%%%%%%%%%%%%%%%%%%%%%%%%%%%%%%%%%%%%%%%%%
\section{Summary/\TODO{}}
\begin{thm}
  The set $\Sto(A^0)$, and thus the classifying set, is in our case isomorphic
  to $\C^{k_1+2\cdot k_2}$.
\end{thm}
We compose this isomorphism as follows:
\begin{enumerate}
  \item Decompose $\C^{k_1+2\cdot k_2}\cong
    \prod_{\theta\in\A^{k_1}} \C \times \prod_{\theta\in\A^{k_2}} \C^2$ and
    define the isomorphism
    \[
      \C^{k_1+2\cdot k_2}\overset{\cong}\longrightarrow
      \prod_{\theta\in\A^{k_1}} \SSto_{\theta}^{k_1}(A^0) \times
      \prod_{\theta\in\A^{k_1}} \SSto_{\theta}^{k_1}(A^0)
    \]
    piecewise as
    \begin{enumerate}
      \item an element
        $\left( (a_{\theta_{\nu_1}},b_{\theta_{\nu_1}})
          ,(a_{\theta_{\nu_2}},b_{\theta_{\nu_2}})
          ,\dots
        \right)\in\prod_{\theta\in\A^{k_2}} \C^2$
        gets mapped to
        \[
          \left.
          \begin{cases}
            \left(
            \begin{pmatrix} 1 & a_{\theta_{\nu_1}} & b_{\theta_{\nu_1}} \\0 & 1 & 0 \\0 & 0 & 1 \end{pmatrix}
            ,\begin{pmatrix} 1 & 0 & 0 \\a_{\theta_{\nu_2}} & 1 & 0 \\b_{\theta_{\nu_2}} & 0 & 1 \end{pmatrix}
            ,\begin{pmatrix} 1 & a_{\theta_{\nu_3}} & b_{\theta_{\nu_3}} \\0 & 1 & 0 \\0 & 0 & 1 \end{pmatrix}
              ,\dots
            \right)
            & ,\substack{\text{~if~} q_1 \underset{\theta_0,\max}{\prec} q_2
              \\\text{~and~} q_1 \underset{\theta_0,\max}{\prec} q_3}
            \\\left(
            \begin{pmatrix} 1 & 0 & 0 \\a_{\theta_{\nu_1}} & 1 & 0 \\b_{\theta_{\nu_1}} & 0 & 1 \end{pmatrix}
            ,\begin{pmatrix} 1 & a_{\theta_{\nu_2}} & b_{\theta_{\nu_2}} \\0 & 1 & 0 \\0 & 0 & 1 \end{pmatrix}
            ,\begin{pmatrix} 1 & 0 & 0 \\a_{\theta_{\nu_3}} & 1 & 0 \\b_{\theta_{\nu_3}} & 0 & 1 \end{pmatrix}
              ,\dots
            \right)
            & ,\substack{\text{~if~} q_2 \underset{\theta_0,\max}{\prec} q_1
              \\\text{~and~} q_3 \underset{\theta_0,\max}{\prec} q_1}
          \end{cases}
          \right\}
          =:\left(
            C_{\theta_{\nu_1}}^{k_2}
            ,C_{\theta_{\nu_2}}^{k_2}
            ,\dots
          \right)
        \]
      \item
        $\left(
          c_{\theta_{\mu_1}}
          ,c_{\theta_{\mu_2}}
          ,\dots
        \right)\in\prod_{\theta\in\A^{k_1}} \C$
        gets mapped to
        \[
          \left.
          \begin{cases}
            \left(
            \begin{pmatrix} 1 & 0 & 0 \\0 & 1 & c_{\theta_{\mu_1}} \\0 & 0 & 1 \end{pmatrix}
            ,\begin{pmatrix} 1 & 0 & 0 \\0 & 1 & 0 \\0 & c_{\theta_{\mu_2}} & 1 \end{pmatrix}
            ,\begin{pmatrix} 1 & 0 & 0 \\0 & 1 & c_{\theta_{\mu_3}} \\0 & 0 & 1 \end{pmatrix}
              ,\dots
            \right)
            &\text{,~if~} q_2 \underset{\theta_0,\max}{\prec} q_3
            \\\left(
            \begin{pmatrix} 1 & 0 & 0 \\0 & 1 & 0 \\0 & c_{\theta_{\mu_1}} & 1 \end{pmatrix}
            ,\begin{pmatrix} 1 & 0 & 0 \\0 & 1 & c_{\theta_{\mu_2}} \\0 & 0 & 1 \end{pmatrix}
            ,\begin{pmatrix} 1 & 0 & 0 \\0 & 1 & 0 \\0 & c_{\theta_{\mu_2}} & 1 \end{pmatrix}
              ,\dots
            \right)
            &\text{,~if~} q_3 \underset{\theta_0,\max}{\prec} q_2
          \end{cases}
        \right\}
        =:\left(
          C_{\theta_{\mu_1}}^{k_1}
          ,C_{\theta_{\mu_2}}^{k_1}
          ,\dots
        \right)
        \]
    \end{enumerate}
  \item We use the inclusions $\A^k\hookrightarrow\A$ and the fact, that
    $\SSto_\theta^k(A^0)=\{\id\}$ if $k\notin\cK_\theta$ to get an isomorphism
    \begin{align*}
      \prod_{\theta\in\A^{k}}\SSto_\theta^{k}(A^0)&\to
      \prod_{\theta\in\A}\SSto_\theta^{k}(A^0)
    \\\left(
        C_{\theta_{\mu_1}}^{k_1}
        ,C_{\theta_{\mu_2}}^{k_1}
        ,\dots
      \right)
      &\mapsto
      \left(
        C_{\theta_{1}}^{k_1}
        ,C_{\theta_{2}}^{k_1}
        ,\dots
      \right)
    \\\left(
        C_{\theta_{\mu_1}}^{k_2}
        ,C_{\theta_{\mu_2}}^{k_2}
        ,\dots
      \right)
      &\mapsto
      \left(
        C_{\theta_{1}}^{k_2}
        ,C_{\theta_{2}}^{k_2}
        ,\dots
      \right)
    \end{align*}
    for every $k\in\cK$.
  \item the next map is induced by the matrix product and given by
    \begin{align*}
      \prod_{\theta\in\A}\SSto_\theta^{k_1}(A^0)
      \times
      \prod_{\theta\in\A}\SSto_\theta^{k_2}(A^0)
      &\to
      \prod_{\theta\in\A}\SSto_\theta(A^0)
    \\\left(\left(
          C_{\theta_{1}}^{k_1}
          ,C_{\theta_{2}}^{k_1}
          ,\dots
        \right),
        \left(
          C_{\theta_{1}}^{k_2}
          ,C_{\theta_{2}}^{k_2}
          ,\dots
      \right)\right)
      &\longmapsto
        \left(
          C_{\theta_{1}}^{k_1}\cdot C_{\theta_{1}}^{k_2}
          ,C_{\theta_{2}}^{k_1}\cdot C_{\theta_{2}}^{k_2}
          ,\dots
      \right)
    \end{align*}
  \item The product of the maps
    \begin{align*}
      \rho_{\tilde\theta}^{-1}:
      \SSto_\theta(A^0)
      &\overset{\cong}\to
      \Sto_\theta(A^0)
      \\C_{\cY_{0,\tilde\theta}} &\mapsto
      t^{L}e^{Q(t^{-1})}C_{\cY_{0,\tilde\theta}} e^{-Q(t^{-1})}t^{-L} \,,
    \end{align*}
    gives an isomorphism $\prod_{\theta\in\A}\SSto_\theta(A^0)
    \overset{\cong}\to \prod_{\theta\in\A}\Sto_\theta(A^0)$.
\end{enumerate}
\begin{comment}
\begin{tikzpicture} %{{{
  [scale=2.5
  ,auto
  ,font=\footnotesize
  ,line width=0.8pt
  ,mybox/.style={rectangle
                ,draw
                ,fill=white
                ,inner sep=5pt
                ,text width=3cm
                ,text badly centered
                ,minimum height=1.2cm
                ,font=\bfseries\footnotesize\sffamily}] 
  \node (a1) at (0,0) {$\C^{k_1+2\cdot k_2}$};
  \node (a2a) at (1,.3) {$\displaystyle\prod_{\theta\in\A^{k_2}} \C^2$};
  \node (a2) at (1,0) {$\times$};
  \node (a2b) at (1,-.3) {$\displaystyle\prod_{\theta\in\A^{k_1}} \C$};
  \draw[right hook->] (a1) to[out=20, in=180] (a2a) {};
  \draw[right hook->] (a1) to[out=-20, in=180] (a2b) {};
  \node (a3a) at (2,.3) {$\displaystyle\prod_{\theta\in\A^{k_2}} \SSto_{\theta}^{k_2}(A^0)$};
  \node (a3) at (2,0) {$\times$};
  \node (a3b) at (2,-.3) {$\displaystyle\prod_{\theta\in\A^{k_1}} \SSto_{\theta}^{k_1}(A^0)$};
  \draw[->] (a2a) -- (a3a) {};
  \draw[->] (a2b) -- (a3b) {};
  \node (a3a2) at (3.2,.3) {$\displaystyle\prod_{\theta\in\A} \SSto^{k_2}_{\theta}(A^0)$};
  \node (a32) at (3.2,0) {$\times$};
  \node (a3b2) at (3.2,-.3) {$\displaystyle\prod_{\theta\in\A} \SSto^{k_1}_{\theta}(A^0)$};
  \draw[->] (a3a) -- (a3a2) {};
  \draw[->] (a3b) -- (a3b2) {};
  \node (a4a) at (4.8,.3) {$\displaystyle\prod_{\theta\in\A} \Sto^{k_2}_{\theta}(A^0)$};
  \node (a4) at (4.8,0) {$\times$};
  \node (a4b) at (4.8,-.3) {$\displaystyle\prod_{\theta\in\A} \Sto^{k_1}_{\theta}(A^0)$};
  \draw[->] (a3a2) -- (a4a) node [midway] {$\displaystyle\prod_{\theta\in\A}\rho_{\theta}^{-1}$};
  \draw[->] (a3b2) -- (a4b) node [midway] {$\displaystyle\prod_{\theta\in\A}\rho_{\theta}^{-1}$}; 
  \draw [decorate,decoration={brace,amplitude=10pt,raise=4pt},yshift=0pt]
    (5.2,.6) -- (5.2,-.6) node [black,midway,xshift=4pt] (a5) {};
  \node (end) at (5.5,-1) {$\St(A^0)$};
  \draw[->] (a5) to[out=0,in=90,"$\mathcal{T}\circ\prod_{k\in\cK}i^k$"] (end) node [midway] {};
\end{tikzpicture} %}}}
\end{comment}
