\chapter{Stokes Structures}
\begin{comment}
  Matrizen Sicht:
  \begin{itemize}
    \item \cite[Chapter 3]{boalch}
      \\ largely following:
      \begin{itemize}
        \item[\textbf{8}] D.G. Babbitt and V.S. Varadarajan.
        \textbf{Formal reduction theory of meromorphic differential equations:
          a group theoretic view.}
        \texttt{euclid.pjm.1102720203.pdf}
        \item[\textbf{11}] W. Balser, W.B. Jurkat, and D.A. Lutz.
        \textbf{Birkhoff invariants and Stokes’ multipliers for meromorphic
          linear differential equations.}
        \item[\textbf{40}] M. Jimbo, T. Miwa, and Kimio Ueno.
        \textbf{Monodromy preserving deformations of linear differential
          equations with rational coefficients I.}
        \item[\textbf{43}] M. Loday-Richaud
        \textbf{Stokes phenomenon, multisummability and differential Galois
          groups.}
        \item[\textbf{50}] J. Martinet and J.P. Ramis.
        \textbf{Elementary acceleration and multisummability.}
      \end{itemize}
    \item \cite{thboalch}
    \item \cite{van2003galois}
    \item Marius van der Put, Kyoshi Saito => Diff.Galois Theory
  \end{itemize}
  Moderne Sicht (sheaf-view):
  \begin{itemize}
    \item Malgrange
    \item \cite{sabbah_cimpa90} and
    \item \cite{sabbah2007isomonodromic}
    \item \cite{sabbah2009introduction} and \cite{sabbah2013introduction}
  \end{itemize}
  See also:
  \begin{itemize}
    \item \cite{Loday1994} for \textbf{Sheaf $\to$ Matrix}
      \begin{thm}[II.2.1]
        The map
        \[
          h:\prod_{\alpha\in\A}\Sto_\alpha(A_0)\to H^1(S^1;\Lambda(A_0))
        \]
        is bijective and natural.
      \end{thm}
      \begin{itemize}
        \item \cite{LodayRichaud2004}
        \item \cite{Loday2014}
      \end{itemize}
    \item \cite{Varadarajan96linearmeromorphic} for \textbf{overview of both}
    \item \cite{sibuya1990Linear} looks at $\infty$
    \item \cite{BJL1979Birkhoff} looks at $\infty$
  \end{itemize}
\end{comment}
A great overview of this topic is found in
\cite{Varadarajan96linearmeromorphic}.

\begin{comment}
  \section{Dictionary}
  \begin{center}
    \begin{tabular}{|c|c||c|c||c|c|}
      \cite{sabbah_cimpa90} & \cite{sabbah2007isomonodromic} &
      \cite{boalch}& \cite{thboalch} &
      \cite{Loday1994} & \cite{Loday2014} \tabularnewline
    \hline
    &  & & & $A_0$ & \tabularnewline{}
    &  & $\A$ & $\A$ &
      $\A$, $\A^{\star}$
      \marginnote{\footnotesize$\star\in\{{}^k,{}^{<k},{}^{\leq k}\}$} &
      $\mathfrak{A}$
      \tabularnewline
    &  & $\Sto_\alpha$ & $\Sto_\alpha$ & $\Sto_\alpha$ & \tabularnewline
    &  &  &  & $\Lambda(A_0)$ & \tabularnewline
      &  &  &  & $[{}^F\!A]$ & \tabularnewline
    &  &  &  &  & \tabularnewline
    &  &  &  &  & \tabularnewline
    &  &  &  &  & \tabularnewline
    &  &  &  &  & \tabularnewline
    &  &  &  &  & \tabularnewline
    \end{tabular}
  \end{center}
\end{comment}

\section{Situation}
Let $\cM^{good}$ be a fixed model with the corresponding connection matrix
$A^0$.
\begin{center}
  \begin{tikzpicture}[scale=3]
    \node[green!40!black] (modSpcSheaf) at (0,0.3) {$\sH(\sM^{good})$};
    \node[blue] (modSpcMat) at (0,-0.3) {$\cH(A^0)$};
    \node[purple] (class) at (0.8,0) {$\G\backslash\tilde\G(B_0)$};
    \node[green!40!black] (sheaf) at (3,1.3) {$\St(\sM^{good})$};
    \node[purple] (sheaf2) at (4,1) {$H^1(S^1;\Lambda^{<0}(B_0)$};
    \node[blue] (mat) at (3,-1.3) {$\prod_{d\in\A}\Sto_d(A^0)$};
    \node[blue] (mat') at (3,-2) {$(U_+\times U_-)^{k-1}$};
    \node[purple] (mat2) at (4,-1) {$\prod_{\alpha\in\mathfrak{A}}\Sto_\alpha(B_0)$};

    \draw[thick,double] (modSpcMat) -- (modSpcSheaf) -- (class);
    \draw[thick,double,blue] (modSpcMat) -- (class);
    \draw[thick,double] (sheaf) -- (sheaf2);
    \draw[thick,double] (mat) -- (mat2);

    \draw[->,green!40!black] (modSpcSheaf) -- (sheaf);
    \draw[->,blue,dashed] (modSpcMat) -- (mat);
    \draw[->,blue,dashed] (class) -- (mat);
    \draw[->,blue] (mat) -- (mat');
    \draw[->,purple,dashed] (class) -- (sheaf2);
    \draw[->,purple,dashed] (class) -- (mat2);
    \draw[->,purple] (mat2) -- (sheaf2);
    \draw[->,purple,dashed] (sheaf2) edge[bend right=20] (mat2);

    \node[green!40!black] at (1,1) {\cite{sabbah2007isomonodromic}};
    \node[blue] at (2.3,-1.6) {\cite{thboalch},\cite{boalch}};
    \node[purple] at (4.5,0) {\cite{Loday1994}{\footnotesize(\cite{Loday2014})}};
  \end{tikzpicture}
\end{center}

%%%%%%%%%%%%%%%%%%%%%%%%%%%%%%%%%%%%%%%%%%%%%%%%%%%%%%%%%%%%%%%%%%%%%%%%%%%%%%%
\section{Stokes Structures: Malgrange-Sibuya isomorphism}
\begin{comment}
  \begin{itemize}
    \item \cite{Loday1994} Thm I.2.1
    \item \cite{Loday2014} Thm. 4.3.9, on p. 78
    \item \cite{sabbah2007isomonodromic} Thm II.6.2
  \end{itemize}
\end{comment}
Here we will look at the classifying set and we will proof, that it is
isomorphic \TODO[as\dots] to the first non abelian cohomology set
$H^1(S^1;\Lambda^{<0}(A^0))=:\St(A^0)$.

Let us first define a Stokes sheaf on $S^1$, as the sheaf of flat
isotropies\TODO[\dots]
\begin{defn}
  The Stokes sheaf $\Lambda(A^0)$ of $A^0$, is the sheaf of groups defined on
  $S^1$ whose stalk at any point $\theta\in S^1$ is the group of germs of
  $f\in\Gl_n(\cO(s))$, $\mathfrak{s}$ a sector containing $\theta$, satisfying
  the conditions:
  \begin{enumerate}
    \item Flatness: $\underset{x\in\mathfrak{s}}{\underset{x\to0}{\lim}}f(x)=1$
      and $f\sim_{\mathfrak{s}} 1$;
    \item Isotropy of $A^0$: ${}^f\!A^0=A^0$.
  \end{enumerate}
\end{defn}

Let $(\sM,\nabla,\hat f)$ be defined on\dots

\begin{thm}
  The homomorphism 
  \[
    \sH\to\St(A^0)=H^1(S^1;\Lambda^{<0}(A^0))
  \]
  is an isomorphism of pointed\footnote{which maps \TODO{} to \TODO{}} sets.
\end{thm}
\TODO[\cite{sabbah2007isomonodromic} cor II.6.4.]
\begin{rem}
  \marginnote{\cite{Loday1994} Remark I.2.2}
  To another normal form $A_1={}^f\!A_0$ there correspond cochains which are
  conjugated via $f$.
  We get the following commutative diagram:
  \TODO{}
\end{rem}



%%%%%%%%%%%%%%%%%%%%%%%%%%%%%%%%%%%%%%%%%%%%%%%%%%%%%%%%%%%%%%%%%%%%%%%%%%%%%%%
\section{Stokes Structures: Matrix version}
\begin{comment}
  See
  \begin{itemize}
    \item \cite{Loday1994}
    \item \cite{boalch} and \cite{thboalch}
    \item \cite{babbitt1989local}
  \end{itemize}
\end{comment}

%%%%%%%%%%%%%%%%%%%%%%%%%%%%%%%%%%%%%%%%%%%%%%%%%%%%%%%%%%%%%%%%%%%%%%%%%%%%%%%
\subsection{Definitions}
\TODO[cyclic covering, nerve]
\begin{defn}[cyclic covering]
  A covering $\cU=\{U_j;j\in J\}$ is called \emph{cyclic covering} if
  \begin{enumerate}
    \item the set $J$ is finite and cyclic $J=\Z/\nu\Z$;
    \item the $U_j$ and, if $\#J>2$, the $U_j\cap U_{j+1}$ are connected arcs on
      $S^1$;
    \item the bisecting directions of the $U_j$ are in ascending order with
      respect to the clockwise orientation of $S^1$;
    \item the $U_j$ are not encased, this means that the arcs
      $U_j\backslash U_l$ are connected arcsz for all $j,l\in J$.
  \end{enumerate}
\end{defn}
\begin{defn}[nerve of a cyclic covering]
  The \emph{nerve} of a cyclic covering $\cU=\{U_j;j\in J\}$ is the family
  $\dot\cU=\{\dot U_j;j\in J\}$ defined by:
  \begin{itemize}
    \item $U_j=U_j\cap U_{j+1}$ when $\#J>2$,
    \item $\dot U_1$ and $\dot U_2$ the connected components of $U_1\cap U_2$
      when $\#J=2$.
  \end{itemize}
\end{defn}
The cyclic coverings correspond one-to-one to nerves of cyclic coverings.

%%%%%%%%%%%%%%%%%%%%%%%%%%%%%%%%%%%%%%%%%%%%%%%%%%%%%%%%%%%%%%%%%%%%%%%%%%%%%%%
\subsection{The main theorem}
\marginnote{\cite[868]{Loday1994}}
Let $\{\theta_j\mid j\in J\}\subset S^1$ be a finite set and
$\dot\phi=(\dot\phi_{\theta_j})_{j\in J}\in\prod_{j\in J}\Lambda_{\theta_j}(A^0)$
be a finite family of germs.
In the following way, one can associate a cohomology class to any $\dot\phi$:
let $\dot\phi_j$ be the function representing the germ $\dot\phi_{\theta_j}$ on
its maximal arc of definition $\Omega_j$ around $\theta_j$.
When a cyclic covering $\cU=\{U_j;j\in J\}$ satisfies
$\dot U_j\subset \Omega_j$ for all $j\in J$ one can define the $1$-cocycle
$(\dot\phi_{j|\dot U_j})_{j\in J}$ on $\cU$.
To a differente $\cU$ this construction yields a cohomologous
$1$-coycle\TODO[Proof], thus the resulting map, in the case
$\{\theta_j\mid j\in J\}=\A$,
\[
  h:\prod_{\alpha\in\A}\Sto_\alpha(A^0)\to \St(A^0)=H^1(A^1;\Lambda(A^0))
\]
is welldefined.
\begin{thm}
  The map
  \[
    h:\prod_{\alpha\in\A}\Sto_\alpha(A^0)\to \St(A^0)=H^1(A^1;\Lambda(A^0))
  \]
  is a bijection and natural.
  \begin{rem}
    \marginnote{\cite[869]{Loday1994}}
    Natural means that $h$ commutes to isomorphisms and constructions over
    systems or connections they represent.
    \begin{comment}
      See \cite{Loday1994} Section III.3.3
    \end{comment}
  \end{rem}
\end{thm}
%%%%%%%%%%%%%%%%%%%%%%%%%%%%%%%%%%%%%%%%%%%%%%%%%%%%%%%%%%%%%%%%%%%%%%%%%%%%%%%
\subsection{Proof}
We will only look at the unramified case, the proof in the ramified case can be
found in \cite{Loday1994} section II.4.
%%%%%%%%%%%%%%%%%%%%%%%%%%%%%%%%%%%%%%%%%%%%%%%%%%%%%%%%%%%%%%%%%%%%%%%%%%%%%%%
\section{From cohomology to matrix version}
