\chapter{Stokes Structures}\label{chap:stokes}
Stokes structures are contain exactly \rewrite{necessary information} to
classify meromorphic classification, i.e.\ with the Stokes structures we are
able to construct a \rewrite{space}\PROBLEM[Why space], which is isomorphic to
the classifying \rewrite{set}.

A great overview of this topic is given by Varadarajan in
\cite{Varadarajan96linearmeromorphic}. Other resources we will use are for
example \rewrite{Sabbah's} book~\cite[section II]{sabbah2007isomonodromic} for
Section~\ref{sec:mainThm1}.
For the Sections~\ref{sec:StokesGroup} and~\ref{sec:mainThm2} will
\rewrite{Loday-Richaud's} paper~\cite{Loday1994} and the book
\cite[Sec.4]{Loday2014} be \rewrite{useful. Also useful} was \rewrite{Boalch's}
paper~\cite{boalch} (resp.\ his thesis~\cite{thboalch}) which looks only at the
single leveled case or the paper~\cite[Thm.13]{Martinet1991} from Martinet and
Ramis.

Let $(\cM^{nf},\nabla^{nf})$ be a fixed model with the corresponding normal
form $A^0$.
Let us also fix a normal solution $\cY_0$ of $A^0$.
The purpose of the next section (Section~\ref{sec:mainThm1}) is, to proof the
Malgrange-Sibuya Theorem.
It states that the classifying set $\cH(A^0)$ is via an map $\exp$ isomorphic
to the first non abelian cohomology $H^1(S^1;\Lambda(A^0))$ of the Stokes sheaf
$\Lambda(A^0)$. It will be denoted by $\St(A^0)$.
In Section~\ref{sec:mainThm2} we will improve the Malgrange-Sibuya Theorem by
showing that each 1-cohomology class in $\St(A^0)$ contains a unique
$1$-cocycle of a special form called \emph{the Stokes cocycle}
(cf.\ Section~\ref{sec:StokesGroup}).
The morphism, which maps each Stokes cocycle to its corresponding $1$-cocycle
will be denoted by $h$.
This will be further improved in Section~\ref{sec:furtherImprovements}.

If one introduces the map $g$, which arises from the theory of summation and
takes an equivalence class (resp.\ an ambassador of such a class) and returns a
corresponding Stokes cocycle in an canonically way
(cf.\ Appendix~\ref{app:multisummability} where the theory of summation  will
be roughly discussed), as a black-box one can write the following commutative
diagram.
\begin{center}
  \begin{tikzpicture}[scale=3]
    % \node[] (modSpcMat) at (0,0.4) {$\cH(\cM^{nf},\nabla^{nf})$};
    \node[] (mat) at (0,-.8) {$\prod_{\theta\in\A}\Sto_\theta(A^0)$};
    \node[] (class) at (0,0) {$\cH(A^0)$};
    % \node[green!40!black] (sheaf) at (3,0.4) {$\St(\cM^{nf})$};
    \node[] (sheaf3) at (1.3,0) {$\St(A^0)$};

    % \draw[thick,double,blue] (modSpcMat) -- (class);
    % \draw[purple,thick,double] (sheaf) -- (sheaf3);

    % \draw[->,green!40!black] (modSpcMat) -- (sheaf)
    %   node[midway,above] {$\exp$};
    \draw[->] (class) -- (mat) node[midway,left] {g};
    \draw[->] (class) -- (sheaf3) node[midway,above] {$\exp$};
    \draw[->] (mat) -- (sheaf3) node[midway, below right] {$h$};
  \end{tikzpicture}
\end{center}\label{page:ofPreDiagram}
This diagram will be enhanced in Section~\ref{sec:theCompleteDiagram} by
adding a couple of isomorphisms.

\ifnum\myDevelopVariable=0
%%%%%%%%%%%%%%%%%%%%%%%%%%%%%%%%%%%%%%%%%%%%%%%%%%%%%%%%%%%%%%%%%%%%%%%%%%%%%%%
\section{Stokes structures: Malgrange-Sibuya isomorphism}\label{sec:mainThm1}
\marginnote{\cite[Thm.I.2.1]{Loday1994},
  \\\cite[Thm.4.3.9]{Loday2014},
  \\\cite[Thm.II.6.2]{sabbah2007isomonodromic}}
Here we will look at the classifying set and we will proof that it is
isomorphic \TODO[as\dots] to the first non abelian cohomology \rewrite{set}
$H^1(S^1;\Lambda(A^0))$, which will be denoted as $\St(A^0)$.
If we talk about cocycles or cochains, we will in the following always mean
$1$-cocycles or $1$-cochains.

Let us first define the Stokes sheaf $\Lambda(A^0)$ on $S^1$, as the sheaf of
flat isotropies of $[A^0]$.
\begin{defn}\label{defn:StokesSheaf}
  \begin{comment}
    The Stokes sheaf $\Lambda(A^0)$ of $A^0$, is the sheaf of groups defined on
    $S^1$ whose stalk at any point $\theta\in S^1$ is the group of germs of
    $f\in\Gl_n(\cO(\mathfrak{s}))$, $\mathfrak{s}$ a sector containing
    $\theta$, satisfying the conditions:
    \begin{enumerate}
      \item Flatness:
        $\underset{x\in\mathfrak{s}}{\underset{x\to0}{\lim}}f(x)=1$ and
        $f\sim_{\mathfrak{s}} 1$;
        \PROBLEM[Why two condition?]
      \item Isotropy of $A^0$: ${}^f\!A^0=A^0$.
    \end{enumerate}
  \end{comment}
  The Stokes sheaf $\Lambda(A^0)$ of $A^0$, is defined as the subsheaf of
  $\GL_n(\cA)$, in the following way.
  For some $\theta\in S^1$ is the stalk at $\theta$ the subgroup of
  $\GL_n(\cA)_\theta$ of elements $f$ which satisfy
  \begin{enumerate}
    \item Multiplicatively flatness: $f$ is asymptotic to the identity,
      i.e.\ $f\sim_{\mathfrak{s}} 1$;
    \item Isotropy of $A^0$: ${}^f\!A^0=A^0$.
  \end{enumerate}
  \begin{s-rem}
    This definition \rewrite{makes also sense as} $\Lambda(A)$ where $A$ stands
    for a systems wich is not in normal form. The elements of $\Lambda(A)$ then
    have to be isotropies of a normal form $A^0$ of $A$.
  \end{s-rem}
  \begin{comment}
    \begin{s-lem}
      This is independent of the choice of the normal form.
    \end{s-lem}
  \end{comment}
  \iffalse
  \begin{comment}
    \begin{s-rem}
      \PROBLEM[remove? need more defs!]
      Sabbah \cite[110]{sabbah2007isomonodromic} talks about (global)
      meromorphic connections $\sM$ on a small disk $D$ around $0$ instead of
      germs of meromorphic connections.

      Define on $S^1$ the sheaf $\Aut^{<0}(\tilde\sM^{nf})$ of automorphisms of
      $\tilde\sM^{nf}:=\cA_D\otimes_{\cO_D}\sM^{nf}$ which
      \begin{itemize}
        \item are compatible with the connection and
        \item are formally equal to the identity, i.e.\ induce the identity on
          $\hat\sM^{nf}:=\hat\cO_D\otimes_{\cO_D}\sM^{nf}$
      \end{itemize}

      The sheaf $\Aut^{<0}(\tilde\sM^{nf})$ corresponds to our $\Lambda(A^0)$.
    \end{s-rem}
  \end{comment}
  \fi
\end{defn}

\subsection{The theorem}
\rewrite{Here we want to state the Malgrange-Sibuya Theorem. We will first give
it in the language of meromorphic connections and after that we will give the
same theorem in the language of systems. The second variant of this theorem
will be proven in the next section.}

In the language of meromorphic connections is the map, in the Malgrange-Sibuya
Theorem bellow, described as follows.

Let $(\cM,\nabla,\hat f)$ be a marked germ of a  meromorphic connection.
\rewrite{By Theorem}~\ref{thm:sectorialDecompFromMAET} there exists an open
covering $\cU=(U_j)_{j\in J}$ and for every open set, an isomorphism
\[
  f_j:(\tilde\cM,\tilde\nabla)_{|U_j}
  \to(\tilde\cM^{nf},\tilde\nabla^{nf})_{|U_j}
\]
such that $f_j\sim_{U_j}\hat f$.
By $(f_kf_j^{-1})_{jk}$ is then a cocycle of the sheaf $\St(A^0)$, relative
to the covering $\cU$, defined.
\comm{For other lifts $f_j'$ of $\hat f$ on $W_j$, $(f_j'f_j^{-1})$ is a
  $0$-cochain of $\Sto(A^0)$ relative to $\cU$. Thus the associated cochians to
  $(f_j)$ and $(f_j')$ are equivalent. One can also check that, if
  $(\cM,\nabla,\hat f)$ and $(\cM',\nabla',\hat f')$ are isomorphic, the
  corresponding cocycles define the same cohomology class.}
This defines a mapping of pointed sets
\[
  \exp:\cH(\cM^{nf},\nabla^{nf})\longrightarrow H^1(S^1;\Lambda(A^0))
\]
to the first non abelian cohomology of $\Lambda(A^0)$, which sends the class of
$(\cM^{nf},\nabla^{nf},\hat\id)$ to that of $\id$, i.e.\ the trivial cohomology
class.

\begin{center}
  \begin{minipage}[t]{0.8\textwidth}
    \begin{tthm}[Malgrange-Sibuya] \label{thm:mainThm1MeromVersion}
      \marginnote{\cite[Thm.I.4.5.1]{babbitt1989local},
        \\\cite[Thm.3.4]{Malgrange1983},
        \\\cite[Thm.13]{Martinet1991}}
      The homomorphism
      \[ \begin{tikzcd}
          \exp:\cH(\cM^{nf},\nabla^{nf}) \rar& \St(A^0):=H^1(S^1;\Lambda(A^0))
      \end{tikzcd} \]
      is an isomorphism of pointed sets.
    \end{tthm}
  \end{minipage}
\end{center}

\subsubsection{The theorem (system version)}
Since the language of meromorphic connections is equivalent to the one of
systems, there is also the translated version of the Malgrange-Sibuya
isomorphism to the language of systems. The corresponding map is then build as
follows.

\marginnote{\cite[855]{Loday1994}}
Let $(A,\hat F)$ be a marked pair, thus $\hat F$ solves $[A^0,A]$.
By the M.A.E.T (Theorem~\ref{thm:maet}) there exists an open covering
$\cU=(U_j)_{j\in J}$ together with, for every open set $U_j$ a lift
$F_j\in\Gl_n(\cA(U_j))$ of $\hat F$ (cf.\ Definition~\ref{defn:lift}), which
solves $[A,A^0]$.
By the cocycle $(F_l^{-1}F_j)_{jl}\in\Gamma(\dot\cU;\Lambda(A^0))$ is then a
cohomology class in $\St(A^0)$, relative to the covering $\cU$, determined.
For other lifts $F_j'$ of $\hat F$ on $U_j$ is $(G_j=F_j^{-1}F_j')$ a
$0$-cochain of $\Lambda(A^0)$ relative to $\cU$, which satisfies
\[
  F_k^{-1}F_j=G_k F_k'^{-1}F'_j G_j^{-1} \,.
\]
Thus the cochians associated to $(F_j)$ and $(F_j')$ determine the same
cohomology class in $\St(A^0)$.
One can also check that, if $(A,\hat F)$ and $(A',\hat F')$ are equivalent, the
corresponding cocycles define the same cohomology class.
This defines a welldefined mapping of pointed sets
\[
  \cH(A^0)\to H^1(S^1;\Lambda(A^0))
\]
to the first non abelian cohomology of $\Lambda(A^0)$, which we call $\exp$.
It maps the class of $(A^0,\hat\id)$ to that of $\id$, i.e.\ the trivial
cohomology class.

\begin{center}
  \begin{minipage}[t]{0.8\textwidth}
    \begin{tthm}[Malgrange-Sibuya \rewrite{(system version)}]\label{thm:mainThm1}
      \marginnote{\cite[Theorem 4.5.1]{babbitt1989local}}
      The homomorphism
      \[ \begin{tikzcd}
          \exp:\cH(A^0) \rar& \St(A^0):=H^1(S^1;\Lambda(A^0))
      \end{tikzcd} \]
      is an isomorphism of pointed sets.
    \end{tthm}
  \end{minipage}
\end{center}
\begin{rem}
  % \TODO[Frage Felix!]
  The Theorem \cite[Thm.III.1.1.2]{babbitt1989local}, in the book from
  Babbitt and Varadarajan, states that $\St(A^0)$ is actually a local moduli
  space for marked pairs, which are formally isomorphic to a given system
  $[A^0]$.
  \begin{comment}
    This means that
    \begin{itemize}
      \item the morphism property,
      \item the criterion of equivalence and
      \item the existence of universal families
    \end{itemize}
    are satisfied (cf.\ \cite[169]{babbitt1989local}).
  \end{comment}
  In fact is the whole third part of \cite{babbitt1989local} dedicated to this
  topic.
\end{rem}
Since the morphism $\exp$ depends on the choice of the normal form, we will
denote that, if it is not clear, by $\exp_{A^0}=\exp$\comm{~(In Loday-Richaud's
paper is this denoted as $\exp_{\mu_0}$)}.
\begin{rem}\label{rem:expNonNormalForm}
  \marginnote{\cite{Loday1994} Remark I.2.2}
  To another normal form $A^1={}^\Phi\!A^0$ there correspond cochains, which
  are conjugated via $\Phi\in G(\!\{t\}\!)$.
  We especially get the following commutative diagram:
  \begin{center}
    \begin{tikzpicture}[scale=2.6]
      \node (tl) at (-1,0.6) {$G\backslash\hat G(A^1)$};
      \node (tr) at (1,0.6) {$G\backslash\hat G(A^0)$};
      \node (bl) at (-1,-0.6) {$H^1(S^1;\Lambda(A^1))$};
      \node (br) at (1,-0.6) {$H^1(S^1;\Lambda(A^0))$};
      \node (tlin) at ($ (tl) - (-0.3,0.2) $)
        {\rotatebox[origin=c]{135}{$\in$}};
      \node (trin) at ($ (tr) - (-0.3,0.2) $)
        {\rotatebox[origin=c]{135}{$\in$}};
      \node (blin) at ($ (bl) - (-0.2,0.2) $)
        {\rotatebox[origin=c]{135}{$\in$}};
      \node (brin) at ($ (br) - (-0.2,0.2) $)
        {\rotatebox[origin=c]{135}{$\in$}};
      \node (tlE) at ($ (tl) - (-0.5,0.4) $) {$\hat F$};
      \node (trE) at ($ (tr) - (-0.5,0.4) $) {$\hat F\Phi$};
      \node (blE) at ($ (bl) - (-0.5,0.4) $) {$\exp_{A^1}(\hat F)$};
      \node (brE) at ($ (br) - (-0.5,0.4) $) {$\exp_{A^0}(\hat F\Phi)$};
      \draw[->] (tl) -- (tr) node [midway,above] {$\cdot\Phi$};
      \draw[->] (tl) -- (bl) node [midway,right] {$\exp_{A^1}$};
      \draw[->] (tr) -- (br) node [midway,right] {$\exp_{A^0}$};
      \draw[->] (bl) -- (br);
      \draw[|->] (tlE) -- (trE);
      \draw[|->] (tlE) -- (blE);
      \draw[|->] (trE) -- (brE);
      \draw[|->] (blE) -- (brE);
    \end{tikzpicture}
  \end{center}
  where $\exp_{A^0}(\hat F\Phi)=\Phi^{-1}\exp_{A^0}(\hat F)\Phi$.
\end{rem}

\subsection{Proof of Theorem~\ref{thm:mainThm1}}
We will mainly refer to \cite[Proof of Theorem 4.5.1]{babbitt1989local} and
\cite[Section 6.d]{sabbah2007isomonodromic}, where a slightly more complicated
case with deformation space is proven. These both resources proof the theorem
using the languages of meromorphic connections whereas we will use systems.
\marginnote{See also \cite{BJL1979Birkhoff} and \cite{babbitt1989local}
  although the proof goes back to work from Malgrange and Sibuya (see for
  example \cite{sibuya1990Linear}).}

We will start by proofing the injectivity of the morphism $\exp$.
\begin{proof}[Proof of the injectivity]
  % \textbf{First look at injectivity:}
  Consider the two marked pairs $(A,\hat F)$ and $(A',\hat F')$ in
  $\hat\Syst_m(A^0)$, whose classes in $\cH(A^0)$ get mapped to same element
  \[
    \exp([(A,\hat F)])=\lambda=\exp([(A',\hat F')])
      \in H^1(S^1;\Lambda(A^0)) \,.
  \]
  By using refined coverings, it is possible to find a common finite covering
  $\cU=\{U_j;j\in J\}$ of $S^1$ such that $\lambda$ is the class of the
  cocycles $(F_l^{-1}F_j)$ and $(F_l'^{-1}F_j')$, where $F_j$ (resp.\ $F_j'$)
  are lifts of $\hat F$ (resp.\ $\hat F'$) on $U_j\in\cU$.
  From $[(F_l^{-1}F_j)]=[(F_l'^{-1}F_j')]$ follows that there exists a
  $0$-cochain $(G_j)_{j\in J}$ of the sheaf $\Lambda(A^0)$ relative to the
  covering $\cU$, such that
  \[
    F_l'^{-1}F_j'=G_lF_l^{-1}F_jG_j^{-1}
    \text{~on~the~arc~} U_j\cap U_l \,,
  \]
  which can be rewritten to
  \begin{equation}\label{eq:inProofOfTHM1Glue}
    F_j'G_jF_j^{-1} = F_l'G_lF_l^{-1} \text{~on~the~arc~} U_j\cap U_l \,.
  \end{equation}
  If we set $H_j:=F_j'G_{j}F_j^{-1}$ on $U_{j}$, we get
  \begin{itemize}
    \item that from equation (\ref{eq:inProofOfTHM1Glue}) that the $H_j$ glue
      together and yield a meromorphic $H$\PROBLEM[holomorphic??],
    \item that $H_j$ is a solution of $[A,A']$ on every $U_j$, i.e.\ it
      satisfies there ${}^{H_j}A=A'$, since
      \begin{align*}
        {}^{H_j}A &= {}^{F_j'G_{j}F_j^{-1}}A
        \\&={}^{F_j'G_{j}}A^0
                  & \text{(since $F_j'$ is a lift of $\hat F'$ on $U_j$)}
        \\&={}^{F_j'}A^0
                  & \text{(since $G_j$ is a is an isotropy of $A^0$)}
        \\&=A'
                  & \text{(since $F_j$ is a lift of $\hat F$ on $U_j$)}
      \end{align*}
      and
    \item which satisfies $\hat F'=\hat H_j\hat F$ on every $U_j$, since
      \begin{align*}
        \hat H_j\hat F&= \widehat{F_j'G_{j}F_j^{-1}} \hat F
        \\            &= \hat{F_j'}
        \underset{\id}{%
          \underset{\text{\rotatebox[origin=c]{-90}{$=$}}}{%
            \underbrace{%
              \hat{G_j}
            }
          }
        }
        \hat{F_j^{-1}} \hat F
                      & \text{(since $G_j$ is flat, i.e.\ $\hat G_j=\id$)}
        \\            &= \hat{F'}
        \underset{\id}{%
          \underset{\text{\rotatebox[origin=c]{-90}{$=$}}}{%
            \underbrace{%
              \hat{F^{-1}} \hat F
            }
          }
        }
        \\            &= \hat{F'}
      \end{align*}
  \end{itemize}
  Therefore are $(A,\hat F)$ and $(A,\hat F')$ equivalent
  (cf.\ page~\pageref{page:ofDefnOfIsomOfMarkedPairs}) and injectivity is
  proven.
  \iffalse
    \begin{comment}
      \textbf{First look at injectivity:}
      Consider the two elements $(\cM,\nabla,\hat f)$ and
      $(\cM',\nabla',\hat f')$ of $\cH(\cM^{nf},\nabla^{nf})$ which map to same
      cohomology class
      \[
        \exp([(\cM,\nabla,\hat f)])=\lambda=\exp([(\cM',\nabla',\hat f')])
          \in H^1(S^1;\Lambda(A^0)) \,.
      \]
      Since we can use refined coverings, it is possible to find a finite
      covering $\cU=\{U_j;j\in J\}$ of $S^1$ such that $\lambda$ is the class
      of the cocycles $(f_lf_j^{-1})$ and $(f_l',f_j'^{-1})$, where
      $f_j$,$f_j'$ are defined on $U_j$.
      Since $[(f_lf_j^{-1})]=[(f_l'f_j'^{-1})]$ there exists a $0$-cochain
      $(g_j)$ of the sheaf $\Aut^{<0}(\tilde\cM^{nf})$ relative to the covering
      $(I_j)$, such that
      \[
        f_l'f_j'^{-1}=g_lf_lf_j^{-1}g_j^{-1} \text{ on } I_j\cap I_l.
      \]
      If we set $\sigma=f_j^{-1}g_{j}^{-1}f_j'$ on $I_{j}$, we get a horizontal
      section\TODO[~on~???], thus\TODO[why?] it satisfies
      $\sigma\circ\hat{f'}=\hat f$. Therefore are $(\cM,\nabla,\hat f)$ and
      $(\cM',\nabla',\hat{f'})$ isomorphic and injectivity is proven.
    \end{comment}
  \fi
\end{proof}

For the proof of the surjectivity we will use another result from Malgrange and
Sibuya, which is also called the Malgrange-Sibuya Theorem
(Theorem~\ref{thm:thm1helpMalgSibuy}). It can for example be found in Babbitt
and Varadarajans's book \cite[65ff]{babbitt1989local} as Theorem 4.2.1.

Let $\hat F\in G(\!(t)\!)$ be a matrix with formally meromorphic entries. By
the Borel-Ritt Lemma (cf.\ Theorem~\ref{thm:borel-ritt}) we then know, that
there exists for every sector $\mathfrak{s}\subsetneq S^1$ a holomorphic
function $G:\mathfrak{s}\to\GL_n(\C)$ which is asymptotic to $\hat F$.
We will denote the set of all such holomorphic functions, which are on
the arc $I$ asymptotic to $\id\in G(\!(t)\!)$ by
\[
  \cG(I)=\left\{G\in\Gl_n(\cA(I))\mid g\sim_I\id\right\}\,,
\]
and this defines a sheaf $\cG$ on $S^1$.
The statement of the (second) Malgrange-Sibuya Theorem
(Theorem~\ref{thm:thm1helpMalgSibuy}) is then, that the \rewrite{difference}
between formal and konvergent invertible matrices is described by the first
sheaf cohomology $H^1(S^1;\cG)$ of $\cG$ via the map
\[
  \Theta: G(\!(t)\!)/G(\!\{t\}\!)\longrightarrow H^1(S^1;\cG) \,,
\]
which will turn out to be an isomorphism. It is set up as follows:
\begin{einr}
  Let $[\hat F]\in G(\!(t)\!)/G(\!\{t\}\!)$ with ambassador $\hat F$ and
  let $\cU=\{U_j\mid j\in J\}$ be a finite covering of $S^1$ by open arcs.
  The Borel-Ritt Lemma yields for every arc $j\in J\subsetneq S^1$ a
  holomorphic function $F_j$ which satisfies $F_j\sim_{U_j}\hat F$.
  By $(F_lF_j^{-1})_{j,l\in J}$ is then a cocycle for $\cG$ defined,
  and write $\Theta([\hat F])$ for the corresponding cohomology class.
\end{einr}
This construction is similar to the definition of the map of
Theorem~\ref{thm:mainThm1}. The difference is, that we instead of M.A.E.D, to
obtain lifts in the sense of Definition~\ref{defn:lift}, we use only the
Borel-Ritt Lemma to obtain only asymptotic lifts.

It can be verified, that the class $\Theta([\hat F])$ does not depend on
\begin{itemize}
  \item the choise of an ambassador $\hat F$ in
    $[\hat F]\in G(\!(t)\!)/G(\!\{t\}\!)$\PROBLEM[proof!],
  \item the choice of the covering $\cU$\PROBLEM[proof!] nor
  \item the choice the $F_j$\PROBLEM[proof!].
\end{itemize}
\begin{lem}
  \TODO[remove this lemma? is not needed/used]
  The mapping $\Theta$ is injective.
\end{lem}
\begin{proof}
  Let $\hat F$ and $\hat F'\in G(\!(t)\!)$ such that
  $\Theta([\hat F])=\Theta([\hat F'])$.
  We then can find a covering $\cU=\{U_j\mid j\in J\}$ together with
  holomorphic functions $F_j$ and $F_j'$, which satisfy
  $F_j\sim_{U_j}\hat F$ and $F_j'\sim_{U_j}\hat F'$, such that
  $(F_l^{-1}F_j)_{j,l\in J}$ and $(F_l'^{-1}F_j')_{j,l\in J}$ determine the
  classes $\Theta([\hat F])$ and $\Theta([\hat F'])$.
  This implies that there are maps $G_j$, which are on $U_j$ holomorphic and
  satisfy $G_j\sim_{U_j}\id$ such that
  \[
    F_l'^{-1}F_j'=G_lF_l^{-1}F_jG_j^{-1}
    \text{~on~the~arc~} U_j\cap U_l
  \]
  This equation can be rewritten to
  \[
    F_j'G_jF_j^{-1}=F_l'G_lF_l^{-1}
    \text{~on~the~arc~} U_j\cap U_l \,.
  \]
  Since this tells us, that the functions $F_j'G_jF_j^{-1}$ coincide on
  the overlapping and define a holomorphic map from the arc $S^1$ (i.e.\ a
  punctured disc with a small radius) into $\gl_n(\C)$, which will be called
  $G$.
  Since $F_j'G_jF_j^{-1}\sim_{U_j}\id$ for all $j\in J$, we have
  $G\sim_{S^1}\id$.
  Thus the defined $G$ meromorphic at $0$ and satisfies $G=F'^{-1}F$, so that
  $[F]=[F']$.
\end{proof}
\begin{thm}[Malgrange-Sibuya]\label{thm:thm1helpMalgSibuy}
  The map $\Theta:G(\!(t)\!)/G(\!\{t\}\!)\to H^1(S^1;\cG)$ is an isomorphism.
\end{thm}
This Theorem is proven in Section 4.4 of Babbitt and Varadarajan's book
\cite{babbitt1989local} or on page 371 of \cite{Martinet1991}.

We are now able to proof the surjectivity of the map from
Theorem~\ref{thm:mainThm1}.
\begin{proof}[Proof of surjectivity]
  \marginnote{\cite[72]{babbitt1989local}}
  Let the cohomology class $\lambda\in H^1(S^1;\Lambda(A^0))$ be represented by
  a cocycle $(F_{jl})_{j,l\in J}$ associated with some finite covering
  $\cU=\{U_j;j\in J\}$ of $S^1$. We especially know, that
  \begin{itemize}
    \item $F_{jl}$ is on $U_j\cap U_l$ asymptotic to $\id$ and
    \item it is a isotropy, i.e.\  ${}^{F_{jl}}A^0=A^0$.
  \end{itemize}
  The cocycle $(F_{jl})_{j,l\in J}$ also determines an element in
  $\sigma\in H^1(S^1;\cG)$.
  From the Theorem~\ref{thm:thm1helpMalgSibuy} we know, that there is a $\hat
  F\in G\llbracket t\rrbracket\subset G(\!(t)\!)$ whose class $[\hat F]$ gets
  via $\Theta$ mapped to $\sigma$.
  Thus there exists holomorphic functions $F_j:\mathfrak{s}_{U_j}\to\gl_n(\C)$
  with $F_j\sim_{U_j}\hat F$ and $F_l^{-1}F_j=F_{jl}$ on
  $\mathfrak{s}_{U_j\cap U_l}$ for all $j,l\in J$.

  Define on every arc $U_j$ the matrix $A_j:={}^{F_j}A^0$.
  On the intersections $U_j\cap U_l$ we know that $A_j=A_l$, since
  from $F_l^{-1}F_j\in\Lambda(A^0)$ follows on $U_j\cap U_l$ that
  \[
    A^0={}^{F_l^{-1}}({}^{F_j}A^0)
    \qquad\Longrightarrow{}\qquad
    \underset{=A_l}{\underbrace{{}^{F_l}A^0}}
    =\underset{=A_j}{\underbrace{{}^{F_j}A^0}}
    \,.
  \]
  Thus the $A_j$ \rewrite{glue to a} section $A$, which satisfies
  ${}^{\hat F}A=A_0$ by construction.
  We have found an element $(A,\hat F)\in\cH(A^0)$ whose image under $\exp$ is
  $\sigma$.
\end{proof}

\iffalse
\begin{comment}
  \TODO[Define $\tilde\cM$ or find other way!]
  \begin{proof}[Proof of surjectivity]
    \begin{multicols}{2}
      % \textbf{Now look at surjectivity:}
      Similar to the proof, given by Sabbah \marginnote{Which refers to
      \cite{Malgrange1983}} in \cite{sabbah2007isomonodromic}, we will start this
      part of the proof by giving a necessary and sufficient condition for a
      class $\lambda$ in $H^1(S^1;\Lambda(A^0))$
      \TODO[$\Lambda(A^0)\sim\Aut^{<0}(\tilde\cM^{nf})$ or $\Aut^{<0}(\cM^{nf})$?]
      to come from an object
      $(\cM,\nabla,\hat f)\in\cH(\cM^{nf},\nabla^{nf})$:
      \begin{einr}
        This is the case if and only if the image of
        $\lambda\in\Aut^{<0}(\cM^{nf})$ in the set
        $H^1(S^1;\Aut_{\cA}(\tilde\cM^{nf}))$ (where $\Aut_{\cA}(\tilde\cM^{nf})$
        contains only the $\cA$-linear automorphisms) is the identity.
      \end{einr}

    \columnbreak

      \textcolor{gray}{%
        \textbf{Now look at surjectivity:}
        Similar to the proof, given by Sabbah \marginnote{Which refers to
        \cite{Malgrange1983}} in \cite{sabbah2007isomonodromic}, we will start
        this part of the proof by giving a necessary and sufficient condition
        for a class $\lambda$ in $H^1(S^1;\Lambda(A^0))$
        to come from an object
        $(A,\hat F)\in\cH(A^0)$:
        \begin{einr}
          \rewrite{This is the case if and only if the image of
            $\lambda\in\Aut^{<0}(\cM^{nf})$ in the set
            $H^1(S^1;\Aut_{\cA}(\tilde\cM^{nf}))$ (where
            $\Aut_{\cA}(\tilde\cM^{nf})$ contains only the $\cA$-linear
            automorphisms) is the identity.}
        \end{einr}
      }
    \end{multicols}
    \begin{proof}
      \textbf{``\Rightarrow{}'':}
      If $\lambda$ is the image of some $(\cM,\nabla,\hat f)$ then there exists
      \begin{itemize}
        \item a covering $(I_{j})$ of $S^1$ and
        \item isomorphisms $f_j:\tilde\cM\overset{\sim}\to\tilde\cM^{nf}$
          inducing $\hat f$
      \end{itemize}
      such that $\lambda$ comes from a cocycle $(\lambda_{j,l})=(f_lf_j^{-1})$
      on $I_j\cap I_l$.
      \TODO{} shows that $(\lambda_{jl})$ is a coboundary of
      $\Aut_\cA(\tilde\cM^{nf})$.

      \textbf{``\Leftarrow{}'':}
      If for some suitable covering $(I_j)$ the cocycle $(\lambda_{jl})$ is a
      coboundary with values in $\Aut_\cA(\tilde\cM^{nf})$, i.e.\
      $\lambda_{jl}=f_lf_j^{-1}$, we define a new connection $\nabla$ on
      $\tilde\cM^{nf}$ by conjugating $\nabla^{nf}$ by $f_j$ on $U_j$.

      \TODO{}

      Moreover $\hat f_j=\hat f_l$ on $U_j\cap U_l$, so that the formal
      isomorphisms
      \[
        \hat f_j:(\hat \cM^{nf},\nabla)
        \overset{\sim}{\longrightarrow}
        (\hat\cM^{nf},\nabla^{nf})
      \]
      can be glued in an isomorphism $\hat f:(\hat \cM^{nf},\nabla)
      \overset{\sim}{\longrightarrow}(\hat\cM^{nf},\nabla^{nf})$.
    \end{proof}

    Thus, the proof of Theorem~\ref{thm:mainThm1} is a consequence of the
    following Theorem by Malgrange and Sibuya.
    \begin{thm}[Malgrange-Sibuya]\label{thm:malgSibuyaHelp}
      \marginnote{\cite[Thm.II.6.10]{sabbah2007isomonodromic}}
      The image of the mapping
      \[
        H^1(S^1;\Gl_d^{<0}(\cA_{\tilde D}))
        \to
        H^1(S^1;\Gl_d(\cA_{\tilde D}))
      \]
      is the identity.
    \end{thm}
    For the proof of Theorem~\ref{thm:malgSibuyaHelp} which we refer to
    \cite[Th.A.1]{Malgrange1983}, \cite[Th.6.4.1]{sibuya1990Linear} and
    \cite{babbitt1989local}.
  \end{proof}
\end{comment}
\fi

\fi

\ifnum\myDevelopVariable=0
%%%%%%%%%%%%%%%%%%%%%%%%%%%%%%%%%%%%%%%%%%%%%%%%%%%%%%%%%%%%%%%%%%%%%%%%%%%%%%%
\section{The Stokes groups}\label{sec:StokesGroup}
\PROBLEM[rewrite!]
Here we want to introduce the notion of Stokes groups. They are for example
also introduced in \cite{Loday1994}, \cite{Loday2014} or section 4
of~\cite{Martinet1991}.

Let us recall, that the normal form $A^0$ can be written as
$A^0=Q'(t^{-1})+L\frac{1}{t}$ and a normal solution is given by
$\cY_0(t)=t^Le^{Q(t^{-1})}$ (cf.\ Proposition~\ref{prop:fundSolBuilder}), where
$Q(t^{-1})=\bigoplus_{j\in\{1,\dots,s\}}q_j(t^{-1})\cdot\id_{n_j}$ and the
block structure of $L$ is finer then the structure of $Q$
(cf.\ Definition~\ref{defn:structureComparison}).
Let $\left\{q_1(t^{-1}),\dots,q_s(t^{-1})\right\}$ be the \emph{set of all
determining polynomials of $[A^0]$} and denote by
\[
  \cQ(A^0):=\left\{\left(q_j-q_l\right)(t^{-1})
    \mid
    \text{$q_j$ and $q_l$ determining polynomials of $[A^0]$, }
    q_j \neq q_l
  \right\}
\]
the \emph{set of all determining polynomials of $[\End A^0]$}.
Instead of $q_j-q_l\in\cQ(A^0)$ we will sometimes talk of (ordered) pairs
$(q_j,q_l)\in\cQ(A^0)$.

\begin{defn}\label{defn:determiningPolysOfEndA}
  We call
  \begin{itemize}
    \item $a_{jl}\in\C\backslash\{0\}$ the \emph{leading factor},
    \item $\frac{a_{jl}}{t^{k_{jl}}}$ the \emph{leading
      coefficient} and
    \item $k_{jl}\in\Q$ the \emph{degree}
  \end{itemize}
  of $q_j-q_l\in\cQ(A^0)$ if
  \[
    q_j-q_l\in\left\{\frac{a_{jl}}{t^{k_{jl}}}+h \mid h \in o(t^{-k_{jl}}),
      a_{jl}\neq0
    \right\}\,.
  \]
  \begin{s-rem}
    \begin{enumerate}
      \item It is obvious that $k_{jl}=k_{lj}$ and
        $\frac{a_{jl}}{t^{k_{jl}}}=\frac{-a_{lj}}{t^{k_{lj}}}$.
      \item In Boalch's paper \cite{boalch} (and also in \cite{thboalch}) are
        the \rewrite{degrees of the pairs always incremented by one}.
        We will prefer the \rewrite{other notion}, which is also used in
        Loday-Richaud's paper \cite{Loday1994}.
      \item In Loday-Richaud's book \cite[Def.4.3.6]{Loday2014} $a_{jl}$ is
        negated to be consistend with calculations at $\infty$.
        Here this is not necessary, since we use the clockwise orientation on
        $S^1$ (cf.\ Definition~\ref{defn:antiStokesDir}).
    \end{enumerate}
  \end{s-rem}
  The degrees of the elements in $\cQ(A^0)$ are defined to be  the
  \emph{levels} of $A^0$.
  The set of all levels of $A^0$ will be denoted by
  \[
    \cK=\{k_1<\dots<k_r\} \subset \Q \,.
  \]
  \begin{s-rem}
    The system $[A^0]$ is unramified if and only if $\cK\subset\Z$.
    Since we only want to consider the unramified case, this will be always the
    case.
  \end{s-rem}
\end{defn}

%%%%%%%%%%%%%%%%%%%%%%%%%%%%%%%%%%%%%%%%%%%%%%%%%%%%%%%%%%%%%%%%%%%%%%%%%%%%%%%
\subsection{Anti-Stokes directions and the Stokes group}
\marginnote{\cite[I.4]{Loday1994}}
\begin{defn}
  \marginnote{\cite[130]{hotta2008}, \cite[79]{Loday2014}}
  Let $k\in\N$ and $a\in\C$.
  We say that an exponential $e^{q(t^{-1})}$, where
  $q(t^{-1})\in\frac{a}{t^{k}}+o(t^{-k})$, has \emph{maximal decay in a
  direction $\theta\in S^1$} if and only if $ae^{-ik\tilde\theta}$ is real
  negative. \rewrite{We say that an matrix has maximal decay, if every entry
  has maximal decay.}
  \begin{comment}
    $be^0\in\C$ corresponding to has maximal decay if and only if \PROBLEM[$1$
    has maximal decay?]
  \end{comment}
\end{defn}

On the determining polynomials of $[A^0]$ we define the following (partial)
order relations:
\begin{defn}\label{defn:definingRelations}
  Let $\tilde\theta$ be a determination of $\theta$.
  \begin{itemize}
    \item We define the relation
      $\boldmath q_j \underset{\tilde\theta}{\prec} q_l$ to be equivalent to
      the condition
      \begin{einr}
        \rewrite{$e^{(q_j-g_l)(t^{-1})}$ is flat at $0$} in a neighbourhood of
        the direction $\tilde\theta$.\PROBLEM[Other flat??]
      \end{einr}
    \item Let us define another relation $\boldmath q_j\myrel{\tilde\theta}q_l$
      equivalent to
      \begin{einr}
        $e^{(q_j-g_l)(t^{-1})}$ is of maximal decay in the direction
        $\tilde\theta$.
      \end{einr}
      \TODO[maybe replace by $\prec\!\!\!\prec$]
  \end{itemize}
  \begin{s-rem}
    In the unramified case do these relations not depend on the determination
    $\tilde\theta$ of $\theta$. As a consequence we will only write
    $\underset{\theta}{\prec}$ and $\myrel{\theta}$.
  \end{s-rem}
\end{defn}
We know that
\begin{enumerate}
  \item the condition $q_j \underset{\theta}{\prec} q_l$ is
    satisfied if and only if $\Re(a_{jl}e^{-ik_{jl}\theta})<0$ and
  \item the condition $q_j\myrel{\theta}q_l$ is
    equivalent to
    \begin{einr}
      $a_{jl}e^{-ik_{jl}\theta}$ is a real negative
      number, i.e.\ $q_j \underset{\theta}{\prec} q_l$ and
      $\Im(a_{jl}e^{-ik_{jl}\theta})=0$.
    \end{einr}
\end{enumerate}
Thus it is convenient to look closer at functions of the form
$f:\theta\mapsto ae^{-ik\theta}$, $k\in\Z$, corresponding to some pair
$(q_j,q_l)$.
Write $a$ as $a=|a|e^{i\arg(a)}$, thus the function writes as
\begin{align*}
  f(\theta)&=|a|e^{i(\arg(a)-k\theta)}
  \\&=|a|(\cos(\arg(a)-k\theta) + i\sin(\arg(a)-k\theta)) \,.
\end{align*}
In the Figure~\ref{fig:functionF}, we illustrate the real and the imaginary
part of $f$.
\begin{figure}[h!] %{{{
  \begin{flushright}
    \tikzmarkc{n2}{purple} here is $\Im(ae^{-ik\theta})=0$
  \end{flushright}
  \begin{center}
    \begin{tikzpicture}[scale=4]
      \pgfmathsetmacro{\k}{3}
      \pgfmathsetmacro{\argA}{0.3}
      \pgfmathsetmacro{\absA}{0.4}

      \begin{scope}[thick]
        \clip (-0.6,{\absA+0.1}) rectangle (2.6,{-\absA-0.1});
        \foreach \x in {-2,-1,...,3}{
          \pgfmathsetmacro{\s}{{\argA + \x / \k * 2 - 1/2/\k}};
          \fill[blue!10!white]
            ({\s + 2/2/\k},0) sin ({\s + 3/2/\k},{-\absA})
                              cos ({\s + 4/2/\k},0);
          \draw[blue!40!black] (\s,0) sin ({\s + 1/2/\k},\absA)
                       cos ({\s + 2/2/\k},0)
                       sin ({\s + 3/2/\k},{-\absA})
                       cos ({\s + 4/2/\k},0);
          \fill[white] ({\s + 3/2/\k},{-\absA}) circle (1pt);
          \fill[blue!40!black] ({\s + 3/2/\k},{-\absA}) circle (.4pt);
          \pgfmathsetmacro{\s}{{\argA + \x / \k * 2}};
          \draw[purple] (\s,0) sin ({\s + 1/2/\k},\absA)
                       cos ({\s + 2/2/\k},0)
                       sin ({\s + 3/2/\k},{-\absA})
                       cos ({\s + 4/2/\k},0);
          \fill[white] (\s,0) circle (1pt);
          \fill[purple] (\s,0) circle (.4pt);
          \fill[white] ({\s + 2/2/\k},0) circle (1pt);
          \fill[purple] ({\s + 2/2/\k},0) circle (.4pt);
        }
      \end{scope}
      \draw [blue!40!black,dashed]
        ({\argA+4/\k + 1/2/\k},{-\absA}) -- ({\argA+4/\k + 1/2/\k},0);
      \draw [blue!40!black,dashed]
        ({\argA+4/\k + 3/2/\k},{-\absA}) -- ({\argA+4/\k + 3/2/\k},0);
      \draw [blue!40!black, thick
            ,decorate
            ,decoration={brace,mirror,amplitude=10pt}
            ,xshift=0pt
            ,yshift=0pt]
        ({\argA+4/\k + 1/2/\k},{-\absA}) -- ({\argA+4/\k + 3/2/\k},{-\absA})
        node [midway,yshift=-7pt] {$\tikzmark{g0}$};

      \node at ({\argA+2 - 3/\k},0) {$\tikzmark{f-1}$};
      \node at ({\argA+2 - 2/\k},0) {$\tikzmark{f0}$};
      \node at ({\argA+2 - 1/\k},0) {$\tikzmark{f1}$};

      \node at ({\argA-2/\k + 2/2/\k},{-\absA}) {$\tikzmark{e-1}$};
      \node at ({\argA + 2/2/\k},{-\absA}) {$\tikzmark{e0}$};
      \node at ({\argA+2/\k + 2/2/\k},{-\absA}) {$\tikzmark{e1}$};
      \node at ({\argA+4/\k + 2/2/\k},{-\absA}) {$\tikzmark{e2}$};

      \draw[-latex'] (-0.7,0) -- (2.7,0);
      \draw[dotted] (-0.6,\absA)node[left,font=\tiny] {$|a|$} -- (2.6,\absA);
      \draw[dotted] (-0.6,{-\absA})node[left,font=\tiny] {$-|a|$} -- (2.6,{-\absA});

      \draw[dotted] ({-0.5},{-\absA}) -- ({-0.5},{\absA + 0.1})
        node [above,font=\tiny,] {-0.5 \pi};
      \draw[-latex'] ({0},{-\absA-0.1}) -- ({0},{\absA + 0.2});
      \foreach \x in {0.5,1,...,2.5}{%
        \draw[dotted] ({\x},{-\absA}) -- ({\x},{\absA + 0.1})
          node [above,font=\tiny,] {\x \pi};
      }

      \draw[thick, dotted, green!50!black] (\argA,{-\absA}) -- (\argA,{\absA + 0.1})
        node [above,font=\tiny,] {$\frac{\arg(a)}{k}$};
    \end{tikzpicture}
  \end{center}
  \begin{flushright}
    \tikzmarkc{n3}{blue} here is $q_j\underset{\theta}{\prec}q_l$
  \end{flushright}
  \begin{flushright}
    \tikzmarkc{n1}{blue} here is $q_j\myrel{\theta}q_l$
  \end{flushright}
  \begin{tikzpicture}[remember picture,overlay]
    \draw[->,blue!50!white,thick] (n3) to[out=150,in=270] (g0);
    \draw[->,purple!50!white,thick] (n2) to[out=240,in=70] (f-1);
    \draw[->,purple!50!white,thick] (n2) to[out=265,in=120] (f0);
    \draw[->,purple!50!white,thick] (n2) to[out=280,in=70] (f1);
    \draw[->,blue!50!white,thick] (n1) to[out=180,in=270] (e-1);
    \draw[->,blue!50!white,thick] (n1) to[out=170,in=270] (e0);
    \draw[->,blue!50!white,thick] (n1) to[out=160,in=270] (e1);
    % \draw[->,blue!40!black,thick] (n1) to[out=150,in=270] (e2);
  \end{tikzpicture}
  \caption{In this plot is the real part of $f(\theta)=ae^{-ik\theta}$,
    corresponding to some pair $(q_j,q_l)$, in
    \textcolor{blue!60!white}{blue} and the imaginary part in
    \textcolor{purple}{purple} sketched.
  }\label{fig:functionF}
\end{figure} %}}}

The graphs, corresponding to the flipped pair $(q_l,q_j)$ are then obtained by
the transformation $\arg(a)\to\arg(-a)=\arg(a)+\pi$, i.e.\ the shift by
$\frac{\pi}{k}$ to the right. This $\frac{\pi}{k}$ is exactly a half period,
thus the new graphs are obtained by \rewrite{mirroring at the line $t=0$.}
\begin{rem}\label{rem:relationDistanceCondition}
  Let $k_{jl}$ be the degree of $q_j-q_l$.
  It is easy to see (cf.\ Figure~\ref{fig:functionF}), that the condition
  \[
    q_j \underset{\theta}{\prec} q_l
  \]
  is equivalent to
  \begin{einr}
    there is a $\theta'\in U(\theta,\frac{\pi}{k_{jl}})$ such that
    $q_j\myrel{\theta}q_l$.
  \end{einr}
\end{rem}

\begin{defn}\label{defn:antiStokesDir}
  % \marginnote{See \cite[Def.I.4.5]{Loday1994}(for the ramified case)
  %   \cite[Def.3.2]{boalch}}
  \begin{enumerate}
    \item $\alpha$ is an \emph{anti-Stokes direction} if there is at least one
      pair $(q_j,q_l)$ in $\cQ(A^0)$, which satisfies $q_j\myrel{\alpha}q_l$.
      \begin{itemize}
        \item Let $\A=\{\alpha_1,\dots,\alpha_{\nu}\}$ denote the set of all
          anti-Stokes directions in a clockwise ordering. For a uniform
          notation later, \rewrite{define $\A$ to contain a single, arbitrary
          direction if $\cK=\{0\}$.}
          \TODO[Is this used somewhere??]
          \begin{s-rem}
            The clockwise ordering is chosen, similar to Loday-Richaud's paper
            \cite{Loday1994}, since the calculations are then compatible with
            the calculations, which look at $\infty$ and take a
            counterclockwise ordering.
            Boalch uses in \cite{boalch} and \cite{thboalch} the inverse
            ordering, but looks also at $0$, thus there \rewrite{might be some}
            incompatibilities.
            In Loday-Richaud's book \cite{Loday2014} this problem is solved by
            an additional minus sign for some koefficients\comm{~to $a_{ij}$}.
            \PROBLEM[mentioned to often?]
          \end{s-rem}
      \end{itemize}
    \item $\theta$ is a \emph{Stokes direction} if there is at least one pair
      $(q_j,q_l)$ in $\cQ(A^0)$, which satisfies neither
      $q_j\underset{\theta}{\prec} q_l$ nor $q_l\underset{\theta}{\prec} q_j$.
      \begin{itemize}
        \item Let $\S=\{\sigma_1<\cdots<\sigma_\mu\}$ be the set of Stokes
          directions.
      \end{itemize}
  \end{enumerate}
\end{defn}
We will use the greek letter $\alpha$ whenever we want to emphasize that a
direction is an anti-Stokes direction.
For generic directions, we will use $\theta$.
In fact will most of the following definitions and
constructions work for every $\theta\in S^1$, but the Stokes group
(cf.\ Definition~\ref{defn:stokesGroup}) for example will be trivial
for every $\theta\notin\A$.
Thus the interesting directions are  only the anti-Stokes directions
$\alpha\in\A$.

\begin{lem}\label{lem:rotationalSym}%\label{rem:rotationalSymPrime}
  Let $\alpha\in\A$ together with a pair $(q_j-q_l)(t^{-1})\in\cQ(A^0)$ of
  degree $k_{jl}$, such that $q_j\myrel{\alpha}q_l$ be given.
  We then know for every $m\in\N$ that
  \[
    \underset{=:\alpha'}{\underbrace{\alpha+m\frac{\pi}{k_{jl}}}} \in \A \,.
  \]
  Especially is either $q_j\myrel{\alpha'}q_l$ (in the case, when $m$ is even)
  or $q_l\myrel{\alpha'}q_j$ (when $m$ is uneven) satisfied
  (see Figure~\ref{fig:functionF}).
  \begin{s-cor}
    It follows that in the case $\cK=\{k\}$, the set $\A$ has
    $\frac{\pi}{k}$-rotational symmetry.
  \end{s-cor}
\end{lem}
\begin{proof}
  \marginnote{\cite[8]{thboalch}}
  Let $(j,l)$ be a pair such that $q_j\myrel{\alpha}q_l$, i.e.\ such that
  $a_{jl}e^{-ik_{jl}\alpha}\in\R_{<0}$.
  Hence, for $m\in\N$,
  \begin{align*}
    a_{jl}e^{-ik_{jl}\left(\alpha+m\frac{\pi}{k_{jl}}\right)}
    &=a_{jl}e^{-ik_{jl}\alpha}e^{-im\pi}
    = \begin{cases}
      a_{jl}e^{-ik_{jl}\alpha}\in\R_{<0}
        & \text{, if $m$ is even}
    \\-a_{jl}e^{-ik_{jl}\alpha}\in\R_{>0}
        & \text{, if $m$ is uneven}
    \end{cases}
  \end{align*}
  is, in the case when $m$ is even, also real and negative. In the other
  case, when $n$ is uneven, we use that $a_{jl}=-a_{lj}$ and $k_{jl}=k_{lj}$ to
  obtain
  $a_{lj}e^{-ik_{lj}\left(\alpha+m\frac{\pi}{k_{lj}}\right)} \in\R_{<0}$.

  Thus, for $\alpha':=\alpha+m\frac{\pi}{k_{jl}}$, we have $\alpha'\in\A$ since
  \begin{itemize}
    \item $q_j\myrel{\alpha'}q_l$ when $m$ is even or
    \item $q_l\myrel{\alpha'}q_j$ when $m$ is uneven.
  \end{itemize}
\end{proof}

As a \rewrite{subgroup of the stalk at $\theta$} of the in
Definition~\ref{defn:StokesSheaf} defined Stokes sheaf $\Lambda(A^0)$ we
define the Stokes group as follows.
\begin{defn}\label{defn:stokesGroup}
  Define the \emph{Stokes group}
  \[
    \Sto_\theta(A^0):=
    \left\{\phi_\theta\in\Lambda_\theta(A^0)
      \mid \phi_\theta \text{~has maximal decay at~} \theta
    \right\}
  \]
  whose elements are called \emph{Stokes germs}.
  \TODO[This is in fact a group, since\dots]
  \begin{s-rem}
    For $\theta\notin\A$ the group $\Sto_\theta(A^0)$ is trivial, since at
    $\theta$ no flat isotropy has maximal decay, but the identity.
  \end{s-rem}
\end{defn}

%%%%%%%%%%%%%%%%%%%%%%%%%%%%%%%%%%%%%%%%%%%%%%%%%%%%%%%%%%%%%%%%%%%%%%%%%%%%%%%
\subsection{Stokes matrices}\label{sec:matrixReps}
\marginnote{\cite[9f]{thboalch}, \cite[??]{Loday1994}}
Stokes matrices, which Wasow calls in his book~\cite{wasow2002asymptotic}
Stokes multipliers and Boalch calls them Stokes factors
in~\cite{boalch,thboalch}, arise either
\begin{einr}
  as faithful representations of Stokes germs
\end{einr}
or, if one starts by comparing the actual fundamental solutions on arcs, as
\begin{einr}
  the matrices describing the blending between two adjacent fundamental
  solutions, with some additional assumptions
  (cf.\ Definition~\cite[80]{Loday2014}).
\end{einr}
\begin{defn}\label{defn:groupOfFaithfullReps}
  Let us use
  \[
    \bdelta_{jl}:=
    \begin{cases}
      0 \in \C^{n_j\times n_l} & \text{,~if~} j\neq l
    \\\id \in \C^{n_j\times n_l} & \text{,~if~} j=l
    \end{cases}
  \]
  as a block version of Kronecker's delta corresponding to the structure of the
  normal solution $\cY_0$, which was fixed.
  Define the group
  \begin{align*}
    \SSto_\theta(A^0)= \Big\{K=(K_{jl})_{j,l\in\{1,\dots,s\}}\in\GL_n(\C) \mid
      K_{jl}=\bdelta_{jl} \text{~unless~} q_j\myrel{\theta}q_l \Big\}
  \end{align*}
  of all \emph{Stokes matrices} of $A^0$ in the direction $\theta$.
  They will arise as a faithful representation
  (cf.\ Section~\ref{sec:faithRepre}) of $\Sto_\theta(A^0)$.
  \begin{s-rem}
    \begin{enumerate}
      \item In Boalch's publications~\cite{boalch,thboalch} are our Stokes
        matrices called Stokes factors, since he introduces other objects, he
        wants to call Stokes matrices
        (cf.\ Section~\ref{sec:furtherImprovements}).
      \item There is obviously a bijection
        $\vartheta_\theta: \prod_{q_j\myrel{\theta}q_l}\C^{n_j \cdot n_l}
        \overset{\cong}{\longrightarrow} \SSto_\theta(A^0)$.
    \end{enumerate}
  \end{s-rem}
\iffalse
  % \begin{s-rem}
  %   \BIGPROBLEM[needs more assumptions!]
  %   \begin{comment}
  %     Loday-Richaud defines Stokes matrices in \cite[Defn.III.3.4]{Loday1994}.
  %   \end{comment}
  %   Let
  %   \begin{itemize}
  %     \item $q_1 = \frac{2}{t^k} + \frac{2}{t^{k-1}}$,
  %     \item $q_2 = \frac{1}{t^k} + \frac{1}{t^{k-1}}$ and
  %     \item $q_3 = \frac{2}{t^k} + \frac{1}{t^{k-1}}$
  %   \end{itemize}
  %   be the three diagonal elements of $Q$ of some system $A$.
  %   Thus the determining polynomials are given by
  %   \begin{itemize}
  %     \item $q_{1,2}=q_1-q_2 = \frac{1}{t^k}$,
  %     \item $q_{1,3}=q_1-q_3 = \frac{1}{t^{k-1}}$ and
  %     \item $q_{3,2}=q_3-q_2 = \frac{1}{t^k}$.
  %   \end{itemize}
  %   Since the leading terms of $q_{1,2}$ and $q_{3,2}$ are equal, they
  %   determine the same anti-Stokes directions. Let $\theta$ be one of these
  %   directions, which satisfies neither
  %   $q_1\underset{\theta,\max}{\prec}q_3$ nor
  %   $q_3\underset{\theta,\max}{\prec}q_1$ (it is always possible, to find such
  %   a direction since the level of $q_{1,3}$ is different to the levels of the
  %   other ones).
  %   The group of Stokes matrices at $\theta$ is then given by
  %   \[
  %     \SSto_\theta(A^0)=\left\{
  %       K= \begin{pmatrix}
  %         1 & a & 0
  %       \\0 & 1 & 0
  %       \\0 & b  & 1
  %       \end{pmatrix}
  %       \mid
  %       a,b \in\C
  %     \right\}
  %   \]
  %   But $\SSto_\theta(A^0)$ is \textbf{not closed under the product}, i.e.\
  %   \[
  %     \begin{pmatrix}
  %       1 & a & 0
  %     \\0 & 1 & 0
  %     \\0 & 0 & 1
  %     \end{pmatrix}
  %     \begin{pmatrix}
  %       1 & 0 & 0
  %     \\0 & 1 & b
  %     \\0 & 0 & 1
  %     \end{pmatrix}
  %     =
  %     \begin{pmatrix}
  %       1 & a & \textcolor{red!60!black}{ab}
  %     \\0 & 1 & b
  %     \\0 & 0 & 1
  %     \end{pmatrix}
  %     \notin \SSto_\theta(A^0)
  %     \,.
  %   \]
  %   \textcolor{green!40!black}{Maybe define the Stokes matrices as:}
  %   \[
  %     \SSto_\theta(A^0)=\left\{
  %       K= \begin{pmatrix}
  %         1 & a & ab
  %       \\0 & 1 & b
  %       \\0 & 0 & 1
  %       \end{pmatrix}
  %       \mid
  %       a,b \in\C
  %     \right\}
  %   \]
  % \end{s-rem}
\fi
\end{defn}

\begin{prop}\label{prop:representation}
  \marginnote{\cite[Def.I.4.7]{Loday1994}\\\cite[78f]{Loday2014}}
  In this situation is the morphism
  \begin{align*}
    \rho_{\theta}:\Sto_\theta(A^0)&\longrightarrow\SSto_\theta(A^0)
    \\\phi_\theta
    &\longmapsto
    C_{\phi_\theta}:=\cY_{0}\phi_\theta\cY_{0}^{-1}
  \end{align*}
  \marginnote{Boalch uses $C_{\phi_\theta}:=\cY_{0}^{-1}\phi_\theta\cY_{0}$}
  an isomorphism which maps a germ of $\Sto_\theta(A^0)$ to the corresponding
  Stokes matrix $C_{\phi_\theta}$ such that
  \begin{equation} \label{eq:representation}
    \phi_\theta(t)\cY_{0}(t)=\cY_{0}(t)C_{\phi_\theta}
  \end{equation}
  near $\theta$.
  The matrix $C_{\phi_\theta}$ is then called a \emph{representation of
  $\phi_\theta$}.
  \begin{s-rem}
    \begin{enumerate}
      \item In the unramified case does this morphism depend on the choice of
        the determination $\tilde\theta$ of $\theta$ and the corresponding
        choice of a realization of the fundamental solution with that
        determination of the argument near the direction $\theta$
        (cf.~\cite{Loday1994} or~\cite[78f]{Loday2014}).
        \TODO[Satz zu lang]
      \item \marginnote{\cite[Defn.I.4.7]{Loday1994}}
        This construction defines also a morphism, which takes a germ
        $\phi_\theta\in\Lambda_\theta(A^0)\supset\Sto_\theta(A^0)$ into its
        unique representation matrix
        \[
          C_{\phi_\theta} \in
          \underset{\SSto_\theta(A^0)}{%
            \underset{\text{\rotatebox[origin=c]{90}{$\subset$}}}{%
              \hat\SSto_{\theta}(A^0)}}
          :=
          \left\{(K_{jl})_{j,l\in\{1,\dots,s\}}
            \in \GL_n(\C)\mid K_{jl}=\bdelta_{jl} \text{ unless }
            q_j \underset{\theta}{\prec} q_l \right\}
        \]
        and there is a bijection $\hat\vartheta_\theta:
        \prod_{q_j\underset{\theta}{\prec}q_l}\C^{n_j \cdot n_l}
        \overset{\cong}{\longrightarrow} \hat\SSto_\theta(A^0)$.
        \begin{comment}
          Does this define a local-constant sheaf
          \[
            I\mapsto \hat\SSto_{I}(A^0)
            :=
            \left\{(K_{jl})_{j,l\in\{1,\dots,s\}}
              \in \GL_n(\C)\mid K_{jl}=\bdelta_{jl} \text{ unless }
              q_j \underset{\theta}{\prec} q_l
              \text{ for some } \theta\in I\right\}
          \]
          and a skyscreaper sheaf
          \[
            I\mapsto \SSto_{I}(A^0) \,.
          \]
          \PROBLEM
        \end{comment}
    \end{enumerate}
  \end{s-rem}
\end{prop}
\begin{proof}
  It is well known (cf.\ \cite[10]{thboalch}), that the morphism
  $\rho_{\theta}$, i.e.\ conjugation by the fundamental solution, relates
  solutions $\phi_\theta$ of $[\End(A^0)]=[A^0,A^0]$ to solutions of $[0,0]$
  which are the constant matrices $\GL_n(\C)$.
  Thus we have to show, that the image of $\Sto_\theta(A^0)$ under
  $\rho_{\theta}$ is $\SSto_\theta(A^0)$.

  To see that the obtained matrix has the necessary zeros, to lie in
  $\SSto_{\theta}(A^0)$ we look at Equation (\ref{eq:representation}) and
  deduce
  \begin{equation}\label{eq:repProof1}
    \phi_\theta(t)
    =t^L e^{Q(t^{-1})}C_{\phi_\theta}e^{-Q(t^{-1})}t^{-L}
  \end{equation}
  with the given choice of the argument near $\theta$.
  After decomposing $C_{\phi_\theta}$ into
  \begin{align*}
    C_{\phi_\theta}&=1_n+\begin{pmatrix}
      c_{(1,1)} & c_{(1,2)} & \cdots &\\
      c_{(2,1} & \ddots\\
      \vdots \\
      & & & c_{(s,s)}
    \end{pmatrix}
  \\&=1_n+
    \underset{C_{\phi_\theta}^{(1,1)}}{\underbrace{%
      \begin{pmatrix}
        c_{(1,1)} & 0 & \cdots &\\
        0\\
        \vdots&\\
        &
      \end{pmatrix}
    }}
    +
    \underset{C_{\phi_\theta}^{(1,2)}}{\underbrace{%
      \begin{pmatrix}
        0 & c_{(1,2)} & 0 & \cdots\\
        & 0 &\\
        &\vdots\\
        &
      \end{pmatrix}
    }}
    +\cdots+
    \underset{C_{\phi_\theta}^{(s,s)}}{\underbrace{%
      \begin{pmatrix}
        &\\
        & & & \vdots\\
        & & & 0\\
        & \cdots & 0 & c_{(s,s)}
      \end{pmatrix}
    }}
  \\&=1_n+\sum_{(l,j)}C_{\phi_\theta}^{(l,j)}
  \end{align*}
  where the $c_{(j,l)}$ are blocks\marginnote{One can ignore the block
  structure by using $1\times1$ sized blocks. But one looses the uniqueness of
  the $q_j$'s.} of size $n_j\times n_l$ which correspond to the structure of
  $Q$. After rewriting the Equation (\ref{eq:repProof1}) we get
  \[
    \phi_\theta=
      t^L\left(
        1_n+\sum_{(l,j)}C_{\phi_\theta}^{(l,j)}e^{(q_l-q_j)(t^{-1})}
      \right)t^{-L} \,.
  \]
  \begin{comment}
    \begin{align*}
      \phi_\theta(t)
      &=t^Le^{Q(t^{-1})}\left(
        1_n+C_{\phi_\theta}
      \right)e^{-Q(t^{-1})}t^{-L}
    \\&=t^Le^{Q(t^{-1})}\left(
        1_n+\sum_{(l,j)}C_{\phi_\theta}^{(l,j)}
      \right)e^{-Q(t^{-1})}t^{-L}
    \\&=t^L\left(
        1_n+\sum_{(l,j)}e^{Q(t^{-1})}C_{\phi_\theta}^{(l,j)}e^{-Q(t^{-1})}
      \right)t^{-L}
    \\&=t^L\left(
          1_n+\sum_{(l,j)}C_{\phi_\theta}^{(l,j)}e^{(q_l-q_j)(t^{-1})}
        \right)t^{-L} \,.
    \end{align*}
  \end{comment}
  Thus, for $\phi_{\theta}$ to be flat in direction $\theta$, it is
  necessary and sufficient that if $e^{(q_l-q_j)(t^{-1})}$ does not have
  maximal decay in direction $\theta$ the corresponding
  block $C_{\phi_\theta}^{(l,j)}$ vanishes.
  Thus we have seen, that $C_{\phi_\theta}$ is an
  element of $\SSto_\theta(A^0)$.

  The \textbf{surjectivity} can \rewrite{be seen easily}, since every constant
  matrix, with zeros at the necessary positions, characterizes an unique
  element of $\Sto_\theta(A^0)$:
  \begin{einr}
    Let $C=1_n + \sum_{(l,j)\mid q_j\myrel{\theta}q_l} C^{(l,j)}$ be an element
    of $\SSto_\theta(A^0)$.
    Then is a pre-image of $C$ given by
    $t^Le^{Q(t^{-1})}Ce^{-Q(t^{-1})}t^{-L}$ which lies in $\Sto_\theta(A^0)$,
    since it satisfies the condition discussed above.
  \end{einr}

  The map $\rho_{\tilde\theta}$ is also \textbf{injective}, since it is the
  conjugation by an invertible matrix.
\end{proof}

From the calculations in the proof it is clear that
\begin{enumerate}
  \item for $j=l$ the (diagonal) blocks $C_{\phi_\theta}^{(l,j)}$ vanish since
    $q_l-q_j=0$ does not have maximal decay and
  \item if $e^{q_j-q_l}$ has has maximal decay, then $e^{q_l-q_j}$ has not.
    Thus if $C_{\phi_\theta}^{(l,j)}$ is not equal to zero, the block
    $C_{\phi_\theta}^{(j,l)}$ is necessarily zero.
\end{enumerate}
This implies that the matrix $C_{\phi_\theta}$ is unipotent, and \rewrite{hence
is $\Sto_\theta(A^0)$ is a unipotent Lie group.}

One can use the Stokes matrices to give an alternative characterization of
Stokes germs:
\begin{einr}
  a germ $\phi_\theta\in\Lambda_\theta(A^0)$ is in $\Sto_\theta(A^0)$ if and
  only if there exists a $C\in\SSto_\theta(A^0)$ such that
  $\phi_\theta=\cY_{0}C\cY_{0}^{-1}$.
\end{einr}
Formulated is this in the following corollary.
\begin{cor}
  \marginnote{\cite[Def.I.4.12]{Loday1994}}
  A germ $\phi_\theta\in\Lambda_\theta(A^0)$ is a Stokes germ, i.e.\ an element
  in $\Sto_\theta(A^0)$, if and only if it has a representation
  $C_{\phi_\theta}$ where
  \[
    C_{\phi_\theta}=
        1_n + \sum_{(l,j)\mid q_j\myrel{\theta}q_l}C_{\phi_\theta}^{(l,j)}
  \]
  and the $C_{\phi_\theta}^{(l,j)}$ have the necessary block structure.
  \begin{s-rem}
    In Loday-Richaud's book~\cite[78]{Loday2014} are the elements of
    $\Sto_\theta(A^0)$ actually characterized as the flat transformations, such
    that Equation (\ref{eq:representation}) is satisfied for some unique
    constant invertible matrix $C\in\SSto_\theta(A^0)$.
  \end{s-rem}
\end{cor}

\begin{defn}
  We denote the set of \emph{levels of the germ}
  $\phi_{\theta}\in\Lambda_\theta(A^0)$ by
  \[
    \cK(\phi_\theta):= \left\{\deg(q_j-q_l)\mid C_{\phi_\theta}^{(l,j)}\neq0
      \text{ in some representation of }\phi_\theta\right\} \subset \cK \,.
  \]
  A germ $\phi_\theta$ is called a \emph{$k$-germ} when
  $\cK(\phi_{\theta})\subset\{k\}$, i.e.\ it has at most the level $k$.
\end{defn}

\begin{comment}
  \PROBLEM[This would be good]
  \begin{lem}
    Every $k$-germ in direction $\alpha$ can be extended to the Stokes arc
    $U(\frac{\pi}{k},\alpha)$.
    \begin{cor}
        Every germ $\phi_\alpha\in\Sto_\alpha(A^0)$ can be extended to the arc
        $U(\frac{\pi}{\max\cK(\phi_\alpha)},\alpha)$, i.e.\ there is a section
        $\phi\in\Gamma\left(U(\frac{\pi}{\max\cK(\phi_\alpha)},\alpha),
          \Lambda(A^0)\right)$ which has $\phi_\alpha$ as its germ at $\alpha$.
    \end{cor}
  \end{lem}
  Let $\phi_\alpha$ be a \textbf{simple} $k$-germ in the sense that it is build
  from a single block, i.e.\
  \[
    \phi_\theta=
      t^L\left(
        1_n+C_{\phi_\theta}^{(l,j)}e^{(q_l-q_j)(t^{-1})}
      \right)t^{-L} \,.
  \]
  for some pair $(j,l)$. Assume also that the block has size $1\times1$.

  Let $\phi$ be the extension of $\phi_\alpha$ around alpha, i.e.\ the matrix
  which has germ $\phi_\alpha$ at $\alpha$ and which solves $[\End A^0]$ and is
  multiplicatively flat.

  \begin{einr}
    The system $[\End A^0]$ is
    \[
    \frac{dF}{dt}=A^0F-FA^0 \,.
    \]
  \end{einr}

  \paragraph{Question:} Which form has the extension $\phi$ around $\alpha$ of
  the germ $\phi_\alpha$, does it retain the structure?
  \begin{einr}
    \begin{enumerate}
    \item Look at a diagonal elemet $\phi^{j,j}$ an $(j,j)$:
      \begin{itemize}
      \item it satisfies $\phi_\alpha^{j,j}=1$ and
      \item is satisfies some complicated equation
        \paragraph{Question:} Is it constantly $1$
        \begin{einr}
          hopefully yes
        \end{einr}
      \end{itemize}
    \item Look at an offdiagonal position at $(j,l)$:
      \begin{itemize}
      \item[case a:] $\phi_\alpha^{j,l}\neq 0$
      \item[case b:] $\phi_\alpha^{j,l}=0$
      \end{itemize}
    \end{enumerate}
  \end{einr}
\end{comment}

%%%%%%%%%%%%%%%%%%%%%%%%%%%%%%%%%%%%%%%%%%%%%%%%%%%%%%%%%%%%%%%%%%%%%%%%%%%%%%%
\subsection{Decomposition of the Stokes group by levels}
\marginnote{\cite{Loday1994}, \cite[362ff]{Martinet1991}}
\rewrite{The goal of this section is, to introduce a filtration of
$\Lambda(A^0)$, wich will be restricted to $\Sto_\theta(A^0)$ and defines there
a filtration.} This leads to a decomposition of $\Sto_\theta(A^0)$ into a
semidirect product (cf.\ Prpopsition~\ref{prop:filtrationOfStokesGroup}).

Let us introduce a couple of notations and definitions, which coincide with the
notations used in Loday-Richaud's paper \cite{Loday1994}.
Another good resource, which uses similar notations, is for example the
paper~\cite[362f]{Martinet1991} from Martinet and Ramis.
\begin{notations}
  \marginnote{\cite[Not.I.4.15]{Loday1994},\\\cite[362]{Martinet1991}}
  For every level $k\in\cK$ and direction $\theta\in S^1$ we set
  \begin{itemize}
    \item $\Lambda^{k}(A^0)$ as the subsheaf of $\Lambda(A^0)$ of all germs,
      which are generated by $k$-germs;
    \item $\Lambda^{\leq k}(A^0)$ (resp. $\Lambda^{<k}(A^0)$ or
      $\Lambda^{\geq k}(A^0)$) as the subsheaf of $\Lambda(A^0)$ generated by
      $k'$-germs for all $k'\leq k$ (resp. $k'<k$ or $k'\geq k$).
  \end{itemize}
  Let $\star\in\{k,<k,\leq k,\dots\}$.
  The restrictions to $\Sto_\theta$ yield the groups
  \[
    \Sto_\theta^\star(A^0):=\Sto_\theta(A^0)\cap\Lambda_\theta^{\star}(A^0)
  \]
  and let us also define $\SSto^\star_\theta(A^0)$ as the groups of
  representations, which correspond to elements of  $\Sto^\star_\theta(A^0)$.
  \begin{comment}
    The corresponding induced isomorphisms are denoted by
    \[
        \rho_\theta^\star:\Sto^\star_\theta(A^0) \to \SSto^\star_\theta(A^0) \,.
    \]
    \TODO[not all are needed, defined somewher else]
  \end{comment}
\end{notations}
Corresponding to the definitions above, one can define
$\A^\star:=\left\{\alpha\in\A\mid\Sto_\alpha^\star(A^0)\neq\{\id\}\right\}$
for $\star\in\{k,<k,\leq k,\dots\}$ and we say that \emph{$\alpha$ is bearing
the level $k$} if $\alpha\in\A^k$.
\begin{rem}
  It is clear that for every $k\in\cK$ we have the canonical inclusions
  $\A^k\hookrightarrow\A^{\leq k}$ and $\A^{<k}\hookrightarrow\A^{\leq k}$.
\end{rem}
Sometimes it is also useful to talk about the \emph{set of levels beared by an
direction $\alpha\in\A$}:
\[
  \cK_\alpha:=\left\{k\in\cK\mid\Sto_\alpha^k(A^0)\neq\{\id\}\right\} \,.
\]
\begin{cor}
  The Lemma~\ref{lem:rotationalSym} implies that from $k\in\cK_{\alpha}$ follows
  that $k\in\cK_{\alpha+m\frac{\pi}{k}}$ for $m\in\N$.
\end{cor}
Let us now study the sheaves $\Lambda^\star(A^0)$, and discuss how they
correlate \rewrite{and how they can be composed from the others}.

The following proposition can be found as~\cite[Prop.I.5.1]{Loday1994} and the
key-statement is also given in~\cite[Prop.4.10]{Martinet1991}.
\begin{prop}\label{prop:PropertiesOfStokesSheafSplitting}
  \marginnote{\cite[Prop.I.5.1]{Loday1994}}
  For any level $k\in\cK$ one has that $\Lambda^{k}(A^0)$,
  $\Lambda^{\leq k}(A^0)$ and $\Lambda^{<k}(A^0)$ are sheaves of subgroups of
  $\Lambda(A^0)$ and the sheaf $\Lambda^k(A^0)$ is normal in
  $\Lambda^{\leq k}(A^0)$.
  \begin{comment}
    A subgroup $N$ is normal in $G$ ($N\vartriangleleft G$) if it is stable
    under conjugation, i.e.
    \[
      N\vartriangleleft G \Leftrightarrow \forall n\in N \forall g\in G,
      gng^{-1}\in N ,.
    \]
  \end{comment}

  \marginnote{\cite[Proposition 10]{Martinet1991}}
  \rewrite{We even know more,} let
  \begin{itemize}
    \item $i:\Lambda^k(A^0)\hookrightarrow\Lambda^{\leq k}(A^0)$ be the
      canonical inclusion and
    \item $p:\Lambda^{\leq k}(A^0)\twoheadrightarrow\Lambda^{<k}(A^0)$ be the
      truncation to terms of levels $<k$.
  \end{itemize}
  Then does the exact sequence of sheaves
  \[
    1\longrightarrow\Lambda^k(A^0)
    \overset{i}\longrightarrow\Lambda^{\leq k}(A^0)
    \overset{p}\longrightarrow\Lambda^{<k}(A^0)
    \longrightarrow 1 \,,
  \]
  split.
\end{prop}
From the splitting of the sequence, we obtain immediately the following
decomposition into a semidirect product.
\begin{cor}\label{cor:factorStokesGerms}
  \marginnote{\cite[Cor.I.5.2]{Loday1994}}
  For any $k\in\cK$, there are the two following ways of factoring
  $\Lambda^{\leq k}(A^0)$ in a semidirect product:
  \begin{align*}
    \Lambda^{\leq k}(A^0)&\cong \Lambda^{<k}(A^0)\ltimes\Lambda^{k}(A^0)
  \\                    &\cong \Lambda^{k}(A^0)\ltimes\Lambda^{<k}(A^0)\,.
  \end{align*}
  Thus any germ $f^{\leq k}\in\Lambda^{\leq k}(A^0)$ can be uniquely written as
  \begin{itemize}
    \item $f^{\leq k}=f^{<k}g^k$, where $f^{<k}\in\Lambda^{<k}$ and
      $g^k\in\Lambda^k$, or
    \item $f^{\leq k}=f^kf^{<k}$, where $f^k\in\Lambda^k$ and
      $f^{<k}\in\Lambda^{<k}$.
  \end{itemize}
  \begin{s-rem}\label{rem:algFactorization}
    \marginnote{\cite[Cor.I.5.2(ii)]{Loday1994}}
    We can get the factor $f^{<k}$ common to both factorizations by truncation
    of $f^{\leq k}$ to terms of level $<k$, i.e.\ the map $p$ from
    Proposition~\ref{prop:PropertiesOfStokesSheafSplitting}.
    This truncation can explicitly be achieved, in terms of Stokes matrices, by
    keeping in representations $1+\sum C^{(j,l)}$ of $f^{\leq k}$ only the
    blocks $C^{(j,l)}$ such that $\deg(q_j-q_l)<k$.

    A factorization algorithm could then be:
    \begin{einr}
      get the factor $f^{<k}$ common to both factorizations by truncation of
      $f^{\leq k}$ to terms of level $<k$ and set $g^k:=(f^{<k})^{-1}f^{\leq k}$
      and $f^k:=f^{\leq k}(f^{<k})^{-1}$.
    \end{einr}
  \end{s-rem}
  This decomposition in a semidirect product can be extended to all levels,
  since $\Lambda^{<k}(A^0)=\Lambda^{\leq\max\{k'\in\cK\mid k' < k\}}$.
  Thus
  \[
    \Lambda(A^0)\cong\underset{k\in\cK}\bigltimes\Lambda^k(A^0) \,,
  \]
  where the semidirect product is taken in an ascending or descending order of
  levels.
\end{cor}
\begin{rem}
  Loday-Richaud states in her paper \cite[Prop.I.5.3]{Loday1994} the following
  proposition, which is a more general version of
  Proposition~\ref{prop:PropertiesOfStokesSheafSplitting}.
  \begin{s-prop}
    \marginnote{\cite[Prop.I.5.3]{Loday1994}}
    For any levels $k$,$k'\in\cK$ with $k'<k$ one has:
    \begin{enumerate}
      \item the sheaf $\Lambda^{\geq k'}(A^0)\cap\Lambda^{\leq k'}(A^0)$ is
        normal in $\Lambda^{\leq k}(A^0)$;
      \item \marginnote{\cite[Proposition 10]{Martinet1991}}
        the exact sequence of sheaves
        \[
          1\longrightarrow\Lambda^{\geq k'}(A^0)\cap\Lambda^{\leq k}(A^0)
          \overset{i}\longrightarrow\Lambda^{\leq k}(A^0)
          \overset{p}\longrightarrow\Lambda^{<k'}(A^0)
          \longrightarrow 1 \,,
        \]
        where
        \begin{itemize}
          \item $i$ is the canonical inclusion and
          \item $p$ is the truncation to terms of levels $<k'$,
        \end{itemize}
        splits.
    \end{enumerate}
    \TODO[is $\Lambda^{\geq k'}(A^0)\cap\Lambda^{\leq k}(A^0)=\Lambda^k(A^0)$
    and thus the first proposition a corollary of this?]
  \end{s-prop}
  We can use this proposition to follow (cf.\ \cite[Cor.I.5.4]{Loday1994}) that
  \begin{enumerate}
    \item
      the filtration
      \[
        \Lambda^{k_r}(A^0)
        =
        \Lambda^{\geq k_r}(A^0)
        \subset
        \Lambda^{\geq k_{r-1}}(A^0)
        \subset
        \cdots
        \subset
        \Lambda^{\geq k_{1}}(A^0)
        =
        \Lambda(A^0)
      \]
      is normal and
    \item we can use this to achieve the decomposition
      \[
        \Lambda(A^0)\cong\underset{k\in\cK}\bigltimes\Lambda^k(A^0)
      \]
      taken in an arbitrary order. In fact, one can also extend the algorithm
      from Remark~\ref{rem:algFactorization} to an arbitrary order of levels.
  \end{enumerate}
\end{rem}
The important statement, which we will use later, is then the following.
\begin{prop}\label{prop:filtrationOfStokesGroup}
  \marginnote{\cite[Prop.I.5.5]{Loday1994}}
  The results can be restricted to the Stokes groups
  (cf.\ \cite[Prop.I.5.5]{Loday1994}). Thus, for $\alpha\in\A$, one has
  \[
    \Sto_\alpha(A^0)\cong\underset{k\in\cK_\alpha}\bigltimes\Sto_\alpha^k(A^0)
  \]
  the semidirect product being taken in an arbitrary order\comm{~(we will only
  be interested in the ascending order)}.
  \begin{s-defn}
    We will denote the map, wich gives the factors of this factorization by
    \[
      i_\alpha:
      \Sto_\alpha(A^0)
      \overset{\cong}\longrightarrow
      \prod_{k\in\cK_\alpha}\Sto_\alpha^k(A^0)\,,
    \]
    where the factorization is taken in ascending order.
  \end{s-defn}
  \begin{s-rem}\label{rem:filtrationOfStokesMats}
    Write $\rho_{\alpha}^k:\Sto^k_\alpha(A^0)\to\SSto^k_\alpha(A^0)$ for
    the \rewrite{restriction} of the map $\rho_{\alpha}$
    (cf.\ Proposition~\ref{prop:representation}) to the level $k$.
    \rewrite{Then, one can} denote the induced decomposition also by
    \[
      i_\alpha:
      \SSto_\alpha(A^0)
      \overset{\cong}\longrightarrow
      \prod_{k\in\cK_\alpha}\SSto_\alpha^k(A^0)
    \]
    and the corresponding diagram
    \[ \begin{tikzcd}
        \Sto_\alpha(A^0)
        \rar{i_\alpha}
        \dar{\rho_{\alpha}}
        & \prod_{k\in\cK_\alpha}\Sto_\alpha^k(A^0)\,,
        \dar{\prod_{k\in\cK}\rho_{\alpha}^k}
      \\\SSto_\alpha(A^0)
        \rar{i_\alpha}
        & \prod_{k\in\cK_\alpha}\SSto_\alpha^k(A^0)\,,
    \end{tikzcd} \]
    commutes.
  \end{s-rem}
\end{prop}

\fi

% \ifnum\myDevelopVariable=0
%%%%%%%%%%%%%%%%%%%%%%%%%%%%%%%%%%%%%%%%%%%%%%%%%%%%%%%%%%%%%%%%%%%%%%%%%%%%%%%
\section{Stokes structures: using Stokes groups}\label{sec:mainThm2}
\marginnote{\cite{Loday1994},~\cite[Thm.4.3.11]{Loday2014}
  \\and~\cite{boalch,thboalch}
  \\and~\cite{babbitt1989local}
  \\and~\cite{BJL1979Birkhoff}
  \\and~\cite[Chapter 4]{Martinet1991}}

The goal in this section is to prove that there is a bijective and natural map
\[
  h:\prod_{\alpha\in\A}\Sto_\alpha(A^0)\longrightarrow\St(A^0) \,.
\]
And since $\Sto_\alpha(A^0)$ has $\SSto_\alpha(A^0)$ as a faithful
representation, we also get the isomorphism
$\prod_{\alpha\in\A}\SSto_\alpha(A^0)\cong\St(A^0)$ as a corollary.
\TODO[This goes back to \cite{BJL1979Birkhoff}?]

Let us recall, that $\St(A^0)$ is defined to be $H^1(S^1;\Lambda(A^0))$
(cf.\ Section~\ref{sec:mainThm1}).
The elements of $\prod_{\alpha\in\A}\Sto_\alpha(A^0)$ define in a canonical way
cocycles of the sheaf $\Lambda(A^0)$ (cf.\ Equation (\ref{eq:mapStoToCocy})),
called Stokes cocycles (cf.\ Definition~\ref{defn:stokesCocycle}).
In fact, will $h$ map such cocycles to the cohomology class, to which they
correspond.
Thus the statement, that $h$ is a bijection, is equivalent to the statement
that
\begin{einr}
  in each cohomology class of $\St(A^0)$ is an unique $1$-cocycle, which is a
  Stokes cocycle.
\end{einr}

%%%%%%%%%%%%%%%%%%%%%%%%%%%%%%%%%%%%%%%%%%%%%%%%%%%%%%%%%%%%%%%%%%%%%%%%%%%%%%%
\subsubsection{Cyclic coverings}
To formulate the following theorem, we use the notion of cyclic coverings and
nerves of such coverings, which are defined as follows.

\begin{defn}
  \marginnote{\cite[Sec.II.1]{Loday1994} and \cite[Sec.II.3.1]{Loday1994}}
  Let $J$ be a finite set, identified to $\{1,\dots,p\}\subset\Z$.
  \begin{enumerate}
    \item A \emph{cyclic covering} of $S^1$ is a finite covering
      $\cU=\left(U_j=U(\theta_j,\epsilon_j)\right)_{j\in J}$ consisting of
      arcs, which satisfies that
      \begin{enumerate}
        \item $\tilde\theta_j \geq \tilde\theta_{j+1}$ for $j\in\{1,\dots,p-1\}$,
        \item $\tilde\theta_j+\frac{\epsilon_j}{2}\geq
          \tilde\theta_{j+1}+\frac{\epsilon_{j+1}}{2}$ for
          $j\in\{1,\dots,p-1\}$ and
          $\tilde\theta_p+\frac{\epsilon_p}{2}\geq
          \tilde\theta_{1}-2\pi+\frac{\epsilon_{1}}{2}$, i.e.\ the arcs are not
          encased by another arc,
      \end{enumerate}
      where the $\tilde\theta_j\in [0,2\pi[$ are determinations of the
      $\theta_j\in S^1$.
      \begin{comment}
        \begin{enumerate}
          \item the $\theta_j$ are in ascending order with respect to the
            clockwise orientation of $S^1$;
          \item the $U_j\cap U_{j+1}$ have only one connected component when
            $\#J>2$;
          \item the $U_j$ are not encased by another arc, this means that the
            open sets $U_j\backslash U_l$ are connected for all $j,l\in J$.
        \end{enumerate}
      \end{comment}
    \item The \emph{nerve} of a cyclic covering $\cU=\{U_j;j\in J\}$ is the
      family $\dot\cU=\{\dot U_j;j\in J\}$ defined by:
      \begin{itemize}
        \item $\dot U_j=U_j\cap U_{j+1}$ when $\#J>2$,
        \item $\dot U_1$ and $\dot U_2$ the connected components of
          $U_1\cap U_2$ when $\#J=2$.
      \end{itemize}
      \begin{s-rem}
        The nerve of the cyclic covering
        $\cU=\left(U(\theta_j,\epsilon_j)\right)_{j\in J}$ is explicitly given
        by
        \[
          \dot\cU=\left(
            \left(\theta_{j+1}-\frac{\epsilon_{j+1}}{2},
            \theta_{j}+\frac{\epsilon_{j}}{2}\right)
          \right)_{j\in J} \,.
        \]
      \end{s-rem}
  \end{enumerate}
\end{defn}
The cyclic coverings correspond one-to-one to nerves of cyclic coverings. If
one starts with a nerve $\{\dot U_j \mid j\in J\}$, one obtains a cyclic
covering as $\cU=\{U_j \mid j\in J\}$ where the arc $U_j$ is the connected
clockwise hull from $\dot U_{j-1}$ to $\dot U_j$.

\begin{defn}
  A covering $\cV$ is said to \emph{refine} a covering $\cU$ if, to each open
  set $V\in\cV$ there is at least one $U\in\cU$ with $V\subset U$.
\end{defn}
\begin{prop}
  \marginnote{\cite[Prop.II.1.3]{Loday1994}}
  The covering $\cV$ refines $\cU$ if and only if the corresponding nerves
  $\dot\cU=\{\dot U_j\}$ and $\dot\cV=\{\dot V_l\}$ satisfy
  \begin{einr}
    each $\dot U_j$ contains at least one $\dot V_l$.
  \end{einr}
\end{prop}

%%%%%%%%%%%%%%%%%%%%%%%%%%%%%%%%%%%%%%%%%%%%%%%%%%%%%%%%%%%%%%%%%%%%%%%%%%%%%%%
\subsection{The theorem}
\marginnote{\cite[868]{Loday1994}}
Let $\{\theta_j\mid j\in J\}\subset S^1$ be a finite set and
$\dot\phi=(\dot\phi_{\theta_j})_{j\in J}
\in\prod_{j\in J}\Lambda_{\theta_j}(A^0)$ be a finite family of germs.
Let $\dot\phi_j$ be the function representing the germ $\dot\phi_{\theta_j}$
on its (maximal) arc of definition $\Omega_j$ around $\theta_j$.
In the following way, one can associate a cohomology class in $\St(A^0)$ to
$\dot\phi$:
\begin{einr}
  for every cyclic covering $\cU=(U_j)_{j\in J}$ which satisfies
  $\dot U_j\subset\Omega_j$ for all $j\in J$, one can define the $1$-cocycle
  $(\dot\phi_{j|\dot U_j})_{j\in J}\in\Gamma(\dot\cU;\Lambda(A^0))$.
\end{einr}
To a different cyclic covering, satisfying the condition above, this
construction yields a cohomologous $1$-cocycle, thus the induced map
\begin{equation}\label{eq:mapStoToCocy}
  \prod_{j\in J}\Lambda_{\theta_j}(A^0)
  \longrightarrow
  H^1(S^1;\Lambda(A^0))=\St(A^0)
\end{equation}
is welldefined (cf.\ \cite[868]{Loday1994}).
\begin{defn}\label{defn:stokesCocycle}
  \marginnote{\cite[Def.II.1.8]{Loday1994}
    \\, \cite[4.3.10]{Loday2014}
    \\, \cite[Defn 6 on p 374]{Martinet1991}}
  Let $\nu=\#\A$ the number of all anti-Stokes directions and write
  $\A=\{\alpha_1,\dots,\alpha_\nu\}$.

  A \emph{Stokes cocycle} is a 1-cocycle $(\phi_j)_{j\in\{1,\dots,\nu\}}\in
  \prod_{j\in\{1,\dots,\nu\}}\Gamma(U_j;\Lambda(A^0))$ corresponding to some
  cyclic covering with nerve $\dot\cU=(\dot U_j)_{j\in\{1,\dots,\nu\}}$,
  which satisfies for every $j\in\{1,\dots,\nu\}$
  \begin{itemize}
    \item $\alpha_j\in\dot U_j$ and
    \item the germ $\phi_{\alpha_j}:=\phi_{j,\alpha_j}$ of $\phi_j$ at
      $\alpha_j$ is an element of $\Sto_{\alpha_j}(A^0)$.
  \end{itemize}
  \begin{s-rem}\label{rem:inclusionGermRemark}
    \PROBLEM[refactor!]
    The sections in $\Gamma(\dot U_j;\Lambda(A^0))$ are uniquely
    determined as the extension of the germ at $\alpha_j$, since the sheaf
    $\Lambda(A^0)$ defined via the system $[A^0,A^0]$
    (cf.\ Definition~\ref{defn:StokesSheaf}).
    We thus have an injective map
    \[
      \prod_{j\in\{1,\dots,\nu\}}\Gamma(\dot U_j;\Lambda(A^0))
      \hookrightarrow
      \prod_{j\in\{1,\dots,\nu\}}\Sto_{\alpha_j}(A^0) \,,
    \]
    which takes an Stokes cocycle and yields the corresponding Stokes germs.
    For a fine enough covering $\cU$, i.e.\ a covering $\cU$ with a nerve
    $\dot\cU$ which consists of small enough arcs satisfying the conditions
    above, is this map a bijection.

    We will use this fact implicitly and assume that the covering is always
    fine enough to call elements of $\prod_{\alpha\in\A}\Sto_\alpha(A^0)$
    Stokes cocycles.
  \end{s-rem}
\end{defn}
We can use the Equation (\ref{eq:mapStoToCocy}) to obtain for Stokes cocycles
a mapping
\[ \begin{tikzcd}
    h:\prod_{\alpha\in\A}\Sto_\alpha(A^0)
    \rar[hook]&
    \prod_{\alpha\in\A}\Lambda_\alpha(A^0)
    \rar{\text{(\ref{eq:mapStoToCocy})}}&
    \St(A^0),
\end{tikzcd} \]
which takes a Stokes cocycle to its corresponding cohomology class.
\begin{center}
  \begin{minipage}[t]{0.8\textwidth}
    \begin{tthm}\label{thm:mainThm2}
      The map
      \[ \begin{tikzcd}
          h:\prod_{\alpha\in\A}\Sto_\alpha(A^0) \rar& \St(A^0)
      \end{tikzcd} \]
      is a bijection and natural.
      \begin{s-rem}
        \marginnote{\cite[869]{Loday1994},\cite[Sec.III.3.3]{Loday1994}}
        Natural means that $h$ commutes to isomorphisms and constructions over
        systems or connections they represent.
      \end{s-rem}
    \end{tthm}
  \end{minipage}
\end{center}
From Theorem~\ref{thm:mainThm2} and Proposition~\ref{prop:representation} we
get the following corollary.
\begin{cor}\label{cor:isomToChochN}
  \PROBLEM[mentioned twice]
  Using the isomorphisms $\Sto_\theta(A^0)\cong\SSto_\theta(A^0)$ from
  Proposition~\ref{prop:representation} we obtain
  \[
    \St(A^0) \cong \prod_{\alpha\in\A}\SSto_\alpha(A^0)
  \]
  which endows $\St(A^0)$ with the structure of an unipotent Lie group with the
  finite complex dimension $N:=\dim_\C\St(A^0)$
  (cf.~\cite[Sec.III.1]{Loday1994}).
  This can be rewritten in the following way:
  \[
    N=\sum_{\alpha\in\A}\dim_\C\SSto_\alpha(A^0)
      =\sum_{\alpha\in\A}\sum_{q_j\myrel{\alpha}q_l}n_j\cdot n_l
      =\sum_{\substack{1\leq j,l\leq n\\j<l}}2\cdot\deg(q_j-q_l)\cdot
        n_j\cdot n_l \,.
  \]
  \begin{s-rem}
    This number $N$ is known to be the \emph{irregularity} of $[\End A^0]$.
  \end{s-rem}
\end{cor}
\begin{comment}
  \marginnote{\cite[880f]{Loday1994}}
  We have also two structures of a linear affine variety on the set $\St(A^0)$.
  \begin{rem}
    Let $\sto_\alpha(A^0)$ be the Lie algebra corresponding to
    $\Sto_\alpha(A^0)$. The exponential map\footnote{This is not the map $\exp$
    from Section~\ref{sec:mainThm1}.} induces an homomorphism
    $\exp:\sto_\alpha(A^0)\to\Sto_\alpha(A^0)$ and denote by $\ln=\exp^{-1}$
    the inverse map.
    \begin{enumerate}
      \item The \emph{tangent linear structure} is defined as \TODO
      \item Using the map
        \begin{align*}
          \sto_{\alpha}(A^0) &\overset{\id+\cdot}\longrightarrow
          \Sto_{\alpha}(A^0)
        \\\dot{f}_\alpha & \longmapsto \id+\dot{f}_\alpha \,.
        \end{align*}
    \end{enumerate}
  \end{rem}
\end{comment}

\begin{rem}
  To define the inverse map of $h$, one has to find in each cocycle in
  $\St(A^0)$ the Stokes cocycle. Loday-Richaud gives an algorithm in section
  II.3.4 of her paper~\cite{Loday1994}, which takes a cocycle over an arbitrary
  cyclic covering and outputs cohomologous Stokes cocycle and thus solves this
  problem.
\end{rem}

%%%%%%%%%%%%%%%%%%%%%%%%%%%%%%%%%%%%%%%%%%%%%%%%%%%%%%%%%%%%%%%%%%%%%%%%%%%%%%%
\subsection{Proof of Theorem~\ref{thm:mainThm2}}\label{sec:proofOfMatrixThm}
We will only look at the unramified case, for which we refer to
\cite[Sec.II.3]{Loday1994}.
The proof in the ramified case can be found in \cite[Sec.II.4]{Loday1994}.
We first have to introduce adequate coverings, which will be used in the proof.

%%%%%%%%%%%%%%%%%%%%%%%%%%%%%%%%%%%%%%%%%%%%%%%%%%%%%%%%%%%%%%%%%%%%%%%%%%%%%%%
\subsubsection{Adequate coverings}
\begin{defn}
  \marginnote{\cite[371]{Martinet1991} defines adapted coverings}
  Let $\star\in\{k,<k,\leq k,\dots\}$.
  A covering $\cU$ beyond which the inductive limit
  $\underset{\cU}{\underrightarrow{\lim}}H^1(\cU;\Lambda^\star(A^0))$ is
  stationary is said to be \emph{adequate} to describe
  $H^1(S^1;\Lambda^\star(A^0))$ or \emph{adequate} to $\Lambda^\star(A^0)$.
  \begin{comment}
    A covering $\cU$ is said to be \emph{adequate} to describe
    $H^1(S^1;\Lambda^\star(A^0))$ or \emph{adequate} to $\Lambda^\star(A^0)$ if
    for every element in
    $\underset{\cU}{\underrightarrow{\lim}}H^1(\cU;\Lambda^\star(A^0))$
    given by some covering $\cU'$ and an element of
    $\Gamma(\cU';\Lambda^\star(A^0))$
    there exists
    \begin{itemize}
      \item an element in $\Gamma(\cU;\Lambda^\star(A^0))$ and
      \item an common refinement of $\cU$ and $\cU'$
    \end{itemize}
    such that \PROBLEM[the elements are~?? on the refined covering.]
  \end{comment}

  In other words is a covering $\cU$ adequate, if and only if the quotient map
  \[
      \Gamma(\cU;\Lambda^\star(A^0))\longrightarrow H^1(S^1;\Lambda^\star(A^0))
  \]
  is surjective.\TODO[Proof? check??]
  \begin{comment}
    \cite[371]{Martinet1991} introduces the following definition
    \begin{s-defn}
      A covering $\cU$ is \emph{adapted} if every anti-Stokes direction is
      contained in exactly one element of the nerve $\dot\cU$.
    \end{s-defn}
  \end{comment}
\end{defn}
The following proposition is in Loday-Richaud's paper~\cite{Loday1994} given as
Proposition II.1.7. It contains a simple characterization, which will be used
to see, that our defined coverings are adequate.
\begin{prop}\label{prop:adeqCovCondition}
  \marginnote{\cite[Prop.II.1.7]{Loday1994}}
  Let $k\in\cK_\alpha$.
  \begin{s-defn}
    Let $\alpha\in\A^k$.
    An arc $U(\alpha,\frac{\pi}{k})$ is called a \emph{Stokes arc of level $k$
    at $\alpha$}.
  \end{s-defn}
  A cyclic covering $\cU=(U_j)_{j\in J}$, which satisfies
  \begin{einr}
    for every $\alpha\in\A^k$ contains the Stokes arc $U(\alpha,\frac{\pi}{k})$
    at least one arc $\dot U_j$ from the nerve $\dot\cU$ of $\cU$
  \end{einr}
  is adequate to $\Lambda^k(A^0)$.

  The covering $\cU$ is adequate to $\Lambda^{\leq k}(A^0)$ (resp.\
  $\Lambda^{<k}(A^0)$) if it is adequate to $\Lambda^{k'}(A^0)$ for every
  $k'\leq k$ (resp.\ $k'<k$).
\end{prop}

Let $k\in\cK$.
We want to define the three cyclic coverings $\cU^{k}$, $\cU^{\leq k}$ and
$\cU^{<k}$ which will be adequate to $\Lambda^k(A^0)$, $\Lambda^{\leq k}(A^0)$
and $\Lambda^{<k}(A^0)$. Furthermore will the coverings be comparable at the
different levels.

\begin{enumerate}
  \item The first covering $\cU^{k}=\{\dot U_\alpha^k\mid\alpha\in\A^k\}$
    is the cyclic covering with nerve
    \[
      \dot\cU^k:=
      \left\{\dot U_\alpha^k=U(\alpha,\frac{\pi}{k})\mid\alpha\in\A^k\right\}
    \]
    consisting of all Stokes arcs of level $k$ for anti-Stokes directions
    bearing the level $k$.
\end{enumerate}

\begin{rem}\label{rem:superSectors}
  Boalch introduces in his publications~\cite[19]{boalch}
  and~\cite[Def.1.23]{thboalch} the notion of \emph{supersectors}, they are in
  the case of a single level $k$, defined as follows:
  \begin{einr}
    write the anti-Stokes directions as $\A=\{\alpha_1,\dots,\alpha_\nu\}$
    arranged according to the clockwise ordering, then is the $i$-th
    supersector defined as the arc
    \[
      \hat\Sect^k_i:=
        \left(\alpha_i-\frac{\pi}{2k},\alpha_{i+1}+\frac{\pi}{2k}\right) \,.
    \]
  \end{einr}
  This yields a cyclic covering $(\hat\Sect^k_i)_{i\in\{1,\dots,\nu\}}$ whose
  nerve is exactly $\dot\cU^k$ defined above.
  \begin{comment}
    \begin{center}
      \begin{tikzpicture}[scale=3]
        \node[] (zero) at (0,0) {};
        \draw[blue] (zero) circle (1cm);

        \fill[fill=green!20!white] (0,0) -- ({cos( 15 )},{sin( 15 )}) arc
          (15:85:1) -- cycle;

        \fill[fill=red!60!black] (0,0) -- ({cos( 15 )*0.5},{sin( 15 )*0.5}) arc
          (15:45:0.5) -- cycle;
        \fill[fill=red!60!black] (0,0) -- ({cos( 55 )*0.5},{sin( 55 )*0.5}) arc
          (55:85:0.5) -- cycle;

        \fill[fill=green!20!white] (0,0) -- ({cos( 15 )*0.4},{sin( 15 )*0.4}) arc
          (15:85:0.4) -- cycle;

        \node[green!40!black] at (1,1) {$\widehat\Sect_i$};

        \node[] (lft) at ({cos( 85 )},{sin( 85 )}) {};
        \node[] (rgt) at ({cos( 15 )},{sin( 15 )}) {};

        \draw[->,red!40!black] ({cos( 45 )},{sin( 45 )})
          to [out=35, in=25] (rgt);

        \draw[->,red!40!black] ({cos( 55 )},{sin( 55 )})
          to [out=65, in=75] (lft);

        \draw[thick,red!40!black,path fading=west] (0,0) -- +({cos( 85 )},{sin( 85 )});
        \draw[thick,red!40!black,path fading=west] (0,0) -- +({cos( 15 )},{sin( 15 )});
        \fill[red!40!black] ({cos( 85 )},{sin( 85 )}) circle (1pt);
        \fill[red!40!black] ({cos( 15 )},{sin( 15 )}) circle (1pt);

        \foreach \w/\str in {10/,
                             20/,
                             45/$\alpha_{i+1}$,
                             55/$\alpha_{i}$}
        {\draw (0,0) -- +({cos( \w )},{sin( \w )}) node[right] {\str};
         \fill[blue!20!white] ({cos( \w )},{sin( \w )}) circle (.7pt);
         \foreach \sep in {60,120,180,240,300}
         {\draw (0,0) -- +({cos( \w + \sep )},{sin( \w + \sep )});
          \fill[blue!20!white]
            ({cos( \w + \sep )},{sin( \w + \sep )}) circle (.7pt);
         }
        };


        \fill (zero) circle (1pt);
      \end{tikzpicture}
    \end{center}
  \end{comment}
\end{rem}
If we extend to more then one level level, $\#\cK>1$, the set
$\bigcup_{k\in\cK}\left\{U(\alpha,\frac{\pi}{k})\mid\alpha\in\A^{k}\right\}$ is
no longer a nerve.
Hence we have to define the coverings $\cU^{\leq k}$ and $\cU^{<k}$ in a
different way.
Denote by
\[
  \left\{K_1<\cdots<K_s=k\right\}
  =\left\{\max\left(\cK_\alpha\cap[0,k]\right)\mid\alpha\in\A^{\leq k}\right\}
\]
the set of all \emph{$k$-maximum levels} for $\alpha\in\A^{\leq k}$.
\begin{enumerate}
  \setcounter{enumi}{1}
  \item The cyclic covering
    $\cU^{\leq k}=\left\{U_\alpha^{\leq k}\mid\alpha\in\A^{\leq k}\right\}$
    will be defined by induction.
    Let us assume that
    \begin{einr}
      the $\dot U_\alpha^{\leq k}$ are defined for all $\alpha\in\A^{\leq k}$
      with $k$-maximum level greater than $K_i$ such that their complete family
      is a nerve.
    \end{einr}
    Let
    \begin{itemize}
      \item $\alpha$ be a anti-Stokes direction with $k$-maximum level $K_i$
        and
      \item $\alpha^-$ (resp.\ $\alpha^+$) be the next anti-Stokes direction
        with $k$-maximum level greater then $K_i$ on the left (resp.\ on the
        right) and define $\dot U_{\alpha^-,\alpha^+}$ as the clockwise hull of
        the arcs $\dot U_{\alpha^-}^{\leq k}$ and $\dot U_{\alpha^+}^{\leq k}$
        already defined by induction.
        If there are no anti-Stokes directions with $k$-maximum level greater
        then $K_i$ we set $\dot U_{\alpha^-,\alpha^+}=S^1$.
    \end{itemize}
    We then set
    \[
      \dot U_\alpha^{\leq k}
        :=U\left(\alpha,\frac{\pi}{K_i}\right)\cap\dot U_{\alpha^-,\alpha^+}
    \]
    and the family of all $\dot U_\alpha^{\leq k}$ is a nerve.
    \begin{rem}
      If $\alpha$ has a $k$-maximum level equal to $k$ then is
      $\dot U_\alpha^{\leq k}$ equal to the Stokes arc
      $U\left(\alpha,\frac{\pi}{k}\right)=\dot U_\alpha^k$.
      \comm{\dots{}and then no $0$-cochain with level $k$ or $\geq k$ can
      exists on the covering $\cU^{\leq{k}}$.}
    \end{rem}
\end{enumerate}

\begin{enumerate}
  \setcounter{enumi}{2}
\item The last cyclic covering,
  $\cU^{<k}=\left\{U_{\alpha}^{<k}\mid\alpha\in\A^{<k}\right\}$, is defined as
  $\cU^{<k}:=\cU^{\leq k'}$ where $k':=\max\{k''\in\cK\mid k''<k\}$.
\end{enumerate}

\begin{figure} %{{{
  \centering

  \def\kOne{7}
  \def\kTwo{10}
  \begin{tikzpicture}[scale=5] %{{{

    \newcommand{\myDrawArcA}[4]{%{{{
      % Parameter: radius , center , width , color
      \pgfmathsetmacro\hwdth{#3 / 2}
      \draw[#4]
        ({cos( #2 )},{sin( #2 )})
        --
        ({cos( #2 ) * #1 },{sin( #2 ) * #1 });
      \draw[ultra thick,#4]
        ({cos(#2 - \hwdth) * #1},{sin(#2 - \hwdth) * #1})
        arc
        ({#2 - \hwdth}:{#2 + \hwdth}:#1);
      \draw[dotted,#4]
        ({cos(#2 - \hwdth) * #1},{sin(#2 - \hwdth) * #1})
        --
        ({cos(#2 - \hwdth)},{sin(#2 - \hwdth)});
      \draw[dotted,#4]
        ({cos(#2 + \hwdth) * #1},{sin(#2 + \hwdth) * #1})
        --
        ({cos(#2 + \hwdth)},{sin(#2 + \hwdth)});
      \filldraw[white] ({cos(#2)},{sin(#2)}) circle (0.5pt);
      \filldraw[red] ({cos(#2)},{sin(#2)}) circle (0.2pt);
    }%}}}
    \newcommand{\myDrawArcB}[5]{%{{{
      % Parameter: radius , center , start , stop , color
      \draw[#5]
        ({cos( #2 )},{sin( #2 )})
        --
        ({cos( #2 ) * #1 },{sin( #2 ) * #1 });
      \draw[ultra thick,#5]
        ({cos( #3 ) * #1},{sin( #3 ) * #1})
        arc
        ({ #3 }:{ #4 }:#1);
      \draw[dotted,#5]
        ({cos( #3 ) * #1},{sin( #3 ) * #1})
        --
        ({cos( #3 )},{sin( #3 )});
      \draw[dotted,#5]
        ({cos( #4 ) * #1},{sin( #4 ) * #1})
        --
        ({cos( #4 )},{sin( #4 )});
      \filldraw[white] ({cos(#2)},{sin(#2)}) circle (0.5pt);
      \filldraw[red] ({cos(#2)},{sin(#2)}) circle (0.2pt);
    }%}}}

    \node (zero) at (0,0) {};
    \draw (zero) circle (1cm);

    %%%%%%%%%%%%%%%%%%%%%%%%%%%%%%%%%%%%%%%%%%%%%%%%%%%%%%%%%%%%%%%%%%%%%%%%%
    %% Inner:
    \foreach \n/\mycol/\baseR in {\kOne/orange/0.8
                                 ,\kTwo/purple/0.9}
    {%{{{
      \pgfmathsetmacro\wdth{180/\n}
      \foreach \i in {1,2,...,\n}}}

    %%%%%%%%%%%%%%%%%%%%%%%%%%%%%%%%%%%%%%%%%%%%%%%%%%%%%%%%%%%%%%%%%%%%%%%%%
    %% Outer:
    \def\mycol{brown}
    \pgfmathsetmacro{\wdth}{180/\kTwo}
    \foreach \i in {1,2,...,\kTwo}{%
      \pgfmathsetmacro\r{{1.05 + mod(\i+1,2)*0.05}}
      \pgfmathsetmacro\angl{\i* \wdth}
      \myDrawArcA{\r}{\angl}{\wdth}{\mycol}

      \pgfmathsetmacro\r{{1.05 + mod(\i + mod(\kTwo+1,2),2)*0.05}}
      \pgfmathsetmacro\angl{\i* \wdth+180}
      \myDrawArcA{\r}{\angl}{\wdth}{\mycol}
    }
    \foreach \i in {1,2,...,\kOne}{%
      \foreach \j in {1,...,\kTwo}{%
        \pgfmathparse{\i/\kOne <= \j/\kTwo ? 0 : 1}
        \ifnum\pgfmathresult=0{%
          \pgfmathparse{\i/\kOne < \j/\kTwo ? 0 : 1}
          \ifnum\pgfmathresult=0{%
            \pgfmathsetmacro\r{{1.15 + mod(\i+1,2)*0.05}}
            \pgfmathsetmacro\center{\i/\kOne*180}
            \pgfmathsetmacro\start{(\i-0.5)*180/\kOne > (\j-1.5)*180/\kTwo
              ? (\i-0.5)*180/\kOne : (\j-1.5)*180/\kTwo}
            \pgfmathsetmacro\stop{(\i+0.5)*180/\kOne > (\j+0.5)*180/\kTwo
              ? (\j+0.5)*180/\kTwo : (\i+0.5)*180/\kOne}
            \myDrawArcB{\r}{\center}{\start}{\stop}{\mycol}

            \pgfmathsetmacro\r{{1.15 + mod(\i,2)*0.05}}
            \pgfmathsetmacro\center{\center+180}
            \pgfmathsetmacro\start{\start+180}
            \pgfmathsetmacro\stop{\stop+180}
            \myDrawArcB{\r}{\center}{\start}{\stop}{\mycol}
          }\fi
          \breakforeach
        }\fi
      }
    }

    \fill[white] (zero) circle (1.5pt);
    \fill (zero) circle (.7pt);
  \end{tikzpicture} %}}}
  \caption{The adequate coverings for an example with $\cK=\{\kOne,\kTwo\}$ and
    $\A=\left\{ \frac{j\cdot\pi}{k}\mid k\in\cK\text{, } j\in\N \right\}$.
    The anti-Stokes directions are marked by the \textcolor{red}{red} dots.
    The arcs of $\dot\cU^{\kOne}=\dot\cU^{\leq\kOne}$ are
    \textcolor{orange}{orange}, the arcs of $\dot\cU^{\kTwo}$ are
    \textcolor{purple}{purple} and the arcs of $\dot\cU^{\leq\kTwo}=\dot\cU$
    are \textcolor{brown}{brown}.
  }\label{fig:adequateCovering}
\end{figure}%}}}

\begin{rem}
  The coverings $\cU^{k}$, $\cU^{\leq k}$ and $\cU^{<k}$ depend only on
  $\cQ(A^0)$. Hence they depend only on the determining polynomials.
\end{rem}
It is obvious, that for every $k\in\cK$ the covering $\cU^{\leq k}$ refines
$\cU^{k}$ and $\cU^{<k}$.
Furthermore are the coverings defined, such that they satisfy the
condition in Proposition~\ref{prop:adeqCovCondition}.
Thus the first property in the following proposition is satisfied. The other
two can be found at \cite[Prop.II.3.1 (iv)]{Loday1994}.
\begin{prop}\label{prop:adequateProperties}
  \marginnote{\cite[Prop.II.3.1]{Loday1994}}
  \PROBLEM[Was bedeutet das?]
  Let $k\in\cK$, then
  \begin{enumerate}
    \item the coverings $\cU^{k}$, $\cU^{\leq k}$ and $\cU^{<k}$ are adequate
      to $\Lambda^k(A^0)$, $\Lambda^{\leq k}(A^0)$ and $\Lambda^{<k}(A^0)$, respectively;
    \item there exists no $0$-cochain in $\Lambda^k(A^0)$ on $\cU^k$;
    \item on $\cU^{\leq k}$ there is no $0$-cochain in $\Lambda^{\leq k}(A^0)$
      of level $k$, i.e.\ all $0$-cochains of $\Lambda^{\leq k}(A^0)$ belong to
      $\Lambda^{<k}(A^0)$.
  \end{enumerate}
\end{prop}
To have a shorter notation, we denote the product
$\prod_{\alpha\in\A^\star}\Gamma(\dot U_\alpha^\star;\Lambda^\star(A^0))$ by
$\Gamma(\dot U^\star;\Lambda^\star(A^0))$ for every
$\star\in\{k,<k,\leq k,\dots\}$.

%%%%%%%%%%%%%%%%%%%%%%%%%%%%%%%%%%%%%%%%%%%%%%%%%%%%%%%%%%%%%%%%%%%%%%%%%%%%%%%
\subsubsection{The case of a unique level}
\marginnote{\cite[II.3.2]{Loday1994}}
First we will proof Theorem~\ref{thm:mainThm2} in the case of a unique level.
This means that
\begin{itemize}
  \item either $\Lambda(A^0)$ has only one level $k$, thus
    \begin{itemize}
      \item $\Lambda(A^0)=\Lambda^k(A^0)$ and
      \item $\Sto_\theta(A^0)=\Sto_\theta^k(A^0)$ for every $\theta$,
    \end{itemize}
  \item or we restrict to a given level $k\in\cK$.
\end{itemize}
\begin{lem}
  Let $k\in\cK$.
  The morphism $h$ from Theorem~\ref{thm:mainThm2} is in the case of an unique
  level build as
  \[ \begin{tikzcd}[row sep=0cm]
    \underset{\alpha\in\A}\prod\Sto_\alpha^k(A^0)
    \rar{i^k}
    & \Gamma(\dot\cU^k;\Lambda^k(A^0))
    \rar{s^k}
    & H^1(S^1;\Lambda^k(A^0))
    \\
    \text{\rotatebox[origin=c]{-90}{$=$}}
    \tikzmark{e1}
    &&
    \tikzmark{e2}
    \text{\rotatebox[origin=c]{-90}{$=$}}
    \\
    \underset{\alpha\in\A}\prod\Sto_\alpha(A^0)
    \arrow{rr}{h}
    && H^1(S^1;\Lambda(A^0))
  \end{tikzcd} \]
  \begin{flushright}
    \tikzmarkc{n1}{blue} only in the single leveled case
    \begin{tikzpicture}[remember picture,overlay]
      \draw[->,blue!50!white,thick] (n1) to[out=180,in=0] (e1);
      \draw[->,blue!50!white,thick] (n1) to[out=180,in=180,distance=3cm] (e2);
    \end{tikzpicture}
  \end{flushright}
  from
  \begin{itemize}
    \item the canonical injective map
      \[ \begin{tikzcd}
        i^k:\underset{\alpha\in\A}\prod\Sto_\alpha^k(A^0) \rar&
        \Gamma(\dot\cU^k;\Lambda^k(A^0)) \,,
      \end{tikzcd} \]
      i.e.\ the map which is the canonical extension of germs to their natural
      arc of definition, and
    \item the quotient map
      \[ \begin{tikzcd}
        s^k:\Gamma(\dot\cU^k;\Lambda^k(A^0)) \rar& H^1(S^1;\Lambda^k(A^0))
      \end{tikzcd} \]
  \end{itemize}
  which are both isomorphisms.
\end{lem}
\begin{proof}
  \begin{enumerate}
    \item The map
      \[ \begin{tikzcd}
        i^k: \underset{\alpha\in\A}\prod\Sto_\alpha^k(A^0)
        \rar &
        \underset{\Gamma(\dot\cU^k;\Lambda^k(A^0))}{%
          \underset{\text{\rotatebox[origin=c]{-90}{$=$}}}{%
            \underbrace{%
              \prod_{\alpha\in\A^k}\Gamma(\dot\cU_\alpha^k;\Lambda^k(A^0))
            }
        }}
      \end{tikzcd} \]
      is welldefined, since the sections of $\Lambda^k(A^0)$ are solution of
      the system $[A^0,A^0]$ and it is very well known from the theory of
      differential equations\TODO[source?] that an element
      $f_\alpha\in\Gamma(\dot\cU_\alpha^k;\Lambda^k(A^0))$ is uniquely
      determined as the extension of its germ at some point $\alpha$.

      It is also a group isomorphism (cf.~\cite{Loday1994}).

      \PROBLEM[$\Sto_\alpha^k(A^0)\subsetneq\Lambda_\alpha^k(A^0)$ SEE bjl p.72]
      \begin{comment}
        It is a group isomorphism, since
        \\Problems:
        \begin{itemize}
          \item Show that a element of $\Lambda_\alpha^k(A^0)$ is extensionable
            to the arc $U_\alpha\in\dot\cU_\alpha^k$ if and only if it has
            maximal decay in direction $\alpha$.
          \item \PROBLEM[$\Sto_\alpha^k(A^0)\subsetneq\Lambda_\alpha^k(A^0)$]
            and thus
            \[
              i^k:
              \underset{\alpha\in\A}\prod\Sto_\alpha^k(A^0)
              \subsetneq
              \underset{\alpha\in\A}\prod\Lambda_\alpha^k(A^0)
              \to
              \prod_{\alpha\in\A^k}\Gamma(\dot\cU_\alpha^k;\Lambda^k(A^0))
            \]
        \end{itemize}
        IDEAS:
        \begin{itemize}
          \item Are sections fully determined by there germs at some points?
          \item $U_\alpha^k$ is the largest arc, to contain no corresponding
            Stokes ray
            \begin{itemize}
              \item Then might \cite[Lemma 1]{BJL1979Birkhoff} on p.\ 73 help
            \end{itemize}
          \item Every arc $U\in\dot\cU_\alpha^k$ has the width $\frac{\pi}{k}$
            and is delimited by Stokes directions.
          \item \textbf{See \cite{babbitt1989local}}
          \item See \cite[375]{Martinet1991} and
            \textbf{\cite[Def.5 on 372]{Martinet1991}}
          \item \textbf{\textcolor{green}{\cite[72]{babbitt1989local}}}
          \item or maybe \cite{wasow2002asymptotic}
        \end{itemize}
      \end{comment}
    \item The second map
      \[ \begin{tikzcd}
        s^k:
        \overset{\Gamma(\dot\cU^k;\Lambda^k(A^0))}{%
          \overset{\text{\rotatebox[origin=c]{-90}{$=$}}}{%
            \overbrace{%
              \prod_{\alpha\in\A^k}\Gamma(\dot\cU_\alpha^k;\Lambda^k(A^0))
            }
        }}
        \rar &
        H^1(S^1;\Lambda^k(A^0))
      \end{tikzcd} \]
      is a bijection, \rewrite{since} from
      Proposition~\ref{prop:adequateProperties} we know that it is
      \begin{itemize}
        \item \textbf{surjective}, \rewrite{since} $\cU^k$ is adequate to
          $\Lambda^k(A^0)$ and
        \item \textbf{injective}, \rewrite{since} on $\cU^k$ there is no
          $0$-cochain in $\Lambda^k(A^0)$.
      \end{itemize}
  \end{enumerate}
  \PROBLEM[Show that this is the correct $h$]
  \PROBLEM[Naturality?]
\end{proof}

%%%%%%%%%%%%%%%%%%%%%%%%%%%%%%%%%%%%%%%%%%%%%%%%%%%%%%%%%%%%%%%%%%%%%%%%%%%%%%%
\subsubsection{The case of several levels}
\rewrite{In the proof of the case of several levels, we will still use}
Loday-Richaud's paper~\cite{Loday1994} as reference.
\begin{defn}\label{defn:firstSetOfInlusions}
  Here we want to define a \emph{product map of cocycles}
  $\mathfrak{S}^{\leq k}$.
  This map will be composed from the following injective maps:
  \begin{enumerate}
    \item The first map is defined as
      \begin{align*}
        \sigma^k:\Gamma(\dot\cU^k;\Lambda^k(A^0))
        &\longrightarrow \Gamma(\dot\cU^{\leq k};\Lambda^{\leq k}(A^0))
      \\\dot f=(\dot f_\alpha)_{\alpha\in\A^k}
        &\longmapsto (\dot G_\alpha)_{\alpha\in\A^{\leq k}}
      \end{align*}
      where
      \[
        \dot G_\alpha=\begin{cases}
          \dot f_\alpha \text{~restricted to } \dot U_\alpha^{\leq k}
          \text{~and seen as being in } \Lambda^{\leq k}(A^0)
          & \text{~when } \alpha\in\A^k
        \\\id \text{~(the identity) }
          & \text{~when } \alpha\notin\A^k
        \end{cases}
      \]
    \item and the second map
      \begin{align*}
        \sigma^{<k}:\Gamma(\dot\cU^{<k};\Lambda^{<k}(A^0))
        &\longrightarrow \Gamma(\dot\cU^{\leq k};\Lambda^{\leq k}(A^0))
      \\\dot f=(\dot f_\alpha)_{\alpha\in\A^{<k}}
        &\longmapsto (\dot F_\alpha)_{\alpha\in\A^{\leq k}}
      \end{align*}
      is defined, in a similar way, as
      \[
        \dot F_\alpha=\begin{cases}
          \dot f_\alpha \text{~restricted to } \dot U_\alpha^{\leq k}
          \text{~and seen as being in } \Lambda^{\leq k}(A^0)
          & \text{~when } \alpha\in\A^{<k}
        \\\id \text{~(the identity) }
          & \text{~when } \alpha\notin\A^{<k}
        \end{cases}
      \]
  \end{enumerate}
  Thus we can define
  \begin{align*}
    \mathfrak{S}^{\leq k}:
    \Gamma(\dot\cU^{<k};\Lambda^{<k}(A^0))
    \times
    \Gamma(\dot\cU^{k};\Lambda^{k}(A^0))
    &\longrightarrow
    \Gamma(\dot\cU^{\leq k};\Lambda^{\leq k}(A^0))
  \\(\dot f, \dot g)
    &\longmapsto
    (\dot F_\alpha\dot G_\alpha)_{\alpha\in\A^{\leq k}}
  \end{align*}
  where $(\dot F_\alpha)_{\alpha\in\A^{\leq k}}=\sigma^{<k}(\dot f)$ and
  $(\dot G_\alpha)_{\alpha\in\A^{\leq k}}=\sigma^k(\dot g)$ are defined as
  above.
  \begin{s-rem}
    This map $\mathfrak{S}^{\leq k}$ is injective, since injectivity for germs
    implies injectivity for sections.
  \end{s-rem}
\end{defn}

\begin{lem}
  \marginnote{\cite[Lem.II.3.3]{Loday1994}}
  Let $k\in\cK$.
  \begin{enumerate}
    \item If the cocycles $\mathfrak{S}^{\leq k}(\dot f,\dot g)$ and
      $\mathfrak{S}^{\leq k}(\dot f',\dot g')$ are cohomologous in
      $\Gamma(\dot\cU^{\leq k};\Lambda^{\leq k}(A^0))$
      then $\dot f$ and $\dot f'$ are cohomologous in
      $\Gamma(\dot\cU^{<k};\Lambda^{<k}(A^0))$.
    \item Any cocycle in $\Gamma(\dot\cU^{\leq k},\Lambda^{\leq k}(A^0))$ is
      cohomologous to a cocycle in the range of $\mathfrak{S}^{\leq k}$.
  \end{enumerate}
\end{lem}
\begin{proof}
  \begin{enumerate}
    \item Denote by $\alpha^+$ the nearest anti-Stokes direction in
      $\A^{\leq k}$ on the right\footnote{In clockwise direction.} of $\alpha$.
      The cocycles $\mathfrak{S}^{\leq k}(\dot f,\dot g)$ and
      $\mathfrak{S}^{\leq k}(\dot f',\dot g')$ are cohomologous if and only
      if there is a $0$-cochain $c=(c_\alpha)_{\alpha\in\A^{\leq k}}
      \in\Gamma(\cU^{\leq k},\Lambda^{\leq k}(A^0)$ such that
      \begin{equation}\label{eq:UniqueString:urdtindfgupndtcn}
        \dot F_\alpha\dot G_\alpha =
        c_\alpha^{-1}\dot F_\alpha'\dot G_\alpha'c_{\alpha^+}
      \end{equation}
      for every $\alpha\in\A$. From Proposition~\ref{prop:adequateProperties}
      follows, that $c$ is with values in $\Lambda^{<k}(A^0)$.
      The fact that $\Lambda^k(A^0)$ is normal in $\Lambda^{\leq k}(A^0)$ in
      Proposition~\ref{prop:PropertiesOfStokesSheafSplitting}, can be used to
      see that $ c_{\alpha^+}^{-1}G_\alpha'c_{\alpha^+}
      \in\Gamma(\cU^{\leq k};\Lambda^{k}(A^0))$.
      Thus, we rewrite the relation (\ref{eq:UniqueString:urdtindfgupndtcn}) to
      \[
        \dot F_\alpha\dot G_\alpha =
        (c_\alpha^{-1}\dot F_\alpha'c_{\alpha^+})
        (c_{\alpha^+}^{-1}G_\alpha'c_{\alpha^+})
        \,,\qquad\text{~for~} \alpha\in\A^{\leq k} \,.
      \]
      Since Corollary~\ref{cor:factorStokesGerms} tells us, that the
      factorization into the factors of the semidirect product are unique, we
      get for all $\alpha\in\A^k$
      \[
        \dot F_\alpha=c_\alpha^{-1}\dot F_\alpha'c_{\alpha^+}
        \qquad \text{~and~} \qquad
        \dot G_\alpha=c_{\alpha^+}^{-1}\dot G_\alpha'c_{\alpha^+}.
      \]
      The former relation implies that $(\dot F_\alpha)$ and $(\dot F_\alpha')$
      are cohomologous with values in $\Lambda^{<k}(A^0)$ on $\cU^{\leq k}$.
      Since $\cU^{<k}$ is already adequate to $\Lambda^{<k}(A^0)$, \rewrite{are
      $(\dot F_\alpha)$ and $(\dot F_\alpha')$} already on $\cU^{<k}$,
      i.e.\ in $\Gamma(\dot\cU^{<k};\Lambda^{<k}(A^0))$, cohomologous.
    \item The proof of part 2.\ (together with a proof of part 1.) can be
      found in Loday-Richaud's paper \cite[Proof of Lem.II.3.3]{Loday1994}.
  \end{enumerate}
\end{proof}
Let $k\in\cK$ and $k'=\max\{k'\in\cK\mid k'<k\}$. We then know by definition
that $\cU^{<k}=\cU^{\leq k'}$ as well as
$\Lambda^{< k}(A^0)=\Lambda^{\leq k'}(A^0)$ and thus
$\Gamma(\dot\cU^{< k};\Lambda^{< k}(A^0))=
\Gamma(\dot\cU^{\leq k'};\Lambda^{\leq k'}(A^0))$ and obtain the following
proposition.
\begin{prop}\label{prop:theMapTau}
  By applying $\mathfrak{S}^{\leq k}$ successively for different $k$'s
  in decending order, one obtains the \emph{product map of single leveled
  cocycles $\tau$} in the following way
  \[ \begin{tikzcd}[column sep=1.4cm,row sep=.9cm]
      \underset{\tikzmark{e1}}{\underbrace{%
        \Gamma(\dot\cU^{<k_r};\Lambda^{<k_r}(A^0))}}
      \times
      \Gamma(\dot\cU^{k_r};\Lambda^{k_r}(A^0))
      \rar{\mathfrak{S}^{\leq k_r}}&
      \overset{\Gamma(\dot\cU;\Lambda(A^0))}{%
        \overset{\text{\rotatebox[origin=c]{-90}{$=$}}}{%
          \Gamma(\dot\cU^{\leq k_r};\Lambda^{\leq k_r}(A^0))
      }}
      \\\underset{\tikzmark{e2}}{\underbrace{%i
        \Gamma(\dot\cU^{<k_{r-1}};\Lambda^{<k_{r-1}}(A^0))}}
      \times
      \Gamma(\dot\cU^{k_{r-1}};\Lambda^{k_{r-1}}(A^0))
      \rar{\mathfrak{S}^{\leq k_{r-1}}}&
      \overset{\tikzmark{n1}}{%
        \Gamma(\dot\cU^{\leq k_{r-1}};\Lambda^{\leq k_{r-1}}(A^0))}
      \\\hspace{6cm}\cdots \rar{\mathfrak{S}^{\leq k_{r-2}}}&
      \overset{\tikzmark{n2}}{%
        \Gamma(\dot\cU^{\leq k_{r-2}};\Lambda^{\leq k_{r-2}}(A^0))}
      \\\underset{\tikzmark{eEND}}{\underbrace{%
        \Gamma(\dot\cU^{<k_3};\Lambda^{<k_3}(A^0))}}
      \times
      \Gamma(\dot\cU^{k_3};\Lambda^{k_3}(A^0))
      \rar{\mathfrak{S}^{\leq k_3}}&
      \cdots\hspace{3cm}
      \\
      \Gamma(\dot\cU^{k_1};\Lambda^{k_1}(A^0))
      \times
      \Gamma(\dot\cU^{k_2};\Lambda^{k_2}(A^0))
      \rar{\mathfrak{S}^{\leq k_2}}&
      \overset{\tikzmark{nEND}}{%
        \Gamma(\dot\cU^{\leq k_2};\Lambda^{\leq k_2}(A^0))}
  \end{tikzcd} \]
  \begin{tikzpicture}[remember picture,overlay]
    \draw[<-] (n1) to[out=90,in=270] node[midway,fill=white]{$=$}
      ([yshift=.3em]e1);
    \draw[<-] (n2) to[out=90,in=270] node[midway,fill=white]{$=$}
      ([yshift=.3em]e2);
    \draw[<-] (nEND) to[out=90,in=270] node[midway,fill=white]{$=$}
      ([yshift=.3em]eEND);
  \end{tikzpicture}
  which can be written in the following compact form
  \begin{align*}
    \tau:\prod_{k\in\cK}\Gamma(\dot \cU^k;\Lambda^k(A^0))
    &\longrightarrow
    \Gamma(\dot\cU;\Lambda(A^0))
  \\(\dot f^k)_{k\in\cK}
    &\longmapsto
    \prod_{k\in\cK}\tau^k(\dot f^k)
  \end{align*}
  where the product is following an ascending order of levels and the maps
  $\tau_k$ are defined as
  \[ \begin{tikzcd}[row sep=0cm]
    \tau^k:\Gamma(\dot\cU^k;\Lambda^k(A^0))
    \arrow{r}{\sigma^k}&
    \Gamma(\dot\cU^{\leq k};\Lambda^{\leq k}(A^0))
    \rar &
    \Gamma(\dot\cU;\Lambda(A^0))
  \\~~~(\dot f_\alpha)_{\alpha\in\A^k}
    \arrow[|->]{rr}
    &&
    (\dot G_\alpha)_{\alpha\in\A}
  \end{tikzcd} \]
  with
  \[
    \dot G_\alpha=\begin{cases}
      \dot f_\alpha \text{~restricted to } \dot U_\alpha
      \text{~and seen as being in } \Lambda(A^0)
      & \text{~when } \alpha\in\A^k
    \\\id \text{~(the identity on $\dot U_\alpha$) }
      & \text{~when } \alpha\notin\A^k
    \end{cases}
  \]
  The defined map $\tau$ is clearly injective and it can be extended to an
  arbitrary order of levels (cf.\ Remark~\cite[Rem.II.3.5]{Loday1994}).
\end{prop}
\begin{comment}
  \begin{defn}\label{defn:theMapTau}
    Define the injective map
    \begin{align*}
      \tau^k:\Gamma(\dot\cU^k;\Lambda^k(A^0))
      &\longrightarrow \Gamma(\dot\cU;\Lambda(A^0))
    \\\dot f=(\dot f_\alpha)_{\alpha\in\A^k}
      &\longmapsto
      (\dot F_\alpha)_{\alpha\in\A}
    \end{align*}
    where
    \[
      \dot F_\alpha=\begin{cases}
        \dot f_\alpha \text{~restricted to } \dot U_\alpha
        \text{~and seen as being in } \Lambda(A^0)
        & \text{~when } \alpha\in\A^k
      \\\id \text{~(the identity on $\dot U_\alpha$) }
        & \text{~when } \alpha\notin\A^k
      \end{cases}
    \]
    \marginnote{\cite[Prop.II.3.4]{Loday1994}}
    The \emph{product map of single-leveled cocycles} is then defined as
    \begin{align*}
      \tau:\prod_{k\in\cK}\Gamma(\dot \cU^k;\Lambda^k(A^0))
      &\longrightarrow
      \Gamma(\dot\cU;\Lambda(A^0))
    \\(\dot f^k)_{k\in\cK}
      &\longmapsto
      \prod_{k\in\cK}\tau^k(\dot f^k)
    \end{align*}
    following an ascending order of levels.
    \begin{s-rem}
      The map $\tau$
      \begin{enumerate}
        \item is injective since it is composed from $\sigma^k$ and the clearly
          injetive mapping
          \begin{align*}
            \Gamma(\dot\cU^{\leq k};\Lambda^{\leq k}(A^0))
            &\longrightarrow
            \Gamma(\dot\cU;\Lambda(A^0))
          \\(\dot F_\alpha)_{\alpha\in\A^{\leq k}}
            &\longmapsto
            (\dot F'_\alpha)_{\alpha\in\A}
          \end{align*}
          where
          \[
            \dot F'_\alpha=\begin{cases}
              \dot F_\alpha \text{~restricted to } \dot U_\alpha
              \text{~and seen as being in } \Lambda(A^0)
              & \text{~when } \alpha\in\A^{\leq k}
              \\\id \text{~(the identity on $\dot U_\alpha$) }
              & \text{~when } \alpha\notin\A^{\leq k}
            \end{cases}
          \]
          and
        \item it can be extended to an arbitrary order of levels
          (cf.\ Remark~\cite[Rem.II.3.5]{Loday1994}).
      \end{enumerate}
    \end{s-rem}
  \end{defn}
\end{comment}
\begin{cor}
  The product map of single-leveled cocycles $\tau$ induces on the cohomology
  a bijective and natural map
  \[ \begin{tikzcd}
    \cT:
    \underset{\underset{k\in\cK}\prod H^1(S^1;\Lambda^k(A^0))}{%
      \underset{\text{\rotatebox[origin=c]{-90}{$\cong$}}}{%
        \prod_{k\in\cK}\Gamma(\dot\cU^k;\Lambda^k(A^0))}}
    \rar&
    \underset{H^1(S^1;\Lambda(A^0))}{%
      \underset{\text{\rotatebox[origin=c]{-90}{$\cong$}}}{%
        H^1(\cU;\Lambda(A^0))}}\,.
  \end{tikzcd} \]
\end{cor}

%%%%%%%%%%%%%%%%%%%%%%%%%%%%%%%%%%%%%%%%%%%%%%%%%%%%%%%%%%%%%%%%%%%%%%%%%%%%%%%
\paragraph{Composing functions to obtain $h$}
We have the ingredients to define the function $h$ from
Theorem~\ref{thm:mainThm2} by composition of already bijective maps.
\begin{proof}[Proof of Theorem~\ref{thm:mainThm2}]
  Let $i_\alpha:\Sto_\alpha(A^0)\to\prod_{k\in\cK}\Sto_\alpha^k(A^0)$ be the
  map which corresponds to the filtration from
  Proposition~\ref{prop:filtrationOfStokesGroup} and
  denote the composition
  \[ \begin{tikzcd}[column sep=1.8cm,row sep=0]
      \displaystyle \prod_{\alpha\in\A}\Sto_\alpha(A^0)
      \rar{\prod_{\alpha\in\A}i_\alpha}
      \arrow[ddrr, out=270,in=200,"\mathfrak{T}"]
      &
      \displaystyle \prod_{\alpha\in\A}\prod_{k\in\cK}\Sto_\alpha^k(A^0)
    \\&\text{\rotatebox[origin=c]{-90}{$\equiv$}}
    \\&\displaystyle \prod_{k\in\cK}\prod_{\alpha\in\A}\Sto_\alpha^k(A^0)
      \rar{\prod_{k\in\cK}i^k}&
      \displaystyle \prod_{k\in\cK}\Gamma(\dot\cU^k;\Lambda^k(A^0))
  \end{tikzcd} \]
  by $\mathfrak{T}$. The bijection $h$ is then obtained as
  \[
    \cT\circ\mathfrak{T}: \prod_{\alpha\in\A}\Sto_\alpha(A^0)
    \longrightarrow H^1(\cU;\Lambda(A^0)) \,.
  \]
  \PROBLEM[naturality (is obvious?)]
  \PROBLEM[Show that this is the correct $h$]
\end{proof}

%%%%%%%%%%%%%%%%%%%%%%%%%%%%%%%%%%%%%%%%%%%%%%%%%%%%%%%%%%%%%%%%%%%%%%%%%%%%%%%
\subsection{Some exemplary calculations}\label{sec:WhichInformationIsNeeded}
Here we want to discuss, which information is required to describe the Stokes
cocycle corresponding to a multileveled system in more depth.
We will look at a sigle-leveled system corresponding to a normal form
$A^0\in \GL_3(\C(\!\{t\}\!))$ with exactly $2$ levels and will apply the
techniques developed in the previous sections in an rather explicit way.

Let $A^0$ be a normal form with dimension $n=3$ and two levels
$\cK=\{k_1<k_2\}$, which satisfies that there is at least one anti-Stokes
direction $\theta$ which is beared by both levels.
Let $q_j(t^{-1})$ be the determining polynomials and let $k_{jl}$ be the
degrees of $(q_j-q_l)(t^{-1})$.
Up to permutation, we know that in our case are the leading terms of
$(q_1-q_2)(t^{-1})$ and $(q_1-q_3)(t^{-1})$ equal and thus
\begin{itemize}
  \item up to permutation is $k_2=k_{1,2}=k_{1,3}$ and $k_1=k_{2,3}$, i.e.\ the
    larger degree appears twice, and
    \begin{comment}
      let $q_1,q_2,q_3$ be polynomials, such that
      \[
        \deg(q_1-q_2) =: k_2 > k_1 := \deg(q_2-q_3)
      \]
      then is the degree of $q_1-q_3$ given by
      \begin{itemize}
        \item[\textbf{case 1}] $\deg(q_1)<k_2$: then is $\deg(q_2)=k_2$ and
          thus $\deg(q_3)=k_2$.
        \item[\textbf{case 2}] $\deg(q_2)<k_2$: then is $\deg(q_1)=k_2$ and
          $\deg(q_3)\leq\deg(q_2)$.
        \item[\textbf{case 3}] $\deg(q_1)=\deg(q_2)=k_2$: thus follows that the
          leading term of $q_1$ and $q_2$ are different.
          \begin{itemize}
            \item[\textbf{subcase 3.a}] $\deg(q_3)<k_2$: everything is clear
            \item[\textbf{subcase 3.b}] $\deg(q_3)=k_2$: here has the leading
              term of $q_3$ be equal to them from $q_1$ to satisfy that
              $k_2>\deg(q_2-q_3)$.
          \end{itemize}
      \end{itemize}
      From this follows that $\deg(q_1-q_3)=k_2$.
    \end{comment}
  % \item $q_1 \underset{\alpha,\max}{\prec} q_2$
  %   \Leftrightarrow{}
  %   $q_1 \underset{\alpha,\max}{\prec} q_3$
  %   \rewrite{respectively}
  %   $q_2 \underset{\alpha,\max}{\prec} q_1$
  %   \Leftrightarrow{}
  %   $q_3 \underset{\alpha,\max}{\prec} q_1$
  \item $q_1\myrel{\alpha}q_2$ (resp.~$q_2\myrel{\alpha}q_1$) if and only if
    $q_1\myrel{\alpha}q_3$ (resp.~$q_3\myrel{\alpha}q_1$) and thus do they
    determine the same anti-Stokes directions.
\end{itemize}
The set of all anti-Stokes directions is then given as
\[
  \A=\left\{\theta+\frac{\pi}{k}\cdot j\mid k\in\cK\text{, }j\in\N\right\}
    % =:\{
    %   \underset{\theta}{%
    %     \underset{\text{\rotatebox[origin=c]{-90}{$=$}}}{%
    %       \alpha_1
    %   }}
    % ,\dots,\alpha_\nu\}\,.
\]
Denote by $\cY_0(t)$ a normal solution of $[A^0]$.

Let us start by looking at a single germ in depth.
The Proposition~\ref{prop:representation} states that every Stokes germ
$\phi_\alpha$ can be written as its matrix representation conjugated by the
normal solution, i.e.\ as $\phi_\alpha=\cY_{0}C_{\phi_\alpha}\cY_{0}^{-1}
=\rho_{\alpha}^{-1}(C_{\phi_\alpha})$.

Look at an example in which we will demonstrate, from which relations on the
determining polynomials which restriction on the form of the Stokes matrices
arise.
\begin{exmp}
  Let $\alpha\in\A$ be an anti-Stokes direction.
  From the definition of $\SSto_\alpha(A^0)$ (cf.\
  Definition~\ref{defn:groupOfFaithfullReps}) we know that, if one has
  $q_1\myrel{\alpha}q_2$, the Stokes matrix has the form
  \[
    \begin{pmatrix}
      1 & \text{\boldmath$c_1$} & \star
    \\\text{\boldmath$0$} & 1 & \star
    \\\star & \star & 1
    \end{pmatrix}
  \]
  where $c_j\in\C$ and $\star\in\C$.

  We have seen that $q_1\myrel{\alpha}q_2$ \Rightarrow{}
  $q_1\myrel{\alpha}q_3$ thus the representation has the
  \rewrite{form}
  \[
    \begin{pmatrix}
      1 & c_1 & \text{\boldmath$c_2$}
    \\0 & 1 & \star
    \\\text{\boldmath$0$} & \star & 1
    \end{pmatrix}
  \]
  and if we also know that neither $q_2\myrel{\alpha}q_3$ nor
  $q_3\myrel{\alpha}q_2$ it has the \rewrite{form}
  \[
    \begin{pmatrix}
      1 & c_1 & c_2
    \\0 & 1 & \text{\boldmath$0$}
    \\0 & \text{\boldmath$0$} & 1
    \end{pmatrix}\,.
  \]
  \begin{comment}
    We also know that every matrix of this \rewrite{form} is a representation
    to some Stokes germ.
    Thus we have an isomorphism
    \begin{align*}
      \vartheta_\alpha:\C^2 &\longrightarrow \SSto_\alpha(A^0)
    \\(c_1,c_2)&\longmapsto
      \begin{pmatrix}
        1 & c_1 & c_2
      \\0 & 1 & 0
      \\0 & 0 & 1
      \end{pmatrix}
    \end{align*}
  \end{comment}
\end{exmp}
In fact, the following $9$ cases of Stokes matrices can arise:
\begin{center}
  \def\arraystretch{1.3}
  \setlength\tabcolsep{4mm}
  \begin{tabular}{r|c|c|c}
    & $q_2\myrel{\alpha}q_3$ & $q_3\myrel{\alpha}q_2$ & else
    \tabularnewline
    \hline
    $\substack{q_1\myrel{\alpha}q_2\\\text{and} \\q_1\myrel{\alpha}q_3}$
    & $\begin{pmatrix} 1 & c_2 & c_3 \\0 & 1 & c_1 \\0 & 0 & 1 \end{pmatrix}$
   \cellcolor{blue!15}
    & $\begin{pmatrix} 1 & c_2 & c_3 \\0 & 1 & 0 \\0 & c_1 & 1 \end{pmatrix}$
   \cellcolor{blue!15}
    & $\begin{pmatrix} 1 & c_2 & c_3 \\0 & 1 & 0 \\0 & 0 & 1 \end{pmatrix}$
   \cellcolor{green!15}
    \tabularnewline
    \hline
    $\substack{q_2\myrel{\alpha}q_1\\\text{and} \\q_3\myrel{\alpha}q_1}$
    & $\begin{pmatrix} 1 & 0 & 0 \\c_2' & 1 & c_1 \\c_3 & 0 & 1 \end{pmatrix}$
   \cellcolor{blue!15}
    & $\begin{pmatrix} 1 & 0 & 0 \\c_2 & 1 & 0 \\c_3' & c_1 & 1 \end{pmatrix}$
   \cellcolor{blue!15}
    & $\begin{pmatrix} 1 & 0 & 0 \\c_2 & 1 & 0 \\c_3 & 0 & 1 \end{pmatrix}$
   \cellcolor{green!15}
    \tabularnewline
    \hline
    else
    & $\begin{pmatrix} 1 & 0 & 0 \\0 & 1 & c_1 \\0 & 0 & 1 \end{pmatrix}$
   \cellcolor{purple!15}
    & $\begin{pmatrix} 1 & 0 & 0 \\0 & 1 & 0 \\0 & c_1 & 1 \end{pmatrix}$
   \cellcolor{purple!15}
    & $\begin{pmatrix} 1 & 0 & 0 \\0 & 1 & 0 \\0 & 0 & 1 \end{pmatrix}$
  \end{tabular}
\end{center}
In the \textcolor{blue!75!black}{blue} cases we have $\cK_\alpha=\cK$ and
$\C^3\overset{\vartheta_\alpha}{\underset{\cong}{\longrightarrow}}\SSto_\alpha(A^0)$.
In the \textcolor{green!50!black}{green} cases $\cK_\alpha=\{k_2\}$ and
$\C^2\overset{\vartheta_\alpha}{\underset{\cong}{\longrightarrow}}\SSto_\alpha(A^0)$
as well as in the \textcolor{purple!75!black}{purple} cases
$\cK_\alpha=\{k_1\}$ and
$\C^1\overset{\vartheta_\alpha}{\underset{\cong}{\longrightarrow}}\SSto_\alpha(A^0)$.
\comm{Thus, for every $\alpha\in\A$, we have an isomorphism
$\rho_{\alpha}^{-1}\circ\vartheta_\alpha$.}
We will replace $c_2'$ by $c_2+c_1c_3$ and $c_3'$ by $c_1c_2+c_3$ to be
consistent with the decomposition in the next part
(cf.\ Example~\ref{exmp:decompositionHere}).
\begin{cor}
  The morphism $\prod_{\alpha\in\A}\vartheta_\alpha$ is an isomorphism of
  pointed sets, which maps the element only containing zeros to
  \[
    (\id,\id,\dots,\id)\in\prod_{\alpha\in\A}\SSto_\alpha(A^0),
  \]
  which gets by $\left(\prod_{\alpha\in\A}\right)^{-1}\circ h$ mapped to the
  trivial cohomology class in $\St(A^0)$.
\end{cor}

%%%%%%%%%%%%%%%%%%%%%%%%%%%%%%%%%%%%%%%%%%%%%%%%%%%%%%%%%%%%%%%%%%%%%%%%%%%%%%%
In proposition~\ref{prop:filtrationOfStokesGroup} and especially
Remark~\ref{rem:filtrationOfStokesMats} we have defined a decomposition of the
Stokes group $\Sto_\alpha(A^0)$ in subgroups generated by $k$-germs for
$k\in\cK$.
In our case, we have at most two nontrivial factors. Especially is this
decomposition given by
\[
  \phi_\alpha=\phi_\alpha^{k_1} \phi_\alpha^{k_2}
  \overset{i_\alpha}\longmapsto
    \left(\phi_\alpha^{k_1},\phi_\alpha^{k_2}\right)
      \in\Sto_\alpha^{k_1}(A^0)\times\Sto_\alpha^{k_2}(A^0) \,,
\]
and $i_\alpha$ is the map, wich gives the factors of this factorization in
ascending order.
This decomposition, of a germ $\phi_\alpha$, is trivial if
$\#\cK(\phi_\alpha)\leq1$, thus the interesting cases are the
\textcolor{blue!75!black}{blue} cases.

\begin{exmp}\label{exmp:decompositionHere}
  Look at the example
  \[
    \vartheta_\alpha(c_1,c_2,c_3)=
    \cY_{0}
    \begin{pmatrix} 1 & 0 & 0 \\c_2 & 1 & 0 \\c_1c_2+c_3 & c_1 & 1 \end{pmatrix}
    \cY_{0}^{-1}
    =\phi_\alpha
    \,.
  \]
  According to Remark~\ref{rem:algFactorization} the factor
  $\phi_\alpha^{k_1}\in\Sto_\alpha^{k_1}(A^0)$, is given by
  \[
    \phi_\alpha^{k_1}=
    \cY_{0}
    \begin{pmatrix}
      1 & 0 & 0
    \\\text{\boldmath $0$} & 1 & 0
    \\\text{\boldmath $0$} & c_1 & 1
    \end{pmatrix}
    \cY_{0}^{-1}
    \,.
  \]
  The other factor $\phi_\alpha^{k_2}$ is then obtained as
  \begin{align*}
    \phi_\alpha^{k_2}&=
    \left(\phi_\alpha^{k_1}\right)^{-1}
    \phi_\alpha^{k_2}
  \\&=\cY_{0}
    \begin{pmatrix}
      1     & 0    & 0
    \\0     & 1    & 0
    \\0     & -c_1 & 1
    \end{pmatrix}
    \underset{=\id}{\underbrace{%
        \cY_{0}^{-1}
        \cY_{0}
    }}
    \begin{pmatrix} 1 & 0 & 0 \\c_2 & 1 & 0 \\c_1c_2+c_3 & c_1 & 1 \end{pmatrix}
    \cY_{0}^{-1}
  \\&=\cY_{0}
    \begin{pmatrix}
      1     & 0 & 0
    \\c_2     & 1          & 0
    \\c_3     & 0          & 1
    \end{pmatrix}
    \cY_{0}^{-1}
    \,.
  \end{align*}
\end{exmp}
The four nontrivial decomposition in our situation, are given by:
\begin{enumerate}
  \item $\begin{pmatrix} 1 & 0 & 0 \\0 & 1 & c_1 \\0 & 0 & 1 \end{pmatrix}
  \cdot\begin{pmatrix} 1 & c_2 & c_3 \\0 & 1 & 0 \\0 & 0 & 1 \end{pmatrix}=
  \begin{pmatrix} 1 & c_2 & c_3 \\0 & 1 & c_1 \\0 & 0 & 1 \end{pmatrix}$
  \item $\begin{pmatrix} 1 & 0 & 0 \\0 & 1 & 0 \\0 & c_1 & 1 \end{pmatrix}
  \cdot\begin{pmatrix} 1 & c_2 & c_3 \\0 & 1 & 0 \\0 & 0 & 1 \end{pmatrix}=
  \begin{pmatrix} 1 & c_2 & c_3 \\0 & 1 & 0 \\0 & c_1 & 1 \end{pmatrix}$
  \item $\begin{pmatrix} 1 & 0 & 0 \\0 & 1 & c_1 \\0 & 0 & 1 \end{pmatrix}
  \cdot\begin{pmatrix} 1 & 0 & 0 \\c_2 & 1 & 0 \\c_3 & 0 & 1 \end{pmatrix}=
  \begin{pmatrix} 1 & 0 & 0 \\c_2+c_1c_3 & 1 & c_1 \\c_3 & 0 & 1 \end{pmatrix}$
  \item $\begin{pmatrix} 1 & 0 & 0 \\0 & 1 & 0 \\0 & c_1 & 1 \end{pmatrix}
  \cdot\begin{pmatrix} 1 & 0 & 0 \\c_2 & 1 & 0 \\c_3 & 0 & 1 \end{pmatrix}=
  \begin{pmatrix} 1 & 0 & 0 \\c_2 & 1 & 0 \\c_1c_2+c_3 & c_1 & 1 \end{pmatrix}$
\end{enumerate}

%%%%%%%%%%%%%%%%%%%%%%%%%%%%%%%%%%%%%%%%%%%%%%%%%%%%%%%%%%%%%%%%%%%%%%%%%%%%%%%
\subsubsection{Explicit example}
\def\kOne{1}
\def\kTwo{3}
\def\zkOnepzKtwo{14} % 2\cdot(\kOne+2\cdot\kTwo
\def\zkOne{2} % 2*\kOne
\def\zkTwo{6} % 2*\kTwo

Even more explicit, we can fix the levels $k_1=\kOne$ and $k_2=\kTwo$ together
with $\theta=0$.
Assume without any restriction that $q_1\myrel{\theta}q_2$ and
$q_1\myrel{\theta}q_3$ as well as $q_2\myrel{\theta}q_3$.
Other choices would result \rewrite{in reordering of the tuples below.}
\comm{Let the matrix $L$ be given as $L=\diag(l_1,l_2,l_3)\in\Gl_n(\C)$.}

The classification space is in this case isomorphic to
$\C^{2\cdot(\kOne+2\cdot\kTwo)}=\C^{\zkOnepzKtwo}$.
The element
\[
  ({}^1c_1,{}^2c_1,
  {}^1c_2,{}^1c_3,{}^2c_2,{}^2c_3,\dots,{}^{\zkTwo}c_2,{}^{\zkTwo}c_3)
  \in\C^{\zkOnepzKtwo}
\]
gets, via the isomorphism $\prod_{\alpha\in\A}j_\alpha$, mapped to
\begin{align*}
  &\left(
  \left(
    \begin{pmatrix} 1 & 0 & 0 \\0 & 1 & {}^1c_1 \\0 & 0 & 1 \end{pmatrix},
    \begin{pmatrix} 1 & 0 & 0 \\0 & 1 & 0 \\0 & {}^2c_1 & 1 \end{pmatrix}
  \right),
  \right.
\\&\qquad\left(
  \left.
    \begin{pmatrix} 1 & {}^1c_2 & {}^1c_3 \\0 & 1 & 0 \\0 & 0 & 1 \end{pmatrix},
    \begin{pmatrix} 1 & 0 & 0 \\{}^2c_2 & 1 & 0 \\{}^2c_3 & 0 & 1 \end{pmatrix},
    \dots,
    \begin{pmatrix} 1 & 0 & 0 \\{}^{\zkTwo}c_2 & 1 & 0 \\{}^{\zkTwo}c_3 & 0 & 1 \end{pmatrix}
  \right)
  \right)
\end{align*}
in $\prod_{\alpha\in\A^{\kOne}}\SSto_{\alpha}^{\kOne}(A^0) \times
\prod_{\alpha\in\A^{\kTwo}}\SSto_{\alpha}^{\kTwo}(A^0)$ and thus the element
\begin{align*}
  &\left(
  \left(
    \begin{pmatrix} 1 & 0 & 0 \\0 & 1 & {}^1c_1 \\0 & 0 & 1 \end{pmatrix},
    \id,\id,
    \begin{pmatrix} 1 & 0 & 0 \\0 & 1 & 0 \\0 & {}^2c_1 & 1 \end{pmatrix},
    \id,\id
  \right),
  \right.
\\&\qquad\left(
  \left.
    \begin{pmatrix} 1 & {}^1c_2 & {}^1c_3 \\0 & 1 & 0 \\0 & 0 & 1 \end{pmatrix},
    \begin{pmatrix} 1 & 0 & 0 \\{}^2c_2 & 1 & 0 \\{}^2c_3 & 0 & 1 \end{pmatrix},
    \dots,
    \begin{pmatrix} 1 & 0 & 0 \\{}^{\zkTwo}c_2 & 1 & 0 \\{}^{\zkTwo}c_3 & 0 & 1 \end{pmatrix}
  \right)
  \right)
\end{align*}
in
$\prod_{\alpha\in\A}\SSto_{\alpha}^{\kOne}(A^0) \times
\prod_{\alpha\in\A}\SSto_{\alpha}^{\kTwo}(A^0)$.
Using the morphism $\prod_{\alpha\in\A}i_\alpha^{-1}$ we get a complete set of
Stokes matrices as
\begin{align*}
  &\left(
    \begin{pmatrix} 1 & {}^1c_2 & {}^1c_3 \\0 & 1 & {}^1c_1 \\0 & 0 & 1 \end{pmatrix},
    \begin{pmatrix} 1 & 0 & 0 \\{}^2c_2 & 1 & 0 \\{}^2c_3 & 0 & 1 \end{pmatrix},
    \begin{pmatrix} 1 & {}^3c_2 & {}^3c_3 \\0 & 1 & 0 \\0 & 0 & 1 \end{pmatrix},
  \right.
\\&\qquad
  \left.
    \begin{pmatrix} 1 & 0 & 0 \\{}^4c_2 & 1 & 0 \\{}^2c_1{}^4c_2+{}^4c_3 & {}^2c_1 & 1 \end{pmatrix},
    \begin{pmatrix} 1 & {}^5c_2 & {}^5c_3 \\0 & 1 & 0 \\0 & 0 & 1 \end{pmatrix},
    \begin{pmatrix} 1 & 0 & 0 \\{}^{\zkTwo}c_2 & 1 & 0 \\{}^{\zkTwo}c_3 & 0 & 1 \end{pmatrix}
  \right)
  \in
  \prod_{\alpha\in\A}\SSto_{\alpha}(A^0) \,.
\end{align*}
Applying the isomorphism $\prod_{\alpha\in\A}\rho_\alpha^{-1}$, i.e.\
conjugation by the fundamental solution $\cY_0(t)=t^Le^{Q(t^{-1})}$
(cf.\ Proposition~\ref{prop:representation}), yields then the corresponding
Stokes cocycle in $\prod_{\alpha\in\A}\Sto_{\alpha}(A^0)$ and thus an element
in $\St(A^0)$.
\begin{comment}
  This element is explicitly given as
  \begin{align*}
    &\left(
      \begin{pmatrix}
          1 & {}^1c_2 t^{l_2-l_1}e^{(q_2-q_1)(t^{-1})}
            & {}^1c_3 t^{l_3-l_1}e^{(q_3-q_1)(t^{-1})}
        \\0 & 1 & {}^1c_1 t^{l_3-l_2}e^{(q_3-q_2)(t^{-1})}
        \\0 & 0 & 1
      \end{pmatrix},
      \cY_0^{-1}
      \begin{pmatrix}
          1 & 0 & 0
        \\{}^2c_2 & 1 & 0
        \\{}^2c_3 & 0 & 1
      \end{pmatrix}
      \cY_0
      ,
    \right.
  \\&\qquad
      \cY_0^{-1}
      \begin{pmatrix}
          1 & {}^3c_2 & {}^3c_3
        \\0 & 1 & 0
        \\0 & 0 & 1
      \end{pmatrix}
      \cY_0
      ,
      \cY_0^{-1}
      \begin{pmatrix}
          1 & 0 & 0
        \\{}^4c_2 & 1 & 0
        \\{}^2c_1{}^4c_2+{}^4c_3 & {}^2c_1 & 1
      \end{pmatrix}
      \cY_0
      ,
  \\&\qquad\qquad
    \left.
      \cY_0^{-1}
      \begin{pmatrix}
          1 & {}^5c_2 & {}^5c_3
        \\0 & 1 & 0
        \\0 & 0 & 1
      \end{pmatrix}
      \cY_0
      ,
      \cY_0^{-1}
      \begin{pmatrix}
          1 & 0 & 0
        \\{}^{\zkTwo}c_2 & 1 & 0
        \\{}^{\zkTwo}c_3 & 0 & 1
      \end{pmatrix}
      \cY_0
    \right)
    \in
    \prod_{\alpha\in\A}\Sto_{\alpha}(A^0) \,.
  \end{align*}
\end{comment}

% \fi

\ifnum\myDevelopVariable=0
%%%%%%%%%%%%%%%%%%%%%%%%%%%%%%%%%%%%%%%%%%%%%%%%%%%%%%%%%%%%%%%%%%%%%%%%%%%%%%%
\section{Further improvements}\label{sec:furtherImprovements}
The set $\prod_{\alpha\in\A}\Sto_\alpha(A^0)$ and thus
$\prod_{\alpha\in\A}\SSto_\alpha(A^0)$ has some bad properties, when small
deformations are applied to $[A^0]$, since under arbitrary small changes, one
Stokes ray can split into two.

In the prove of Theorem~\ref{thm:mainThm2} we have seen, \rewrite{that not only
$\cH(A^0)\cong\prod_{\alpha\in\A}\Sto_\alpha(A^0)$ but} also that
\begin{align*}
  \cH(A^0)&\cong\prod_{k\in\cK}\prod_{\alpha\in\A}\Sto_\alpha^k(A^0)
  \\&\cong\prod_{k\in\cK}\prod_{\alpha\in\A^k}\Sto_\alpha^k(A^0)
  &\text{(since $\SSto_{\alpha}^{k}(A^0)=\{\id\}$ when $k\notin\cK_\alpha$.)}
  \\&\cong\prod_{k\in\cK}\prod_{\alpha\in\A^k}\SSto_\alpha^k(A^0) \,.
\end{align*}
This representation can be used to achieve further improvements, since one is
able to multiply some succeeding Stokes matrices (resp. Stokes germs) of the
same level without loss of information.
This is based on the Corollary~\ref{cor:composeLevelwise} stated bellow.

Let us fix a level $k\in\cK$. We want to rewrite the product
$\prod_{\alpha\in\A^k}\SSto_\alpha^k(A^0)$ by collecting the information of
multiple Stokes matrices into one product of these matrices.

\TODO[move?]
Boalch, which looks in his publications \cite{boalch,thboalch} only at the
single-leveled case, uses this extensively to obtain better Stokes matrices,
which are stable under small deformations. Our Stokes matrices are in his
publications called \emph{Stokes factors}.

\begin{defn}
  A subset of $\A^k$ consisting of $\frac{\#\A^k}{2k}$ consecutive anti-Stokes
  directions of level $k$ will be called a \emph{half-period (of level $k$)}.
  \begin{s-rem}
    The set $\A^k$ can clearly be split into $2k$ distinct half-periods.
  \end{s-rem}
\end{defn}
From the definition of anti-Stokes directions of level $k$ it is clear that
every arc of width $\frac{\pi}{k}$, which has no anti-Stokes direction of level
$k$ on its border, contains $\frac{\#\A^k}{2k}$ anti-Stokes directions
since
\begin{einr}
  for every sector $I=U(\theta,\frac{\pi}{k})$, of width $\frac{\pi}{k}$ and
  center $\theta$, which satisfies $\theta\pm\frac{\pi}{2k}\notin\A^k$,
  there is to any pair $(q_j,q_l)$ in $\cQ(A^0)$, which has level $k_{jl}=k$,
  exactly one direction $\alpha\in \A^k\cap I$ which satisfies, that
  $\Im(a_{jl}e^{-ik_{jl\alpha}})=0$.
  At such a direction $\alpha$, corresponding to the pair $(q_j,q_l)$, is then
  either $q_j\myrel{\alpha}q_l$ or $q_l\myrel{\alpha}q_j$ satisfied.
\end{einr}
This implies that for every $\theta$, which satisfies
$\theta\pm\frac{\pi}{2k}\notin\A^k$, is $\A^k\cap U(\theta,\frac{\pi}{k})$
a half-period and by
$\dot\bigcup_{1\leq m\leq 2k}\A^k\cap U(\theta,\frac{m\pi}{k})=\A^k$ is the
corresponding \rewrite{decomposition} in half-periods of $\A^k$ given.

Let $I=U(\theta,\frac{\pi}{k})$ be an arc width $\frac{\pi}{k}$ such that
$\theta\pm\frac{\pi}{2k}\notin\A^k$.
\begin{rem}\label{rem:remarkAboutCommonStructure}
  For every $\alpha\in\A^k\cap I$ we can write the Stokes matrix as
  \[
    K^\alpha=(K^\alpha_{jl})_{j,l\in\{1,\dots,s\}}\in\SSto_\alpha^k(A^0)
  \]
  where the $K^\alpha_{jl}\in\C^{n_j\times n_l}$ are blocks corresponding to
  the structure of $Q$ (cf.\ Definition~\ref{defn:groupOfFaithfullReps}).
  We then know for $j\neq l$ that, if $K^\alpha_{jl}\neq0$ then
  \begin{itemize}
    \item is $K^\alpha_{lj}=0$ and
    \item for every $\alpha'\in(\A^k\cap I)\backslash\{\alpha\}$ is
      $K^{\alpha'}_{jl}=0$ as well as $K^{\alpha'}_{lj}=0$.
  \end{itemize}
\end{rem}
\begin{rem}
  In the situation of Remark~\ref{rem:remarkAboutCommonStructure}
  there exists a common (block) permutation matrix $P\in\GL_n(\C)$ given by
  $(P)_{jl}=\bdelta_{\pi(j)l}$ where
  \begin{itemize}
    \item $\bdelta_{jl}$ is the block version of Kronecker's delta
      which was introduced in Definition~\ref{defn:groupOfFaithfullReps} and
    \item $\pi$ is the permutation of $\{1,\dots,s\}$ corresponding to
      $q_j\underset{\theta}{\prec}q_l$ \Leftrightarrow{} $\pi(j)<\pi(l)$
      \marginnote{\cite[18]{boalch}}
  \end{itemize}
  such that every matrix $P^{-1}K^{\alpha}P$ is upper triangular and the
  observation from Remark~\ref{rem:remarkAboutCommonStructure} is still
  satisfied.
  \begin{s-rem}
    After moving the sector $U(\theta,\frac{\pi}{k})$ to
    $U(\theta,\frac{\pi}{k})+\frac{\pi}{k}
    =U(\theta+\frac{\pi}{k},\frac{\pi}{k})$ we obtain also a corresponding
    permutation $\pi'$ which is inverse to $\pi$.
    The corresponding permutation matrix $P'$ is then given by $P'=P^{-1}$.

    Such that the permutation $P$ transforms Stokes matrices on the sector
    $I+\frac{\pi}{k}$ to lower triangular matrices
  \end{s-rem}
  In the single leveled case of this phenomenon is discussed on page 18 of
  Boalch's paper~\cite{boalch}.
\end{rem}

In the paper \cite{BJL1979Birkhoff} from Balser, Jurkat and Lutz is the
following Lemma stated as Lemma 2 on page 75.
\begin{lem}\label{lem:UniqueDecompositionWotBlocks}
  Let $T\subset\{1,\dots,n\}\times\{1,\dots,n\}$ be a position set, which
  satisfies the \emph{completeness property}:
  \begin{einr}
    if $(j,k)$ and $(k,l)\in T$ then is also $(j,l)\in T$.
  \end{einr}
  Choose a indexing $i:\{1,\dots,
  \underset{\#T}{\underset{\text{\rotatebox[origin=c]{-90}{$=$}}}{\mu}}\}
  \overset{\cong}{\to}T$ of the position set and denote by
  $\delta_{jl}\in\C$ the \rewrite{ordinary} Kronecker's delta,
  then there exists for every
  \[
    K\in \left\{K=(K_{jl})_{j,l\in\{1,\dots,n\}}\in\GL_n(\C)\mid
      K_{jl}=\delta_{jl} \text{~unless~} (j,l)\in T \right\}
  \]
  unique scalars $t_j\in\C$ such that
  \[
    K=(\id + t_1E_{i(1)})\cdots(\id + t_{\mu}E_{i(\mu)}) \,.
  \]
  \begin{s-rem}
    The completeness property is reasonable, since for example
    $S_{\{(2,3),(3,2)\}}$, corresponding to the not complete set
    $\{(2,3),(3,2)\}$, is not stable under the product:
    \begin{align*}
      \begin{pmatrix}
        1 & a & 0
      \\0 & 1 & 0
      \\0 & 0 & 1
      \end{pmatrix}
      \begin{pmatrix}
        1 & 0 & 0
      \\0 & 1 & b
      \\0 & 0 & 1
      \end{pmatrix}
      =
      \begin{pmatrix}
        1 & a & \textcolor{red!60!black}{ab}
      \\0 & 1 & b
      \\0 & 0 & 1
      \end{pmatrix}
      \notin S_{\{(2,3),(3,2)\}}
      \,.
    \end{align*}
  \end{s-rem}
\end{lem}
\begin{lem}
  Every position set, corresponding to some block, is complete.
\end{lem}
\begin{proof}
  Such a position set, corresponding to some block, is given by
  \[
    T=\{(j,l)\mid j_1\leq j\leq j_2, l_1\leq l\leq l_2\}
  \]
  for some $j_1,j_2,l_2$ and $l_2$ in $\{1,\dots,n\}$.

  Let $(j,k)$, $(k,l)\in T$ be two positions.
  It is obvious that $(j,l)\in T$, since $j_1\leq j\leq j_2$ and
  $l_1\leq l\leq l_2$ are satisfied.
\end{proof}
\begin{cor}
  We can write the Lemma~\ref{lem:UniqueDecompositionWotBlocks} in block form,
  corresponding to the structure of $Q$.
  Let $T\subset\{1,\dots,s\}\times\{1,\dots,s\}$ be a position set and choose a
  indexing
  \[
    i:\{1,\dots, \underset{\#T}{\underset{%
      \text{\rotatebox[origin=c]{-90}{$=$}}}{\mu}}\}\overset{\cong}{\to}T
  \]
  of $T$.
  \begin{s-defn}
    Define the group of matrices, corresponding by a complete position set, by
    \[
      S_T:=\left\{K=(K_{jl})_{j,l\in\{1,\dots,n\}}\in\GL_n(\C)\mid
      K_{jl}=\bdelta_{jl} \text{~unless~} (j,l)\in T \right\} \,,
    \]
    where $\bdelta_{jl}$ is the block version of Kronecker's delta,
    corresponding to the structure of $Q$.
  \end{s-defn}
  From the Lemma~\ref{lem:UniqueDecompositionWotBlocks} then follows that for
  every $K\in S_T$ is the decomposition
  \[
    K=K_1\cdot K_2\cdots K_\mu \,,
  \]
  where $K_j\in S_{\{i(j)\}}$, is unique.
\end{cor}
From the previous corollary we deduce the following corollary.
\begin{cor}\label{cor:composeLevelwise}
  Let $T_1,\dots,T_r\subset\{1,\dots,s\}\times\{1,\dots,s\}$ be a distinct
  position sets, such that
  \begin{einr}
    $T:=T_1\dot\cup\dots\dot\cup T_r$ \rewrite{as well as} every $T_m$ satisfy
    the completeness property.
  \end{einr}
  Then is by
  \begin{align*}
    S_{T_1}\times\cdots\times S_{T_m} &\longrightarrow S_T
  \\(K_1,\dots,K_m) &\longmapsto K_1\cdots K_m
  \end{align*}
  an isomorphism defined.
\end{cor}
We can apply the previous corollary in our situation. This yields the following
theorem.
\begin{thm}\label{thm:theoremForlargerDecomp}
  Let $\theta\in S^1$ be a fixed direction which satisfies
  $\theta\pm\frac{\pi}{2k}\notin\A^k$, then
  \[
    \cH(A^0)\cong\prod_{k\in\cK}\prod_{j\in\{1,\dots,2k\}}
    \hat\Sto_{\theta+j\frac{\pi}{k}}^k(A^0) \,,
  \]
  where
  \begin{align*}
    \hat\SSto_{\theta}^k(A^0)
    :\!\!&= \left\{K=(K_{jl})_{j,l\in\{1,\dots,s\}}\in\GL_n(\C)\mid
        K_{jl}=\delta_{jl} \text{~unless~}
        q_j \underset{\theta}{\prec} q_l \text{, } k_{jl}=k\right\}
    \\&=S_{\{(j,l) \mid q_j \underset{\theta}{\prec} q_l \}}
  \end{align*}
  and
  \[
    \hat\Sto_{\theta}^k(A^0):=\{\rho_\theta^{-1}(K)
      \mid K\in\hat\SSto_{\theta}^k(A^0) \} \,.
  \]
  \begin{s-rem}
    \begin{enumerate}
      \item This means that the information of all Stokes-matrices on every
        level $k\in\cK$ can be grouped into $2k$ matrices, which are products
        of the corresponding Stokes matrices.
      \item By the definition above, it is obvious that
        \[
          \Sto_{\theta}^k(A^0)=
          \bigcap_{\theta'\in \A^k\cap U(\theta,\frac{\pi}{k})}
          \hat\Sto_{\theta'}^k(A^0) \,.
        \]
    \end{enumerate}
  \end{s-rem}
\end{thm}
\begin{proof}
  We have an isomorphism
  \[ \begin{tikzcd}
    \eta_\theta^k:
    \underset{\alpha\in\A^k\cap U(\theta,\frac{\pi}{k})}\prod
    \underset{S_{\{(j,l)\mid q_j\myrel{\alpha}q_l\}}}{%
      \underset{\text{\rotatebox[origin=c]{-90}{$=$}}}{\underbrace{%
          \SSto_\alpha^k(A^0)}}}
    \rar{\cong}&
    \underset{S_{\{(j,l) \mid q_j \underset{\theta}{\prec} q_l \}}}{%
      \underset{\text{\rotatebox[origin=c]{-90}{$=$}}}{\underbrace{%
          \hat\SSto_\theta^k(A^0)}}}
    \,.
  \end{tikzcd} \]
  from Corollary~\ref{cor:composeLevelwise}, since
  \begin{itemize}
    \item for every $\alpha\in\A$ satisfies the set
      $\{(j,l)\mid q_j\myrel{\alpha}q_l\}$ the completeness
      property, since it is defined via a \rewrite{transitiv relation}, and
    \item the union of all $\{(j,l)\mid q_j\myrel{\alpha}q_l\}$
      for $\alpha\in I\cap\A^k$ is then
      \begin{align*}
        \dot\bigcup_{\alpha\in I\cap\A}
          \{(j,l)\mid q_j\myrel{\alpha}q_l\}
          &=\{(j,l) \mid q_j \myrel{\alpha} q_l
            \text{~for some~}\alpha\in\A^k\cap U(\theta,\frac{\pi}{k}) \}
        \\&= \{(j,l) \mid q_j \underset{\theta}{\prec} q_l \}
          \qquad\qquad\text{(cf.\ Remark~\ref{rem:relationDistanceCondition})}
      \end{align*}
      and is also complete, since $\underset{\theta}{\prec}$ is also a
      \rewrite{transitive relation}.
  \end{itemize}
  The isomorphism of the theorem is then the Stokes germ version of
  \[ \begin{tikzcd}
    \underset{\eta}{%
      \underset{\text{\rotatebox[origin=c]{90}{$:=$}}}{\underbrace{%
        \underset{k\in\cK}{\prod}\underset{j\in\{1,\dots,2k\}}{\prod}
        \eta_{\theta+j\frac{\pi}{k}}^k
    }}}:
    \underset{\cH(A^0)}{%
      \underset{\text{\rotatebox[origin=c]{-90}{$\cong$}}}{\underbrace{%
        \prod_{k\in\cK}\prod_{\alpha\in\A^k}\SSto_\alpha^k(A^0)}}}
    \rar{\cong}&
    \underset{k\in\cK}{\prod}\underset{j\in\{1,\dots,2k\}}{\prod}
    \hat\SSto_{\theta+j\frac{\pi}{k}}^k(A^0) \,.
  \end{tikzcd} \]
\end{proof}
\begin{cor}
  This does also induce an isomorphism
  \[ \begin{tikzcd}
    \eta:
    \displaystyle\prod_{k\in\cK}\prod_{\alpha\in\A}\Sto_{\alpha}^{k}(A^0)
    \rar{\cong}&
    \displaystyle\prod_{k\in\cK} \prod_{j\in\{1,\dots,2k\}}
      \hat\Sto_{\theta+j\frac{\pi}{k}}^k(A^0)
  \end{tikzcd} \]
  on the level of Stokes germs instead of Stokes matrices.
\end{cor}
\begin{rem}
  The, in the proof of Theorem~\ref{thm:theoremForlargerDecomp}, obtained
  isomorphism
  \[
    \eta: \prod_{\alpha\in\A}\SSto_{\alpha}(A^0) \longrightarrow
    \prod_{k\in\cK} \prod_{j\in\{1,\dots,2k\}}
    \hat\SSto_{\theta+j\frac{\pi}{k}}^k(A^0)
  \]
  is in the single-leveled case with $n=s$, i.e.\ all diagonal elements of
  $Q$ are different, given in Lemma 3.2 of Boalch's paper
  \cite[Lem.3.2]{boalch}.
  In this case is every $\hat\SSto_{\theta+j\frac{\pi}{k}}^k(A^0)$ isomorphic
  to some $PU_+P^{-1}$ where $U_+$ is the group of all upper triangular
  matrices, with ones on the diagonal, and $P$ is a permutation matrix.
\end{rem}

%%%%%%%%%%%%%%%%%%%%%%%%%%%%%%%%%%%%%%%%%%%%%%%%%%%%%%%%%%%%%%%%%%%%%%%%%%%%%%%
\section{The complete Diagram}\label{sec:theCompleteDiagram}
If we start with the diagram from Page~\pageref{page:ofPreDiagram}, we can add
all the defined isomorphisms and rewrite into the following commutative
diagram of \textbf{isomorphisms of pointed sets}.\PROBLEM[pointed sets?]
\begin{center}
  \begin{tikzpicture}
    \pgfmathsetmacro{\dx}{4cm}
    \pgfmathsetmacro{\ldx}{6cm}
    \pgfmathsetmacro{\dy}{2.5cm}
    \pgfmathsetmacro{\mdy}{-1.5cm}

    \node (cH) at (-6,3.5) {$\cH(A^0)$};
    \node (cH2) at ([yshift=-2cm,xshift=2cm]cH) {$G(\!\{t\}\!)\backslash\hat G(A^0)$};
    \node (ProdOfStos) at (-6,-4)
      {$\displaystyle\prod_{\alpha\in\A}\Sto_{\alpha}(A^0)$};
    \node (S3P) at ([xshift={\dx},yshift={\dy}]ProdOfStos)
      {$\displaystyle\prod_{k\in\cK}\prod_{\alpha\in\A}\Sto_{\alpha}^{k}(A^0)$};
    \node (S2P) at ([xshift={\dx},yshift={\dy}]S3P)
      {$\displaystyle\prod_{k\in\cK}\Gamma(\dot\cU^k;\Lambda^k(A^0))$};
    \node (StA0) at ([xshift={\dx},yshift={\dy}]S2P)
      {$\St(A^0)$};
    \node (ProdOfSStos) at ([yshift=-{\dy}]ProdOfStos)
      {$\displaystyle\prod_{\alpha\in\A}\SSto_{\alpha}(A^0)$};
    \node (SS3P) at ([yshift=-{\dy}]S3P)
      {$\displaystyle\prod_{k\in\cK}\prod_{\alpha\in\A}\SSto_{\alpha}^{k}(A^0)$};
    \node (SS2P) at ([yshift={\mdy},xshift={\ldx}]S3P)
      {$\displaystyle\prod_{k\in\cK} \prod_{j\in\{1,\dots,2k\}}
        \hat\Sto_{\theta+j\frac{\pi}{k}}^k(A^0)$};
    \node (SSS2P) at ([yshift={\mdy},xshift={\ldx}]SS3P)
      {$\displaystyle\prod_{k\in\cK} \prod_{j\in\{1,\dots,2k\}}
        \hat\SSto_{\theta+j\frac{\pi}{k}}^k(A^0)$};
    \node[red] (SSS3P) at ([yshift={\mdy},xshift={\ldx}]ProdOfSStos)
      {$\displaystyle\prod_{\alpha\in\A}
        \C^{\sum_{q_j\myrel{\alpha}q_l}
          \deg(q_j-q_l)\cdot n_j\cdot n_l}$};

    \draw[->] (cH2) -- (cH)
      node[midway,above right] {Cor.~\ref{cor:isomorphyOfClassfset}};
    \draw[->] (cH) -- (ProdOfStos)
      node[midway,left] {$g$};
    \draw[->] (cH) -- (StA0)
      node[midway,above] {$\exp_{A^0}$};
    \draw[->,purple] (ProdOfStos) -- (S3P)
      node[midway,above left] {\rewrite{$\displaystyle\chi\circ\prod_{\alpha\in\A}i_\alpha$}};
    \draw[->,purple] (S3P) -- (S2P)
      node[midway,above left] {$\prod_{k\in\cK}i^k$};
    \draw[->,purple] (S2P) -- (StA0)
      node[midway,above left] {$\cT$};
    \draw[->] (ProdOfStos) -- (ProdOfSStos)
      node[midway,left] {$\prod_{\alpha\in\A}\rho_\alpha$};
    \draw[->] (ProdOfSStos) -- (SS3P)
      node[midway,below right] {\rewrite{$\displaystyle\chi\circ\prod_{\alpha\in\A}i_\alpha$}};
    \draw[->] (S3P) -- (SS3P)
      node[midway,right] {$\prod_{k\in\cK}\prod_{\alpha\in\A}\rho_{\alpha}^k$};
    \draw[->] (S3P) -- (SS2P)
      node[midway,above] {$\eta$};
    \draw[->] (SS3P) -- (SSS2P)
      node[midway,above] {$\eta$};
    \draw[->] (SS2P) -- (SSS2P)
      node[midway,right] {(induced by $\rho$)};
    \draw[->,red] (SSS3P) -- (ProdOfSStos)
      node[midway,below left,red] {$\prod_{\alpha\in\A}\vartheta_\alpha$};
  \end{tikzpicture}
\end{center}
Where
\begin{itemize}
  \item the \textcolor{purple}{purple path} is the isomorphism $h$ from
    Theorem~\ref{thm:mainThm2} and
  \item we denote
    \begin{itemize}
      \item $\chi:\displaystyle \prod_{\alpha\in\A}\prod_{k\in\cK}\Sto_\alpha^k(A^0)
        \equiv\displaystyle \prod_{k\in\cK}\prod_{\alpha\in\A}\Sto_\alpha^k(A^0)$
        the reordering and
      \item by abuse of notation we also denote the Stokes matrix version in
        the same way
        $\chi:\displaystyle \prod_{\alpha\in\A}\prod_{k\in\cK}\SSto_\alpha^k(A^0)
        \equiv\displaystyle
        \prod_{k\in\cK}\prod_{\alpha\in\A}\SSto_\alpha^k(A^0)$.
    \end{itemize}
\end{itemize}
\PROBLEM[Der unterste respektiert die Struktur nicht]

\fi
