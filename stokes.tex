\chapter{Stokes Structures}\label{chap:stokes}
Stokes structures are the \rewrite{necessary information} for meromorphic
classification, i.e.\ with the Stokes structures we are able to construct a
\rewrite{space}, which is isomorphic to the classifying \rewrite{set}.

A great overview of this topic is given by Varadarajan in
\cite{Varadarajan96linearmeromorphic}. Other resources we will use are for
example \rewrite{Sabbah's} book \cite[section II]{sabbah2007isomonodromic} for
section~\ref{sec:mainThm1}.
For the sections~\ref{sec:StokesGroup} and~\ref{sec:mainThm2} will
\rewrite{Loday-Richaud's} paper \cite{Loday1994} and the book \cite{Loday2014}
be \rewrite{useful. Also useful} was \rewrite{Boalch's} paper \cite{boalch}
(resp.\ his thesis \cite{thboalch}) which looks only at the single leveled
case.

Let $\cM^{nf}$ be a fixed model with the corresponding normal form $A^0$.
The first objective is to proof the Malgrange-Sibuya theorem
in section~\ref{sec:mainThm1}. It states that the classifying set $\cH(A^0)$
is via $\exp$ isomorphic to the first cohomology $H^1(S^1;\Lambda^{<0}(A^0))$.
In section~\ref{sec:mainThm2} we will improve the Malgrange-Sibuya Theorem by
showing that each 1-cohomology class in $\St(A^0)$ contains a unique
$1$-cocycle of a special form called \emph{the Stokes cocycle}
(cf.\ Section~\ref{sec:StokesGroup}).
The morphism, which maps each Stokes cocycle to its corresponding $1$-cocycle
will be denoted by $h$.

We \rewrite{have the following} commutative diagram of isomorphisms.
\begin{center}
  \begin{tikzpicture}[scale=3]
    \node[] (modSpcMat) at (0,0) {$\cH(A^0)$};
    \node[purple] (class) at (1,0) {$G\backslash\hat G(A^0)$};
    \node[] (sheaf2) at (2,1) {$H^1(S^1;\Lambda^{<0}(A^0))$};
    \node[green!40!black] (sheaf) at (3,1) {$\St(\cM^{nf})$};
    \node[] (sheaf3) at (3,0.7) {\textcolor{purple}{$\St(A^0)$}};
    \node[blue] (mat) at (3,-1) {$\prod_{\theta\in\A}\Sto_\theta(A^0)$};
    % \node[blue] (mat') at (3.5,-1.5) {$(U_+\times U_-)^{k-1}$};

    % \draw[thick,double] (modSpcMat) -- (modSpcSheaf) -- (class) -- (class2) --
    %   (modSpcMat);
    \draw[thick,double,blue] (modSpcMat) -- (class);
    \draw[thick,double] (sheaf) -- (sheaf2) node[midway,above] {\tiny def.};
    \draw[purple,thick,double] (sheaf) -- (sheaf3) node[midway,right] {\tiny def.};

    \draw[->,green!40!black] (modSpcMat) -- (sheaf)
      node[midway,above left] {$\exp$};
    \draw[->,blue,dashed] (modSpcMat) edge[bend right=20] (mat);
    \draw[->,blue] (class) -- (mat) node[midway,below left] {g};
    % \draw[->,blue,dotted] (mat) -- (mat') node[midway,right] {$\cK=\{k-1\}$};
    \draw[->,purple,dashed] (class) -- (sheaf3)
      node[midway,below right] {$\exp_{A^0}$};
    % \draw[->,purple,dashed] (class) -- (mat);
    \draw[->,purple] (mat) -- (sheaf3) node[midway, right] {$h$};
    \draw[->,purple,dashed] (sheaf3) edge[bend right=20] (mat);
  \end{tikzpicture}
\end{center}
The map $g$, which makes the diagram commute, is defined using the theory of
summation, which will be roughly discussed in
section~\ref{sec:multisummability}.

%%%%%%%%%%%%%%%%%%%%%%%%%%%%%%%%%%%%%%%%%%%%%%%%%%%%%%%%%%%%%%%%%%%%%%%%%%%%%%%
\section{Stokes structures: Malgrange-Sibuya isomorphism}\label{sec:mainThm1}
\marginnote{\cite[Thm.I.2.1]{Loday1994},
  \\\cite[Thm.4.3.9]{Loday2014},
  \\\cite[Thm.II.6.2]{sabbah2007isomonodromic}}
Here we will look at the classifying set and we will proof that it is
isomorphic \TODO[as\dots] to the first non abelian cohomology \rewrite{set}
$H^1(S^1;\Lambda^{<0}(A^0))=:\St(A^0)$.
If we talk about cocycles or cochains, we will in the following always mean
$1$-cocycles or $1$-cochains.

Let us first define the Stokes sheaf $\Lambda(A^0)$ on $S^1$, as the sheaf of
flat isotropies.
\begin{defn}\label{defn:StokesSheaf}
  The Stokes sheaf $\Lambda(A^0)$ of $A^0$, is the sheaf of groups defined on
  $S^1$ whose stalk at any point $\theta\in S^1$ is the group of germs of
  $f\in\Gl_n(\cO(\mathfrak{s}))$, $\mathfrak{s}$ a sector containing $\theta$,
  satisfying
  the conditions:
  \begin{enumerate}
    \item Flatness: $\underset{x\in\mathfrak{s}}{\underset{x\to0}{\lim}}f(x)=1$
      and $f\sim_{\mathfrak{s}} 1$;
      \TODO[Why two condition?]
    \item Isotropy of $A^0$: ${}^f\!A^0=A^0$.
  \end{enumerate}
  \begin{s-rem}
    This definition \rewrite{makes also sense} for systems wich are not in
    normal form.
  \end{s-rem}
  \begin{s-rem}
    \PROBLEM[remove? need more defs!]
    Sabbah \cite[110]{sabbah2007isomonodromic} talks about (global) meromorphic
    connections $\sM$ on a small disk $D$ around $0$ instead of germs of
    meromorphic connections.

    Define on $S^1$ the sheaf $\Aut^{<0}(\tilde\sM^{nf})$ of automorphisms of
    $\tilde\sM^{nf}=\cA_D\otimes_{\cO_D}\sM^{nf}$ which
    \begin{itemize}
      \item are compatible with the connection and
      \item are formally equal to the identity, i.e.\ induce the identity on
        $\hat\sM^{nf}=\hat\cO_D\otimes_{\cO_D}\sM^{nf}$
    \end{itemize}

    The sheaf $\Aut^{<0}(\tilde\sM^{nf})$ corresponds to our $\Lambda(A^0)$.
  \end{s-rem}
\end{defn}

\subsection{The theorem}
In the language of meromorphic connections is the Malgrange-Sibuya Theorem
described as follows.

Let $(\cM,\nabla,\hat f)$ be a marked germ of a  meromorphic connection.
There exists an open covering $\cU=(U_j)_{j\in J}$ and for every open set an
isomorphism
\[
  f_j:(\tilde\cM,\tilde\nabla)_{|U_j}
  \to(\tilde\cM^{nf},\tilde\nabla^{nf})_{|U_j}
\]
such that $\hat f_j=\hat f$. By $(f_kf_j^{-1})_{jk}$ is then a cocycle of the
sheaf $\St(A^0)$, relative to the covering $\cU$, defined.
\begin{comment}
  For other lifts $f_j'$ of $\hat f$ on $W_j$, $(f_j'f_j^{-1})$ is a
  $0$-cochain of $\Sto(A^0)$ relative to $\cU$. Thus the associated cochians to
  $(f_j)$ and $(f_j')$ are equivalent. One can also check that, if
  $(\cM,\nabla,\hat f)$ and $(\cM',\nabla',\hat f')$ are isomorphic, the
  corresponding cocycles define the same cohomology class.
\end{comment}
This defines a mapping of pointed sets
\[
  \cH\to H^1(S^1;\Lambda(A^0))
\]
to the first non abelian cohomology of $\Lambda(A^0)$, which sends the class of
$(\cM^{nf},\nabla^{nf},\hat\id)$ to that of $\id$, i.e.\ the trivial cohomology
class.

\begin{tthm}[Malgrange-Sibuya] \label{thm:mainThm1MeromVersion}
  \marginnote{\cite[Thm.I.4.5.1]{babbitt1989local},
    \\\cite[Thm.3.4]{Malgrange1983}}
  The homomorphism
  \[
    \exp:\cH\to\St(A^0):=H^1(S^1;\Lambda(A^0))
  \]
  is an isomorphism of pointed sets.
\end{tthm}
\begin{rem}
  The theorem \cite[Thm.III.1.1.2]{babbitt1989local}, in the book from
  Babbitt and Varadarajan, states that $\St(A^0)$
  is actually a local moduli space for marked pairs, which are formally
  isomorphic to a given connection $(\cM^{nf},\nabla^{nf})$. This means that
  \begin{itemize}
    \item the morphism property,
    \item the criterion of equivalence and
    \item the existence of universal families
  \end{itemize}
  are satisfied (cf.\ \cite[169]{babbitt1989local}).
  In fact is the whole third part of \cite{babbitt1989local} dedicated to this
  topic.
\end{rem}

\subsubsection{The theorem (system version)}
Since the language of meromorphic connections is equivalent to the one of
systems, there is also the translated version of the Malgrange-Sibuya
isomorphism to the language of systems.

Let $(A,\hat F)$ be a marked pair, i.e.\ $\hat F$ solves $[A^0,A]$.
There exists an open covering $\cU=(U_j)_{j\in J}$ together with, for every
open set $U_j$ a lift $F_j\in\Gl_n(\cA(U_j))$
(cf.\ Definition~\ref{defn:lift}), which solves $[A,A^0]$.
By $(F_kF_j^{-1})_{jk}$ is then a cocycle of the sheaf $\St(A^0)$, relative to
the covering $\cU$, defined.
For other lifts $F_j'$ of $\hat F$ on $U_j$ is $(F_j'F_j^{-1})$ a $0$-cochain
of $\Sto(A^0)$ relative to $\cU$, thus the associated cochians to $(F_j)$ and
$(F_j')$ are equivalent.
One can also check that, if $(A,\hat F)$ and $(A',\hat F')$ are equivalent, the
corresponding cocycles define the same cohomology class.
This defines a welldefined mapping of pointed sets
\[
  \cH(A^0)\to H^1(S^1;\Lambda(A^0))
\]
to the first non abelian cohomology of $\Lambda(A^0)$, which we call $\exp$.

\begin{tthm}[Malgrange-Sibuya \rewrite{(system version)}] \label{thm:mainThm1}
  \marginnote{\cite[Theorem 4.5.1]{babbitt1989local}}
  The homomorphism
  \[
    \exp:\cH(A^0)\to\St(A^0):=H^1(S^1;\Lambda(A^0))
  \]
  is an isomorphism of pointed\footnote{Which maps the class of
  $(A^0,\hat\id)$ to that of $\id$, i.e.\ the trivial cohomology class.} sets.
\end{tthm}
Since the morphism $\exp$ depends on the choice of the normal form, we will
denote that, if it is not clear, by
$\exp_{A^0}=\exp$\footnote{In Loday's book~\cite{Loday1994} this is denoted as
$\exp_{\mu_0}$.}.
\begin{rem}\label{rem:expNonNormalForm}
  \begin{enumerate}
    \item \marginnote{\cite{Loday1994} Remark I.2.2}
      To another normal form $A^1={}^\Phi\!A^0$ there correspond cochains,
      which are conjugated via $\Phi$.
      We get the following commutative diagram:
      \[ \begin{tikzcd}
          G\backslash\hat G(A^1) \rar{\cdot\Phi}\dar{\exp_{A^1}}
          & G\backslash\hat G(A^0) \dar{\exp_{A^0}}
          & \hat F \arrow[|->]{r}\arrow[|->]{d}
          & \hat F\Phi \arrow[|->]{d}
        \\ H^1(S^1;\lambda(A^1)) \rar
          & H^1(S^1;\lambda(A^0))
          & \exp_{A^1}(\hat F) \arrow[|->]{r}
          & \exp_{A^0}(\hat F\Phi)
      \end{tikzcd} \]
      where $\exp_{A^0}(\hat F\Phi)=\Phi^{-1}\exp_{A^0}(\hat F)\Phi$.
      \TODO[$\Phi\in G(\!(t)\!)$???]
    \item The isomorphism $\exp_A$ makes also sense, if $A$ is not in normal
      form (cf.\ \cite[883]{Loday1994}).
  \end{enumerate}
\end{rem}

\subsection{Proof}
We will mainly refer to \cite[Proof of Theorem 4.5.1]{babbitt1989local} and
\cite[Section 6.d]{sabbah2007isomonodromic}, where a slightly more complicated
case with deformation space is proofen. These both resources proof the theorem
using the languages of meromorphic connections, whereas we will use systems.
\marginnote{See also \cite{BJL1979Birkhoff} and \cite{babbitt1989local}
  although the proof goes back to work from Malgrange and Sibuya (see for
  example \cite{sibuya1990Linear}).}

We will start by proofing the injectivity of the morphism $\exp$.
\begin{proof}[Proof of the injectivity]
  % \textbf{First look at injectivity:}
  Consider the two elements $(A,\hat F)$ and $(A',\hat F')$ of
  $\hat\Syst_m(A^0)$, whose classes in $\cH(A^0)$ get mapped to same element
  \[
    \exp((A,\hat F))=\lambda=\exp((A',\hat F'))
      \in H^1(S^1;\Lambda(A^0)) \,.
  \]
  Since we can use refined coverings, it is possible to find a finite covering
  $\cU=\{U_j;j\in J\}$ of $S^1$ such that $\lambda$ is the class of the
  cocycles $(F_l^{-1}F_j)$ and $(F_l'^{-1}F_j')$, where $F_j$, $F_j'$ are
  defined relative to $U_j\in\cU$.
  Since $[(F_l^{-1}F_j)]=[(F_l'^{-1}F_j')]$ there exists a
  $0$-cochain $(G_j)_{j\in J}$ of the sheaf $\Lambda(A^0)$ relative to the
  covering $\cU$, such that
  \[
    F_l'^{-1}F_j'=G_lF_l^{-1}F_jG_j^{-1}
    \text{~on~the~arc~} U_j\cap U_l \,,
  \]
  which can be rewritten to
  \[
    F_j'G_jF_j^{-1} = F_l'G_lF_l^{-1}
    \text{~on~the~arc~} U_j\cap U_l \,.
  \]
  If we set $H_j:=F_j'G_{j}F_j^{-1}$ on $U_{j}$, we get
  \begin{itemize}
    \item a solution of $[A,A']$ on every $U_j$, i.e.\ it
      satisfies there ${}^{H_j}A=A'$, since
      \begin{align*}
        {}^{H_j}A &= {}^{F_j'G_{j}F_j^{-1}}A
        \\&={}^{F_j'G_{j}}A^0
        & \text{(since $F_j'$ is a lift of $\hat F'$ on $U_j$)}
        \\&={}^{F_j'}A^0
        & \text{(since $G_j$ is a is an isotropy of $A^0$)}
        \\&=A'
        & \text{(since $F_j$ is a lift of $\hat F$ on $U_j$)}
      \end{align*}
      and
      % \begin{comment}
      %   \begin{align*}
      %     {}^{H_j}A &= (dH_j)H_j^{-1}+H_jAH_j^{-1}
      %     \\&=(d(F_j^{-1}G_{j}^{-1}F_j'))F_j'^{-1}G_{j}F_j
      %         + F_j^{-1}G_{j}^{-1}F_j'AF_j'^{-1}G_{j}F_j
      %     \\&=\dots
      %     \\&=\left(d\left(\hat F\hat F'^{-1}\right)\right)
      %         \left(\hat F\hat F'^{-1}\right)^{-1}
      %         +\hat F\hat F'^{-1} A \left(\hat F\hat F'^{-1}\right)^{-1}
      %     \\&={}^{\hat F\hat F'^{-1}}A'
      %     \\&={}^{\hat F}\left({}^{\hat F'^{-1}}A'\right)
      %     \\&=A'
      %   \end{align*}
      % \end{comment}
    \item which satisfies $\hat F'=\hat H_j\hat F$ on every $U_j$, since
      \begin{align*}
        \hat H_j\hat F&= \widehat{F_j'G_{j}F_j^{-1}} \hat F
        \\&= \hat{F_j'}
        \underset{\id}{%
          \underset{\text{\rotatebox[origin=c]{-90}{$=$}}}{%
            \underbrace{%
              \hat{G_{j}}
            }
          }
        }
        \hat{F_j^{-1}} \hat F
        & \text{(since $G_j$ is flat, i.e.\ $\hat G_j=\id$)}
        \\&= \hat{F'}
        \underset{\id}{%
          \underset{\text{\rotatebox[origin=c]{-90}{$=$}}}{%
            \underbrace{%
              \hat{F^{-1}} \hat F
            }
          }
        }
        \\&= \hat{F'}
      \end{align*}
  \end{itemize}
  Therefore are $(A,\hat F)$ and $(A,\hat F')$ equivalent and injectivity is
  proven.
  \iffalse
    \begin{comment}
      \textbf{First look at injectivity:}
      Consider the two elements $(\cM,\nabla,\hat f)$ and $(\cM',\nabla',\hat
      f')$ of $\cH$ which map to same cohomology class
      \[
        \exp((\cM,\nabla,\hat f))=\lambda=\exp((\cM',\nabla',\hat f'))
          \in H^1(S^1;\Lambda(A^0)) \,.
      \]
      Since we can use refined coverings, it is possible to find a finite
      covering $\cU=\{U_j;j\in J\}$ of $S^1$ such that $\lambda$ is the class
      of the cocycles $(f_lf_j^{-1})$ and $(f_l',f_j'^{-1})$, where
      $f_j$,$f_j'$ are defined on $U_j$.
      Since $[(f_lf_j^{-1})]=[(f_l'f_j'^{-1})]$ there exists a $0$-cochain
      $(g_j)$ of the sheaf $\Aut^{<0}(\tilde\cM^{nf})$ relative to the covering
      $(I_j)$, such that
      \[
        f_l'f_j'^{-1}=g_lf_lf_j^{-1}g_j^{-1} \text{ on } I_j\cap I_l.
      \]
      If we set $\sigma=f_j^{-1}g_{j}^{-1}f_j'$ on $I_{j}$, we get a horizontal
      section\TODO[~on~???], thus\TODO[why?] it satisfies
      $\sigma\circ\hat{f'}=\hat f$. Therefore are $(\cM,\nabla,\hat f)$ and
      $(\cM',\nabla',\hat{f'})$ isomorphic and injectivity is proven.
    \end{comment}
  \fi
\end{proof}

For the proof of the surjectivity we will use another result from Malgrange and
Sibuya, which is also called the Malgrange-Sibuya Theorem. It can for example
be found in Babbitt and Varadarajans's book \cite[65ff]{babbitt1989local} as
Theorem 4.2.1.

Let $\hat F\in G(\!(t)\!)$ be a matrix with formally meromorphic entries. By
the Borel-Ritt Lemma (cf.\ Theorem~\ref{thm:borel-ritt}) we then know, that
there exists for every sector $\mathfrak{s}$ a holomorphic function
$g:\mathfrak{s}\to G$ which is asymptotic to $\hat F$.
We will denote the set of all such holomorphic functions, which are on
the arc $I$ asymptotic to $\id\in G(\!(t)\!)$ by
\[
  \cG(I)=\left\{g\in\Gl_n(\cA(I))\mid g\sim_I\id\right\}\,,
\]
and this defines a sheaf $\cG$ on $S^1$.
The statement of the (second) Malgrange-Sibuya Theorem is then, that the
\rewrite{difference} between formal and konvergent invertible matrices is
described by the first sheaf cohomology $H^1(S^1,\cG)$ of $\cG$ via the map
\[
  \Theta: G\llbracket t\rrbracket/G\{t\}\to H^1(S^1,\cG) \,,
\]
which is set up as follows:
\begin{einr}
  Let $[\hat F]\in G\llbracket t\rrbracket/G\{t\}$ with ambassador $\hat F$ and
  let $\cU=\{U_j\mid j\in J\}$ be a finite covering of $S^1$ by open arcs.
  The Borel-Ritt Lemma yields for every arc $j\in J$ a holomorphic function
  $F_j$ which satisfies $F_j\sim_{U_j}\hat F$.
  By $(F_lF_j^{-1})_{j,l\in J}$ is then a cocycle for $\cG$ defined,
  and write $\Theta([\hat F])$ for the corresponding cohomology class.
\end{einr}
It can be verified, that the cass $\Theta([\hat F])$ does not depend on
\begin{itemize}
  \item the choise of an ambassador $\hat F$ in
    $[\hat F]\in G\llbracket t\rrbracket/G\{t\}$\PROBLEM[proof!],
  \item the choice of the covering $\cU$\PROBLEM[proof!] nor
  \item the choice the $F_j$\PROBLEM[proof!].
\end{itemize}
\begin{lem}
  \TODO[remove this lemma? is not needed/used]
  The mapping $\Theta$ is injective.
\end{lem}
\begin{proof}
  Let $\hat F$ and $\hat F'\in G\llbracket t\rrbracket$ such that
  $\Theta([\hat F])=\Theta([\hat F'])$.
  We then can find a covering $\cU=\{U_j\mid j\in J\}$ together with
  holomorphic functions $F_j$ and $F_j'$, which satisfy
  $F_j\sim_{U_j}\hat F$ and $F_j'\sim_{U_j}\hat F'$, such that
  $(F_l^{-1}F_j)_{j,l\in J}$ and $(F_l'^{-1}F_j')_{j,l\in J}$ determine the
  classes $\Theta([\hat F])$ and $\Theta([\hat F'])$.
  This implies that there are maps $g_j$, which are on $U_j$ holomorphic and
  satisfy $g_j\sim_{U_j}\id$ such that
  \[
    F_l^{-1}F_j=g_l^{-1}F_l'^{-1}F_j'g_j
    \text{~on~the~arc~} U_j\cap U_l
  \]
  This equation can be rewritten to
  \[
    F_l'g_lF_l^{-1}=F_j'g_jF_j^{-1}
    \text{~on~the~arc~} U_j\cap U_l \,.
  \]
  Since this tells us, that the functions $F_j'g_jF_j^{-1}$ coincide on
  the overlapping and define a holomorphic map from the arc $S^1$ (i.e.\ a
  punctured disc with a small radius) into $G$, which will be called $g$.
  Since $F_j'g_jF_j^{-1}\sim_{U_j}\id$ for all $j\in J$, we have
  $g\sim\id$.
  \PROBLEM[Riemannscher hebbarkeits satz]
  Thus the defined $g$ satisfies $g=F'^{-1}F$, so that $[F]=[F']$.
\end{proof}
\TODO[\cite{babbitt1989local} on page 66]
\begin{thm}[Malgrange-Sibuya]\label{thm:thm1helpMalgSibuy}
  The map $\Theta$ is an isomorphism
  \[
    G\llbracket t\rrbracket/G\{t\}\cong H^1(S^1,\cG).
  \]
  \begin{s-rem}
    Since we know that
    $G\llbracket t\rrbracket/G\{t\}\cong G(\!(t)\!)/G(\!\{t\}\!)$\TODO[(cf~??)],
    we get also
    \[
      G(\!(t)\!)/G(\!\{t\}\!)\cong H^1(S^1,\cG).
    \]
  \end{s-rem}
\end{thm}
This theorem is proven in Section 4.4 of Babbitt and Varadarajan's book
\cite{babbitt1989local}.

We are now able to proof the surjectivity part of Theorem~\ref{thm:mainThm1}.
\begin{proof}[Proof of surjectivity]
  \marginnote{\cite[72]{babbitt1989local}}
  Let the cohomology class $\lambda\in H^1(S^1,\Lambda(A^0))$ be represented by
  a cocycle $(F_{jl})_{j,l\in J}$ associated with some finite covering
  $\cU=\{U_j;j\in J\}$ of $S^1$. We especially know, that
  \begin{itemize}
    \item $F_{jl}$ is on $U_j\cap U_l$ asymptotic to $\id$ and
    \item it is a  ${}^{F_{jl}}A^0=A^0$.
  \end{itemize}
  The cocycle $(F_{jl})_{j,l\in J}$ also determines an element in
  $\sigma\in H^1(S^1,\cG)$.
  From the Theorem~\ref{thm:thm1helpMalgSibuy} we know, that there is a $\hat
  F\in G\llbracket t\rrbracket\subset G(\!(t)\!)$ whose class $[\hat F]$ gets
  via $\Theta$ mapped to $\sigma$.
  Thus there exists holomorphic functions $F_j:\mathfrak{s}_{U_j}\to G$ with
  $F_j\sim_{U_j}\hat F$ and $F_l^{-1}F_j=F_{jl}$ on
  $\mathfrak{s}_{U_j\cap U_l}$ for all $j,l\in J$.

  Define on every arc $U_j$ the matrix $A_j:={}^{F_j}A^0$.
  On the intersections $U_j\cap U_l$ we know that $A_j=A_l$, since
  $F_l^{-1}F_j\in\Lambda(A^0)$ and thus
  $A^0={}^{F_l^{-1}}\Big(\underset{=A_j}{\underbrace{{}^{F_j}A^0}}\Big)$.
  Thus the $A_j$ \rewrite{glue to a} section $A$, which satisfies
  ${}^{\hat F}A=A_0$ by construction.
  We have found an element $(A,\hat F)\in\cH(A^0)$ whose image under $\exp$ is
  $\sigma$.
\end{proof}

\begin{comment}
  \TODO[Define $\tilde\cM$ or find other way!]
  \begin{proof}[Proof of surjectivity]
    \begin{multicols}{2}
      % \textbf{Now look at surjectivity:}
      Similar to the proof, given by Sabbah \marginnote{Which refers to
      \cite{Malgrange1983}} in \cite{sabbah2007isomonodromic}, we will start this
      part of the proof by giving a necessary and sufficient condition for a
      class $\lambda$ in $H^1(S^1,\Lambda(A^0))$
      \TODO[$\Lambda(A^0)\sim\Aut^{<0}(\tilde\cM^{nf})$ or $\Aut^{<0}(\cM^{nf})$?]
      to come from an object
      $(\cM,\nabla,\hat f)\in\cH$:
      \begin{einr}
        This is the case if and only if the image of
        $\lambda\in\Aut^{<0}(\cM^{nf})$ in the set
        $H^1(S^1,\Aut_{\cA}(\tilde\cM^{nf}))$ (where $\Aut_{\cA}(\tilde\cM^{nf})$
        contains only the $\cA$-linear automorphisms) is the identity.
      \end{einr}

    \columnbreak

      \textcolor{gray}{%
        \textbf{Now look at surjectivity:}
        Similar to the proof, given by Sabbah \marginnote{Which refers to
        \cite{Malgrange1983}} in \cite{sabbah2007isomonodromic}, we will start
        this part of the proof by giving a necessary and sufficient condition
        for a class $\lambda$ in $H^1(S^1,\Lambda(A^0))$
        to come from an object
        $(A,\hat F)\in\cH(A^0)$:
        \begin{einr}
          \rewrite{This is the case if and only if the image of
            $\lambda\in\Aut^{<0}(\cM^{nf})$ in the set
            $H^1(S^1,\Aut_{\cA}(\tilde\cM^{nf}))$ (where
            $\Aut_{\cA}(\tilde\cM^{nf})$ contains only the $\cA$-linear
            automorphisms) is the identity.}
        \end{einr}
      }
    \end{multicols}
    \begin{proof}
      \textbf{``\Rightarrow{}'':}
      If $\lambda$ is the image of some $(\cM,\nabla,\hat f)$ then there exists
      \begin{itemize}
        \item a covering $(I_{j})$ of $S^1$ and
        \item isomorphisms $f_j:\tilde\cM\overset{\sim}\to\tilde\cM^{nf}$
          inducing $\hat f$
      \end{itemize}
      such that $\lambda$ comes from a cocycle $(\lambda_{j,l})=(f_lf_j^{-1})$
      on $I_j\cap I_l$.
      \TODO{} shows that $(\lambda_{jl})$ is a coboundary of
      $\Aut_\cA(\tilde\cM^{nf})$.

      \textbf{``\Leftarrow{}'':}
      If for some suitable covering $(I_j)$ the cocycle $(\lambda_{jl})$ is a
      coboundary with values in $\Aut_\cA(\tilde\cM^{nf})$, i.e.\
      $\lambda_{jl}=f_lf_j^{-1}$, we define a new connection $\nabla$ on
      $\tilde\cM^{nf}$ by conjugating $\nabla^{nf}$ by $f_j$ on $U_j$.

      \TODO{}

      Moreover $\hat f_j=\hat f_l$ on $U_j\cap U_l$, so that the formal
      isomorphisms
      \[
        \hat f_j:(\hat \cM^{nf},\nabla)
        \overset{\sim}{\longrightarrow}
        (\hat\cM^{nf},\nabla^{nf})
      \]
      can be glued in an isomorphism $\hat f:(\hat \cM^{nf},\nabla)
      \overset{\sim}{\longrightarrow}(\hat\cM^{nf},\nabla^{nf})$.
    \end{proof}

    Thus, the proof of~\ref{thm:mainThm1} is a consequence of the following
    theorem by Malgrange and Sibuya.
    \begin{thm}[Malgrange-Sibuya]\label{thm:malgSibuyaHelp}
      \marginnote{\cite[Thm.II.6.10]{sabbah2007isomonodromic}}
      The image of the mapping
      \[
        H^1(S^1,\Gl_d^{<0}(\cA_{\tilde D}))
        \to
        H^1(H^1,\Gl_d(\cA_{\tilde D}))
      \]
      is the identity.
    \end{thm}
    For the proof of Theorem~\ref{thm:malgSibuyaHelp} which we refer to
    \cite[Th.A.1]{Malgrange1983}, \cite[Th.6.4.1]{sibuya1990Linear} and
    \cite{babbitt1989local}.
  \end{proof}
\end{comment}

%%%%%%%%%%%%%%%%%%%%%%%%%%%%%%%%%%%%%%%%%%%%%%%%%%%%%%%%%%%%%%%%%%%%%%%%%%%%%%%
\section{The Stokes group}\label{sec:StokesGroup}
%%%%%%%%%%%%%%%%%%%%%%%%%%%%%%%%%%%%%%%%%%%%%%%%%%%%%%%%%%%%%%%%%%%%%%%%%%%%%%%
\subsection{Anti-Stokes directions and the Stokes group}
\marginnote{\cite[I.4]{Loday1994}}
Let us recall, that the normal form $A^0$ is written as
$A^0=t^LQ'(t^{-1})t^{-L}+L\frac{1}{t}$ and a normal solution is given by
$\cY_0(t)=t^Le^{Q(t^{-1})}$ (cf.\ Proposition~\ref{prop:fundSolBuilder}), where
$Q(t^{-1})=\diag(\underset{n_1\text{-times}}{\underbrace{%
  q_1,\dots,q_1}},q_2,\dots ,q_s)$.
Let us denote by
\[
  \cQ_{[A^0]} := \left\{q_1(t^{-1}),\dots,q_s(t^{-1})\right\}
\]
the \emph{set of all determining polynomials of $[A^0]$} and by
\[
  \cQ_{[\End A^0]}:=\left\{\left(q_j-q_l\right)(t^{-1})
    \mid q_j \neq q_l \in\cQ_{[A^0]}
  \right\}
\]
the \emph{set of all determining polynomials of $[\End A^0]$}.
\begin{defn}\label{defn:determiningPolysOfEndA}
  We call
  \begin{itemize}
    \item $a_{jl}\in\C\backslash\{0\}$ the \emph{leading factor},
    \item $\frac{a_{jl}}{t^{k}}=:q_{jl}(t^{-1})$ the \emph{leading
      coefficient} and
    \item $k=k_{jl}$ the \emph{degree} $\deg(q_j-q_l)$ of $(q_j-q_l)$ or a
      \emph{level} of $A^0$
  \end{itemize}
  of $q_j-q_l\in\cQ_{[\End A^0]}$ if
  \[
    q_j-q_l\in\left\{\frac{a_{jl}}{t^{k}}+h \mid h \in o(t^{-k}), a_{jl}\neq0
    \right\}\,.
  \]
  \begin{s-rem}
    \begin{enumerate}
      \item It is obvious that $k_{jl}=k_{lj}$ and
        $q_{jl}(t^{-1})=-q_{lj}(t^{-1})$.
      \item In Boalch's paper \cite{boalch} (and also in \cite{thboalch}) is
        the \rewrite{$k$ always incremented by one}.
        We will prefer the other definition, which is also used in
        Loday-Richaud's paper \cite{Loday1994}.
      \item In Loday-Richaud's book \cite[Def.4.3.6]{Loday2014} $a_{jl}$ gets a
        negative sign to be consistend with calculations at $\infty$. Here this
        is not necessary, since we use the clockwise orientation on $S^1$
        (cf.\ Definition~\ref{defn:antiStokesDir}).
        \begin{comment}
          Does that mean that, to be consistend with \cite{boalch}, one has to
          invert the permutation matrix?
        \end{comment}
    \end{enumerate}
  \end{s-rem}
  The degrees of the elements in $\cQ_{[\End A^0]}$ are defined to be  the
  \emph{levels} of $A^0$.
  The set of all levels of $A^0$ will be denoted by
  \[
    \cK=\{k_1<\dots<k_r\} \subset \Q \,.
  \]
  \begin{s-rem}
    \begin{enumerate}
      \item If we assume that the level of $q_j-q_l$ is given by
        $k_{jl}=\max\{\deg(q_j),\deg(q_l)\}$ is then $\cK$ given by
        \[
          \cK=\left\{
            k_{1,2},
            k_{1,3},\dots,
            k_{1,n},
            k_{2,3},\dots,
            k_{n-1,n}
          \right\}\,.
        \]
      \item $A^0$ is unramified if and only if $\cK\subset\Z$. Since we only
        want to consider the unramified case, this will be always the case.
    \end{enumerate}
  \end{s-rem}
\end{defn}

\begin{comment}
  \begin{defn}
    \marginnote{\cite[130]{hotta2008}, \cite[79]{Loday2014}}
    A function is \emph{of maximal decay}, if \PROBLEM{}
    \begin{s-rem}
      \marginnote{\cite[79]{Loday2014}}
      An exponential $e^{q(1/t)}=e^{-\frac{a}{t^{k}}(x+o(t^{-k}))}$ has maximal
      decay in a direction $\tilde\theta\in S^1$ if and only if
      $-ae^{-ik\tilde\theta}$ is real negative.
    \end{s-rem}

    \comm{On the other hand, is a function is \emph{flat}, if\dots}
  \end{defn}
\end{comment}
\begin{defn}
  \TODO[better definition?]
  \marginnote{\cite[79]{Loday2014}}
  An exponential $e^{q(1/t)}=e^{-\frac{a}{t^{k}}(x+o(t^{-k}))}$ has
  \emph{maximal decay in a direction $\tilde\theta\in S^1$} if and only if
  $-ae^{-ik\tilde\theta}$ is real negative.
\end{defn}

On the elements of $\cQ(A^0)$ we define the following (partial) order
relations:
\begin{defn}
  Let $\tilde\theta$ be a determination of $\theta$.
  \begin{itemize}
    \item We define the relation
      $\boldmath q_j \underset{\tilde\theta}{\prec} q_l$ to be equivalent to
      the condition
      \begin{einr}
        $e^{(q_j-g_l)(t^{-1})}$ is flat at $0$ in a neighbourhood of the
        direction $\tilde\theta$.
      \end{einr}
    \item Let us define another relation
      $\boldmath q_j \underset{\tilde\theta,\max}{\prec} q_l$ equivalent to
      \begin{einr}
        $e^{(q_j-g_l)(t^{-1})}$ is of maximal decay in the direction
        $\tilde\theta$.
      \end{einr}
  \end{itemize}
  \begin{s-rem}
    \begin{enumerate}
      \item The condition $q_j \underset{\tilde\theta}{\prec} q_l$ is
        satisfied if and only if $\Re(a_{jl}e^{-ik_{jl}\tilde\theta})<0$.
      \item The condition $q_j \underset{\tilde\theta,\max}{\prec} q_l$ is
        equivalent to
        \begin{einr}
          $a_{jl}e^{-ik_{jl}\tilde\theta}$ is a real negative
          number, i.e.\ $q_j \underset{\tilde\theta}{\prec} q_l$ and
          $\Im(a_{jl}e^{-ik_{jl}\theta})=0$.
        \end{einr}
      \item In the unramified case do these relations not depend on the
        determination $\tilde\theta$ of $\theta$. As a consequence we will only
        write $\underset{\theta}{\prec}$ and $\underset{\theta,\max}{\prec}$.
    \end{enumerate}
  \end{s-rem}
\end{defn}
Let us look closer at functions of the form $f:\theta\mapsto ae^{-ik\theta}$,
corresponding to some pair $(q_j,q_l)$.
Write $a$ as $a=|a|e^{i\arg(a)}$, thus the function writes as
\[
  f(\theta)=|a|e^{i(\arg(a)-k\theta)}=
  |a|(\cos(\arg(a)-k\theta) + i\sin(\arg(a)-k\theta)) \,.
\]
In the figure~\ref{fig:functionF}, we illustrate the real and the imaginary
part of $f$.
\begin{figure}[h!] %{{{
  \begin{flushright}
    \tikzmarkc{n2}{purple} here is $\Im(ae^{-ik\theta})=0$
  \end{flushright}
  \begin{center}
    \begin{tikzpicture}[scale=4]
      \pgfmathsetmacro{\k}{3}
      \pgfmathsetmacro{\argA}{0.3}
      \pgfmathsetmacro{\absA}{0.4}

      \begin{scope}[thick]
        \clip (-0.6,{\absA+0.1}) rectangle (2.6,{-\absA-0.1});
        \foreach \x in {-2,-1,...,3}{
          \pgfmathsetmacro{\s}{{\argA + \x / \k * 2 - 1/2/\k}};
          \fill[blue!10!white]
            ({\s + 2/2/\k},0) sin ({\s + 3/2/\k},{-\absA})
                              cos ({\s + 4/2/\k},0);
          \draw[blue!40!black] (\s,0) sin ({\s + 1/2/\k},\absA)
                       cos ({\s + 2/2/\k},0)
                       sin ({\s + 3/2/\k},{-\absA})
                       cos ({\s + 4/2/\k},0);
          \fill[white] ({\s + 3/2/\k},{-\absA}) circle (1pt);
          \fill[blue!40!black] ({\s + 3/2/\k},{-\absA}) circle (.4pt);
          \pgfmathsetmacro{\s}{{\argA + \x / \k * 2}};
          \draw[purple] (\s,0) sin ({\s + 1/2/\k},\absA)
                       cos ({\s + 2/2/\k},0)
                       sin ({\s + 3/2/\k},{-\absA})
                       cos ({\s + 4/2/\k},0);
          \fill[white] (\s,0) circle (1pt);
          \fill[purple] (\s,0) circle (.4pt);
          \fill[white] ({\s + 2/2/\k},0) circle (1pt);
          \fill[purple] ({\s + 2/2/\k},0) circle (.4pt);
        }
      \end{scope}
      \draw [blue!40!black,dashed]
        ({\argA+4/\k + 1/2/\k},{-\absA}) -- ({\argA+4/\k + 1/2/\k},0);
      \draw [blue!40!black,dashed]
        ({\argA+4/\k + 3/2/\k},{-\absA}) -- ({\argA+4/\k + 3/2/\k},0);
      \draw [blue!40!black, thick
            ,decorate
            ,decoration={brace,mirror,amplitude=10pt}
            ,xshift=0pt
            ,yshift=0pt]
        ({\argA+4/\k + 1/2/\k},{-\absA}) -- ({\argA+4/\k + 3/2/\k},{-\absA})
        node [midway,yshift=-7pt] {$\tikzmark{g0}$};

      \node at ({\argA+2 - 3/\k},0) {$\tikzmark{f-1}$};
      \node at ({\argA+2 - 2/\k},0) {$\tikzmark{f0}$};
      \node at ({\argA+2 - 1/\k},0) {$\tikzmark{f1}$};

      \node at ({\argA-2/\k + 2/2/\k},{-\absA}) {$\tikzmark{e-1}$};
      \node at ({\argA + 2/2/\k},{-\absA}) {$\tikzmark{e0}$};
      \node at ({\argA+2/\k + 2/2/\k},{-\absA}) {$\tikzmark{e1}$};
      \node at ({\argA+4/\k + 2/2/\k},{-\absA}) {$\tikzmark{e2}$};

      \draw (-0.6,0) -- (2.6,0);
      \draw[dotted] (-0.6,\absA)node[left,font=\tiny] {$|a|$} -- (2.6,\absA);
      \draw[dotted] (-0.6,{-\absA})node[left,font=\tiny] {$-|a|$} -- (2.6,{-\absA}); 
      \foreach \x in {-0.5,0,...,2.5}{
        \draw[dotted] ({\x},{-\absA}) -- ({\x},{\absA + 0.1})
          node [above,font=\tiny,] {\x \pi};
      }
      \draw[thick, dotted, green!50!black] (\argA,{-\absA}) -- (\argA,{\absA + 0.1})
        node [above,font=\tiny,] {$\frac{\arg(a)}{k}$};
    \end{tikzpicture}
  \end{center}
  \begin{flushright}
    \tikzmarkc{n3}{blue} here is $q_j \underset{\theta}{\prec} q_l$
  \end{flushright}
  \begin{flushright}
    \tikzmarkc{n1}{blue} here is $q_j \underset{\theta,\max}{\prec} q_l$
  \end{flushright}
  \begin{tikzpicture}[remember picture,overlay]
    \draw[->,blue!50!white,thick] (n3) to[out=150,in=270] (g0);
    \draw[->,purple!50!white,thick] (n2) to[out=240,in=60] (f-1);
    \draw[->,purple!50!white,thick] (n2) to[out=265,in=120] (f0);
    \draw[->,purple!50!white,thick] (n2) to[out=290,in=60] (f1);
    \draw[->,blue!50!white,thick] (n1) to[out=180,in=270] (e-1);
    \draw[->,blue!50!white,thick] (n1) to[out=170,in=270] (e0);
    \draw[->,blue!50!white,thick] (n1) to[out=160,in=270] (e1);
    % \draw[->,blue!40!black,thick] (n1) to[out=150,in=270] (e2);
  \end{tikzpicture}
  \caption{In this plot is the real part of $f(\theta)=ae^{-ik\theta}$,
    corresponding to some pair $(q_j,q_l)$, in
    \textcolor{blue!40!white}{blue} and the imaginary part in
    \textcolor{purple}{purple} sketched.
  }\label{fig:functionF}
\end{figure} %}}}
The graphs, corresponding to the pair $(q_l,q_j)$ are then obtained by the
transformation $\arg(a)\to\arg(-a)=\arg(a)+\pi$, i.e.\ the shift by
$\frac{\pi}{k}$ to the right. This $\frac{\pi}{k}$ is exactly a halve period.

\begin{defn}\label{defn:antiStokesDir}
  % \marginnote{See \cite[Def.I.4.5]{Loday1994}(for the ramified case)
  %   \cite[Def.3.2]{boalch}}
  \begin{enumerate}
    \item $\theta$ is an \emph{anti-Stokes direction} if there is at least one
      pair $(q_j,q_l)$ in $\cQ(A^0)$, which satisfies
      $q_j \underset{\theta,\max}{\prec} q_l$.
      \begin{itemize}
        \item Let $\A=\{\theta_1,\dots,\theta_{\nu}\}$ denote the set of all
          anti-Stokes directions in a clockwise ordering. For a uniform
          notation later, \rewrite{define $\A$ to contain a single, arbitrary
          direction if $\cK=\{0\}$.}
          \TODO[make all Stokes dirs $\alpha$ instead of $\theta$?]
          \begin{s-rem}
            The clockwise ordering is chosen, similar to Loday-Richaud's paper
            \cite{Loday1994}, since the calculations are then compatible with
            the calculations, which look at $\infty$ and take a
            counterclockwise ordering.
            Boalch uses in \cite{boalch} and \cite{thboalch} the inverse
            ordering, but looks also at $0$, thus there might be some
            incompatibilities.
            In Loday-Richaud's book \cite{Loday2014} this problem is solved by
            an additional minus sign for some koefficients\comm{~to $a_{ij}$}.
          \end{s-rem}
      \end{itemize}
    \item $\theta$ is an \emph{Stokes direction} if there is at least one pair
      $(q_j,q_l)$ in $\cQ(A^0)$, which satisfies neither
      $q_j\underset{\theta}{\prec} q_l$ nor $q_l\underset{\theta}{\prec} q_j$.
      \begin{itemize}
        \item Let $\S=\{\sigma_1<\cdots<\sigma_\mu\}$ be the set of Stokes
          directions.
      \end{itemize}
  \end{enumerate}
\end{defn}

\begin{lem}\label{lem:rotationalSym}%\label{rem:rotationalSymPrime}
  If $\theta\in\A$ together with a pair $(q_j-q_l)(t^{-1})\in\cQ_{[\End A^0]}$
  of degree $k$ is given, we have for every $m\in\N$ that
  \[
    \underset{=:\theta'}{\underbrace{\theta+m\frac{\pi}{k}}} \in \A \,.
  \]
  Especially is either $q_j \underset{\theta',\max}{\prec} q_l$ (in the
  case, when $m$ is even) or $q_l \underset{\theta',\max}{\prec} q_j$
  (when $m$ is uneven) satisfied (see figure~\ref{fig:functionF}).
  \begin{s-cor}
    It follows that in the case $\cK=\{k\}$, the set $\A$ has
    $\frac{\pi}{k}$-rotational symmetry.
  \end{s-cor}
\end{lem}
\begin{proof}
  \marginnote{\cite[8]{thboalch}}
  Let $\theta\in S^1$ with determination $\tilde\theta$.
  Let $(j,l)$ be a pair such that
  $q_j \underset{\tilde\theta,\max}{\prec} q_l$, i.e.\ such that
  $a_{jl}e^{-ik_{jl}\tilde\theta}\in\R_{<0}$.
  Hence, for $m\in\N$,
  \begin{align*}
    a_{jl}e^{-ik_{jl}\left(\tilde\theta+m\frac{\pi}{k_{jl}}\right)}
    &=a_{jl}e^{-ik_{jl}\tilde\theta}e^{-im\pi}
    = \begin{cases}
      a_{jl}e^{-ik_{jl}\tilde\theta}\in\R_{<0}
        & \text{, if $m$ is even}
    \\-a_{jl}e^{-ik_{jl}\tilde\theta}\in\R_{>0}
        & \text{, if $m$ is uneven}
    \end{cases}
  \end{align*}
  is, in the case when $n$ is even, also real and negative. In the other
  case, when $n$ is uneven, we use that $a_{jl}=-a_{lj}$ and $k_{jl}=k_{lj}$ to
  obtain
  $a_{lj}e^{-ik_{lj}\left(\tilde\theta+m\frac{\pi}{k_{lj}}\right)} \in\R_{<0}$.

  Thus, for $\tilde\theta':=\tilde\theta+m\frac{\pi}{k_{jl}}$ as determination
  of $\theta':=\theta+m\frac{\pi}{k_{jl}}$, we have $\theta'\in\A$ since
  \begin{itemize}
    \item $q_j \underset{\tilde\theta',\max}{\prec} q_l$ when $m$ is even or
    \item $q_l \underset{\tilde\theta',\max}{\prec} q_j$ when $m$ is uneven.
  \end{itemize}
\end{proof}

As a \rewrite{subgroup of the stalk at $\theta$} of the in
Definition~\ref{defn:StokesSheaf} defined Stokes sheaf $\Lambda(A^0)$ we
define the Stokes group as follows.
\begin{defn}\label{defn:stokesGroup}
  Define the \emph{Stokes group}
  \[
    \Sto_\theta(A^0):=
    \left\{\phi_\theta\in\Lambda_\theta(A^0)
      \mid \phi_\theta \text{ has maximal decay at } \theta
    \right\}
  \]
  whose elements are called \emph{Stokes germs}.
  \begin{s-rem}
    For $\theta\notin\A$ the group $\Sto_\theta(A^0)$ is trivial, since no flat
    isotropy has maximal decay, but the identity.

    \TODO[This is in fact a group, since\dots]
  \end{s-rem}
\end{defn}

%%%%%%%%%%%%%%%%%%%%%%%%%%%%%%%%%%%%%%%%%%%%%%%%%%%%%%%%%%%%%%%%%%%%%%%%%%%%%%%
\subsection{Stokes matrices}\label{sec:matrixReps}
\marginnote{\cite[9f]{thboalch}, \cite[??]{Loday1994}}
\begin{defn}\label{defn:groupOfFaithfullReps}
  Let us use
  \[
    \delta_{jl}:=
    \begin{cases}
      0 \in \C^{n_j\times n_l} & \text{,~if~} j\neq l
    \\\id \in \C^{n_j\times n_l} & \text{,~if~} j=l
    \end{cases}
  \]
  as a block version of Kronecker's delta.
  Define the group
  \begin{align*}
    \SSto_\theta(A^0)= \Big\{K=(K_{jl})_{j,l\in\{1,\dots,s\}}\in G \mid
      K_{jl}=\delta_{jl} \text{~unless~}
      q_j \underset{\theta,\max}{\prec} q_l \Big\}
  \end{align*}
  of all \emph{Stokes matrices} of $A^0$ in the direction $\theta$.
  They will arise as a faithful representation
  (cf.\ Section~\ref{sec:faithRepre}) of $\Sto_\theta(A^0)$.
\end{defn}
Let $\cY_0=t^L e^{Q(t^{-1})}$ be a normal solution corresponding to
$A^0$ (cf.\ Definition~\ref{defn:normSol}).
For every determination $\tilde\theta$ of $\theta$ we
get a function $e^{Q(1/|t|e^{i\tilde\theta})}$ as a realization of the formal
exponential $e^{Q(1/t)}$.
Let us denote by $\cY_{0,\tilde\theta}$ the function defined by $\cY_0$ with
that determination of the argument near the direction
$\theta$.\label{page:alreadyUsedDefn}

\begin{prop}\label{prop:representation}
  \marginnote{\cite[Def.I.4.7]{Loday1994}\\\cite[78f]{Loday2014}}
  In this situation is the morphism
  \begin{align*}
    \rho_{\tilde\theta}:\Sto_\theta(A^0)&\to\SSto_\theta(A^0)
    \\\phi_\theta
    &\mapsto
    C_{\cY_{0,\tilde\theta}}
    :=1+C_{\tilde\theta}
    :=\cY_{0,\tilde\theta}^{-1}\phi_\theta\cY_{0,\tilde\theta}
  \end{align*}
  an isomorphism which maps a germ of $\Sto_\theta(A^0)$ to the corresponding
  Stokes matrix $C_{\cY_{0,\tilde\theta}}$ such that
  \begin{equation} \label{eq:representation}
    \phi_\theta(t)\cY_{0,\tilde\theta}(t)
    =\cY_{0,\tilde\theta}(t)C_{\cY_{0,\tilde\theta}}
  \end{equation}
  near $\theta$.
  The matrix $C_{\cY_{0,\tilde\theta}}$ is then called a \emph{representation
  of $\phi_\theta$}.
  \begin{s-rem}
    \begin{enumerate}
      \item This morphism depends on the choice of the determination
        $\tilde\theta$ of $\theta$.
      % \item The inverse morphism of $\rho_{\tilde\theta}$ is given by
      %   \begin{align*}
      %     \rho_{\tilde\theta}^{-1}:
      %     \SSto_\theta(A^0)
      %     &\overset{\cong}\to
      %     \Sto_\theta(A^0)
      %     \\C_{\cY_{0,\tilde\theta}} &\mapsto
      %     \cY_{0,\tilde\theta}C_{\cY_{0,\tilde\theta}}\cY_{0,\tilde\theta}^{-1}
      %     \,.
      %   \end{align*}
      \item In Loday-Richaud's book \cite[78]{Loday2014} are the elements of
        $\Sto_\theta(A^0)$ characterized as the flat transformations, such that
        equation (\ref{eq:representation}) is satisfied for some unique
        constant invertible matrix
        $C_{\cY_{0,\tilde\theta}}\in\SSto_\theta(A^0)$.
      \item \marginnote{\cite[Defn.I.4.7]{Loday1994}}
        This construction defines also a morphism, which takes a germ
        $\phi_\theta\in\Lambda_\theta(A^0)$ relative to $\cY_{0,\tilde\theta}$
        into its unique representation matrix
        \[
          C_{\cY_{0,\tilde\theta}}
          \in\left\{K\in G\mid K_{jl}=\delta_{jl} \text{ unless }
          q_j \underset{\theta}{\prec} q_l \right\} \,.
        \]
    \end{enumerate}
  \end{s-rem}
\end{prop}
\begin{proof}
  It is well known (cf.\ \cite[10]{thboalch}), that the morphism
  $\rho_{\tilde\theta}$, i.e.\ conjugation by the fundamental solution,
  relates solutions $\phi_\theta$ of $[\End(A^0)]=[A^0,A^0]$ to solutions of
  $[0,0]$ which are the constant matrices $G$.
  Thus we have to show, that the image of $\Sto_\theta(A^0)$ under
  $\rho_{\tilde\theta}$ is $\SSto_\theta(A^0)$.

  To see that the obtained matrix has the necessary zeros, to lie in
  $\SSto_{\theta}(A^0)$ we look at equation (\ref{eq:representation}) and
  deduce
  \begin{equation}\label{eq:repProof1}
    \phi_\theta(t)
    =t^L e^{Q(t^{-1})}(1_n+C_{\tilde\theta})e^{-Q(t^{-1})}t^{-L}
  \end{equation}
  with the given choice of the argument near $\theta$.
  After decomposing $C_{\tilde\theta}$ into
  \begin{align*}
    C_{\tilde\theta}&=\begin{pmatrix}
      c_{(1,1)} & c_{(1,2)} & \cdots &\\
      c_{(2,1} & \ddots\\
      \vdots \\
      & & & c_{(s,s)}
    \end{pmatrix}
  \\&=
    \underset{C_{\tilde\theta}^{(1,1)}}{\underbrace{%
      \begin{pmatrix}
        c_{(1,1)} & 0 & \cdots &\\
        0\\
        \vdots&\\
        &
      \end{pmatrix}
    }}
    +
    \underset{C_{\tilde\theta}^{(1,2)}}{\underbrace{%
      \begin{pmatrix}
        0 & c_{(1,2)} & 0 & \cdots\\
        & 0 &\\
        &\vdots\\
        &
      \end{pmatrix}
    }}
    +\cdots+
    \underset{C_{\tilde\theta}^{(s,s)}}{\underbrace{%
      \begin{pmatrix}
        &\\
        & & & \vdots\\
        & & & 0\\
        & \cdots & 0 & c_{(s,s)}
      \end{pmatrix}
    }}
  \\&=\sum_{(l,j)}C_{\tilde\theta}^{(l,j)}
  \end{align*}
  where the $c_{(j,l)}$ are blocks\marginnote{One can ignore the block
  structure by using $1\times1$ sized blocks. But one looses the uniqueness of
  the $q_j$'s.} of size $n_j\times n_l$ which correspond to the structure of
  $Q$. After rewriting the Equation (\ref{eq:repProof1}) we get
  \[
    \phi_\theta=
      t^L\left(
        1_n+\sum_{(l,j)}C_{\tilde\theta}^{(l,j)}e^{(q_l-q_j)(t^{-1})}
      \right)t^{-L} \,.
  \]
  \begin{comment}
    \begin{align*}
      \phi_\theta(t)
      &=t^Le^{Q(t^{-1})}\left(
        1_n+C_{\tilde\theta}
      \right)e^{-Q(t^{-1})}t^{-L}
    \\&=t^Le^{Q(t^{-1})}\left(
        1_n+\sum_{(l,j)}C_{\tilde\theta}^{(l,j)}
      \right)e^{-Q(t^{-1})}t^{-L}
    \\&=t^L\left(
        1_n+\sum_{(l,j)}e^{Q(t^{-1})}C_{\tilde\theta}^{(l,j)}e^{-Q(t^{-1})}
      \right)t^{-L}
    \\&=t^L\left(
          1_n+\sum_{(l,j)}C_{\tilde\theta}^{(l,j)}e^{(q_l-q_j)(t^{-1})}
        \right)t^{-L} \,.
    \end{align*}
  \end{comment}
  Thus, for $\phi_{\theta}$ to be flat in direction $\theta$, it is
  necessary and sufficient that if $e^{(q_l-q_j)(t^{-1})}$ does not have
  maximal decay in direction $\tilde\theta$ the corresponding
  block $C_{\tilde\theta}^{(l,j)}$ vanishes.
  Thus we have seen, that $(1_n+C_{\tilde\theta})$ is an element of
  $\SSto_\theta(A^0)$.

  The map $\rho_{\tilde\theta}$ is \textbf{injective}, and thus a faithful
  representation, since conjugation by an invertible matrix defines an
  isomorphism to its \rewrite{image}.

  The \textbf{surjectivity} can also \rewrite{be seen easily}, since every
  constant (unipotent) matrix, with zeros at the necessary positions,
  characterizes a unique element of $\Sto_\theta(A^0)$.
  \TODO[Isomorphic to $\C^{?}$?]
\end{proof}

In particular, from the proof, we get that
\begin{enumerate}
  \item for $j=l$ the (diagonal) blocks $C_{\tilde\theta}^{(l,j)}$ vanish since
    $q_l-q_j=0$ does not have maximal decay and
  \item if $e^{q_j-q_l}$ has has maximal decay, then $e^{q_l-q_j}$ has not.
    Thus if $C_{\tilde\theta}^{(l,j)}$ is not equal to zero, the block
    $C_{\tilde\theta}^{(j,l)}$ is necessarily zero.
\end{enumerate}
\begin{rem}
  This implies that the matrix $1_n+C_{\tilde\theta}$ is unipotent, and
  \rewrite{hence is $\Sto_\alpha(A^0)$ is a unipotent Lie group.}
\end{rem}

\begin{cor}
  \marginnote{\cite[Def.I.4.12]{Loday1994}}
  A germ $\phi_\theta\in\Lambda_\theta(A^0)$ is a Stokes germ, i.e.\ an element
  in $\Sto_\theta(A^0)$, if and only if for some, hence all determinations
  $\tilde\theta$, it has a representation $1+C_{\tilde\theta}$ where
  \[
    C_{\tilde\theta}=\sum_{(l,j)\mid q_j\underset{\tilde\theta,\max}{\prec}q_l}
    C_{\tilde\theta}^{(l,j)}
  \]
  and $C_{\tilde\theta}^{(l,j)}$ have the necessary block structure.

  In other words:
  \begin{einr}
    a germ $\phi_\theta\in\Lambda_\theta(A^0)$ is in $\Sto_\theta(A^0)$ if and
    only if there exists a $C_{\tilde\theta}\in\SSto_\theta(A^0)$ such that
    $\phi_\theta=\cY_{0,\tilde\theta}C_{\tilde\theta}\cY_{0,\tilde\theta}^{-1}$
    in some representation $\tilde\theta$ of $\theta$.
  \end{einr}
\end{cor}

%%%%%%%%%%%%%%%%%%%%%%%%%%%%%%%%%%%%%%%%%%%%%%%%%%%%%%%%%%%%%%%%%%%%%%%%%%%%%%%
\subsection{Filtration of the Stokes group by levels}
\marginnote{\cite{Loday1994}, \cite[362ff]{Martinet1991}}
Let us introduce a couple of notations and definitions, which coincide with the
notations used in Loday-Richaud's paper \cite{Loday1994}.

\begin{defn}
  We denote the set of \emph{levels of the germ}
  $\phi_{\theta}\in\Sto_\theta(A^0)$ by
  \[
    \cK(\phi_\theta):= \left\{\deg(q_j-q_l)\mid C_{\tilde\theta}^{(l,j)}\neq0
      \text{ in some determination of }\phi_\theta\right\} \subset \cK \,.
  \]
  A germ $\phi_\theta$ is called a \emph{$k$-germ} when
  $\cK(\phi_{\theta})\subset\{k\}$, i.e.\ its only level, if present, is $k$.
\end{defn}
\begin{notations}
  \marginnote{\cite[Not.I.4.15]{Loday1994},\\\cite[362]{Martinet1991}}
  For $k\in\cK$ and $\theta\in\A$ we set:
  \begin{itemize}
    \item $\Lambda^{k}(A^0):=$ the subsheaf of $\Lambda(A^0)$ of all germs,
      which are generated by $k$-germs;
    \item $\Lambda^{\leq k}(A^0)$ (resp. $\Lambda^{<k}(A^0)$ or
      $\Lambda^{\geq k}(A^0)$) as the subsheaf of $\Lambda(A^0)$ generated by
      $k'$-germs for all $k'\leq k$ (resp. $k'<k$ or $k'\geq k$);
  \end{itemize}
  The restriction to $\Sto_\theta$ yields the groups
  \[
    \Sto_\theta^\star(A^0):=\Sto_\theta(A^0)\cap\Lambda_\theta^{\star}(A^0)
  \]
  for $\star\in\{k,<k,\leq k,\dots\}$.
  % \begin{itemize}
  %   \item $\Sto_\theta^k(A^0):=\Sto_\theta(A^0)\cap\Lambda_\theta^{k}(A^0)$;
  %   \item $\Sto_\theta^{\leq k}(A^0):=\Sto_\theta(A^0)\cap\Lambda_\theta^{\leq k}(A^0)$;
  %   \item $\Sto_\theta^{<k}(A^0):=\Sto_\theta(A^0)\cap\Lambda_\theta^{<k}(A^0)$;
  % \end{itemize}
  We define also
  \begin{itemize}
    \item $\SSto^\star_\theta(A^0)$ as the groups of representations, which
      correspond to elements of  $\Sto^\star_\theta(A^0)$;
    \item $\A^k:=\left\{\theta\in\A\mid\Sto_\theta^k(A^0)\neq\{\id\}\right\}$
      as the set of anti-Stokes directions \emph{bearing the level $k$};
    \item $\A^{\leq k}:=\underset{k'\leq k}\bigcup\A^{k'}$ (resp.
      $\A^{<k}:=\underset{k'<k}\bigcup\A^{k'}$ and
      $\A^{\geq k}:=\underset{k'\geq k}\bigcup\A^{k'}$);
      \begin{s-rem}
        It is clear that
        \begin{itemize}
          \item $\A^{\star}=\left\{\theta\in\A
            \mid\Sto_{\theta}^\star(A^0)\neq\{\id\}\right\}$ and
          \item we have the canonical inclusions
            $\A^k\hookrightarrow\A^{\leq k}$ and
            $\A^{<k}\hookrightarrow\A^{\leq k}$.
        \end{itemize}
      \end{s-rem}
    \item $\cK_\theta:=\left\{k\in\cK\mid\Sto_\theta^k(A^0)\neq\{\id\}\right\}$
      the set of levels \emph{beared by $\theta\in\A$}, and
      $K_\theta=\max\cK_\theta$ the \emph{maximal level beared by $\theta$}.
      \begin{s-rem}
        The Lemma~\ref{lem:rotationalSym} implies that from $k\in\cK_{\theta}$
        follows that $k\in\cK_{\theta+m\frac{\pi}{k}}$ for $m\in\N$.
      \end{s-rem}
  \end{itemize}
\end{notations}

\begin{comment}
  See \cite[I.5]{Loday1994} on p.\ 861f (See [LR91])
\end{comment}
\rewrite{We will now introduce a filtration of $\Lambda(A^0)$, wich will be
restricted to $\Sto_\theta(A^0)$ and defines there the filtration, in which we
are interested.}
\begin{prop}\label{prop:PropertiesOfStokesSheafSplitting}
  \marginnote{\cite[Prop.I.5.1]{Loday1994}}
  For any level $k\in\cK$ one has
  \begin{enumerate}
    \item $\Lambda^{k}(A^0)$, $\Lambda^{\leq k}(A^0)$ and $\Lambda^{<k}(A^0)$
      are sheaves of subgroups of $\Lambda(A^0)$;
    \item the sheaf $\Lambda^k(A^0)$ is normal in $\Lambda^{\leq k}(A^0)$;
      \begin{comment}
        A subgroup $N$ is normal in $G$ ($N\vartriangleleft G$) if it is stable
        under conjugation, i.e.
        \[
          N\vartriangleleft G \Leftrightarrow \forall n\in N \forall g\in G,
          gng^{-1}\in N ,.
        \]
      \end{comment}
    \item \marginnote{\cite[Proposition 10]{Martinet1991}}
      the exact sequence of sheaves
      \[
        1\longrightarrow\Lambda^k(A^0)
        \overset{i}\longrightarrow\Lambda^{\leq k}(A^0)
        \overset{p}\longrightarrow\Lambda^{<k}(A^0)
        \longrightarrow 1 \,,
      \]
      where
      \begin{itemize}
        \item $i:\Lambda^k(A^0)\hookrightarrow\Lambda^{\leq k}(A^0)$ is the
          canonical inclusion and
        \item $p$ is the truncation to terms of levels $<k'$,
      \end{itemize}
      splits.
  \end{enumerate}
\end{prop}
\begin{proof}
  \begin{comment}
    \begin{enumerate}
      \item \PROBLEM[clear, since germwise clear?]
    \end{enumerate}
  \end{comment}
  \begin{comment}
    \begin{enumerate}
      \setcounter{enumi}{1}
      \item
        \PROBLEM[Germwise or sectionwise??, check only on generators?]
        Let $\alpha$ be an direction in $S^1$.
        Let $n\in\Lambda^k(A^0)_\alpha$ and $g\in\Lambda^{\leq k}(A^0)_\alpha$.
        We have to show, that $gng^{-1}\in\Lambda^{\leq k}(A^0)_\alpha$,
        i.e.\ that $gng^{-1}$ is a $k$-germ.
        \TODO{}
    \end{enumerate}
  \end{comment}
  \begin{enumerate}
    \setcounter{enumi}{2}
    \item Splitting is clear, since we have the canonical inclusion
      $\Lambda^{<k}(A^0)\hookrightarrow\Lambda^{\leq k}(A^0)$ and
      $p\circ\substack{\text{the}\\\text{inclusion}}=\id_{\Lambda^{<k}(A^0)}$.
  \end{enumerate}
\end{proof}
\begin{cor}\label{cor:factorStokesGerms}
  \marginnote{\cite[Cor.I.5.2]{Loday1994}}
  \begin{enumerate}
    \item For any $k\in\cK$, there are the two following ways of factoring
      $\Lambda^{\leq k}(A^0)$ in a semi-direct product:
      \begin{align*}
        \Lambda^{\leq k}(A^0)&\cong \Lambda^{<k}(A^0)\ltimes\Lambda^{k}(A^0)
      \\                     &\cong \Lambda^{k}(A^0)\ltimes\Lambda^{<k}(A^0)\,.
      \end{align*}
      Thus any germ $f^{\leq k}\in\Lambda^{\leq k}(A^0)$ can be uniquely
      written as
      \begin{itemize}
        \item $f^{\leq k}=f^{<k}g^k$, where $f^{<k}\in\Lambda^{<k}$ and
          $g^k\in\Lambda^k$, or
        \item $f^{\leq k}=f^kf^{<k}$, where $f^k\in\Lambda^k$ and
          $f^{<k}\in\Lambda^{<k}$.
      \end{itemize}
      \begin{s-rem}\label{rem:algFactorization}
        \marginnote{\cite[Cor.I.5.2(ii)]{Loday1994}}
        We can get the factor $f^{<k}$ common to both factorizations by
        truncation of $f^{\leq k}$ to terms of level $k$.
        This truncation can explicitly be achieved in terms of Stokes matrices,
        by keeping in any representation $1+\sum C_{\tilde\theta}^{(j,l)}$ of
        $f^{\leq k}$ only the terms $C_{\tilde\theta}^{(j,l)}$ such that
        $\deg(q_j-q_l)<k$.

        A factorization algorithm could then be:
        \begin{einr}
          get the factor $f^{<k}$ common to both factorizations by truncation
          of $f^{\leq k}$ to terms of level $k$ and set
          $g^k:=(f^{<k})^{-1}f^{\leq k}$ and $f^k:=f^{\leq k}(f^{<k})^{-1}$.
        \end{einr}
      \end{s-rem}
    \item This decomposition in semi-direct product can be extended to all
      levels. Thus
      \[
        \Lambda(A^0)\cong\underset{k\in\cK_\theta}\bigltimes\Lambda^k(A^0)
      \]
      where the semi-direct product is taken in an ascending or descending
      order of levels $k$.
  \end{enumerate}
\end{cor}
\begin{rem}
  Loday-Richaud states in his paper \cite[Prop.I.5.3]{Loday1994} the following
  proposition.
  \begin{s-prop}
    \marginnote{\cite[Prop.I.5.3]{Loday1994}}
    For any levels $k$,$k'\in\cK$ with $k'<k$ one has:
    \begin{enumerate}
      \item the sheaf $\Lambda^{\geq k'}(A^0)\cap\Lambda^{\leq k'}(A^0)$ is
        normal in $\Lambda^{\leq k}(A^0)$;
      \item \marginnote{\cite[Proposition 10]{Martinet1991}}
        the exact sequence of sheaves
        \[
          1\longrightarrow\Lambda^{\geq k'}(A^0)\cap\Lambda^{\leq k'}(A^0)
          \overset{i}\longrightarrow\Lambda^{\leq k}(A^0)
          \overset{p}\longrightarrow\Lambda^{<k'}(A^0)
          \longrightarrow 1 \,,
        \]
        where
        \begin{itemize}
          \item $i$ is the canonical inclusion and
          \item $p$ is the truncation to terms of levels $<k'$,
        \end{itemize}
        splits.
    \end{enumerate}
    \TODO[is $\Lambda^{\geq k'}(A^0)\cap\Lambda^{\leq k}(A^0)=\Lambda^k(A^0)$
    and thus the first proposition a corollary of this?]
  \end{s-prop}
  We can use this proposition to follow (cf.\ \cite[Cor.I.5.4]{Loday1994}) that
  \begin{enumerate}
    \item
      the filtration
      \[
        \Lambda^{k_r}(A^0)
        =
        \Lambda^{\geq k_r}(A^0)
        \subset
        \Lambda^{\geq k_{r-1}}(A^0)
        \subset
        \cdots
        \subset
        \Lambda^{\geq k_{1}}(A^0)
        =
        \Lambda(A^0)
      \]
      is normal and
    \item we can use this to achieve the decomposition
      \[
        \Lambda(A^0)\cong\underset{k\in\cK_\theta}\bigltimes\Lambda^k(A^0)
      \]
      taken in an arbitrary order. In fact, one can also extend the algorithm
      from Remark~\ref{rem:algFactorization} to an arbitrary order of levels.
  \end{enumerate}
\end{rem}
\begin{prop}\label{prop:filtrationOfStokesGroup}
  \marginnote{\cite[Prop.I.5.5]{Loday1994}}
  The results can be restricted to the Stokes groups. Thus, for
  $\theta\in\A$, one has
  \[
    \Sto_\theta(A^0)\cong\underset{k\in\cK_\theta}\bigltimes\Sto_\theta^k(A^0)
  \]
  the semi-direct product being taken in an arbitrary order \comm{(we will only
  be interested in the ascending order)}.
  \begin{s-defn}
    We will denote the map, wich gives the factors of this factorization by
    \[
      i_\theta:
      \Sto_\theta(A^0)
      \overset{\cong}\longrightarrow
      \prod_{k\in\cK_\theta}\Sto_\theta^k(A^0)\,,
    \]
    where the factorization is taken in ascending order.
  \end{s-defn}
  \begin{s-rem}\label{rem:filtrationOfStokesMats}
    Write $\rho_{\tilde\theta}^k:\Sto^k_\theta(A^0)\to\SSto^k_\theta(A^0)$ for
    the \rewrite{restriction} of the map $\rho_{\tilde\theta}$
    (cf.\ Proposition~\ref{prop:representation}) to the level $k$.
    \rewrite{Then, one can} denote the induced decomposition also by
    \[
      i_\theta:
      \SSto_\theta(A^0)
      \overset{\cong}\longrightarrow
      \underset{k\in\cK_\theta}\bigltimes\SSto_\theta^k(A^0)
    \]
    and the corresponding diagram
    \[ \begin{tikzcd}
        \Sto_\theta(A^0)
        \rar{i_\theta}
        \dar{\rho_{\tilde\theta}}
        & \prod_{k\in\cK_\theta}\Sto_\theta^k(A^0)\,,
        \dar{\prod_{k\in\cK}\rho_{\tilde\theta}^k}
      \\\SSto_\theta(A^0)
        \rar{i_\theta^k}
        & \prod_{k\in\cK_\theta}\SSto_\theta^k(A^0)\,,
    \end{tikzcd} \]
    commutes.
  \end{s-rem}
\end{prop}
\begin{comment}
  \begin{proof}
    \TODO{}
  \end{proof}
\end{comment}

%%%%%%%%%%%%%%%%%%%%%%%%%%%%%%%%%%%%%%%%%%%%%%%%%%%%%%%%%%%%%%%%%%%%%%%%%%%%%%%
\section{Stokes structures: using Stokes matrices}\label{sec:mainThm2}
\marginnote{\cite{Loday1994}, \cite[Thm.4.3.11]{Loday2014}
  \\and \cite{boalch,thboalch}
  \\and \cite{babbitt1989local}
  \\and \cite{BJL1979Birkhoff}}
The goal in this section is to prove that there is a bijective and natural map
\[
  h:\prod_{\theta\in\A}\Sto_\theta(A^0)\to\St(A^0) \,.
\]
And since $\Sto_\theta(A^0)$ has a faithful representation $\SSto_\theta(A^0)$
we also get
\[
  \prod_{\theta\in\A}\SSto_\theta(A^0)\cong\St(A^0)
\]
as a corollary.
\TODO[This goes back to \cite{BJL1979Birkhoff}?]

Let us recall, that $\St(A^0)$ is defined to be $H^1(S^1;\Lambda^{<0}(^0))$
(cf.\ Section~\ref{sec:mainThm1}).
The elements of $\prod_{\theta\in\A}\Sto_\theta(A^0)$ define in a canonical way
cocycles of the sheaf $\Lambda^{<0}(A^0)$ (cf.\ Remark~\ref{rem:mapStoToCocy}),
called Stokes cocycles (cf.\ Definition~\ref{defn:stokesCocycle}).
In fact, will $h$ map such cocycles to the cohomology class, to which they
correspond.
Thus the statement, that $h$ is a bijection, is equivalent to the statement
that
\begin{einr}
  in each cohomology class of $\St(A^0)$ is an unique $1$-cocycle, which is a
  Stokes cocycle.
\end{einr}

%%%%%%%%%%%%%%%%%%%%%%%%%%%%%%%%%%%%%%%%%%%%%%%%%%%%%%%%%%%%%%%%%%%%%%%%%%%%%%%
\subsection{The theorem}
\begin{comment}
  \begin{lem}
    Since the sections of the sheaf $\Lambda(A^0)$ are solutions of the system
    $[A^0,A^0]$ (cf.\ Definition~\ref{defn:StokesSheaf}),
    the theory of differential equations\PROBLEM[source?] tells us that
    \rewrite{sections extend uniquely}.
    Thus we can extend any germ $\phi_\theta\in\Lambda_\theta(A^0)$ uniquely to
    a section $\phi\in\Gamma(\Omega;\Lambda(A^0))$, where $\Omega$ is the
    maximal arc of definition around $\theta$.

    \PROBLEM[The flatness proprerty implicitly satisfied?]
  \end{lem}
\end{comment}
\begin{rem}\label{rem:mapStoToCocy}
  \marginnote{\cite[868]{Loday1994}}
  Let $\{\theta_j\mid j\in J\}\subset S^1$ be a finite set and
  $\dot\phi=(\dot\phi_{\theta_j})_{j\in J}
  \in\prod_{j\in J}\Lambda_{\theta_j}(A^0)$ be a finite family of germs.
  In the following way, one can associate a cohomology class in $\St(A^0)$ to
  any $\dot\phi$:
  \begin{einr}
    let $\dot\phi_j$ be the function representing the germ
    $\dot\phi_{\theta_j}$ on its maximal arc of definition $\Omega_j$ around
    $\theta_j$.
    Then, for every cyclic covering $\cU=\{U_j;j\in J\}$ which satisfies
    $\dot U_j\subset \Omega_j$ for all $j\in J$, one can define the $1$-cocycle
    $(\dot\phi_{j|\dot U_j})_{j\in J}$ on $\cU$.
  \end{einr}
  To a different $\cU$ this construction yields a cohomologous
  $1$-cocycle\PROBLEM[Proof], thus the induced map
  \[
    \prod_{\alpha\in J}\Sto_\alpha(A^0)\to H^1(S^1;\Lambda(A^0))=\St(A^0)
  \]
  is welldefined.
\end{rem}
\begin{comment}
  \cite[Thm.6.3.1]{sibuya1990Linear} says: if two differential equations have
  the same stokes phenomenon, they are analytically equivalent.
\end{comment}
\begin{defn}\label{defn:stokesCocycle}
  \TODO[Why only germ? Sections?]
  \marginnote{\cite[Def.II.1.8]{Loday1994}, \cite[4.3.10]{Loday2014}}
  A \emph{Stokes cocycle} is a 1-cocycle $(\phi_\theta)_{\theta\in\A}$ of
  $\Lambda(A^0)$ which is
  \begin{itemize}
    \item indexed by the set of anti-Stokes directions and
    \item each component $\phi_\theta$ determines an element of
      $\Sto_\theta(A^0)$.
  \end{itemize}
  \begin{s-rem}
    The set of all Stokes cocycles can be identified with
    $\prod_{\alpha\in\A}\Sto_\alpha(A^0)$.
  \end{s-rem}
\end{defn}
We can use Remark~\ref{rem:mapStoToCocy} to obtain in the case $J=\A$ a
mapping
\[
  h:\prod_{\alpha\in\A}\Sto_\alpha(A^0)\to\St(A^0),
\]
which takes a Stokes cocycle to its corresponding cohomology class.
\begin{tthm}
  \label{thm:mainThm2}
  The map
  \[
    h:\prod_{\alpha\in\A}\Sto_\alpha(A^0)\to\St(A^0)
  \]
  is a bijection and natural.
  \begin{s-rem}
    \marginnote{\cite[869]{Loday1994},\cite[Sec.III.3.3]{Loday1994}}
    Natural means that $h$ commutes to isomorphisms and constructions over
    systems or connections they represent.
  \end{s-rem}
\end{tthm}
From theorem~\ref{thm:mainThm2} and Proposition~\ref{prop:representation} we
get the following corollary.
\begin{cor}
  Using the isomorphisms from the Stokes groups to the group of all Stokes
  matrices (cf.\ Proposition~\ref{prop:representation}) we obtain
  \[
    \St(A^0) \cong \prod_{\theta\in\A}\SSto_\theta(A^0)
  \]
  and thus $\St(A^0)$ has complex dimension
  $\dim_\C\St(A^0)=\sum_{\theta\in\A}\dim_\C\SSto_\theta(A^0)$.
\end{cor}
\begin{rem}
  The set $\prod_{\alpha\in\A}\Sto_\alpha(A^0)$ and thus
  $\prod_{\alpha\in\A}\SSto_\alpha(A^0)$ has some bad properties, when small
  deformations are applied to $[A^0]$, since under arbitrary small changes, one
  Stokes ray can split into two. Boalch~\cite{boalch,thboalch} solves this in
  the single-leveled case by \rewrite{renaming our Stokes matrices as Stokes
  factors} and introducing new Stokes matrices which are build by the product
  of consecutive Stokes factors. He uses the rotational symmetry of the
  single-leveled case \rewrite{thus this approach is not} applicable to the
  multi-leveled case.
  \comm{But one can \dots.}
\end{rem}

%%%%%%%%%%%%%%%%%%%%%%%%%%%%%%%%%%%%%%%%%%%%%%%%%%%%%%%%%%%%%%%%%%%%%%%%%%%%%%%
\subsection{Proof}\label{sec:proofOfMatrixThm}
We will only look at the unramified case, for which we refer to
\cite[Sec.II.3]{Loday1994}.
The proof in the ramified case can be found in \cite[Sec.II.4]{Loday1994}.
We first have to introduce adequate coverings, which will be used in the proof.

%%%%%%%%%%%%%%%%%%%%%%%%%%%%%%%%%%%%%%%%%%%%%%%%%%%%%%%%%%%%%%%%%%%%%%%%%%%%%%%
\subsubsection{Coverings}
\marginnote{\cite[Sec.II.1]{Loday1994} and \cite[Sec.II.3.1]{Loday1994}}
A covering $\cV$ is said to \emph{refine} a covering $\cU$ if, to each open set
$V\in\cV$ there is at least one $U\in\cU$ with $V\subset U$.
\begin{defn}
  \begin{enumerate}
    \item A covering $\cU=\{U_j;j\in J\}$ is called \emph{cyclic covering} if
      \begin{enumerate}
        \item the set $J$ is finite and cyclic $J=\Z/\nu\Z$;
        \item the $U_j$ and, if $\#J>2$, the $U_j\cap U_{j+1}$ are connected
          arcs on $S^1$;
        \item the bisecting directions of the $U_j$ are in ascending order with
          respect to the clockwise orientation of $S^1$;
        \item the $U_j$ are not encased, this means that the arcs
          $U_j\backslash U_l$ are connected arcs for all $j,l\in J$.
      \end{enumerate}
    \item The \emph{nerve} of a cyclic covering $\cU=\{U_j;j\in J\}$ is the
      family $\dot\cU=\{\dot U_j;j\in J\}$ defined by:
      \begin{itemize}
        \item $\dot U_j=U_j\cap U_{j+1}$ when $\#J>2$,
        \item $\dot U_1$ and $\dot U_2$ the connected components of
          $U_1\cap U_2$ when $\#J=2$.
      \end{itemize}
  \end{enumerate}
\end{defn}
The cyclic coverings correspond one-to-one to nerves of cyclic coverings.
\begin{prop}
  \marginnote{\cite[Prop.II.1.3]{Loday1994}}
  The covering $\cV$ refines $\cU$ if and only if the corresponding nerves
  $\dot\cU=\{\dot U_j\}$ and $\dot\cV=\{\dot V_l\}$ satisfy
  \begin{einr}
    each $\dot U_j$ contains at least one $\dot V_l$.
  \end{einr}
\end{prop}

\begin{defn}
  A covering $\cU$ beyond which the inductive limit
  $\underset{\cU}{\underrightarrow{\lim}}H^1(\cU;\Lambda(A^0))$ is stationary
  is said to be \emph{adequate} to describe $H^1(S^1;\Lambda(A^0))$ or
  \emph{adequate} to $\Lambda(A^0)$.
\end{defn}

\begin{prop}\label{prop:adeqCovCondition}
  \marginnote{\cite[Prop.II.1.7]{Loday1994}}
  Let $\theta\in\A$ and $k\in\cK_\theta$.
  \begin{s-defn}
    An arc $U(\theta,\frac{\pi}{k})$ is called a \emph{Stokes arc of level $k$
    at $\theta$}.
  \end{s-defn}
  A cyclic covering $\cU=(U_j)_{j\in J}$, which satisfies
  \begin{einr}
    each Stokes arc of level $k$ (resp.\ of level $\leq k$ or of any level)
    contains at least one arc $\dot U_j$ from the nerve $\dot\cU$ of $\cU$
  \end{einr}
  is adequate to $\Lambda^k(A^0)$ (resp.\ to $\Lambda^{\leq k}(A^0)$ or to
  $\Lambda(A^0)$).
\end{prop}
\begin{comment}
  \begin{proof}
    \TODO{}
  \end{proof}
\end{comment}

%%%%%%%%%%%%%%%%%%%%%%%%%%%%%%%%%%%%%%%%%%%%%%%%%%%%%%%%%%%%%%%%%%%%%%%%%%%%%%%
\subsubsection{Adequate coverings}
Let $k\in\cK$.
We want to define the three cyclic coverings $\cU^{k}$, $\cU^{\leq k}$ and
$\cU^{<k}$ which will be adequate to $\Lambda^k(A^0)$, $\Lambda^{\leq k}(A^0)$
and $\Lambda^{<k}(A^0)$. Furthermore will the coverings be comparable at the
different levels.

\begin{enumerate}
  \item The first covering $\cU^{k}=\{\dot U_\theta^k\mid\theta\in\A^k\}$
    is the cyclic covering with nerve
    \[
      \dot\cU^k:=
      \left\{\dot U_\theta^k=U(\theta,\frac{\pi}{k})\mid\theta\in\A^k\right\}
    \]
    consisting of all Stokes arcs of level $k$ for anti-Stokes directions
    bearing the level $k$.
\end{enumerate}

If we extend to several levels, the set
$\bigcup_k\left\{U(\theta,\frac{\pi}{k})\mid\theta\in\A^{k}\right\}$ is no
longer a nerve.
Hence we have to define the coverings $\cU^{\leq k}$ and $\cU^{<k}$ in a
different way.
Denote by
\[
  \left\{K_1<\cdots<K_s=k\right\}
  =\left\{\max\left(\cK_\theta\cap[0,k]\right)\mid\theta\in\A^{\leq k}\right\}
\]
the set of all \emph{$k$-maximum levels} for $\theta\in\A^{\leq k}$.
\begin{enumerate}
  \setcounter{enumi}{1}
  \item The cyclic covering
    $\cU^{\leq k}=\left\{U_\theta^{\leq k}\mid\theta\in\A^{\leq k}\right\}$
    will be defined by induction.
    Let us assume that
    \begin{einr}
      the $\dot U_\theta^{\leq k}$ are defined for all $\theta\in\A^{\leq k}$
      with $k$-maximum level greater than $K_i$ such that their complete family
      is a nerve.
    \end{einr}
    Let
    \begin{itemize}
      \item $\theta$ be a anti-Stokes direction with $k$-maximum level $K_i$
        and
      \item $\theta^-$ (resp.\ $\theta^+$) be the next anti-Stokes direction
        with $k$-maximum level greater then $K_i$ on the left (resp.\ on the
        right) and define $\dot U_{\theta^-,\theta^+}$ as the clockwise hull of
        the arcs $\dot U_{\theta^-}^{\leq k}$ and $\dot U_{\theta^+}^{\leq k}$.
        If there are no anti-Stokes directions with $k$-maximum level greater
        then $K_i$ we set $\dot U_{\theta^-,\theta^+}=S^1$.
    \end{itemize}
    We then set
    \[
      \dot U_\theta^{\leq k}
        :=U\left(\theta,\frac{\pi}{K_i}\right)\cap\dot U_{\theta^-,\theta^+}
    \]
    and the family of all $\dot U_\theta^{\leq k}$ is a nerve.\PROBLEM[proof!]
    \begin{rem}
      If $\theta$ has a $k$-maximum level equal to $k$ then is
      $\dot U_\theta^{\leq k}$ equal to the Stokes arc
      $U\left(\theta,\frac{\pi}{k}\right)=\dot U_\theta^k$.
      \comm{\dots{}and then no $0$-cochain with level $k$ or $\geq k$ can
      exists on the covering $\cU^{\leq{k}}$.}
    \end{rem}
\end{enumerate}

\begin{enumerate}
  \setcounter{enumi}{2}
\item The last cyclic covering,
  $\cU^{<k}=\left\{U_{\theta}^k\mid\theta\in\A^{<k}\right\}$, is defined as
  $\cU^{<k}:=\cU^{\leq k'}$ where $k':=\max\{k''\in\cK\mid k''<k\}$.
\end{enumerate}

\begin{figure} %{{{
  \centering

  \def\kOne{7}
  \def\kTwo{10}
  \begin{tikzpicture}[scale=5] %{{{

    \newcommand{\myDrawArcA}[4]{%{{{
      % Parameter: radius , center , width , color
      \pgfmathsetmacro\hwdth{#3 / 2}
      \draw[ultra thick,#4]
        ({cos( #2 )},{sin( #2 )})
        --
        ({cos( #2 ) * #1 },{sin( #2 ) * #1 });
      \draw[thick,#4]
        ({cos(#2 - \hwdth) * #1},{sin(#2 - \hwdth) * #1})
        arc
        ({#2 - \hwdth}:{#2 + \hwdth}:#1);
      \draw[dotted,#4]
        ({cos(#2 - \hwdth) * #1},{sin(#2 - \hwdth) * #1})
        --
        ({cos(#2 - \hwdth)},{sin(#2 - \hwdth)});
      \draw[dotted,#4]
        ({cos(#2 + \hwdth) * #1},{sin(#2 + \hwdth) * #1})
        --
        ({cos(#2 + \hwdth)},{sin(#2 + \hwdth)});
      \filldraw[white] ({cos(#2)},{sin(#2)}) circle (0.5pt);
      \filldraw[red] ({cos(#2)},{sin(#2)}) circle (0.2pt);
    }%}}}
    \newcommand{\myDrawArcB}[5]{%{{{
      % Parameter: radius , center , start , stop , color
      \draw[ultra thick,#5]
        ({cos( #2 )},{sin( #2 )})
        --
        ({cos( #2 ) * #1 },{sin( #2 ) * #1 });
      \draw[thick,#5]
        ({cos( #3 ) * #1},{sin( #3 ) * #1})
        arc
        ({ #3 }:{ #4 }:#1);
      \draw[dotted,#5]
        ({cos( #3 ) * #1},{sin( #3 ) * #1})
        --
        ({cos( #3 )},{sin( #3 )});
      \draw[dotted,#5]
        ({cos( #4 ) * #1},{sin( #4 ) * #1})
        --
        ({cos( #4 )},{sin( #4 )});
      \filldraw[white] ({cos(#2)},{sin(#2)}) circle (0.5pt);
      \filldraw[red] ({cos(#2)},{sin(#2)}) circle (0.2pt);
    }%}}}

    \node (zero) at (0,0) {};
    \draw (zero) circle (1cm);

    %%%%%%%%%%%%%%%%%%%%%%%%%%%%%%%%%%%%%%%%%%%%%%%%%%%%%%%%%%%%%%%%%%%%%%%%%
    %% Inner:
    \foreach \n/\mycol/\baseR in {\kOne/orange/0.8
                                 ,\kTwo/purple/0.9}
    {%{{{
      \pgfmathsetmacro\wdth{180/\n}
      \foreach \i in {1,2,...,\n}}}

    %%%%%%%%%%%%%%%%%%%%%%%%%%%%%%%%%%%%%%%%%%%%%%%%%%%%%%%%%%%%%%%%%%%%%%%%%
    %% Outer:
    \def\mycol{brown}
    \pgfmathsetmacro{\wdth}{180/\kTwo}
    \foreach \i in {1,2,...,\kTwo}{%
      \pgfmathsetmacro\r{{1.05 + mod(\i+1,2)*0.05}}
      \pgfmathsetmacro\angl{\i* \wdth}
      \myDrawArcA{\r}{\angl}{\wdth}{\mycol}

      \pgfmathsetmacro\r{{1.05 + mod(\i + mod(\kTwo+1,2),2)*0.05}}
      \pgfmathsetmacro\angl{\i* \wdth+180}
      \myDrawArcA{\r}{\angl}{\wdth}{\mycol}
    }
    \foreach \i in {1,2,...,\kOne}{%
      \foreach \j in {1,...,\kTwo}{%
        \pgfmathparse{\i/\kOne <= \j/\kTwo ? 0 : 1}
        \ifnum\pgfmathresult=0{%
          \pgfmathparse{\i/\kOne < \j/\kTwo ? 0 : 1}
          \ifnum\pgfmathresult=0{%
            \pgfmathsetmacro\r{{1.15 + mod(\i+1,2)*0.05}}
            \pgfmathsetmacro\center{\i/\kOne*180}
            \pgfmathsetmacro\start{(\i-0.5)*180/\kOne > (\j-1.5)*180/\kTwo
              ? (\i-0.5)*180/\kOne : (\j-1.5)*180/\kTwo}
            \pgfmathsetmacro\stop{(\i+0.5)*180/\kOne > (\j+0.5)*180/\kTwo
              ? (\j+0.5)*180/\kTwo : (\i+0.5)*180/\kOne}
            \myDrawArcB{\r}{\center}{\start}{\stop}{\mycol}

            \pgfmathsetmacro\r{{1.15 + mod(\i,2)*0.05}}
            \pgfmathsetmacro\center{\center+180}
            \pgfmathsetmacro\start{\start+180}
            \pgfmathsetmacro\stop{\stop+180}
            \myDrawArcB{\r}{\center}{\start}{\stop}{\mycol}
          }\fi
          \breakforeach
        }\fi
      }
    }

    \fill[white] (zero) circle (1.5pt);
    \fill (zero) circle (.7pt);
  \end{tikzpicture} %}}}
  \caption{The adequate coverings for an example with $\cK=\{\kOne,\kTwo\}$ and
    $\A=\left\{ \frac{j\cdot\pi}{k}\mid k\in\cK\text{, } j\in\N \right\}$.
    The anti-Stokes directions are marked by the \textcolor{red}{red} dots.
    The arcs of $\dot\cU^{\kOne}=\dot\cU^{\leq\kOne}$ are
    \textcolor{orange}{orange}, the arcs of $\dot\cU^{\kTwo}$ are
    \textcolor{purple}{purple} and the arcs of $\dot\cU^{\leq\kTwo}=\dot\cU$
    are \textcolor{brown}{brown}.
  }\label{fig:adequateCovering}
\end{figure}%}}}

\begin{rem}
  The coverings $\cU^{k}$, $\cU^{\leq k}$ and $\cU^{<k}$ depend only on
  $\cQ_{[\End A^0]}$. Hence they depend only on $\cQ_{[A^0]}$.
\end{rem}
It is obvious, that for every $k\in\cK$ the covering $\cU^{\leq k}$ refines
$\cU^{k}$ and $\cU^{<k}$.
Furthermore are the coverings defined, such that they satisfy the
condition in Proposition~\ref{prop:adeqCovCondition}.
Thus the \rewrite{properties in} the following proposition \rewrite{are
satisfied}.
\begin{prop}\label{prop:adequateProperties}
  \marginnote{\cite[Prop.II.3.1]{Loday1994}}
  Let $k\in\cK$, then
  \begin{enumerate}
    \item the coverings $\cU^{k}$, $\cU^{\leq k}$ and $\cU^{<k}$ are adequate
      to $\Lambda^k(A^0)$, $\Lambda^{\leq k}(A^0)$ and $\Lambda^{<k}(A^0)$;
    \item there exists no $0$-cochain in $\Lambda^k(A^0)$ on $\cU^k$;
    \item on $\cU^{\leq k}$ there is no $0$-cochain of level $k$, i.e.\ belong
      all $0$-cochains of  $\cU^{\leq k}$ to $\cU^{<k}$
      (cf. Proposition \cite[Prop.II.3.1 (iv)]{Loday1994}).
  \end{enumerate}
\end{prop}
\begin{comment}
  \begin{proof}
    \begin{enumerate}
      \item This is clear, since the coverings are build to satisfy the
        condition from proposition~\ref{prop:adeqCovCondition}.
      \item \PROBLEM{}
      \item \PROBLEM{}
    \end{enumerate}
  \end{proof}
\end{comment}
To have a shorter notation, we denote the product
$\prod_{\theta\in\A^\star}\Gamma(\dot U_\theta^\star;\Lambda^\star(A^0))$ by
$\Gamma(\dot U^\star;\Lambda^\star(A^0))$ for $\star\in\{k,<k,\leq k,\dots\}$.

%%%%%%%%%%%%%%%%%%%%%%%%%%%%%%%%%%%%%%%%%%%%%%%%%%%%%%%%%%%%%%%%%%%%%%%%%%%%%%%
\subsubsection{The case of a unique level}
\marginnote{\cite[II.3.2]{Loday1994}}
First we will proof theorem~\ref{thm:mainThm2} in the case of a unique level.
This means that
\begin{itemize}
  \item either $\Lambda(A^0)$ has only one level $k$, thus
    \begin{itemize}
      \item $\Lambda(A^0)=\Lambda^k(A^0)$ and
      \item $\Sto_\theta(A^0)=\Sto_\theta^k(A^0)$,
    \end{itemize}
  \item or we restrict to a given level $k\in\cK$.
\end{itemize}
\begin{lem}
  Let $k\in\cK$.
  The morphism $h$ from Theorem~\ref{thm:mainThm2} is in the case of an unique
  level build as
  \[ \begin{tikzcd}[row sep=0cm]
    \underset{\theta\in\A}\prod\Sto_\theta^k(A^0)
    \rar{i^k}
    & \Gamma(\dot\cU^k;\Lambda^k(A^0))
    \rar{s^k}
    & H^1(S^1;\Lambda^k(A^0))
    \\
    \text{\rotatebox[origin=c]{-90}{$=$}}
    &&
    \text{\rotatebox[origin=c]{-90}{$=$}}
    \\
    \underset{\theta\in\A}\prod\Sto_\theta(A^0)
    \arrow{rr}{h}
    &
    & H^1(S^1;\Lambda(A^0))
  \end{tikzcd} \]
  from
  \begin{itemize}
    \item the canonical injective map
      \[
        i^k:\underset{\theta\in\A}\prod\Sto_\theta^k(A^0) \to
        \Gamma(\dot\cU^k;\Lambda^k(A^0)) \,,
      \]
      i.e.\ the map which is the canonical extension of germs to their natural
      arc of definition, and
    \item the quotient map
      \[
        s^k:\Gamma(\dot\cU^k;\Lambda^k(A^0))\to H^1(S^1;\Lambda^k(A^0))
      \]
  \end{itemize}
  which are both bijections.
\end{lem}
\begin{proof}
  \begin{enumerate}
    \item The map
      \[
        i^k: \underset{\theta\in\A}\prod\Sto_\theta^k(A^0)
        \to
        \underset{\Gamma(\dot\cU^k;\Lambda^k(A^0))}{%
          \underset{\text{\rotatebox[origin=c]{-90}{$=$}}}{%
            \underbrace{%
              \prod_{\theta\in\A^k}\Gamma(\dot\cU_\theta^k;\Lambda^k(A^0))
            }
        }}
      \]
      is welldefined, since the sections of $\Lambda^k(A^0)$ are solution of
      the system $[A^0,A^0]$ and we know from the theory of differential
      equations\TODO[source?] that an element
      $f_\theta\in\Gamma(\dot\cU_\theta^k;\Lambda^k(A^0))$ is uniquely
      determined as the extension of its germ at some point $\theta$.

      It is a group isomorphism, since
      \PROBLEM[$\Sto_\theta^k(A^0)\subsetneq\Lambda_\theta^k(A^0)$ SEE bjl p.72]
      \begin{comment}
        Problems:
        \begin{itemize}
          \item \PROBLEM[$\Sto_\theta^k(A^0)\subsetneq\Lambda_\theta^k(A^0)$]
            and thus
            \[
              i^k:
              \underset{\theta\in\A}\prod\Sto_\theta^k(A^0)
              \subsetneq
              \underset{\theta\in\A}\prod\Lambda_\theta^k(A^0)
              \to
              \prod_{\theta\in\A^k}\Gamma(\dot\cU_\theta^k;\Lambda^k(A^0))
            \]
        \end{itemize}
        IDEAS:
        \begin{itemize}
          \item Are sections fully determined by there germs at some points?
            \begin{itemize}
              \item since sections are defined by differential equations??

                ODE-Theory, says that a germ defines a solution in a
                \textbf{neighbourhood}
              \item \PROBLEM[why is the neighbourhood large enough?] I.e.\
                \[
                  \substack{the\\neighbourhood}\supset\cU_\theta^k \,?
                \]
            \end{itemize}
          \item $U_\alpha^k$ is the largest arc, to contain no corresponding
            Stokes ray 
            \begin{itemize}
              \item Then might \cite[Lemma 1]{BJL1979Birkhoff} on p.\ 73 help
            \end{itemize}
          \item maybe follows from \cite[Prop.1.24]{thboalch}
          \item \textbf{See \cite{babbitt1989local}}
        \end{itemize}
      \end{comment}
    \item The second map
      \[
        s^k:
        \overset{\Gamma(\dot\cU^k;\Lambda^k(A^0))}{%
          \overset{\text{\rotatebox[origin=c]{-90}{$=$}}}{%
            \overbrace{%
              \prod_{\theta\in\A^k}\Gamma(\dot\cU_\theta^k;\Lambda^k(A^0))
            }
        }}
        \to
        H^1(S^1;\Lambda^k(A^0))
      \]
      is a bijection, since from proposition~\ref{prop:adequateProperties} we
      know that it is
      \begin{itemize}
        \item \textbf{surjective}, since $\cU^k$ is adequate to
          $\Lambda^k(A^0)$ and
        \item \textbf{injective}, since on $\cU^k$ there is no $0$-cochain in
          $\Lambda^k(A^0)$.
      \end{itemize}
  \end{enumerate}
  \PROBLEM[Show that this is the correct $h$]
  \PROBLEM[Naturality?]
\end{proof}

%%%%%%%%%%%%%%%%%%%%%%%%%%%%%%%%%%%%%%%%%%%%%%%%%%%%%%%%%%%%%%%%%%%%%%%%%%%%%%%
\subsubsection{The case of several levels}
\rewrite{In the proof of the case of several levels, we will still use}
Loday-Richaud's paper~\cite{Loday1994} as reference.
\begin{defn}
  Here we want to define a \emph{product map of cocycles}
  $\mathfrak{S}^\leq k$.
  This map will be composed from the following injective maps:
  \begin{enumerate}
    \item The first map is defined as
      \begin{align*}
        \sigma^k:\Gamma(\dot\cU^k;\Lambda^k(A^0))
        &\to \Gamma(\dot\cU^{\leq k};\Lambda^{\leq k}(A^0))
      \\\dot f=(\dot f_\theta)_{\theta\in\A^k}
        &\mapsto (\dot G_\theta)_{\theta\in\A^{\leq k}}
      \end{align*}
      where
      \begin{itemize}
        \item $\dot G_\theta=\begin{cases}
            \dot f_\alpha \text{~restricted to } \dot U_\theta^{\leq k}
            \text{~and seen as being in } \Lambda^{\leq k}(A^0)
            & \text{~when } \theta\in\A^k
          \\\id \text{~(the identity) }
            & \text{~when } \theta\notin\A^k
          \end{cases}$
      \end{itemize}
    \item and the second map
      \begin{align*}
        \sigma^{<k}:\Gamma(\dot\cU^{<k};\Lambda^{<k}(A^0))
        &\to \Gamma(\dot\cU^{\leq k};\Lambda^{\leq k}(A^0))
      \\\dot f=(\dot f_\theta)_{\theta\in\A^{<k}}
        &\mapsto (\dot F_\theta)_{\theta\in\A^{\leq k}}
      \end{align*}
      is defined, in a similar way, as
      \begin{itemize}
        \item $\dot F_\theta=\begin{cases}
            \dot f_\alpha \text{~restricted to } \dot U_\theta^{\leq k}
            \text{~and seen as being in } \Lambda^{\leq k}(A^0)
            & \text{~when } \theta\in\A^{<k}
          \\\id \text{~(the identity) }
            & \text{~when } \theta\notin\A^{<k}
          \end{cases}$
      \end{itemize}
  \end{enumerate}
  Thus we can define
  \begin{align*}
    \mathfrak{S}^{\leq k}:
    \Gamma(\dot\cU^{<k};\Lambda^{<k}(A^0))
    \times
    \Gamma(\dot\cU^{k};\Lambda^{k}(A^0))
    &\to
    \Gamma(\dot\cU^{\leq k};\Lambda^{\leq k}(A^0))
  \\(\dot f, \dot g)
    &\mapsto
    (\dot F_\theta\dot G_\theta)_{\theta\in\A^{\leq k}}
  \end{align*}
  where $(\dot F_\theta)_{\theta\in\A^{\leq k}}=\sigma^{<k}(\dot f)$ and
  $(\dot G_\theta)_{\theta\in\A^{\leq k}}=\sigma^k(\dot g)$ are defined as
  above.
  \begin{s-rem}
    This map $\mathfrak{S}^{\leq k}$ is injective, since injectivity for germs
    implies injectivity for sections.
  \end{s-rem}
\end{defn}

\begin{lem}
  \marginnote{\cite[Lem.II.3.3]{Loday1994}}
  Let $k\in\cK$.
  \begin{enumerate}
    \item If the cocycles $\mathfrak{S}^{\leq k}(\dot f,\dot g)$ and
      $\mathfrak{S}^{\leq k}(\dot f',\dot g')$ are cohomologous in
      $\Gamma(\dot\cU^{\leq k};\Lambda^{\leq k}(A^0))$
      then $\dot f$ and $\dot f'$ are cohomologous in
      $\Gamma(\dot\cU^{<k};\Lambda^{<k}(A^0))$.
    \item Any cocycle in $\Gamma(\dot\cU^{\leq k},\Lambda^{\leq k}(A^0))$ is
      cohomologous to a cocycle in the range of $\mathfrak{S}^{\leq k}$.
  \end{enumerate}
\end{lem}
\begin{proof}
  \begin{enumerate}
    \item Denote by $\theta^+$ the nearest anti-Stokes direction in
      $\A^{\leq k}$ on the right\footnote{In clockwise direction.} of $\theta$.
      The cocycles $\mathfrak{S}^{\leq k}(\dot f,\dot g)$ and
      $\mathfrak{S}^{\leq k}(\dot f',\dot g')$ are cohomologous if and only
      if there is a $0$-cochain $c=(c_\theta)_{\theta\in\A^{\leq k}}
      \in\Gamma(\cU^{\leq k},\Lambda^{\leq k}(A^0)$ such that
      \begin{equation}\label{eq:UniqueString:urdtindfgupndtcn}
        \dot F_\theta\dot G_\theta =
        c_\theta^{-1}\dot F_\theta'\dot G_\theta'c_{\theta^+}
      \end{equation}
      for every $\theta\in\A$. From Proposition~\ref{prop:adequateProperties}
      follows, that $c$ is with values in $\Lambda^{<k}(A^0)$.
      The fact that $\Lambda^k(A^0)$ is normal in $\Lambda^{\leq k}(A^0)$ in
      Proposition~\ref{prop:PropertiesOfStokesSheafSplitting}, can be used to
      see that $ c_{\theta^+}^{-1}G_\theta'c_{\theta^+}
      \in\Gamma(\cU^{\leq k};\Lambda^{k}(A^0))$.
      Thus, we rewrite the relation (\ref{eq:UniqueString:urdtindfgupndtcn}) to
      \[
        \dot F_\theta\dot G_\theta =
        (c_\theta^{-1}\dot F_\theta'c_{\theta^+})
        (c_{\theta^+}^{-1}G_\theta'c_{\theta^+})
        \,,\qquad\text{~for~} \theta\in\A^{\leq k} \,.
      \]
      Since Corollary~\ref{cor:factorStokesGerms} tells us, that the
      factorization into the factors of the semi-direct product are unique, we
      get for all $\theta\in\A^k$
      \[
        \dot F_\theta=c_\theta^{-1}\dot F_\theta'c_{\theta^+}
        \qquad \text{~and~} \qquad
        \dot G_\theta=c_{\theta^+}^{-1}G_\theta'c_{\theta^+}.
      \]
      The former relation implies that $(\dot F_\theta)$ and $(\dot F_\theta')$
      are cohomologous with values in $\Lambda^{<k}(A^0)$ on $\cU^{\leq k}$.
      Since $\cU^{<k}$ is already adequate to $\Lambda^{<k}(A^0)$, \rewrite{are
      $(\dot F_\theta)$ and $(\dot F_\theta')$} already on $\cU^{<k}$,
      i.e.\ in $\Gamma(\dot\cU^{<k};\Lambda^{<k}(A^0))$, cohomologous.
    \item The proof of part 2.\ (together with the proof of part 1.) can be
      found in Loday-Richaud's paper \cite[Proof of Lem.II.3.3]{Loday1994}.
      \TODO[PROOF!]
  \end{enumerate}
\end{proof}
\begin{defn}\label{defn:theMapTau}
  Define the injective map
  \begin{align*}
    \tau^k:\Gamma(\dot\cU^k;\Lambda^k(A^0)) &\to \Gamma(\dot\cU;\Lambda(A^0))
  \\\dot f=(\dot f_\theta)_{\theta\in\A^k} &\mapsto
    (\dot F_\theta)_{\theta\in\A}
  \end{align*}
  where
  \[
    \dot F_\theta=\begin{cases}
      \dot f_\theta \text{~restricted to } \dot U_\theta
      \text{~and seen as being in } \Lambda(A^0)
      & \text{~when } \theta\in\A^k
    \\\id \text{~(the identity on $\dot U_\theta$) }
      & \text{~when } \theta\notin\A^k
    \end{cases}
  \]
  \marginnote{\cite[Prop.II.3.4]{Loday1994}}
  The \emph{product map of single-leveled cocycles} is then defined as
  \begin{align*}
    \tau:\prod_{k\in\cK}\Gamma(\dot \cU^k;\Lambda^k(A^0))
    &\to
    \Gamma(\dot\cU;\Lambda(A^0))
  \\(\dot f^k)_{k\in\cK}
    &\mapsto
    \prod_{k\in\cK}\tau^k(\dot f^k)
  \end{align*}
  following an ascending order of levels.
  \begin{s-rem}
    The map $\tau$
    \begin{enumerate}
      \item is injective since it is composed from $\sigma^k$ and the clearly
        injetive mapping
        \[ \begin{tikzcd}[row sep=0cm]
            \Gamma(\dot\cU^{\leq k};\Lambda^{\leq k}(A^0))
            \rar
            & \Gamma(\dot\cU;\Lambda(A^0))
          \\(\dot F_\theta)_{\theta\in\A^{\leq k}}
            \rar[maps to] 
            & (\dot F'_\theta)_{\theta\in\A}
        \end{tikzcd} \]
        where 
        \[
          \dot F'_\theta=\begin{cases}
            \dot F_\theta \text{~restricted to } \dot U_\theta
            \text{~and seen as being in } \Lambda(A^0)
            & \text{~when } \theta\in\A^{\leq k}
            \\\id \text{~(the identity on $\dot U_\theta$) }
            & \text{~when } \theta\notin\A^{\leq k}
          \end{cases}
        \]
        and
      \item it can be extended to an arbitrary order of levels
        (cf.\ Remark~\cite[Rem.II.3.5]{Loday1994}).
    \end{enumerate}
  \end{s-rem}
\end{defn}
\begin{cor}
  The product map of single-leveled cocycles $\tau$ induces on the cohomology
  a bijective and natural map
  \[
    \cT:
    \underset{\underset{k\in\cK}\prod H^1(S^1;\Lambda^k(A^0))}{%
      \underset{\text{\rotatebox[origin=c]{-90}{$\cong$}}}{%
        \prod_{k\in\cK}\Gamma(\dot\cU^k;\Lambda^k(A^0))}}
    \longrightarrow
    \underset{H^1(S^1;\Lambda(A^0))}{%
      \underset{\text{\rotatebox[origin=c]{-90}{$\cong$}}}{%
        H^1(\cU;\Lambda(A^0))}}\,.
  \]
\end{cor}
\begin{comment}
  \begin{proof}
    \PROBLEM[Follows from the last lemmas?]
  \end{proof}
\end{comment}
\paragraph{Composing functions to obtain $h$}
We have the ingredients to define the function $h$ from
Theorem~\ref{thm:mainThm2} by composition of already bijective maps.
\begin{proof}[Proof of theorem~\ref{thm:mainThm2}]
  Let $i_\theta:\Sto_\theta(A^0)\to\prod_{k\in\cK}\Sto_\theta^k(A^0)$ be the
  map which corresponds to the filtration from
  proposition~\ref{prop:filtrationOfStokesGroup} and
  denote the composition
  \[ \begin{tikzcd}[column sep=1.8cm,row sep=0]
      \displaystyle \prod_{\theta\in\cA}\Sto_\theta(A^0)
      \rar{\prod_{\theta\in\A}i_\theta}
      \arrow[ddrr, out=270,in=200,"\mathfrak{T}"]
      &
      \displaystyle \prod_{\theta\in\cA}\prod_{k\in\cK}\Sto_\theta^k(A^0)
    \\&\text{\rotatebox[origin=c]{-90}{$\equiv$}}
    \\&\displaystyle \prod_{k\in\cK}\prod_{\theta\in\cA}\Sto_\theta^k(A^0)
      \rar{\prod_{k\in\cK}i^k}&
      \displaystyle \prod_{k\in\cK}\Gamma(\dot\cU^k;\Lambda^k(A^0))
  \end{tikzcd} \]
  by $\mathfrak{T}$.
  Thus we obtain the bijection
  \[
    \cT\circ\mathfrak{T}:
    \prod_{\theta\in\cA}\Sto_\theta(A^0)
    \to
    H^1(\cU;\Lambda(A^0))
  \]
  as the map $h$.
  \PROBLEM[naturality (is obvious?)]
  \PROBLEM[Show that this is the correct $h$]
\end{proof}

\subsection{Some remarks}
\marginnote{\cite[III.1]{Loday1994}}
\begin{rem}
  The morphism $h$ endows the set $H^1(S^1;\Lambda(A^0))$ , resp.\ the
  classifying set, with the structure of a unipotent Lie group with finite
  dimension
  \[
    N=\sum_{1\leq j,l\leq n}\deg(q_j-q_l)
     =\sum_{\substack{1\leq j,l\leq n\\j<l}}2\cdot\deg(q_j-q_l)
  \]
  \begin{rem}
    This number $N$ is known to be the \emph{irregularity} of $[\End A^0]$.
  \end{rem}
\end{rem}

\begin{comment}
  \marginnote{\cite[880f]{Loday1994}}
  We have also two structures of a linear affine variety on the set $\St(A^0)$.
  \begin{rem}
    Let $\sto_\theta(A^0)$ be the Lie algebra corresponding to
    $\Sto_\theta(A^0)$. The exponential map\footnote{This is not the map $\exp$
    from section~\ref{sec:mainThm1}.} induces an homomorphism
    $\exp:\sto_\theta(A^0)\to\Sto_\theta(A^0)$ and denote by $\ln=\exp^{-1}$
    the inverse map.
    \begin{enumerate}
      \item The \emph{tangent linear structure} is defined as \TODO
      \item Using the map
        \begin{align*}
          \sto_{\theta}(A^0) &\overset{\id+\cdot}\longrightarrow
          \Sto_{\theta}(A^0)
        \\\dot{f}_\theta & \longmapsto \id+\dot{f}_\theta \,.
        \end{align*}
    \end{enumerate}
  \end{rem}
\end{comment}

\begin{rem}
  To define the inverse map of $h$, one has to find in each cocycle in
  $\St(A^0)$ the Stokes cocycle. Loday-Richaud~\cite[section~II.3.4]{Loday1994}
  gives an algorithm, which takes a cocycle over an arbitrary cyclic covering
  and outputs cohomologous Stokes cocycle and thus solves this problem.
\end{rem}

%%%%%%%%%%%%%%%%%%%%%%%%%%%%%%%%%%%%%%%%%%%%%%%%%%%%%%%%%%%%%%%%%%%%%%%%%%%%%%%
\section{Which information is required to describe a Stokes cocycle in the
  multileveled case?}\label{sec:WhichInformationIsNeeded}
Let $A^0$ be a normal form with dimension $n=3$ and two levels
$\cK=\{k_1<k_2\}$, which satisfies that there is at least one anti-Stokes
direction $\theta_0$ which is beared by both levels.
Let $q_j(t^{-1})$ be the determining polynomials and let $k_{jl}$ be the
degrees of $(q_j-q_l)(t^{-1})$.
Since it is the only way to get two levels in dimension $3$ we have, up to
permutation: $k_1=k_{1,2}\geq k_{1,3}$ and $k_2=k_{2,3}$.
We know also that the leading terms of $(q_1-q_2)(t^{-1})$ and
$(q_1-q_3)(t^{-1})$ are equal and thus determine the same anti-Stokes
directions.
The set of all anti-Stokes directions is then given as
\[
  \A=\left\{\theta_0+\frac{\pi}{k}\cdot j\mid k\in\cK\text{, }j\in\N\right\}
    =:\{
      \underset{\theta_0}{%
        \underset{\text{\rotatebox[origin=c]{-90}{$=$}}}{%
          \theta_1
      }}
    ,\dots,\theta_\nu\}\,.
\]
Denote by $\cY_0(t)$ a normal solution of $[A^0]$.

\marginnote{\cite[79]{Loday2014}}
Fix a set of determinations $\tilde\theta_1,\dots,\tilde\theta_\nu$ of the
anti-Stokes directions $\theta_1,\dots,\theta_\nu$ such that all
$\tilde\theta_j$ lie in the same interval $[2m\pi,2(m+1)\pi[$.
For every $\theta\in\A$ denote by $\cY_{0,\tilde\theta}(t)$ the normal solution
with that choice of a determination of the argument. We have already used this
notation on page~\pageref{page:alreadyUsedDefn}.

%%%%%%%%%%%%%%%%%%%%%%%%%%%%%%%%%%%%%%%%%%%%%%%%%%%%%%%%%%%%%%%%%%%%%%%%%%%%%%%
\subsection{What do the Stokes germs $\phi_\theta$ look like?}
\rewrite{First} we want to use the Stokes matrices of Stokes germs, discussed
in Section~\ref{sec:matrixReps}, to \rewrite{obtain some statements} about the
\rewrite{structure} of the Stokes germs in our situation.

The Proposition~\ref{prop:representation} states that every Stokes germ
$\phi_\theta$ can be written as its matrix representation conjugated by the
normal solution. That means that there is a Stokes matrix in
$\SSto_\theta(A^0)$, which gets via
\begin{align*}
  \rho_{\tilde\theta}^{-1}:
  \SSto_\theta(A^0)
  &\overset{\cong}\longrightarrow
  \Sto_\theta(A^0)
  \\C_{\cY_{0,\tilde\theta}} &\longmapsto
  \cY_{0,\tilde\theta} C_{\cY_{0,\tilde\theta}} \cY_{0,\tilde\theta}^{-1} \,,
\end{align*}
mapped to the Stokes germ $\phi_\theta$.
\TODO[This depends on the determination $\tilde\theta$ of $\theta$]

\paragraph{Using the structure of $\SSto_\theta(A^0)$}
First look at an example in which we will explain, from which relations on the
determining polynomials which restriction on the form of the Stokes matrices
follow.
\begin{exmp}
  Let $\theta\in\A$ be an anti-Stokes direction.
  From the definition of $\SSto_\theta(A^0)$ (cf.\
  Definition~\ref{defn:groupOfFaithfullReps}) we know that, if one has $q_1
  \underset{\theta,\max}{\prec} q_2$, the Stokes matrix has the form
  \[
    \begin{pmatrix}
      1 & \text{\boldmath$c_1$} & \star
    \\\text{\boldmath$0$} & 1 & \star
    \\\star & \star & 1
    \end{pmatrix}
  \]
  where $c_j\in\C$ and $\star\in\C$.
  \begin{s-rem}
    In our situation, we also know that
    \begin{einr}
      $q_1 \underset{\theta,\max}{\prec} q_2$
      \Leftrightarrow{}
      $q_1 \underset{\theta,\max}{\prec} q_3$
      \qquad \rewrite{respectively} \qquad
      $q_2 \underset{\theta,\max}{\prec} q_1$
      \Leftrightarrow{}
      $q_3 \underset{\theta,\max}{\prec} q_1$
    \end{einr}
    since the leading terms of $q_1-q_2$ and $q_1-q_3$ are equal.
  \end{s-rem}
  Thus the representation has the \rewrite{form}
  \[
    \begin{pmatrix}
      1 & c_1 & \text{\boldmath$c_2$}
    \\0 & 1 & \star
    \\\text{\boldmath$0$} & \star & 1
    \end{pmatrix}\,.
  \]
  If we also know that neither $q_2 \underset{\theta,\max}{\prec} q_3$ nor
  $q_3 \underset{\theta,\max}{\prec} q_2$ it has the \rewrite{form}
  \[
    \begin{pmatrix}
      1 & c_1 & c_2
    \\0 & 1 & \text{\boldmath$0$}
    \\0 & \text{\boldmath$0$} & 1
    \end{pmatrix}\,.
  \]
  We also know that every matrix of this \rewrite{form} is a representation to
  some Stokes germ.
  Thus we have an isomorphism
  \begin{align*}
    \vartheta_\theta:\C^2 &\longrightarrow \SSto_\theta(A^0)
  \\(c_1,c_2)&\longmapsto
    \begin{pmatrix}
      1 & c_1 & c_2
    \\0 & 1 & 0
    \\0 & 0 & 1
    \end{pmatrix}
  \end{align*}
\end{exmp}
In fact, the following $9$ cases of Stokes matrices can arise.
\begin{center}
  \def\arraystretch{1.3}
  \setlength\tabcolsep{4mm}
  \begin{tabular}{r|c|c|c}
    & $q_2 \underset{\theta,\max}{\prec} q_3$
    & $q_3 \underset{\theta,\max}{\prec} q_2$
    & else
    \tabularnewline
    \hline
    $\substack{q_1 \underset{\theta,\max}{\prec} q_2\\\text{and}
    \\q_1 \underset{\theta,\max}{\prec} q_3}$
    & $\begin{pmatrix} 1 & c_2 & c_3 \\0 & 1 & c_1 \\0 & 0 & 1 \end{pmatrix}$
   \cellcolor{blue!15}
    & $\begin{pmatrix} 1 & c_2 & c_3 \\0 & 1 & 0 \\0 & c_1 & 1 \end{pmatrix}$
   \cellcolor{blue!15}
    & $\begin{pmatrix} 1 & c_2 & c_3 \\0 & 1 & 0 \\0 & 0 & 1 \end{pmatrix}$
   \cellcolor{green!15}
    \tabularnewline
    \hline
    $\substack{q_2 \underset{\theta,\max}{\prec} q_1\\\text{and}
    \\q_3 \underset{\theta,\max}{\prec} q_1}$
    & $\begin{pmatrix} 1 & 0 & 0 \\c_2+c_1c_3 & 1 & c_1 \\c_3 & 0 & 1 \end{pmatrix}$
   \cellcolor{blue!15}
    & $\begin{pmatrix} 1 & 0 & 0 \\c_2 & 1 & 0 \\c_1c_2+c_3 & c_1 & 1 \end{pmatrix}$
   \cellcolor{blue!15}
    & $\begin{pmatrix} 1 & 0 & 0 \\c_2 & 1 & 0 \\c_3 & 0 & 1 \end{pmatrix}$
   \cellcolor{green!15}
    \tabularnewline
    \hline
    else
    & $\begin{pmatrix} 1 & 0 & 0 \\0 & 1 & c_1 \\0 & 0 & 1 \end{pmatrix}$
   \cellcolor{purple!15}
    & $\begin{pmatrix} 1 & 0 & 0 \\0 & 1 & 0 \\0 & c_1 & 1 \end{pmatrix}$
   \cellcolor{purple!15}
    & $\begin{pmatrix} 1 & 0 & 0 \\0 & 1 & 0 \\0 & 0 & 1 \end{pmatrix}$
  \end{tabular}
\end{center}
In the \textcolor{blue!75!black}{blue} cases we have $\cK_\theta=\cK$ and
$\C^3\overset{\vartheta_\theta}{\underset{\cong}{\longrightarrow}}\SSto_\theta(A^0)$.
In the \textcolor{green!50!black}{green} cases $\cK_\theta=\{k_2\}$ and
$\C^2\overset{\vartheta_\theta}{\underset{\cong}{\longrightarrow}}\SSto_\theta(A^0)$ as
well as in the \textcolor{purple!75!black}{purple} cases $\cK_\theta=\{k_1\}$
and $\C^1\overset{\vartheta_\theta}{\underset{\cong}{\longrightarrow}}\SSto_\theta(A^0)$.
Thus, for every $\theta\in\A$, we have an isomorphism
$\rho_{\tilde\theta}^{-1}\circ\vartheta_\theta$.
Two of the cases are more complicated than necessary, to be consistent with
the decomposition in the next part (cf.\ Example~\ref{exmp:decompositionHere}).

\paragraph{Decomposition by levels}
In proposition~\ref{prop:filtrationOfStokesGroup} and especially
remark~\ref{rem:filtrationOfStokesMats} we have defined a decomposition of the
Stokes group $\Sto_\theta(A^0)$ in subgroups generated by $k$-germs for
$k\in\cK$.
In our case, we have at most two level, such that this decomposition is given
by
\[
  \phi_\theta=\phi_\theta^{k_1} \phi_\theta^{k_2}
  \overset{i_\theta}\longmapsto
    \left(\phi_\theta^{k_1},\phi_\theta^{k_2}\right)
      \in\Sto_\theta^{k_1}(A^0)\times\Sto_\theta^{k_2}(A^0) \,,
\]
where $\phi_\theta^{k_1}\in\Sto_\theta^{k_1}(A^0)=\Sto_\theta^{<k_2}(A^0)$,
$\phi_\theta^{k_2}\in\Sto_\theta^{k_2}(A^0)$  and $i_\theta$ is the map, wich
gives the factors of this factorization in ascending order.

This decomposition of a germ $\phi_\theta$ is trivial if
$\#\cK(\phi_\theta)\leq1$, thus the interesting cases are the
\textcolor{blue!75!black}{blue} cases.

\begin{exmp}\label{exmp:decompositionHere}
  Look at the example
  \[
    \vartheta_\theta(c_1,c_2,c_3)=
    \cY_{0,\tilde\theta}
    \begin{pmatrix} 1 & 0 & 0 \\c_2 & 1 & 0 \\c_1c_2+c_3 & c_1 & 1 \end{pmatrix}
    \cY_{0,\tilde\theta}^{-1}
    =\phi_\theta
    \,.
  \]
  According to remark~\ref{rem:algFactorization} the factor
  $\phi_\theta^{k_1}$, generated by the $k_1$-germs, is given by
  \[
    \phi_\theta^{k_1}=
    \cY_{0,\tilde\theta}
    \begin{pmatrix}
      1 & 0 & 0
    \\\text{\boldmath $0$} & 1 & 0
    \\\text{\boldmath $0$} & c_1 & 1
    \end{pmatrix}
    \cY_{0,\tilde\theta}^{-1}
    \,.
  \]
  The other factor $\phi_\theta^{k_2}$ is then obtained as
  \begin{align*}
    \phi_\theta^{k_2}&=
    \left(\phi_\theta^{k_1}\right)^{-1}
    \phi_\theta^{k_2}
  \\&=\cY_{0,\tilde\theta}
    \begin{pmatrix}
      1     & 0    & 0
    \\0     & 1    & 0
    \\0     & -c_1 & 1
    \end{pmatrix}
    \underset{=\id}{\underbrace{%
        \cY_{0,\tilde\theta}^{-1}
        \cY_{0,\tilde\theta}
    }}
    \begin{pmatrix} 1 & 0 & 0 \\c_2 & 1 & 0 \\c_1c_2+c_3 & c_1 & 1 \end{pmatrix}
    \cY_{0,\tilde\theta}^{-1}
  \\&=\cY_{0,\tilde\theta}
    \begin{pmatrix}
      1     & 0 & 0
    \\c_2     & 1          & 0
    \\c_3     & 0          & 1
    \end{pmatrix}
    \cY_{0,\tilde\theta}^{-1}
    \,.
  \end{align*}
\end{exmp}
The four nontrivial decomposition in our situation, are given by
\begin{enumerate}
  \item $\begin{pmatrix} 1 & 0 & 0 \\0 & 1 & c_1 \\0 & 0 & 1 \end{pmatrix}
  \cdot\begin{pmatrix} 1 & c_2 & c_3 \\0 & 1 & 0 \\0 & 0 & 1 \end{pmatrix}=
  \begin{pmatrix} 1 & c_2 & c_3 \\0 & 1 & c_1 \\0 & 0 & 1 \end{pmatrix}$
  \item $\begin{pmatrix} 1 & 0 & 0 \\0 & 1 & 0 \\0 & c_1 & 1 \end{pmatrix}
  \cdot\begin{pmatrix} 1 & c_2 & c_3 \\0 & 1 & 0 \\0 & 0 & 1 \end{pmatrix}=
  \begin{pmatrix} 1 & c_2 & c_3 \\0 & 1 & 0 \\0 & c_1 & 1 \end{pmatrix}$
  \item $\begin{pmatrix} 1 & 0 & 0 \\0 & 1 & c_1 \\0 & 0 & 1 \end{pmatrix}
  \cdot\begin{pmatrix} 1 & 0 & 0 \\c_2 & 1 & 0 \\c_3 & 0 & 1 \end{pmatrix}=
  \begin{pmatrix} 1 & 0 & 0 \\c_2+c_1c_3 & 1 & c_1 \\c_3 & 0 & 1 \end{pmatrix}$
  \item $\begin{pmatrix} 1 & 0 & 0 \\0 & 1 & 0 \\0 & c_1 & 1 \end{pmatrix}
  \cdot\begin{pmatrix} 1 & 0 & 0 \\c_2 & 1 & 0 \\c_3 & 0 & 1 \end{pmatrix}=
  \begin{pmatrix} 1 & 0 & 0 \\c_2 & 1 & 0 \\c_1c_2+c_3 & c_1 & 1 \end{pmatrix}$
\end{enumerate}
Next, we will use this decompositions to write the isomorphisms
$\vartheta_\theta:\C^\star\to \SSto_\theta(A^0)$ as
\[ \begin{tikzcd}
  \C^\star
  \rar{j_\theta}
  \arrow[rr,out=-30,in=-150,"\vartheta_\theta"]
  &
  \SSto_\theta^{k_1}(A^0)\times\SSto_\theta^{k_2}(A^0)
  \rar{i_\theta^{-1}}
  &
  \SSto_\theta(A^0)
\end{tikzcd} \]
where $\star\in\{1,2,3\}$.

\subsection{What do Stokes cocycles look like?}
\begin{prop}
  By taking the product over $\A$ of all
  $\rho_{\tilde\theta}^{-1}\circ\vartheta_\theta$ we obtain the isomorphism
  \[
    \prod_{\theta\in\A}\rho_{\tilde\theta}^{-1}\circ\vartheta_\theta:
    \C^{k_1+2\cdot k_2}
    \overset\cong\longrightarrow \prod_{\theta\in\A}\SSto_\theta(A^0)
    \overset\cong\longrightarrow \prod_{\theta\in\A}\Sto_\theta(A^0)
  \]
  which can be concatenated with $h$ to obtain an element in $\St(A^0)$.
  We can also define the isomorphism
  $\C^{2\cdot(k_1+2\cdot k_2)}\textcolor{purple}{\longrightarrow} \St(A^0)$
  which makes the following diagram commute.
  \[ \begin{tikzcd}[column sep=1.5cm,row sep=2cm]
      &\C^{2\cdot(k_1+2\cdot k_2)}
      \dlar[blue]{\prod_{\theta\in\A}\rho_{\tilde\theta}^{-1}\circ\vartheta_\theta}
      \arrow[dr,purple]
    \\\prod_{\theta\in\A}\Sto_\theta(A^0)
      \arrow[rr,green!50!black,"h"]
      &&\St(A^0)
  \end{tikzcd} \]
  Using the knowledge, how the morphism were build and the fact that some of
  the morphism are equal, we can rewrite this diagram as follows
  \[ \begin{tikzcd}
      &\C^{2\cdot(k_1+2\cdot k_2)}
      \dar[xshift=-1pt,blue]
      \dar[xshift=1pt,purple,"\prod_{\theta\in\A}j_\theta"]
    \\
      &\prod_{k\in\{k_1,k_2\}}\prod_{\theta\in\A}\SSto_{\theta}^{k}(A^0)
       \arrow[dl,blue,"\prod_{\theta\in\A}i_\theta^{-1}"]
       \arrow[ddr,purple,
         "\prod_{k\in\{k_1,k_2\}}\prod_{\theta\in\A}(\rho_{\tilde\theta}^k)^{-1}"]
    \\ \prod_{\theta\in\A} \SSto_{\theta}(A^0)
        \dar[blue]{\prod_{\theta\in\A}\rho_{\tilde\theta}^{-1}}
    \\\prod_{\theta\in\A} \Sto_{\theta}(A^0)
      \arrow[rr,green!50!black,"\prod_{\theta\in\A}i_\theta"]
      &&\prod_{k\in\{k_1,k_2\}}\prod_{\theta\in\A}\Sto_{\theta}^{k}(A^0)
      \dar[xshift=-1pt,green!50!black]
      \dar[xshift=1pt,purple]{\prod_{k\in\{k_1,k_2\}}i^k}
    \\
      && \prod_{k\in\cK}\Gamma(\dot\cU^k;\Lambda^k(A^0))
      \dar[xshift=-1pt,green!50!black]
      \dar[xshift=1pt,purple]{\cT}
    \\&& \St(A^0)
  \end{tikzcd} \]
\end{prop}

\begin{comment}
  Decompose $\C^{k_1+2\cdot k_2}\cong
  \prod_{\theta\in\A^{k_1}} \C \times \prod_{\theta\in\A^{k_2}} \C^2$ and
  define the isomorphism
  \[
    \C^{k_1+2\cdot k_2}\overset{\cong}\longrightarrow
    \prod_{\theta\in\A^{k_1}} \SSto_{\theta}^{k_1}(A^0) \times
    \prod_{\theta\in\A^{k_2}} \SSto_{\theta}^{k_2}(A^0)
  \]
  levelwise as
  \begin{enumerate}
    \item an element
      $\left( (a_{\theta_{\nu_1}},b_{\theta_{\nu_1}})
        ,(a_{\theta_{\nu_2}},b_{\theta_{\nu_2}})
        ,\dots
      \right)\in\prod_{\theta\in\A^{k_2}} \C^2$
      gets mapped to
      \[
        \left.
        \begin{cases}
          \left(
          \begin{pmatrix} 1 & a_{\theta_{\nu_1}} & b_{\theta_{\nu_1}} \\0 & 1 & 0 \\0 & 0 & 1 \end{pmatrix}
          ,\begin{pmatrix} 1 & 0 & 0 \\a_{\theta_{\nu_2}} & 1 & 0 \\b_{\theta_{\nu_2}} & 0 & 1 \end{pmatrix}
          ,\begin{pmatrix} 1 & a_{\theta_{\nu_3}} & b_{\theta_{\nu_3}} \\0 & 1 & 0 \\0 & 0 & 1 \end{pmatrix}
            ,\dots
          \right)
          & ,\substack{\text{~if~} q_1 \underset{\theta_0,\max}{\prec} q_2
            \\\text{~and~} q_1 \underset{\theta_0,\max}{\prec} q_3}
          \\\left(
          \begin{pmatrix} 1 & 0 & 0 \\a_{\theta_{\nu_1}} & 1 & 0 \\b_{\theta_{\nu_1}} & 0 & 1 \end{pmatrix}
          ,\begin{pmatrix} 1 & a_{\theta_{\nu_2}} & b_{\theta_{\nu_2}} \\0 & 1 & 0 \\0 & 0 & 1 \end{pmatrix}
          ,\begin{pmatrix} 1 & 0 & 0 \\a_{\theta_{\nu_3}} & 1 & 0 \\b_{\theta_{\nu_3}} & 0 & 1 \end{pmatrix}
            ,\dots
          \right)
          & ,\substack{\text{~if~} q_2 \underset{\theta_0,\max}{\prec} q_1
            \\\text{~and~} q_3 \underset{\theta_0,\max}{\prec} q_1}
        \end{cases}
        \right\}
        =:\left(
          C_{\theta_{\nu_1}}^{k_2}
          ,C_{\theta_{\nu_2}}^{k_2}
          ,\dots
        \right)
      \]
    \item
      $\left(
        c_{\theta_{\mu_1}}
        ,c_{\theta_{\mu_2}}
        ,\dots
      \right)\in\prod_{\theta\in\A^{k_1}} \C$
      gets mapped to
      \[
        \left.
        \begin{cases}
          \left(
          \begin{pmatrix} 1 & 0 & 0 \\0 & 1 & c_{\theta_{\mu_1}} \\0 & 0 & 1 \end{pmatrix}
          ,\begin{pmatrix} 1 & 0 & 0 \\0 & 1 & 0 \\0 & c_{\theta_{\mu_2}} & 1 \end{pmatrix}
          ,\begin{pmatrix} 1 & 0 & 0 \\0 & 1 & c_{\theta_{\mu_3}} \\0 & 0 & 1 \end{pmatrix}
            ,\dots
          \right)
          &\text{,~if~} q_2 \underset{\theta_0,\max}{\prec} q_3
          \\\left(
          \begin{pmatrix} 1 & 0 & 0 \\0 & 1 & 0 \\0 & c_{\theta_{\mu_1}} & 1 \end{pmatrix}
          ,\begin{pmatrix} 1 & 0 & 0 \\0 & 1 & c_{\theta_{\mu_2}} \\0 & 0 & 1 \end{pmatrix}
          ,\begin{pmatrix} 1 & 0 & 0 \\0 & 1 & 0 \\0 & c_{\theta_{\mu_2}} & 1 \end{pmatrix}
            ,\dots
          \right)
          &\text{,~if~} q_3 \underset{\theta_0,\max}{\prec} q_2
        \end{cases}
      \right\}
      =:\left(
        C_{\theta_{\mu_1}}^{k_1}
        ,C_{\theta_{\mu_2}}^{k_1}
        ,\dots
      \right)
      \]
  \end{enumerate}
\end{comment}

%%%%%%%%%%%%%%%%%%%%%%%%%%%%%%%%%%%%%%%%%%%%%%%%%%%%%%%%%%%%%%%%%%%%%%%%%%%%%%%
\subsubsection{Explicit example}
\def\kOne{1}
\def\kTwo{3}
\def\zkOnepzKtwo{14} % 2\cdot(\kOne+2\cdot\kTwo
\def\zkOne{2} % 2*\kOne
\def\zkTwo{6} % 2*\kTwo

Even more explicit, we can fix the levels $k_1=\kOne$ and $k_2=\kTwo$ together
with $\theta_0=0$.
Assume without any restriction that $q_1 \underset{\theta_0,\max}{\prec} q_2$
and $q_1 \underset{\theta_0,\max}{\prec} q_3$ as well as
$q_2 \underset{\theta_0,\max}{\prec} q_3$.

The classification space is in this case isomorphic to
$\C^{2\cdot(\kOne+2\cdot\kTwo)}=\C^{\zkOnepzKtwo}$.
The element
\[
  ({}^1c_1,{}^2c_1,
  {}^1c_2,{}^1c_3,{}^2c_2,{}^2c_3,\dots,{}^{\zkTwo}c_2,{}^{\zkTwo}c_3)
  \in\C^{\zkOnepzKtwo}
\]
gets, via the morphism $\prod_{\theta\in\A}j_\theta$, mapped to
\begin{align*}
  &\left(
  \left(
    \begin{pmatrix} 1 & 0 & 0 \\0 & 1 & {}^1c_1 \\0 & 0 & 1 \end{pmatrix},
    \begin{pmatrix} 1 & 0 & 0 \\0 & 1 & 0 \\0 & {}^2c_1 & 1 \end{pmatrix}
  \right),
  \right.
\\&\qquad\left(
  \left.
    \begin{pmatrix} 1 & {}^1c_2 & {}^1c_3 \\0 & 1 & 0 \\0 & 0 & 1 \end{pmatrix},
    \begin{pmatrix} 1 & 0 & 0 \\{}^2c_2 & 1 & 0 \\{}^2c_3 & 0 & 1 \end{pmatrix},
    \dots,
    \begin{pmatrix} 1 & 0 & 0 \\{}^{\zkTwo}c_2 & 1 & 0 \\{}^{\zkTwo}c_3 & 0 & 1 \end{pmatrix}
  \right)
  \right)
\end{align*}
in $\prod_{\theta\in\A^{\kOne}}\SSto_{\theta}^{\kOne}(A^0) \times
\prod_{\theta\in\A^{\kTwo}}\SSto_{\theta}^{\kTwo}(A^0)$ and thus the element
\begin{align*}
  &\left(
  \left(
    \begin{pmatrix} 1 & 0 & 0 \\0 & 1 & {}^1c_1 \\0 & 0 & 1 \end{pmatrix},
    \id,\id,
    \begin{pmatrix} 1 & 0 & 0 \\0 & 1 & 0 \\0 & {}^2c_1 & 1 \end{pmatrix},
    \id,\id
  \right),
  \right.
\\&\qquad\left(
  \left.
    \begin{pmatrix} 1 & {}^1c_2 & {}^1c_3 \\0 & 1 & 0 \\0 & 0 & 1 \end{pmatrix},
    \begin{pmatrix} 1 & 0 & 0 \\{}^2c_2 & 1 & 0 \\{}^2c_3 & 0 & 1 \end{pmatrix},
    \dots,
    \begin{pmatrix} 1 & 0 & 0 \\{}^{\zkTwo}c_2 & 1 & 0 \\{}^{\zkTwo}c_3 & 0 & 1 \end{pmatrix}
  \right)
  \right)
\end{align*}
in
$\prod_{\theta\in\A}\SSto_{\theta}^{\kOne}(A^0) \times
\prod_{\theta\in\A}\SSto_{\theta}^{\kTwo}(A^0)$.
Using the morphism $\prod_{\theta\in\A}i_\theta^{-1}$ we get a complete set of
Stokes matrices as
\begin{align*}
  &\left(
    \begin{pmatrix} 1 & {}^1c_2 & {}^1c_3 \\0 & 1 & {}^1c_1 \\0 & 0 & 1 \end{pmatrix},
    \begin{pmatrix} 1 & 0 & 0 \\{}^2c_2 & 1 & 0 \\{}^2c_3 & 0 & 1 \end{pmatrix},
    \begin{pmatrix} 1 & {}^3c_2 & {}^3c_3 \\0 & 1 & 0 \\0 & 0 & 1 \end{pmatrix},
  \right.
\\&\qquad
  \left.
    \begin{pmatrix} 1 & 0 & 0 \\{}^4c_2 & 1 & 0 \\{}^2c_1{}^4c_2+{}^4c_3 & {}^2c_1 & 1 \end{pmatrix},
    \begin{pmatrix} 1 & {}^5c_2 & {}^5c_3 \\0 & 1 & 0 \\0 & 0 & 1 \end{pmatrix},
    \begin{pmatrix} 1 & 0 & 0 \\{}^{\zkTwo}c_2 & 1 & 0 \\{}^{\zkTwo}c_3 & 0 & 1 \end{pmatrix}
  \right)
  \in
  \prod_{\theta\in\A}\SSto_{\theta}(A^0)
\end{align*}

