\chapter{Stokes Structures}
\begin{comment}
  Matrizen Sicht:
  \begin{itemize}
    \item \cite[Chapter 3]{boalch}
      \\ largely following:
      \begin{itemize}
        \item[\textbf{8}] D.G. Babbitt and V.S. Varadarajan.
        \textbf{Formal reduction theory of meromorphic differential equations:
          a group theoretic view.}
        \texttt{euclid.pjm.1102720203.pdf}
        \item[\textbf{11}] W. Balser, W.B. Jurkat, and D.A. Lutz.
        \textbf{Birkhoff invariants and Stokes’ multipliers for meromorphic
          linear differential equations.}
        \item[\textbf{40}] M. Jimbo, T. Miwa, and Kimio Ueno.
        \textbf{Monodromy preserving deformations of linear differential
          equations with rational coefficients I.}
        \item[\textbf{43}] M. Loday-Richaud
        \textbf{Stokes phenomenon, multisummability and differential Galois
          groups.}
        \item[\textbf{50}] J. Martinet and J.P. Ramis.
        \textbf{Elementary acceleration and multisummability.}
      \end{itemize}
    \item \cite{thboalch}
    \item \cite{van2003galois}
    \item Marius van der Put, Kyoshi Saito => Diff.Galois Theory
  \end{itemize}
  Moderne Sicht (sheaf-view):
  \begin{itemize}
    \item Malgrange
    \item \cite{sabbah_cimpa90} and
    \item \cite{sabbah2007isomonodromic}
    \item \cite{sabbah2009introduction} and \cite{sabbah2013introduction}
  \end{itemize}
  See also:
  \begin{itemize}
    \item \cite{Loday1994} for \textbf{Sheaf $\to$ Matrix}
      \begin{thm}[II.2.1]
        The map
        \[
          h:\prod_{\alpha\in\A}\Sto_\alpha(A_0)\to H^1(S^1;\Lambda(A_0))
        \]
        is bijective and natural.
      \end{thm}
      \begin{itemize}
        \item \cite{LodayRichaud2004}
        \item \cite{Loday2014}
      \end{itemize}
    \item \cite{Varadarajan96linearmeromorphic} for \textbf{overview of both}
  \end{itemize}
\end{comment}

\begin{comment}
  \section{Dictionary}
  \begin{center}
    \begin{tabular}{|c|c||c|c||c|c|}
      \cite{sabbah_cimpa90} & \cite{sabbah2007isomonodromic} &
      \cite{boalch}& \cite{thboalch} &
      \cite{Loday1994} & \cite{Loday2014} \tabularnewline
    \hline
    &  & & & $A_0$ & \tabularnewline{}
    &  & $\A$ & $\A$ &
      $\A$, $\A^{\star}$
      \marginnote{\footnotesize$\star\in\{{}^k,{}^{<k},{}^{\leq k}\}$} &
      $\mathfrak{A}$
      \tabularnewline
    &  & $\Sto_\alpha$ & $\Sto_\alpha$ & $\Sto_\alpha$ & \tabularnewline
    &  &  &  & $\Lambda(A_0)$ & \tabularnewline
      &  &  &  & $[{}^F\!A]$ & \tabularnewline
    &  &  &  &  & \tabularnewline
    &  &  &  &  & \tabularnewline
    &  &  &  &  & \tabularnewline
    &  &  &  &  & \tabularnewline
    &  &  &  &  & \tabularnewline
    \end{tabular}
  \end{center}
\end{comment}

\section{Situation}
\begin{center}
  \begin{tikzpicture}[scale=3]
    \node[green!40!black] (modSpcSheaf) at (0,0.3) {$\sH(\sM^{good})$};
    \node[blue] (modSpcMat) at (0,-0.3) {$\cH(A^0)$};
    \node[purple] (class) at (0.8,0) {$\G\backslash\tilde\G(B_0)$};
    \node[green!40!black] (sheaf) at (3,1.3) {$\St(\sM^{good})$};
    \node[purple] (sheaf2) at (4,1) {$H^1(S^1;\Lambda^{<0}(B_0)$};
    \node[blue] (mat) at (3,-1.3) {$\prod_{d\in\A}\Sto_d(A^0)$};
    \node[blue] (mat') at (3,-2) {$(U_+\times U_-)^{k-1}$};
    \node[purple] (mat2) at (4,-1) {$\prod_{\alpha\in\mathfrak{A}}\Sto_\alpha(B_0)$};

    \draw[thick,double] (modSpcMat) -- (modSpcSheaf) -- (class);
    \draw[thick,double,blue] (modSpcMat) -- (class);
    \draw[thick,double] (sheaf) -- (sheaf2);
    \draw[thick,double] (mat) -- (mat2);

    \draw[->,green!40!black] (modSpcSheaf) -- (sheaf);
    \draw[->,blue] (modSpcMat) -- (mat);
    \draw[->,blue] (class) -- (mat);
    \draw[->,blue] (mat) -- (mat');
    \draw[->,purple] (class) -- (sheaf2);
    \draw[->,purple] (class) -- (mat2);
    \draw[->,purple] (mat2) -- (sheaf2);

    \node[green!40!black] at (1,1) {\cite{sabbah2007isomonodromic}};
    \node[blue] at (1,-1) {\cite{thboalch},\cite{boalch}};
    \node[purple] at (4.5,0) {\cite{Loday1994}{\footnotesize(\cite{Loday2014})}};
  \end{tikzpicture}
\end{center}

\section{The classifying set}
\begin{comment}
  \begin{itemize}
    \item \cite{thboalch} p.6
      \begin{itemize}
        \item \cite{boalch} p.19
      \end{itemize}
    \item \cite{Loday1994} p.852
    \item \cite{sabbah2007isomonodromic} p.111
  \end{itemize}
\end{comment}
Roughly speaking, we want to understand the difference between the formal
Classification of meromophic connections and the finer meromophic
classification.
It is natural to fix a model $A^0$ and restrict to the meromophic connections
formally isomorphic to this model.


Our goal is to understand \TODO.
To achieve this goal, we look at the slightly larger space \TODO.

\TODO[Define $\sH$]
\begin{comment} \footnotesize
  \begin{lem}
    The presheaf $\sH$ is a sheaf on $\{0\}$.
  \end{lem}
  \begin{proof}
    See \cite{sabbah2007isomonodromic} lemma II.6.2. 
  \end{proof}
\end{comment}

\begin{comment}
  \TODO: left quotient? $G\backslash\hat G(A_0)$
\end{comment}
\section{Stokes Structures: Cohomology version}
\TODO[The stokes sheaf]
Let $(\sM,\nabla,\hat f)$ be defined on\dots

\begin{thm}
  The homomorphism $\sH\to\St(\sM^{good})$ is an isomorphism of pointed sets.
\end{thm}
\TODO[\cite{sabbah2007isomonodromic} cor II.6.4.]
\section{Stokes Structures: Matrix version}
\section{From cohomology to matrix version}
