%%%%%%%%%%%%%%%%%%%%%%%%%%%%%%%%%%%%%%%%%%%%%%%%%%%%%%%%%%%%%%%%%%%%%%%%%%%%%%%
\section{Further improvements}\label{sec:furtherImprovements}
The set $\prod_{\alpha\in\A}\Sto_\alpha(A^0)$ and thus
$\prod_{\alpha\in\A}\SSto_\alpha(A^0)$ has some bad properties, when small
deformations are applied to $[A^0]$, since under arbitrary small changes, one
Stokes ray can split into two.

In the prove of Theorem~\ref{thm:mainThm2} we have seen, \rewrite{that not only
$\cH(A^0)\cong\prod_{\alpha\in\A}\Sto_\alpha(A^0)$ but} also that
\begin{align*}
  \cH(A^0)&\cong\prod_{k\in\cK}\prod_{\alpha\in\A}\Sto_\alpha^k(A^0)
  \\&\cong\prod_{k\in\cK}\prod_{\alpha\in\A^k}\Sto_\alpha^k(A^0)
  &\text{(since $\SSto_{\alpha}^{k}(A^0)=\{\id\}$ when $k\notin\cK_\alpha$.)}
  \\&\cong\prod_{k\in\cK}\prod_{\alpha\in\A^k}\SSto_\alpha^k(A^0) \,.
\end{align*}
This representation can be used to achieve further improvements, since one is
able to multiply some succeeding Stokes matrices (resp. Stokes germs) of the
same level without loss of information.
This is precisely stated in Corollary~\ref{cor:composeLevelwise} below.

\TODO[move?]
Boalch, who looks in his publications \cite{boalch,thboalch} only at the
single-leveled case, uses this extensively to obtain better Stokes matrices,
which are stable under small deformations. Our Stokes matrices are in his
publications called \emph{Stokes factors}.

Let us fix a level $k\in\cK$. We want to rewrite the product
$\prod_{\alpha\in\A^k}\SSto_\alpha^k(A^0)$ by collecting the information of a
subsets of these Stokes matrices into one product of these matrices.

\begin{defn}
  A subset of $\A^k$ consisting of $\frac{\#\A^k}{2k}$ consecutive anti-Stokes
  directions of level $k$ will be called a \emph{half-period (of level $k$)}.
\end{defn}
From the definition of anti-Stokes directions of level $k$ it is clear that
every arc of width $\frac{\pi}{k}$, which has no anti-Stokes direction of level
$k$ on its border, contains $\frac{\#\A^k}{2k}$ anti-Stokes directions
since
\begin{einr}
  for every sector $I=U(\theta,\frac{\pi}{k})$ of width $\frac{\pi}{k}$ and
  center $\theta$ which satisfies $\theta\pm\frac{\pi}{2k}\notin\A^k$
  there is to any pair $(q_j,q_l)$ in $\cQ(A^0)$ which has level $k_{jl}=k$
  exactly one direction $\alpha\in \A^k\cap I$ which satisfies that
  $\Im(a_{jl}e^{-ik_{jl\alpha}})=0$.
  At such a direction $\alpha$ corresponding to the pair $(q_j,q_l)$ is then
  either $q_j\myrel{\alpha}q_l$ or $q_l\myrel{\alpha}q_j$ satisfied.
\end{einr}
This implies that for every $\theta$, which satisfies
$\theta\pm\frac{\pi}{2k}\notin\A^k$, is $\A^k\cap U(\theta,\frac{\pi}{k})$
a half-period and by
$\dot\bigcup_{1\leq m\leq 2k}\A^k\cap U(\theta+\frac{m\pi}{k},\frac{\pi}{k})
=\A^k$ is a \rewrite{decomposition} in
distinct half-periods of $\A^k$ given.

\begin{figure}[h!] %{{{
  \begin{center}
    \begin{tikzpicture}[scale=4]
      \pgfmathsetmacro{\k}{1}
      \pgfmathsetmacro{\abs}{0.4}

      \draw[thick, dotted, green!50!black] (0.9,0) -- (0.9,{\abs + 0.1})
        node [above,font=\tiny,] {$\theta$};

      \foreach \x in {-2,-1,...,3}{
        \foreach \argA/\absA in {-0.5/0.4
                                ,0.3/0.35
                                ,0.1/0.27
                                ,0.65/0.23}{
          \begin{scope}
            \clip (-0.65,{\abs+0.1}) rectangle (2.63,{-\abs-0.1});
            \pgfmathsetmacro{\s}{{\argA + \x / \k * 2 - 1/2/\k}};

            \draw[blue!40!white] (\s,0)
                sin ({\s + 1/2/\k},{-\absA})
                cos ({\s + 2/2/\k},0)
                sin ({\s + 3/2/\k},\absA)
                cos ({\s + 4/2/\k},0);
            \draw[dotted] ({\s + 1/2/\k},{-\absA}) -- ({\s + 1/2/\k},-0.5);
            \fill[white] ({\s + 1/2/\k},{-\absA}) circle (1pt);
            \fill[red] ({\s + 1/2/\k},{-\absA}) circle (.4pt);

            \draw[blue!40!black] (\s,0)
                sin ({\s + 1/2/\k},\absA)
                cos ({\s + 2/2/\k},0)
                sin ({\s + 3/2/\k},{-\absA})
                cos ({\s + 4/2/\k},0);
            \draw[dotted] ({\s + 3/2/\k},{-\absA}) -- ({\s + 3/2/\k},-0.5);
            \fill[white] ({\s + 3/2/\k},{-\absA}) circle (1pt);
            \fill[red] ({\s + 3/2/\k},{-\absA}) circle (.4pt);
          \end{scope}
          \begin{scope}[thick]
            \clip (0.4,{\abs+0.1}) rectangle (1.4,{-\abs-0.1});
            \pgfmathsetmacro{\s}{{\argA + \x / \k * 2 - 1/2/\k}};

            \draw[blue!40!white] (\s,0)
                sin ({\s + 1/2/\k},{-\absA})
                cos ({\s + 2/2/\k},0)
                sin ({\s + 3/2/\k},\absA)
                cos ({\s + 4/2/\k},0);
            \fill[white] ({\s + 1/2/\k},{-\absA}) circle (1pt);
            % \draw[dotted] ({\s + 1/2/\k},{-\absA}) -- ({\s + 1/2/\k},0.5);
            \fill[red] ({\s + 1/2/\k},{-\absA}) circle (.4pt);

            \draw[blue!40!black] (\s,0)
                sin ({\s + 1/2/\k},\absA)
                cos ({\s + 2/2/\k},0)
                sin ({\s + 3/2/\k},{-\absA})
                cos ({\s + 4/2/\k},0);
            \fill[white] ({\s + 3/2/\k},{-\absA}) circle (1pt);
            % \draw[dotted] ({\s + 3/2/\k},{-\absA}) -- ({\s + 3/2/\k},0.5);
            \fill[red] ({\s + 3/2/\k},{-\absA}) circle (.4pt);
          \end{scope}
        }
      }
      \node[font=\tiny] at (-0.5,{-\abs - 0.15}) {$\alpha_{\nu-1}$};
      \node[font=\tiny] at (-0.35,{-\abs - 0.15}) {$\alpha_{\nu}$};
      \node[font=\tiny] at (0.1,{-\abs - 0.15}) {$\alpha_1$};
      \node[font=\tiny] at (0.3,{-\abs - 0.15}) {$\alpha_2$};
      \node[font=\tiny] at (0.5,{-\abs - 0.15}) {$\alpha_3$};
      \node[font=\tiny] at (0.65,{-\abs - 0.15}) {$\alpha_4$};
      \node[font=\tiny] at (1.1,{-\abs - 0.15}) {$\alpha_5$};
      \node[font=\tiny] at (1.3,{-\abs - 0.15}) {$\alpha_6$};
      \node[font=\tiny] at (1.5,{-\abs - 0.15}) {$\alpha_7$};
      \node[font=\tiny] at (1.65,{-\abs - 0.15}) {$\alpha_8$};
      \node[font=\tiny] at (2.1,{-\abs - 0.15}) {$\alpha_9$};
      \node[font=\tiny] at (2.3,{-\abs - 0.15}) {$\alpha_{10}$};
      \node[font=\tiny] at (2.5,{-\abs - 0.15}) {$\alpha_{11}$};

      \draw[-latex'] (-0.7,0) -- (2.7,0) node [right] {$S^1$};
      \draw[-latex'] ({0},{-\abs-0.1}) -- ({0},{\abs + 0.2});

      \begin{scope}[dashed]
        \clip (-0.65,{\abs+0.1}) rectangle (2.6,{-\abs-0.4});
        \foreach \x in {-2,-1,1,2} {
          % \draw [purple]
          %   ({0.4 + \x},{-\abs - 0.2}) -- ({0.4 + \x},0);
          \draw [purple]
            ({1.4 + \x},{-\abs - 0.2}) -- ({1.4 + \x},0);
          \draw [purple
                ,decorate
                ,decoration={brace,mirror,amplitude=10pt}
                ,xshift=0pt
                ,yshift=0pt]
            ({0.4 + \x},{-\abs - 0.2}) -- ({1.4 + \x},{-\abs - 0.2});
        }
      \end{scope}
      \draw [purple,dashed]
        (1.4,{-\abs - 0.2}) -- (1.4,0);
      \draw [purple, thick
            ,decorate
            ,decoration={brace,mirror,amplitude=10pt}
            ,xshift=0pt
            ,yshift=0pt]
        (0.4,{-\abs - 0.2}) -- (1.4,{-\abs - 0.2})
        node [midway,yshift=-7pt] {$\tikzmark{g0}$};
      \draw [purple,dashed,
            ,decorate
            ,decoration={brace,mirror,amplitude=10pt}
            ,xshift=0pt
            ,yshift=0pt]
        (1.4,{-\abs - 0.2}) -- (2.4,{-\abs - 0.2})
        node [midway,yshift=-7pt] {$\tikzmark{g1}$};

      % \draw[dotted] (-0.6,\abs)node[left,font=\tiny] {$|a|$} -- (2.6,\abs);
      % \draw[dotted] (-0.6,{-\abs})node[left,font=\tiny] {$-|a|$} -- (2.6,{-\abs});

      % \draw[dotted] ({-0.5},{-\abs}) -- ({-0.5},{\abs + 0.1})
      %   node [above,font=\tiny,] {-0.5 \pi};
      % \foreach \x in {0.5,1,...,2.5}{%
      %   \draw[dotted] ({\x},{-\abs}) -- ({\x},{\abs + 0.1})
      %     node [above,font=\tiny,] {\x \pi};
      % }
    \end{tikzpicture}
  \end{center}
  \begin{flushright}
    \tikzmarkc{n1}{purple}
    $\A^k\cap U(\theta+\frac{\pi}{k},\frac{\pi}{k})=\{\alpha_7,\alpha_8,\alpha_9,\alpha_{10}\}$
  \end{flushright}
  \begin{flushright}
    \tikzmarkc{n0}{purple}
    $\A^k\cap U(\theta,\frac{\pi}{k})=\{\alpha_3,\alpha_4,\alpha_5,\alpha_6\}$
  \end{flushright}
  \begin{tikzpicture}[remember picture,overlay]
    \draw[->,purple!50!white,thick,dashed] (n1) to[out=150,in=270] (g1);
    \draw[->,purple!50!white,thick] (n0) to[out=180,in=270] (g0);
  \end{tikzpicture}
  \caption{Assume that $\cQ(A^0)$ contains exactly eight ordered pairs of degree
    $k$. Choose four of them, which are up to order different. In this picture
    are the real parts of the leading terms corresponding to the pairs in
    \textcolor{blue!40!black}{blue} plotted. In \textcolor{blue!60!white}{light
      blue} are the real parts corresponding to the flipped pairs drawn. All
    graphs are just horizontally shifted and vertically scaled sinus-functions
    with the same period $\frac{2\pi}{k}$. The points, where the real part is
    negative and the imaginary part vanishes, i.e.\ where the pairs determine an
    anti-Stokes direction, are marked by the \textcolor{red}{red}
    dots.}\label{fig:halfPeriod}
\end{figure} %}}}

\begin{cor}
  The set $\A^k$ can be split into $2k$ distinct half-periods (cf.\
  Figure~\ref{fig:halfPeriod}).
\end{cor}

Let $I=U(\theta,\frac{\pi}{k})$ be an arc width $\frac{\pi}{k}$ such that
$\theta\pm\frac{\pi}{2k}\notin\A^k$.
\begin{rem}\label{rem:remarkAboutCommonStructure}
  For every $\alpha\in\A^k\cap I$ we can write the Stokes matrix as
  \[
    K^\alpha=(K^\alpha_{jl})_{j,l\in\{1,\dots,s\}}\in\SSto_\alpha^k(A^0)
  \]
  where the $K^\alpha_{jl}\in\C^{n_j\times n_l}$ are blocks corresponding to
  the structure of $Q$ (cf.\ Definition~\ref{defn:groupOfFaithfullReps}).
  We then know for $j\neq l$ that, if $K^\alpha_{jl}\neq0$ then
  \begin{itemize}
    \item is $K^\alpha_{lj}=0$ and
    \item for every $\alpha'\in(\A^k\cap I)\backslash\{\alpha\}$ is
      $K^{\alpha'}_{jl}=0$ as well as $K^{\alpha'}_{lj}=0$.
  \end{itemize}
\end{rem}
\begin{rem}
  In the situation of Remark~\ref{rem:remarkAboutCommonStructure}
  there exists a common (block) permutation matrix $P\in\GL_n(\C)$ given by
  $(P)_{jl}=\bdelta_{\pi(j)l}$ where
  \begin{itemize}
    \item $\bdelta_{jl}$ is the block version of Kronecker's delta
      which was introduced in Definition~\ref{defn:groupOfFaithfullReps} and
    \item $\pi$ is the permutation of $\{1,\dots,s\}$ corresponding to
      $q_j\underset{\theta}{\prec}q_l$ \Leftrightarrow{} $\pi(j)<\pi(l)$
      \marginnote{\cite[18]{boalch}}
  \end{itemize}
  such that every matrix $P^{-1}K^{\alpha}P$ is upper triangular and the
  observation from Remark~\ref{rem:remarkAboutCommonStructure} is still
  satisfied.
  \begin{s-rem}
    After moving the sector $U(\theta,\frac{\pi}{k})$ to
    $U(\theta,\frac{\pi}{k})+\frac{\pi}{k}
    =U(\theta+\frac{\pi}{k},\frac{\pi}{k})$ we obtain also a corresponding
    permutation $\pi'$ which is inverse to $\pi$.
    The corresponding permutation matrix $P'$ is then given by $P'=P^{-1}$.

    Such that the permutation $P$ transforms Stokes matrices on the sector
    $I+\frac{\pi}{k}$ to lower triangular matrices
  \end{s-rem}
  In the single leveled case of this phenomenon is discussed on page 18 of
  Boalch's paper~\cite{boalch}.
\end{rem}

In the paper \cite{BJL1979Birkhoff} from Balser, Jurkat and Lutz is the
following Lemma stated as Lemma 2 on page 75.
\begin{lem}\label{lem:UniqueDecompositionWotBlocks}
  Let $T\subset\{1,\dots,n\}\times\{1,\dots,n\}$ be a position set, which
  satisfies the \emph{completeness property}:
  \begin{einr}
    if $(j,k)$ and $(k,l)\in T$ then is also $(j,l)\in T$.
  \end{einr}
  Choose a indexing $i:\{1,\dots,
  \underset{\#T}{\underset{\text{\rotatebox[origin=c]{-90}{$=$}}}{\mu}}\}
  \overset{\cong}{\to}T$ of the position set and denote by
  $\delta_{jl}\in\C$ the \rewrite{ordinary} Kronecker's delta,
  then there exists for every
  \[
    K\in \left\{K=(K_{jl})_{j,l\in\{1,\dots,n\}}\in\GL_n(\C)\mid
      K_{jl}=\delta_{jl} \text{~unless~} (j,l)\in T \right\}
  \]
  unique scalars $t_j\in\C$ such that
  \[
    K=(\id + t_1E_{i(1)})\cdots(\id + t_{\mu}E_{i(\mu)}) \,.
  \]
  \begin{s-rem}
    The completeness property is reasonable, since for example
    $S_{\{(2,3),(3,2)\}}$, corresponding to the not complete set
    $\{(2,3),(3,2)\}$, is not stable under the product:
    \begin{align*}
      \begin{pmatrix}
        1 & a & 0
      \\0 & 1 & 0
      \\0 & 0 & 1
      \end{pmatrix}
      \begin{pmatrix}
        1 & 0 & 0
      \\0 & 1 & b
      \\0 & 0 & 1
      \end{pmatrix}
      =
      \begin{pmatrix}
        1 & a & \textcolor{red!60!black}{ab}
      \\0 & 1 & b
      \\0 & 0 & 1
      \end{pmatrix}
      \notin S_{\{(2,3),(3,2)\}}
      \,.
    \end{align*}
  \end{s-rem}
\end{lem}
\begin{lem}
  Every position set, corresponding to some block, is complete.
\end{lem}
\begin{proof}
  Such a position set, corresponding to some block, is given by
  \[
    T=\{(j,l)\mid j_1\leq j\leq j_2, l_1\leq l\leq l_2\}
  \]
  for some $j_1,j_2,l_2$ and $l_2$ in $\{1,\dots,n\}$.

  Let $(j,k)$, $(k,l)\in T$ be two positions.
  It is obvious that $(j,l)\in T$, since $j_1\leq j\leq j_2$ and
  $l_1\leq l\leq l_2$ are satisfied.
\end{proof}
\begin{cor}
  We can write the Lemma~\ref{lem:UniqueDecompositionWotBlocks} in block form,
  corresponding to the structure of $Q$.
  Let $T\subset\{1,\dots,s\}\times\{1,\dots,s\}$ be a position set and choose a
  indexing
  \[
    i:\{1,\dots, \underset{\#T}{\underset{%
      \text{\rotatebox[origin=c]{-90}{$=$}}}{\mu}}\}\overset{\cong}{\to}T
  \]
  of $T$.
  \begin{s-defn}
    Define the group of matrices, corresponding by a complete position set, by
    \[
      S_T:=\left\{K=(K_{jl})_{j,l\in\{1,\dots,n\}}\in\GL_n(\C)\mid
      K_{jl}=\bdelta_{jl} \text{~unless~} (j,l)\in T \right\} \,,
    \]
    where $\bdelta_{jl}$ is the block version of Kronecker's delta,
    corresponding to the structure of $Q$.
  \end{s-defn}
  From the Lemma~\ref{lem:UniqueDecompositionWotBlocks} then follows that for
  every $K\in S_T$ is the decomposition
  \[
    K=K_1\cdot K_2\cdots K_\mu \,,
  \]
  where $K_j\in S_{\{i(j)\}}$, is unique.
\end{cor}
From the previous corollary we deduce the following corollary.
\begin{cor}\label{cor:composeLevelwise}
  Let $T_1,\dots,T_r\subset\{1,\dots,s\}\times\{1,\dots,s\}$ be a distinct
  position sets, such that
  \begin{einr}
    $T:=T_1\dot\cup\dots\dot\cup T_r$ \rewrite{as well as} every $T_m$ satisfy
    the completeness property.
  \end{einr}
  Then is by
  \begin{align*}
    S_{T_1}\times\cdots\times S_{T_m} &\longrightarrow S_T
  \\(K_1,\dots,K_m) &\longmapsto K_1\cdots K_m
  \end{align*}
  an isomorphism defined.
\end{cor}
We can apply the previous corollary in our situation. This yields the following
theorem.
\begin{thm}\label{thm:theoremForlargerDecomp}
  Let $\theta\in S^1$ be a fixed direction which satisfies
  $\theta\pm\frac{\pi}{2k}\notin\A^k$, then
  \[
    \cH(A^0)\cong\prod_{k\in\cK}\prod_{j\in\{1,\dots,2k\}}
    \hat\Sto_{\theta+j\frac{\pi}{k}}^k(A^0) \,,
  \]
  where
  \begin{align*}
    \hat\SSto_{\theta}^k(A^0)
    :\!\!&= \left\{K=(K_{jl})_{j,l\in\{1,\dots,s\}}\in\GL_n(\C)\mid
        K_{jl}=\delta_{jl} \text{~unless~}
        q_j \underset{\theta}{\prec} q_l \text{, } k_{jl}=k\right\}
    \\&=S_{\{(j,l) \mid q_j \underset{\theta}{\prec} q_l \}}
  \end{align*}
  and
  \[
    \hat\Sto_{\theta}^k(A^0):=\{\rho_\theta^{-1}(K)
      \mid K\in\hat\SSto_{\theta}^k(A^0) \} \,.
  \]
  \begin{s-rem}
    \begin{enumerate}
      \item This means that the information of all Stokes-matrices on every
        level $k\in\cK$ can be grouped into $2k$ matrices, which are products
        of the corresponding Stokes matrices.
      \item By the definition above, it is obvious that
        \[
          \Sto_{\theta}^k(A^0)\cong
          \bigcap_{\theta'\in \A^k\cap U(\theta,\frac{\pi}{k})}
          \hat\Sto_{\theta'}^k(A^0) \,.
        \]
    \end{enumerate}
  \end{s-rem}
\end{thm}
But there is in the multileveled case no obvious way to find a subset $J\in S^1$
and an isomorphism
\[
  \prod_{\theta\in J} \hat\Sto_{\theta}(A^0)
  ~~\not{\!\!\!\!\longrightarrow}
  \prod_{k\in\cK}\prod_{j\in\{1,\dots,2k\}}
  \hat\Sto_{\theta+j\frac{\pi}{k}}^k(A^0)\,,
\]
where $\hat\Sto_{\theta}(A^0)$ was defined in Remark~\ref{rem:representation}.
\begin{proof}
  We have an isomorphism
  \[ \begin{tikzcd}
    \eta_\theta^k:
    \underset{\alpha\in\A^k\cap U(\theta,\frac{\pi}{k})}\prod
    \underset{S_{\{(j,l)\mid q_j\myrel{\alpha}q_l\}}}{%
      \underset{\text{\rotatebox[origin=c]{-90}{$=$}}}{\underbrace{%
          \SSto_\alpha^k(A^0)}}}
    \rar{\cong}&
    \underset{S_{\{(j,l) \mid q_j \underset{\theta}{\prec} q_l \}}}{%
      \underset{\text{\rotatebox[origin=c]{-90}{$=$}}}{\underbrace{%
          \hat\SSto_\theta^k(A^0)}}}
    \,.
  \end{tikzcd} \]
  from Corollary~\ref{cor:composeLevelwise}, since
  \begin{itemize}
    \item for every $\alpha\in\A$ satisfies the set
      $\{(j,l)\mid q_j\myrel{\alpha}q_l\}$ the completeness
      property, since it is defined via a \rewrite{transitiv relation}, and
    \item the union of all $\{(j,l)\mid q_j\myrel{\alpha}q_l\}$
      for $\alpha\in I\cap\A^k$ is then
      \begin{align*}
        \dot\bigcup_{\alpha\in I\cap\A}
          \{(j,l)\mid q_j\myrel{\alpha}q_l\}
          &=\{(j,l) \mid q_j \myrel{\alpha} q_l
            \text{~for some~}\alpha\in\A^k\cap U(\theta,\frac{\pi}{k}) \}
        \\&= \{(j,l) \mid q_j \underset{\theta}{\prec} q_l \}
          \qquad\qquad\text{(cf.\ Remark~\ref{rem:relationDistanceCondition})}
      \end{align*}
      and is also complete, since $\underset{\theta}{\prec}$ is also a
      \rewrite{transitive relation}.
  \end{itemize}
  The isomorphism of the theorem is then the Stokes germ version of
  \[ \begin{tikzcd}
    \underset{\eta}{%
      \underset{\text{\rotatebox[origin=c]{90}{$:=$}}}{\underbrace{%
        \underset{k\in\cK}{\prod}\underset{j\in\{1,\dots,2k\}}{\prod}
        \eta_{\theta+j\frac{\pi}{k}}^k
    }}}:
    \underset{\cH(A^0)}{%
      \underset{\text{\rotatebox[origin=c]{-90}{$\cong$}}}{\underbrace{%
        \prod_{k\in\cK}\prod_{\alpha\in\A^k}\SSto_\alpha^k(A^0)}}}
    \rar{\cong}&
    \underset{k\in\cK}{\prod}\underset{j\in\{1,\dots,2k\}}{\prod}
    \hat\SSto_{\theta+j\frac{\pi}{k}}^k(A^0) \,.
  \end{tikzcd} \]
\end{proof}
\begin{cor}
  This does also induce an isomorphism
  \[ \begin{tikzcd}
    \eta:
    \displaystyle\prod_{k\in\cK}\prod_{\alpha\in\A}\Sto_{\alpha}^{k}(A^0)
    \rar{\cong}&
    \displaystyle\prod_{k\in\cK} \prod_{j\in\{1,\dots,2k\}}
      \hat\Sto_{\theta+j\frac{\pi}{k}}^k(A^0)
  \end{tikzcd} \]
  on the level of Stokes germs instead of Stokes matrices.
\end{cor}
\begin{rem}
  The, in the proof of Theorem~\ref{thm:theoremForlargerDecomp}, obtained
  isomorphism
  \[
    \eta: \prod_{\alpha\in\A}\SSto_{\alpha}(A^0) \longrightarrow
    \prod_{k\in\cK} \prod_{j\in\{1,\dots,2k\}}
    \hat\SSto_{\theta+j\frac{\pi}{k}}^k(A^0)
  \]
  is in the single-leveled case with $n=s$, i.e.\ all diagonal elements of
  $Q$ are different, given in Lemma 3.2 of Boalch's paper
  \cite[Lem.3.2]{boalch}.
  In this case is every $\hat\SSto_{\theta+j\frac{\pi}{k}}^k(A^0)$ isomorphic
  to some $PU_+P^{-1}$ where $U_+$ is the group of all upper triangular
  matrices, with ones on the diagonal, and $P$ is a permutation matrix.
\end{rem}

%%% Local Variables:
%%% TeX-master: "Maximilian_Huber-Masters_Thesis-with_notes.tex"
%%% End:
