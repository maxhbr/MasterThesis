%%%%%%%%%%%%%%%%%%%%%%%%%%%%%%%%%%%%%%%%%%%%%%%%%%%%%%%%%%%%%%%%%%%%%%%%%%%%%%%
\section{The complete Diagram}\label{sec:theCompleteDiagram}
If we start with the diagram from Page~\pageref{page:ofPreDiagram}, we can add
all the defined isomorphisms and rewrite into the following commutative
diagram of \textbf{isomorphisms of pointed sets}.
\begin{center}
  \begin{tikzpicture}[scale=0.95,transform shape]
    \pgfmathsetmacro{\dx}{4.5cm}
    \pgfmathsetmacro{\ldx}{6cm}
    \pgfmathsetmacro{\dy}{2.5cm}
    \pgfmathsetmacro{\mdy}{-1.5cm}

    \node (cH) at (-6,3.5) {$\cH(A^0)$};
    \node (cH2) at ([yshift=-\dy]cH)
      {$G(\!\{t\}\!)\backslash\hat G(A^0)$};
    \node (cH3) at ([yshift=-\dy]cH2)
      {$\Gamma(\dot\cU;\Lambda(A^0))$};
    \node (ProdOfStos) at (-6,-4)
      {$\displaystyle\prod_{\alpha\in\A}\Sto_{\alpha}(A^0)$};
    \node (S3P) at ([xshift={\dx},yshift={\dy}]ProdOfStos)
      {$\displaystyle\prod_{k\in\cK}\prod_{\alpha\in\A}\Sto_{\alpha}^{k}(A^0)$};
    \node (S2P) at ([xshift={\dx},yshift={\dy}]S3P)
      {$\displaystyle\prod_{k\in\cK}\Gamma(\dot\cU^k;\Lambda^k(A^0))$};
    \node (StA0) at ([xshift={\dx},yshift={\dy}]S2P)
      {$\St(A^0)$};
    \node (ProdOfSStos) at ([yshift=-{\dy}]ProdOfStos)
      {$\displaystyle\prod_{\alpha\in\A}\SSto_{\alpha}(A^0)$};
    \node (SS3P) at ([yshift=-{\dy}]S3P)
      {$\displaystyle\prod_{k\in\cK}\prod_{\alpha\in\A}\SSto_{\alpha}^{k}(A^0)$};
    \node (SS2P) at ([yshift={\mdy},xshift={\ldx}]S3P)
      {$\displaystyle\prod_{k\in\cK} \prod_{j\in\{1,\dots,2k\}}
        \hat\Sto_{\theta+j\frac{\pi}{k}}^k(A^0)$};
    \node (SSS2P) at ([yshift={\mdy},xshift={\ldx}]SS3P)
      {$\displaystyle\prod_{k\in\cK} \prod_{j\in\{1,\dots,2k\}}
        \hat\SSto_{\theta+j\frac{\pi}{k}}^k(A^0)$};
    \node[] (SSS3P) at ([yshift={\mdy},xshift={\ldx}]ProdOfSStos)
      {$\displaystyle\prod_{\alpha\in\A}
        \C^{\sum_{q_j\myrel{\alpha}q_l}
          \deg(q_j-q_l)\cdot n_j\cdot n_l}$};

    \draw[->] (cH2) -- (cH)
      node[midway,right]
      {\footnotesize(given by Corollary~\ref{cor:isomorphyOfClassfset})};
    \draw[->] (cH) -- (StA0) node[midway,above] {$\exp_{A^0}$};
    \draw[->,purple] (ProdOfStos) to
      node[midway,above,sloped]
      {$\chi\circ\prod_{\alpha\in\A}i_\alpha$}
      (S3P);
    \draw[->,purple] (S3P) -- (S2P)
      node[midway,above,sloped] {$\prod_{k\in\cK}i^k$};
    \draw[->,purple] (S2P) -- (StA0)
      node[midway,above left,sloped] {$\cT$};
    \draw[->] (ProdOfStos) -- (ProdOfSStos)
      node[midway,left] {$\prod_{\alpha\in\A}\rho_\alpha$};
    \draw[->] (ProdOfSStos) -- (SS3P)
      node[midway,below,sloped] {$\chi\circ\prod_{\alpha\in\A}i_\alpha$};
    \draw[->] (S3P) -- (SS3P)
      node[midway,right] {$\prod_{k\in\cK}\prod_{\alpha\in\A}\rho_{\alpha}^k$};
    \draw[->] (S3P) -- (SS2P) node[midway,above,sloped] {$\eta$};
    \draw[->] (SS3P) -- (SSS2P) node[midway,above,sloped] {$\eta$};
    \draw[->] (SS2P) -- (SSS2P)
      node[midway,right] {\footnotesize(induced by $\rho$)};
    \draw[->] (SSS3P) -- (ProdOfSStos)
      node[midway,below,sloped] {$\prod_{\alpha\in\A}\vartheta_\alpha$};

    % \draw[->] (cH) to[out=220,in=100]  node[midway,left] {$g$} (ProdOfStos);
    \path[blue] (cH) edge [->,transform canvas={xshift=-1.3mm}] (cH2);
    \draw[->,blue] (cH3) -- (ProdOfStos)
      node[midway,left] {\footnotesize(take germs)};
    \draw[->,blue] (cH2) -- (cH3)
      node[midway,right] {$\tilde g$};
    \draw[->] (S2P) to[in=45,out=180] node[midway,above,sloped] {$\tau$} (cH3);
  \end{tikzpicture}
\end{center}
Where
\begin{itemize}
  \item the map $g$, which arises from the theory of summation
  \item the \textcolor{purple}{purple path} is the isomorphism $h$ from
    Theorem~\ref{thm:mainThm2} and
  \item the \textcolor{blue}{blue path} is the isomorphism $g$ from
    Theorem~\ref{thm:mainThm2}, where $\tilde g$ arises from the theory of
    multisummability and is introduced Definition~\ref{defn:theMapG} in the
    Appendix~\ref{app:multisummability}, and
  \item we denote
    \begin{itemize}
      \item $\chi:\displaystyle \prod_{\alpha\in\A}\prod_{k\in\cK}\Sto_\alpha^k(A^0)
        \equiv\displaystyle \prod_{k\in\cK}\prod_{\alpha\in\A}\Sto_\alpha^k(A^0)$
        the reordering and
      \item by abuse of notation we also denote the Stokes matrix version in
        the same way
        $\chi:\displaystyle \prod_{\alpha\in\A}\prod_{k\in\cK}\SSto_\alpha^k(A^0)
        \equiv\displaystyle
        \prod_{k\in\cK}\prod_{\alpha\in\A}\SSto_\alpha^k(A^0)$.
    \end{itemize}
\end{itemize}

%%% Local Variables:
%%% TeX-master: "Maximilian_Huber-Masters_Thesis-with_notes.tex"
%%% End:
