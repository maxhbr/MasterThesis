\section{A taste of summability}\label{app:multisummability}
For the sake of completeness, let us speak a little bit about summability in
this section. We will only mention some facts to give an rough idea how the map
$g$ in the diagram on page~\pageref{page:ofPreDiagram} arises.
The understanding of this section is not necessary for the rest of this
chapter. It is enough to think of the map $g$ (resp.\ $\tilde g$) as a black box
which uses some theory to make the asymptotic expansion unique and yields an
element of $\prod_{\alpha\in\cA}\Sto_\alpha(A^0)$ (resp.\ of
$\Gamma(\dot\cU;\Lambda(A^0))$) corresponding to some marked pair.

\marginnote{\cite[111]{Loday2014}}
The aim of a theory of summation is to associate with any series an asymptotic
function uniquely determined in a way as much natural as possible.
A useful and extensive resources for this topic are Loday-Richaud's
book~\cite{Loday2014}.
A resource which looks only at an special case, which naturally arises from the
previous sections, is Section III of the paper~\cite{Loday1994} and some of the
most important statements, i.e.\ exactly those we need, are found in many places
like Boalch's publications~\cite{boalch,thboalch}, although he looks only at the
single-leveled case.

The following Proposition can be found in~\cite[Prop.III.2.1]{Loday1994}
and~\cite[Thm.4.3.13]{Loday2014}.
\begin{prop}\label{prop:multisummability}
  \marginnote{\cite[Sec.III.2.1]{Loday1994}}
  Let $\cV=(V_j)_{j\in J}$ be a cyclic covering,
  Let $\hat F\in\hat G(A^0)$ be a formal transformation and let
  $\dot\phi=(\dot\phi_j)_{j\in J}\in\Gamma(\dot{\mathcal{V}};\Lambda(A^0))$,
  be a $1$-cocycle in the cohomology class $\exp(\hat F)$.

  There exists a \textbf{unique} family of realizations $(F_j)_{j\in J}$ of
  $\hat F$ over $\mathcal{V}$, i.e.\ matrices $F_j$ which are analytic on
  $V_j$, satisfy $[A^0,A]$ and are asymptotic to $\hat F$ on $V_j$, such that
  \[
    \dot\phi_j=F_{j-1}F_j^{-1} \in\Gamma(V_{j-1}\cap V_j;\Lambda(A^0))
  \]
  for every $j\in J$.
\end{prop}
When $\mathcal{V}$ is the, in Section~\ref{sec:proofOfMatrixThm} defined,
cyclic covering $\cU^{\leq k_r}=:\cU$ and $\dot\phi$ is in its Stokes
form\TODO[~(cf.~??)], we call the realizations $F_{\alpha}$, which were obtained
by the previous proposition, the \emph{sums of $\hat F$}.
\begin{defn}\label{defn:sumsLeftRight}
  \marginnote{\cite[Defn.III.2.2]{Loday1994}}
  Denote by $\alpha^+$ the next anti-Stokes direction on the right of $\alpha$
  we can define
  \begin{itemize}
  \item $S_\alpha^-(\hat F):=F_\alpha\in\Gl_n(\cA(U_\alpha))$ as the \emph{sum
      of $\hat F$ on the left of $\alpha$} and
  \item $S_\alpha^+(\hat F):=F_{\alpha^+}\in\Gl_n(\cA(U_{\alpha^+}))$ as the
    \emph{sum of $\hat F$ on the right of $\alpha$}.
  \end{itemize}
\end{defn}
\begin{rem}
  \marginnote{\cite[Defn.3.6]{boalch}}
  $S^+_\alpha(\hat F)t^Le^{Q(t^{-1})}$ is a solution of $[A]$ on the
  corresponding sector.
  \TODO[ or $S^-_\alpha(\hat F)t^Le^{Q(t^{-1})}$?  or
    $t^Le^{Q(t^{-1})}S^+_\alpha(\hat F)$?  or
  $t^Le^{Q(t^{-1})}S^-_\alpha(\hat F)$?]
  Using the equation (\ref{eq:representation}) we can write
  \[
    S^+_\alpha(\hat F)
    \cY_{0,\alpha}(t)
    =
    S^-_\alpha(\hat F)
    \cY_{0,\alpha}(t)C_{\cY_{0,\alpha}}
  \]
  thus the Stokes matrix $C_{\cY_{0,\alpha}}$ is a matrix, which
  describes the \rewrite{blending} between the two adjacent sectors.
  \TODO[see Loday's remarks on thm.4.3.13 in \cite{Loday2014}]
\end{rem}

\subsubsection{The map
  $g:\cH(A^0)\to\prod_{\alpha\in\cA}\Sto_\alpha(A^0)$}\label{sect:multisummmap}
There are alternative and more constructive definitions, which are equivalent to
Definition~\ref{defn:sumsLeftRight}.
Since some of them are more direct and do not use the $1$-cocycle, to which they
correspond, they can be used to obtain the corresponding
element in $\prod_{\alpha\in\cA}\Sto_\alpha(A^0)$.
Many different approaches of (multi-)summation can be found in~\cite{Loday2014}.

Let us fix an ambassador $\hat F$ in $G\backslash\hat G(A^0)$ corresponding to
equivalence class of marked pairs in $\cH(A^0)$.
The theory of summation says that there is a map which takes $\hat F$ to a
unique and natural set of functions
\[
  (S_{\alpha_1}^-(\hat F),
  \underset{S_{\alpha_1}^+(\hat F)}{
    \underset{\text{\rotatebox[origin=c]{-90}{$=$}}}{\underbrace{
        S_{\alpha_2}^-(\hat F)
      }}}
  ,\dots,S_{\alpha_\nu}^-(\hat F))\in
  \prod_{\alpha\in\A}\Gl_n(\cA(U_\alpha)) \,,
\]
which have $\hat F$ as asymptotic functions.
The corresponding element in $\prod_{\alpha\in\A}\Sto_\alpha(A^0)$ is then found
as
\[
  (\phi_{\alpha_1},\dots,\phi_{\alpha_\nu})
  \in\Gamma(\cU;\Lambda(A^0))
\]
by setting $\phi_\alpha:=
\left(S^-_\alpha(\hat F)\right)̂^{̀-1}S^+_\alpha(\hat F)\in\Gamma(U_\alpha;\Lambda(A^0))$.
\begin{defn}\label{defn:theMapG}
  The corresponding map is denoted by
  \[
    \tilde g:G(\!\{t\}\!)\backslash\hat G(A^0)
    \longrightarrow
    \Gamma(\dot\cU;\Lambda(A^0)) \,.
  \]
  The map $g$ mentioned in the beginning of Chapter~\ref{chap:stokes} is build
  as shown in Section~\ref{sec:theCompleteDiagram}.
\end{defn}
%%%%%%%%%%%%%%%%%%%%%%%%%%%%%%%%%%%%%%%%%%%%%%%%%%%%%%%%%%%%%%%%%%%%%%%%%%%%%%%
\section{The complete diagram}\label{sec:theCompleteDiagram}
In this section we want to write the improved version of the diagram from
page~\pageref{page:ofPreDiagram} down. It will contain nearly all of the
isomorphisms which were defined in the previous sections. Additionally will
isomorphisms like $g$ be decomposed into their building blocks.

In the following diagram of \textbf{isomorphisms of pointed sets} was the
isomorphism $\exp_{A^0}$ discussed in Section~\ref{sec:mainThm1}. The
\textcolor{purple}{purple} path together with the lower left part was discussed
in Section~\ref{sec:mainThm2}. The lower right part was discussed in
Section~\label{sec:furtherImprovements} and the \textcolor{blue}{blue} part on
the left was discussed in the previous section.

\begin{center}
  \begin{tikzpicture}[scale=0.95,transform shape]
    \pgfmathsetmacro{\dx}{4.5cm}
    \pgfmathsetmacro{\ldx}{6cm}
    \pgfmathsetmacro{\dy}{2.5cm}
    \pgfmathsetmacro{\mdy}{-1.5cm}

    \node (cH) at (-6,3.5) {$\cH(A^0)$};
    \node (cH2) at ([yshift=-\dy]cH)
      {$G(\!\{t\}\!)\backslash\hat G(A^0)$};
    \node (cH3) at ([yshift=-\dy]cH2)
      {$\Gamma(\dot\cU;\Lambda(A^0))$};
    \node (ProdOfStos) at (-6,-4)
      {$\displaystyle\prod_{\alpha\in\A}\Sto_{\alpha}(A^0)$};
    \node (S3P) at ([xshift={\dx},yshift={\dy}]ProdOfStos)
      {$\displaystyle\prod_{k\in\cK}\prod_{\alpha\in\A}\Sto_{\alpha}^{k}(A^0)$};
    \node (S2P) at ([xshift={\dx},yshift={\dy}]S3P)
      {$\displaystyle\prod_{k\in\cK}\Gamma(\dot\cU^k;\Lambda^k(A^0))$};
    \node (StA0) at ([xshift={\dx},yshift={\dy}]S2P)
      {$\St(A^0)$};
    \node (ProdOfSStos) at ([yshift=-{\dy}]ProdOfStos)
      {$\displaystyle\prod_{\alpha\in\A}\SSto_{\alpha}(A^0)$};
    \node (SS3P) at ([yshift=-{\dy}]S3P)
      {$\displaystyle\prod_{k\in\cK}\prod_{\alpha\in\A}\SSto_{\alpha}^{k}(A^0)$};
    \node (SS2P) at ([yshift={\mdy},xshift={\ldx}]S3P)
      {$\displaystyle\prod_{k\in\cK} \prod_{j\in\{1,\dots,2k\}}
        \hat\Sto_{\theta+j\frac{\pi}{k}}^k(A^0)$};
    \node (SSS2P) at ([yshift={\mdy},xshift={\ldx}]SS3P)
      {$\displaystyle\prod_{k\in\cK} \prod_{j\in\{1,\dots,2k\}}
        \hat\SSto_{\theta+j\frac{\pi}{k}}^k(A^0)$};
    \node[] (SSS3P) at ([yshift={\mdy},xshift={\ldx}]ProdOfSStos)
      {$\displaystyle\prod_{\alpha\in\A}
        \C^{\sum_{q_j\myrel{\alpha}q_l}
          \deg(q_j-q_l)\cdot n_j\cdot n_l}$};

    \draw[->] (cH2) edge [->,transform canvas={xshift=2mm}]
      node[midway,right]
      {\footnotesize(cf.\ Cor.\ref{cor:isomorphyOfClassfset})}
      (cH);
    \draw[->] (cH) -- (StA0) node[midway,above] {$\exp_{A^0}$};
    \draw[->,purple] (ProdOfStos) to
      node[midway,above,sloped]
      {$\chi\circ\prod_{\alpha\in\A}i_\alpha$}
      (S3P);
    \draw[->,purple] (S3P) -- (S2P)
      node[midway,above,sloped] {$\prod_{k\in\cK}i^k$};
    \draw[->,purple] (S2P) -- (StA0)
      node[midway,above left] {$\cT$};
    \draw[->] (ProdOfStos) -- (ProdOfSStos)
      node[midway,left] {$\prod_{\alpha\in\A}\rho_\alpha$};
    \draw[->] (ProdOfSStos) -- (SS3P)
      node[midway,below,sloped] {$\chi\circ\prod_{\alpha\in\A}i_\alpha$};
    \draw[->] (S3P) -- (SS3P)
      node[midway,right] {$\prod_{k\in\cK}\prod_{\alpha\in\A}\rho_{\alpha}^k$};
    \draw[->] (S3P) -- (SS2P) node[midway,above] {$\eta$};
    \draw[->] (SS3P) -- (SSS2P) node[midway,above] {$\eta$};
    \draw[->] (SS2P) -- (SSS2P)
      node[midway,right]
      {$\hat\rho$};
    \draw[->] (SSS3P) -- (ProdOfSStos)
      node[midway,below,sloped] {$\prod_{\alpha\in\A}\vartheta_\alpha$};

    % \draw[->] (cH) to[out=220,in=100]  node[midway,left] {$g$} (ProdOfStos);
    \draw[->,blue] (cH) -- (cH2);
    \draw[->,blue] (cH3) -- (ProdOfStos)
      node[midway,left] {\footnotesize(take germs)};
    \draw[->,blue] (cH2) -- (cH3)
      node[midway,right] {$\tilde g$};

    \draw[->,gray] (ProdOfStos) to[out=60,in=185] node[midway,above] {$\mathfrak{T}$} (S2P);
    \ifnum\myDevelopVariable=1
        \draw[right hook->,dashed,red] (S2P) to[in=39,out=180] node[midway,above] {$\tau$} (cH3);
        \draw[->>,dashed,red] (cH3) to[out=52,in=195] node[midway,above,sloped] {(quotient map)} (StA0);
    \fi
  \end{tikzpicture}
\end{center}
Where
\begin{itemize}
  \item the \textcolor{purple}{purple path} is the isomorphism $h$ from
    Theorem~\ref{thm:mainThm2},
  \item the \textcolor{blue}{blue path} is the isomorphism $g$, where $\tilde g$
    arises from the theory of multisummability and is introduced
    Definition~\ref{defn:theMapG} and
  \item we denote
    \begin{itemize}
      \item $\hat\rho:=\prod_{k\in\cK}\prod_{j\in\{1,\dots,2k\}}\hat\rho_{\theta+j\frac{\pi}{k}}^k$,
      \item $\chi:\displaystyle \prod_{\alpha\in\A}\prod_{k\in\cK}\Sto_\alpha^k(A^0)
        \equiv\displaystyle \prod_{k\in\cK}\prod_{\alpha\in\A}\Sto_\alpha^k(A^0)$
        the reordering and
      \item by abuse of notation we also denote the Stokes matrix version in
        the same way
        $\chi:\displaystyle \prod_{\alpha\in\A}\prod_{k\in\cK}\SSto_\alpha^k(A^0)
        \equiv\displaystyle
        \prod_{k\in\cK}\prod_{\alpha\in\A}\SSto_\alpha^k(A^0)$.
    \end{itemize}
\end{itemize}

%%% Local Variables:
%%% TeX-master: "Maximilian_Huber-Masters_Thesis-with_notes.tex"
%%% End:
