\chapter{Asymptotic expansion}
\begin{comment}
  ´classical'
  \begin{itemize}
    \item \cite[60]{sabbah_cimpa90} Chapter II.2.2
      \begin{itemize}
        \item \cite{sabbah2000equations}
      \end{itemize}
    \item \textbf{\textcolor{blue}{Van der Put:
          \cite[Chapter 7]{van2003galois}: Exact Asymptotics}}
    \item \cite{majima1984asymptotic}
    \item \cite{Balser2000Formal}
    \item \cite{Loday1994}
    \item \textbf{\textcolor{blue}{\cite{Loday2014} Chapter 2}}
  \end{itemize}
  ´sheafical'
  \begin{itemize}
    \item \textbf{\textcolor{red}{\cite[II.5]{sabbah2007isomonodromic}}}
  \end{itemize}
  \TODO:
  \begin{itemize}
    \item \cite{sibuya1990Linear} Appendix A.3
  \end{itemize}
\end{comment}
In this chapter, we want to look ak asymptotic expansions (at $0$).
There are various references for Poincarè asymptotic expansions, or for shor,
asymptotic expansions.
We will mostly refer to \cite[chapter 2]{Loday2014} and
\cite[chapter 7]{van2003galois}.
\TODO[For the second definition we will also use
\cite{sabbah2007isomonodromic}.]

\TODO[Define Sect command]

We introduce the following notations:
\begin{notations} We denote by
  \begin{itemize}
    \item $\mathfrak{s}=\mathfrak{s}_{a,b}(r)$, $a,b\in\S^1$ and $r\in\R_{>0}$
      the open sector
      \[
        \mathfrak{s}_{a,b}(r):=\left\{t\in\C \mid a<\arg(t)<b, 0<|t|<r\right\}
      \]
      of all points $t\in\C$ satisfying $a<\arg(t)<b$ and $0<|t|<r$;
    \item $\mathfrak{s}_{I}(r)=\mathfrak{s}_{a,b}(r)$ for $I=(a,b)$;
    \item $\bar{\mathfrak{s}}=\bar{\mathfrak{s}}_{a,b}(r)$ the closure of
      $\mathfrak{s}_{a,b}(r)$ in $\C^*:=\C\backslash\{0\}$.
  \end{itemize}
  \begin{center}
    \begin{tikzpicture}[scale=4]
      \node (zero) at (0,0) {};
      \node[below left] at (zero) {$0$};
      \draw[blue,dashed] (zero) circle (1cm);
      \node[blue,right] at (-1,-.7) {$S^1$};

      \filldraw[fill=green!20!white
        ,draw=green!60!black
        ,thick
        ,path fading=west] (0,0)
      -- ({cos( -30 )*.65},{sin( -30 )*.65}) arc (-30:70:.65) -- cycle;
      \draw[blue!60!black,thick] (0,0) -- ({cos( -30 )*.65},{sin( -30)*.65});
      \node[green!40!black] at (.4,.3)
        {$\mathfrak{s}_{\textcolor{red!60!black}{a,b}}
        (\textcolor{blue!60!black}{r})$};
      \node[green!40!black] at (.3,-.25) {$\textcolor{blue!60!black}{r}$};

      \draw[thick,red!60!black] ({cos( -30 )},{sin( -30 )}) arc (-30:70:1);

      \fill[red!60!black] ({cos( -30 )},{sin( -30 )}) circle(.7pt);
      \fill[red!60!black] ({cos( 70 )},{sin( 70 )}) circle(.7pt);

      \node[red!60!black] at ({1.1 * cos(70)},{1.1 * sin(70)}) {$a$};
      \node[red!60!black] at ({1.1 * cos(-30)},{1.1 * sin(-30)}) {$b$};

      \node[red!60!black,right] at (.8,.6) {$I$};

      \fill[white] (zero) circle (1.5pt);
      \fill (zero) circle (.7pt);
    \end{tikzpicture}
  \end{center}
\end{notations}
\begin{rem}
  \begin{itemize}
    \item We will say, that \emph{a sector $\mathfrak{s}_{I}(r)$ contains
      $\theta\in S^1$} if $\theta\in I$. In the same way we will say, that
      $\mathfrak{s}_I(r)$ contains $U\subset S^1$ if $U\subset I$ or that
      $U\subset S^1$ contains a sector $\mathfrak{s}_I(r)$ if $I\subset U$.
    \item Since the explicit value $r$ of the radius does not matter, we shall
      simply speak of an \emph{arc} $I$ on $S^1$ in place of a sector
      $\mathfrak{s}_{I}(r)$ for a sufficiently small $r$.
  \end{itemize}
\end{rem}
\begin{defn}
  A sector $\mathfrak{s}_{a',b'}(r')$ is said to be a proper sub-sector of the
  sector $\mathfrak{s}_{a,b}(r)$,
  $\mathfrak{s}_{a',b'}(r')\Subset\mathfrak{s}_{a,b}(r)$, if its closure
  $\bar{\mathfrak{s}}_{a',b'}(r')$ is included in $\mathfrak{s}_{a,b}(r)$.
  \begin{rem}
    $\mathfrak{s}_{a',b'}(r')\Subset\mathfrak{s}_{a,b}(r)$ if and only if
    \begin{itemize}
      \item $a<a'<b'<b$ and $r'<r$.
    \end{itemize}
  \end{rem}
\end{defn}

\begin{defn}
  \def\myN{\textbf{\textcolor{blue!40!black}{N}}}
  \def\mySect{\textcolor{red!40!black}{\mathfrak{s}'}}
  \def\myConst{\textcolor{green!40!black}{C(\myN,\mySect)}}
  A function $f\in\cO(\mathfrak{s})$ is said to \emph{have the formal Laurent
  series $\sum_{n\geq n_0}a_nt^n$ as asymptotic expansion} (or \emph{to be
  asymptotic to the series}) on the sector $\mathfrak{s}$ if
  \begin{itemize}
    \item for all proper sub-sectors $\mySect\Subset\mathfrak{s}$ and
    \item all $\myN\in\N$,
  \end{itemize}
  there exists a constant
  $\myConst$ such that the following estimate holds
  \[
    \left|
      f(x)-\sum_{n_0\leq n\leq \myN-1}c_nt^n
    \right|
    \leq \myConst|t|^{\myN} \qquad \text{ for all }t\in \mySect \,.
  \]
  \begin{rem}
    Sometimes, for example in \cite{sabbah_cimpa90}, this condition is written
    as
    \[
      \lim_{z\to0,z\in{\mySect}}
      |t|^{-(\myN-1)}
      \left|
        f(x)-\sum_{n_0\leq n\leq \myN-1}c_nt^n
      \right|=0
      \qquad \text{ for all } t\in \mySect \,.
    \]
  \end{rem}
  \begin{enumerate}
    \item We will say, that \emph{$f$ has the formal Laurent series $\hat g$ as
      asymptotic expansion on the interval $I\subset S^1$} if there exists a
      $r\in\R_{>0}$ such that $f$ has the formal Laurent series $\hat g$ as
      asymptotic expansion on $\mathfrak{s}_I(r)$.
    \item We will say, that \emph{$f$ has the formal Laurent series $\hat g$ as
      asymptotic expansion in the direction $\theta$} if there exists an
      interval $\theta\in I\subset S^1$ such that $f$ has formal Laurent series
      $\hat g$ as asymptotic expansion on the interval $I$.
  \end{enumerate}
\end{defn}
If $f$ has $\hat g$ as asymptotic expansion on the sector $\mathfrak{s}$ (or on
the interval $I\subset S^1$, or in the direction $\theta$), we denote that by
$f\sim_{\mathfrak{s}}\hat g$ (or $f\sim_{I}\hat g$, or $f\sim_{\theta}\hat g$).
\begin{notations}
  \begin{itemize}
    \item Define $\cA(\mathfrak{s})$ as the set of holomorphic functions
      $f\in\cO(\mathfrak{s})$ which admit an asymptotic expansion at $0$ on
      $\mathfrak{s}$.
    \item One writes
      \[
        T_{\mathfrak{s}}:\cA(\mathfrak{s})\to\C(\!(t)\!)
      \]
      for the mapping, which associates to each
      function $f\in\cA(\mathfrak{s})$ its asymptotic expansion.
    \item A function $f$ with $T_{\mathfrak{s}}(f)=\id_{\mathfrak{s}}$ is
      called \emph{flat (on $\mathfrak{s}$)}
    \item Denote by $\cA^{<0}(\mathfrak{s})$ the set
      $\ker(T_{\mathfrak{s}})\subset\cA(\mathfrak{s})$ of functions asymptotic
      to zero at $0$.
    \item Define $\cA(a,b)\!\!\overset{(a,b)=I}{=}\!\!\cA(I)$
      as the limit
      $\underset{r\to0}{\underrightarrow{\lim}}\cA(\mathfrak{s}_{a,b}(r))$.
      \begin{itemize}
        \item[] In more detail this means that the elements of $\cA(a,b)$
          are pairs $(f,\mathfrak{s}_{a,b}(r))$ mit
          $f\in\cA(\mathfrak{s}_{a,b}(r))$. The equivalence relation is given
          by $(f_1,\mathfrak{s}_{a,b}(r_1))\sim(f_2,\mathfrak{s}_{a,b}(r_2))$
          if there is a pair $(f_3,\mathfrak{s}_{a,b}(r_3))$ such that
          $r_3<\min(r_1,r_2)$ and $f_3=f_1=f_2$ on $\mathfrak{s}_{a,b}(r_3)$.
      \end{itemize}
      Analogous $\cA^{<0}(I)$ is defined.
  \end{itemize}
\end{notations}
\begin{rem}
  One can also define 
  \begin{itemize}
    \item $\cA(I)$ and $\cA(\theta)$,
    \item $T_I$ and $T_\theta$,
    \item $\cA^{<0}(I)$ and $\cA^{<0}(\theta)$.
  \end{itemize}
\end{rem}
It can easily be seen, that $\cA:U\to\cA(U)$ defines a sheaf on $S^1$.
\begin{comment}
  \begin{rem}
    $\cA_\theta=\cA(\theta)$ and $\cA^{<0}_\theta=\cA^{<0}(\theta)$.
  \end{rem}
\end{comment}
\begin{itemize}
    \marginnote[194]{\cite{van2003galois}}
  \item $\cA(S^1)=$\TODO
  \item\dots
\end{itemize}

\TODO[Properties,Examples,\dots]
\begin{thm}[Borel-Ritt]
  Let $\mathfrak{s}$ be an open sector with a opening less than $2\pi$.
  Then the Taylor map
  \[
    T_{\mathfrak{s}}:\cA(\mathfrak{s})\to\C(\!(t)\!)
  \]
  is onto.
\end{thm}
\begin{proof}
  \begin{comment}
    \begin{itemize}
      \item \cite[Lem.II.2.2.5]{sabbah_cimpa90}: $T_I$
      \item \cite[Th.7.3]{van2003galois}: $T_I=T_{(a,b)}$
      \item \cite[Th.2.4.1]{Loday2014}: $T_{\mathfrak{s}}$
    \end{itemize}
  \end{comment}
  See \cite[Lem.II.2.2.5]{sabbah_cimpa90}, \cite[Th.7.3]{van2003galois} and
  \cite[Th.2.4.1]{Loday2014}.
\end{proof}

\begin{comment}
%%%%%%%%%%%%%%%%%%%%%%%%%%%%%%%%%%%%%%%%%%%%%%%%%%%%%%%%%%%%%%%%%%%%%%%%%%%%%%%
  \section{Sheaf definition from \cite{sabbah2007isomonodromic}}
  See \cite[II.5.c]{sabbah2007isomonodromic}.
  Let
  \begin{itemize}
    \item $D$ the open disc with coordinate $t$ centered at the origin and of
      radius $r^0>0$.
    \item $\tilde D:=[0,r^0[\times S^1$
    \item $\pi:\tilde D\to D,~(r,\theta)\mapsto t=re^{i\theta}$ 
  \end{itemize}
  \dots we can define the derivations $t\partial/\partial t$ and 
  $\bar t\partial/\partial\bar t$.
  \begin{defn}
    The sheaf of rings $\cA_{\tilde D}$ is defined as the subsheaf of
    $\sC_{\tilde D}^\infty$ of germs killed by $\bar t\partial/\partial\bar t$.
  \end{defn}
  \begin{rem}
    \cite{sabbah2007isomonodromic}: Remark 5.11
  \end{rem}
  \begin{rem}
    On the sheaf $\cA_{\tilde D}$ is defined the action of the derivation
    $\partial/\partial t$\dots
  \end{rem}
  \dots
\end{comment}
