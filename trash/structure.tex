\chapter{Structure of Stokes germs and Stokes
  matrices}\label{chp:WhichInformationIsNeeded}
Here we want to discuss, which information is required to describe the Stokes
cocycle corresponding to a multileveled system in more depth.
We will start be looking at a sigle-leveled system determined by a normal form
$A^0\in \GL_3(\C(\!\{t\}\!))$ with exactly $2$ levels. We will apply the
techniques developed in the previous chapter in an rather explicit way. 

\rewrite{Later we will look at a more general multileveled case.}

\section{Simplest, multileveled case}
Let $A^0$ be a normal form with dimension $n=3$ and two levels
$\cK=\{k_1<k_2\}$, which satisfies that there is at least one anti-Stokes
direction $\theta$ which is beared by both levels.
Let $q_j(t^{-1})$ be the determining polynomials and let $k_{jl}$ be the
degrees of $(q_j-q_l)(t^{-1})$.
Since it is the only way to get two levels in dimension $3$ we have, up to
permutation: $k_1=k_{1,2}\geq k_{1,3}$ and $k_2=k_{2,3}$.
We know also that the leading terms of $(q_1-q_2)(t^{-1})$ and
$(q_1-q_3)(t^{-1})$ are equal and thus determine the same anti-Stokes
directions.
The set of all anti-Stokes directions is then given as
\[
  \A=\left\{\theta+\frac{\pi}{k}\cdot j\mid k\in\cK\text{, }j\in\N\right\}
    =:\{
      \underset{\theta}{%
        \underset{\text{\rotatebox[origin=c]{-90}{$=$}}}{%
          \alpha_1
      }}
    ,\dots,\alpha_\nu\}\,.
\]
Denote by $\cY_0(t)$ a normal solution of $[A^0]$.

\subsection{What do the Stokes germs $\phi_\alpha$ look like?}
\rewrite{First} we want to use the Stokes matrices of Stokes germs, discussed
in Section~\ref{sec:matrixReps}, to \rewrite{obtain some statements} about the
\rewrite{structure} of the Stokes germs in our situation.

The Proposition~\ref{prop:representation} states that every Stokes germ
$\phi_\alpha$ can be written as its matrix representation conjugated by the
normal solution. That means that there is a Stokes matrix in
$\SSto_\alpha(A^0)$, which gets via
\begin{align*}
  \rho_{\alpha}^{-1}:
  \SSto_\alpha(A^0) &\overset{\cong}\longrightarrow \Sto_\alpha(A^0)
  \\C_{\phi_\alpha} &\longmapsto \cY_{0}C_{\phi_\alpha}\cY_{0}^{-1} \,,
\end{align*}
mapped to the Stokes germ $\phi_\alpha$.

\paragraph{Using the structure of $\SSto_\alpha(A^0)$}
First look at an example in which we will explain, from which relations on the
determining polynomials which restriction on the form of the Stokes matrices
follow.
\begin{exmp}
  Let $\alpha\in\A$ be an anti-Stokes direction.
  From the definition of $\SSto_\alpha(A^0)$ (cf.\
  Definition~\ref{defn:groupOfFaithfullReps}) we know that, if one has $q_1
  \underset{\alpha,\max}{\prec} q_2$, the Stokes matrix has the form
  \[
    \begin{pmatrix}
      1 & \text{\boldmath$c_1$} & \star
    \\\text{\boldmath$0$} & 1 & \star
    \\\star & \star & 1
    \end{pmatrix}
  \]
  where $c_j\in\C$ and $\star\in\C$.
  \begin{s-rem}
    In our situation, we also know that
    \begin{einr}
      $q_1 \underset{\alpha,\max}{\prec} q_2$
      \Leftrightarrow{}
      $q_1 \underset{\alpha,\max}{\prec} q_3$
      \qquad \rewrite{respectively} \qquad
      $q_2 \underset{\alpha,\max}{\prec} q_1$
      \Leftrightarrow{}
      $q_3 \underset{\alpha,\max}{\prec} q_1$
    \end{einr}
    since the leading terms of $q_1-q_2$ and $q_1-q_3$ are equal.
  \end{s-rem}
  Thus the representation has the \rewrite{form}
  \[
    \begin{pmatrix}
      1 & c_1 & \text{\boldmath$c_2$}
    \\0 & 1 & \star
    \\\text{\boldmath$0$} & \star & 1
    \end{pmatrix}\,.
  \]
  If we also know that neither $q_2 \underset{\alpha,\max}{\prec} q_3$ nor
  $q_3 \underset{\alpha,\max}{\prec} q_2$ it has the \rewrite{form}
  \[
    \begin{pmatrix}
      1 & c_1 & c_2
    \\0 & 1 & \text{\boldmath$0$}
    \\0 & \text{\boldmath$0$} & 1
    \end{pmatrix}\,.
  \]
  We also know that every matrix of this \rewrite{form} is a representation to
  some Stokes germ.
  Thus we have an isomorphism
  \begin{align*}
    \vartheta_\alpha:\C^2 &\longrightarrow \SSto_\alpha(A^0)
  \\(c_1,c_2)&\longmapsto
    \begin{pmatrix}
      1 & c_1 & c_2
    \\0 & 1 & 0
    \\0 & 0 & 1
    \end{pmatrix}
  \end{align*}
\end{exmp}
In fact, the following $9$ cases of Stokes matrices can arise.
\begin{center}
  \def\arraystretch{1.3}
  \setlength\tabcolsep{4mm}
  \begin{tabular}{r|c|c|c}
    & $q_2 \underset{\alpha,\max}{\prec} q_3$
    & $q_3 \underset{\alpha,\max}{\prec} q_2$
    & else
    \tabularnewline
    \hline
    $\substack{q_1 \underset{\alpha,\max}{\prec} q_2\\\text{and}
    \\q_1 \underset{\alpha,\max}{\prec} q_3}$
    & $\begin{pmatrix} 1 & c_2 & c_3 \\0 & 1 & c_1 \\0 & 0 & 1 \end{pmatrix}$
   \cellcolor{blue!15}
    & $\begin{pmatrix} 1 & c_2 & c_3 \\0 & 1 & 0 \\0 & c_1 & 1 \end{pmatrix}$
   \cellcolor{blue!15}
    & $\begin{pmatrix} 1 & c_2 & c_3 \\0 & 1 & 0 \\0 & 0 & 1 \end{pmatrix}$
   \cellcolor{green!15}
    \tabularnewline
    \hline
    $\substack{q_2 \underset{\alpha,\max}{\prec} q_1\\\text{and}
    \\q_3 \underset{\alpha,\max}{\prec} q_1}$
    & $\begin{pmatrix} 1 & 0 & 0 \\c_2+c_1c_3 & 1 & c_1 \\c_3 & 0 & 1 \end{pmatrix}$
   \cellcolor{blue!15}
    & $\begin{pmatrix} 1 & 0 & 0 \\c_2 & 1 & 0 \\c_1c_2+c_3 & c_1 & 1 \end{pmatrix}$
   \cellcolor{blue!15}
    & $\begin{pmatrix} 1 & 0 & 0 \\c_2 & 1 & 0 \\c_3 & 0 & 1 \end{pmatrix}$
   \cellcolor{green!15}
    \tabularnewline
    \hline
    else
    & $\begin{pmatrix} 1 & 0 & 0 \\0 & 1 & c_1 \\0 & 0 & 1 \end{pmatrix}$
   \cellcolor{purple!15}
    & $\begin{pmatrix} 1 & 0 & 0 \\0 & 1 & 0 \\0 & c_1 & 1 \end{pmatrix}$
   \cellcolor{purple!15}
    & $\begin{pmatrix} 1 & 0 & 0 \\0 & 1 & 0 \\0 & 0 & 1 \end{pmatrix}$
  \end{tabular}
\end{center}
In the \textcolor{blue!75!black}{blue} cases we have $\cK_\alpha=\cK$ and
$\C^3\overset{\vartheta_\alpha}{\underset{\cong}{\longrightarrow}}\SSto_\alpha(A^0)$.
In the \textcolor{green!50!black}{green} cases $\cK_\alpha=\{k_2\}$ and
$\C^2\overset{\vartheta_\alpha}{\underset{\cong}{\longrightarrow}}\SSto_\alpha(A^0)$
as well as in the \textcolor{purple!75!black}{purple} cases
$\cK_\alpha=\{k_1\}$ and
$\C^1\overset{\vartheta_\alpha}{\underset{\cong}{\longrightarrow}}\SSto_\alpha(A^0)$.
Thus, for every $\alpha\in\A$, we have an isomorphism
$\rho_{\alpha}^{-1}\circ\vartheta_\alpha$.
Two of the cases are more complicated than necessary, to be consistent with
the decomposition in the next part (cf.\ Example~\ref{exmp:decompositionHere}).

\paragraph{Decomposition by levels}
In proposition~\ref{prop:filtrationOfStokesGroup} and especially
remark~\ref{rem:filtrationOfStokesMats} we have defined a decomposition of the
Stokes group $\Sto_\alpha(A^0)$ in subgroups generated by $k$-germs for
$k\in\cK$.
In our case, we have at most two level, such that this decomposition is given
by
\[
  \phi_\alpha=\phi_\alpha^{k_1} \phi_\alpha^{k_2}
  \overset{i_\alpha}\longmapsto
    \left(\phi_\alpha^{k_1},\phi_\alpha^{k_2}\right)
      \in\Sto_\alpha^{k_1}(A^0)\times\Sto_\alpha^{k_2}(A^0) \,,
\]
where $\phi_\alpha^{k_1}\in\Sto_\alpha^{k_1}(A^0)=\Sto_\alpha^{<k_2}(A^0)$,
$\phi_\alpha^{k_2}\in\Sto_\alpha^{k_2}(A^0)$  and $i_\alpha$ is the map, wich
gives the factors of this factorization in ascending order.

This decomposition of a germ $\phi_\alpha$ is trivial if
$\#\cK(\phi_\alpha)\leq1$, thus the interesting cases are the
\textcolor{blue!75!black}{blue} cases.

\begin{exmp}\label{exmp:decompositionHere}
  Look at the example
  \[
    \vartheta_\alpha(c_1,c_2,c_3)=
    \cY_{0}
    \begin{pmatrix} 1 & 0 & 0 \\c_2 & 1 & 0 \\c_1c_2+c_3 & c_1 & 1 \end{pmatrix}
    \cY_{0}^{-1}
    =\phi_\alpha
    \,.
  \]
  According to remark~\ref{rem:algFactorization} the factor
  $\phi_\alpha^{k_1}$, generated by the $k_1$-germs, is given by
  \[
    \phi_\alpha^{k_1}=
    \cY_{0}
    \begin{pmatrix}
      1 & 0 & 0
    \\\text{\boldmath $0$} & 1 & 0
    \\\text{\boldmath $0$} & c_1 & 1
    \end{pmatrix}
    \cY_{0}^{-1}
    \,.
  \]
  The other factor $\phi_\alpha^{k_2}$ is then obtained as
  \begin{align*}
    \phi_\alpha^{k_2}&=
    \left(\phi_\alpha^{k_1}\right)^{-1}
    \phi_\alpha^{k_2}
  \\&=\cY_{0}
    \begin{pmatrix}
      1     & 0    & 0
    \\0     & 1    & 0
    \\0     & -c_1 & 1
    \end{pmatrix}
    \underset{=\id}{\underbrace{%
        \cY_{0}^{-1}
        \cY_{0}
    }}
    \begin{pmatrix} 1 & 0 & 0 \\c_2 & 1 & 0 \\c_1c_2+c_3 & c_1 & 1 \end{pmatrix}
    \cY_{0}^{-1}
  \\&=\cY_{0}
    \begin{pmatrix}
      1     & 0 & 0
    \\c_2     & 1          & 0
    \\c_3     & 0          & 1
    \end{pmatrix}
    \cY_{0}^{-1}
    \,.
  \end{align*}
\end{exmp}
The four nontrivial decomposition in our situation, are given by
\begin{enumerate}
  \item $\begin{pmatrix} 1 & 0 & 0 \\0 & 1 & c_1 \\0 & 0 & 1 \end{pmatrix}
  \cdot\begin{pmatrix} 1 & c_2 & c_3 \\0 & 1 & 0 \\0 & 0 & 1 \end{pmatrix}=
  \begin{pmatrix} 1 & c_2 & c_3 \\0 & 1 & c_1 \\0 & 0 & 1 \end{pmatrix}$
  \item $\begin{pmatrix} 1 & 0 & 0 \\0 & 1 & 0 \\0 & c_1 & 1 \end{pmatrix}
  \cdot\begin{pmatrix} 1 & c_2 & c_3 \\0 & 1 & 0 \\0 & 0 & 1 \end{pmatrix}=
  \begin{pmatrix} 1 & c_2 & c_3 \\0 & 1 & 0 \\0 & c_1 & 1 \end{pmatrix}$
  \item $\begin{pmatrix} 1 & 0 & 0 \\0 & 1 & c_1 \\0 & 0 & 1 \end{pmatrix}
  \cdot\begin{pmatrix} 1 & 0 & 0 \\c_2 & 1 & 0 \\c_3 & 0 & 1 \end{pmatrix}=
  \begin{pmatrix} 1 & 0 & 0 \\c_2+c_1c_3 & 1 & c_1 \\c_3 & 0 & 1 \end{pmatrix}$
  \item $\begin{pmatrix} 1 & 0 & 0 \\0 & 1 & 0 \\0 & c_1 & 1 \end{pmatrix}
  \cdot\begin{pmatrix} 1 & 0 & 0 \\c_2 & 1 & 0 \\c_3 & 0 & 1 \end{pmatrix}=
  \begin{pmatrix} 1 & 0 & 0 \\c_2 & 1 & 0 \\c_1c_2+c_3 & c_1 & 1 \end{pmatrix}$
\end{enumerate}
We will use this decompositions to write the isomorphisms
$\vartheta_\alpha:\C^\star\to \SSto_\alpha(A^0)$ as
\[ \begin{tikzcd}
  \C^\star
  \rar{j_\alpha}
  \arrow[rr,out=-30,in=-150,"\vartheta_\alpha"]
  &
  \SSto_\alpha^{k_1}(A^0)\times\SSto_\alpha^{k_2}(A^0)
  \rar{i_\alpha^{-1}}
  &
  \SSto_\alpha(A^0)
\end{tikzcd} \]
where $\star\in\{1,2,3\}$.

\subsection{What do Stokes cocycles look like?}
\begin{prop}\label{prop:decompositionDiagram}
  By taking the product over $\A$ of all
  $\rho_{\alpha}^{-1}\circ\vartheta_\alpha$ we obtain the isomorphism
  \[
    \prod_{\alpha\in\A}\rho_{\alpha}^{-1}\circ\vartheta_\alpha:
    \C^{k_1+2\cdot k_2}
    \overset\cong\longrightarrow \prod_{\alpha\in\A}\SSto_\alpha(A^0)
    \overset\cong\longrightarrow \prod_{\alpha\in\A}\Sto_\alpha(A^0)
  \]
  which can be concatenated with $h$ to obtain an element in $\St(A^0)$.
  We can also define the isomorphism
  $\C^{2\cdot(k_1+2\cdot k_2)}\textcolor{purple}{\longrightarrow} \St(A^0)$
  which makes the following diagram commute.
  \[ \begin{tikzcd}[column sep=1.5cm,row sep=2cm]
      &\C^{2\cdot(k_1+2\cdot k_2)}
      \dlar[blue]{\prod_{\alpha\in\A}\rho_{\alpha}^{-1}\circ\vartheta_\alpha}
      \arrow[dr,purple]
    \\\prod_{\alpha\in\A}\Sto_\alpha(A^0)
      \arrow[rr,green!50!black,"h"]
      &&\St(A^0)
  \end{tikzcd} \]
  Using the knowledge, how the morphism were build and the fact that some of
  the morphism are equal, we can rewrite this diagram as follows
  \[ \begin{tikzcd}
      &\C^{2\cdot(k_1+2\cdot k_2)}
      \dar[xshift=-1pt,blue]
      \dar[xshift=1pt,purple,"\prod_{\alpha\in\A}j_\alpha"]
    \\
      &\prod_{k\in\{k_1,k_2\}}\prod_{\alpha\in\A}\SSto_{\alpha}^{k}(A^0)
       \arrow[dl,blue,"\prod_{\alpha\in\A}i_\alpha^{-1}"]
       \arrow[ddr,purple,
         "\prod_{k\in\{k_1,k_2\}}\prod_{\alpha\in\A}(\rho_{\alpha}^k)^{-1}"]
    \\ \prod_{\alpha\in\A} \SSto_{\alpha}(A^0)
        \dar[blue]{\prod_{\alpha\in\A}\rho_{\alpha}^{-1}}
    \\\prod_{\alpha\in\A} \Sto_{\alpha}(A^0)
      \arrow[rr,green!50!black,"\prod_{\alpha\in\A}i_\alpha"]
      &&\prod_{k\in\{k_1,k_2\}}\prod_{\alpha\in\A}\Sto_{\alpha}^{k}(A^0)
      \dar[xshift=-1pt,green!50!black]
      \dar[xshift=1pt,purple]{\prod_{k\in\{k_1,k_2\}}i^k}
    \\
      && \prod_{k\in\{k_1,k_2\}}\Gamma(\dot\cU^k;\Lambda^k(A^0))
      \dar[xshift=-1pt,green!50!black]
      \dar[xshift=1pt,purple]{\cT}
    \\&& \St(A^0)
  \end{tikzcd} \]
\end{prop}

\begin{comment}
  Decompose $\C^{k_1+2\cdot k_2}\cong
  \prod_{\alpha\in\A^{k_1}} \C \times \prod_{\alpha\in\A^{k_2}} \C^2$ and
  define the isomorphism
  \[
    \C^{k_1+2\cdot k_2}\overset{\cong}\longrightarrow
    \prod_{\alpha\in\A^{k_1}} \SSto_{\alpha}^{k_1}(A^0) \times
    \prod_{\alpha\in\A^{k_2}} \SSto_{\alpha}^{k_2}(A^0)
  \]
  levelwise as
  \begin{enumerate}
    \item an element
      $\left( (a_{\alpha_{\nu_1}},b_{\alpha_{\nu_1}})
        ,(a_{\alpha_{\nu_2}},b_{\alpha_{\nu_2}})
        ,\dots
      \right)\in\prod_{\alpha\in\A^{k_2}} \C^2$
      gets mapped to
      \[
        \left.
        \begin{cases}
          \left(
          \begin{pmatrix} 1 & a_{\alpha_{\nu_1}} & b_{\alpha_{\nu_1}} \\0 & 1 & 0 \\0 & 0 & 1 \end{pmatrix}
          ,\begin{pmatrix} 1 & 0 & 0 \\a_{\alpha_{\nu_2}} & 1 & 0 \\b_{\alpha_{\nu_2}} & 0 & 1 \end{pmatrix}
          ,\begin{pmatrix} 1 & a_{\alpha_{\nu_3}} & b_{\alpha_{\nu_3}} \\0 & 1 & 0 \\0 & 0 & 1 \end{pmatrix}
            ,\dots
          \right)
          & ,\substack{\text{~if~} q_1 \underset{\theta,\max}{\prec} q_2
            \\\text{~and~} q_1 \underset{\theta,\max}{\prec} q_3}
          \\\left(
          \begin{pmatrix} 1 & 0 & 0 \\a_{\alpha_{\nu_1}} & 1 & 0 \\b_{\alpha_{\nu_1}} & 0 & 1 \end{pmatrix}
          ,\begin{pmatrix} 1 & a_{\alpha_{\nu_2}} & b_{\alpha_{\nu_2}} \\0 & 1 & 0 \\0 & 0 & 1 \end{pmatrix}
          ,\begin{pmatrix} 1 & 0 & 0 \\a_{\alpha_{\nu_3}} & 1 & 0 \\b_{\alpha_{\nu_3}} & 0 & 1 \end{pmatrix}
            ,\dots
          \right)
          & ,\substack{\text{~if~} q_2 \underset{\theta,\max}{\prec} q_1
            \\\text{~and~} q_3 \underset{\theta,\max}{\prec} q_1}
        \end{cases}
        \right\}
        =:\left(
          C_{\alpha_{\nu_1}}^{k_2}
          ,C_{\alpha_{\nu_2}}^{k_2}
          ,\dots
        \right)
      \]
    \item
      $\left(
        c_{\alpha_{\mu_1}}
        ,c_{\alpha_{\mu_2}}
        ,\dots
      \right)\in\prod_{\alpha\in\A^{k_1}} \C$
      gets mapped to
      \[
        \left.
        \begin{cases}
          \left(
          \begin{pmatrix} 1 & 0 & 0 \\0 & 1 & c_{\alpha_{\mu_1}} \\0 & 0 & 1 \end{pmatrix}
          ,\begin{pmatrix} 1 & 0 & 0 \\0 & 1 & 0 \\0 & c_{\alpha_{\mu_2}} & 1 \end{pmatrix}
          ,\begin{pmatrix} 1 & 0 & 0 \\0 & 1 & c_{\alpha_{\mu_3}} \\0 & 0 & 1 \end{pmatrix}
            ,\dots
          \right)
          &\text{,~if~} q_2 \underset{\theta,\max}{\prec} q_3
          \\\left(
          \begin{pmatrix} 1 & 0 & 0 \\0 & 1 & 0 \\0 & c_{\alpha_{\mu_1}} & 1 \end{pmatrix}
          ,\begin{pmatrix} 1 & 0 & 0 \\0 & 1 & c_{\alpha_{\mu_2}} \\0 & 0 & 1 \end{pmatrix}
          ,\begin{pmatrix} 1 & 0 & 0 \\0 & 1 & 0 \\0 & c_{\alpha_{\mu_2}} & 1 \end{pmatrix}
            ,\dots
          \right)
          &\text{,~if~} q_3 \underset{\theta,\max}{\prec} q_2
        \end{cases}
      \right\}
      =:\left(
        C_{\alpha_{\mu_1}}^{k_1}
        ,C_{\alpha_{\mu_2}}^{k_1}
        ,\dots
      \right)
      \]
  \end{enumerate}
\end{comment}

%%%%%%%%%%%%%%%%%%%%%%%%%%%%%%%%%%%%%%%%%%%%%%%%%%%%%%%%%%%%%%%%%%%%%%%%%%%%%%%
\subsubsection{Explicit example}
\def\kOne{1}
\def\kTwo{3}
\def\zkOnepzKtwo{14} % 2\cdot(\kOne+2\cdot\kTwo
\def\zkOne{2} % 2*\kOne
\def\zkTwo{6} % 2*\kTwo

Even more explicit, we can fix the levels $k_1=\kOne$ and $k_2=\kTwo$ together
with $\theta=0$.
Assume without any restriction that $q_1 \underset{\theta,\max}{\prec} q_2$
and $q_1 \underset{\theta,\max}{\prec} q_3$ as well as
$q_2 \underset{\theta,\max}{\prec} q_3$.

The classification space is in this case isomorphic to
$\C^{2\cdot(\kOne+2\cdot\kTwo)}=\C^{\zkOnepzKtwo}$.
The element
\[
  ({}^1c_1,{}^2c_1,
  {}^1c_2,{}^1c_3,{}^2c_2,{}^2c_3,\dots,{}^{\zkTwo}c_2,{}^{\zkTwo}c_3)
  \in\C^{\zkOnepzKtwo}
\]
gets, via the morphism $\prod_{\alpha\in\A}j_\alpha$, mapped to
\begin{align*}
  &\left(
  \left(
    \begin{pmatrix} 1 & 0 & 0 \\0 & 1 & {}^1c_1 \\0 & 0 & 1 \end{pmatrix},
    \begin{pmatrix} 1 & 0 & 0 \\0 & 1 & 0 \\0 & {}^2c_1 & 1 \end{pmatrix}
  \right),
  \right.
\\&\qquad\left(
  \left.
    \begin{pmatrix} 1 & {}^1c_2 & {}^1c_3 \\0 & 1 & 0 \\0 & 0 & 1 \end{pmatrix},
    \begin{pmatrix} 1 & 0 & 0 \\{}^2c_2 & 1 & 0 \\{}^2c_3 & 0 & 1 \end{pmatrix},
    \dots,
    \begin{pmatrix} 1 & 0 & 0 \\{}^{\zkTwo}c_2 & 1 & 0 \\{}^{\zkTwo}c_3 & 0 & 1 \end{pmatrix}
  \right)
  \right)
\end{align*}
in $\prod_{\alpha\in\A^{\kOne}}\SSto_{\alpha}^{\kOne}(A^0) \times
\prod_{\alpha\in\A^{\kTwo}}\SSto_{\alpha}^{\kTwo}(A^0)$ and thus the element
\begin{align*}
  &\left(
  \left(
    \begin{pmatrix} 1 & 0 & 0 \\0 & 1 & {}^1c_1 \\0 & 0 & 1 \end{pmatrix},
    \id,\id,
    \begin{pmatrix} 1 & 0 & 0 \\0 & 1 & 0 \\0 & {}^2c_1 & 1 \end{pmatrix},
    \id,\id
  \right),
  \right.
\\&\qquad\left(
  \left.
    \begin{pmatrix} 1 & {}^1c_2 & {}^1c_3 \\0 & 1 & 0 \\0 & 0 & 1 \end{pmatrix},
    \begin{pmatrix} 1 & 0 & 0 \\{}^2c_2 & 1 & 0 \\{}^2c_3 & 0 & 1 \end{pmatrix},
    \dots,
    \begin{pmatrix} 1 & 0 & 0 \\{}^{\zkTwo}c_2 & 1 & 0 \\{}^{\zkTwo}c_3 & 0 & 1 \end{pmatrix}
  \right)
  \right)
\end{align*}
in
$\prod_{\alpha\in\A}\SSto_{\alpha}^{\kOne}(A^0) \times
\prod_{\alpha\in\A}\SSto_{\alpha}^{\kTwo}(A^0)$.
Using the morphism $\prod_{\alpha\in\A}i_\alpha^{-1}$ we get a complete set of
Stokes matrices as
\begin{align*}
  &\left(
    \begin{pmatrix} 1 & {}^1c_2 & {}^1c_3 \\0 & 1 & {}^1c_1 \\0 & 0 & 1 \end{pmatrix},
    \begin{pmatrix} 1 & 0 & 0 \\{}^2c_2 & 1 & 0 \\{}^2c_3 & 0 & 1 \end{pmatrix},
    \begin{pmatrix} 1 & {}^3c_2 & {}^3c_3 \\0 & 1 & 0 \\0 & 0 & 1 \end{pmatrix},
  \right.
\\&\qquad
  \left.
    \begin{pmatrix} 1 & 0 & 0 \\{}^4c_2 & 1 & 0 \\{}^2c_1{}^4c_2+{}^4c_3 & {}^2c_1 & 1 \end{pmatrix},
    \begin{pmatrix} 1 & {}^5c_2 & {}^5c_3 \\0 & 1 & 0 \\0 & 0 & 1 \end{pmatrix},
    \begin{pmatrix} 1 & 0 & 0 \\{}^{\zkTwo}c_2 & 1 & 0 \\{}^{\zkTwo}c_3 & 0 & 1 \end{pmatrix}
  \right)
  \in
  \prod_{\alpha\in\A}\SSto_{\alpha}(A^0)
\end{align*}

\section{In the more general multileveled context}
Let us assume, that our normal form $[A^0]$ of dimension $n$ has the levels
$\cK$.

\begin{cor}
  We now can improve the diagram of isomorphisms from
  Proposition~\ref{prop:decompositionDiagram} by writing it in a more general
  form and adding the found isomorphism.
  Let $\theta$ be a direction, which satisfies
  $\theta\pm\frac{\pi}{2k}\notin\A^k$ for every $k\in\cK$.
  \[ \begin{tikzcd}
      & \C^{\sum_{q_j\underset{\alpha,\max}{\prec}q_l}\deg(q_j-q_l)}
      \dar["\prod_{\alpha\in\A}j_\alpha"]
    \\
      &\displaystyle\prod_{k\in\cK}\prod_{\alpha\in\A}\SSto_{\alpha}^{k}(A^0)
       \arrow[dl,"\prod_{\alpha\in\A}i_\alpha^{-1}"]
       \arrow[ddr,
         "\prod_{k\in\cK}\prod_{\alpha\in\A}(\rho_{\alpha}^k)^{-1}"]
       \arrow[r]
      &\displaystyle\prod_{k\in\cK} \prod_{j\in\{1,\dots,2k\}}
       \hat\SSto_{\theta+j\frac{\pi}{k}}^k(A^0)
    \\ \displaystyle\prod_{\alpha\in\A} \SSto_{\alpha}(A^0)
        \dar{\prod_{\alpha\in\A}\rho_{\alpha}^{-1}}
    \\\displaystyle\prod_{\alpha\in\A} \Sto_{\alpha}(A^0)
      \arrow[rr,"\prod_{\alpha\in\A}i_\alpha"]
      &&\displaystyle\prod_{k\in\cK}\prod_{\alpha\in\A}\Sto_{\alpha}^{k}(A^0)
      \dar{\prod_{k\in\cK}i^k}
    \\
      && \displaystyle\prod_{k\in\cK}\Gamma(\dot\cU^k;\Lambda^k(A^0))
      \dar{\cT}
    \\&& \St(A^0)
      \arrow[purple,uuuu,out=30,in=-60,thick]
  \end{tikzcd} \]
  \[
    \St(A^0) \textcolor{purple}{\longrightarrow}
    \prod_{k\in\cK} \prod_{j\in\{1,\dots,2k\}}
    \hat\SSto_{\theta+j\frac{\pi}{k}}^k(A^0)
  \]

  \begin{s-rem}
    The contained isomorphism
    \[
      \prod_{\alpha\in\A}\SSto_{\alpha}(A^0) \longrightarrow
      \prod_{k\in\cK} \prod_{j\in\{1,\dots,2k\}}
      \hat\SSto_{\theta+j\frac{\pi}{k}}^k(A^0)
    \]
    is in the single-leveled case with $n=s$, i.e.\ all diagonal elements of
    $Q$ are different, given in Lemma 3.2 of Boalch's paper
    \cite[Lem.3.2]{boalch}.
    In this case is every
    $\hat\SSto_{\theta+j\frac{\pi}{k}}^k(A^0)$ isomorphic to some
    $PU_+P^{-1}$ where $U_+$ is the group of all upper triangular matrices,
    with ones on the diagonal, and $P$ is a permutation matrix.
  \end{s-rem}
\end{cor}
