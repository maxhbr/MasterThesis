\chapter{Symplektische Geometrie}
\begin{comment}
Siehe
\begin{itemize}
\item Eilenberg.pdf
\end{itemize}
TODO:
\begin{itemize}
\item symplektische Struktur
\item hamiltonsche Wirkung / Abbildung \cite[p161ff]{mcduff1998introduction}
\item momenten Abbildung \cite[p164ff]{mcduff1998introduction}
\item symplektische quotienten
\begin{itemize}
  \item \cite[Section 5.4]{mcduff1998introduction}
  \item Part IX aus 
    \textbf{[Ana Cannas da Silva] Lectures on Symplectic Geometry}
  \item \cite[chapter 1]{citeulike:4402830}
\end{itemize}
\end{itemize}
\textbf{Es muss Lie (teilweise) vor Symplektische Geometrie!}
\end{comment}

\ccite[Section 3.1]{mcduff1998introduction}
Eine \emph{Symplektische Struktur} auf einer glatten Mannigfaltigkeit $M$ ist
eine nichtdegenerierte $2$-Form $\omega\in\Omega^2(M)$.
Nichtdegeneriertheit bedeutet dass jeder Tangentialraum $(T_q,M,\omega_q)$ ein
symplektischer Vektorraum ist.
Die Mannigfaltigkeit $M$ ist nötigerweise gerade dimensional.
\begin{comment}
Nach \cite[Corollary 2.5]{mcduff1998introduction} gilt, dass das $n$-fache
Wedge Produkt $\omega \wedge\dots\wedge\omega$ nicht verschwindet. Damit ist
$M$ orientierbar.
\end{comment}
\section{Hamiltonsche Wirkung / Abbildung}
\ccite[p161ff]{mcduff1998introduction}
\section{Momenten Abbildung}
\ccite[p164ff]{mcduff1998introduction}
\begin{comment}
The concept of a moment map is a generalization of that of a hamiltonian
function. The notion of a moment map associated to a group action on a
symplectic manifold formalizes the Noether principle, which states that to
every symmetry (such as a group action) in a mechanical system, there
corresponds a conserved quantity.
\\
Aus p127 
\textbf{[Ana Cannas da Silva] Lectures on Symplectic Geometry}
\end{comment}

%
% vim:set ft=tex foldmethod=marker foldmarker={{{,}}}:
