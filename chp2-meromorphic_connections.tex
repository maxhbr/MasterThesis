\chapter{Systems and meromorphic connections}
Here we will start with defining (global) meromorphic connetions although we
will only be interested in local information.
For local description we will use \textbf{germs of meromorphic connections}, the
coordinate dependent \textbf{systems} and \textbf{connection matrices}.
Other coordinate independent approaches arise for example from the theory of
(localized holonomic) $\cD$-modules.

Meromorphic connections are introduced and discussed in many resources.
A good starting point are Sabbah's lecture notes~\cite{sabbah_cimpa90}.
More advanced resources are for example
Sabbah's book~\cite{sabbah2007isomonodromic},
Varadarajan's book~\cite{Varadarajan96linearmeromorphic} or the
book~\cite{hotta2008} from Hotta et al.
The necessary facts about meromorphic connections are also stated in
Boalch's paper~\cite{boalch} (resp.\ his thesis~\cite{thboalch}),
and Loday-Richaud's paper~\cite{Loday1994}.

Although the language of meromorphic connections is often preferred, we will
use the language of systems most of the time.
Systems are, for example, discussed in the book~\cite{hotta2008} from Hotta et
al, Loday-Richaud's paper~\cite{Loday1994} and her book~\cite{Loday2014} and
Boalch's publications~\cite{boalch,thboalch}. Another resource \rewrite{might}
Remy's paper~\cite{Remy2014} be.

We will use \rewrite{all of the} above mentioned \rewrite{resources} in this
chapter.

\section{(Global) meromorphic connections}
Let $M$ be a riemanian surface and let $Z=k_1(a_1)+\cdots+k_m(a_m)>0$ be an
effective divisor\footnote{The $a_i\in M$ are distinct points and the $k_i$ are
positive integers.} on $M$.
It is sufficient to think $M=\P^1$ and $0\in|Z|$\footnote{If
$Z=k_1(a_1)+\cdots+k_m(a_m)$ then $|Z|:=\{a_1,\dots,a_m\}$.}, since we will
only be interested in local information (at $0$).

Let $\sM$ be a holomorphic Bundle over $M$ i.e.\ a locally free $\cO_M$-module
of rank $n$.
A (global) meromorphic connection is then defined as follows.
\begin{defn}\label{defn:mercon}
  A \emph{meromorphic connection $(\sM,\nabla)$ on $\sM$ with poles on $Z$}
  is defined by a $\C$-linear morphism of sheaves
  \[
    \nabla:\sM\to\Omega_M^1(*Z)\otimes\sM
  \]
  satisfying, for each $U\underset{\text{op.}}{\subset} M$, the
  \emph{Leibniz rule}
    \[
      \nabla(fs)=f\nabla s+(df)\otimes s
    \]
  for $s\in\Gamma(U,\sM)$ and $f\in\cO_M(U)$.
  The \emph{rank} of the meromorphic connection $(\sM,\nabla)$ is defined to be
  the rank of the Bundle $\sM$.
  \begin{s-rem}
    Some authors use the factors $k_i$ of the divisor $Z$ to limit the pole
    orders at the points $a_i$. Since we do not need this restriction, we allow
    arbitrary pole orders. Denoted is this by the $*$ in $\Omega_M^1(*Z)$.
    The sheaf $\cO_M(*Z)$ of functions, which are meromorphic along $Z$, is
    defined in Sabbah's book~\cite[Sec.0.8]{sabbah2007isomonodromic} and
    \[
      \Omega_M^1(*Z):=\cO_M(*Z)\underset{\cO_M}\otimes\Omega_M^1
    \]
    is the \emph{sheaf of meromorphic differential $1$-forms}
    (cf.\ \cite[Sec.0.9.b]{sabbah2007isomonodromic}).
  \end{s-rem}
\end{defn}
The word ``global'' used to emphasize that the connection is ``on $M$'', i.e.\
not only a germ at some point of $M$, \rewrite{like it will be used} later
(cf.\ Remark~\ref{rem:GlobalNothingGerm}).
It will occasionally be convenient to omit the $\nabla$ and simply call $\sM$
the meromorphic connection.
\begin{rem}
  Here, the variant `holomorphic bundle with meromorphic connection' is chosen,
  like in Boalch's paper~\cite{boalch}. There is also the transposed description
  `meromorphic bundle with holomorphic connection' which is for example used in
  Sabbah's book~\cite{sabbah2007isomonodromic}.

  By choosing a lattice of a meromorphic bundle
  (cf.\ \cite[Def.0.8.3]{sabbah2007isomonodromic}), one gets a holomorphic bundle
  but if the meromorphic bundle had a holomorphic connection, the induced
  connection \rewrite{on the lattice} is no longer guaranteed to be holomorphic.
  Thus we obtain a meromorphic connection on a holomorphic bundle in our sense.
\end{rem}

\begin{defn}
  \marginnote{\cite[0.12.2]{sabbah2007isomonodromic}}
  The connection $\nabla:\cM\to \Omega_M^1\otimes_{\cO_M}\cM$ is said to be
  \emph{integrable} or \emph{flat}, if its curvature vanishes, i.e.\
  $R_\nabla\equiv0$
  where $R_\nabla:=\nabla\circ\nabla:\cE\to\Omega_M^2\otimes_{\cO_M}\cE$ is a
  $\cO_M$-linear morphism.
  \begin{s-rem}
    Here are all connections flat, since all connections will be of Dimension
    one.
  \end{s-rem}
\end{defn}

%%%%%%%%%%%%%%%%%%%%%%%%%%%%%%%%%%%%%%%%%%%%%%%%%%%%%%%%%%%%%%%%%%%%%%%%%%%%%%%
\section{Local expression of meromorphic connections and systems}
\marginnote{\cite[28]{sabbah2007isomonodromic},
  \\\cite[2]{thboalch} and
  \\\cite[11]{babbitt1989local}}
We will only be interested in local information of meromorphic connections.
This means that we look at a connection in a neighbourhood of $0$ which has its
unique singularity at $0$.
There are multiple ways of expressing the local information without the need of
an fixed neigbourhood.
We will either talk about germs of meromorphic connections or systems, which
depend on the choice of a trivialization.
\begin{prop}
  \marginnote{\textbf{\cite[Rem.5.2.4]{hotta2008}}\\\cite[Def.4.2.1]{Loday2014}}
  A germ of a meromorphic connection $(\sM,\nabla)$ is the sheaf-theoretic
  germ (at $t=0$) and thus is given by a tuple $(\cM,\nabla)$ where
  \begin{itemize}
    \item $\cM$ is the germ at $0$ of the holomorphic bundle $\sM$ and thus a
      $\C(\!\{t\}\!)$-vectorspace of dimension $n$, since the ring of germs of
      meromorphic functions with poles at $0$ is the ring $\C(\!\{t\}\!)$, and
    \item $\nabla:\cM\to \cM$ is a additive map, which
      satisfies the \emph{Leibniz rule}
      \[
        \nabla(fm)=\frac{d}{dt} f\cdot m + f\nabla(m)
      \]
      for all $f\in\C(\!\{t\}\!)$ and $m\in \cM$.
  \end{itemize}
  \begin{s-rem}
    Loday-Richaud calls the meromorphic connections in her
    book~\cite[Def.4.2.1]{Loday2014} \emph{differential modules}.
  \end{s-rem}
\end{prop}
\begin{rem}\label{rem:GlobalNothingGerm}
  From now on we will mostly talk about \textbf{germs of} meromorphic
  connections $(\cM,\nabla)$ and we will call them meromorphic connection. If
  we want to talk about meromorphic connections in the sense of
  Definition~\ref{defn:mercon} we will emphasize this by the word `global' or
  by talking about a meromorphic connection \textbf{on $M$}.
\end{rem}
\begin{defn}
  \marginnote{\cite[Def.5.2.1]{hotta2008}}
  A \emph{(iso-)morphism of meromorphic connections}
  $\Phi:(\cM,\nabla)\overset{\sim}\to(\cM',\nabla')$ is a (iso-)morphism of
  $\C(\!\{t\}\!)$-vectorspaces $\Phi:\cM\overset{\sim}\to\cM'$ which commutes
  with the connections, i.e.\ which satisfies
  $\nabla\circ\phi = (\id\otimes\phi)\circ\nabla$.
\end{defn}
\marginnote{\cite[65]{Loday2014}, \cite[129]{hotta2008}}
Choose a $\C(\!\{t\}\!)$-basis $\underline{e}=(e_1,e_2,\dots,e_n)$ of $\cM$.
Let $A=(a_{jk})_{j,k\in\{1,\dots,n\}}$ be a $n\times n$ matrix with entries in
$\C(\!\{t\}\!)$ such that the it describes the action of $\nabla$ corresponding
to the chosen basis, i.e.\ satisfies
 $\nabla e_k=-\sum_{1\leq j\leq n} a_{jk}(t)e_j$ for every $k\in\{1,\dots,n\}$.
\begin{defn}
  The matrix $A$ is called a \emph{connection matrix} of $(\cM,\nabla)$.
\end{defn}
The connection $\nabla$ is fully determined by $A$. Indeed, let
$x=\sum_{0\leq j\leq n}x_je_j=\underline{e}\cdot X$ be an arbitrary element of
$\cM$, where $X={}^t\!(x_1,x_2 ,\dots,x_n)$ is a column matrix.
Then, applying $\nabla$ using the Leibniz rule, yields
\begin{align*}
  \nabla x&=\nabla\left(\underline{e}\cdot X\right)
  \\&=\underline{e} \cdot dX + \nabla \underline{e} \cdot X
  \\&=\underline{e}\left(dX-AX\right).
\end{align*}
Such that horizontal sections of $(\cM,\nabla)$, i.e.\ sections which satisfy
$\nabla x=0$, correspond to solutions of
\begin{equation}\label{eq:ode}
  \frac{d}{dt}x=Ax \,.
\end{equation}
Thus, with the connection $\nabla$ and the $\C(\!\{t\}\!)$-basis
$\underline{e}$ is naturally associate the differential operator
$\triangle=d-A$, which has order one and dimension $n$.
\begin{defn}
  \marginnote{\cite[Sec.4]{Martinet1991}}
  We call (\ref{eq:ode}), determined by the differential operator
  $\triangle=d-A$, a \emph{germ of a meromorphic linear differential
  system\footnotemark} of rank $n$, or just a
  \emph{system}.
  \begin{s-prop}
    Thus, the set of systems is isomorphic to the set
    \[
      \End(E)\otimes\C(\!\{t\}\!)=\gl_n(\C(\!\{t\}\!))
    \]
    of all connection matrices.
  \end{s-prop}
  Such a system will be denoted by $[A]=d-A$ and we will call $A$ the
  connection matrix of the system $[A]$.
\end{defn}
  \footnotetext{Martinet and Ramis call them in~\cite{Martinet1991}
  \emph{germs of meromorphic differential operators.}}
\TODO[possibly multivalued solutions ($\tilde K$)?~\cite{hotta2008} on page
128]

If we start with a system $[A]$ and we want a meromorphic connection
$(\cM,\nabla)$ which has $A$ as connection matrix we can do this in the
following way.
\begin{prop}\label{prop:systToMeromConn}
  If we start with either a system $[A]$ of rank $n$, or a connection matrix
  $A\in\gl_n(\C(\!\{t\}\!))$, we get a germ of a meromorphic connection via
  \[
    (\cM_A,\nabla_A)=(\C(\!\{t\}\!)^n,d-A)
  \]
  which has $A$ as its connection matrix.
\end{prop}

\begin{prop}\label{prop:MatOfSumOfMerCon}
  Let $(\cM_1,\nabla_1)$ and $(\cM_2,\nabla_2)$ be two meromorphic
  connections
  with the connection matrices $A_1$ and $A_2$.
  A connection matrix of $(\cM_1,\nabla_1)\oplus(\cM_1,\nabla_1)$ is then
  given
  by the block-diagonal matrix $\diag(A_1,A_2)$.
\end{prop}
\begin{proof}
  Use Proposition~\ref{prop:systToMeromConn} to write the connection as
  \[
    (\C(\!\{t\}\!)^{n_1},d-A_1)\oplus(\C(\!\{t\}\!)^{n_2},d-A_2)
  \]
  and we want to show that it is isomorphic to
  \[
    \left(\C(\!\{t\}\!)^{n_1+n_2},d-
    \begin{pmatrix} A_1 & 0 \\0 & A_2 \end{pmatrix}\right) \,.
  \]
  Denote by $A:=\diag(A_1,A_2)$ the block-diagonal matrix build from $A_1$ and
  $A_2$.
  For every $j\in\{1,2\}$ we have the corresponding inclusion
  $i_j:\C(\!\{t\}\!)^{n_j}\hookrightarrow\C(\!\{t\}\!)^{n_1+n_2}$ and the
  diagram
  \[ \begin{tikzcd}
      \C(\!\{t\}\!)^{n_j} \rar{d-A_j}\dar{i_j} & \C(\!\{t\}\!)^{n_j} \dar{i_j}
    \\\C(\!\{t\}\!)^{n_1+n_2} \rar{d-A} & \C(\!\{t\}\!)^{n_1+n_2}
  \end{tikzcd} \]
  which commutes, since the derivation commutes with the inclusion and the
  matrix $A$ is build in the \rewrite{correct way}, to satisfy
  $i_j(A_jx)=A_j(i_j(x))$:
  \begin{align*}
    i_j(dx-A_jx) &= i_j(dx)-i_j(A_jx)
    \\&=d(i_j(x))-A_j(i_j(x))
    \\&=((d-A)\circ i_j)(x) \,.
  \end{align*}
\end{proof}

\begin{rem}
  \marginnote{\cite[129f]{hotta2008}}
  Let $(\cM_1,\nabla_1)$ and $(\cM_2,\nabla_2)$ be meromorphic connections.
  Then $\cM_1\otimes\cM_2$ is endowed with the structure of a meromorphic
  connection by
  \[
  \nabla(u_1\otimes u_2)=\nabla_1u_1\otimes u_2+u_1\otimes\nabla_2u_2
  \]
  where $u_i\in\cM_i$.
  \begin{comment}
    $\Hom_{\C(\!\{t\}\!)}(\cM_1,\cM_2)$ is endowed with the structure of a
    meromorphic connection by
    \[
    (\nabla\phi)(u_1)=\nabla_2(\phi(u_1))-\phi(\nabla_1 u_1)
    \]
    where $\phi\in\Hom_{\C(\!\{t\}\!)}(\cM_1,\cM_2)$ and $u_i\in\cM_i$.
  \end{comment}
\end{rem}

\begin{comment}
  \begin{lem}
    \PROBLEM[how looks the connection matrix? Simply the sum?]
    Let $(\cM_1,\nabla_1)$ and $(\cM_2,\nabla_2)$ be meromorphic connections
    with connection matrices $A_1$ and $A_2$.
    The connection matrix of $(\cM_1\otimes\cM_2,\nabla)$ is then given by
    $A_1+A_2$.
  \end{lem}
  \begin{proof}
    Let $u_1\otimes u_2\in\cM_1\otimes\cM_2$.
    \begin{align*}
      \nabla(u_1\otimes u_2) &=\nabla_1u_1\otimes u_2+u_1\otimes\nabla_2u_2
      \\&=(\nabla_1u_1)\otimes u_2+u_1\otimes\nabla_2u_2
    \end{align*}
  \end{proof}
\end{comment}

\begin{comment}
%%%%%%%%%%%%%%%%%%%%%%%%%%%%%%%%%%%%%%%%%%%%%%%%%%%%%%%%%%%%%%%%%%%%%%%%%%%%%%%
  \subsubsection{Formalization}
  \begin{multicols}{2}
    Let $[A]$ be a system. We \rewrite{view it as a formal system, by} allowing
    formal solutions.
    \TODO{}

    \columnbreak

    Let $(\cM,\nabla)$ be a meromorphic connection. The connection $\nabla$
    naturally extends to $\hat\cM:=\cM\otimes\C(\!(t)\!)$ and
    $\tilde\cM_\theta:=\cM\otimes\cA_\theta$.
    \TODO{}
  \end{multicols}
\end{comment}


\begin{comment}
%%%%%%%%%%%%%%%%%%%%%%%%%%%%%%%%%%%%%%%%%%%%%%%%%%%%%%%%%%%%%%%%%%%%%%%%%%%%%%%
  \subsubsection{As differential operator}
  \marginnote{\cite[Sec.4.2]{Loday2014}}

  From the theory of ordinary differential equations we know that to
  (\ref{eq:ode}) there is a equivalent ordinary differential equation of order
  $n$ which can be written as
  \[
    \underset{=:P}{\underbrace{%
        (a_n\partial_t^n+a_{n-1}\partial_t^{n-1}+\cdots a_{1}\partial_t+a_{0})
    }} \cdot v=0
  \]
  where $a_i\in\C(\!\{t\}\!)$.
  This leads to the theory of $\cD$-modules.

  \marginnote{See \cite[Sec.1.4]{babbitt1983} for \textbf{ode of rank $n$} to
  \textbf{system}.}
\end{comment}

%%%%%%%%%%%%%%%%%%%%%%%%%%%%%%%%%%%%%%%%%%%%%%%%%%%%%%%%%%%%%%%%%%%%%%%%%%%%%%%
\subsection{Transformation of systems}
\begin{notations}
  We will use the following notations
  \begin{itemize}
    \item $G[t]=\Gl_n(\C[t])$;
    \item $G\{t\}=\Gl_n(\C\{t\})$ the analytic transformations;
    \item $G(\!\{t\}\!)=\Gl_n\left(\C\{t\}[t^{-1}]\right)$ the
      meromorphic\footnote{We use the term meromorphic in the sens of
      convergent meromorphic. Otherwise we say formal meromorphic.}
      transformations;
    \item $G\llbracket t\rrbracket=\Gl_n\left(\C\llbracket t\rrbracket\right)$
      the maybe not applicable formal transformations;
    \item $G(\!(t)\!)=\Gl_n\left(\C\llbracket t\rrbracket[t^{-1}]\right)$
      the maybe not applicable formal meromorphic transformations.
  \end{itemize}
\end{notations}
\marginnote{\cite[Sec.4.3.1]{Loday2014}}
By \emph{meromorphic transformation}, or just \emph{transformation}, of a
system we mean a $\C(\!\{t\}\!)$-linear change of the trivialization.
Such a change is given by a matrix $F\in G(\!\{t\}\!)$ and the transformed
connection matrix ${}^F\!A$ is obtained through the Gauge transformation
\[
  {}^F\!A=(dF)F^{-1} + FAF^{-1} \,.
\]
If $F$ is formal i.e.\ $F\in G(\!(t)\!)$, it will usually be denoted by
$\hat F$.
The transformation of $A$ by $\hat F$ is not guaranteed to have convergent
entries.
We denote by $\hat G(A)$ the set of all \emph{(applicable) formal
transformations}
\[
  \hat G(A):=\bigl\{\hat F\in G(\!(t)\!)
    \mid {}^{\hat F}\!A
    \text{ has convergent entries, i.e.\ ${}^{\hat F}\!A\in G(\!\{t\}\!)$}
  \bigr\}\,.
\]
Let $\hat{F'}\in\hat G(A^0)$ and $A':={}^{\hat{F'}}\!A$, then are the sets
$\hat G(A)$ and $\hat G(A')$ related by
\[
  \hat G(A')=\hat G(A)\hat{F'}^{-1}=\bigl\{
    \hat F\in G(\!\{t\}\!) \mid \hat F\hat{F'}\in\hat G(A)
  \bigr\} \,.
\]

\begin{rem}
  The condition
  \begin{einr}
    $B$ is obtained from $A$ by transformation $F$
  \end{einr}
  is clearly equivalent to
  \begin{einr}
    $F$ solves the linear differential system
    \[
      \frac{dF}{dt}=BF-FA
    \]
    which is denoted by $[A,B]$.
  \end{einr}
  \begin{s-rem}
    If we start with
    \begin{itemize}
      \item two base choices $\cM\overset{\sim}\longrightarrow\C(\!\{t\}\!)^n$
        and $\cM'\overset{\sim}\longrightarrow\C(\!\{t\}\!)^n$ and
      \item an isomorphism
        $\Phi:(\cM,\nabla)\overset{\sim}\longrightarrow(\cM',\nabla')$ together
        with the corresponding base change $F\in G(\!\{t\}\!)$
    \end{itemize}
    we have the following commutative diagram:
    \[ \begin{tikzcd}[column sep=.5cm,row sep=.5cm]
        \C(\!\{t\}\!)^n
        \arrow[green!60!black,thick]{rrrrr}{F}
        \arrow[green!60!black,thick]{dddd}{d-A}&&&&&
          \C(\!\{t\}\!)^n\arrow[green!60!black,thick]{dddd}{d-B}
          \\ & \cM \arrow{dd}{\nabla}\arrow{rrr}{\Phi}\ular&&&
          \cM'\arrow{dd}{\nabla'}\urar
          \\
        \\ & \cM \arrow{rrr}{\Phi}\dlar&&& \cM'\drar
        \\ \C(\!\{t\}\!)^n \arrow[green!60!black,thick]{rrrrr}{F}&&&&&
          \C(\!\{t\}\!)^n
    \end{tikzcd} \]
    \marginnote{\begin{align*}
          ((d-B)\circ F) x &= (F\circ(d-A)) x
        \\(d\circ F) x - (B\circ F) x &= F(dx -Ax)
        \\d(Fx) - B(Fx) &= F(x' - Ax)
        \\Fx'+F'x - BFx &= Fx' - FAx
        \\Fx'+F'x - Fx' &= BFx - FAx
        \\F'x &= (BF - FA)x
      \end{align*}}
    Thus the commutation property for the outer rectangle reads
    \[
      (d-B)\circ F=F\circ(d-A)
    \]
    which is equivalent to
    \[
      \frac{dF}{dt}=BF-FA.
    \]
  \end{s-rem}
\end{rem}
\begin{rem}\label{rem:distributingTransforamtionRule}
  The simple but useful rule
  \[
    {}^{(F_2F_1)}\!A =
    {}^{F_2}\!\left({}^{F_1}\!A\right)
  \]
  can be seen by calculation:
  \begin{align*}
    {}^{(F_2F_1)}\!A
    &= d(F_2F_1)(F_2F_1)^{-1}+F_2F_1 A(F_2F_1)^{-1}
  \\&=\left(
      \left(dF_2\right)̂F_1
      +F_2\left(dF_1\right)̂
    \right) F_1^{̀-1} F_2^{̀-1}
    +F_2F_1 A F_1^{̀-1}F_2^{̀-1}
  \\&= \left(dF_2\right)̂F_2^{-1}
     +F_2\left(dF_1\right)̂ F_1^{̀-1} F_2^{̀-1}
     +F_2
     \left(
       {}^{F_1}\!A-\left(dF_1\right)F^{-1}
     \right)
     F_2^{̀-1}
  \\&= \left(dF_2\right)̂F_2^{-1} +F_2 {}^{F_1}\!A F_2^{̀-1}
  \\&= {}^{F_2}\!\left({}^{F_1}\!A\right) \,.
  \end{align*}
\end{rem}
\begin{defn}
  We define the \emph{(formal) equivalence relation on the connection matrices
  (resp.\ on the corresponding systems)} as
  \begin{einr}
    \textbf{\boldmath$A$ is (formally) equivalent to $B$}
  \end{einr}
  if and only if
  \begin{einr}
    \textbf{\boldmath$B$ is obtained from $A$ by (formal) transformation}.
  \end{einr}
  The \emph{class of a connection matrix} is the orbit under the Gauge
  transformation by $G(\!\{t\}\!)$. The \emph{formal class} is the orbit by
  $\hat G(A)$.
  \begin{s-rem}
    Thus $A$ is (formally) equivalent to $B$ if and only if there is a (formal)
    solution of $[A,B]$.
  \end{s-rem}
  \begin{comment}
    \begin{s-rem}
      This implies also an equivalence relation and a classification on the
      systems.
    \end{s-rem}
  \end{comment}
\end{defn}

The defined equivalence relations are compatible with the \rewrite{isomorphisms
relations} on meromorphic connections (cf.\ \cite[Lem.5.1.3]{hotta2008}),
i.e.\ the following proposition holds.
\begin{prop}
  Two germs of meromorphic connections are (formally) isomorphic if and only if
  their corresponding connection matrices are (formally) equivalent.
\end{prop}

\begin{defn}\label{defn:isotropies}
  \marginnote{\cite[853]{Loday1994}}
  An \emph{isotropy} of $A^0$ or of $[A^0]$ is a transformation
  $\hat F$ which satisfies ${}^{\hat F}\!A^0=A^0$.
  Thus, the isotropies are the solutions of the system
  $[\End A^0]:=[A^0,A^0]$.

  Let $G_0(A^0)$ denote the set of all isotropies of $A^0$.
  \begin{s-rem}
    The isotropies are, a priori, formal transformations.
    Loday-Richaud mentions in her paper~\cite[853]{Loday1994}, that
    actually $G_0(A^0)$ is a subgroup of $\Gl_n(\C[1/t,t])$.
  \end{s-rem}
\end{defn}
\begin{lem}
  \marginnote{\cite[854]{Loday1994}}
  Two formal transformations $\hat F_1$ and $\hat F_2$ take $A^0$ into
  equivalent matrices ${}^{\hat F_1}\!A^0$ and  ${}^{\hat F_2}\!A^0$ if and
  only if there exists isotropy $f_0\in G_0(A^0)$ such that
  $\hat F_1=\hat F_2f_0$
  (cf.~\cite[854]{Loday1994}).
\end{lem}

\marginnote{\cite[Defn.5.1.6]{hotta2008}}
The meromorphic connections are distinguished into regular and irregular
meromorphic connections.
\TODO[more filltext!]
\begin{defn}
  \marginnote{\cite[86]{sabbah2007isomonodromic}}
  A connection with connection matrix $A$ has \emph{regular singularity} at $0$
  if there exists a konvergent transformation, by which $A$ is obtained from a
  matrix with at most a simple pole at $t=0$.
  Otherwise, the singularity is called \emph{irregular}.
  \begin{s-rem}
    This implies that, if $A$ has irregular (resp.\ regular) singularity, then
    also all meromorphic equivalent matrices ${}^{F}\!A$ have irregular
    (resp.\ regular) singularity.
  \end{s-rem}
\end{defn}
\begin{thm}
  Let $(\cM,\nabla)$ be a regular singular meromorphic connection and $A$ its
  connection matrix.
  Then there exists a formal matrix $F\in G(\!\{t\}\!)$ such that after
  transformation by $F$ the matrix $B={}^F\!A$ is constant i.e.\
  ${}^F\!A\in\Gl_n(\C)$ (cf.~\cite[Thm.II.2.8]{sabbah2007isomonodromic}
  or~\cite[Sec.5.1.2]{hotta2008}).
\end{thm}

%%%%%%%%%%%%%%%%%%%%%%%%%%%%%%%%%%%%%%%%%%%%%%%%%%%%%%%%%%%%%%%%%%%%%%%%%%%%%%%
\subsection{Fundamental solutions and the monodromy of a system}
\marginnote{\cite[Sec.4.3.2]{Loday2014},
           \\\cite[130]{hotta2008},
           \\\textbf{\cite[50]{BJL1979Birkhoff}}
           \\,\cite[26]{sibuya1990Linear}}
It is well known in the theory of linear ODEs that the solutions to a system
like (\ref{eq:ode}) form a vector space of dimension $n$ over $\C$, i.e.\ if
$x_0(t)$ and $x_{00}(t)$ are two solution of (\ref{eq:ode}) and $c_1$,
$c_2\in\C$ are two constants, then is also $c_1x_0(t)+c_2x_{00}(t)$ a solution
of the same system.
\begin{defn}
  \marginnote{\cite[4f]{thboalch}}
  A \emph{fundamental matrix of (formal) solutios} or \emph{(formal)
  fundamental solution} $\cY$ on a sector $\mathfrak{s}$ of the
  system $[A]$ is an invertible $n\times n$ matrix (with formal entries), which
  solves $[0,A]$ on $\mathfrak{s}$.
  \begin{s-rem}
    \begin{itemize}
      \item This means that the columns of $\cY$ are $n$ $\C$-linearly
        independent solutions of the system $[A]$ on $\mathfrak{s}$.
      \item Some authors\PROBLEM[?] introduce multi-valued solutions, to avoid
        the restriction to sectors.

        \begin{einr}
          \emph{Multi-valued} are functions, which are not single-valued and
          \emph{single-valued} is a function $f$, which satisfies
          \[
            f(t)=f(t\exp(2\pi i)) \qquad\text{whenever both sides are defined.}
          \]
        \end{einr}
        The function $t\to t^\alpha$, for example, is single-valued whenever
        $\alpha\in\Z$.
    \end{itemize}
  \end{s-rem}
\end{defn}
The Stokes phenomenon, which will be discussed in the
chapter~\ref{chap:stokes}, results from the fact that there is always a formal
fundamental solution, which solves a system on the full arc $S^1$.
But holomorphic solutions, which are asymptotic to the formal solution, may
exist only on small sub-sectors of $S^1$.
\begin{rem}
  \marginnote{\cite[2.1.3]{Zein2009}}
  If the trivialization is changed by $F$ (resp.\ $\hat F$) the fundamental
  solution $\cY\in G(\!\{t\}\!)$ changes to $F\cY$ (resp.\ $\hat F\cY$).
  (cf.\ \cite[Thm.4.3.1]{Loday2014} or \cite[2.1.3]{Zein2009})
\end{rem}

\marginnote{\cite[130]{hotta2008},\\\cite[6]{heu2010}}
Choose a fundamental solution $\cY$. Then analytic continuation along a closed
path $\gamma$ in $M\backslash Z$ provides another fundamental solution
$\cY'=\rho^{-1}(\gamma)\cY$ where $\rho(\gamma)$ is called the \emph{monodromy
along the path $\gamma$}.
\begin{defn}
  Let $[A]$ be a system with fundamental solution $\cY$.
  The analytic continuation of $\cY$ along a small circle around $t=0$ yields
  the fundamental solution
  \[
    \lim_{s\to2\pi}\cY(e^{\sqrt{-1}s}t)=\cY(t)L
  \]
  where $L\in\GL_n(\C)$ is called the \emph{\TODO[(formal)?]monodromy matrix}
  of $[A]$.
  \PROBLEM[wie passt das zu normal-formen?]
\end{defn}

%%%%%%%%%%%%%%%%%%%%%%%%%%%%%%%%%%%%%%%%%%%%%%%%%%%%%%%%%%%%%%%%%%%%%%%%%%%%%%%
\section{Formal classification}\label{sec:formalClassification}
\marginnote{\cite[Thm.4.3.1]{Loday2014}}
In every formal equivalence class of meromorphic connections, there are some
meromorphic connections of special form, which we will call models. They are
not unique but all of them, which are formally isomorphic to a given
meromorphic connection, lie in the same convergent equivalence class.
In fact, every element of this convergent equivalence class will be a model in
our definition.

The models will be used to classify meromorphic connections up to formal
isomorphism. Two meromorphic connections are formally isomorphic, if they are
isomorphic to the (up to convergent isomorphism) same model.

This is essentially given by the Levelt-Turittin Theorem, which states that each
meromorphic connection is, after potentially needed ramification, formally
isomorphic to such a model. Thus the Levelt-Turittin Theorem solves the formal
classification problem.

\subsection{In the language of meromorphic connections: models}
We will start by defining the models in the language of meromorphic connections.
This approach is discussed by Sabbah in~\cite{sabbah_cimpa90} and in a more
general context in~\cite[Sec.II.5]{sabbah2007isomonodromic}.
\begin{defn}\label{defn:elemnMerConnBausteine}
  \begin{itemize}
    \item For a $\phi\in\C(\!\{t\}\!)$ we use $\cE^{\phi}$ to denote the germ
      \[
        (\cE^{\phi},\nabla)=(\C(\!\{t\}\!),d-\phi')\,.
      \]
      This corresponds to the system satisfied by the function $e^\phi$, since
      $(d-\phi')e^\phi=0$.
      \begin{s-prop}\label{cor:uniqueAmbassadorForPhi}
        $\cE^\phi$ is determined by the class of $\phi$ in
        $\C(\!\{t\}\!)/\C\{t\}\cong t^{-1}\C[t^{-1}]$. In the following, we
        will only consider the unique ambassador $\phi$ in each class which has
        no holomorphic part.
      \end{s-prop}
    \item For $\alpha\in\C$, define the \emph{elementary regular meromorphic
      connection of rank one} $\cN_{\alpha,0}$ as the germ
      \[
        (\cN_{\alpha,0},\nabla)=\left(\C(\!\{t\}\!),d+\frac{\alpha}{t}\right)
        \,.
      \]
      This corresponds to the system satisfied by $t^{-\alpha}$.

      \marginnote{\cite[Defn.II.2.5]{sabbah2007isomonodromic}}
      An \emph{elementary regular model of arbitrary rank} is a meromorphic
      connection which has a basis, in which the connection Matrix can be
      written as
      \[
        \frac{1}{t} (\alpha\id+N)
      \]
      where $N\in\gl_{d+1}(\C)$ is a nilpotent matrix.
      If \rewrite{$\alpha\id + N$} is a single Jordan Block, we denote the
      corresponding connection by $\cN_{\alpha,d}$.
      \begin{s-rem}
        The Corollary~\cite[Cor.II.2.9]{sabbah2007isomonodromic} states, that
        every germ of a regular meromorphic connection $(\cR,\nabla)$ is
        isomorphic to some direct sum
        \[
          (\cR,\nabla)=\bigoplus_{\alpha,d}(\cN_{\alpha,d},\nabla)\,.
        \]
        of elementary regular meromorphic connections.
        This solves the classification problem, for regular meromorphic
        connections.
        For a detailed analysis of regular meromorphic connections see Sabbah's
        book~\cite[Sec.II.2]{sabbah2007isomonodromic} or the
        book~\cite[Sec.5.2]{hotta2008} from Hotta et al.
      \end{s-rem}
  \end{itemize}
\end{defn}
\begin{defn}
  \marginnote{\cite[Def.II.5.2]{sabbah2007isomonodromic}}
  A germ $(\cM,\nabla)$ is called \emph{elementary} if it is isomorphic to
  some germ $(\cE^\phi,\nabla)\otimes(\cR,\nabla)$ where
  \begin{itemize}
    \item $(\cE^\phi,\nabla)$ is defined \rewrite{as in}
      definition~\ref{defn:elemnMerConnBausteine} and
    \item $(\cR,\nabla)$ has regular singularity at $\{0\}$, i.e.\ is
      isomorphic to a direct sum of regular elementary meromorphic connections.
  \end{itemize}
  \marginnote{\cite[II.2.f]{sabbah2007isomonodromic}}
\end{defn}
\begin{defn}\label{defn:model}
  \def\myPhi{\textcolor{red!60!black}{\phi}}
  \def\myE{\textcolor{green!40!black}{\cE^{\myPhi}}}
  A germ $(\cM',\nabla')$ is a \emph{model} if there exists, after ramification
  $\cM=\pi^*\cM'$ by $\pi$, a isomorphism to a direct sum of elementary
  meromorphic connections:
  \begin{multicols}{2}
    \[
      \lambda:(\cM,\nabla)
      \overset{\cong}{\longrightarrow}
      % \cong
      \bigoplus_{~\tikzmark{e3}\!\!\myPhi}
      \overset{\tikzmark{e2}}{\myE}
      \otimes
      \overset{\tikzmark{e1}}{\textcolor{blue!40!black}{\cR_{\myPhi}}}
      \,.
    \]
    \columnbreak{}
    \begin{itemize}
      \item[\tikzmarkb{n2}{green}] is irregular singular
      \item[\tikzmarkc{n1}{blue}] has regular singularity at $\{0\}$
      \item[\tikzmarkc{n3}{red}] $\myPhi\in t^{-1}\C[t^{-1}]$ pairwise distinct
    \end{itemize}
    \begin{tikzpicture}[remember picture,overlay]
      \draw[->,blue!50!white,thick] (n1) to[out=180,in=70] (e1);
      \draw[->,green!40!black,thick] (n2) to[out=180,in=70] (e2);
      \draw[->,red!50!white,thick] (n3) to[out=205,in=-30] (e3);
    \end{tikzpicture}
  \end{multicols}
  \begin{s-rem}
    Here it is not necessary to understand ramification, since we restrict to
    the unramified case.
  \end{s-rem}
\end{defn}
The important theorem here is the Levelt-Turittin Theorem, which solves the
formal classification problem.
\begin{thm}[Levelt-Turittin]\label{thm:leveltTurittin}
  To each germ $(\cM',\nabla')$ of a meromorphic connection there exists, after
  potentially needed pullback $\pi^{*}\textcolor{black}{\cM'}=:\cM$ by some
  suitable ramification $t=z^q$ of order $q\geq1$, a
  \textcolor{green!30!black}{\textbf{formal}} isomorphism
  \[
    \textcolor{green!30!black}{\hat{\textcolor{black}{\lambda}}}:
    \textcolor{green!30!black}{\hat{\textcolor{black}{\cM}}}
    \overset{\cong}\longrightarrow
    \textcolor{green!30!black}{\hat{\textcolor{black}{\cM}}^{nf}}
    :=\textcolor{green!30!black}{\hat\cO_M\otimes}\cM^{nf}
  \]
  to a model $\cM^{nf}$.
  We then call $\cM^{nf}$ a \emph{formal decomposition} or \emph{formal model}
  of $\cM$ or $\cM'$.
  \begin{s-rem}\label{rem:leveltTurittin}
    But there is in general \textbf{no} lift of the isomorphism $\hat\lambda$,
    i.e.\ there is no isomorphism making the diagram
    \[ \begin{tikzcd}
        \cM \dar \arrow[dotted]{r}[description]{?} & \cM^{nf} \dar
        \\\hat\cM \rar{\hat\lambda} & \hat\cM^{nf}
    \end{tikzcd} \]
    kommutative.
    But sectorwise, there are lifts given by the main asymptotic existence
    theorem (cf.\ Theorem~\ref{thm:maet}).
  \end{s-rem}
\end{thm}
Proofs of this theorem can be found in multiple places, for example in
\cite[Thm.5.4.7]{sabbah_cimpa90}.

\TODO[Full set of formal invariants]

%%%%%%%%%%%%%%%%%%%%%%%%%%%%%%%%%%%%%%%%%%%%%%%%%%%%%%%%%%%%%%%%%%%%%%%%%%%%%%%
\subsection{In the language of systems: normal forms}\label{sec:normalForms}
\marginnote{\cite{Remy2014}
           \\,\cite[146]{sibuya1990Linear}}
Here we will see, that the fundamental solution of a model can be written in a
special form and that this property characterizes models.
\begin{defn}
  The fundamental solution corresponding to a model (resp.\ a normal form, see
  Definition~\ref{defn:normSol}) will be called a
  \emph{normal solution}.
\end{defn}

Let $(\cM,\nabla)$ be an unramified meromorphic connection which is via some
$\hat\lambda$ formally isomorphic to some model
$\cM^{nf}=\bigoplus_{j=1}^s\cE^{\phi_j}\otimes\cR_j$, where we assume that
$\phi_j\in t^{-1}\C[t^{-1}]$ (cf.~Corollary~\ref{cor:uniqueAmbassadorForPhi}).
For every $j$ is a connection matrix, corresponding to $\cR_j$ given by
$\frac{1}{t}L_j$, where $L$ is in Jordan normal form.
A connection matrix to $\cE^{\phi_j}\otimes\cR_j$ is then by
$q_j'(t^{-1})\cdot\id_{n_j}+\frac{1}{t}L_j$ given\TODO[(cf.~???)], where
$q_j(t^{-1})=\phi_j(t)$ is a polynomial in $t^{-1}$ without constant term and
$q_j'(t^{-1})=\frac{d}{dt}\phi_j(t)$\footnote{By abuse of notation we denote by
  $q_j'(t^{-1})$ the derivation $\frac{d}{dt}(q_j(t^{-1}))=
  \left(\left(\frac{d}{dt}q_j\right)\circ t^{-1}\right) \cdot\frac{d}{dt}t^{-1}$.
  The same does apply for $Q'(t^{-1})$.}.
A connection matrix for $\cM^{nf}$ is then obtained by
\begin{align*}
  A^0&= \bigoplus_{j=1}^s
       \left(q_j'(t^{-1})\cdot\id_{n_j}+\frac{1}{t}L_j\right)
  \\ &=\bigoplus_{j=1}^sq_j'(t^{-1})\cdot\id_{n_j}
       +\frac{1}{t}\bigoplus_{j=1}^sL_j
  \\ &=Q'(t^{-1})+\frac{1}{t}L \,,
\end{align*}
where
$Q(t^{-1}):=\bigoplus_{j=1}^sq_j(t^{-1})\cdot\id_{n_j}$
and $L:=\bigoplus_{j=1}^sL_j$.
\begin{defn}\label{defn:structureComparison}
  Let $L$ be a block diagonal matrix $L=\bigoplus_{j=1}^sL_j$, where the $L_j$
  are of size $n_j\times n_j$ and $Q$ be a diagonal matrix.
  We will say, that \emph{the block structure of $L$ is finer then the
  structure of $Q$} if there are $q_j$'s such that
  $Q=\bigoplus_{j=1}q_j\cdot\id_{n_j}$.
  \begin{s-rem}
    \begin{enumerate}\label{rem:structureComparison}
      \item The matrices $Q$ and $L$ defined above clearly satisfy this
        condition.
      \item When this condition is satisfied, do $L$ and $Q$ commute.
    \end{enumerate}
  \end{s-rem}
\end{defn}
\begin{prop}\label{prop:fundSolBuilder}
  Let $L\in \GL_n(\C)$ be constant and
  $Q(t^{-1})=\diag(q_1(t^{-1}),\dots,q_n(t^{-1}))$ a diagonal matrix of
  polynomials in $t^{-1}$ where $L$ is in Jordan normal form and its block
  structure is finer then the structure of $Q(t^{-1})$.

  The matrix $\cY_0:=t^L e^{Q(t^{-1})}$ is then a fundamental solution of the
  system determined by the matrix
  \[
    A^0=Q'(t^{-1})+L\frac{1}{t} \,.
  \]
\end{prop}
\begin{proof}
  It is easy to see (or cf.~\cite[Appendix C.1]{Balser2000Formal}) that the
  block structure of $t^L$ is finer then the structure of $Q'(t^{-1})$ and thus
  $t^LQ'(t^{-1})t^{-L}=Q'(t^{-1})$, since the (block) structure is preserved
  under the exponential and derivation and thus we can prove that $\cY_0$ is a
  matrix consisting of solutions:
  \begin{align*}
    \frac{d}{dt}\cY_0
    &=\frac{d}{dt}\left(t^Le^{Q(t^{-1})}\right)
  \\&= t^L\frac{d}{dt}e^{Q(t^{-1})} + \frac{d}{dt}t^Le^{Q(t^{-1})}
  \\&= t^LQ'(t^{-1})e^{Q(t^{-1})} + L\frac{1}{t}t^Le^{Q(t^{-1})}
  % \\&=L\frac{1}{t}t^Le^{Q(t^{-1})}+Q'(t^{-1})t^Le^{Q(t^{-1})}
  \\&=\left(t^LQ'(t^{-1})t^{-L}+L\frac{1}{t}\right)t^Le^{Q(t^{-1})}
  \\&=\left(Q'(t^{-1})+L\frac{1}{t}\right)t^Le^{Q(t^{-1})}
  \\&=A^0\cY_0 \,.
  \end{align*}
  The invertability condition is clear.
\end{proof}
\begin{rem}
  If one starts without the assumption on the structure, one only obtains
  \[
    A=t^LQ'(t^{-1})t^{-L}+L\frac{1}{t}
  \]
  as possible system matrix. But the obtained matrix may not define a matrix,
  since it could contain entrys, which are not in $\C(\!\{t\}\!)$.

  A simple example is:
  $L=\begin{pmatrix} \alpha & 1 \\ 0 & \alpha \end{pmatrix}$
  and
  $Q=\begin{pmatrix} \phi_1 & 0 \\ 0 & \phi_2 \end{pmatrix}$,
  with $\phi_1\neq\phi_2$.
  The matrix $A$ is then
  \begin{align*}
    A  &=t^LQ'(t^{-1})t^{-L}+L\frac{1}{t}
    \\ &=
    \begin{pmatrix}
      \phi_1' & (-\phi_1'+\phi_2')\ln(t)
      \\0     & \phi_2'
    \end{pmatrix}
    +
    \begin{pmatrix}
        \alpha & 1
        \\ 0   & \alpha
    \end{pmatrix} \frac{1}{t}
    \notin\gl_n(\C(\!\{t\}\!))
    \,.
  \end{align*}
\end{rem}
To $\hat\lambda$ and the (implicitly) used choices of bases corresponds a
formal transformation $\hat F$ and thus we have
\begin{itemize}
  \item a fundamental solution $\hat Ft^L e^{Q(t^{-1})}$ and
  \item a connection matrix ${}^{\hat F}\!A^0$
\end{itemize}
of $(\cM,\nabla)$, which was defined at the beginning of this subsection.
\begin{cor}
  From the Levelt-Turittin theorem we deduce that every meromorphic connection
  $(\cM,\nabla)$ and thus every system $[A]$ has a fundamental solution of the
  form
  \[
    \mathcal{Y}=\hat F t^L e^{Q(t^{-1})}
  \]
  where $\hat F$ is a formal transformation, solving the differentia equation
  corresponding to the formal isomorphism
  $(\cM^{nf},\nabla^{nf})\to(\cM,\nabla)$.
  In the other way is a system, which has $\cY$ as fundamental solution is
  given by
  \[
    \bar A={}^{\hat F}\!\left(Q'(t^{-1})+L\frac{1}{t}\right) \,.
  \]
  \begin{comment}
    \begin{s-rem}
      It is always possible to permutate the columns of a fundamental solution
      by
      \[
        P^{-1}\mathcal{Y}P=\hat F t^{P^{-1}LP} e^{P^{-1}Q(t^{-1})P}
      \]
      with a permutation matrix $P$ and \rewrite{obtain another fundamental
      solution for the same system} (cf.\ \cite[73]{Loday2014}).
    \end{s-rem}
  \end{comment}
\end{cor}
In the special case when $\hat\lambda$ is a convergent isomorphism, i.e.\ in
the case, when $A$ is already a model, we see that it has a fundamental
solution of the form $Ft^L e^{Q(t^{-1})}$, where $F$ is a convergent
transformation corresponding to $\hat\lambda$.
In fact are the models uniquely characterized as the meromorphic connections,
with a fundamental solution which can be written as $\cY_0=Ft^Le^{Q(t^{-1})}$
and we will use this fact, to say when a system is something like a model.
\begin{defn}\label{defn:normSol}
  Let $[A]$ be a system.
  We call $[A]$ (or $A$) a \emph{normal form} if a fundamental solution of
  $[A]$ can be written as
  \[
    \mathcal{Y}_0(t)=F t^L e^{Q(t^{-1})}
  \]
  with
  \begin{itemize}
    \item an \emph{irregular part} $e^{Q(t^{-1})}$ of $\mathcal{Y}_0$
      determined by
      \[
        Q(t^{-1})=\underset{j=1}{\overset{s}{\bigoplus}}q_j(t^{-1})\id_{n_j}
          =\diag(\underset{n_1\text{-times}}{\underbrace{%
          q_1,\dots,q_1}},q_2,\dots,q_s)
      \]
      where the $q_i(t^{-1})$ are polynomials in $\frac{1}{t}$ (or in a
      fractional power $\frac{1}{s}=\frac{1}{t^{1/p}}$ of $\frac{1}{t}$ for the
      ramified case) such that $q_j(0)=0$, i.e.\ without constant term,
    \item a constant matrix $L\in\gl_n(\C)$ called the \emph{matrix of formal
      monodromy}, where $t^L$ means $e^{L\ln t}$ and
    \item a (convergent) transformation $F\in G(\!\{t\}\!)$.
      \begin{s-rem}
        By changing the basis via $F^{-1}$ we obtain from $[A^0]$ the
        equivalent system $[{}^{F^{-1}}\!A^0]$ with the normal solution
        \[
          F^{-1}\mathcal{Y}_0(t)=t^L e^{Q(t^{-1})} \,.
        \]
        Thus it is always possible to assume that the transformation matrix $F$
        is trivial.
      \end{s-rem}
  \end{itemize}
  The normal forms will often be denoted $A^0$.
  If $A$ is formally equivalent to a normal form $A^0$ we say that $A^0$
  \emph{is a normal form for} $A$ and for $[A]$.
\end{defn}
\begin{cor}\label{cor:structuralAssumptions}
  Since we only look at unramified connections, by the Levelt-Turittin Theorem
  we are able to assume that $L$ is in Jordan normal form and that it has a
  block structure, which is finer then the structure of
  $Q=\bigoplus_{j=1}^sq_j(t^{-1})\cdot\id_{n_j}$
  (cf.~\cite[Sec.1]{Remy2014} or~\cite[Sec.4]{Martinet1991}).
\end{cor}
The following corollary enables us to use normal forms in place of models.
\begin{cor}
  A meromorphic connection $(\cM,\nabla)$ is a model if and only if its
  connection matrix $A$ is a normal form.
\end{cor}

% \begin{comment}
% %%%%%%%%%%%%%%%%%%%%%%%%%%%%%%%%%%%%%%%%%%%%%%%%%%%%%%%%%%%%%%%%%%%%%%%%%%%%%%%
% \subsection{In the language of systems: normal forms}
% In the language of system, the equivalent of models are normal forms, which are
% characterizes by the structure of their fundamental solutions.

% The normal forms are then the systems $[A^0]$, with a fundamental solution
% $\cY_0$ in a special form and from the Levelt-Turittin we then can deduce that
% every system $[{}^{\hat F}\!A^0]$, with $\hat F\in G(\!(t)\!)$, has a
% fundamental solution in the form $\hat F\cY_0$.
% \begin{defn}\label{defn:normSol}
%   Let $[A]$ be a system.
%   We call $[A]$ (or $A$) a \emph{normal form} if its fundamental solution can
%   be written as
%   \[
%     \mathcal{Y}_0(t)=F t^L e^{Q(t^{-1})}
%   \]
%   with
%   \begin{itemize}
%     \item \emph{irregular part} $e^{Q(t^{-1})}$ of $\mathcal{Y}_0$ determined
%       by
%       \[
%         Q(t^{-1})=\underset{j=1}{\overset{s}{\bigoplus}}q_j(t^{-1})\id_{n_j}
%           =\diag(\underset{n_1\text{-times}}{\underbrace{%
%           q_1,\dots,q_1}},q_2,\dots,q_s)
%       \]
%       where the $q_i(t^{-1})$ are polynomials in $\frac{1}{t}$ (or in a
%       fractional power $\frac{1}{s}=\frac{1}{t^{1/p}}$ of $\frac{1}{t}$ for the
%       ramified case) such that $q_j(0)=0$, i.e.\ without constant term,
%     \item $L\in\gl_n(\C)$ constant
%       \PROBLEM[block diagonal (corresponding to $Q$)]
%       matrix called the \emph{matrix of formal
%       monodromy}, where $t^L$ means $e^{L\ln t}$ and
%       \marginnote{in \cite[1]{Remy2014} $L$ is just a Jordan normal form.  Is
%       this generic enough?}
%     \item a transformation $F\in G(\!\{t\}\!)$.
%   \end{itemize}
%   The normal forms will often be denoted $A^0$.
%   If $A$ is formally equivalent to a normal form $A^0$ we say that $A^0$
%   \emph{is a normal form for} $A$ and for $[A]$.

%   The fundamental solution $F t^L e^{Q(t^{-1})}$ of a normal form will be
%   called \emph{normal solution} and will often be denoted by $\cY_0$.
%   \begin{s-rem}
%     \begin{enumerate}
%       \item \TODO[Cor from base change for fundamental solutions] By changing the
%         basis via $F^{-1}$ wo obtain from $[A^0]$ the equivalent system
%         $[{}^{F^{-1}}\!A^0]$ with the normal solution
%       \[
%         F^{-1}\mathcal{Y}_0(t)=t^L e^{Q(t^{-1})} \,.
%       \]
%       Thus it is always possible to assume that the transformation matrix $F$ is
%       trivial.
%       \item In the unramified case $Q$ and $L$ commute thus $L$ can be supposed
%         in Jordan form (cf.\ \cite[Sec.4]{Martinet1991}).
%     \end{enumerate}
%   \end{s-rem}
% \end{defn}
% \begin{thm}\label{thm:modelEqNormalForm}
%   A meromorphic connection $(\cM,\nabla)$ is a model if and only if its
%   connection matrix $A$ defines a normal form $[A]$.
%   \begin{s-rem}
%     \begin{itemize}
%       \item This allows us, to say that a model $(\cM^{nf},\nabla^{nf})$ is a
%         normal form and \rewrite{vice versa}.
%         \rewrite{This} explains, why we mark models with ${}^{nf}$.
%       \item In the unramified case are \rewrite{the $q_i(t^{-1})$ the
%         $\phi_i(t)$} from definition~\ref{defn:model}.
%         \TODO[Why/realy?]
%     \end{itemize}
%   \end{s-rem}
% \end{thm}
% \begin{proof}
%   \textbf{``\Rightarrow{}'':}
%   Let $\cM=\bigoplus_{\phi}\cE^\phi\otimes\cR_\phi$ be a model and let us
%   first fix a $\phi$ and look at the connection $\cE^\phi\otimes\cR_\phi$.
%   We know that $\cR_\phi$ is a direct sum of regular elementary meromorphic
%   connections whose fundamental matrices are of the form
%   $\frac{1}{t}(\alpha\id+N)$ where $\alpha\in\C$ and $N$ is nilpotent
%   (cf.\ Definition~\ref{defn:elemnMerConnBausteine}).
%   Thus the connection matrix for $\cR_\phi$ writes as
%   \[
%     \frac{1}{t} \bigoplus_{(\alpha,N)}\left( \alpha\id+N\right)
%     \,.
%   \]
%   A connection matrix of $\cE^\phi\otimes\cR_\phi$ is then given\PROBLEM[!]
%   by
%   \[
%     t^{\bigoplus_{(\alpha,N)}\left( \alpha\id+N\right)}
%     \left(-\phi'\id\right)
%     t^{-\bigoplus_{(\alpha,N)}\left( \alpha\id+N\right)}
%     + \frac{1}{t} \bigoplus_{(\alpha,N)}\left( \alpha\id+N\right) \,.
%   \]
%   Since $\cM$ is a sum of such elementary connections, The connection matrix
%   of $\cM$ writes as
%   \[
%     \bigoplus_\phi\left(
%       - t^{\bigoplus_{(\alpha,N)}\left( \alpha\id+N\right)}
%       \left(\phi'\id\right)
%       t^{-\bigoplus_{(\alpha,N)}\left( \alpha\id+N\right)}
%       +
%       \frac{1}{t} \bigoplus_{(\alpha,N)}\left( \alpha\id+N\right)
%     \right) \,,
%   \]
%   which can be rewritten as
%   \[
%     \underset{t^LQ'(t^{-1})t^{-L}}{
%       \underset{\text{\rotatebox[origin=c]{-90}{$=$}}}{%
%         \underbrace{-\bigoplus_\phi
%           \left(
%             t^{\bigoplus_{(\alpha,N)}\left( \alpha\id+N\right)}
%             \left(\phi'\id\right)
%             t^{-\bigoplus_{(\alpha,N)}\left( \alpha\id+N\right)}
%           \right)
%         }
%       }
%     }
%     +
%     \frac{1}{t}
%     \underset{L}{
%       \underset{\text{\rotatebox[origin=c]{-90}{$=$}}}{%
%         \underbrace{%
%           \bigoplus_\phi\left(
%             \bigoplus_{(\alpha,N)}\left( \alpha\id+N\right)
%           \right)
%         }
%       }
%     }
%     \,.
%   \]
%   Thus it has
%   \[
%     t^{\bigoplus_\phi\left(
%         \bigoplus_{(\alpha,N)}\left( \alpha\id+N\right)
%     \right)}
%     e^{-\bigoplus_\phi\phi'\id}
%   \]
%   as fundamental solution (cf.\ proposition~\ref{prop:fundSolBuilder}).
%   \PROBLEM[Construction only yields unipotent $N$]

%   \PROBLEM{}

%   \textbf{``\Leftarrow{}'':}
%   Let $A^0$ be a normal form with $\mathcal{Y}_0(t)=F t^L e^{Q(t^{-1})}$ as a
%   normal solution. We use Proposition~\ref{prop:fundSolBuilder} to assume that
%   the connection matrix $A^0$ has the form
%   \[
%     A^0=t^LQ'(t^{-1})t^{-L}+L\frac{1}{t}
%   \]
%   and the Proposition~\ref{prop:systToMeromConn} yields our meromorphic
%   connection $(\cM_{A^0},\nabla_{A^0})=(\C(\!\{t\}\!)^n,d-A^0)$ for
%   \rewrite{which we want to prove, to be} a model.
%   \TODO[We can use Proposition~\ref{prop:MatOfSumOfMerCon} to assume, that $L$ is
%     \textbf{only one Jordan block}.]

%   \PROBLEM[Jordan-NF]
% \end{proof}
% \begin{cor}
%   \TODO[Start with this, bevore defn of normal forms?]
%   From the Levelt-Turittin theorem\comm{, the fundamental solution
%   transformation rules} and Theorem~\ref{thm:modelEqNormalForm} we
%   deduce that every system $[A]$ with normal form $[A^0]$ and normal solution
%   $t^L e^{Q(t^{-1})}$, has the fundamental solution
%   \[
%     \mathcal{Y}=\hat F t^L e^{Q(t^{-1})}
%   \]
%   where $\hat F$ is a solution of $[A^0,A]$.
%   \begin{s-rem}
%     It is always possible to permutate the columns of a fundamental solution by
%     \[
%       P^{-1}\mathcal{Y}P=\hat F t^{P^{-1}LP} e^{P^{-1}Q(t^{-1})P}
%     \]
%     with a permutation matrix $P$ and \rewrite{obtain another fundamental
%     solution for the same system} (cf.\ \cite[73]{Loday2014}).
%   \end{s-rem}
% \end{cor}
% \end{comment}

%%%%%%%%%%%%%%%%%%%%%%%%%%%%%%%%%%%%%%%%%%%%%%%%%%%%%%%%%%%%%%%%%%%%%%%%%%%%%%%
\section{The main asymptotic existence theorem (M.A.E.T)}\label{sec:MAET}
\begin{comment}
  \begin{multicols}{2}
    \textbf{Classical:}
    \begin{itemize}
      \item \cite[Thm.4.4.1]{Loday2014}
      \item \cite[Thm.7.10]{van2003galois}{\tiny\cite[Thm.7.12]{van2003galois}}
      \item \cite[Thm.IV.12.1]{wasow2002asymptotic}
      \item \cite[5.3.Thm.1]{Varadarajan96linearmeromorphic}
      \item \cite[207]{Balser2000Formal}: Some historical remarks
      \item \cite[Thm.A]{BJL1979Birkhoff}
    \end{itemize}
  \columnbreak
    \textbf{Sheafical:}
    \begin{itemize}
      \item \cite[Thm.2.3.1]{sabbah_cimpa90}
      \item \cite[Sec.4.4]{Loday2014}
    \end{itemize}
  \end{multicols}
\end{comment}
Here we want to state the main asymptotic expansion theorem (or often M.A.E.T.)
which is essentially a deduction from the Borel-Ritt Lemma.
\marginnote{\cite[207]{Balser2000Formal}}
It states that to every formal solution of a system of meromorphic differential
equations and every sector with sufficiently small opening, one can find a
holomorphic solution of the system having the formal one as its asymptotic
expansion.

\begin{defn}\label{defn:lift}
  \marginnote{\cite[855]{Loday1994}}
  Let $A$ be via $\hat F$ formally equivalent to $A^0$.
  We call $F$ a \emph{lift of $\hat F$ on $I\subset S^1$} if
  \begin{itemize}
    \item $F\sim_I\hat F$
      (cf.\ Page~\pageref{page:notationForAsymptoticExpansion}) and
    \item $F$ satisfies the same system $[A^0,A]$ as $\hat F$.
  \end{itemize}
\end{defn}
The following theorem, often called the main asymptotic existence theorem, can
be for example found in as Theorem A in the paper~\cite{BJL1979Birkhoff} from
Balser, Jurkat and Lutz, Theorem 3.1 in Boalch's paper~\cite{boalch}
or Theorem 4.4.1 in Loday-Richaud's Book~\cite{Loday2014}.
\begin{thm}[M.A.E.T]\label{thm:maet}
  To every $\hat F\in G(\!(t)\!)$ and to every small enough arc
  $I\subsetneq S^1$ there exists a lift $F$ on $I$.
  \begin{s-rem}
    \marginnote{\cite[Thm.4.4.1]{Loday2014}}
    \PROBLEM[Differential operator is $\triangle$]
    If we write the system $[A^0,A]$ as a differential operator $D$.
    The theorem then states that $D$ acts linearly and surjectively on the
    sheaf $\cA^{<0}$, i.e.\ the sequence
    \[
      \cA^{<0}\overset{D}\longrightarrow\cA^{<0} \longrightarrow 0
    \]
    are exact sequences of sheaves of $\C$-vector spaces.
    \begin{comment}
      (cf.~\cite[App.1;Thm.1]{malgrange1991})
    \end{comment}
  \end{s-rem}
\end{thm}
\begin{rem}
  \PROBLEM[Think about it?]
  In the language of meromorphic connections is this theorem sometimes called
  \emph{sectorial decomposition} and stated for example
  in~\cite[Thm.II.5.12]{sabbah2007isomonodromic}
  and~\cite[Sec.II.2.4]{sabbah_cimpa90}:
  \begin{s-thm}[Sectorial decomposition]\label{thm:sectorialDecompFromMAET}
    \PROBLEM[needs more defns]
    Let $(\cM,\nabla)$ be a meromorphic connection and let
    $\hat\lambda:\hat\cM\to\hat\cM^{nf}$ be the isomorphism given by
    Theorem~\ref{thm:leveltTurittin} together with the model $\cM^{nf}$.
    There exists then, for any $e^{i\theta^0}\in S^1$, an isomorphism
    $\tilde\lambda_{\theta^0}:
    \tilde\cM_{\theta^0}=\cA_{\theta^0}\otimes\cM\to\tilde\cM_{\theta^0}^{nf}$
    lifting $\hat\lambda$ \rewrite{that is, such that} the following diagram
    \[ \begin{tikzcd}
        \tilde\cM_{\theta^o} \dar \rar{\tilde\lambda_{\theta^o}} &
        \tilde\cM^{nf}_{\theta^o} \dar
        \\\hat\cM \rar{\hat\lambda} &
        \hat\cM^{nf}
    \end{tikzcd} \]
    commutes
  \end{s-thm}
  This is exactly the solution to the problem stated in
  Remark~\ref{rem:leveltTurittin}.
\end{rem}

%%%%%%%%%%%%%%%%%%%%%%%%%%%%%%%%%%%%%%%%%%%%%%%%%%%%%%%%%%%%%%%%%%%%%%%%%%%%%%%
\section{The classifying set}\label{sec:classifyingSet}
We want to understand the Set
$\bigl\{\big[(\cM,\nabla)\big]\bigr\}$ of the (convergent)
isomorphism classes of all meromorphic connections. Since we can use the formal
classification (cf.\ Section~\ref{sec:formalClassification}) and we know that
all elements in a convergent isomorphism class lie in the same formal
isomorphism class (In other words: the convergent classification is
\rewrite{finer} than the formal classification) we can reduce the problem by
fixing a model $(\cM^{nf},\nabla^{nf})$ with the corresponding normal form
$A^0$.
\rewrite{Thus we can} restrict ourself to the subset
\begin{multline*}
  {}^0C(\cM^{nf},\nabla^{nf})=\bigl\{
    \bigl[(\cM,\nabla)\bigr]
    \mid \text{there exists a formal isomorphism }
  \\\qquad\hat f:(\hat\cM,\hat\nabla)
      \overset{\sim}\longrightarrow
      (\hat\cM^{nf},\hat\nabla^{nf})
  \bigr\}
\end{multline*}
of all isomorphism classes of meromorphic connections, which are formally
isomorphic to $(\cM^{nf},\nabla^{nf})$. This is the set that we will be
calling the \emph{classifying set (to $(\cM^{nf},\nabla^{nf})$)} and we will
also denote it also by ${}^0C(A^0)$, if we are using the language of systems.
\begin{rem}
  Note that we classify
  \begin{einr}
    meromorphic connections within fixed \textbf{formal meromorphic classes,
    modulo meromorphic equivalence}.
  \end{einr}
  Whereas for example Boalch in~\cite{boalch} and~\cite{thboalch} classifies
  \begin{einr}
    meromorphic connections within fixed \textbf{formal analytic classes,
    modulo analytic equivalence}
  \end{einr}
  as it was done in the older literature.
  This makes no difference, since the resulting classifying sets are isomorphic
  (cf.~\cite{thboalch} or~\cite{babbitt1989local}).

  This distinction relates to the difference between \textbf{‘regular
  singular’} connections and \textbf{‘logarithmic’} connections.
\end{rem}
\marginnote{\cite[6]{thboalch}
  \\\tiny{(\cite[19]{boalch})}
  \\\cite[852]{Loday1994}
  \\\cite[111]{sabbah2007isomonodromic}
  \\\cite{babbitt1983}}
It is convenient to look at the slightly larger space of isomorphism classes of
\emph{marked (meromorphic) pairs}
\[
  \cH(\cM^{nf},\nabla^{nf})=\bigl\{
    \bigl[(\cM,\nabla,\hat f)\bigr]
      \mid
      \hat f:(\hat\cM,\hat\nabla)
        \overset{\sim}\longrightarrow
        (\hat\cM^{nf},\hat\nabla^{nf})
  \bigr\}
\]
in which we also handle the additional information, of the formal isomorphism,
by which the meromorphic connection is isomorphic to the model.
The isomorphisms of marked pairs are defined as follows:
\begin{defn}\label{defn:isomsOfPairs}
  Two germs $(\cM,\nabla,\hat f)$ and $(\cM',\nabla',\hat f')$ are
  isomorphic if there exists an isomorphism
  $g:(\cM,\nabla)\overset{\sim}\longrightarrow(\cM',\nabla')$ such that
  $\hat f=\hat f'\circ \hat g$.
  \begin{s-rem}
    \rewrite{Sabbah states in} \cite[111]{sabbah2007isomonodromic} that such
    an isomorphism is then unique.
  \end{s-rem}
\end{defn}

Equivalently, one can talk in terms of systems. We then denote by
\[
  \Syst_m(A^0):=\bigl\{[A]
    \mid A={}^{\hat F}\!A^0 \text{ for some } \hat F\in G(\!(t)\!)\bigr\}
\]
the set of systems formally meromorphic equivalent to $A^0$.
Since we use meromorphic equivalences, in contrast to \cite{boalch,thboalch},
we denote that in $\Syst_m$ by the subscript ${}_m$.
Thus ${}^0C(\cM^{nf},\nabla^{nf})$ corresponds to
the set ${}^0C(A^0):=\Syst_m(A^0)/G(\!\{t\}\!)$ of meromorphic classes which
are formally equivalent to $A^0$.
Analogous, $\cH(\cM^{nf},\nabla^{nf})$ corresponds to the set $\cH(A^0)$ of
equivalence classes, i.e.\ orbits of $G(\!\{t\}\!)$, in
\[
  \hat\Syst_m(A^0):=\bigl\{\bigl(A,\hat F\bigl)
    \mid A={}^{\hat F}\!A^0 \text{ for some } \hat F\in G(\!(t)\!)\bigl\} \,.
\]
\label{page:ofDefnOfIsomOfMarkedPairs}
The isomorphisms defined in Definition~\ref{defn:isomsOfPairs} translate into
the following equivalence relation:
\begin{einr}
  two marked pairs $(A,\hat F)$ and $(A',\hat F')$ are equivalent, if and only
  if there is a base change $H$ such that
  \PROBLEM[$H\in G(\!\{t\}\!)$? or asymptotic functions?]
  \begin{itemize}
  \item $A'={}^{H}\!A$, i.e.\ $H$ is a solution of $[A,A']$, and
  \item $\hat F'=\hat H\hat F$ (cf.\ \cite[71]{babbitt1989local}).
  \end{itemize}
\end{einr}
In the following diagram the first condition is equivalent to the commutativity
of the \textcolor{green!60!black}{green} square. The second property corresponds
to the commutation property of the top (resp.\ bottom) triangle on the level of
asymptotic expansions\PROBLEM[?].
\[ \begin{tikzcd}[column sep=1cm,row sep=.5cm]
  &&&\C(\{t\})^n \arrow[green!60!black, thick]{ddd}{d-A}
    \arrow[green!60!black, thick]{ddr}{H}
\\\C(\{t\})^n \arrow{rrru}{\hat F}
  \arrow[ultra thick, white]{rrrrd}{\hat F'}
  \arrow{rrrrd}{\hat F'}
  \arrow{ddd}{d-A^0}
\\&&&&\C(\{t\})^n \arrow[green!60!black, thick]{ddd}{d-A'}
\\&&&\C(\{t\})^n \arrow[green!60!black, thick]{ddr}{H}
\\\C(\{t\})^n \arrow{rrru}{\hat F} \arrow{rrrrd}{\hat F'}
\\&&&&\C(\{t\})^n
\end{tikzcd} \]

\begin{lem}
  \marginnote{\url{http://mathworld.wolfram.com/Stabilizer.html}}
  Since $G_0(A^0)$ is by definition the stabilizer of $A^0$ (cf.\
  definition~\ref{defn:isotropies}) and $\Syst_m(A^0)$ is the corresponding
  orbit we can use the Proposition 3.1 in the
  book~\cite{wielandt1964finite} from Wielandt and deduce
  \[
    \Syst_m(A^0)\cong \hat G(A^0)/G_0(A^0) \,.
  \]
  \begin{s-cor}\label{cor:isomorphyOfClassfset}
    \marginnote{\cite[Eq.1.9b]{babbitt1983}}
    Thus the \emph{set of meromorphic classas of systems formally equivalent
    to $A^0$} are just the orbits of $G(\!\{t\}\!)$ in $\Syst_m(A^0)$ that is
    \[
      {}^0C(A^0)\cong G(\!\{t\}\!)\backslash\hat G(A^0)/G_0(A^0)
    \]
    whereas the \emph{set of meromorphic classes of marked pairs $\cH(A^0)$
    of $[A^0]$} is canonically isomorphic to the left quotient
    $G(\!\{t\}\!)\backslash\hat G(A^0)$ (cf.\ \cite[Lem.1.17]{thboalch}).
  \end{s-cor}
\end{lem}

The group $G_0(A^0)$ is easy to compute and is often trivial. In fact, the
elements are block-diagonal corresponding to the structure of $Q$,
see~\cite[77]{Loday2014}.
Thus the structure of ${}^0C(A^0)$ is easily deduced from the structure of
$G(\!\{t\}\!)\backslash\hat G(A^0)$.
