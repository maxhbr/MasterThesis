\section*{vom 19.12.2013}

\begin{ex}
Formal sauber: Wähle Trivialisierungen
\end{ex}
Sei $V\to\Sigma=\P^1$ ein holomorphes Vektor Bündel.
Sei $i\in\{1,\dots,m\}$ fix.
Hier wollen wir das wählen einer lokalen Trivialisierung beziehungsweise
zunächst das wählen einer lokalen Koordinate auf $\P^1$ formal sauber machen,
dazu verwenden wir die $\Proj$-Konstruktion.\\
Das Wählen einer lokalen Koordinate entspricht der Wahl eines homogenen
$f\in\C[T_0,T_1]_+$ so dass
\[
\A_z^1=\Spec\C[z] \overset{\tau_1}\cong D_+(f)
\overset{\tau_2}\hookrightarrow
\Proj\C[T_0,T_1]=\P^1 \,.
\]
Dies soll so gewählt werden, dass \textbf{\boldmath die lokale Koordinate $z$
auf $a_i=(a_{i,0}:a_{i,1})$ verschwindet,} also dass $\tau_2\circ\tau_1(0)=a_i$
gilt.

\textbf{IDEE:} 
\[ \begin{tikzcd}[row sep=0]
\A_z^1 \rar{\cong}& D_+\left(a_{i,0}T_0 + a_{i,1}T_1\right) \rar[hook]& \P^1\\
0 \arrow[|->]{r}& a_i\\
& a \arrow[|->]{r}& a
\end{tikzcd} \]
Dazu:
\begin{comment}
\begin{align*}
D_+\left(a_{i,0}T_0 + a_{i,1}T_1\right)
  &=\Proj \C[T_0,T_1]\Big\backslash
    V_+\Big(\left(a_{i,0}T_0 + a_{i,1}T_1\right)\Big)
\\&\hookrightarrow\P^1
    \qquad \text{~durch~} \qquad a\mapsto a
\end{align*}
\textbf{Fall:} $a=(1:0)$ dann ist $\A^1\cong D_+(T_1)\hookrightarrow \P^1$
durch $x \mapsto (1,z)$
\end{comment}
\textbf{Weitere IDEE:}
Verwende eine Standart Inclusion und verkette diese mit mit einer
Transformation, welche s.B. $(1:0)$ auf $a_i$ verschiebt.

Aus dieser lokalen Koordinate erhält man (in einer Umgebung $U$ von $a_i$) eine
lokale Trivialisierung des Vektor Bündels $V$ auf $\P^1$.
\begin{defn}
Eine \emph{lokale Trivialisierung} eines Vektorbündels $V\to\Sigma$ ist \dots
\end{defn}
\begin{comment}
Wähle an einem Punkt und setze von dort aus fort?!?
\end{comment}
Bezogen auf diese lokale Trivialisierung von $V$ hat $\nabla$ die Form
$\nabla=d-A$ wobei
\[
A=\underset{\text{Hauptteil}}{\underbrace{%
    A_{k_1}\frac{dz}{z^{k_i}} +\dots+ A_{1}\frac{dz}{z}
  }} +
  \underset{\text{Nebenteil}}{\underbrace{A_0dz +\dots}}
\]
eine Matrix meromorpher $1$-Formen und $A_j\in\End(\C^n)$.
\begin{comment}
\begin{defn}
Eine meromorphe $1$-Form ist ein Element $\omega\in\Omega^1(V;\C)$ welche sich
als $\omega=udz$ mit meromorphen $u$ schreiben lässt.
\end{defn}
\end{comment}
\TODO

\begin{ex}
\begin{enumerate}
\item
\comm{(von Def 2.7)} Definition von Erweitertem Orbit $\tilde O_i$ als top.
Raum
\item
\comm{(von Lemma 2.2)} $\tilde O\cong T^*O_u\times O_B$
\end{enumerate}
\end{ex}
Es sei $n\in\N$ fix (Rang des Vektorbündels).
Es sei $\GL_n(R)$ die Gruppe der $n\times n$ Matrizen mit Einträgen in $R$
deren Determinanten eine Einheit in $R$ sind.
Es sei
\begin{align*}
G_k&:=\GL_n\left(\C[\xi]/\xi^k\right)
\\&=\left\{A\in\left( \C[\xi]/\xi^k \right)^{n\times n}\mid
    \det(A)\in \left(\C[\xi]/\xi^k\right)^\times \right\}
\\&\overset{\text{\color{red}?}}=
    \left\{A\in\left( \C[\xi]/\xi^k \right)^{n\times n}\mid
    \det(A)\in \C^\times \right\}
\\&=\left(\left( \C[\xi]/\xi^k \right)^{n\times n}\right)^\times
\end{align*}
die Gruppe der $(k-1)$-Jets von Bündel Automorphismen und
\begin{align*}
B_k&=\left\{ A \in G_k \mid A(0)=1\in \C^{n\times n} \right\}
\\&=\left\{ \sum_{i=1}^{k-1}A_i\xi^i + 1 \in G_k \right\}
\end{align*}
die Untergruppe aller Elemente mit konstantem Term $1$.
Weiter sei $G:=\Gl_n(\C)=G_1$. Damit haben wir die kurze exakte Sequenz
\[
\begin{tikzcd}[row sep=0]
1 \rar & B_k \arrow[hook]{r} & G_k \rar & G \arrow[hook,bend right=40]{l}
  \rar & 1\\
& & A \arrow[|->]{r}& A(0)
\end{tikzcd}
\]
Diese Sequenz splitet, da $G$ in $G_k$ als die Untergruppe der konstanten
Matritzen enthalten ist und die entsprechende Verknüpfung der Morphismen der
Identität $\Id_G$ entspricht.
Es folgt, dass $G_k$ das semi-direkte Produkt $G\ltimes B_k$
ist, wobei $G$ auf $B_k$ durch Konjungation wirkt.
\begin{comment}
Die Kategorie der Gruppen ist nicht Abelsch, deshalb folgt nur der Zerfall in
ein semidirektes Produkt.\\
Siehe \cite[p.148]{hatcher2002algebraic}
\end{comment}
\begin{comment}
Es gilt $B_k\vartriangleleft G_k$, $G_k=B_kG$ und $G\cap B_k=\{1\}$,
deshalb ist $G_k=B_k\rtimes G$ bzw. $G_k=G\ltimes B_k$, wie es hier geschrieben
wird.
\begin{cor}
Aus $G_k=G\ltimes B_k$ folgt, dass $G_k\slash B_k\cong G$.
\end{cor}
\begin{lem}
Falls $G_k=G\ltimes B_k$ und $G_k=G\rtimes B_k$ so folgt, dass $G_k=G\times
B_k$
\end{lem}
\end{comment}

\begin{comment}
\begin{lem}
Wenn $G = TH$ semidirektes Produkt zweier Untergruppen mit normalem $T$ ist und
$\alpha:H\to \Aut(T)$ durch $\alpha_h(t) = hth^{-1}$ definiert wird, dann ist
die Abbildung $T \rtimes_\alpha H \to G, (t,h)\to th$ ein Gruppenisomorphismus.
\end{lem}
\begin{align*}
G\ltimes B_k &=\Gl_n(\C)\ltimes
    \left\{ \sum_{i=1}^{k-1}A_i\xi^i + 1 \in G_k \right\}\\
  &\overset{\text{\color{red}?}}=
    \left\{ g\left(\sum_{i=1}^{k-1}\cdot A_i\xi^i + 1\right)g^{-1}
    \overset{\text{\color{red}?}}\in G_k
    \mid g\in\Gl_n(\C) \right\}\\
  &\overset{\text{\color{red}?}}=
    \left\{ \sum_{i=1}^{k-1}\left(g\cdot A_i\cdot g^{-1}\right)\xi^i + gg^{-1}
    \in G_k \mid  g\in\Gl_n(\C) \right\}\\
  &\overset{\text{\color{red}?}}= G_k \,.
\end{align*}
Sei $n=3$ und $k=2$.
Nun suche $g\in G$ und $A_1\xi + 1\in B_2$ so dass
\begin{align*}
g\left(A_1\xi + 1\right)g^{-1}
  &\overset{!}= \begin{pmatrix}2 & 0 & \xi\\
    \xi & 2 & 0\\
    0 & \xi & 2
    \end{pmatrix}\overset{\text{\boldmath$(\star)$}}\in G_{2}
&\text{~bzw.~}&&
\underset{\in\xi\C^{n\times n}}{\underbrace{%
  \underset{\in\C^{n\times n}}{\underbrace{\left(gA_1g^{-1}\right)}}\xi
}}
  &\overset{!}= \begin{pmatrix}1 & 0 & \xi\\
    \xi & 1 & 0\\
    0 & \xi & 1
    \end{pmatrix}
 \notin\xi\C^{n\times n}
\end{align*}
\textbf{\boldmath$(\star)$}
da nach \cite[Chapter 2, Section 1]{thboalch}
gilt, dass ein Element $\sum_{i=0}^{k-1}A_i\xi^i$ in $G_k$ ist, genau dann,
wenn $\det(A_0)\neq 0$ ist. Hier ist $\det(A_0)=8$.
\end{comment}

Ein Element $A$ aus $G_k$ hat die Form $A=A_0+A_1\xi+\dots+A_{k-1}\xi^{k-1}$,
wobei die $A_i$ aus $\End(\C^n)$ sind. Anderstherum ist ein solches Element
$A_0+A_1\xi+\dots+A_{k-1}\xi^{k-1}$ in $G_k$ genau dann, wenn $\det(A_0)\neq0$.
\begin{comment}
Zu Zeigen: $\det(\sum^{k-1}_{i=0}A_{i}\xi^{i})\neq 0$ $\Leftrightarrow$
$\det(A_0)\neq0$.\\
\textbf{Denn} $\det(\sum^{k-1}_{i=0}A_{i}\xi^{i})=\overset{\text{\color{red}?}}\dots
=\det(A_0)$.
\end{comment}
Die Lie Algebra $\mathfrak g_k=\Lie(G_k)$ zu $G_k$ besteht aus den Elementen
\[
X=X_0+X_1\xi+\dots+X_{k-1}\xi^{k-1}
\]
mit $X_i\in\End(\C^n)$ beliebig.
\begin{comment}
Dazu: \TODO
\end{comment}
Die Elemente aus $\mathfrak g_k^*$, dem Vektorraum Dual von $\mathfrak g_k$
werden als
\[
A=\left(\frac{A_{k}}{\xi^{k}}+\dots+\frac{A_{1}}{\xi}\right)d\xi
\]
geschrieben, wobei die $A_i$ wieder aus $\End(\C)$ beliebig sind. \comm{abusing
notation. Die Elemente aus $G_k$ sollten wohl besser $g$ genannt werden, wie in
der Diss}
\begin{comment}
\ccite[p. 22]{thboalch}
Die Paarung zwischen $\mathfrak g_k^*$ und $\mathfrak g_k$ ist gegeben durch
\[
<A,X>=\Res_0(\Tr(AX))=\sum^{k}_{i=1}\Tr(A_iX_{i-1})
\]
Wobei $\Res_0$ die Residuen Abbildung ist, welche den Koeffizient vor
$d\xi/\xi$ ausgibt.
\textbf{Observe that the product $AX$ is a well defined element of $\mathfrak
g_k^∗$, where $A\in\mathfrak g_k^*$ and $X\in\mathfrak g_k$.}
Ähnlich ist das Produkt $XA$ wohldefiniert in $\mathfrak g_k^∗$. Damit ist
$\mathfrak g_k^*$ ein Bimodul über $\mathfrak g_k$.
\end{comment}

\paragraph{Koadjunggierte Orbiten} (\cite[pp.23-26]{thboalch})
\begin{defn}
\begin{itemize}
\item
Der \emph{$G_k$ koadjungierte Orbit durch $A\in\mathfrak g_k^*$} ist
\[
O(A):=\{gAg^{-1}\mid g\in G_k\}\subset\mathfrak g_k^*
\]
\item
Die \emph{schönen} $G_k$ koadjungierte Orbiten sind die Orbiten, die einen
Erzeuger haben, dessen
führender Koeffizient
\begin{enumerate}
\item
diagonalisierbar mit unterschiedlichen Eigenwerten ist, falls $k\geq2$, oder
\item
diagonalisierbar mit modulo $\Z$ unterschiedlichen Eigenwerten, falls $k=1$.
\end{enumerate}
\end{itemize}
\end{defn}
Diese koadjunggierten Orbiten sind homogene Räume für $G_k$ und deshalb sind
diese glatte komplexe Mannigfaltigkeiten.
Nach symplektischer Geometrie haben koadjungierte Orbiten eine natürliche
(Kostant-Kirillov) symplektische Struktur.
\comm{they are the symplectic leaves of the (Lie) Poisson bracket on the dual
of the Lie algebra. Since everything is complex here, $O(A)$ is naturally a
complex symplectic manifold.}

\TODO

\begin{comment}
Sei $X$ ein Element der Lie Algebra $\mathfrak g_k$ und setze das
charakteristische Polynom
\[
P_X(\lambda):=\det(\lambda1-X)\in\C[\lambda,\xi]/(\xi^k)
\]
von $X$ über dem Ring $\C[\xi]/(\xi)$.
Dieses Polynom hat $\lambda$-Grad $n$ mit Koeffizienten in $\C[\xi]/(\xi^k)$.
\begin{lem}
\ccite[Lemma 2.5]{thboalch}
Sei $X\in\mathfrak g_k$ ein Element, bei dem der Konstante Term
unterschiedliche Eigenwerten hat.
\begin{enumerate}
\item
Falls $g\in G_k$ dann ist $P_{gXg^{-1}}=P_X$
\item
Falls $Y\in\mathfrak g_k$ und $P_Y=P_X$ dann ist $Y=gXg^{-1}$ für ein $g\in
G_k$.
\end{enumerate}
\end{lem}
Und damit erhalten wir das folgende Korollar.
\begin{cor}
\ccite[Corollary 2.6]{thboalch}
Die schönen Orbiten sind affine algebraische Varietäten.
\end{cor}
\begin{cor}\ccite[Corollary 2.7]{thboalch}
Wir nehmen an, dass dar führende Koeffizient von $A\in\mathfrak g_k^*$
unterschiedliche Eigenwerte habe und das für diese bereits eine Anordnung
$e_1,\dots,e_n$ gewählt wurde.
Dann setzt sich die Wahl der Diagonalisierung $\diag(e_1,\dots,e_n)$ des
führenden Koeffizienten von $A$ eindeutig zu einer Diagonalisierung von $A$
fort.
Das bedeutet, dass es eindeutige Elemente $f_i,\dots,f_n\in\C[\xi]/(\xi^k)$
gibt, so dass
\begin{itemize}
\item
$f_i(0)=e_i$ für alle $i$ gilt und \item das entsprechende diagonale Elemente
von $\mathfrak g_k^*$ ist in dem glechem Orbit wie $A$:
\[
\diag(f_i,\dots,f_n)d\xi/\xi^k\in O(A)
\]
\end{itemize}
\end{cor}
\end{comment}

\TODO
\paragraph{Erwieterte Orbiten} (\cite[pp.26-36]{thboalch})
Sei nun $k$ mindestens $2$ und wähle ein diagonales Element
\[
A^0:=\left(\frac{A_{k}^0}{\xi^{k}}+\dots+\frac{A_{2}^0}{\xi^2}\right)d\xi
\in \mathfrak b_k^*
\]
\begin{comment}
Ein diagonales Element ist ein Element, dessen Vorfaktoren alle diagonale
Matritzen sind?
\end{comment}
so dass der führende Koeffizient $A_k^0$ unterschiedliche Diagonaleinträge hat.
Sei $O_B=O_B(A^0)$ der $B_k$ coadjungierte Orbit durch $A^0$.
\begin{defn}
Die \emph{Erweiterung} oder der \emph{erwieterte Orbit} assoziiert zu dem $B_k$
koadjungierten Orbit $O_B$ ist die Menge:
\[
\tilde O = \tilde O(A^0) :=\{ (g_0,A)\in\Gl_n(\C)\times\mathfrak g_k^*
  \mid \pi_{irr}(g_0Ag_0^{-1})\in O_B \}
\]
wobei $\pi_{irr}:\mathfrak g_k^*\to\mathfrak b_k^*$ die natürliche Projektion
ist, welche das Residuum entfernt.
\end{defn}
\begin{comment}
Warum denn \textbf{"erwieterte"} (extended) Orbiten, Fragen:
\begin{itemize}
\item ist für jedes Element von $\tilde O$ auch dessen $G_k$ Orbit komplett in
$\tilde O$ enthalten? Dazu:\\
Sei $A\in \tilde O$, sei $g\in G_k$ dann ist $gAg^{-1}\in O(A)$.
Dann ist $(g^{-1},gAg^{-1})$ in $\tilde O$.\\
$\tilde O \to \mathfrak g_k^*$
\end{itemize}
\end{comment}
\begin{comment}
\textbf{Symplectic aspects of the group cotangent bundles.}
\ccite[p. 19]{thboalch}
Let $G$ be a Lie group.
The \textbf{left multiplication} $L_g:G\to G; h\mapsto gh$ gives an isomrphism
\[
(dL_g)_i:\mathfrak g=T_1G\to T_gG
\]
and induces a trivialization:
\[
G\times\mathfrak g\cong TG; \qquad (g,X)\mapsto (g,(dL_g)_1X)
\]
which will be referred to as the \emph{left trivialization} of $TG$.

By taking the duals the left trivialization of the cotangent bundle is also
obtained:
\[
G\times \mathfrak g^* \cong T^*G;\qquad(g,A)\mapsto(g,(dL_{g^{-1}})_1^\vee A)
\]
where $(dL_{g^{-1}})_1^\vee$ denotes the inverse of the dual linear map to
$(dV_g)_1$.

Now we can write down the natural symplectic structure on $T^∗G$ explicitly:
\begin{lem}
\ccite[Lemma 1.45]{thboalch}
\end{lem}
\begin{proof}
\TODO
\end{proof}
If the right trivializations are used instead, the formula looks the same upto
one sign:
\begin{lem}
\ccite[Lemma 1.46]{thboalch}
\end{lem}
\begin{proof}
\TODO
\end{proof}
\end{comment}
\begin{lem} \comm{(Decoupling)} \ccite[Lemma 2.13]{thboalch}
Die folgende Abbildung ist ein complex analytischer Isomorphismus:
\[
\tilde O\cong T^*\Gl_n(\C)\times O_B;\qquad
  (g_0,A)\mapsto ((g_0,\pi_{res}(A)),\pi_{irr}(g_0Ag_0^{-1}))
\]
wobei $T^*\Gl_n(\C)\cong \Gl_n(\C)\times\mathfrak{gl}_n(\C)^*$ via der links
Trivialisierung und $\pi_{irr}$, $\pi_{res}$ die Projektionen von $\mathfrak
g_k^*$ nach $\mathfrak b_k^*$, $\mathfrak{gl}_n(\C)^*$ passend.
\end{lem}
\begin{proof}
Die Abbildung ist wohldefiniert.
\begin{comment}
denn??
\end{comment}
\textbf{Behauptung:} Die Abbildung:
\[
((g_0,S),B)\overset\psi\mapsto(g_0,g_0^{-1}Bg_0+S)\in\tilde O \,,
\]
mit $((g_0,S),B)\in T^*\Gl_n(\C)\times O_B$, ist ein Inverses zu der oben
definierten Abbildung.\\
Sei $\phi$ die oben definierte Abbildung, dann
\begin{align*}
((g_0,S),B)&\overset\psi\mapsto(g_0,g_0^{-1}Bg_0+S)
\\&\overset\phi\mapsto((g_0,\pi_{res}(g_0^{-1}Bg_0+S)),
    \pi_{irr}(g_0(g_0^{-1}Bg_0+S)g_0^{-1}))
\\&=((g_0,\pi_{res}(g_0^{-1}Bg_0)+\pi_{res}(S)),
    \pi_{irr}( g_0g_0^{-1}Bg_0g_0^{-1})+\pi_{irr}(g_0Sg_0^{-1}))
\\&=((g_0,S), \pi_{irr}(B)+ \pi_{irr}(g_0Sg_0^{-1}))
\\&=((g_0,S), B)
\end{align*}
In der Anderen Richtung gilt:
\begin{align*}
(g_0,A)&\overset\phi\mapsto ((g_0,\pi_{res}(A)),\pi_{irr}(g_0Ag_0^{-1}))
\\&\overset\psi\mapsto (g_0,g_0^{-1}(\pi_{irr}(g_0Ag_0^{-1}))g_0+\pi_{res}(A))
\\&= (g_0,g_0^{-1}g_0(A - \pi_{res}(A))g_0^{-1}g_0+\pi_{res}(A))
\\&= (g_0,A)
\end{align*}
Damit folgt die Behauptung.
\end{proof}

\begin{comment}
\begin{beh}
$\dim(\tilde O)=\dim(O)+2n$
\end{beh}
\end{comment}
\begin{comment}
\ccite[p27]{thboalch}
\[
\Theta :=\left\{ A\in\mathfrak g_k^*\mid (g_0,A)\in\tilde O
  \text{ für ein passendes } g_0\in\GL_n(\C) \right\}
\]
ist der Kern der Projektion
\[
\tilde O\to\mathfrak g_k^*;\qquad (g_0,A)\mapsto A
\]
\begin{cor}
\ccite[Corollary 2.15]{thboalch}
The extended orbit $\tilde O$ is a principal $T$ bundle over $\theta$.
\end{cor}
\end{comment}
\begin{comment}
Wie sieht denn die $B_k$ Wirkung auf $T^*G_k\times O_B$ aus?\\
Siehe: \cite[Definition 1.47]{thboalch}
\begin{defn} \ccite[Definition 1.47]{thboalch}
Die \emph{links Wirkung} von $G$ auf $T^*G$ ist \comm{(in terms of left
trivialisation)}:
\[
\beta(g,A)=(\beta g,A)\,.
\]
Die \emph{rechts Wirkung} \dots
\end{defn}
\begin{lem} \ccite[Lemma 1.48]{thboalch}
Die links Wirkung von $G$ auf $T^*G$ ist Hamiltonisch mit einer
equivarianten momenten Abbildung gegeben \comm{(in terms of left
trivialisation)} durch
\[
\mu_L:G\times\mathfrak g^* \to \mathfrak g^*; \qquad (g,A)\mapsto -\Ad_g^*(A)
\]
Die rechts Wirkung \dots
\end{lem}
Und $B_k$ wirkt auf $O_B$ durch Konjungation.\\
Damit ergibt sich zusammen die $B_k$ Wirkung auf $T^*G_k\times O_B$ durch
\[
\beta(g,A,B)=(\beta g,A,\beta B\beta^{-1})
\]
\end{comment}
\begin{lem}[Lemma 2.2 aus \protect\cite{boalch}]
Der erweiterte Orbit $\tilde O$ ist kanonisch isomorph zu dem symplektischem
quotienten $\symplquot{T^*G_k\times O_B}{B_k}$ wobei das kotangentialbündel
$T^*G_k$ und der koadjungierte Orbit $O_B$ ihre natürliche symplektische
Struktur tragen.
\begin{comment}
Entspricht
\cite[Proposition 2.19]{thboalch}
zusammen mit
\cite[Remark 2.20]{thboalch}
\end{comment}
\end{lem}
\begin{proof}
Die Momenten Abbildung der $B_k$-Wirkung auf $O_B$ ist nur die Inklusion
$O_B\to\mathfrak b_k^*$ und die Momenten Abbildung für die $B_k$-Wirkung auf
$T^*G_k$ ist die Komposition der Momenten Abbildung der links-$G_k$-Wirkung mit
der Projektion $\pi_{irr}:\mathfrak g_k^*\to\mathfrak b_k^*$. \comm{in thboalch
ist $\pi_{irr}$ das $\pi$}
Damit ist die Momenten Abbildung der $B_k$-Wirkung auf dem Produkt die Summe
dieser beiden Wirkungen:
\begin{equation}\label{eq:pfOf2.2-1}
\mu:T^*G_k\times O_B\to \mathfrak b_k^*;\qquad (g,A,B)\mapsto
  -\pi_{irr}\left(\Ad_g^*(A)\right)+B
\end{equation}
Damit ist das Urbild von $0\in\mathfrak b_k^*$ unter $\mu$:
\[
\mu^{-1}(0)=\left\{(g,A,B)\mid\pi_{irr}(gAg^{-1})=B\right\} \,.
\]
\begin{comment}
Sei $(g,A,B)\in T^*G_k\times O_B$ im Kern, also
\begin{align*}
0=\mu((g,A,B))&=-\pi_{irr}\left(\Ad_g^*(A)\right)+B
\\&=-\pi_{irr}\left(gAg^{-1}\right)+B \qquad \text{also} \qquad
    \pi_{irr}\left(gAg^{-1}\right)=B
\end{align*}
\end{comment}
Um den symplektischen Quotienten
$\symplquot{T^*G_k\times O_B}{B_k}:=\mu^{-1}(0)/B_k$
mit $\tilde O$ zu identifizieren, betrachten wir die Abbildung:
\[
\chi:\mu^{-1}(0)\to\tilde O;\qquad(g,A,B)\mapsto(g(0),A)\,.
\]
Diese Abbildung ist wohldefiniert da die Bedingung $\pi_{irr}(gAg^{-1})=B$ aus
(\ref{eq:pfOf2.2-1}) impliziert, dass
\[
\pi_{irr}(g(0)Ag(0)^{-1})=g(0)g^{-1}Bgg(0)^{-1}
\]
\begin{comment}
\begin{align*}
\pi_{irr}(g(0)Ag(0)^{-1})
  &=g(0)\pi_{irr}(A)g(0)^{-1}
\\&=g(0)\pi_{irr}(g^{-1}gAg^{-1}g)g(0)^{-1}
\\&\overset{\text{\color{red}?}}=g(0)g^{-1}\pi_{irr}(gAg^{-1})gg(0)^{-1}
\\&=g(0)g^{-1}Bgg(0)^{-1}
\end{align*}
\end{comment}
und dieses Element ist in dem Gleichem $B_k$ koadjunggierten Orbit wie $B$, da
$g(0)g^{-1}\in B_k$.
\begin{comment}
\begin{align*}
g(0)g^{-1}
  &=g(0)\left(g(0)+g_i\xi^1+\dots+g_k\xi^k\right)^{-1}
\\&\overset{\text{\color{red}?}}=
    g(0)\left(g(0)^{-1}+(g_i)^{-1}\xi^1+\dots+(g_k)^{-1}\xi^k\right)
  & \text{\color{red}so ist das falsch?!?}
\\&=g(0)g(0)^{-1}+g(0)(g_i)^{-1}\xi^1+\dots+g(0)(g_k)^{-1}\xi^k
\\&=1+g(0)(g_i)^{-1}\xi^1+\dots+g(0)(g_k)^{-1}\xi^k \in B_k
\end{align*}
ODER.
\[
(g(0))^{-1} \overset{\text{\color{red}?}}= g^{-1}(0)
\]
\end{comment}
\begin{beh}
$\chi$ ist surjektiv und hat genau die $B_k$ Orbiten in $\mu^{-1}(0)$ als
Fasern.
\end{beh}
Surjektivität ist klar, da wir einen Schnitt von $\chi$ angeben können:
\[
s:(g_0,A)\mapsto (g_0,A,\pi_{irr}(g_0Ag_0^{-1}))\in\mu^{-1}(0)
\]
wobei $(g_0,A)\in\tilde O$.
\begin{comment}
\[
\chi\circ s=\Big((g_0,A)\overset s\mapsto(g_0,A,\pi_{irr}(g_0Ag_0^{-1}))
  \overset\chi\mapsto(\underset{=g_0}{\underbrace{g_0(0)}},A)\Big)
  =\id_{\mu^{-1}(0)}
\]
\end{comment}
Nun wollen wir die Fasern betrachten, sei dazu $\chi(g,A,B)=\chi(g',A',B')$.
Also ist $A=A'$, $g(0)=g'(0)$ und damit $g'=\beta g$ wobei $\beta:=g'g^{-1}$ in
$B_k$ ist.
Dann, erneut nach der Bedingung in (\ref{eq:pfOf2.2-1}), gilt:
\[
B'=\pi_{irr}(g'A'g'^{-1})=\pi_{irr}(\beta gAg^{-1}\beta^{-1})=\beta B\beta^{-1}
\]
\begin{comment}
\begin{align*}
B'&\overset{(\ref{eq:pfOf2.2-1})}=
  \pi_{irr}(g'A'g'^{-1})  \overset{A'=A}{=}
    \pi_{irr}\left(g'\left(g^{-1}g\right)A\left(g^{-1}g\right)g'^{-1}\right)
    =\pi_{irr}(\beta gAg^{-1}\beta^{-1})
\\& =\beta \pi_{irr}(gAg^{-1}) \beta^{-1}
    \overset{(\ref{eq:pfOf2.2-1})}=\beta B\beta^{-1}
\end{align*}
\end{comment}
Damit gilt, dass $(g',A',B')=\beta(g,A,B)$ und damit ist jede Faser von $\chi$
in einem $B_k$ Oribt enthalten.
\begin{comment}
Also(?):
\begin{align*}
\beta(g,A,B)
  &\overset{!}=(\beta g,A,\beta B\beta^{-1})
\\&=(g',A',B')
\end{align*}
\end{comment}
Anderstherum ist klar dass $B_k$ innerhalb der Fasern von $\chi$ wirkt.
\begin{comment}
Wie sieht die $B_k$-Wirkung auf $\mu^{-1}(0)$ aus?
Sei $\beta\in B_k$ und $(g,A,B)\in\mu^{-1}(0)$ so \dots
\end{comment}
Dies entspricht der anderen Inclusion und damit ist die Behauptung gezeigt.
{\hfill \textbf{Beh.} \ensuremath{\qed}}

Dies identifiziert den Quotienten $\mu^{-1}(0)/B_k$ mit $\tilde O$ \textbf{als
Mannigfaltigkeit.}

\TODO
\end{proof}

\begin{ex}
Proof (2.1)
\end{ex}
\begin{prop}\ccite[Prop 2.1]{boalch}
\begin{itemize}
\item
\ccite[Theorem 2.35]{thboalch}
$\mathcal M(\textbf a)$ ist isomorph zu dem komplexen symplektischem Quotienten
\[
\mathcal M(\textbf a)\cong\symplquot{O_1\times\dots\times O_m}{G} \,.
\]
\item
\ccite[Theorem 2.43]{thboalch}
Analog gibt es komplexe symplektische Mannigfaltigkeiten (erweiterte Orbiten)
$\tilde O_i$ mit $\dim(\tilde O_i)=\dim(O_i)+2n$ und freie Hamiltonische
$G$-Wirkungen, so dass
\[
\widetilde{\mathcal M}(\textbf a)\cong
\symplquot{\tilde O_1\times\dots\times\tilde O_m}{G} \,.
\]
\item \TODO[Unterpunkt 3]
\end{itemize}
\end{prop}
\begin{proof}
Wir wählen eine Koordinate $z$ und identifizieren $\P^1$ mit
$\C\cup\{\infty\}$, so dass jedes $a_i$ endlich ist.
Definiere $z_i:=z-a_i$.
Die gewählten Keime $d-{}^iA^0$ des meromorphen Zusammenhangs bestimmen
$G_{k_i}$ koadjunggierte Orbiten $O_i$ und setzen sich zu erweiterten Orbiten
$\tilde O_i$ fort.
\begin{comment}
Setze $O_i$ als den koadjunggierten Orbit durch den Punkt von $\mathfrak
g_k^*$, welcher durch den Hauptteil von ${}^iA^0$ \comm{~in (5)~} gegeben ist.
Analog bestimmt der irreguläre Teil von ${}^iA^0$ einen Punkt in $\mathfrak
b_{k_i}^*$ und $\tilde O_i$ ist der erwieterte Orbit assoziiert zu dem
$B_{k_i}$ koadjunggierten Orbit durch den Punkt
\end{comment}
\end{proof}

% vim:set ft=tex foldmethod=marker foldmarker={{{,}}}:
