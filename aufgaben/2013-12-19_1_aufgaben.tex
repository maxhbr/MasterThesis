\section*{vom 19.12.2013}

\begin{ex}
Formal sauber: Wähle Trivialisierungen, also
\begin{itemize}
\item Wähle lokale Koordinaten und
\item erhalte daraus eine Trivialisierung
\end{itemize}
\end{ex}
Sei $V\to\Sigma=\P^1$ ein holomorphes Vektor Bündel.
Sei $i\in\{1,\dots,m\}$ fix.
Hier wollen wir das wählen einer lokalen Trivialisierung beziehungsweise
zunächst das wählen einer lokalen Koordinate auf $\P^1$ formal sauber machen,
dazu verwenden wir die $\Proj$-Konstruktion.\\
Das Wählen einer lokalen Koordinate entspricht der Wahl eines homogenen
$f=f_0T_0+f_1T_1\in\C{[T_0,T_1]}_+$ vom Grad $1$ sowie eines Isomorphismus
$\tau_1$ und einer Inclusion $\tau_2$ so dass
\[ \begin{tikzcd}
\A_z^1\bydef\Spec\C[z] \rar{\overset{\tau_1}\cong} 
  &D_+(f) \bydef \Spec \C[T_1,T_2]_{(f)} \rar[hook]{\tau_2}
  & \P^1 \bydef \Proj\C[T_0,T_1] \,.
\end{tikzcd} \]
Dies soll so gewählt werden, dass \textbf{\boldmath{}die lokale Koordinate $z$
auf $a=[a_{0}:a_{1}]\in\P^1$ verschwindet,} also dass $\tau_2\circ\tau_1(0)=a$
gilt.

Um $\tau_1$ zu finden, suche einen Isomorphismus $t_1:\C[z]\to
\C{[T_0,T_1]}_{(f)}$, so dass dieser nach anwenden von $\Spec$ den gesuchten
Isomorphismus ergibt.
Wir suchen also ein homogenes Element, auf welches wir $z$ abbilden.
Ein allgemeines solches Element ist gegeben durch $\frac{\alpha T_0+\beta
T_1}{f}$ mit $[\alpha:\beta]\in\P^1$.

Wie sieht $\tau_2$ aus?

\begin{comment}
BIS HIER HIN NEU!!
\end{comment}

\textbf{IDEE:}
\[ \begin{tikzcd}[row sep=0]
\A_z^1 \rar{\cong}& D_+\left(a_{i,0}T_0 + a_{i,1}T_1\right) \rar[hook]{\iota}
  & \P^1\\
0 \arrow[|->]{r}& a_i
\end{tikzcd} \]
Dazu:
\begin{comment}
\begin{align*}
D_+\left(a_{i,0}T_0 + a_{i,1}T_1\right)
  &=\Proj \C[T_0,T_1]\Big\backslash
    V_+\Big(\left(a_{i,0}T_0 + a_{i,1}T_1\right)\Big)
\\&\hookrightarrow\P^1
    \qquad \text{~durch~} \qquad a\mapsto a
\end{align*}
\textbf{Fall:} $a=[1:0]$ dann ist $\A^1\cong D_+(T_1)\hookrightarrow \P^1$
durch $x \mapsto (1,z)$
\end{comment}

\begin{center} \rule{0.7\textwidth}{0.4pt} \end{center}

\textbf{Weitere IDEE:}
Verwende eine standard Inclusion und verkette diese mit mit einer
Transformation, welche z.B. $[1:0]$ auf $a_i$ verschiebt.
\[ \begin{tikzcd}[row sep=0]
\A_z^1 \rar{\cong}& D_+\left(T_0\right) \rar{\cong}
  &{\text{\huge???}} \rar[hook]{\iota} & \P^1\\
z \arrow[|->]{r}& {[1:z]} \rar[|->] & {[a_{i,0},a_{i,1}+z]}\\
  & {[1:0]} \arrow[|->]{r}& a_i
\end{tikzcd} \]
Wobei der zweite Isomorphismus durch eine invertierbare Matrix repräsentiert
wird. Beispielsweise $\footnotesize\begin{pmatrix}a_{i,0} & 0\\ a_{i,1} & 1
\end{pmatrix}$, diese schickt $[1:0]$ auf $a_i$.

\begin{center} \rule{0.7\textwidth}{0.4pt} \end{center}

\textbf{Neuer Start:}
\begin{comment}
Frage: Welche Automorphismen hat man auf $D_+(T_0)$ oder $\A_z^1$?\\
Antwort: \TODO
\end{comment}

Betrachte das Diagram:
\[
\begin{tikzcd}
\C[T_0,T_1]_{(T_0)} \dar[dashed]{?}
& \C[z] \lar \arrow[dashed]{ld}{?}
\\
\C[T_0,T_1]_{(\alpha T_0+\beta T_1)}
\end{tikzcd}
\qquad\overset{\Spec}{\text{\Huge$\rightsquigarrow$}}\qquad
\begin{tikzcd}
D_+(T_0) \rar
& \A_z^1
\\
D_+(\alpha T_0+\beta T_1) \uar{\cong} \arrow{ur}
\end{tikzcd}
\]

Weiteres Diagram:
\[
\begin{tikzcd}[row sep=0,column sep=1.5em]
\Spec \C[z] \rar{\Psi}
    %\arrow[out=340,in=180]{rrdd} \arrow[bend right=40]{rrrr}{?}
  & D_+(\alpha T_0+\beta T_1) \rar[hook]
  & \P^1 
  & D_+(T_0) \lar[hook]
  & \Spec \C[t] \lar{\Phi}
    %\arrow[out=200,in=0]{lldd}
\\ && \cup
\\ 
&& D_+(\alpha T_0+\beta T_1) \cap D_+(T_0)
\\
\Psi^{-1}(\cap) \ar[hook]{uuu} \ar{rru}{\Psi} \ar[bend right=10,dashed]{rrrr}{?}
& & & &
\Phi^{-1}(\cap) \ar[hook]{uuu} \ar{llu}{\Phi}
\end{tikzcd}
\]

\begin{center} \rule{0.7\textwidth}{0.4pt} \end{center}

Aus dieser lokalen Koordinate erhält man (in einer Umgebung $U$ von $a_i$) eine
lokale Trivialisierung des Vektor Bündels $V$ auf $\P^1$.
\begin{defn}
Eine \emph{lokale Trivialisierung} eines Vektorbündels $\pi:V\to\Sigma$ ist:
\begin{itemize}
\item eine Teilmenge $U$ von $\Sigma$ und dazu
\item ein Isomorphismus $\pi^{-1}(U)\to U\times\C^r$ wenn $r$ der Rang ist.
\end{itemize}
\end{defn}
\begin{comment}
Wähle an einem Punkt und setze von dort aus fort?!?
\end{comment}
Bezogen auf diese lokale Trivialisierung von $V$ hat $\nabla$ die Form
$\nabla=d-A$ wobei
\[
A=\underset{\text{Hauptteil}}{\underbrace{%
    A_{k_1}\frac{dz}{z^{k_i}} +\cdots+ A_{1}\frac{dz}{z}
  }} +
  \underset{\text{Nebenteil}}{\underbrace{A_0dz +\dots}}
\]
eine Matrix meromorpher $1$-Formen und $A_j\in\End(\C^n)$.
\begin{comment}
\begin{defn}
Eine meromorphe $1$-Form ist ein Element $\omega\in\Omega^1(V;\C)$ welche sich
als $\omega=udz$ mit meromorphen $u$ schreiben lässt.
\end{defn}
\end{comment}
\TODO{}

% vim:set ft=tex foldmethod=marker foldmarker={{{,}}}:
