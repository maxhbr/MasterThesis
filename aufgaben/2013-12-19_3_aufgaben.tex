\begin{ex}
Proof (2.1)
\end{ex}
\begin{prop}\ccite[Prop 2.1]{boalch}
\begin{itemize}
\item
\ccite[Theorem 2.35]{thboalch}
$\mathcal M(\textbf a)$ ist isomorph zu dem komplexen symplektischem Quotienten
\[
\mathcal M(\textbf a)\cong\symplquot{O_1\times\cdots\times O_m}{G} \,.
\]
\item
\ccite[Theorem 2.43]{thboalch}
Analog gibt es komplexe symplektische Mannigfaltigkeiten (erweiterte Orbiten)
$\tilde O_i$ mit $\dim(\tilde O_i)=\dim(O_i)+2n$ und freie Hamiltonische
$G$-Wirkungen, so dass
\[
\widetilde{\mathcal M}(\textbf a)\cong
\symplquot{\tilde O_1\times\cdots\times\tilde O_m}{G} \,.
\]
\item \TODO[Unterpunkt 3]
\end{itemize}
\end{prop}
\begin{proof}
Wir wählen eine Koordinate $z$ und identifizieren $\P^1$ mit
$\C\cup\{\infty\}$, so dass jedes $a_i$ endlich ist.
Definiere $z_i:=z-a_i$.
Die gewählten Keime $d-{{}^i}A^0$ des meromorphen Zusammenhangs bestimmen
$G_{k_i}$ koadjunggierte Orbiten $O_i$ und setzen sich zu erweiterten Orbiten
$\tilde O_i$ fort.
\begin{comment}
Setze $O_i$ als den koadjunggierten Orbit durch den Punkt von $\mathfrak
g_k^*$, welcher durch den Hauptteil von ${}^iA^0$ \comm{~in (5)~} gegeben ist.
Analog bestimmt der irreguläre Teil von ${}^iA^0$ einen Punkt in $\mathfrak
b_{k_i}^*$ und $\tilde O_i$ ist der erwieterte Orbit assoziiert zu dem
$B_{k_i}$ koadjunggierten Orbit durch den Punkt
\end{comment}
\end{proof}

% vim:set ft=tex foldmethod=marker foldmarker={{{,}}}:
