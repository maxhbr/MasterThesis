%%%%%%%%%%%%%%%%%%%%%%%%%%%%%%%%%%%%%%%%%%%%%%%%%%%%%%%%%%%%%%%%%%%%%%%%%%%%%%%
\section{Stokes structures: Malgrange-Sibuya isomorphism}\label{sec:mainThm1}
\marginnote{\cite[Thm.I.2.1]{Loday1994},
  \\\cite[Thm.4.3.9]{Loday2014},
  \\\cite[Thm.II.6.2]{sabbah2007isomonodromic}}
Here we will look at the classifying set and we will proof that it is
isomorphic \TODO[as\dots] to the first non abelian cohomology \rewrite{set}
$H^1(S^1;\Lambda(A^0))$, which will be denoted as $\St(A^0)$.
If we talk about cocycles or cochains, we will in the following always mean
$1$-cocycles or $1$-cochains.

Let us first define the Stokes sheaf $\Lambda(A^0)$ on $S^1$, as the sheaf of
flat isotropies of $[A^0]$.
\begin{defn}\label{defn:StokesSheaf}
  \begin{comment}
    The Stokes sheaf $\Lambda(A^0)$ of $A^0$, is the sheaf of groups defined on
    $S^1$ whose stalk at any point $\theta\in S^1$ is the group of germs of
    $f\in\Gl_n(\cO(\mathfrak{s}))$, $\mathfrak{s}$ a sector containing
    $\theta$, satisfying the conditions:
    \begin{enumerate}
      \item Flatness:
        $\underset{x\in\mathfrak{s}}{\underset{x\to0}{\lim}}f(x)=1$ and
        $f\sim_{\mathfrak{s}} 1$;
        \PROBLEM[Why two condition?]
      \item Isotropy of $A^0$: ${}^f\!A^0=A^0$.
    \end{enumerate}
  \end{comment}
  The Stokes sheaf $\Lambda(A^0)$ of $A^0$, is defined as the subsheaf of
  $\GL_n(\cA)$, in the following way.
  For some $\theta\in S^1$ is the stalk at $\theta$ the subgroup of
  $\GL_n(\cA)_\theta$ of elements $f$ which satisfy
  \begin{enumerate}
    \item Multiplicatively flatness: $f$ is asymptotic to the identity,
      i.e.\ $f\sim_{\mathfrak{s}} 1$;
    \item Isotropy of $A^0$: ${}^f\!A^0=A^0$.
  \end{enumerate}
  \begin{s-rem}
    This definition \rewrite{makes also sense as} $\Lambda(A)$ where $A$ stands
    for a systems wich is not in normal form. The elements of $\Lambda(A)$ then
    have to be isotropies of a normal form $A^0$ of $A$.
  \end{s-rem}
  \begin{comment}
    \begin{s-lem}
      This is independent of the choice of the normal form.
    \end{s-lem}
  \end{comment}
  \iffalse
  \begin{comment}
    \begin{s-rem}
      \PROBLEM[remove? need more defs!]
      Sabbah \cite[110]{sabbah2007isomonodromic} talks about (global)
      meromorphic connections $\sM$ on a small disk $D$ around $0$ instead of
      germs of meromorphic connections.

      Define on $S^1$ the sheaf $\Aut^{<0}(\tilde\sM^{nf})$ of automorphisms of
      $\tilde\sM^{nf}:=\cA_D\otimes_{\cO_D}\sM^{nf}$ which
      \begin{itemize}
        \item are compatible with the connection and
        \item are formally equal to the identity, i.e.\ induce the identity on
          $\hat\sM^{nf}:=\hat\cO_D\otimes_{\cO_D}\sM^{nf}$
      \end{itemize}

      The sheaf $\Aut^{<0}(\tilde\sM^{nf})$ corresponds to our $\Lambda(A^0)$.
    \end{s-rem}
  \end{comment}
  \fi
\end{defn}

\subsection{The theorem}
\rewrite{Here we want to state the Malgrange-Sibuya Theorem. We will first give
it in the language of meromorphic connections and after that we will give the
same theorem in the language of systems. The second variant of this theorem
will be proven in the next section.}

In the language of meromorphic connections is the map, in the Malgrange-Sibuya
Theorem bellow, described as follows.

Let $(\cM,\nabla,\hat f)$ be a marked germ of a  meromorphic connection.
\rewrite{By Theorem}~\ref{thm:sectorialDecompFromMAET} there exists an open
covering $\cU=(U_j)_{j\in J}$ and for every open set, an isomorphism
\[
  f_j:(\tilde\cM,\tilde\nabla)_{|U_j}
  \to(\tilde\cM^{nf},\tilde\nabla^{nf})_{|U_j}
\]
such that $f_j\sim_{U_j}\hat f$.
By $(f_kf_j^{-1})_{jk}$ is then a cocycle of the sheaf $\St(A^0)$, relative
to the covering $\cU$, defined.
\comm{For other lifts $f_j'$ of $\hat f$ on $W_j$, $(f_j'f_j^{-1})$ is a
  $0$-cochain of $\Sto(A^0)$ relative to $\cU$. Thus the associated cochians to
  $(f_j)$ and $(f_j')$ are equivalent. One can also check that, if
  $(\cM,\nabla,\hat f)$ and $(\cM',\nabla',\hat f')$ are isomorphic, the
  corresponding cocycles define the same cohomology class.}
This defines a mapping of pointed sets
\[
  \exp:\cH(\cM^{nf},\nabla^{nf})\longrightarrow H^1(S^1;\Lambda(A^0))
\]
to the first non abelian cohomology of $\Lambda(A^0)$, which sends the class of
$(\cM^{nf},\nabla^{nf},\hat\id)$ to that of $\id$, i.e.\ the trivial cohomology
class.

\begin{center}
  \begin{minipage}[t]{0.8\textwidth}
    \begin{tthm}[Malgrange-Sibuya] \label{thm:mainThm1MeromVersion}
      \marginnote{\cite[Thm.I.4.5.1]{babbitt1989local},
        \\\cite[Thm.3.4]{Malgrange1983},
        \\\cite[Thm.13]{Martinet1991}}
      The homomorphism
      \[ \begin{tikzcd}
          \exp:\cH(\cM^{nf},\nabla^{nf}) \rar& \St(A^0):=H^1(S^1;\Lambda(A^0))
      \end{tikzcd} \]
      is an isomorphism of pointed sets.
    \end{tthm}
  \end{minipage}
\end{center}

\subsubsection{The theorem (system version)}
Since the language of meromorphic connections is equivalent to the one of
systems, there is also the translated version of the Malgrange-Sibuya
isomorphism to the language of systems. The corresponding map is then build as
follows.

\marginnote{\cite[855]{Loday1994}}
Let $(A,\hat F)$ be a marked pair, thus $\hat F$ solves $[A^0,A]$.
By the M.A.E.T (Theorem~\ref{thm:maet}) there exists an open covering
$\cU=(U_j)_{j\in J}$ together with, for every open set $U_j$ a lift
$F_j\in\Gl_n(\cA(U_j))$ of $\hat F$ (cf.\ Definition~\ref{defn:lift}), which
solves $[A,A^0]$.
By the cocycle $(F_l^{-1}F_j)_{jl}\in\Gamma(\dot\cU;\Lambda(A^0))$ is then a
cohomology class in $\St(A^0)$, relative to the covering $\cU$, determined.
For other lifts $F_j'$ of $\hat F$ on $U_j$ is $(G_j=F_j^{-1}F_j')$ a
$0$-cochain of $\Lambda(A^0)$ relative to $\cU$, which satisfies
\[
  F_k^{-1}F_j=G_k F_k'^{-1}F'_j G_j^{-1} \,.
\]
Thus the cochians associated to $(F_j)$ and $(F_j')$ determine the same
cohomology class in $\St(A^0)$.
One can also check that, if $(A,\hat F)$ and $(A',\hat F')$ are equivalent, the
corresponding cocycles define the same cohomology class.
This defines a welldefined mapping of pointed sets
\[
  \cH(A^0)\to H^1(S^1;\Lambda(A^0))
\]
to the first non abelian cohomology of $\Lambda(A^0)$, which we call $\exp$.
It maps the class of $(A^0,\hat\id)$ to that of $\id$, i.e.\ the trivial
cohomology class.

\begin{center}
  \begin{minipage}[t]{0.8\textwidth}
    \begin{tthm}[Malgrange-Sibuya \rewrite{(system version)}]\label{thm:mainThm1}
      \marginnote{\cite[Theorem 4.5.1]{babbitt1989local}}
      The homomorphism
      \[ \begin{tikzcd}
          \exp:\cH(A^0) \rar& \St(A^0):=H^1(S^1;\Lambda(A^0))
      \end{tikzcd} \]
      is an isomorphism of pointed sets.
    \end{tthm}
  \end{minipage}
\end{center}
\begin{rem}
  % \TODO[Frage Felix!]
  The Theorem \cite[Thm.III.1.1.2]{babbitt1989local}, in the book from
  Babbitt and Varadarajan, states that $\St(A^0)$ is actually a local moduli
  space for marked pairs, which are formally isomorphic to a given system
  $[A^0]$.
  \begin{comment}
    This means that
    \begin{itemize}
      \item the morphism property,
      \item the criterion of equivalence and
      \item the existence of universal families
    \end{itemize}
    are satisfied (cf.\ \cite[169]{babbitt1989local}).
  \end{comment}
  In fact is the whole third part of \cite{babbitt1989local} dedicated to this
  topic.
\end{rem}
Since the morphism $\exp$ depends on the choice of the normal form, we will
denote that, if it is not clear, by $\exp_{A^0}=\exp$\comm{~(In Loday-Richaud's
paper is this denoted as $\exp_{\mu_0}$)}.
\begin{rem}\label{rem:expNonNormalForm}
  \marginnote{\cite{Loday1994} Remark I.2.2}
  To another normal form $A^1={}^\Phi\!A^0$ there correspond cochains, which
  are conjugated via $\Phi\in G(\!\{t\}\!)$.
  We especially get the following commutative diagram:
  \begin{center}
    \begin{tikzpicture}[scale=2.6]
      \node (tl) at (-1,0.6) {$G\backslash\hat G(A^1)$};
      \node (tr) at (1,0.6) {$G\backslash\hat G(A^0)$};
      \node (bl) at (-1,-0.6) {$H^1(S^1;\Lambda(A^1))$};
      \node (br) at (1,-0.6) {$H^1(S^1;\Lambda(A^0))$};
      \node (tlin) at ($ (tl) - (-0.3,0.2) $)
        {\rotatebox[origin=c]{135}{$\in$}};
      \node (trin) at ($ (tr) - (-0.3,0.2) $)
        {\rotatebox[origin=c]{135}{$\in$}};
      \node (blin) at ($ (bl) - (-0.2,0.2) $)
        {\rotatebox[origin=c]{135}{$\in$}};
      \node (brin) at ($ (br) - (-0.2,0.2) $)
        {\rotatebox[origin=c]{135}{$\in$}};
      \node (tlE) at ($ (tl) - (-0.5,0.4) $) {$\hat F$};
      \node (trE) at ($ (tr) - (-0.5,0.4) $) {$\hat F\Phi$};
      \node (blE) at ($ (bl) - (-0.5,0.4) $) {$\exp_{A^1}(\hat F)$};
      \node (brE) at ($ (br) - (-0.5,0.4) $) {$\exp_{A^0}(\hat F\Phi)$};
      \draw[->] (tl) -- (tr) node [midway,above] {$\cdot\Phi$};
      \draw[->] (tl) -- (bl) node [midway,right] {$\exp_{A^1}$};
      \draw[->] (tr) -- (br) node [midway,right] {$\exp_{A^0}$};
      \draw[->] (bl) -- (br);
      \draw[|->] (tlE) -- (trE);
      \draw[|->] (tlE) -- (blE);
      \draw[|->] (trE) -- (brE);
      \draw[|->] (blE) -- (brE);
    \end{tikzpicture}
  \end{center}
  where $\exp_{A^0}(\hat F\Phi)=\Phi^{-1}\exp_{A^0}(\hat F)\Phi$.
\end{rem}

\subsection{Proof of Theorem~\ref{thm:mainThm1}}
We will mainly refer to \cite[Proof of Theorem 4.5.1]{babbitt1989local} and
\cite[Section 6.d]{sabbah2007isomonodromic}, where a slightly more complicated
case with deformation space is proven. These both resources proof the theorem
using the languages of meromorphic connections whereas we will use systems.
\marginnote{See also \cite{BJL1979Birkhoff} and \cite{babbitt1989local}
  although the proof goes back to work from Malgrange and Sibuya (see for
  example \cite{sibuya1990Linear}).}

We will start by proofing the injectivity of the morphism $\exp$.
\begin{proof}[Proof of the injectivity]
  % \textbf{First look at injectivity:}
  Consider the two marked pairs $(A,\hat F)$ and $(A',\hat F')$ in
  $\hat\Syst_m(A^0)$, whose classes in $\cH(A^0)$ get mapped to same element
  \[
    \exp([(A,\hat F)])=\lambda=\exp([(A',\hat F')])
      \in H^1(S^1;\Lambda(A^0)) \,.
  \]
  By using refined coverings, it is possible to find a common finite covering
  $\cU=\{U_j;j\in J\}$ of $S^1$ such that $\lambda$ is the class of the
  cocycles $(F_l^{-1}F_j)$ and $(F_l'^{-1}F_j')$, where $F_j$ (resp.\ $F_j'$)
  are lifts of $\hat F$ (resp.\ $\hat F'$) on $U_j\in\cU$.
  From $[(F_l^{-1}F_j)]=[(F_l'^{-1}F_j')]$ follows that there exists a
  $0$-cochain $(G_j)_{j\in J}$ of the sheaf $\Lambda(A^0)$ relative to the
  covering $\cU$, such that
  \[
    F_l'^{-1}F_j'=G_lF_l^{-1}F_jG_j^{-1}
    \text{~on~the~arc~} U_j\cap U_l \,,
  \]
  which can be rewritten to
  \begin{equation}\label{eq:inProofOfTHM1Glue}
    F_j'G_jF_j^{-1} = F_l'G_lF_l^{-1} \text{~on~the~arc~} U_j\cap U_l \,.
  \end{equation}
  If we set $H_j:=F_j'G_{j}F_j^{-1}$ on $U_{j}$, we get
  \begin{itemize}
    \item that from equation (\ref{eq:inProofOfTHM1Glue}) that the $H_j$ glue
      together and yield a meromorphic $H$\PROBLEM[holomorphic??],
    \item that $H_j$ is a solution of $[A,A']$ on every $U_j$, i.e.\ it
      satisfies there ${}^{H_j}A=A'$, since
      \begin{align*}
        {}^{H_j}A &= {}^{F_j'G_{j}F_j^{-1}}A
        \\&={}^{F_j'G_{j}}A^0
                  & \text{(since $F_j'$ is a lift of $\hat F'$ on $U_j$)}
        \\&={}^{F_j'}A^0
                  & \text{(since $G_j$ is a is an isotropy of $A^0$)}
        \\&=A'
                  & \text{(since $F_j$ is a lift of $\hat F$ on $U_j$)}
      \end{align*}
      and
    \item which satisfies $\hat F'=\hat H_j\hat F$ on every $U_j$, since
      \begin{align*}
        \hat H_j\hat F&= \widehat{F_j'G_{j}F_j^{-1}} \hat F
        \\            &= \hat{F_j'}
        \underset{\id}{%
          \underset{\text{\rotatebox[origin=c]{-90}{$=$}}}{%
            \underbrace{%
              \hat{G_j}
            }
          }
        }
        \hat{F_j^{-1}} \hat F
                      & \text{(since $G_j$ is flat, i.e.\ $\hat G_j=\id$)}
        \\            &= \hat{F'}
        \underset{\id}{%
          \underset{\text{\rotatebox[origin=c]{-90}{$=$}}}{%
            \underbrace{%
              \hat{F^{-1}} \hat F
            }
          }
        }
        \\            &= \hat{F'}
      \end{align*}
  \end{itemize}
  Therefore are $(A,\hat F)$ and $(A,\hat F')$ equivalent
  (cf.\ page~\pageref{page:ofDefnOfIsomOfMarkedPairs}) and injectivity is
  proven.
  \iffalse
    \begin{comment}
      \textbf{First look at injectivity:}
      Consider the two elements $(\cM,\nabla,\hat f)$ and
      $(\cM',\nabla',\hat f')$ of $\cH(\cM^{nf},\nabla^{nf})$ which map to same
      cohomology class
      \[
        \exp([(\cM,\nabla,\hat f)])=\lambda=\exp([(\cM',\nabla',\hat f')])
          \in H^1(S^1;\Lambda(A^0)) \,.
      \]
      Since we can use refined coverings, it is possible to find a finite
      covering $\cU=\{U_j;j\in J\}$ of $S^1$ such that $\lambda$ is the class
      of the cocycles $(f_lf_j^{-1})$ and $(f_l',f_j'^{-1})$, where
      $f_j$,$f_j'$ are defined on $U_j$.
      Since $[(f_lf_j^{-1})]=[(f_l'f_j'^{-1})]$ there exists a $0$-cochain
      $(g_j)$ of the sheaf $\Aut^{<0}(\tilde\cM^{nf})$ relative to the covering
      $(I_j)$, such that
      \[
        f_l'f_j'^{-1}=g_lf_lf_j^{-1}g_j^{-1} \text{ on } I_j\cap I_l.
      \]
      If we set $\sigma=f_j^{-1}g_{j}^{-1}f_j'$ on $I_{j}$, we get a horizontal
      section\TODO[~on~???], thus\TODO[why?] it satisfies
      $\sigma\circ\hat{f'}=\hat f$. Therefore are $(\cM,\nabla,\hat f)$ and
      $(\cM',\nabla',\hat{f'})$ isomorphic and injectivity is proven.
    \end{comment}
  \fi
\end{proof}

For the proof of the surjectivity we will use another result from Malgrange and
Sibuya, which is also called the Malgrange-Sibuya Theorem
(Theorem~\ref{thm:thm1helpMalgSibuy}). It can for example be found in Babbitt
and Varadarajans's book \cite[65ff]{babbitt1989local} as Theorem 4.2.1.

Let $\hat F\in G(\!(t)\!)$ be a matrix with formally meromorphic entries. By
the Borel-Ritt Lemma (cf.\ Theorem~\ref{thm:borel-ritt}) we then know, that
there exists for every sector $\mathfrak{s}\subsetneq S^1$ a holomorphic
function $G:\mathfrak{s}\to\GL_n(\C)$ which is asymptotic to $\hat F$.
We will denote the set of all such holomorphic functions, which are on
the arc $I$ asymptotic to $\id\in G(\!(t)\!)$ by
\[
  \cG(I)=\left\{G\in\Gl_n(\cA(I))\mid g\sim_I\id\right\}\,,
\]
and this defines a sheaf $\cG$ on $S^1$.
The statement of the (second) Malgrange-Sibuya Theorem
(Theorem~\ref{thm:thm1helpMalgSibuy}) is then, that the \rewrite{difference}
between formal and konvergent invertible matrices is described by the first
sheaf cohomology $H^1(S^1;\cG)$ of $\cG$ via the map
\[
  \Theta: G(\!(t)\!)/G(\!\{t\}\!)\longrightarrow H^1(S^1;\cG) \,,
\]
which will turn out to be an isomorphism. It is set up as follows:
\begin{einr}
  Let $[\hat F]\in G(\!(t)\!)/G(\!\{t\}\!)$ with ambassador $\hat F$ and
  let $\cU=\{U_j\mid j\in J\}$ be a finite covering of $S^1$ by open arcs.
  The Borel-Ritt Lemma yields for every arc $j\in J\subsetneq S^1$ a
  holomorphic function $F_j$ which satisfies $F_j\sim_{U_j}\hat F$.
  By $(F_lF_j^{-1})_{j,l\in J}$ is then a cocycle for $\cG$ defined,
  and write $\Theta([\hat F])$ for the corresponding cohomology class.
\end{einr}
This construction is similar to the definition of the map of
Theorem~\ref{thm:mainThm1}. The difference is, that we instead of M.A.E.D, to
obtain lifts in the sense of Definition~\ref{defn:lift}, we use only the
Borel-Ritt Lemma to obtain only asymptotic lifts.

It can be verified, that the class $\Theta([\hat F])$ does not depend on
\begin{itemize}
  \item the choise of an ambassador $\hat F$ in
    $[\hat F]\in G(\!(t)\!)/G(\!\{t\}\!)$\PROBLEM[proof!],
  \item the choice of the covering $\cU$\PROBLEM[proof!] nor
  \item the choice the $F_j$\PROBLEM[proof!].
\end{itemize}
\begin{lem}
  \TODO[remove this lemma? is not needed/used]
  The mapping $\Theta$ is injective.
\end{lem}
\begin{proof}
  Let $\hat F$ and $\hat F'\in G(\!(t)\!)$ such that
  $\Theta([\hat F])=\Theta([\hat F'])$.
  We then can find a covering $\cU=\{U_j\mid j\in J\}$ together with
  holomorphic functions $F_j$ and $F_j'$, which satisfy
  $F_j\sim_{U_j}\hat F$ and $F_j'\sim_{U_j}\hat F'$, such that
  $(F_l^{-1}F_j)_{j,l\in J}$ and $(F_l'^{-1}F_j')_{j,l\in J}$ determine the
  classes $\Theta([\hat F])$ and $\Theta([\hat F'])$.
  This implies that there are maps $G_j$, which are on $U_j$ holomorphic and
  satisfy $G_j\sim_{U_j}\id$ such that
  \[
    F_l'^{-1}F_j'=G_lF_l^{-1}F_jG_j^{-1}
    \text{~on~the~arc~} U_j\cap U_l
  \]
  This equation can be rewritten to
  \[
    F_j'G_jF_j^{-1}=F_l'G_lF_l^{-1}
    \text{~on~the~arc~} U_j\cap U_l \,.
  \]
  Since this tells us, that the functions $F_j'G_jF_j^{-1}$ coincide on
  the overlapping and define a holomorphic map from the arc $S^1$ (i.e.\ a
  punctured disc with a small radius) into $\gl_n(\C)$, which will be called
  $G$.
  Since $F_j'G_jF_j^{-1}\sim_{U_j}\id$ for all $j\in J$, we have
  $G\sim_{S^1}\id$.
  Thus the defined $G$ meromorphic at $0$ and satisfies $G=F'^{-1}F$, so that
  $[F]=[F']$.
\end{proof}
\begin{thm}[Malgrange-Sibuya]\label{thm:thm1helpMalgSibuy}
  The map $\Theta:G(\!(t)\!)/G(\!\{t\}\!)\to H^1(S^1;\cG)$ is an isomorphism.
\end{thm}
This Theorem is proven in Section 4.4 of Babbitt and Varadarajan's book
\cite{babbitt1989local} or on page 371 of \cite{Martinet1991}.

We are now able to proof the surjectivity of the map from
Theorem~\ref{thm:mainThm1}.
\begin{proof}[Proof of surjectivity]
  \marginnote{\cite[72]{babbitt1989local}}
  Let the cohomology class $\lambda\in H^1(S^1;\Lambda(A^0))$ be represented by
  a cocycle $(F_{jl})_{j,l\in J}$ associated with some finite covering
  $\cU=\{U_j;j\in J\}$ of $S^1$. We especially know, that
  \begin{itemize}
    \item $F_{jl}$ is on $U_j\cap U_l$ asymptotic to $\id$ and
    \item it is a isotropy, i.e.\  ${}^{F_{jl}}A^0=A^0$.
  \end{itemize}
  The cocycle $(F_{jl})_{j,l\in J}$ also determines an element in
  $\sigma\in H^1(S^1;\cG)$.
  From the Theorem~\ref{thm:thm1helpMalgSibuy} we know, that there is a $\hat
  F\in G\llbracket t\rrbracket\subset G(\!(t)\!)$ whose class $[\hat F]$ gets
  via $\Theta$ mapped to $\sigma$.
  Thus there exists holomorphic functions $F_j:\mathfrak{s}_{U_j}\to\gl_n(\C)$
  with $F_j\sim_{U_j}\hat F$ and $F_l^{-1}F_j=F_{jl}$ on
  $\mathfrak{s}_{U_j\cap U_l}$ for all $j,l\in J$.

  Define on every arc $U_j$ the matrix $A_j:={}^{F_j}A^0$.
  On the intersections $U_j\cap U_l$ we know that $A_j=A_l$, since
  from $F_l^{-1}F_j\in\Lambda(A^0)$ follows on $U_j\cap U_l$ that
  \[
    A^0={}^{F_l^{-1}}({}^{F_j}A^0)
    \qquad\Longrightarrow{}\qquad
    \underset{=A_l}{\underbrace{{}^{F_l}A^0}}
    =\underset{=A_j}{\underbrace{{}^{F_j}A^0}}
    \,.
  \]
  Thus the $A_j$ \rewrite{glue to a} section $A$, which satisfies
  ${}^{\hat F}A=A_0$ by construction.
  We have found an element $(A,\hat F)\in\cH(A^0)$ whose image under $\exp$ is
  $\sigma$.
\end{proof}

\iffalse
\begin{comment}
  \TODO[Define $\tilde\cM$ or find other way!]
  \begin{proof}[Proof of surjectivity]
    \begin{multicols}{2}
      % \textbf{Now look at surjectivity:}
      Similar to the proof, given by Sabbah \marginnote{Which refers to
      \cite{Malgrange1983}} in \cite{sabbah2007isomonodromic}, we will start this
      part of the proof by giving a necessary and sufficient condition for a
      class $\lambda$ in $H^1(S^1;\Lambda(A^0))$
      \TODO[$\Lambda(A^0)\sim\Aut^{<0}(\tilde\cM^{nf})$ or $\Aut^{<0}(\cM^{nf})$?]
      to come from an object
      $(\cM,\nabla,\hat f)\in\cH(\cM^{nf},\nabla^{nf})$:
      \begin{einr}
        This is the case if and only if the image of
        $\lambda\in\Aut^{<0}(\cM^{nf})$ in the set
        $H^1(S^1;\Aut_{\cA}(\tilde\cM^{nf}))$ (where $\Aut_{\cA}(\tilde\cM^{nf})$
        contains only the $\cA$-linear automorphisms) is the identity.
      \end{einr}

    \columnbreak

      \textcolor{gray}{%
        \textbf{Now look at surjectivity:}
        Similar to the proof, given by Sabbah \marginnote{Which refers to
        \cite{Malgrange1983}} in \cite{sabbah2007isomonodromic}, we will start
        this part of the proof by giving a necessary and sufficient condition
        for a class $\lambda$ in $H^1(S^1;\Lambda(A^0))$
        to come from an object
        $(A,\hat F)\in\cH(A^0)$:
        \begin{einr}
          \rewrite{This is the case if and only if the image of
            $\lambda\in\Aut^{<0}(\cM^{nf})$ in the set
            $H^1(S^1;\Aut_{\cA}(\tilde\cM^{nf}))$ (where
            $\Aut_{\cA}(\tilde\cM^{nf})$ contains only the $\cA$-linear
            automorphisms) is the identity.}
        \end{einr}
      }
    \end{multicols}
    \begin{proof}
      \textbf{``\Rightarrow{}'':}
      If $\lambda$ is the image of some $(\cM,\nabla,\hat f)$ then there exists
      \begin{itemize}
        \item a covering $(I_{j})$ of $S^1$ and
        \item isomorphisms $f_j:\tilde\cM\overset{\sim}\to\tilde\cM^{nf}$
          inducing $\hat f$
      \end{itemize}
      such that $\lambda$ comes from a cocycle $(\lambda_{j,l})=(f_lf_j^{-1})$
      on $I_j\cap I_l$.
      \TODO{} shows that $(\lambda_{jl})$ is a coboundary of
      $\Aut_\cA(\tilde\cM^{nf})$.

      \textbf{``\Leftarrow{}'':}
      If for some suitable covering $(I_j)$ the cocycle $(\lambda_{jl})$ is a
      coboundary with values in $\Aut_\cA(\tilde\cM^{nf})$, i.e.\
      $\lambda_{jl}=f_lf_j^{-1}$, we define a new connection $\nabla$ on
      $\tilde\cM^{nf}$ by conjugating $\nabla^{nf}$ by $f_j$ on $U_j$.

      \TODO{}

      Moreover $\hat f_j=\hat f_l$ on $U_j\cap U_l$, so that the formal
      isomorphisms
      \[
        \hat f_j:(\hat \cM^{nf},\nabla)
        \overset{\sim}{\longrightarrow}
        (\hat\cM^{nf},\nabla^{nf})
      \]
      can be glued in an isomorphism $\hat f:(\hat \cM^{nf},\nabla)
      \overset{\sim}{\longrightarrow}(\hat\cM^{nf},\nabla^{nf})$.
    \end{proof}

    Thus, the proof of Theorem~\ref{thm:mainThm1} is a consequence of the
    following Theorem by Malgrange and Sibuya.
    \begin{thm}[Malgrange-Sibuya]\label{thm:malgSibuyaHelp}
      \marginnote{\cite[Thm.II.6.10]{sabbah2007isomonodromic}}
      The image of the mapping
      \[
        H^1(S^1;\Gl_d^{<0}(\cA_{\tilde D}))
        \to
        H^1(S^1;\Gl_d(\cA_{\tilde D}))
      \]
      is the identity.
    \end{thm}
    For the proof of Theorem~\ref{thm:malgSibuyaHelp} which we refer to
    \cite[Th.A.1]{Malgrange1983}, \cite[Th.6.4.1]{sibuya1990Linear} and
    \cite{babbitt1989local}.
  \end{proof}
\end{comment}
\fi
