\chapter{Stokes Structures}\label{chap:stokes}
\begin{comment}
  \begin{itemize}
  \item
  Characterized completely
  \end{itemize}
\end{comment}
Let $(\cM^{nf},\nabla^{nf})$ be a fixed model with the corresponding normal
form $A^0$ and let us also fix a normal solution $\cY_0$ of $A^0$.
Here we want to introduce the Stokes structures, which will characterize the
isomorphism classes of meromorphic connections uniquely, i.e.\ an space which is
isomorphic to the set $\cH(A^0)$ of isomorphism classes of marked pairs.
A great overview of this topic is given by Varadarajan
in~\cite{Varadarajan96linearmeromorphic}.

One of the most important theorems, which is fundamental for the whole chapter,
is the Malgrange-Sibuya Theorem.
It states that the classifying set $\cH(A^0)$ is via an map $\exp$ isomorphic
to the first non abelian cohomology space $H^1(S^1;\Lambda(A^0))=:\St(A^0)$ of
the Stokes sheaf $\Lambda(A^0)$ and will be proven in the first section.
In Section~\ref{sec:mainThm2} we will improve the Malgrange-Sibuya Theorem by
showing that each $1$-cohomology class in $\St(A^0)$ contains a unique
$1$-cocycle of a special form called \emph{the Stokes cocycle}.
We will further show that such cocycles can be identified with their germs at
some special directions, i.e.\ anti-Stokes directions. These germs are called
Stokes germs and for an anti-Stokes direction $\alpha$ do these germs form the
Stokes groups $\Sto_\alpha(A^0)$ (cf.\ Section~\ref{sec:StokesGroup}).
The morphism, which maps each product of Stokes germs to its corresponding
$1$-cocycle will be denoted by $h$.
This will be further improved in Section~\ref{sec:furtherImprovements}, where we
will collect multiple Stokes germs to their product to obtain a more robust
version of the Stokes space.

If one introduces the map $g$, which arises from the theory of
summation\footnote{We will not use the theory of summation and only think of it
  as a black box, although we will roughly discuss the theory of summation in
  Appendix~\ref{app:multisummability}.} and takes an equivalence
class\footnote{resp.\ an ambassador of such a class.} and returns the
corresponding Stokes cocycle in an canonically way one obtains the following
commutative diagram.
\begin{center}
  \begin{tikzpicture}[scale=3]
    % \node[] (modSpcMat) at (0,0.4) {$\cH(\cM^{nf},\nabla^{nf})$};
    \node[] (mat) at (0,-.8) {$\prod_{\theta\in\A}\Sto_\theta(A^0)$};
    \node[] (class) at (0,0) {$\cH(A^0)$};
    % \node[green!40!black] (sheaf) at (3,0.4) {$\St(\cM^{nf})$};
    \node[] (sheaf3) at (1.3,0) {$\St(A^0)$};

    % \draw[thick,double,blue] (modSpcMat) -- (class);
    % \draw[purple,thick,double] (sheaf) -- (sheaf3);

    % \draw[->,green!40!black] (modSpcMat) -- (sheaf)
    %   node[midway,above] {$\exp$};
    \draw[->] (class) -- (mat) node[midway,left] {g};
    \draw[->] (class) -- (sheaf3) node[midway,above] {$\exp$};
    \draw[->] (mat) -- (sheaf3) node[midway, below right] {$h$};
  \end{tikzpicture}
\end{center}\label{page:ofPreDiagram}
This diagram will be enhanced in Section~\ref{sec:theCompleteDiagram} by
adding a couple of isomorphisms.

In the first section of this chapter we will use
\rewrite{Sabbah's} book~\cite[section II]{sabbah2007isomonodromic} as its main
resource together with \cite{babbitt1989local} for the main proof.
In the Sections~\ref{sec:StokesGroup} and~\ref{sec:mainThm2} will
\rewrite{Loday-Richaud's} paper~\cite{Loday1994} and her book
\cite[Sec.4]{Loday2014} be useful.
Stokes groups are also discussed \rewrite{Boalch's} paper~\cite{boalch} (resp.\
his thesis~\cite{thboalch}) which looks only at the single leveled case or the
paper~\cite{Martinet1991} from Martinet and Ramis.
Another very important paper is~\cite{BJL1979Birkhoff} from Balser, Jurkat and
Lutz.

%%% Local Variables:
%%% TeX-master: "Maximilian_Huber-Masters_Thesis-with_notes.tex"
%%% End:
