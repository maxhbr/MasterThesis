\section{Meromorphe Zusammenhänge vs. Systeme}
Sei $\Sigma$ eine Riemann Fläche, $D=k_1(a_1)+\dots+k_m(a_m)$ ein effektiver
Divisor darauf.
\begin{paracol}{2}
\begin{defn}[Meromorpher Zusammenhang]
Ein meromorpher Zusammenhang $\nabla$ auf $V$ mit Polen auf $D$ ist eine
Abbildung
\[
  \nabla: V\to V\otimes K(D)
\]
so dass die Leibnitzregel
\begin{equation}
  \nabla(fv)=(df)\otimes v + f\nabla v
\end{equation}
erfüllt ist. Hierbei ist $v$ ein lokaler Schnitt von $V$, $f$ eine lokale
holomorphe Funktion und $K$ die Garbe der holomorphen Eins-Formen.
\end{defn}
\switchcolumn %%%%%%%%%%%%%%%%%%%%%%%%%%%%%%%%%%%%%%%%%%%%%%%%%%%%%%%%%%%%%%%%%
\begin{defn}
Ein \emph{System} ist \dots
\end{defn}
Dieses wird erhalten durch lokale Trivialisierung.
\end{paracol}

\begin{paracol}{2}
\begin{defn}[compatible framing]
\end{defn}
\switchcolumn %%%%%%%%%%%%%%%%%%%%%%%%%%%%%%%%%%%%%%%%%%%%%%%%%%%%%%%%%%%%%%%%%
\begin{defn}[marked pair]
\end{defn}
\end{paracol}
