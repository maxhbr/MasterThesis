\chapter{Stokes Structures}\label{chap:stokes}
Stokes structures are contain exactly \rewrite{necessary information} to
classify meromorphic classification, i.e.\ with the Stokes structures we are
able to construct a \rewrite{space}\PROBLEM[Why space], which is isomorphic to
the classifying \rewrite{set}.

A great overview of this topic is given by Varadarajan in
\cite{Varadarajan96linearmeromorphic}. Other resources we will use are for
example \rewrite{Sabbah's} book~\cite[section II]{sabbah2007isomonodromic} for
Section~\ref{sec:mainThm1}.
For the Sections~\ref{sec:StokesGroup} and~\ref{sec:mainThm2} will
\rewrite{Loday-Richaud's} paper~\cite{Loday1994} and the book
\cite[Sec.4]{Loday2014} be \rewrite{useful. Also useful} was \rewrite{Boalch's}
paper~\cite{boalch} (resp.\ his thesis~\cite{thboalch}) which looks only at the
single leveled case or the paper~\cite[Thm.13]{Martinet1991} from Martinet and
Ramis.

Let $(\cM^{nf},\nabla^{nf})$ be a fixed model with the corresponding normal
form $A^0$.
Let us also fix a normal solution $\cY_0$ of $A^0$.
The purpose of the next section (Section~\ref{sec:mainThm1}) is, to proof the
Malgrange-Sibuya Theorem.
It states that the classifying set $\cH(A^0)$ is via an map $\exp$ isomorphic
to the first non abelian cohomology $H^1(S^1;\Lambda(A^0))$ of the Stokes sheaf
$\Lambda(A^0)$. It will be denoted by $\St(A^0)$.
In Section~\ref{sec:mainThm2} we will improve the Malgrange-Sibuya Theorem by
showing that each 1-cohomology class in $\St(A^0)$ contains a unique
$1$-cocycle of a special form called \emph{the Stokes cocycle}
(cf.\ Section~\ref{sec:StokesGroup}).
The morphism, which maps each Stokes cocycle to its corresponding $1$-cocycle
will be denoted by $h$.
This will be further improved in Section~\ref{sec:furtherImprovements}.

If one introduces the map $g$, which arises from the theory of summation and
takes an equivalence class (resp.\ an ambassador of such a class) and returns a
corresponding Stokes cocycle in an canonically way
(cf.\ Appendix~\ref{app:multisummability} where the theory of summation  will
be roughly discussed), as a black-box one can write the following commutative
diagram.
\begin{center}
  \begin{tikzpicture}[scale=3]
    % \node[] (modSpcMat) at (0,0.4) {$\cH(\cM^{nf},\nabla^{nf})$};
    \node[] (mat) at (0,-.8) {$\prod_{\theta\in\A}\Sto_\theta(A^0)$};
    \node[] (class) at (0,0) {$\cH(A^0)$};
    % \node[green!40!black] (sheaf) at (3,0.4) {$\St(\cM^{nf})$};
    \node[] (sheaf3) at (1.3,0) {$\St(A^0)$};

    % \draw[thick,double,blue] (modSpcMat) -- (class);
    % \draw[purple,thick,double] (sheaf) -- (sheaf3);

    % \draw[->,green!40!black] (modSpcMat) -- (sheaf)
    %   node[midway,above] {$\exp$};
    \draw[->] (class) -- (mat) node[midway,left] {g};
    \draw[->] (class) -- (sheaf3) node[midway,above] {$\exp$};
    \draw[->] (mat) -- (sheaf3) node[midway, below right] {$h$};
  \end{tikzpicture}
\end{center}\label{page:ofPreDiagram}
This diagram will be enhanced in Section~\ref{sec:theCompleteDiagram} by
adding a couple of isomorphisms.
