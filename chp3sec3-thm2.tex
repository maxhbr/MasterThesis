%%%%%%%%%%%%%%%%%%%%%%%%%%%%%%%%%%%%%%%%%%%%%%%%%%%%%%%%%%%%%%%%%%%%%%%%%%%%%%%
\section{Stokes structures: using Stokes groups}\label{sec:mainThm2}
\marginnote{\cite{Loday1994},~\cite[Thm.4.3.11]{Loday2014}
  \\and~\cite{boalch,thboalch}
  \\and~\cite{babbitt1989local}
  \\and~\cite{BJL1979Birkhoff}
  \\and~\cite[Chapter 4]{Martinet1991}}

The goal in this section is to prove that there is a bijective and natural map
\[
  h:\prod_{\alpha\in\A}\Sto_\alpha(A^0)\longrightarrow\St(A^0)
\]
which endows $\St(A^0)$ with the structure of a unipotent Lie group.
And since $\Sto_\alpha(A^0)$ has $\SSto_\alpha(A^0)$ as a faithful
representation, we also get the isomorphism
$\prod_{\alpha\in\A}\SSto_\alpha(A^0)\cong\St(A^0)$ as a corollary.
\TODO[This goes back to \cite{BJL1979Birkhoff}?]

Let us recall, that $\St(A^0)$ is defined to be $H^1(S^1;\Lambda(A^0))$
(cf.\ Section~\ref{sec:mainThm1}).
The elements of $\prod_{\alpha\in\A}\Sto_\alpha(A^0)$ define in a canonical way
cocycles of the sheaf $\Lambda(A^0)$ (cf.\ Equation (\ref{eq:mapStoToCocy})),
called Stokes cocycles (cf.\ Definition~\ref{defn:stokesCocycle}).
In fact, will $h$ map such cocycles to the cohomology class, to which they
correspond.
Thus the statement, that $h$ is a bijection, is equivalent to the statement
that
\begin{einr}
  in each cohomology class of $\St(A^0)$ is a unique $1$-cocycle, which is a
  Stokes cocycle.
\end{einr}

\vspace{-2pt}%%%%%%%%%%%%%%%%%%%%%%%%%%%%%%%%%%%%%%%%%%%%%%%%%%%%%%%%%%%%%%%%%%
%%%%%%%%%%%%%%%%%%%%%%%%%%%%%%%%%%%%%%%%%%%%%%%%%%%%%%%%%%%%%%%%%%%%%%%%%%%%%%%
\subsubsection{Cyclic coverings}
To formulate the theorem in the next section, we use the notion of cyclic
coverings and nerves of such coverings, which are defined as follows.
\begin{defn}
  \marginnote{\cite[Sec.II.1]{Loday1994} and \cite[Sec.II.3.1]{Loday1994}}
  Let $J$ be a finite set, identified to $\{1,\dots,p\}\subset\Z$.
  \begin{enumerate}
    \item A \emph{cyclic covering} of $S^1$ is a finite covering
      $\cU=\big(U_j:=U(\theta_j,\epsilon_j)\big)_{j\in J}$ consisting of
      arcs, which satisfies that
      \begin{enumerate}
        \item $\tilde\theta_j \geq \tilde\theta_{j+1}$ for
          $j\in\{1,\dots,p-1\}$, i.e.\ the center points are ordered in
          ascending order with respect to the clockwise orientation of $S^1$ and
        \item $\tilde\theta_j+\frac{\epsilon_j}{2}\geq
          \tilde\theta_{j+1}+\frac{\epsilon_{j+1}}{2}$ for
          $j\in\{1,\dots,p-1\}$ and
          $\tilde\theta_p+\frac{\epsilon_p}{2}\geq
          \tilde\theta_{1}-2\pi+\frac{\epsilon_{1}}{2}$, i.e.\ the arcs are not
          encased by another arc,
      \end{enumerate}
      where the $\tilde\theta_j\in [0,2\pi[$ are determinations of the
      $\theta_j\in S^1$.
      \begin{comment}
        \begin{enumerate}
          \item the $\theta_j$ are in ascending order with respect to the
            clockwise orientation of $S^1$;
          \item the $U_j\cap U_{j+1}$ have only one connected component when
            $\#J>2$;
          \item the $U_j$ are not encased by another arc, this means that the
            open sets $U_j\backslash U_l$ are connected for all $j,l\in J$.
        \end{enumerate}
      \end{comment}
    \item The \emph{nerve} of a cyclic covering $\cU=\{U_j;j\in J\}$ is the
      family $\dot\cU=\{\dot U_j;j\in J\}$ defined by:
      \begin{itemize}
        \item $\dot U_j=U_j\cap U_{j+1}$ when $\#J>2$,
        \item $\dot U_1$ and $\dot U_2$ the connected components of
          $U_1\cap U_2$ when $\#J=2$.
      \end{itemize}
      \begin{s-rem}
        The nerve of the cyclic covering
        $\cU=\left(U(\theta_j,\epsilon_j)\right)_{j\in J}$ is explicitly given
        by
        \[
          \dot\cU=\left(
            \bigl(\theta_{j}-\frac{\epsilon_{j}}{2},
            \theta_{j+1}+\frac{\epsilon_{j+1}}{2}\bigr)
          \right)_{j\in J} \,.
        \]
      \end{s-rem}
  \end{enumerate}
\end{defn}
The cyclic coverings correspond one-to-one to nerves of cyclic coverings. If
one starts with a nerve $\{\dot U_j \mid j\in J\}$, one obtains a cyclic
covering as $\cU=\{U_j \mid j\in J\}$ where the arc $U_j$ are the connected
clockwise hulls from $\dot U_{j-1}$ to $\dot U_j$.

\begin{defn}
  A covering $\cV$ is said to \emph{refine} a covering $\cU$ if, to each open
  set $V\in\cV$ there is at least one $U\in\cU$ with $V\subset U$.
\end{defn}
Each refined covering of $\cU$ is obtained by successively
\begin{enumerate}
\item narrowing an arc $U\in\cU$ to a smaller arc $\tilde U\subset U$ or
\item splitting an arc $U\in\cU$ into two smaller arcs $U'$ and $U''$
  satisfying $U=U'\cup U''$.
\end{enumerate}
This can be used to see the following proposition
(cf.\ \cite[Prop.II.1.3]{Loday1994}).
\begin{prop}
  The covering $\cV$ refines $\cU$ if and only if the corresponding nerves
  $\dot\cU=\{\dot U_j\}$ and $\dot\cV=\{\dot V_l\}$ satisfy
  \begin{einr}
    each $\dot U_j$ contains at least one $\dot V_l$.
  \end{einr}
\end{prop}

The cyclic coverings and especially the nerves of such coverings will be
useful, since we have the following proposition
(cf.~\cite[Prop.2.6]{Loday2004}).
\begin{prop}\label{prop:cocycleIsomNerve}
  The set of $1$-cocycles of $\cU$ is canonically isomorphic to the set of
  $1$-cochains restricted to $\dot\cU=\bigl(\dot U_j\bigr)_{j\in J}$ without any
  cocycle condition, i.e.\ the product
  $\prod_{j\in J}\Gamma(\dot U_j,\Lambda(A^0))$.
\end{prop}

%%%%%%%%%%%%%%%%%%%%%%%%%%%%%%%%%%%%%%%%%%%%%%%%%%%%%%%%%%%%%%%%%%%%%%%%%%%%%%%
\subsection{The theorem}
\marginnote{\cite[868]{Loday1994}}
Let $\{\theta_j\mid j\in J\}\subset S^1$ be a finite set and
$\dot\phi=(\dot\phi_{\theta_j})_{j\in J}
\in\prod_{j\in J}\Lambda_{\theta_j}(A^0)$ be a finite family of germs.
Let $\dot\phi_j$ be the function representing the germ $\dot\phi_{\theta_j}$
on its (maximal) arc of definition $\Omega_j$ around $\theta_j$.
In the following way, one can associate a cohomology class in $\St(A^0)$ to
$\dot\phi$:
\begin{einr}
  for every cyclic covering $\cU=(U_j)_{j\in J}$, which satisfies
  $\dot U_j\subset\Omega_j$ for all $j\in J$, one can define the $1$-cocycle
  $(\dot\phi_{j|\dot U_j})_{j\in J}\in\Gamma(\dot\cU;\Lambda(A^0))$.
\end{einr}
To a different cyclic covering, satisfying the condition above, this
construction yields a cohomologous $1$-cocycle, thus the induced map
\begin{equation}\label{eq:mapStoToCocy}
  \prod_{j\in J}\Lambda_{\theta_j}(A^0)
  \longrightarrow
  H^1(S^1;\Lambda(A^0))=\St(A^0)
\end{equation}
is welldefined (cf.\ \cite[868]{Loday1994}).

Recall that $\nu=\#\A$ is the number of all anti-Stokes directions and the set
of all anti-Stokes directions is denoted by $\A=\{\alpha_1,\dots,\alpha_\nu\}$.
\begin{defn}\label{defn:stokesCocycle}
  \marginnote{\cite[Def.II.1.8]{Loday1994}
    \\, \cite[4.3.10]{Loday2014}
    \\, \cite[Defn 6 on p 374]{Martinet1991}}
  A \emph{Stokes cocycle} is a 1-cocycle $(\phi_j)_{j\in\{1,\dots,\nu\}}\in
  \prod_{j\in\{1,\dots,\nu\}}\Gamma(U_j;\Lambda(A^0))$ corresponding to some
  cyclic covering with nerve $\dot\cU=(\dot U_j)_{j\in\{1,\dots,\nu\}}$, which
  satisfies for every $j\in\{1,\dots,\nu\}$
  \begin{itemize}
  \item $\alpha_j\in\dot U_j$ and
  \item the germ $\phi_{\alpha_j}:=\phi_{j,\alpha_j}$ of $\phi_j$ at $\alpha_j$
    is an element of $\Sto_{\alpha_j}(A^0)$.
  \end{itemize}
  \begin{comment}
    \begin{s-rem}\label{rem:inclusionGermRemark}
      \BIGPROBLEM[refactor!remove?]  The sections
      $\Gamma(\dot U_j;\Lambda(A^0))$ are uniquely determined as the extension
      of the germ at $\alpha_j$, since the sheaf $\Lambda(A^0)$ defined via the
      system $[A^0,A^0]$ (cf.\ Definition~\ref{defn:StokesSheaf}).  We thus have
      an injective map
      \[
        \prod_{j\in\{1,\dots,\nu\}}\Gamma(\dot U_j;\Lambda(A^0))
        \hookrightarrow
        \prod_{j\in\{1,\dots,\nu\}}\Sto_{\alpha_j}(A^0) \,,
      \]
      which takes an Stokes cocycle and yields the corresponding Stokes germs.
      For a fine enough covering $\cU$, i.e.\ a covering $\cU$ with a nerve
      $\dot\cU$ which consists of small enough arcs satisfying the conditions
      above, is this map a bijection.

      We will use this fact implicitly and assume that the covering is always
      fine enough to call elements of $\prod_{\alpha\in\A}\Sto_\alpha(A^0)$
      Stokes cocycles.
    \end{s-rem}
  \end{comment}
\end{defn}
We can use the mapping (\ref{eq:mapStoToCocy}), corresponding to the construction
at the beginning of this section, to obtain a mapping
\[ \begin{tikzcd}
  h:\prod_{\alpha\in\A}\Sto_\alpha(A^0)
  \rar[hook]&
  \prod_{\alpha\in\A}\Lambda_\alpha(A^0)
  \rar{\text{(\ref{eq:mapStoToCocy})}}&
  \St(A^0),
\end{tikzcd} \]
which takes a complete set of Stokes germs to its corresponding cohomology
class, given by a Stokes cocycle.
\begin{center}
  \begin{minipage}[t]{0.8\textwidth}
    \begin{tthm}\label{thm:mainThm2}
      The map
      \[ \begin{tikzcd}
          h:\prod_{\alpha\in\A}\Sto_\alpha(A^0) \rar& \St(A^0)
      \end{tikzcd} \]
      is a bijection and natural.
      \begin{s-rem}
        \marginnote{\cite[869]{Loday1994},\cite[Sec.III.3.3]{Loday1994}}
        Natural means that $h$ commutes to isomorphisms and constructions over
        systems or connections they represent.
      \end{s-rem}
    \end{tthm}
  \end{minipage}
\end{center}
To define the inverse map of $h$, one has to find in each cocycle in $\St(A^0)$
the Stokes cocycle and take the germs. Loday-Richaud gives an algorithm in
Section II.3.4 of her paper~\cite{Loday1994}, which takes a cocycle over an
arbitrary cyclic covering and outputs cohomologous Stokes cocycle and thus
solves this problem. This means, that the inverse of $h$ is constructible.

\begin{cor}
  \PROBLEM[mentioned twice]
  \BIGPROBLEM[remove?]
  Using the isomorphisms $\Sto_\theta(A^0)\cong\SSto_\theta(A^0)$ from
  Proposition~\ref{prop:representation} we obtain
  \[
    \St(A^0)\cong\prod_{\alpha\in\A}\SSto_\alpha(A^0) \,.
  \]
\end{cor}
Since the product of Lie groups is in an obvious way again a Lie group, this
endows $\St(A^0)$ with the structure of an unipotent Lie group with the
finite complex dimension $N:=\dim_\C\St(A^0)$
(cf.~\cite[Sec.III.1]{Loday1994}).
\begin{rem}
  This number $N$ is known to be the \emph{irregularity} of $[\End A^0]$ and
  can be rewritten in the following way:
  \[
    N=\sum_{\alpha\in\A}\dim_\C\SSto_\alpha(A^0)
     =\sum_{\alpha\in\A}\sum_{q_j\myrel{\alpha}q_l}n_j\cdot n_l
     =\sum_{\substack{1\leq j,l\leq n\\j<l}}2\cdot\deg(q_j-q_l)\cdot
       n_j\cdot n_l \,.
  \]
\end{rem}

One can also define the structure of a linear affine variety on the set
$\St(A^0)$. This was for example done in
Section~\cite[Sec.II.3]{babbitt1989local} or in~\cite[Sec.III.1]{Loday1994}, where
actually multiple structures of linear affine varieties on $\St(A^0)$ are
defined. In~\cite[35ff]{Varadarajan96linearmeromorphic} is mentioned that one
can also define a scheme structure on $\St(A^0)$.

%%%%%%%%%%%%%%%%%%%%%%%%%%%%%%%%%%%%%%%%%%%%%%%%%%%%%%%%%%%%%%%%%%%%%%%%%%%%%%%
\subsection{Proof of Theorem~\ref{thm:mainThm2}}\label{sec:proofOfMatrixThm}
We will only look at the unramified case, for which we refer
to~\cite[Sec.II.3]{Loday1994}.
The proof in the ramified case can be found in Section II.4 of Loday-Richaud's
Paper~\cite{Loday1994} and a sketch of the complete proof is also in
\cite{Loday2004}.
We first have to introduce the notion of adequate coverings, which will be used
in the proof.

%%%%%%%%%%%%%%%%%%%%%%%%%%%%%%%%%%%%%%%%%%%%%%%%%%%%%%%%%%%%%%%%%%%%%%%%%%%%%%%
\subsubsection{Adequate coverings}
\TODO[Adequate is acyclic in Loday2004?]
\begin{comment}
  \begin{enumerate}
  \item \cite[371]{Martinet1991} defines adapted coverings
  \item \cite[5269]{Loday2004} defines acyclic coverings
    \begin{itemize}
    \item uses the \textbf{theorem of Leray}
      \url{https://de.wikipedia.org/wiki/Satz_von_Leray}
    \end{itemize}
  \end{enumerate}
\end{comment}
\begin{defn}
  Let $\star\in\{k,<k,\leq k,\dots\}$.
  A covering $\cU$ beyond which the inductive limit
  \[
    \underset{\cU}{\underrightarrow{\lim}}H^1(\cU;\Lambda^\star(A^0))
  \]
  is stationary is said to be \emph{adequate} to describe
  $H^1(S^1;\Lambda^\star(A^0))$ or \emph{adequate} to $\Lambda^\star(A^0)$.
  \begin{comment}
    A covering $\cU$ is said to be \emph{adequate} to describe
    $H^1(S^1;\Lambda^\star(A^0))$ or \emph{adequate} to $\Lambda^\star(A^0)$ if
    for every element in
    $\underset{\cU}{\underrightarrow{\lim}}H^1(\cU;\Lambda^\star(A^0))$
    given by some covering $\cU'$ and an element of
    $\Gamma(\cU';\Lambda^\star(A^0))$
    there exists
    \begin{itemize}
      \item an element in $\Gamma(\cU;\Lambda^\star(A^0))$ and
      \item an common refinement of $\cU$ and $\cU'$
    \end{itemize}
    such that \PROBLEM[the elements are~?? on the refined covering.]
  \end{comment}

  In other words is a covering $\cU$ adequate to $\Lambda^\star(A^0)$, if and
  only if the quotient map
  \[
      \Gamma(\cU;\Lambda^\star(A^0))\longrightarrow H^1(S^1;\Lambda^\star(A^0))
  \]
  is surjective.\TODO[Proof? check??]
  \begin{comment}
    \cite[371]{Martinet1991} introduces the following definition
    \begin{s-defn}
      A covering $\cU$ is \emph{adapted} if every anti-Stokes direction is
      contained in exactly one element of the nerve $\dot\cU$.
    \end{s-defn}
  \end{comment}
\end{defn}
Adequate coverings are sometimes called acyclic, for example
in~\cite{Loday2004}, or adapted, for example in~\cite{Martinet1991}.

The following proposition is in Loday-Richaud's paper~\cite{Loday1994} given as
Proposition II.1.7. It contains a simple characterization, which will be used
to see, that our defined coverings are adequate.
\begin{prop}\label{prop:adeqCovCondition}
  \marginnote{\cite[Prop.II.1.7]{Loday1994}}
  Let $k\in\cK_\alpha$.
  \begin{s-defn}
    Let $\alpha\in\A^k$.
    An arc $U\bigl(\alpha,\frac{\pi}{k}\bigr)$ is called a \emph{Stokes arc of
      level $k$ at $\alpha$}.

    If $(q_j,q_l)\in\cQ(A^0)$ is a pair, such that $q_j\myrel{\alpha}q_l$, is
    the arc $U\bigl(\alpha,\frac{\pi}{k}\bigr)$ exactly the arc of decay of the
    exponential $e^{q_j-q_l}$ (cf.~\cite[5269]{Loday2004} and
    Remark~\ref{rem:relationDistanceCondition}).
  \end{s-defn}
  A cyclic covering $\cU=(U_j)_{j\in J}$ which satisfies
  \begin{einr}
    for every $\alpha\in\A^k$ contains the Stokes arc
    $U\bigl(\alpha,\frac{\pi}{k}\bigr)$ at least one arc $\dot U_j$ from the
    nerve $\dot\cU$ of $\cU$
  \end{einr}
  is adequate to $\Lambda^k(A^0)$.

  The covering $\cU$ is adequate to $\Lambda^{\leq k}(A^0)$ (resp.\
  $\Lambda^{<k}(A^0)$) if it is adequate to $\Lambda^{k'}(A^0)$ for every
  $k'\leq k$ (resp.\ $k'<k$).
\end{prop}
\begin{comment}
  \begin{proof}
    Show that for every $U_j,U_l\in\cU$ is
    \[
      H^1(U_j\cap U_l;\Lambda^k(A^0))=0 \,.
    \]
    Then the theorem of Leray implies that
    $H^1(S^0;\Lambda^k(A^0))=H^1(\cU;\Lambda^k(A^0))$
  \end{proof}
\end{comment}

Let $k\in\cK$.
We want to define the three cyclic coverings $\cU^{k}$, $\cU^{\leq k}$ and
$\cU^{<k}$ which will be adequate to $\Lambda^k(A^0)$, $\Lambda^{\leq k}(A^0)$
and $\Lambda^{<k}(A^0)$,respectively . Furthermore will the coverings be
comparable at the different levels.

\begin{itemize}
\item[\textbf{1.}] The first covering
  $\cU^{k}=\{\dot U_\alpha^k\mid\alpha\in\A^k\}$ is the cyclic covering
  determined by the nerve
  \[
    \dot\cU^k:=
    \bigl\{\dot U_\alpha^k=U\bigl(\alpha,\frac{\pi}{k}\bigr)\mid\alpha\in\A^k\bigr\}
  \]
  consisting of all Stokes arcs of level $k$ for anti-Stokes directions bearing
  the level $k$.
\end{itemize}
Since $\dot\cU^k$ consists only of arcs with equal opening is this canonically a
nerve.

\begin{rem}\label{rem:superSectors}
  Boalch, who looks only at the single-leveled case, introduces in his
  publications~\cite[19]{boalch} and~\cite[Def.1.23]{thboalch} the notion of
  supersectors which are defined as follows:
  \begin{einr}
    write the anti-Stokes directions as $\A=\{\alpha_1,\dots,\alpha_\nu\}$
    arranged according to the clockwise ordering, then is the $i$-th
    \emph{supersector} defined as the arc
    \[
      \hat\Sect^k_i:=
        \left(\alpha_{i+1}-\frac{\pi}{2k},\alpha_{i}+\frac{\pi}{2k}\right) \,.
    \]
  \end{einr}
  This yields a cyclic covering $(\hat\Sect^k_i)_{i\in\{1,\dots,\nu\}}$ whose
  nerve is exactly $\dot\cU^k$, which was defined above.
\end{rem}
  \begin{figure}[h]
    \begin{center}
      \begin{tikzpicture}[scale=3.5]
        \clip (-1.1,-0.2) rectangle (1.3,1.3);
        \node[] (zero) at (0,0) {};
        \draw[blue,dashed] (zero) circle (1cm);
        \node[blue] at (-0.9,0) {$S^1$};

        \draw[thick,red!60!black] ({cos( 15 )},{sin( 15 )}) arc (15:85:1);

        \filldraw[fill=green!20!white
          ,draw=green!60!black
          ,thick
          ,path fading=west,fading angle=50] (0,0)
        -- ({cos( 15 )*.65},{sin( 15 )*.65}) arc (15:85:.65) -- cycle;

        \node[green!40!black] at (-0.15,0.34) {$\widehat\Sect_i$};

        \node[] (lft) at ({cos( 85 )},{sin( 85 )}) {};
        \node[] (rgt) at ({cos( 15 )},{sin( 15 )}) {};

        \draw[->,red!40!black] ({cos( 45 )},{sin( 45 )})
          to [out=-5, in=25] node[sloped,midway,above] {$-\frac{\pi}{2k}$} (rgt);

        \draw[->,red!40!black] ({cos( 55 )},{sin( 55 )})
          to [out=105, in=75] node[sloped,midway,above] {$+\frac{\pi}{2k}$} (lft);

        \draw[thick,red!40!black,path fading=south]
          ({cos( 85 ) * 0.65},{sin( 85 ) * 0.65}) -- ({cos( 85 )},{sin( 85 )});
        \draw[thick,red!40!black,path fading=west]
          ({cos( 15 ) * 0.65},{sin( 15 ) * 0.65}) -- ({cos( 15 )},{sin( 15 )});
        \fill[red!40!black] ({cos( 85 )},{sin( 85 )}) circle (.7pt);
        \fill[white] ({cos( 85 )},{sin( 85 )}) circle (.3pt);
        \fill[red!40!black] ({cos( 15 )},{sin( 15 )}) circle (.7pt);
        \fill[white] ({cos( 15 )},{sin( 15 )}) circle (.3pt);

        \draw[path fading=south]
        ({cos( 45 ) * 0.65},{sin( 45 ) * 0.65})
        -- ({cos(45)},{sin(45)});
        \node[above right] at ({cos(45)},{sin(45)}) {$\alpha_{i+1}$};
        \fill[blue!20!white] ({cos(45)},{sin(45)}) circle (.7pt);
        \fill[white] ({cos( 45 )},{sin( 45 )}) circle (.3pt);
        \draw[path fading=south]
        ({cos( 55 ) * 0.65},{sin( 55 ) * 0.65})
        -- ({cos(55)},{sin(55)});
        \node[above right] at ({cos(55)},{sin(55)}) {$\alpha_{i}$};
        \fill[blue!20!white] ({cos(55)},{sin(55)}) circle (.7pt);
        \fill[white] ({cos( 55 )},{sin( 55 )}) circle (.3pt);

        \fill (zero) circle (1pt);
      \end{tikzpicture}
    \end{center}
    \caption{An exemplary supersector $\widehat\Sect_i$ corresponding to the
      anti-Stokes directions $\alpha_i$ and $\alpha_{i+1}$.}
  \end{figure}
If we extend to a subset $J$ of $\cK$ containing more then one level level,
$\#J>1$, the set
\[
  \bigcup_{k\in J}\bigl\{U\bigl(\alpha,\frac{\pi}{k}\bigr)\mid\alpha\in\A^{k}\bigl\}
  =\bigcup_{k\in J}\dot\cU^k
\]
is no longer guaranteed to be a nerve.
Hence we have to define the coverings $\cU^{\leq k}$ and $\cU^{<k}$ in a
different way. We will start by adding the arcs corresponding to the largest
degree first and continue by adding the arcs corresponding to smaller degrees
successively as described below.

Denote by $\left\{K_1<\cdots<K_s=k\right\}
=\left\{\max\left(\cK_\alpha\cap[0,k]\right)\mid\alpha\in\A^{\leq k}\right\}$
the set of all \emph{$k$-maximum levels} for all $\alpha\in\A^{\leq k}$.
\begin{itemize}
\item[\textbf{2.}] The cyclic covering
  $\cU^{\leq k}=\left\{U_\alpha^{\leq k}\mid\alpha\in\A^{\leq k}\right\}$
  will be defined by decreasing induction on the levels.
  Let us assume that
  \begin{einr}
    the $\dot U_\alpha^{\leq k}$ are defined for all $\alpha\in\A^{\leq k}$ with
    $k$-maximum level greater than $K_i$ such that their complete family is a
    nerve.
  \end{einr}
  For every anti-Stokes direction $\alpha$ with $k$-maximum level $K_i$ let
  $\alpha^-$ (resp.\ $\alpha^+$) be the next anti-Stokes direction with
  $k$-maximum level greater then $K_i$ on the left (resp.\ on the right).

  Define $\dot U_{\alpha^-,\alpha^+}$ as the clockwise hull of the arcs
  $\dot U_{\alpha^-}^{\leq k}$ and $\dot U_{\alpha^+}^{\leq k}$ already defined
  by induction.
  If there are no anti-Stokes directions with $k$-maximum level greater then
  $K_i$ we set $\dot U_{\alpha^-,\alpha^+}=S^1$.

  We then add for every anti-Stokes direction $\alpha$ with $k$-maximum level
  $K_i$ the arc
  \[
    \dot U_\alpha^{\leq k}
    :=U\bigl(\alpha,\frac{\pi}{K_i}\bigr)\cap\dot U_{\alpha^-,\alpha^+}
  \]
  to $\dot\cU^{\leq k}$ and the received family is still a nerve.
  \begin{figure}[h]
    \begin{center}
      \begin{tikzpicture}[scale=3.5]
        \node[green!40!black] at (-0.3,0.4) {$\dot U_{\alpha^-}^{\leq k}$};
        \node[brown!40!black] at (0.65,0.3) {$\dot U_{\alpha^+}^{\leq k}$};
        \node[blue] at (0.4,-0.13) {$U\bigl(\alpha,\frac{\pi}{K_i}\bigr)$};
        \node[red!40!black] at (1.05,0.5) {$\dot U_\alpha^{\leq k}$};

        \clip (-1.1,-0.2) rectangle (1.3,1.3);
        \node[] (zero) at (0,0) {};
        \draw[blue,dashed] (zero) circle (1cm);
        \node[blue] at (-0.9,0) {$S^1$};

        \draw[thick,red!60!black] ({cos( 15 )},{sin( 15 )}) arc (15:100:1);

        \foreach \n/\nn/\mycol/\w/\txt in {48/108/green/0.65/$\alpha^-$
                                     ,15/75/brown/0.55/$\alpha^+$
                                     ,5/100/blue/0.45/$\alpha$}
        {
          \pgfmathsetmacro\mid{{(\n + \nn)/2}}
          \filldraw[fill=\mycol!20!white
            ,draw=\mycol!60!black
            ,thick
            ,path fading=west,fading angle=\mid] (0,0)
            -- ({cos( \n ) * \w},{sin( \n ) * \w}) arc ({\n}:{\nn}:{\w}) -- cycle;
          \draw[path fading=west]
            ({cos( \mid ) * \w},{sin( \mid ) * \w}) -- ({cos( \mid )},{sin( \mid )});
          \fill[blue!20!white] ({cos(\mid)},{sin(\mid)}) circle (.7pt);
          \fill[white] ({cos(\mid)},{sin(\mid)}) circle (.3pt);
          \node[above right] at ({cos(\mid)},{sin(\mid)}) {\txt};
        }

        \draw[thick,red!40!black,path fading=south]
          ({cos( 100 ) * 0.45},{sin( 100 ) * 0.45}) -- ({cos( 100 )},{sin( 100 )});
        \draw[thick,red!40!black,path fading=west]
          ({cos( 15 ) * 0.55},{sin( 15 ) * 0.55}) -- ({cos( 15 )},{sin( 15 )});

        \fill[red!40!black] ({cos( 100 )},{sin( 100 )}) circle (.7pt);
        \fill[white] ({cos( 100 )},{sin( 100 )}) circle (.3pt);
        \fill[red!40!black] ({cos( 15 )},{sin( 15 )}) circle (.7pt);
        \fill[white] ({cos( 15 )},{sin( 15 )}) circle (.3pt);

        \fill (zero) circle (1pt);
      \end{tikzpicture}
    \end{center}
    \caption{An exemplary construction of an arc $\dot U_\alpha^{\leq k}$
      corresponding (in \textcolor{red!40!black}{red}) to the anti-Stokes
      direction $\alpha$.  The \textcolor{green!40!black}{green} arc is
      $\dot U_{\alpha^-}^{\leq k}$, the \textcolor{brown!70!black}{brown} arc is
      $\dot U_{\alpha^+}^{\leq k}$ and the \textcolor{blue}{blue} arc is the
      Stokes arc of level $K_i$ at $\alpha$, where $K_i$ is the $k$-maximum
      level of $\alpha$.}
  \end{figure}
\end{itemize}

\begin{itemize}
\item[\textbf{3.}] The last cyclic covering,
  $\cU^{<k}=\left\{U_{\alpha}^{<k}\mid\alpha\in\A^{<k}\right\}$, is defined as
  $\cU^{<k}:=\cU^{\leq k'}$ where $k':=\max\{k''\in\cK\mid k''<k\}$.
\end{itemize}

\begin{figure} %{{{
  \centering

  \def\kOne{7}
  \def\kTwo{10}
  \begin{tikzpicture}[scale=6] %{{{

    \newcommand{\myDrawArcA}[4]{%{{{
      % Parameter: radius , center , width , color
      \pgfmathsetmacro\hwdth{#3 / 2}
      \draw[#4]
        ({cos( #2 )},{sin( #2 )})
        --
        ({cos( #2 ) * #1 },{sin( #2 ) * #1 });
      \draw[ultra thick,#4]
        ({cos(#2 - \hwdth) * #1},{sin(#2 - \hwdth) * #1})
        arc
        ({#2 - \hwdth}:{#2 + \hwdth}:#1);
      \draw[dotted,#4]
        ({cos(#2 - \hwdth) * #1},{sin(#2 - \hwdth) * #1})
        --
        ({cos(#2 - \hwdth)},{sin(#2 - \hwdth)});
      \draw[dotted,#4]
        ({cos(#2 + \hwdth) * #1},{sin(#2 + \hwdth) * #1})
        --
        ({cos(#2 + \hwdth)},{sin(#2 + \hwdth)});
      % \filldraw[white] ({cos(#2)},{sin(#2)}) circle (0.5pt);
      % \filldraw[red] ({cos(#2)},{sin(#2)}) circle (0.2pt);
      \fill[blue!20!white] ({cos(#2)},{sin(#2)}) circle (.5pt);
      \fill[white] ({cos(#2)},{sin(#2)}) circle (.2pt);
    }%}}}
    \newcommand{\myDrawArcB}[5]{%{{{
      % Parameter: radius , center , start , stop , color
      \draw[#5]
        ({cos( #2 )},{sin( #2 )})
        --
        ({cos( #2 ) * #1 },{sin( #2 ) * #1 });
      \draw[ultra thick,#5]
        ({cos( #3 ) * #1},{sin( #3 ) * #1})
        arc
        ({ #3 }:{ #4 }:#1);
      \draw[dotted,#5]
        ({cos( #3 ) * #1},{sin( #3 ) * #1})
        --
        ({cos( #3 )},{sin( #3 )});
      \draw[dotted,#5]
        ({cos( #4 ) * #1},{sin( #4 ) * #1})
        --
        ({cos( #4 )},{sin( #4 )});
      % \filldraw[white] ({cos(#2)},{sin(#2)}) circle (0.5pt);
      % \filldraw[red] ({cos(#2)},{sin(#2)}) circle (0.2pt);
      \fill[blue!20!white] ({cos(#2)},{sin(#2)}) circle (.5pt);
      \fill[white] ({cos(#2)},{sin(#2)}) circle (.2pt);
    }%}}}

    \node (zero) at (0,0) {};
    \draw (zero) circle (1cm);

    %%%%%%%%%%%%%%%%%%%%%%%%%%%%%%%%%%%%%%%%%%%%%%%%%%%%%%%%%%%%%%%%%%%%%%%%%
    %% Inner:
    \foreach \n/\mycol/\baseR in {\kOne/orange/0.8
                                 ,\kTwo/purple/0.9}
    {%{{{
      \pgfmathsetmacro\wdth{180/\n}
      \foreach \i in {1,2,...,\n}}}

    %%%%%%%%%%%%%%%%%%%%%%%%%%%%%%%%%%%%%%%%%%%%%%%%%%%%%%%%%%%%%%%%%%%%%%%%%
    %% Outer:
    \def\mycol{brown}
    \pgfmathsetmacro{\wdth}{180/\kTwo}
    \foreach \i in {1,2,...,\kTwo}{%
      \pgfmathsetmacro\r{{1.05 + mod(\i+1,2)*0.05}}
      \pgfmathsetmacro\angl{\i* \wdth}
      \myDrawArcA{\r}{\angl}{\wdth}{\mycol}

      \pgfmathsetmacro\r{{1.05 + mod(\i + mod(\kTwo+1,2),2)*0.05}}
      \pgfmathsetmacro\angl{\i* \wdth+180}
      \myDrawArcA{\r}{\angl}{\wdth}{\mycol}
    }
    \foreach \i in {1,2,...,\kOne}{%
      \foreach \j in {1,...,\kTwo}{%
        \pgfmathparse{\i/\kOne <= \j/\kTwo ? 0 : 1}
        \ifnum\pgfmathresult=0{%
          \pgfmathparse{\i/\kOne < \j/\kTwo ? 0 : 1}
          \ifnum\pgfmathresult=0{%
            \pgfmathsetmacro\r{{1.15 + mod(\i+1,2)*0.05}}
            \pgfmathsetmacro\center{\i/\kOne*180}
            \pgfmathsetmacro\start{(\i-0.5)*180/\kOne > (\j-1.5)*180/\kTwo
              ? (\i-0.5)*180/\kOne : (\j-1.5)*180/\kTwo}
            \pgfmathsetmacro\stop{(\i+0.5)*180/\kOne > (\j+0.5)*180/\kTwo
              ? (\j+0.5)*180/\kTwo : (\i+0.5)*180/\kOne}
            \myDrawArcB{\r}{\center}{\start}{\stop}{\mycol}

            \pgfmathsetmacro\r{{1.15 + mod(\i,2)*0.05}}
            \pgfmathsetmacro\center{\center+180}
            \pgfmathsetmacro\start{\start+180}
            \pgfmathsetmacro\stop{\stop+180}
            \myDrawArcB{\r}{\center}{\start}{\stop}{\mycol}
          }\fi
          \breakforeach
        }\fi
      }
    }

    \fill[white] (zero) circle (1.5pt);
    \fill (zero) circle (.7pt);
  \end{tikzpicture} %}}}
  \caption{The adequate coverings for an example with $\cK=\{\kOne,\kTwo\}$ and
    $\A=\left\{ \frac{j\cdot\pi}{k}\mid k\in\cK\text{, } j\in\N \right\}$.
    The anti-Stokes directions are marked by the \textcolor{blue}{blue} circles.
    The arcs of $\dot\cU^{\kOne}=\dot\cU^{\leq\kOne}$ are
    \textcolor{orange}{orange}, the arcs of $\dot\cU^{\kTwo}$ are
    \textcolor{purple}{purple} and the arcs of $\dot\cU^{\leq\kTwo}=\dot\cU$
    are \textcolor{brown}{brown}.
  }\label{fig:adequateCovering}
\end{figure}%}}}

\begin{rem}
  The coverings $\cU^{k}$, $\cU^{\leq k}$ and $\cU^{<k}$ depend only on
  $\cQ(A^0)$. Hence they depend only on the determining polynomials.
\end{rem}
For every $k'\leq k\in\cK$ such that $k'\leq k$ and every
$\alpha\in\A^{k'}\subset\A^{\leq k}$ (resp.\
$\alpha\in\A^{<k'}\subset\A^{\leq k}$) is
$\dot U^{\leq k}_\alpha\subset\dot U^{k'}_\alpha$ (resp.\
$\dot U^{\leq k}_\alpha\subset\dot U^{<k'}_\alpha$), thus the covering
$\cU^{\leq k}$ refines $\cU^{k'}$ and $\cU^{<k'}$.
Thus the coverings are comparable on the different levels.
Furthermore are the coverings defined, such that they satisfy the
condition in Proposition~\ref{prop:adeqCovCondition}, such that the first
property in the following proposition is satisfied. The other two can be found
at~\cite[Prop.II.3.1 (iv)]{Loday1994} or on page 5269 of~\cite{Loday2004}.
\begin{prop}\label{prop:adequateProperties}
  \marginnote{\cite[Prop.II.3.1]{Loday1994}}
  \PROBLEM[Was bedeutet das?]
  Let $k\in\cK$, then
  \begin{enumerate}
    \item the coverings $\cU^{k}$, $\cU^{\leq k}$ and $\cU^{<k}$ are adequate
      to $\Lambda^k(A^0)$, $\Lambda^{\leq k}(A^0)$ and $\Lambda^{<k}(A^0)$, respectively;
    \item there exists no $0$-cochain in $\Lambda^k(A^0)$ on $\cU^k$;
    \item on $\cU^{\leq k}$ there is no $0$-cochain in $\Lambda^{\leq k}(A^0)$
      of level $k$, i.e.\ all $0$-cochains of $\Lambda^{\leq k}(A^0)$ belong to
      $\Lambda^{<k}(A^0)$.
  \end{enumerate}
\end{prop}
To have a shorter notation, we denote the product
$\prod_{\alpha\in\A^\star}\Gamma(\dot U_\alpha^\star;\Lambda^\star(A^0))$ by
$\Gamma(\dot U^\star;\Lambda^\star(A^0))$ for every
$\star\in\{k,<k,\leq k,\dots\}$.
Let us also denote $\cU:=\cU^k$ where $k$ is the maximal degree, i.e.\
$k=\max\cK$.

%%%%%%%%%%%%%%%%%%%%%%%%%%%%%%%%%%%%%%%%%%%%%%%%%%%%%%%%%%%%%%%%%%%%%%%%%%%%%%%
\subsubsection{The case of a unique level}
\marginnote{\cite[II.3.2]{Loday1994}}
First we will proof Theorem~\ref{thm:mainThm2} in the case of a unique level.
This means that
\begin{itemize}
  \item either $\Lambda(A^0)$ has only one level $k$, thus
    \begin{einr}
      $\Lambda(A^0)=\Lambda^k(A^0)$ and $\Sto_\theta(A^0)=\Sto_\theta^k(A^0)$
      for every $\theta$,
    \end{einr}
  \item or we restrict to a given level $k\in\cK$.
\end{itemize}
The following lemma, which will also be required for the multileveled case,
solves the case of a unique level.
\begin{lem}\label{lem:solutionOfSingleLeveledCase}
  Let $k\in\cK$.
  The morphism $h$ from Theorem~\ref{thm:mainThm2} is in the case of a unique
  level build as
  \[ \begin{tikzcd}[row sep=0cm]
    \underset{\alpha\in\A}\prod\Sto_\alpha^k(A^0)
    \rar{i^k}
    & \Gamma(\dot\cU^k;\Lambda^k(A^0))
    \rar{s^k}
    & H^1(S^1;\Lambda^k(A^0))
    \\
    \text{\rotatebox[origin=c]{-90}{$=$}}
    \tikzmark{e1}
    &&
    \tikzmark{e2}
    \text{\rotatebox[origin=c]{-90}{$=$}}
    \\
    \underset{\alpha\in\A}\prod\Sto_\alpha(A^0)
    \arrow{rr}{h}
    && H^1(S^1;\Lambda(A^0))
  \end{tikzcd} \]
  \begin{flushright}
    \tikzmarkc{n1}{blue} only in the single leveled case
    \begin{tikzpicture}[remember picture,overlay]
      \draw[->,blue!50!white,thick] (n1) to[out=180,in=0] (e1);
      \draw[->,blue!50!white,thick] (n1) to[out=180,in=180,distance=3cm] (e2);
    \end{tikzpicture}
  \end{flushright}
  from
  \begin{itemize}
  \item the canonical injective map $i^k$ i.e.\ the map which is the canonical
    extension of germs to their natural arc of definition, and
  \item the quotient map $s^k$
  \end{itemize}
  which are both isomorphisms.
\end{lem}
\begin{proof}
  \begin{enumerate}
  \item The map
    \[ \begin{tikzcd}
      i^k: \underset{\alpha\in\A}\prod\Sto_\alpha^k(A^0)
      \rar &
      % \underset{\Gamma(\dot\cU^k;\Lambda^k(A^0))}{%
      %   \underset{\text{\rotatebox[origin=c]{-90}{$=$}}}{%
      %     \underbrace{%
            \prod_{\alpha\in\A^k}\Gamma(\dot\cU_\alpha^k;\Lambda^k(A^0))
          % }}}
    \end{tikzcd} \]
    is welldefined, i.e.\ the germs at $\alpha$ can be uniquely
    extended to the arc $\dot\cU_\alpha^k$, since
    \begin{itemize}
    \item the sheaf $\Lambda(A^0)$, and thus $\Lambda^k(A^0)$, is a
      piecewise-constant sheaf, thus we know that the sections on arcs are
      uniquely determined by their germs and
    \item arcs of $\dot\cU_\alpha^k$ are the natural domains of existence of the
      corresponding Stokes germs.
    \end{itemize}
    The second fact can be found in Loday-Richaud's paper~\cite{Loday2004} on
    page 5269.
    In~\cite[5269]{Loday2004} is also stated, that only the extensions of Stokes
    germs at $\alpha$ are sections of $\Lambda(A^0)$ on $\dot U_\alpha^k$ and
    this implies surjectivity.
  \item The second map
    \[ \begin{tikzcd}
      s^k:
      % \overset{\Gamma(\dot\cU^k;\Lambda^k(A^0))}{%
      %   \overset{\text{\rotatebox[origin=c]{-90}{$=$}}}{%
      %     \overbrace{%
            \prod_{\alpha\in\A^k}\Gamma(\dot\cU_\alpha^k;\Lambda^k(A^0))
          % }}}
      \rar &
      H^1(S^1;\Lambda^k(A^0))
    \end{tikzcd} \]
    is a bijection, \rewrite{since} from
    Proposition~\ref{prop:adequateProperties} we know that it is
    \begin{itemize}
    \item \textbf{surjective}, \rewrite{since} $\cU^k$ is adequate to
      $\Lambda^k(A^0)$ and
    \item \textbf{injective}, \rewrite{since} on $\cU^k$ there is no
      $0$-cochain in $\Lambda^k(A^0)$.
    \end{itemize}
  \end{enumerate}
  \PROBLEM[Show that this is the correct $h$]
  \PROBLEM[Naturality?]
\end{proof}

%%%%%%%%%%%%%%%%%%%%%%%%%%%%%%%%%%%%%%%%%%%%%%%%%%%%%%%%%%%%%%%%%%%%%%%%%%%%%%%
\subsubsection{The case of several levels}
Let us now proof the Theorem~\ref{thm:mainThm2} in the case of several levels,
We will begin by defining some maps, which will be composed to obtain the
isomorphism $h$ from the theorem.
\begin{defn}\label{defn:firstSetOfInlusions}
  Begin with the \emph{product map of cocycles} $\mathfrak{S}^{\leq k}$.
  This map will be composed from the following injective maps:
  \begin{enumerate}
    \item The first map is defined as
      \begin{align*}
        \sigma^k:\Gamma(\dot\cU^k;\Lambda^k(A^0))
        &\longrightarrow \Gamma(\dot\cU^{\leq k};\Lambda^{\leq k}(A^0))
      \\\dot f=(\dot f_\alpha)_{\alpha\in\A^k}
        &\longmapsto (\dot G_\alpha)_{\alpha\in\A^{\leq k}}
      \end{align*}
      where
      \[
        \dot G_\alpha=\begin{cases}
          \dot f_\alpha \text{~restricted to } \dot U_\alpha^{\leq k}
          \text{~and seen as being in } \Lambda^{\leq k}(A^0)
          & \text{~when } \alpha\in\A^k
        \\\id \text{~(the identity) }
          & \text{~when } \alpha\notin\A^k
        \end{cases}
      \]
    \item and the second map
      \begin{align*}
        \sigma^{<k}:\Gamma(\dot\cU^{<k};\Lambda^{<k}(A^0))
        &\longrightarrow \Gamma(\dot\cU^{\leq k};\Lambda^{\leq k}(A^0))
      \\\dot f=(\dot f_\alpha)_{\alpha\in\A^{<k}}
        &\longmapsto (\dot F_\alpha)_{\alpha\in\A^{\leq k}}
      \end{align*}
      is defined, in a similar way, as
      \[
        \dot F_\alpha=\begin{cases}
          \dot f_\alpha \text{~restricted to } \dot U_\alpha^{\leq k}
          \text{~and seen as being in } \Lambda^{\leq k}(A^0)
          & \text{~when } \alpha\in\A^{<k}
        \\\id \text{~(the identity) }
          & \text{~when } \alpha\notin\A^{<k}
        \end{cases}
      \]
  \end{enumerate}
  Thus we can define
  \begin{align*}
    \mathfrak{S}^{\leq k}:
    \Gamma(\dot\cU^{<k};\Lambda^{<k}(A^0))
    \times
    \Gamma(\dot\cU^{k};\Lambda^{k}(A^0))
    &\longrightarrow
    \Gamma(\dot\cU^{\leq k};\Lambda^{\leq k}(A^0))
  \\(\dot f, \dot g)
    &\longmapsto
    (\dot F_\alpha\dot G_\alpha)_{\alpha\in\A^{\leq k}}
  \end{align*}
  where $(\dot F_\alpha)_{\alpha\in\A^{\leq k}}=\sigma^{<k}(\dot f)$ and
  $(\dot G_\alpha)_{\alpha\in\A^{\leq k}}=\sigma^k(\dot g)$ are defined as
  above.
  \begin{s-rem}
    This map $\mathfrak{S}^{\leq k}$ is injective, since injectivity for germs
    implies injectivity for sections.
  \end{s-rem}
\end{defn}

\begin{lem}
  \marginnote{\cite[Lem.II.3.3]{Loday1994}}
  Let $k\in\cK$.
  \begin{enumerate}
    \item If the cocycles $\mathfrak{S}^{\leq k}(\dot f,\dot g)$ and
      $\mathfrak{S}^{\leq k}(\dot f',\dot g')$ are cohomologous in
      $\Gamma(\dot\cU^{\leq k};\Lambda^{\leq k}(A^0))$
      then $\dot f$ and $\dot f'$ are cohomologous in
      $\Gamma(\dot\cU^{<k};\Lambda^{<k}(A^0))$.
    \item Any cocycle in $\Gamma(\dot\cU^{\leq k};\Lambda^{\leq k}(A^0))$ is
      cohomologous to a cocycle in the range of $\mathfrak{S}^{\leq k}$.
  \end{enumerate}
\end{lem}
\begin{proof}
  \begin{enumerate}
    \item Denote by $\alpha^+$ the nearest anti-Stokes direction in
      $\A^{\leq k}$ on the right of $\alpha$.
      The cocycles $\mathfrak{S}^{\leq k}(\dot f,\dot g)$ and
      $\mathfrak{S}^{\leq k}(\dot f',\dot g')$ are cohomologous if and only
      if there is a $0$-cochain $c=(c_\alpha)_{\alpha\in\A^{\leq k}}
      \in\Gamma(\cU^{\leq k},\Lambda^{\leq k}(A^0)$ such that
      \begin{equation}\label{eq:UniqueString:urdtindfgupndtcn}
        \dot F_\alpha\dot G_\alpha =
        c_\alpha^{-1}\dot F_\alpha'\dot G_\alpha'c_{\alpha^+}
      \end{equation}
      for every $\alpha\in\A$. From Proposition~\ref{prop:adequateProperties}
      follows, that $c$ is with values in $\Lambda^{<k}(A^0)$.
      The fact that $\Lambda^k(A^0)$ is normal in $\Lambda^{\leq k}(A^0)$ in
      Proposition~\ref{prop:PropertiesOfStokesSheafSplitting}, can be used to
      see that $ c_{\alpha^+}^{-1}G_\alpha'c_{\alpha^+}
      \in\Gamma(\cU^{\leq k};\Lambda^{k}(A^0))$.
      Thus, we rewrite the relation (\ref{eq:UniqueString:urdtindfgupndtcn}) to
      \[
        \dot F_\alpha\dot G_\alpha =
        (c_\alpha^{-1}\dot F_\alpha'c_{\alpha^+})
        (c_{\alpha^+}^{-1}G_\alpha'c_{\alpha^+})
        \,,\qquad\text{~for~} \alpha\in\A^{\leq k} \,.
      \]
      Since Corollary~\ref{cor:factorStokesGerms} tells us that this
      factorization corresponds to a semidirect product and thus is unique we
      obtain
      \[
        \dot F_\alpha=c_\alpha^{-1}\dot F_\alpha'c_{\alpha^+}
        \qquad \text{~and~} \qquad
        \dot G_\alpha=c_{\alpha^+}^{-1}\dot G_\alpha'c_{\alpha^+}
      \]
      for all $\alpha\in\A^{\leq k}$.
      The former relation implies that $(\dot F_\alpha)$ and $(\dot F_\alpha')$
      are cohomologous with values in $\Lambda^{<k}(A^0)$ on $\cU^{\leq k}$.
      Since the coarser covering $\cU^{<k}$ is already adequate to
      $\Lambda^{<k}(A^0)$ are $(\dot F_\alpha)$ and $(\dot F_\alpha')$ already
      in $\Gamma(\dot\cU^{<k};\Lambda^{<k}(A^0))$ cohomologous.
    \item The proof of part 2.\ (together with a proof of part 1.) can be
      found in Loday-Richaud's paper \cite[Proof of Lem.II.3.3]{Loday1994}.
  \end{enumerate}
\end{proof}
Let $k\in\cK=\{k_1<k_2<\dots<k_r\}$ and $k'=\max\{k'\in\cK\mid k'<k\}$. We then
know by definition that $\cU^{<k}=\cU^{\leq k'}$ as well as
$\Lambda^{< k}(A^0)=\Lambda^{\leq k'}(A^0)$ and thus
$\Gamma(\dot\cU^{< k};\Lambda^{< k}(A^0))=
\Gamma(\dot\cU^{\leq k'};\Lambda^{\leq k'}(A^0))$ and we obtain the following
proposition.
\begin{prop}\label{prop:theMapTau}
  By applying $\mathfrak{S}^{\leq k}$ successively for different $k$'s
  in decending order, one obtains the \emph{product map of single leveled
  cocycles $\tau$} in the following way
  \[ \begin{tikzcd}[column sep=1.4cm,row sep=.4cm]
      \underset{\tikzmark{e1}}{\underbrace{%
        \Gamma(\dot\cU^{<k_r};\Lambda^{<k_r}(A^0))}}
      \times
      \Gamma(\dot\cU^{k_r};\Lambda^{k_r}(A^0))
      \rar{\mathfrak{S}^{\leq k_r}}&
      \overset{\normalsize\Gamma(\dot\cU;\Lambda(A^0))}{%
        \overset{\normalsize\text{\rotatebox[origin=c]{-90}{$=$}}}{%
          \Gamma(\dot\cU^{\leq k_r};\Lambda^{\leq k_r}(A^0))
      }}
      \\\\\underset{\tikzmark{e2}}{\underbrace{%i
        \Gamma(\dot\cU^{<k_{r-1}};\Lambda^{<k_{r-1}}(A^0))}}
      \times
      \Gamma(\dot\cU^{k_{r-1}};\Lambda^{k_{r-1}}(A^0))
      \rar{\mathfrak{S}^{\leq k_{r-1}}}&
      \overset{\tikzmark{n1}}{%
        \Gamma(\dot\cU^{\leq k_{r-1}};\Lambda^{\leq k_{r-1}}(A^0))}
      \\\\\hspace{6cm}\cdots \rar{\mathfrak{S}^{\leq k_{r-2}}}&
      \overset{\tikzmark{n2}}{%
        \Gamma(\dot\cU^{\leq k_{r-2}};\Lambda^{\leq k_{r-2}}(A^0))}
      \\\underset{\tikzmark{eEND}}{\underbrace{%
        \Gamma(\dot\cU^{<k_3};\Lambda^{<k_3}(A^0))}}
      \times
      \Gamma(\dot\cU^{k_3};\Lambda^{k_3}(A^0))
      \rar{\mathfrak{S}^{\leq k_3}}&
      \cdots\hspace{3cm}
      \\\\
      \Gamma(\dot\cU^{k_1};\Lambda^{k_1}(A^0))
      \times
      \Gamma(\dot\cU^{k_2};\Lambda^{k_2}(A^0))
      \rar{\mathfrak{S}^{\leq k_2}}&
      \overset{\tikzmark{nEND}}{%
        \Gamma(\dot\cU^{\leq k_2};\Lambda^{\leq k_2}(A^0))}
  \end{tikzcd} \]
  \begin{tikzpicture}[remember picture,overlay]
    \draw[<-] (n1) to[out=90,in=270,distance=1.6cm] node[midway,fill=white,sloped]{$\cong$}
      ([yshift=.3em]e1);
    \draw[<-] (n2) to[out=90,in=270,distance=1.6cm] node[midway,fill=white,sloped]{$\cong$}
      ([yshift=.3em]e2);
    \draw[<-] (nEND) to[out=90,in=270,distance=1.6cm] node[midway,fill=white,sloped]{$\cong$}
      ([yshift=.3em]eEND);
  \end{tikzpicture}
  which can be written in the following compact form
  \begin{align*}
    \tau:\prod_{k\in\cK}\Gamma(\dot \cU^k;\Lambda^k(A^0))
    &\longrightarrow
    \Gamma(\dot\cU;\Lambda(A^0))
  \\(\dot f^k)_{k\in\cK}
    &\longmapsto
    \prod_{k\in\cK}\tau^k(\dot f^k)
  \end{align*}
  where the product is following an ascending order of levels and the maps
  $\tau_k$ are defined as
  \[ \begin{tikzcd}[row sep=0cm]
    \tau^k:\Gamma(\dot\cU^k;\Lambda^k(A^0))
    \arrow{r}{\sigma^k}&
    \Gamma(\dot\cU^{\leq k};\Lambda^{\leq k}(A^0))
    \rar &
    \Gamma(\dot\cU;\Lambda(A^0))
  \\~~~(\dot f_\alpha)_{\alpha\in\A^k}
    \arrow[|->]{rr}
    &&
    (\dot G_\alpha)_{\alpha\in\A}
  \end{tikzcd} \]
  with
  \[
    \dot G_\alpha=\begin{cases}
      \dot f_\alpha \text{~restricted to } \dot U_\alpha
      \text{~and seen as being in } \Lambda(A^0)
      & \text{~when } \alpha\in\A^k
    \\\id \text{~(the identity on $\dot U_\alpha$) }
      & \text{~when } \alpha\notin\A^k
    \end{cases}
  \]
  The defined map $\tau$ is clearly injective and it can be extended to an
  arbitrary order of levels (cf.\ Remark~\cite[Rem.II.3.5]{Loday1994}).
\end{prop}
From the previous statements one obtains the following Corollary (cf.\
\cite[Prop.II.3.4]{Loday1994}).
\begin{cor}
  The product map of single-leveled cocycles $\tau$ induces on the cohomology
  a bijective and natural map
  \[ \begin{tikzcd}
    \cT:
    \underset{\underset{k\in\cK}\prod H^1(S^1;\Lambda^k(A^0))}{%
      \underset{\text{\rotatebox[origin=c]{-90}{$\cong$}}}{%
        \prod_{k\in\cK}\Gamma(\dot\cU^k;\Lambda^k(A^0))}}
    \rar&
    \underset{H^1(S^1;\Lambda(A^0))}{%
      \underset{\text{\rotatebox[origin=c]{-90}{$\cong$}}}{%
        H^1(\cU;\Lambda(A^0))}}\,.
  \end{tikzcd} \]
\end{cor}

%%%%%%%%%%%%%%%%%%%%%%%%%%%%%%%%%%%%%%%%%%%%%%%%%%%%%%%%%%%%%%%%%%%%%%%%%%%%%%%
\paragraph{Composing functions to obtain $h$}
We have the ingredients to define the function $h$ from
Theorem~\ref{thm:mainThm2} by composition of already bijective maps.
\begin{proof}[Proof of Theorem~\ref{thm:mainThm2}]
  Let $i_\alpha:\Sto_\alpha(A^0)\to\prod_{k\in\cK}\Sto_\alpha^k(A^0)$ be the
  map which corresponds to the filtration from
  Proposition~\ref{prop:filtrationOfStokesGroup} and
  denote the composition
  \[ \begin{tikzcd}[column sep=1.8cm,row sep=-0.2cm]
      \displaystyle \prod_{\alpha\in\A}\Sto_\alpha(A^0)
      \rar{\prod_{\alpha\in\A}i_\alpha}
      \arrow[ddrr, out=270,in=200,"\mathfrak{T}"]
      & \displaystyle \prod_{\alpha\in\A}\prod_{k\in\cK}\Sto_\alpha^k(A^0)
    \\ &\text{\rotatebox[origin=c]{-90}{$\equiv$}}
    \\ &\displaystyle \prod_{k\in\cK}\prod_{\alpha\in\A}\Sto_\alpha^k(A^0)
      \rar{\prod_{k\in\cK}i^k}
      & \displaystyle \prod_{k\in\cK}\Gamma(\dot\cU^k;\Lambda^k(A^0))
  \end{tikzcd} \]
  by $\mathfrak{T}$, where $i^k$ was defined in
  Lemma~\ref{lem:solutionOfSingleLeveledCase}.
  The bijection $h$ is then obtained as the composition
  \[
    \cT\circ\mathfrak{T}: \prod_{\alpha\in\A}\Sto_\alpha(A^0)
    \longrightarrow H^1(\cU;\Lambda(A^0)) \,.
  \]
  \PROBLEM[naturality (is obvious?)]
  \PROBLEM[Show that this is the correct $h$]
\end{proof}

%%%%%%%%%%%%%%%%%%%%%%%%%%%%%%%%%%%%%%%%%%%%%%%%%%%%%%%%%%%%%%%%%%%%%%%%%%%%%%%
\subsection{Some exemplary calculations}\label{sec:WhichInformationIsNeeded}
Here we want to discuss which information is required to describe the Stokes
cocycle corresponding to a multileveled system in more depth.
We will look at the simplest case of multi-leveled systems, i.e. systems
corresponding to a normal form in $\GL_3(\C(\!\{t\}\!))$ with exactly two
levels, and will apply the techniques developed in the previous sections in an
rather explicit way.

Let $A^0$ be a normal form in $\GL_3(\C(\!\{t\}\!))$ with two levels
$\cK=\{k_1<k_2\}$ and assume that there is at least one anti-Stokes
direction $\theta$ which is beared by both levels.
\comm{This assumption implies that the set of anti-Stokes directions is uniquely
  determined by the levels and $\theta$.}
Let $q_j(t^{-1})$ be the determining polynomials and let $k_{jl}$ be the
degrees of $(q_j-q_l)(t^{-1})$.
\begin{lem}\label{lem:simpleLemmaOnStructure}
  Up to permutation we know that the leading coefficients of
  $(q_1-q_2)(t^{-1})$ and $(q_1-q_3)(t^{-1})$ are equal.
\end{lem}
\begin{proof}
  Let $q_1,q_2,q_3$ be sorted, such that
  \[
    \deg(q_1-q_2) =: k_2 > k_1 := \deg(q_2-q_3),
  \]
  i.e.\ such that the leading coefficients $\frac{a_{1,2}}{t^{k_2}}$ and
  $\frac{a_{2,3}}{t^{k_1}}$ are of different degree, seen as polynomials in
  $t^{-1}$. To find the leading coefficient of $q_1-q_3$ we have to
  distinguish different cases:
  \begin{enumerate}
  \item $\deg(q_1)=\deg(q_2)=k_2$: in this case has
    $q_3$ to be of degree $k_2$.
    Even further, $q_2$ and $q_3$ have to have the same leading coefficients,
    since they have to cancel out.
  \item $\deg(q_1)<k_2$: then has $q_2$ to be of
    degree $k_2$ and we can continue similar to the first case.
  \item $\deg(q_2)<k_2$: then can't $q_3$ be of
    degree $k_2=\deg(q_1)$.
  \end{enumerate}
  Thus we know for both, $q_2$ and $q_3$, that the coefficients of degree $k_2$
  are either equal or both zero.
\end{proof}
From the previous lemma follows that
\begin{itemize}
\item up to permutation is $k_2=k_{1,2}=k_{1,3}$ and $k_1=k_{2,3}$, i.e.\ the
  larger degree appears twice, and
\item $q_1\myrel{\alpha}q_2$ (resp.~$q_2\myrel{\alpha}q_1$) if and only if
  $q_1\myrel{\alpha}q_3$ (resp.~$q_3\myrel{\alpha}q_1$) and thus do the pairs
  $(q_1,q_2)$ and $(q_1,q_3)$ determine the same anti-Stokes directions.
\end{itemize}
The set of all anti-Stokes directions is then given as
\[
  \A=\left\{\theta+\frac{\pi}{k}\cdot j\mid k\in\cK\text{, }j\in\N\right\} \,.
\]
Denote by $\cY_0(t)$ a normal solution of $[A^0]$.

Let us start by looking at a single germ in depth.
The Proposition~\ref{prop:representation} states that every Stokes germ
$\phi_\alpha$ can be written as its matrix representation conjugated by the
normal solution, i.e.\ as $\phi_\alpha=\cY_{0}C_{\phi_\alpha}\cY_{0}^{-1}
=\rho_{\alpha}^{-1}(C_{\phi_\alpha})$.

Look at an example in which we will demonstrate, from which relations on the
determining polynomials which restriction on the form of the Stokes matrices
arise.
\begin{exmp}
  Let $\alpha\in\A$ be an anti-Stokes direction.
  From the definition of $\SSto_\alpha(A^0)$ (cf.\
  Definition~\ref{defn:groupOfFaithfullReps}) we know that, if one has
  $q_1\myrel{\alpha}q_2$, the Stokes matrix has the form
  \[
    \begin{pmatrix}
      1 & \text{\boldmath$c_1$} & \star
    \\\text{\boldmath$0$} & 1 & \star
    \\\star & \star & 1
    \end{pmatrix}
  \]
  where $c_j\in\C$ and $\star\in\C$.
  We have seen that $q_1\myrel{\alpha}q_2$ \Rightarrow{}
  $q_1\myrel{\alpha}q_3$ (cf.\ Lemma~\ref{lem:simpleLemmaOnStructure}) thus the
  representation has the structure
  \[
    \begin{pmatrix}
      1 & c_1 & \text{\boldmath$c_2$}
    \\0 & 1 & \star
    \\\text{\boldmath$0$} & \star & 1
    \end{pmatrix}
  \]
  and if we also know that neither $q_2\myrel{\alpha}q_3$ nor
  $q_3\myrel{\alpha}q_2$ it has the \rewrite{form}
  \[
    \begin{pmatrix}
      1 & c_1 & c_2
    \\0 & 1 & \text{\boldmath$0$}
    \\0 & \text{\boldmath$0$} & 1
    \end{pmatrix}\,.
  \]
  The obvious isomorphism $\vartheta_\alpha$ from
  Remark~\ref{rem:groupOfFaithfullReps} can explicitly given as
  \begin{align*}
    \vartheta_\alpha:\C^2 &\longrightarrow \SSto_\alpha(A^0)
    \\(c_1,c_2)&\longmapsto
                 \begin{pmatrix}
                   1 & c_1 & c_2
                   \\0 & 1 & 0
                   \\0 & 0 & 1
                 \end{pmatrix}
                             \,.
  \end{align*}
\end{exmp}
The following table gives an overview of the $9$ possible combinations and the
corresponding forms of the Stokes matrices which can arise in our situation.
The case in the lower right occurs when the direction $\alpha$ is not an
anti-Stokes direction.
\begin{center}
  \def\arraystretch{1.3}
  \setlength\tabcolsep{4mm}
  \begin{tabular}{r|c|c|c}
    & $q_2\myrel{\alpha}q_3$ & $q_3\myrel{\alpha}q_2$ & else
    \tabularnewline
    \hline
    $\substack{q_1\myrel{\alpha}q_2\\\text{and} \\q_1\myrel{\alpha}q_3}$
    & $\begin{pmatrix} 1 & c_2 & c_3 \\0 & 1 & c_1 \\0 & 0 & 1 \end{pmatrix}$
   \cellcolor{blue!15}
    & $\begin{pmatrix} 1 & c_2 & c_3 \\0 & 1 & 0 \\0 & c_1 & 1 \end{pmatrix}$
   \cellcolor{blue!15}
    & $\begin{pmatrix} 1 & c_2 & c_3 \\0 & 1 & 0 \\0 & 0 & 1 \end{pmatrix}$
   \cellcolor{green!15}
    \tabularnewline
    \hline
    $\substack{q_2\myrel{\alpha}q_1\\\text{and} \\q_3\myrel{\alpha}q_1}$
    & $\begin{pmatrix} 1 & 0 & 0 \\c_2' & 1 & c_1 \\c_3 & 0 & 1 \end{pmatrix}$
   \cellcolor{blue!15}
    & $\begin{pmatrix} 1 & 0 & 0 \\c_2 & 1 & 0 \\c_3' & c_1 & 1 \end{pmatrix}$
   \cellcolor{blue!15}
    & $\begin{pmatrix} 1 & 0 & 0 \\c_2 & 1 & 0 \\c_3 & 0 & 1 \end{pmatrix}$
   \cellcolor{green!15}
    \tabularnewline
    \hline
    else
    & $\begin{pmatrix} 1 & 0 & 0 \\0 & 1 & c_1 \\0 & 0 & 1 \end{pmatrix}$
   \cellcolor{purple!15}
    & $\begin{pmatrix} 1 & 0 & 0 \\0 & 1 & 0 \\0 & c_1 & 1 \end{pmatrix}$
   \cellcolor{purple!15}
    & $\begin{pmatrix} 1 & 0 & 0 \\0 & 1 & 0 \\0 & 0 & 1 \end{pmatrix}$
  \end{tabular}
\end{center}
In the \textcolor{blue!75!black}{blue} cases we have $\cK_\alpha=\cK$ and
$\C^3\overset{\vartheta_\alpha}{\underset{\cong}{\longrightarrow}}\SSto_\alpha(A^0)$.
In the \textcolor{green!50!black}{green} cases $\cK_\alpha=\{k_2\}$ and
$\C^2\overset{\vartheta_\alpha}{\underset{\cong}{\longrightarrow}}\SSto_\alpha(A^0)$
as well as in the \textcolor{purple!75!black}{purple} cases
$\cK_\alpha=\{k_1\}$ and
$\C^1\overset{\vartheta_\alpha}{\underset{\cong}{\longrightarrow}}\SSto_\alpha(A^0)$.
We will replace $c_2'$ by $c_2+c_1c_3$ and $c_3'$ by $c_1c_2+c_3$ to be
consistent with the decomposition in the next part
(cf.\ Example~\ref{exmp:decompositionHere}).
\begin{cor}
  The morphism $\prod_{\alpha\in\A}\vartheta_\alpha$ is an isomorphism of
  pointed sets, which maps the element only containing zeros to
  \[
    (\id,\id,\dots,\id)\in\prod_{\alpha\in\A}\SSto_\alpha(A^0),
  \]
  which gets by $\prod_{\alpha\in\A}\rho_\alpha^{-1}\circ h$ mapped to the
  trivial cohomology class in $\St(A^0)$.
\end{cor}

%%%%%%%%%%%%%%%%%%%%%%%%%%%%%%%%%%%%%%%%%%%%%%%%%%%%%%%%%%%%%%%%%%%%%%%%%%%%%%%
In Proposition~\ref{prop:filtrationOfStokesGroup} we have defined a
decomposition of the Stokes group $\Sto_\alpha(A^0)$ in subgroups generated by
$k$-germs for $k\in\cK$.
In our case, we have at most two nontrivial factors.
The map which gives the factors of this factorization in ascending order is
denoted by $i_\alpha$ and the decomposition is then given by
\[
  \phi_\alpha=\phi_\alpha^{k_1} \phi_\alpha^{k_2}
  \overset{i_\alpha}\longmapsto
    \left(\phi_\alpha^{k_1},\phi_\alpha^{k_2}\right)
    \in\Sto_\alpha^{k_1}(A^0)\times\Sto_\alpha^{k_2}(A^0) \,.
\]
This decomposition applied to some germ $\phi_\alpha$ is trivial if
$\#\cK(\phi_\alpha)\leq1$, thus the interesting cases are the
\textcolor{blue!75!black}{blue} cases from the table above.

The lower right blue case will be discussed in the following example in more
detail.
\begin{exmp}\label{exmp:decompositionHere}
  Look at an anti-Stokes direction $\alpha\in\A$ and assume that
  $q_3\myrel{\alpha}q_2$, $q_2\myrel{\alpha}q_1$ and $q_3\myrel{\alpha}q_1$. We
  then know that $\phi_\alpha\in \Sto_\alpha(A^0)$ can be written as
  \begin{align*}
    \phi_\alpha &= \cY_{0}\vartheta_\alpha(c_1,c_2,c_3) \cY_{0}^{-1}
    \\&
    =\cY_{0}
    \begin{pmatrix} 1 & 0 & 0 \\c_2 & 1 & 0 \\c_1c_2+c_3 & c_1 & 1 \end{pmatrix}
    \cY_{0}^{-1}
    \,.
  \end{align*}
  According to Remark~\ref{rem:algFactorization} the factor
  $\phi_\alpha^{k_1}\in\Sto_\alpha^{k_1}(A^0)$ is obtained by truncation to
  terms of level $k_1$, i.e.\ as
  \[
    \phi_\alpha^{k_1}=
    \cY_{0}
    \begin{pmatrix}
      1 & 0 & 0
    \\\text{\boldmath $0$} & 1 & 0
    \\\text{\boldmath $0$} & c_1 & 1
    \end{pmatrix}
    \cY_{0}^{-1}
    \,.
  \]
  The other factor $\phi_\alpha^{k_2}$ is then obtained as
  \begin{align*}
    \phi_\alpha^{k_2}&=
    \left(\phi_\alpha^{k_1}\right)^{-1}
    \phi_\alpha
  \\&=\cY_{0}
    \begin{pmatrix}
      1     & 0    & 0
    \\0     & 1    & 0
    \\0     & -c_1 & 1
    \end{pmatrix}
    \underset{=\id}{\underbrace{%
        \cY_{0}^{-1}
        \cY_{0}
    }}
    \begin{pmatrix} 1 & 0 & 0 \\c_2 & 1 & 0 \\c_1c_2+c_3 & c_1 & 1 \end{pmatrix}
    \cY_{0}^{-1}
  \\&=\cY_{0}
    \begin{pmatrix}
      1     & 0 & 0
    \\c_2     & 1          & 0
    \\c_3     & 0          & 1
    \end{pmatrix}
    \cY_{0}^{-1}
    \,.
  \end{align*}
\end{exmp}
\begin{rem}
  The four nontrivial decomposition in our situation are given by:
  \begin{enumerate}
  \item $\begin{pmatrix} 1 & 0 & 0 \\0 & 1 & c_1 \\0 & 0 & 1 \end{pmatrix}
    \cdot\begin{pmatrix} 1 & c_2 & c_3 \\0 & 1 & 0 \\0 & 0 & 1 \end{pmatrix}=
    \begin{pmatrix} 1 & c_2 & c_3 \\0 & 1 & c_1 \\0 & 0 & 1 \end{pmatrix}$
  \item $\begin{pmatrix} 1 & 0 & 0 \\0 & 1 & 0 \\0 & c_1 & 1 \end{pmatrix}
    \cdot\begin{pmatrix} 1 & c_2 & c_3 \\0 & 1 & 0 \\0 & 0 & 1 \end{pmatrix}=
    \begin{pmatrix} 1 & c_2 & c_3 \\0 & 1 & 0 \\0 & c_1 & 1 \end{pmatrix}$
  \item $\begin{pmatrix} 1 & 0 & 0 \\0 & 1 & c_1 \\0 & 0 & 1 \end{pmatrix}
    \cdot\begin{pmatrix} 1 & 0 & 0 \\c_2 & 1 & 0 \\c_3 & 0 & 1 \end{pmatrix}=
    \begin{pmatrix} 1 & 0 & 0 \\c_2+c_1c_3 & 1 & c_1 \\c_3 & 0 & 1 \end{pmatrix}$
  \item $\begin{pmatrix} 1 & 0 & 0 \\0 & 1 & 0 \\0 & c_1 & 1 \end{pmatrix}
    \cdot\begin{pmatrix} 1 & 0 & 0 \\c_2 & 1 & 0 \\c_3 & 0 & 1 \end{pmatrix}=
    \begin{pmatrix} 1 & 0 & 0 \\c_2 & 1 & 0 \\c_1c_2+c_3 & c_1 & 1 \end{pmatrix}$
  \end{enumerate}
\end{rem}

%%%%%%%%%%%%%%%%%%%%%%%%%%%%%%%%%%%%%%%%%%%%%%%%%%%%%%%%%%%%%%%%%%%%%%%%%%%%%%%
% \subsubsection{Explicit example}
\begin{exmp}
    \def\kOne{1}
    \def\kTwo{3}
    \def\zkOnepzKtwo{14} % 2\cdot(\kOne+2\cdot\kTwo
    \def\zkOne{2} % 2*\kOne
    \def\zkTwo{6} % 2*\kTwo

    Even more explicit, we can fix the levels $k_1=\kOne$ and $k_2=\kTwo$ together
    with $\theta=0$.
    Assume without any restriction that $q_1\myrel{\theta}q_2$ and
    $q_1\myrel{\theta}q_3$ as well as $q_2\myrel{\theta}q_3$.
    Other choices would result \rewrite{in reordering of the tuples below.}
    \comm{Let the matrix $L$ be given as $L=\diag(l_1,l_2,l_3)\in\Gl_n(\C)$.}

    The classification space is in this case isomorphic to
    $\C^{2\cdot(\kOne+2\cdot\kTwo)}=\C^{\zkOnepzKtwo}$.
    The element $({}^1c_1,{}^2c_1,
    {}^1c_2,{}^1c_3,{}^2c_2,{}^2c_3,\dots,{}^{\zkTwo}c_2,{}^{\zkTwo}c_3)
    \in\C^{\zkOnepzKtwo}$
    gets\comm{, via the isomorphism $\prod_{\alpha\in\A}j_\alpha$,} mapped to
    \begin{align*}
    &\left(
    \left(
        \begin{pmatrix} 1 & 0 & 0 \\0 & 1 & {}^1c_1 \\0 & 0 & 1 \end{pmatrix},
        \begin{pmatrix} 1 & 0 & 0 \\0 & 1 & 0 \\0 & {}^2c_1 & 1 \end{pmatrix}
    \right),
    \right.
    \\&\qquad\left(
    \left.
        \begin{pmatrix} 1 & {}^1c_2 & {}^1c_3 \\0 & 1 & 0 \\0 & 0 & 1 \end{pmatrix},
        \begin{pmatrix} 1 & 0 & 0 \\{}^2c_2 & 1 & 0 \\{}^2c_3 & 0 & 1 \end{pmatrix},
        \dots,
        \begin{pmatrix} 1 & 0 & 0 \\{}^{\zkTwo}c_2 & 1 & 0 \\{}^{\zkTwo}c_3 & 0 & 1 \end{pmatrix}
    \right)
    \right)
    \end{align*}
    in $\prod_{\alpha\in\A^{\kOne}}\SSto_c{\alpha}^{\kOne}(A^0) \times
    \prod_{\alpha\in\A^{\kTwo}}\SSto_{\alpha}^{\kTwo}(A^0)$ and thus the element
    \begin{align*}
    &\left(
    \left(
        \begin{pmatrix} 1 & 0 & 0 \\0 & 1 & {}^1c_1 \\0 & 0 & 1 \end{pmatrix},
        \id,\id,
        \begin{pmatrix} 1 & 0 & 0 \\0 & 1 & 0 \\0 & {}^2c_1 & 1 \end{pmatrix},
        \id,\id
    \right),
    \right.
    \\&\qquad\left(
    \left.
        \begin{pmatrix} 1 & {}^1c_2 & {}^1c_3 \\0 & 1 & 0 \\0 & 0 & 1 \end{pmatrix},
        \begin{pmatrix} 1 & 0 & 0 \\{}^2c_2 & 1 & 0 \\{}^2c_3 & 0 & 1 \end{pmatrix},
        \dots,
        \begin{pmatrix} 1 & 0 & 0 \\{}^{\zkTwo}c_2 & 1 & 0 \\{}^{\zkTwo}c_3 & 0 & 1 \end{pmatrix}
    \right)
    \right)
    \end{align*}
    in
    $\prod_{\alpha\in\A}\SSto_{\alpha}^{\kOne}(A^0) \times
    \prod_{\alpha\in\A}\SSto_{\alpha}^{\kTwo}(A^0)$.
    Using the morphism $\prod_{\alpha\in\A}i_\alpha^{-1}$ we get a complete set of
    Stokes matrices as
    \begin{align*}
    &\left(
        \begin{pmatrix} 1 & {}^1c_2 & {}^1c_3 \\0 & 1 & {}^1c_1 \\0 & 0 & 1 \end{pmatrix},
        \begin{pmatrix} 1 & 0 & 0 \\{}^2c_2 & 1 & 0 \\{}^2c_3 & 0 & 1 \end{pmatrix},
        \begin{pmatrix} 1 & {}^3c_2 & {}^3c_3 \\0 & 1 & 0 \\0 & 0 & 1 \end{pmatrix},
    \right.
    \\&\qquad
    \left.
        \begin{pmatrix} 1 & 0 & 0 \\{}^4c_2 & 1 & 0 \\{}^2c_1{}^4c_2+{}^4c_3 & {}^2c_1 & 1 \end{pmatrix},
        \begin{pmatrix} 1 & {}^5c_2 & {}^5c_3 \\0 & 1 & 0 \\0 & 0 & 1 \end{pmatrix},
        \begin{pmatrix} 1 & 0 & 0 \\{}^{\zkTwo}c_2 & 1 & 0 \\{}^{\zkTwo}c_3 & 0 & 1 \end{pmatrix}
    \right)
    \in
    \prod_{\alpha\in\A}\SSto_{\alpha}(A^0) \,.
    \end{align*}
    Applying the isomorphism $\prod_{\alpha\in\A}\rho_\alpha^{-1}$, i.e.\
    conjugation by the fundamental solution $\cY_0(t)=t^Le^{Q(t^{-1})}$ (cf.\
    Proposition~\ref{prop:representation}), yields then the corresponding element in
    $\prod_{\alpha\in\A}\Sto_{\alpha}(A^0)$.
\end{exmp}
% \begin{comment}
%   This element is explicitly given as
%   \begin{align*}
%     &\left(
%       \begin{pmatrix}
%           1 & {}^1c_2 t^{l_2-l_1}e^{(q_2-q_1)(t^{-1})}
%             & {}^1c_3 t^{l_3-l_1}e^{(q_3-q_1)(t^{-1})}
%         \\0 & 1 & {}^1c_1 t^{l_3-l_2}e^{(q_3-q_2)(t^{-1})}
%         \\0 & 0 & 1
%       \end{pmatrix},
%       \cY_0^{-1}
%       \begin{pmatrix}
%           1 & 0 & 0
%         \\{}^2c_2 & 1 & 0
%         \\{}^2c_3 & 0 & 1
%       \end{pmatrix}
%       \cY_0
%       ,
%     \right.
%   \\&\qquad
%       \cY_0^{-1}
%       \begin{pmatrix}
%           1 & {}^3c_2 & {}^3c_3
%         \\0 & 1 & 0
%         \\0 & 0 & 1
%       \end{pmatrix}
%       \cY_0
%       ,
%       \cY_0^{-1}
%       \begin{pmatrix}
%           1 & 0 & 0
%         \\{}^4c_2 & 1 & 0
%         \\{}^2c_1{}^4c_2+{}^4c_3 & {}^2c_1 & 1
%       \end{pmatrix}
%       \cY_0
%       ,
%   \\&\qquad\qquad
%     \left.
%       \cY_0^{-1}
%       \begin{pmatrix}
%           1 & {}^5c_2 & {}^5c_3
%         \\0 & 1 & 0
%         \\0 & 0 & 1
%       \end{pmatrix}
%       \cY_0
%       ,
%       \cY_0^{-1}
%       \begin{pmatrix}
%           1 & 0 & 0
%         \\{}^{\zkTwo}c_2 & 1 & 0
%         \\{}^{\zkTwo}c_3 & 0 & 1
%       \end{pmatrix}
%       \cY_0
%     \right)
%     \in
%     \prod_{\alpha\in\A}\Sto_{\alpha}(A^0) \,.
%   \end{align*}
% \end{comment}

%%% Local Variables:
%%% TeX-master: "Maximilian_Huber-Masters_Thesis-with_notes.tex"
%%% End:
